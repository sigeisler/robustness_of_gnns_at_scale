%%
%% This is file `sample-authordraft.tex',
%% generated with the docstrip utility.
%%
%% The original source files were:
%%
%% samples.dtx  (with options: `authordraft')
%% 
%% IMPORTANT NOTICE:
%% 
%% For the copyright see the source file.
%% 
%% Any modified versions of this file must be renamed
%% with new filenames distinct from sample-authordraft.tex.
%% 
%% For distribution of the original source see the terms
%% for copying and modification in the file samples.dtx.
%% 
%% This generated file may be distributed as long as the
%% original source files, as listed above, are part of the
%% same distribution. (The sources need not necessarily be
%% in the same archive or directory.)
%%
%% The first command in your LaTeX source must be the \documentclass command.
\documentclass[sigconf,authordraft]{acmart}

\usepackage{algorithm}                  % algorithms
\usepackage{algorithmic}                % algorithms
\usepackage{booktabs}                   % pandas
\usepackage{graphicx}
\usepackage{hyperref}
\usepackage{multirow}                   % mulirows
\usepackage{makecell}                   % Linebreaks in rows
\usepackage{natbib}
\usepackage{nicefrac}                   % compact symbols for 1/2, etc.
\usepackage{pgfplots}
\usepackage{subfig}


\newcommand{\adj}{\mA}
\newcommand{\weight}{\mW}
\newcommand{\features}{\mX}
\newcommand{\featset}{\sX}
\newcommand{\softout}{\vs}
\newcommand{\neighbors}{\sN}
\newcommand{\lone}{\text{L}_1}
\newcommand{\pertm}{\tilde{\mX}_\epsilon}
\newcommand{\pertmset}{\tilde{\sX}_\epsilon}

% \providecommand*\theoremautorefname{Theorem}
% \providecommand*\propositionautorefname{Proposition}
% \providecommand*\conjectureautorefname{Conjecture}
% \providecommand*\corollaryautorefname{Corollary}
% \providecommand*\lemmaautorefname{Lemma}

\renewcommand{\equationautorefname}{Eq.}
\renewcommand{\figureautorefname}{Fig.}
\newcommand{\algorithmautorefname}{Algorithm}
\renewcommand{\sectionautorefname}{\S}
\renewcommand{\subsectionautorefname}{\S}
\renewcommand{\appendixautorefname}{\S}

\newcommand{\dz}[1]{\textcolor{violet}{(DZ: #1)}}
\newcommand{\sg}[1]{\textcolor{blue}{(SG: #1)}}
\newcommand{\todo}[1]{\textcolor{red}{(Todo: #1)}}

\input{math_commands.tex}

%%
%% \BibTeX command to typeset BibTeX logo in the docs
\AtBeginDocument{%
  \providecommand\BibTeX{{%
    \normalfont B\kern-0.5em{\scshape i\kern-0.25em b}\kern-0.8em\TeX}}}

%% Rights management information.  This information is sent to you
%% when you complete the rights form.  These commands have SAMPLE
%% values in them; it is your responsibility as an author to replace
%% the commands and values with those provided to you when you
%% complete the rights form.
\setcopyright{acmcopyright}
\copyrightyear{2021}
\acmYear{2021}
\acmDOI{TBD}

%% These commands are for a PROCEEDINGS abstract or paper.
\acmConference[KDD ’21]{27th ACM SIGKDD Conference On Knowledge Discovery and Data Mining}{August 14--18, 2021}{Online}
% \acmBooktitle{Woodstock '18: ACM Symposium on Neural Gaze Detection,
%   June 03--05, 2018, Woodstock, NY}
% \acmPrice{TBD}
% \acmISBN{TBD}


%%
%% Submission ID.
%% Use this when submitting an article to a sponsored event. You'll
%% receive a unique submission ID from the organizers
%% of the event, and this ID should be used as the parameter to this command.
%%\acmSubmissionID{123-A56-BU3}

%%
%% The majority of ACM publications use numbered citations and
%% references.  The command \citestyle{authoryear} switches to the
%% "author year" style.
%%
%% If you are preparing content for an event
%% sponsored by ACM SIGGRAPH, you must use the "author year" style of
%% citations and references.
%% Uncommenting
%% the next command will enable that style.
%%\citestyle{acmauthoryear}

%%
%% end of the preamble, start of the body of the document source.
\begin{document}

%%
%% The "title" command has an optional parameter,
%% allowing the author to define a "short title" to be used in page headers.
\title{Robust Graph Neural Networks at Scale}

%%
%% The "author" command and its associated commands are used to define
%% the authors and their affiliations.
%% Of note is the shared affiliation of the first two authors, and the
%% "authornote" and "authornotemark" commands
%% used to denote shared contribution to the research.
\author{Simon Geisler, Hakan \c{S}irin, Daniel Zuegner, Tobias Schmidt, Aleksandar Bojchevski, Stephan Guennemann}
\email{{geisler, sirin, zuegnerd, schmidtt, a.bojchevski, guennemann}@in.tum.de}
\affiliation{%
  \institution{Technical University of Munich}
  \country{Germany}
}

%%
%% By default, the full list of authors will be used in the page
%% headers. Often, this list is too long, and will overlap
%% other information printed in the page headers. This command allows
%% the author to define a more concise list
%% of authors' names for this purpose.
\renewcommand{\shortauthors}{Geisler et al.}

%%
%% The abstract is a short summary of the work to be presented in the
%% article.
\begin{abstract}
  Adversarial robustness of Graph Neural Networks (GNNs) has become exceedingly important due to the popularity and diverse applications of GNNs. Specifically, structure perturbations are very effective, but designing attacks that add or remove edges is difficult because of the discrete optimization domain. Existing adversarial attacks for structure perturbations that rely on first-order optimization require a dense adjacency matrix and, therefore, can only be applied to small graphs (space complexity \(\Theta(n^2)\) in the number of nodes \(n\)).
  In this work, we address the question of how to scale adversarial attacks for evaluating the robustness for such applications.
  First, we show that the widely used surrogate losses are not well-suited for global attacks on GNNs at scale and propose two-alternatives that overcome these limitations.
  Second, we propose three attacks based on first-order optimization that do not require a dense adjacency matrix. Hence, we use our methods for global attacks on graphs more than 100 times larger than previously evaluated and scale local attacks to a graph 500 times larger than before. Moreover, one of the proposed attacks considers the very practical case of node injection.
  Last, we leverage recent advances in differentiable sorting for robust aggregation in message passing that scales linearly with the neighborhood size. We, therefore, improve one of the most effective defense strategies relying on a robust message-passing aggregation. Consequently, we also show how to improve the robustness of a graph neural network at scale.
\end{abstract}

%%
%% The code below is generated by the tool at http://dl.acm.org/ccs.cfm.
%% Please copy and paste the code instead of the example below.
%%
% \begin{CCSXML}
%   <ccs2012>
%   <concept>
%   <concept_id>10010520.10010553.10010562</concept_id>
%   <concept_desc>Computer systems organization~Embedded systems</concept_desc>
%   <concept_significance>500</concept_significance>
%   </concept>
%   <concept>
%   <concept_id>10010520.10010575.10010755</concept_id>
%   <concept_desc>Computer systems organization~Redundancy</concept_desc>
%   <concept_significance>300</concept_significance>
%   </concept>
%   <concept>
%   <concept_id>10010520.10010553.10010554</concept_id>
%   <concept_desc>Computer systems organization~Robotics</concept_desc>
%   <concept_significance>100</concept_significance>
%   </concept>
%   <concept>
%   <concept_id>10003033.10003083.10003095</concept_id>
%   <concept_desc>Networks~Network reliability</concept_desc>
%   <concept_significance>100</concept_significance>
%   </concept>
%   </ccs2012>
% \end{CCSXML}

% \ccsdesc[500]{Computer systems organization~Embedded systems}
% \ccsdesc[300]{Computer systems organization~Redundancy}
% \ccsdesc{Computer systems organization~Robotics}
% \ccsdesc[100]{Networks~Network reliability}

%%
%% Keywords. The author(s) should pick words that accurately describe
%% the work being presented. Separate the keywords with commas.
\keywords{Adversarial robustness, graph neural networks, scalability, semi-supervised learning}

%% A "teaser" image appears between the author and affiliation
%% information and the body of the document, and typically spans the
%% page.
% \begin{teaserfigure}
%   \includegraphics[width=\textwidth]{sampleteaser}
%   \caption{Seattle Mariners at Spring Training, 2010.}
%   \Description{Enjoying the baseball game from the third-base
%     seats. Ichiro Suzuki preparing to bat.}
%   \label{fig:teaser}
% \end{teaserfigure}

%%
%% This command processes the author and affiliation and title
%% information and builds the first part of the formatted document.
\maketitle

\section{Introduction} % Open

The adversarial robustness from the perspective of attack and defenses had been widely studied in recent research~\todo{cite}. However, most of these works analyzed graphs with less then 20,000 nodes. In particular from the perspective of real-world citation networks or internet-scale applications those graphs are tiny. In this work, we set the foundation for the study of adversarial robustness of GNNs on real-world citation/social networks. Importantly, it is unknown how the adversarial robustness/vulnerability relates to the graph size. Our first, experiments show that the adversarial robustness of a GNN depends on the graph size.

We show that contemporary surrogate losses are problematic for global attacks on large graphs. Especially in combination with realistic budgets, previous surrogate losses lead to weak attacks. We propose two new surrogate losses to overcome these limitations. On the traditional graphs, the new losses easily improve the strength of the attack by 33\% and for larger datasets we observe gains of more than 100\%. Hence, GNNs are even more fragile than previously believed.

Attacks based on combinatorial approaches easily become computationally infeasible because of the vast amount of potential adjacency matrices (\(2^{n^2}\)). With first-order optimization we can approximate the discrete optimization problem. First-order optimization attacks typically require the gradient towards the entries in the adjacency matrix and, hence, reduce the complexity to \(\Theta(n^2)\). To attack a small dataset such as PubMed (19,717 nodes), typically more than 20~GB are required if using a dense adjacency matrix. We argue that such memory requirements are still impractical and hinder practitioners to assess adversarial robustness. We identify the necessity to research scalable attacks for GNNs, due to the lack of such attacks for real-world graphs We propose two strategies how one may apply first-order optimization, without the burden of a dense adjacency matrix. In Section~\ref{sec:prbcd}, we describe how one might add/remove edges between existing nodes based on Randomized Block Coordinate Descent (R-BCD). Thereafter in Section~\ref{sec:attackkdd} we propose an attack that adds adversarial nodes and was one of the top 5 attacks of the KDD Cup 2020~\citep{Biendata2020}. In this work, we focus on the important task of node classification. Since we cover the task of global attacks our attacks can easily be generalized to graph classification.

The recent defense of~\citet{Geisler2020}, based on robust aggregations in the message passing step showed supervisor performance over the other compared defenses. We use the very recent advancements in differential sorting~\todo{cite}, to propose a computationally less demanding robust, differentiable aggregations. We call this aggregation Soft Median, observe a similar robustness to~\citep{Geisler2020} and can leverage the lower memory footprint in GNNs when memory is at premium.

Our contributions are the following:
\begin{itemize}
  \item We show that the widely used cross entropy is not a good surrogate loss for attacking the graph structure on graph neural networks.
  \item We propose two novel losses for global attacks on graph neural networks that overcome the limitations and empirically boosts the attack strength by up to 100\%.
  \item We propose two scalable adversarial attacks that add/remove edges between the existing nodes. One relies on projected gradient descent and the other uses a greedy FGSM-like optimization scheme (space complexity of \(\Theta(m)\) in the number of edges \(m\)).
  \item We propose one scalable adversarial attack that adds adversarial nodes (space complexity of \(\Theta(n)\)).
  \item We propose a differentiable robust aggregation that scales linearity w.r.t.\ the neighborhood size of the message passing aggregation and performs au par to the previous defense. This enables us to defend a GNN when memory is at premium.
  \item We study the adversarial robustness on graphs substantially larger than PubMed. Empirically, we find that the graph size negatively related to the adversarial robustness.
\end{itemize}

\section{Cross Entropy is a Bad Surrogate}\label{sec:ceisbad} % Simon

In the context of images, typically a single sample is attacked. In the context of graph neural networks this corresponds to a local attack. For such a scenario an untargeted attack it is often sufficient to maximize the cross entropy 
\begin{equation}\label{eq:crossentropy}
\text{CE}^{(n)}(y, \vp) = \sum_{c \in \sC} \mathbb{I}[y^{(n)} = c] \log(\evp_{c})^{(n)} = \log(\evp_{c^*}^{(n)})\,.
\end{equation}
Many \emph{global} attacks~\citep{Chen2018, Wu2019, Xu2018, Zugner2019a} also perform a global attack on a GNN via maximizing the cross entropy \(\max_{\adj} \text{CE}(f_{\theta}(\adj, \features))\). However, in our experiments, especially on large graphs, we have often observed that the CE loss increased while the accuracy did not decline. This can be explained by a bias of CE towards nodes which had a low confidence score (misclassified). Maximizing the CE is equivalent to minimizing the data likelihood and poorly correlates with the accuracy given this limited budget. This is apparent in Figure~\ref{fig:negceprob}.

\begin{figure}[t]
  \centering
  \makebox[\linewidth][c]{
    \(\begin{array}{cc}
      \subfloat[Clean graph]{\resizebox{0.5\linewidth}{!}{%% Creator: Matplotlib, PGF backend
%%
%% To include the figure in your LaTeX document, write
%%   \input{<filename>.pgf}
%%
%% Make sure the required packages are loaded in your preamble
%%   \usepackage{pgf}
%%
%% and, on pdftex
%%   \usepackage[utf8]{inputenc}\DeclareUnicodeCharacter{2212}{-}
%%
%% or, on luatex and xetex
%%   \usepackage{unicode-math}
%%
%% Figures using additional raster images can only be included by \input if
%% they are in the same directory as the main LaTeX file. For loading figures
%% from other directories you can use the `import` package
%%   \usepackage{import}
%%
%% and then include the figures with
%%   \import{<path to file>}{<filename>.pgf}
%%
%% Matplotlib used the following preamble
%%   \usepackage[utf8]{inputenc}
%%   \usepackage[T1]{fontenc}
%%   \usepackage{amsmath}
%%   \newcommand*{\mat}[1]{\boldsymbol{#1}}
%%
\begingroup%
\makeatletter%
\begin{pgfpicture}%
\pgfpathrectangle{\pgfpointorigin}{\pgfqpoint{10.063632in}{9.626613in}}%
\pgfusepath{use as bounding box, clip}%
\begin{pgfscope}%
\pgfsetbuttcap%
\pgfsetmiterjoin%
\definecolor{currentfill}{rgb}{1.000000,1.000000,1.000000}%
\pgfsetfillcolor{currentfill}%
\pgfsetlinewidth{0.000000pt}%
\definecolor{currentstroke}{rgb}{1.000000,1.000000,1.000000}%
\pgfsetstrokecolor{currentstroke}%
\pgfsetstrokeopacity{0.000000}%
\pgfsetdash{}{0pt}%
\pgfpathmoveto{\pgfqpoint{0.000000in}{0.000000in}}%
\pgfpathlineto{\pgfqpoint{10.063632in}{0.000000in}}%
\pgfpathlineto{\pgfqpoint{10.063632in}{9.626613in}}%
\pgfpathlineto{\pgfqpoint{0.000000in}{9.626613in}}%
\pgfpathclose%
\pgfusepath{fill}%
\end{pgfscope}%
\begin{pgfscope}%
\pgfsetbuttcap%
\pgfsetmiterjoin%
\definecolor{currentfill}{rgb}{1.000000,1.000000,1.000000}%
\pgfsetfillcolor{currentfill}%
\pgfsetlinewidth{0.000000pt}%
\definecolor{currentstroke}{rgb}{0.000000,0.000000,0.000000}%
\pgfsetstrokecolor{currentstroke}%
\pgfsetstrokeopacity{0.000000}%
\pgfsetdash{}{0pt}%
\pgfpathmoveto{\pgfqpoint{0.663632in}{0.466613in}}%
\pgfpathlineto{\pgfqpoint{9.963632in}{0.466613in}}%
\pgfpathlineto{\pgfqpoint{9.963632in}{9.526613in}}%
\pgfpathlineto{\pgfqpoint{0.663632in}{9.526613in}}%
\pgfpathclose%
\pgfusepath{fill}%
\end{pgfscope}%
\begin{pgfscope}%
\pgfpathrectangle{\pgfqpoint{0.663632in}{0.466613in}}{\pgfqpoint{9.300000in}{9.060000in}}%
\pgfusepath{clip}%
\pgfsetroundcap%
\pgfsetroundjoin%
\pgfsetlinewidth{0.501875pt}%
\definecolor{currentstroke}{rgb}{0.800000,0.800000,0.800000}%
\pgfsetstrokecolor{currentstroke}%
\pgfsetdash{}{0pt}%
\pgfpathmoveto{\pgfqpoint{1.086359in}{0.466613in}}%
\pgfpathlineto{\pgfqpoint{1.086359in}{9.526613in}}%
\pgfusepath{stroke}%
\end{pgfscope}%
\begin{pgfscope}%
\definecolor{textcolor}{rgb}{0.150000,0.150000,0.150000}%
\pgfsetstrokecolor{textcolor}%
\pgfsetfillcolor{textcolor}%
\pgftext[x=1.086359in,y=0.376335in,,top]{\color{textcolor}\rmfamily\fontsize{8.000000}{9.600000}\selectfont \(\displaystyle {0.0}\)}%
\end{pgfscope}%
\begin{pgfscope}%
\pgfpathrectangle{\pgfqpoint{0.663632in}{0.466613in}}{\pgfqpoint{9.300000in}{9.060000in}}%
\pgfusepath{clip}%
\pgfsetroundcap%
\pgfsetroundjoin%
\pgfsetlinewidth{0.501875pt}%
\definecolor{currentstroke}{rgb}{0.800000,0.800000,0.800000}%
\pgfsetstrokecolor{currentstroke}%
\pgfsetdash{}{0pt}%
\pgfpathmoveto{\pgfqpoint{2.777269in}{0.466613in}}%
\pgfpathlineto{\pgfqpoint{2.777269in}{9.526613in}}%
\pgfusepath{stroke}%
\end{pgfscope}%
\begin{pgfscope}%
\definecolor{textcolor}{rgb}{0.150000,0.150000,0.150000}%
\pgfsetstrokecolor{textcolor}%
\pgfsetfillcolor{textcolor}%
\pgftext[x=2.777269in,y=0.376335in,,top]{\color{textcolor}\rmfamily\fontsize{8.000000}{9.600000}\selectfont \(\displaystyle {0.2}\)}%
\end{pgfscope}%
\begin{pgfscope}%
\pgfpathrectangle{\pgfqpoint{0.663632in}{0.466613in}}{\pgfqpoint{9.300000in}{9.060000in}}%
\pgfusepath{clip}%
\pgfsetroundcap%
\pgfsetroundjoin%
\pgfsetlinewidth{0.501875pt}%
\definecolor{currentstroke}{rgb}{0.800000,0.800000,0.800000}%
\pgfsetstrokecolor{currentstroke}%
\pgfsetdash{}{0pt}%
\pgfpathmoveto{\pgfqpoint{4.468180in}{0.466613in}}%
\pgfpathlineto{\pgfqpoint{4.468180in}{9.526613in}}%
\pgfusepath{stroke}%
\end{pgfscope}%
\begin{pgfscope}%
\definecolor{textcolor}{rgb}{0.150000,0.150000,0.150000}%
\pgfsetstrokecolor{textcolor}%
\pgfsetfillcolor{textcolor}%
\pgftext[x=4.468180in,y=0.376335in,,top]{\color{textcolor}\rmfamily\fontsize{8.000000}{9.600000}\selectfont \(\displaystyle {0.4}\)}%
\end{pgfscope}%
\begin{pgfscope}%
\pgfpathrectangle{\pgfqpoint{0.663632in}{0.466613in}}{\pgfqpoint{9.300000in}{9.060000in}}%
\pgfusepath{clip}%
\pgfsetroundcap%
\pgfsetroundjoin%
\pgfsetlinewidth{0.501875pt}%
\definecolor{currentstroke}{rgb}{0.800000,0.800000,0.800000}%
\pgfsetstrokecolor{currentstroke}%
\pgfsetdash{}{0pt}%
\pgfpathmoveto{\pgfqpoint{6.159091in}{0.466613in}}%
\pgfpathlineto{\pgfqpoint{6.159091in}{9.526613in}}%
\pgfusepath{stroke}%
\end{pgfscope}%
\begin{pgfscope}%
\definecolor{textcolor}{rgb}{0.150000,0.150000,0.150000}%
\pgfsetstrokecolor{textcolor}%
\pgfsetfillcolor{textcolor}%
\pgftext[x=6.159091in,y=0.376335in,,top]{\color{textcolor}\rmfamily\fontsize{8.000000}{9.600000}\selectfont \(\displaystyle {0.6}\)}%
\end{pgfscope}%
\begin{pgfscope}%
\pgfpathrectangle{\pgfqpoint{0.663632in}{0.466613in}}{\pgfqpoint{9.300000in}{9.060000in}}%
\pgfusepath{clip}%
\pgfsetroundcap%
\pgfsetroundjoin%
\pgfsetlinewidth{0.501875pt}%
\definecolor{currentstroke}{rgb}{0.800000,0.800000,0.800000}%
\pgfsetstrokecolor{currentstroke}%
\pgfsetdash{}{0pt}%
\pgfpathmoveto{\pgfqpoint{7.850001in}{0.466613in}}%
\pgfpathlineto{\pgfqpoint{7.850001in}{9.526613in}}%
\pgfusepath{stroke}%
\end{pgfscope}%
\begin{pgfscope}%
\definecolor{textcolor}{rgb}{0.150000,0.150000,0.150000}%
\pgfsetstrokecolor{textcolor}%
\pgfsetfillcolor{textcolor}%
\pgftext[x=7.850001in,y=0.376335in,,top]{\color{textcolor}\rmfamily\fontsize{8.000000}{9.600000}\selectfont \(\displaystyle {0.8}\)}%
\end{pgfscope}%
\begin{pgfscope}%
\pgfpathrectangle{\pgfqpoint{0.663632in}{0.466613in}}{\pgfqpoint{9.300000in}{9.060000in}}%
\pgfusepath{clip}%
\pgfsetroundcap%
\pgfsetroundjoin%
\pgfsetlinewidth{0.501875pt}%
\definecolor{currentstroke}{rgb}{0.800000,0.800000,0.800000}%
\pgfsetstrokecolor{currentstroke}%
\pgfsetdash{}{0pt}%
\pgfpathmoveto{\pgfqpoint{9.540912in}{0.466613in}}%
\pgfpathlineto{\pgfqpoint{9.540912in}{9.526613in}}%
\pgfusepath{stroke}%
\end{pgfscope}%
\begin{pgfscope}%
\definecolor{textcolor}{rgb}{0.150000,0.150000,0.150000}%
\pgfsetstrokecolor{textcolor}%
\pgfsetfillcolor{textcolor}%
\pgftext[x=9.540912in,y=0.376335in,,top]{\color{textcolor}\rmfamily\fontsize{8.000000}{9.600000}\selectfont \(\displaystyle {1.0}\)}%
\end{pgfscope}%
\begin{pgfscope}%
\definecolor{textcolor}{rgb}{0.150000,0.150000,0.150000}%
\pgfsetstrokecolor{textcolor}%
\pgfsetfillcolor{textcolor}%
\pgftext[x=5.313632in,y=0.222655in,,top]{\color{textcolor}\rmfamily\fontsize{10.000000}{12.000000}\selectfont Probability of attacked node}%
\end{pgfscope}%
\begin{pgfscope}%
\pgfpathrectangle{\pgfqpoint{0.663632in}{0.466613in}}{\pgfqpoint{9.300000in}{9.060000in}}%
\pgfusepath{clip}%
\pgfsetroundcap%
\pgfsetroundjoin%
\pgfsetlinewidth{0.501875pt}%
\definecolor{currentstroke}{rgb}{0.800000,0.800000,0.800000}%
\pgfsetstrokecolor{currentstroke}%
\pgfsetdash{}{0pt}%
\pgfpathmoveto{\pgfqpoint{0.663632in}{0.466613in}}%
\pgfpathlineto{\pgfqpoint{9.963632in}{0.466613in}}%
\pgfusepath{stroke}%
\end{pgfscope}%
\begin{pgfscope}%
\definecolor{textcolor}{rgb}{0.150000,0.150000,0.150000}%
\pgfsetstrokecolor{textcolor}%
\pgfsetfillcolor{textcolor}%
\pgftext[x=0.514325in, y=0.428350in, left, base]{\color{textcolor}\rmfamily\fontsize{8.000000}{9.600000}\selectfont \(\displaystyle {0}\)}%
\end{pgfscope}%
\begin{pgfscope}%
\pgfpathrectangle{\pgfqpoint{0.663632in}{0.466613in}}{\pgfqpoint{9.300000in}{9.060000in}}%
\pgfusepath{clip}%
\pgfsetroundcap%
\pgfsetroundjoin%
\pgfsetlinewidth{0.501875pt}%
\definecolor{currentstroke}{rgb}{0.800000,0.800000,0.800000}%
\pgfsetstrokecolor{currentstroke}%
\pgfsetdash{}{0pt}%
\pgfpathmoveto{\pgfqpoint{0.663632in}{1.886837in}}%
\pgfpathlineto{\pgfqpoint{9.963632in}{1.886837in}}%
\pgfusepath{stroke}%
\end{pgfscope}%
\begin{pgfscope}%
\definecolor{textcolor}{rgb}{0.150000,0.150000,0.150000}%
\pgfsetstrokecolor{textcolor}%
\pgfsetfillcolor{textcolor}%
\pgftext[x=0.337239in, y=1.848575in, left, base]{\color{textcolor}\rmfamily\fontsize{8.000000}{9.600000}\selectfont \(\displaystyle {2000}\)}%
\end{pgfscope}%
\begin{pgfscope}%
\pgfpathrectangle{\pgfqpoint{0.663632in}{0.466613in}}{\pgfqpoint{9.300000in}{9.060000in}}%
\pgfusepath{clip}%
\pgfsetroundcap%
\pgfsetroundjoin%
\pgfsetlinewidth{0.501875pt}%
\definecolor{currentstroke}{rgb}{0.800000,0.800000,0.800000}%
\pgfsetstrokecolor{currentstroke}%
\pgfsetdash{}{0pt}%
\pgfpathmoveto{\pgfqpoint{0.663632in}{3.307061in}}%
\pgfpathlineto{\pgfqpoint{9.963632in}{3.307061in}}%
\pgfusepath{stroke}%
\end{pgfscope}%
\begin{pgfscope}%
\definecolor{textcolor}{rgb}{0.150000,0.150000,0.150000}%
\pgfsetstrokecolor{textcolor}%
\pgfsetfillcolor{textcolor}%
\pgftext[x=0.337239in, y=3.268799in, left, base]{\color{textcolor}\rmfamily\fontsize{8.000000}{9.600000}\selectfont \(\displaystyle {4000}\)}%
\end{pgfscope}%
\begin{pgfscope}%
\pgfpathrectangle{\pgfqpoint{0.663632in}{0.466613in}}{\pgfqpoint{9.300000in}{9.060000in}}%
\pgfusepath{clip}%
\pgfsetroundcap%
\pgfsetroundjoin%
\pgfsetlinewidth{0.501875pt}%
\definecolor{currentstroke}{rgb}{0.800000,0.800000,0.800000}%
\pgfsetstrokecolor{currentstroke}%
\pgfsetdash{}{0pt}%
\pgfpathmoveto{\pgfqpoint{0.663632in}{4.727285in}}%
\pgfpathlineto{\pgfqpoint{9.963632in}{4.727285in}}%
\pgfusepath{stroke}%
\end{pgfscope}%
\begin{pgfscope}%
\definecolor{textcolor}{rgb}{0.150000,0.150000,0.150000}%
\pgfsetstrokecolor{textcolor}%
\pgfsetfillcolor{textcolor}%
\pgftext[x=0.337239in, y=4.689023in, left, base]{\color{textcolor}\rmfamily\fontsize{8.000000}{9.600000}\selectfont \(\displaystyle {6000}\)}%
\end{pgfscope}%
\begin{pgfscope}%
\pgfpathrectangle{\pgfqpoint{0.663632in}{0.466613in}}{\pgfqpoint{9.300000in}{9.060000in}}%
\pgfusepath{clip}%
\pgfsetroundcap%
\pgfsetroundjoin%
\pgfsetlinewidth{0.501875pt}%
\definecolor{currentstroke}{rgb}{0.800000,0.800000,0.800000}%
\pgfsetstrokecolor{currentstroke}%
\pgfsetdash{}{0pt}%
\pgfpathmoveto{\pgfqpoint{0.663632in}{6.147509in}}%
\pgfpathlineto{\pgfqpoint{9.963632in}{6.147509in}}%
\pgfusepath{stroke}%
\end{pgfscope}%
\begin{pgfscope}%
\definecolor{textcolor}{rgb}{0.150000,0.150000,0.150000}%
\pgfsetstrokecolor{textcolor}%
\pgfsetfillcolor{textcolor}%
\pgftext[x=0.337239in, y=6.109247in, left, base]{\color{textcolor}\rmfamily\fontsize{8.000000}{9.600000}\selectfont \(\displaystyle {8000}\)}%
\end{pgfscope}%
\begin{pgfscope}%
\pgfpathrectangle{\pgfqpoint{0.663632in}{0.466613in}}{\pgfqpoint{9.300000in}{9.060000in}}%
\pgfusepath{clip}%
\pgfsetroundcap%
\pgfsetroundjoin%
\pgfsetlinewidth{0.501875pt}%
\definecolor{currentstroke}{rgb}{0.800000,0.800000,0.800000}%
\pgfsetstrokecolor{currentstroke}%
\pgfsetdash{}{0pt}%
\pgfpathmoveto{\pgfqpoint{0.663632in}{7.567733in}}%
\pgfpathlineto{\pgfqpoint{9.963632in}{7.567733in}}%
\pgfusepath{stroke}%
\end{pgfscope}%
\begin{pgfscope}%
\definecolor{textcolor}{rgb}{0.150000,0.150000,0.150000}%
\pgfsetstrokecolor{textcolor}%
\pgfsetfillcolor{textcolor}%
\pgftext[x=0.278211in, y=7.529471in, left, base]{\color{textcolor}\rmfamily\fontsize{8.000000}{9.600000}\selectfont \(\displaystyle {10000}\)}%
\end{pgfscope}%
\begin{pgfscope}%
\pgfpathrectangle{\pgfqpoint{0.663632in}{0.466613in}}{\pgfqpoint{9.300000in}{9.060000in}}%
\pgfusepath{clip}%
\pgfsetroundcap%
\pgfsetroundjoin%
\pgfsetlinewidth{0.501875pt}%
\definecolor{currentstroke}{rgb}{0.800000,0.800000,0.800000}%
\pgfsetstrokecolor{currentstroke}%
\pgfsetdash{}{0pt}%
\pgfpathmoveto{\pgfqpoint{0.663632in}{8.987957in}}%
\pgfpathlineto{\pgfqpoint{9.963632in}{8.987957in}}%
\pgfusepath{stroke}%
\end{pgfscope}%
\begin{pgfscope}%
\definecolor{textcolor}{rgb}{0.150000,0.150000,0.150000}%
\pgfsetstrokecolor{textcolor}%
\pgfsetfillcolor{textcolor}%
\pgftext[x=0.278211in, y=8.949695in, left, base]{\color{textcolor}\rmfamily\fontsize{8.000000}{9.600000}\selectfont \(\displaystyle {12000}\)}%
\end{pgfscope}%
\begin{pgfscope}%
\definecolor{textcolor}{rgb}{0.150000,0.150000,0.150000}%
\pgfsetstrokecolor{textcolor}%
\pgfsetfillcolor{textcolor}%
\pgftext[x=0.222655in,y=4.996613in,,bottom,rotate=90.000000]{\color{textcolor}\rmfamily\fontsize{10.000000}{12.000000}\selectfont Probability density}%
\end{pgfscope}%
\begin{pgfscope}%
\pgfpathrectangle{\pgfqpoint{0.663632in}{0.466613in}}{\pgfqpoint{9.300000in}{9.060000in}}%
\pgfusepath{clip}%
\pgfsetbuttcap%
\pgfsetmiterjoin%
\definecolor{currentfill}{rgb}{0.298039,0.447059,0.690196}%
\pgfsetfillcolor{currentfill}%
\pgfsetlinewidth{1.003750pt}%
\definecolor{currentstroke}{rgb}{1.000000,1.000000,1.000000}%
\pgfsetstrokecolor{currentstroke}%
\pgfsetdash{}{0pt}%
\pgfpathmoveto{\pgfqpoint{1.086359in}{0.466613in}}%
\pgfpathlineto{\pgfqpoint{1.931813in}{0.466613in}}%
\pgfpathlineto{\pgfqpoint{1.931813in}{4.958071in}}%
\pgfpathlineto{\pgfqpoint{1.086359in}{4.958071in}}%
\pgfpathclose%
\pgfusepath{stroke,fill}%
\end{pgfscope}%
\begin{pgfscope}%
\pgfpathrectangle{\pgfqpoint{0.663632in}{0.466613in}}{\pgfqpoint{9.300000in}{9.060000in}}%
\pgfusepath{clip}%
\pgfsetbuttcap%
\pgfsetmiterjoin%
\definecolor{currentfill}{rgb}{0.298039,0.447059,0.690196}%
\pgfsetfillcolor{currentfill}%
\pgfsetlinewidth{1.003750pt}%
\definecolor{currentstroke}{rgb}{1.000000,1.000000,1.000000}%
\pgfsetstrokecolor{currentstroke}%
\pgfsetdash{}{0pt}%
\pgfpathmoveto{\pgfqpoint{1.931813in}{0.466613in}}%
\pgfpathlineto{\pgfqpoint{2.777268in}{0.466613in}}%
\pgfpathlineto{\pgfqpoint{2.777268in}{2.986800in}}%
\pgfpathlineto{\pgfqpoint{1.931813in}{2.986800in}}%
\pgfpathclose%
\pgfusepath{stroke,fill}%
\end{pgfscope}%
\begin{pgfscope}%
\pgfpathrectangle{\pgfqpoint{0.663632in}{0.466613in}}{\pgfqpoint{9.300000in}{9.060000in}}%
\pgfusepath{clip}%
\pgfsetbuttcap%
\pgfsetmiterjoin%
\definecolor{currentfill}{rgb}{0.298039,0.447059,0.690196}%
\pgfsetfillcolor{currentfill}%
\pgfsetlinewidth{1.003750pt}%
\definecolor{currentstroke}{rgb}{1.000000,1.000000,1.000000}%
\pgfsetstrokecolor{currentstroke}%
\pgfsetdash{}{0pt}%
\pgfpathmoveto{\pgfqpoint{2.777268in}{0.466613in}}%
\pgfpathlineto{\pgfqpoint{3.622723in}{0.466613in}}%
\pgfpathlineto{\pgfqpoint{3.622723in}{2.903717in}}%
\pgfpathlineto{\pgfqpoint{2.777268in}{2.903717in}}%
\pgfpathclose%
\pgfusepath{stroke,fill}%
\end{pgfscope}%
\begin{pgfscope}%
\pgfpathrectangle{\pgfqpoint{0.663632in}{0.466613in}}{\pgfqpoint{9.300000in}{9.060000in}}%
\pgfusepath{clip}%
\pgfsetbuttcap%
\pgfsetmiterjoin%
\definecolor{currentfill}{rgb}{0.298039,0.447059,0.690196}%
\pgfsetfillcolor{currentfill}%
\pgfsetlinewidth{1.003750pt}%
\definecolor{currentstroke}{rgb}{1.000000,1.000000,1.000000}%
\pgfsetstrokecolor{currentstroke}%
\pgfsetdash{}{0pt}%
\pgfpathmoveto{\pgfqpoint{3.622722in}{0.466613in}}%
\pgfpathlineto{\pgfqpoint{4.468177in}{0.466613in}}%
\pgfpathlineto{\pgfqpoint{4.468177in}{2.853299in}}%
\pgfpathlineto{\pgfqpoint{3.622722in}{2.853299in}}%
\pgfpathclose%
\pgfusepath{stroke,fill}%
\end{pgfscope}%
\begin{pgfscope}%
\pgfpathrectangle{\pgfqpoint{0.663632in}{0.466613in}}{\pgfqpoint{9.300000in}{9.060000in}}%
\pgfusepath{clip}%
\pgfsetbuttcap%
\pgfsetmiterjoin%
\definecolor{currentfill}{rgb}{0.298039,0.447059,0.690196}%
\pgfsetfillcolor{currentfill}%
\pgfsetlinewidth{1.003750pt}%
\definecolor{currentstroke}{rgb}{1.000000,1.000000,1.000000}%
\pgfsetstrokecolor{currentstroke}%
\pgfsetdash{}{0pt}%
\pgfpathmoveto{\pgfqpoint{4.468177in}{0.466613in}}%
\pgfpathlineto{\pgfqpoint{5.313631in}{0.466613in}}%
\pgfpathlineto{\pgfqpoint{5.313631in}{2.920050in}}%
\pgfpathlineto{\pgfqpoint{4.468177in}{2.920050in}}%
\pgfpathclose%
\pgfusepath{stroke,fill}%
\end{pgfscope}%
\begin{pgfscope}%
\pgfpathrectangle{\pgfqpoint{0.663632in}{0.466613in}}{\pgfqpoint{9.300000in}{9.060000in}}%
\pgfusepath{clip}%
\pgfsetbuttcap%
\pgfsetmiterjoin%
\definecolor{currentfill}{rgb}{0.298039,0.447059,0.690196}%
\pgfsetfillcolor{currentfill}%
\pgfsetlinewidth{1.003750pt}%
\definecolor{currentstroke}{rgb}{1.000000,1.000000,1.000000}%
\pgfsetstrokecolor{currentstroke}%
\pgfsetdash{}{0pt}%
\pgfpathmoveto{\pgfqpoint{5.313632in}{0.466613in}}%
\pgfpathlineto{\pgfqpoint{6.159087in}{0.466613in}}%
\pgfpathlineto{\pgfqpoint{6.159087in}{2.888095in}}%
\pgfpathlineto{\pgfqpoint{5.313632in}{2.888095in}}%
\pgfpathclose%
\pgfusepath{stroke,fill}%
\end{pgfscope}%
\begin{pgfscope}%
\pgfpathrectangle{\pgfqpoint{0.663632in}{0.466613in}}{\pgfqpoint{9.300000in}{9.060000in}}%
\pgfusepath{clip}%
\pgfsetbuttcap%
\pgfsetmiterjoin%
\definecolor{currentfill}{rgb}{0.298039,0.447059,0.690196}%
\pgfsetfillcolor{currentfill}%
\pgfsetlinewidth{1.003750pt}%
\definecolor{currentstroke}{rgb}{1.000000,1.000000,1.000000}%
\pgfsetstrokecolor{currentstroke}%
\pgfsetdash{}{0pt}%
\pgfpathmoveto{\pgfqpoint{6.159086in}{0.466613in}}%
\pgfpathlineto{\pgfqpoint{7.004541in}{0.466613in}}%
\pgfpathlineto{\pgfqpoint{7.004541in}{2.919340in}}%
\pgfpathlineto{\pgfqpoint{6.159086in}{2.919340in}}%
\pgfpathclose%
\pgfusepath{stroke,fill}%
\end{pgfscope}%
\begin{pgfscope}%
\pgfpathrectangle{\pgfqpoint{0.663632in}{0.466613in}}{\pgfqpoint{9.300000in}{9.060000in}}%
\pgfusepath{clip}%
\pgfsetbuttcap%
\pgfsetmiterjoin%
\definecolor{currentfill}{rgb}{0.298039,0.447059,0.690196}%
\pgfsetfillcolor{currentfill}%
\pgfsetlinewidth{1.003750pt}%
\definecolor{currentstroke}{rgb}{1.000000,1.000000,1.000000}%
\pgfsetstrokecolor{currentstroke}%
\pgfsetdash{}{0pt}%
\pgfpathmoveto{\pgfqpoint{7.004541in}{0.466613in}}%
\pgfpathlineto{\pgfqpoint{7.849995in}{0.466613in}}%
\pgfpathlineto{\pgfqpoint{7.849995in}{3.255933in}}%
\pgfpathlineto{\pgfqpoint{7.004541in}{3.255933in}}%
\pgfpathclose%
\pgfusepath{stroke,fill}%
\end{pgfscope}%
\begin{pgfscope}%
\pgfpathrectangle{\pgfqpoint{0.663632in}{0.466613in}}{\pgfqpoint{9.300000in}{9.060000in}}%
\pgfusepath{clip}%
\pgfsetbuttcap%
\pgfsetmiterjoin%
\definecolor{currentfill}{rgb}{0.298039,0.447059,0.690196}%
\pgfsetfillcolor{currentfill}%
\pgfsetlinewidth{1.003750pt}%
\definecolor{currentstroke}{rgb}{1.000000,1.000000,1.000000}%
\pgfsetstrokecolor{currentstroke}%
\pgfsetdash{}{0pt}%
\pgfpathmoveto{\pgfqpoint{7.849995in}{0.466613in}}%
\pgfpathlineto{\pgfqpoint{8.695450in}{0.466613in}}%
\pgfpathlineto{\pgfqpoint{8.695450in}{4.399213in}}%
\pgfpathlineto{\pgfqpoint{7.849995in}{4.399213in}}%
\pgfpathclose%
\pgfusepath{stroke,fill}%
\end{pgfscope}%
\begin{pgfscope}%
\pgfpathrectangle{\pgfqpoint{0.663632in}{0.466613in}}{\pgfqpoint{9.300000in}{9.060000in}}%
\pgfusepath{clip}%
\pgfsetbuttcap%
\pgfsetmiterjoin%
\definecolor{currentfill}{rgb}{0.298039,0.447059,0.690196}%
\pgfsetfillcolor{currentfill}%
\pgfsetlinewidth{1.003750pt}%
\definecolor{currentstroke}{rgb}{1.000000,1.000000,1.000000}%
\pgfsetstrokecolor{currentstroke}%
\pgfsetdash{}{0pt}%
\pgfpathmoveto{\pgfqpoint{8.695450in}{0.466613in}}%
\pgfpathlineto{\pgfqpoint{9.540904in}{0.466613in}}%
\pgfpathlineto{\pgfqpoint{9.540904in}{9.095184in}}%
\pgfpathlineto{\pgfqpoint{8.695450in}{9.095184in}}%
\pgfpathclose%
\pgfusepath{stroke,fill}%
\end{pgfscope}%
\begin{pgfscope}%
\pgfsetrectcap%
\pgfsetmiterjoin%
\pgfsetlinewidth{0.752812pt}%
\definecolor{currentstroke}{rgb}{0.700000,0.700000,0.700000}%
\pgfsetstrokecolor{currentstroke}%
\pgfsetdash{}{0pt}%
\pgfpathmoveto{\pgfqpoint{0.663632in}{0.466613in}}%
\pgfpathlineto{\pgfqpoint{0.663632in}{9.526613in}}%
\pgfusepath{stroke}%
\end{pgfscope}%
\begin{pgfscope}%
\pgfsetrectcap%
\pgfsetmiterjoin%
\pgfsetlinewidth{0.752812pt}%
\definecolor{currentstroke}{rgb}{0.700000,0.700000,0.700000}%
\pgfsetstrokecolor{currentstroke}%
\pgfsetdash{}{0pt}%
\pgfpathmoveto{\pgfqpoint{9.963632in}{0.466613in}}%
\pgfpathlineto{\pgfqpoint{9.963632in}{9.526613in}}%
\pgfusepath{stroke}%
\end{pgfscope}%
\begin{pgfscope}%
\pgfsetrectcap%
\pgfsetmiterjoin%
\pgfsetlinewidth{0.752812pt}%
\definecolor{currentstroke}{rgb}{0.700000,0.700000,0.700000}%
\pgfsetstrokecolor{currentstroke}%
\pgfsetdash{}{0pt}%
\pgfpathmoveto{\pgfqpoint{0.663632in}{0.466613in}}%
\pgfpathlineto{\pgfqpoint{9.963632in}{0.466613in}}%
\pgfusepath{stroke}%
\end{pgfscope}%
\begin{pgfscope}%
\pgfsetrectcap%
\pgfsetmiterjoin%
\pgfsetlinewidth{0.752812pt}%
\definecolor{currentstroke}{rgb}{0.700000,0.700000,0.700000}%
\pgfsetstrokecolor{currentstroke}%
\pgfsetdash{}{0pt}%
\pgfpathmoveto{\pgfqpoint{0.663632in}{9.526613in}}%
\pgfpathlineto{\pgfqpoint{9.963632in}{9.526613in}}%
\pgfusepath{stroke}%
\end{pgfscope}%
\end{pgfpicture}%
\makeatother%
\endgroup%
}} &
      \subfloat[Attacked nodes]{\resizebox{0.5\linewidth}{!}{%% Creator: Matplotlib, PGF backend
%%
%% To include the figure in your LaTeX document, write
%%   \input{<filename>.pgf}
%%
%% Make sure the required packages are loaded in your preamble
%%   \usepackage{pgf}
%%
%% and, on pdftex
%%   \usepackage[utf8]{inputenc}\DeclareUnicodeCharacter{2212}{-}
%%
%% or, on luatex and xetex
%%   \usepackage{unicode-math}
%%
%% Figures using additional raster images can only be included by \input if
%% they are in the same directory as the main LaTeX file. For loading figures
%% from other directories you can use the `import` package
%%   \usepackage{import}
%%
%% and then include the figures with
%%   \import{<path to file>}{<filename>.pgf}
%%
%% Matplotlib used the following preamble
%%   \usepackage[utf8]{inputenc}
%%   \usepackage[T1]{fontenc}
%%   \usepackage{amsmath}
%%   \newcommand*{\mat}[1]{\boldsymbol{#1}}
%%
\begingroup%
\makeatletter%
\begin{pgfpicture}%
\pgfpathrectangle{\pgfpointorigin}{\pgfqpoint{2.000721in}{1.819295in}}%
\pgfusepath{use as bounding box, clip}%
\begin{pgfscope}%
\pgfsetbuttcap%
\pgfsetmiterjoin%
\definecolor{currentfill}{rgb}{1.000000,1.000000,1.000000}%
\pgfsetfillcolor{currentfill}%
\pgfsetlinewidth{0.000000pt}%
\definecolor{currentstroke}{rgb}{1.000000,1.000000,1.000000}%
\pgfsetstrokecolor{currentstroke}%
\pgfsetstrokeopacity{0.000000}%
\pgfsetdash{}{0pt}%
\pgfpathmoveto{\pgfqpoint{-0.000000in}{0.000000in}}%
\pgfpathlineto{\pgfqpoint{2.000721in}{0.000000in}}%
\pgfpathlineto{\pgfqpoint{2.000721in}{1.819295in}}%
\pgfpathlineto{\pgfqpoint{-0.000000in}{1.819295in}}%
\pgfpathclose%
\pgfusepath{fill}%
\end{pgfscope}%
\begin{pgfscope}%
\pgfsetbuttcap%
\pgfsetmiterjoin%
\definecolor{currentfill}{rgb}{1.000000,1.000000,1.000000}%
\pgfsetfillcolor{currentfill}%
\pgfsetlinewidth{0.000000pt}%
\definecolor{currentstroke}{rgb}{0.000000,0.000000,0.000000}%
\pgfsetstrokecolor{currentstroke}%
\pgfsetstrokeopacity{0.000000}%
\pgfsetdash{}{0pt}%
\pgfpathmoveto{\pgfqpoint{0.604603in}{0.473545in}}%
\pgfpathlineto{\pgfqpoint{1.883353in}{0.473545in}}%
\pgfpathlineto{\pgfqpoint{1.883353in}{1.719295in}}%
\pgfpathlineto{\pgfqpoint{0.604603in}{1.719295in}}%
\pgfpathclose%
\pgfusepath{fill}%
\end{pgfscope}%
\begin{pgfscope}%
\pgfpathrectangle{\pgfqpoint{0.604603in}{0.473545in}}{\pgfqpoint{1.278750in}{1.245750in}}%
\pgfusepath{clip}%
\pgfsetroundcap%
\pgfsetroundjoin%
\pgfsetlinewidth{0.501875pt}%
\definecolor{currentstroke}{rgb}{0.800000,0.800000,0.800000}%
\pgfsetstrokecolor{currentstroke}%
\pgfsetdash{}{0pt}%
\pgfpathmoveto{\pgfqpoint{0.662728in}{0.473545in}}%
\pgfpathlineto{\pgfqpoint{0.662728in}{1.719295in}}%
\pgfusepath{stroke}%
\end{pgfscope}%
\begin{pgfscope}%
\definecolor{textcolor}{rgb}{0.150000,0.150000,0.150000}%
\pgfsetstrokecolor{textcolor}%
\pgfsetfillcolor{textcolor}%
\pgftext[x=0.662728in,y=0.383267in,,top]{\color{textcolor}\rmfamily\fontsize{8.000000}{9.600000}\selectfont \(\displaystyle {0.0}\)}%
\end{pgfscope}%
\begin{pgfscope}%
\pgfpathrectangle{\pgfqpoint{0.604603in}{0.473545in}}{\pgfqpoint{1.278750in}{1.245750in}}%
\pgfusepath{clip}%
\pgfsetroundcap%
\pgfsetroundjoin%
\pgfsetlinewidth{0.501875pt}%
\definecolor{currentstroke}{rgb}{0.800000,0.800000,0.800000}%
\pgfsetstrokecolor{currentstroke}%
\pgfsetdash{}{0pt}%
\pgfpathmoveto{\pgfqpoint{1.244012in}{0.473545in}}%
\pgfpathlineto{\pgfqpoint{1.244012in}{1.719295in}}%
\pgfusepath{stroke}%
\end{pgfscope}%
\begin{pgfscope}%
\definecolor{textcolor}{rgb}{0.150000,0.150000,0.150000}%
\pgfsetstrokecolor{textcolor}%
\pgfsetfillcolor{textcolor}%
\pgftext[x=1.244012in,y=0.383267in,,top]{\color{textcolor}\rmfamily\fontsize{8.000000}{9.600000}\selectfont \(\displaystyle {0.5}\)}%
\end{pgfscope}%
\begin{pgfscope}%
\pgfpathrectangle{\pgfqpoint{0.604603in}{0.473545in}}{\pgfqpoint{1.278750in}{1.245750in}}%
\pgfusepath{clip}%
\pgfsetroundcap%
\pgfsetroundjoin%
\pgfsetlinewidth{0.501875pt}%
\definecolor{currentstroke}{rgb}{0.800000,0.800000,0.800000}%
\pgfsetstrokecolor{currentstroke}%
\pgfsetdash{}{0pt}%
\pgfpathmoveto{\pgfqpoint{1.825296in}{0.473545in}}%
\pgfpathlineto{\pgfqpoint{1.825296in}{1.719295in}}%
\pgfusepath{stroke}%
\end{pgfscope}%
\begin{pgfscope}%
\definecolor{textcolor}{rgb}{0.150000,0.150000,0.150000}%
\pgfsetstrokecolor{textcolor}%
\pgfsetfillcolor{textcolor}%
\pgftext[x=1.825296in,y=0.383267in,,top]{\color{textcolor}\rmfamily\fontsize{8.000000}{9.600000}\selectfont \(\displaystyle {1.0}\)}%
\end{pgfscope}%
\begin{pgfscope}%
\definecolor{textcolor}{rgb}{0.150000,0.150000,0.150000}%
\pgfsetstrokecolor{textcolor}%
\pgfsetfillcolor{textcolor}%
\pgftext[x=1.243978in,y=0.229588in,,top]{\color{textcolor}\rmfamily\fontsize{10.000000}{12.000000}\selectfont Prob. score \(\displaystyle p^*\)}%
\end{pgfscope}%
\begin{pgfscope}%
\pgfpathrectangle{\pgfqpoint{0.604603in}{0.473545in}}{\pgfqpoint{1.278750in}{1.245750in}}%
\pgfusepath{clip}%
\pgfsetroundcap%
\pgfsetroundjoin%
\pgfsetlinewidth{0.501875pt}%
\definecolor{currentstroke}{rgb}{0.800000,0.800000,0.800000}%
\pgfsetstrokecolor{currentstroke}%
\pgfsetdash{}{0pt}%
\pgfpathmoveto{\pgfqpoint{0.604603in}{0.473545in}}%
\pgfpathlineto{\pgfqpoint{1.883353in}{0.473545in}}%
\pgfusepath{stroke}%
\end{pgfscope}%
\begin{pgfscope}%
\definecolor{textcolor}{rgb}{0.150000,0.150000,0.150000}%
\pgfsetstrokecolor{textcolor}%
\pgfsetfillcolor{textcolor}%
\pgftext[x=0.455297in, y=0.435283in, left, base]{\color{textcolor}\rmfamily\fontsize{8.000000}{9.600000}\selectfont \(\displaystyle {0}\)}%
\end{pgfscope}%
\begin{pgfscope}%
\pgfpathrectangle{\pgfqpoint{0.604603in}{0.473545in}}{\pgfqpoint{1.278750in}{1.245750in}}%
\pgfusepath{clip}%
\pgfsetroundcap%
\pgfsetroundjoin%
\pgfsetlinewidth{0.501875pt}%
\definecolor{currentstroke}{rgb}{0.800000,0.800000,0.800000}%
\pgfsetstrokecolor{currentstroke}%
\pgfsetdash{}{0pt}%
\pgfpathmoveto{\pgfqpoint{0.604603in}{0.810790in}}%
\pgfpathlineto{\pgfqpoint{1.883353in}{0.810790in}}%
\pgfusepath{stroke}%
\end{pgfscope}%
\begin{pgfscope}%
\definecolor{textcolor}{rgb}{0.150000,0.150000,0.150000}%
\pgfsetstrokecolor{textcolor}%
\pgfsetfillcolor{textcolor}%
\pgftext[x=0.278211in, y=0.772528in, left, base]{\color{textcolor}\rmfamily\fontsize{8.000000}{9.600000}\selectfont \(\displaystyle {1000}\)}%
\end{pgfscope}%
\begin{pgfscope}%
\pgfpathrectangle{\pgfqpoint{0.604603in}{0.473545in}}{\pgfqpoint{1.278750in}{1.245750in}}%
\pgfusepath{clip}%
\pgfsetroundcap%
\pgfsetroundjoin%
\pgfsetlinewidth{0.501875pt}%
\definecolor{currentstroke}{rgb}{0.800000,0.800000,0.800000}%
\pgfsetstrokecolor{currentstroke}%
\pgfsetdash{}{0pt}%
\pgfpathmoveto{\pgfqpoint{0.604603in}{1.148035in}}%
\pgfpathlineto{\pgfqpoint{1.883353in}{1.148035in}}%
\pgfusepath{stroke}%
\end{pgfscope}%
\begin{pgfscope}%
\definecolor{textcolor}{rgb}{0.150000,0.150000,0.150000}%
\pgfsetstrokecolor{textcolor}%
\pgfsetfillcolor{textcolor}%
\pgftext[x=0.278211in, y=1.109773in, left, base]{\color{textcolor}\rmfamily\fontsize{8.000000}{9.600000}\selectfont \(\displaystyle {2000}\)}%
\end{pgfscope}%
\begin{pgfscope}%
\pgfpathrectangle{\pgfqpoint{0.604603in}{0.473545in}}{\pgfqpoint{1.278750in}{1.245750in}}%
\pgfusepath{clip}%
\pgfsetroundcap%
\pgfsetroundjoin%
\pgfsetlinewidth{0.501875pt}%
\definecolor{currentstroke}{rgb}{0.800000,0.800000,0.800000}%
\pgfsetstrokecolor{currentstroke}%
\pgfsetdash{}{0pt}%
\pgfpathmoveto{\pgfqpoint{0.604603in}{1.485281in}}%
\pgfpathlineto{\pgfqpoint{1.883353in}{1.485281in}}%
\pgfusepath{stroke}%
\end{pgfscope}%
\begin{pgfscope}%
\definecolor{textcolor}{rgb}{0.150000,0.150000,0.150000}%
\pgfsetstrokecolor{textcolor}%
\pgfsetfillcolor{textcolor}%
\pgftext[x=0.278211in, y=1.447018in, left, base]{\color{textcolor}\rmfamily\fontsize{8.000000}{9.600000}\selectfont \(\displaystyle {3000}\)}%
\end{pgfscope}%
\begin{pgfscope}%
\definecolor{textcolor}{rgb}{0.150000,0.150000,0.150000}%
\pgfsetstrokecolor{textcolor}%
\pgfsetfillcolor{textcolor}%
\pgftext[x=0.222655in,y=1.096420in,,bottom,rotate=90.000000]{\color{textcolor}\rmfamily\fontsize{10.000000}{12.000000}\selectfont Frequency}%
\end{pgfscope}%
\begin{pgfscope}%
\pgfpathrectangle{\pgfqpoint{0.604603in}{0.473545in}}{\pgfqpoint{1.278750in}{1.245750in}}%
\pgfusepath{clip}%
\pgfsetbuttcap%
\pgfsetmiterjoin%
\definecolor{currentfill}{rgb}{0.298039,0.447059,0.690196}%
\pgfsetfillcolor{currentfill}%
\pgfsetlinewidth{1.003750pt}%
\definecolor{currentstroke}{rgb}{1.000000,1.000000,1.000000}%
\pgfsetstrokecolor{currentstroke}%
\pgfsetdash{}{0pt}%
\pgfpathmoveto{\pgfqpoint{0.662728in}{0.473545in}}%
\pgfpathlineto{\pgfqpoint{0.778978in}{0.473545in}}%
\pgfpathlineto{\pgfqpoint{0.778978in}{1.659974in}}%
\pgfpathlineto{\pgfqpoint{0.662728in}{1.659974in}}%
\pgfpathclose%
\pgfusepath{stroke,fill}%
\end{pgfscope}%
\begin{pgfscope}%
\pgfpathrectangle{\pgfqpoint{0.604603in}{0.473545in}}{\pgfqpoint{1.278750in}{1.245750in}}%
\pgfusepath{clip}%
\pgfsetbuttcap%
\pgfsetmiterjoin%
\definecolor{currentfill}{rgb}{0.298039,0.447059,0.690196}%
\pgfsetfillcolor{currentfill}%
\pgfsetlinewidth{1.003750pt}%
\definecolor{currentstroke}{rgb}{1.000000,1.000000,1.000000}%
\pgfsetstrokecolor{currentstroke}%
\pgfsetdash{}{0pt}%
\pgfpathmoveto{\pgfqpoint{0.778978in}{0.473545in}}%
\pgfpathlineto{\pgfqpoint{0.895228in}{0.473545in}}%
\pgfpathlineto{\pgfqpoint{0.895228in}{1.064399in}}%
\pgfpathlineto{\pgfqpoint{0.778978in}{1.064399in}}%
\pgfpathclose%
\pgfusepath{stroke,fill}%
\end{pgfscope}%
\begin{pgfscope}%
\pgfpathrectangle{\pgfqpoint{0.604603in}{0.473545in}}{\pgfqpoint{1.278750in}{1.245750in}}%
\pgfusepath{clip}%
\pgfsetbuttcap%
\pgfsetmiterjoin%
\definecolor{currentfill}{rgb}{0.298039,0.447059,0.690196}%
\pgfsetfillcolor{currentfill}%
\pgfsetlinewidth{1.003750pt}%
\definecolor{currentstroke}{rgb}{1.000000,1.000000,1.000000}%
\pgfsetstrokecolor{currentstroke}%
\pgfsetdash{}{0pt}%
\pgfpathmoveto{\pgfqpoint{0.895228in}{0.473545in}}%
\pgfpathlineto{\pgfqpoint{1.011478in}{0.473545in}}%
\pgfpathlineto{\pgfqpoint{1.011478in}{0.896788in}}%
\pgfpathlineto{\pgfqpoint{0.895228in}{0.896788in}}%
\pgfpathclose%
\pgfusepath{stroke,fill}%
\end{pgfscope}%
\begin{pgfscope}%
\pgfpathrectangle{\pgfqpoint{0.604603in}{0.473545in}}{\pgfqpoint{1.278750in}{1.245750in}}%
\pgfusepath{clip}%
\pgfsetbuttcap%
\pgfsetmiterjoin%
\definecolor{currentfill}{rgb}{0.298039,0.447059,0.690196}%
\pgfsetfillcolor{currentfill}%
\pgfsetlinewidth{1.003750pt}%
\definecolor{currentstroke}{rgb}{1.000000,1.000000,1.000000}%
\pgfsetstrokecolor{currentstroke}%
\pgfsetdash{}{0pt}%
\pgfpathmoveto{\pgfqpoint{1.011478in}{0.473545in}}%
\pgfpathlineto{\pgfqpoint{1.127728in}{0.473545in}}%
\pgfpathlineto{\pgfqpoint{1.127728in}{0.868122in}}%
\pgfpathlineto{\pgfqpoint{1.011478in}{0.868122in}}%
\pgfpathclose%
\pgfusepath{stroke,fill}%
\end{pgfscope}%
\begin{pgfscope}%
\pgfpathrectangle{\pgfqpoint{0.604603in}{0.473545in}}{\pgfqpoint{1.278750in}{1.245750in}}%
\pgfusepath{clip}%
\pgfsetbuttcap%
\pgfsetmiterjoin%
\definecolor{currentfill}{rgb}{0.298039,0.447059,0.690196}%
\pgfsetfillcolor{currentfill}%
\pgfsetlinewidth{1.003750pt}%
\definecolor{currentstroke}{rgb}{1.000000,1.000000,1.000000}%
\pgfsetstrokecolor{currentstroke}%
\pgfsetdash{}{0pt}%
\pgfpathmoveto{\pgfqpoint{1.127728in}{0.473545in}}%
\pgfpathlineto{\pgfqpoint{1.243978in}{0.473545in}}%
\pgfpathlineto{\pgfqpoint{1.243978in}{0.857667in}}%
\pgfpathlineto{\pgfqpoint{1.127728in}{0.857667in}}%
\pgfpathclose%
\pgfusepath{stroke,fill}%
\end{pgfscope}%
\begin{pgfscope}%
\pgfpathrectangle{\pgfqpoint{0.604603in}{0.473545in}}{\pgfqpoint{1.278750in}{1.245750in}}%
\pgfusepath{clip}%
\pgfsetbuttcap%
\pgfsetmiterjoin%
\definecolor{currentfill}{rgb}{0.298039,0.447059,0.690196}%
\pgfsetfillcolor{currentfill}%
\pgfsetlinewidth{1.003750pt}%
\definecolor{currentstroke}{rgb}{1.000000,1.000000,1.000000}%
\pgfsetstrokecolor{currentstroke}%
\pgfsetdash{}{0pt}%
\pgfpathmoveto{\pgfqpoint{1.243978in}{0.473545in}}%
\pgfpathlineto{\pgfqpoint{1.360228in}{0.473545in}}%
\pgfpathlineto{\pgfqpoint{1.360228in}{0.829339in}}%
\pgfpathlineto{\pgfqpoint{1.243978in}{0.829339in}}%
\pgfpathclose%
\pgfusepath{stroke,fill}%
\end{pgfscope}%
\begin{pgfscope}%
\pgfpathrectangle{\pgfqpoint{0.604603in}{0.473545in}}{\pgfqpoint{1.278750in}{1.245750in}}%
\pgfusepath{clip}%
\pgfsetbuttcap%
\pgfsetmiterjoin%
\definecolor{currentfill}{rgb}{0.298039,0.447059,0.690196}%
\pgfsetfillcolor{currentfill}%
\pgfsetlinewidth{1.003750pt}%
\definecolor{currentstroke}{rgb}{1.000000,1.000000,1.000000}%
\pgfsetstrokecolor{currentstroke}%
\pgfsetdash{}{0pt}%
\pgfpathmoveto{\pgfqpoint{1.360228in}{0.473545in}}%
\pgfpathlineto{\pgfqpoint{1.476478in}{0.473545in}}%
\pgfpathlineto{\pgfqpoint{1.476478in}{0.809779in}}%
\pgfpathlineto{\pgfqpoint{1.360228in}{0.809779in}}%
\pgfpathclose%
\pgfusepath{stroke,fill}%
\end{pgfscope}%
\begin{pgfscope}%
\pgfpathrectangle{\pgfqpoint{0.604603in}{0.473545in}}{\pgfqpoint{1.278750in}{1.245750in}}%
\pgfusepath{clip}%
\pgfsetbuttcap%
\pgfsetmiterjoin%
\definecolor{currentfill}{rgb}{0.298039,0.447059,0.690196}%
\pgfsetfillcolor{currentfill}%
\pgfsetlinewidth{1.003750pt}%
\definecolor{currentstroke}{rgb}{1.000000,1.000000,1.000000}%
\pgfsetstrokecolor{currentstroke}%
\pgfsetdash{}{0pt}%
\pgfpathmoveto{\pgfqpoint{1.476478in}{0.473545in}}%
\pgfpathlineto{\pgfqpoint{1.592728in}{0.473545in}}%
\pgfpathlineto{\pgfqpoint{1.592728in}{0.800336in}}%
\pgfpathlineto{\pgfqpoint{1.476478in}{0.800336in}}%
\pgfpathclose%
\pgfusepath{stroke,fill}%
\end{pgfscope}%
\begin{pgfscope}%
\pgfpathrectangle{\pgfqpoint{0.604603in}{0.473545in}}{\pgfqpoint{1.278750in}{1.245750in}}%
\pgfusepath{clip}%
\pgfsetbuttcap%
\pgfsetmiterjoin%
\definecolor{currentfill}{rgb}{0.298039,0.447059,0.690196}%
\pgfsetfillcolor{currentfill}%
\pgfsetlinewidth{1.003750pt}%
\definecolor{currentstroke}{rgb}{1.000000,1.000000,1.000000}%
\pgfsetstrokecolor{currentstroke}%
\pgfsetdash{}{0pt}%
\pgfpathmoveto{\pgfqpoint{1.592728in}{0.473545in}}%
\pgfpathlineto{\pgfqpoint{1.708978in}{0.473545in}}%
\pgfpathlineto{\pgfqpoint{1.708978in}{0.826641in}}%
\pgfpathlineto{\pgfqpoint{1.592728in}{0.826641in}}%
\pgfpathclose%
\pgfusepath{stroke,fill}%
\end{pgfscope}%
\begin{pgfscope}%
\pgfpathrectangle{\pgfqpoint{0.604603in}{0.473545in}}{\pgfqpoint{1.278750in}{1.245750in}}%
\pgfusepath{clip}%
\pgfsetbuttcap%
\pgfsetmiterjoin%
\definecolor{currentfill}{rgb}{0.298039,0.447059,0.690196}%
\pgfsetfillcolor{currentfill}%
\pgfsetlinewidth{1.003750pt}%
\definecolor{currentstroke}{rgb}{1.000000,1.000000,1.000000}%
\pgfsetstrokecolor{currentstroke}%
\pgfsetdash{}{0pt}%
\pgfpathmoveto{\pgfqpoint{1.708978in}{0.473545in}}%
\pgfpathlineto{\pgfqpoint{1.825228in}{0.473545in}}%
\pgfpathlineto{\pgfqpoint{1.825228in}{0.899823in}}%
\pgfpathlineto{\pgfqpoint{1.708978in}{0.899823in}}%
\pgfpathclose%
\pgfusepath{stroke,fill}%
\end{pgfscope}%
\begin{pgfscope}%
\pgfsetrectcap%
\pgfsetmiterjoin%
\pgfsetlinewidth{0.752812pt}%
\definecolor{currentstroke}{rgb}{0.700000,0.700000,0.700000}%
\pgfsetstrokecolor{currentstroke}%
\pgfsetdash{}{0pt}%
\pgfpathmoveto{\pgfqpoint{0.604603in}{0.473545in}}%
\pgfpathlineto{\pgfqpoint{0.604603in}{1.719295in}}%
\pgfusepath{stroke}%
\end{pgfscope}%
\begin{pgfscope}%
\pgfsetrectcap%
\pgfsetmiterjoin%
\pgfsetlinewidth{0.752812pt}%
\definecolor{currentstroke}{rgb}{0.700000,0.700000,0.700000}%
\pgfsetstrokecolor{currentstroke}%
\pgfsetdash{}{0pt}%
\pgfpathmoveto{\pgfqpoint{1.883353in}{0.473545in}}%
\pgfpathlineto{\pgfqpoint{1.883353in}{1.719295in}}%
\pgfusepath{stroke}%
\end{pgfscope}%
\begin{pgfscope}%
\pgfsetrectcap%
\pgfsetmiterjoin%
\pgfsetlinewidth{0.752812pt}%
\definecolor{currentstroke}{rgb}{0.700000,0.700000,0.700000}%
\pgfsetstrokecolor{currentstroke}%
\pgfsetdash{}{0pt}%
\pgfpathmoveto{\pgfqpoint{0.604603in}{0.473545in}}%
\pgfpathlineto{\pgfqpoint{1.883353in}{0.473545in}}%
\pgfusepath{stroke}%
\end{pgfscope}%
\begin{pgfscope}%
\pgfsetrectcap%
\pgfsetmiterjoin%
\pgfsetlinewidth{0.752812pt}%
\definecolor{currentstroke}{rgb}{0.700000,0.700000,0.700000}%
\pgfsetstrokecolor{currentstroke}%
\pgfsetdash{}{0pt}%
\pgfpathmoveto{\pgfqpoint{0.604603in}{1.719295in}}%
\pgfpathlineto{\pgfqpoint{1.883353in}{1.719295in}}%
\pgfusepath{stroke}%
\end{pgfscope}%
\end{pgfpicture}%
\makeatother%
\endgroup%
}}  \\
    \end{array}\)
  }
  \caption{In (a) we show the distribution of confidence scores for the correct class \(p^*\) over all test nodes on the clean graph. We observe a large fraction of very confident nodes. In stark contrast, in (b) we analyze the distribution of the directly attacked test nodes before the evasion attack started (i.e.\ if the attack would randomly attack nodes this distribution should match (a)). Here we small budget of one percent of edges (\(\Delta=\epsilon=0.01\)) on the arXiv dataset (see~Tab.\ref{tab:datasets})\label{fig:negceprob}}
\end{figure}

In contrast of attacking a single image/node, a global structure attack has to 1) keep house with the budget \(\Delta\) and 2) find edges that degrade the accuracy maximally (i.e.\ potentially target ``fragile'' nodes). Without additional information, intuitively, one would first attack low confidence nodes close to the boundary. Analogously, a first-order optimization algorithm focuses on the nodes where loss surface is steep. In general, attacking a GNN is non-convex and presumably NP hard and hence it is hard to make any statements that hold in general. We propose to study surrogate losses for applying first-order optimization to the collective attack of multiple nodes with a global budget \(\Delta\) based on a greedy scheme. This scheme attacks one node at a time given its confidence scores \(\vp\) or margin \(\psi = \min_{c \ne c^*} \evp_{c^*} - \evp_{c}\). Note that such a greedy algorithm is equivalent to a scheduling algorithm where we have the jobs \(i \in \sV\) of length \(\Delta_i\), with the set of nodes \(\sV\).

\begin{theorem}\label{theorem:goodsurrogate}
  Let \(f_{\theta}(\adj, \features)\) be a node classification algorithm applied to a graph with the adjacency matrix \(\adj\), \(\mathcal{L}\) is the negative 0/1 loss, and let \(\Delta_i\) be the budget to move the prediction of any arbitrary node \(i\) (independently for each node) over the decision boundary is given by \(\Delta_i = |\psi_i| + \eta\) (with an arbitrary small constant \(\eta\)). 
  %with a strictly monotonic increasing function \(g(\eta)\). 
  We can obtain the global optimum of
  \begin{equation}\label{eq:goodsurrogate}
    \max_{\tilde{\adj}\text{ s.t.\ }\|\tilde{\adj} - \adj\|_0 < \Delta} \mathcal{L}(f_{\theta}(\tilde{\adj}, \features))\,
  \end{equation}
  via greedily attacking the nodes in order of \(\mathcal{L}'(-\eta) - \mathcal{L}'(\psi_0) \ge \mathcal{L}'(-\eta) - \mathcal{L}'(\psi_1) \ge \dots \ge \mathcal{L}'(-\eta) - \mathcal{L}'(\psi_l)\) until the budget is exceeded \(\Delta < \sum_{i=0}^{l+1} \Delta_i\). \(\mathcal{L}'\) denotes the monotonically decreasing surrogate loss and its derivative is (a) \(\nicefrac{\partial \mathcal{L}'}{\partial p^*} |_{p^* < 0} \le 0\), has (b) its minimum for \(\psi \to 0^+\), and (c) \(\nicefrac{\partial \mathcal{L}'}{\partial p^*}|_{p^* > 0}\) is monotonically decreasing.
\end{theorem}

\begin{proof}
  An optimal solution is obtained by the greedy algorithm that executes the tasks in order \(\nicefrac{\mathcal{L}(-\eta) - \mathcal{L}(\psi_0)}{\psi_0} \ge \nicefrac{\mathcal{L}(-\eta) - \mathcal{L}(\psi_1)}{\psi_1} \ge \dots \ge \nicefrac{\mathcal{L}(-\eta) - \mathcal{L}(\psi_l)}{\psi_l}\) until the budget is exceeded: \(\Delta < \sum_{i=0}^{l+1} \Delta_i\). We can easily see that this greedy solution obtains the optimal solution by an exchange argument. Lets suppose we are given the optimal plan \(\sigma^*\) and the greedy solution has the plan \(\sigma\). In case \(\sigma^*\) would contain one or more tasks for that \(w > l\) instead of \(b \le l\). We know that \(\psi_w > \psi_b\) and hence \(\Delta_w > \Delta_b\). Thus, replacing \(\psi_b\) by \(\psi_w\) would either lead to the an equal good solution or would be even better (contradiction!). Hence, the greedy plan \(\sigma\) is at least as good as the optimal plan \(\sigma^*\). Moreover, the maximum is unique except for ties s.t.\ \(\psi_i = \psi_j \ge 0, \forall i, j \in \sV\).

  Consequently, a surrogate loss \(\mathcal{L}`\) must maintain the order above: \(\mathcal{L}'(-\eta) - \mathcal{L}'(\psi_0) \ge \mathcal{L}'(-\eta) - \mathcal{L}'(\psi_1) \ge \dots \ge \mathcal{L}'(-\eta) - \mathcal{L}'(\psi_l)\) in order to reach the global optimum as well (possibly allowing for equivalently good exchanged tasks). The order is preserved if (a) \(\mathcal{L}'(-a) \le \mathcal{L}'(-\eta)\) for every \(a \in (\eta,1]\), (c) \(\nicefrac{\partial \mathcal{L}'}{\partial p^*}|_{p^* > 0}\) is strictly monotonically decreasing or in other words \(\mathcal{L}'\) is strictly concave for positive inputs. From this it follows that (b) \(\nicefrac{\partial \mathcal{L}'}{\partial p^*}\) is minimal for \(\psi \to 0^+\). \todo{Shorten? Reference?}
\end{proof}

\begin{corollary}\label{corollary:ce}
  The cross entropy surrogate loss \(\mathcal{L} = \text{Accuracy} \approx CE\) (Eq.~\ref{eq:crossentropy}) does not obtain the global optimum since it violates properties (a) and (b).
\end{corollary}

\begin{corollary}\label{corollary:margin}
  The margin loss \(\mathcal{L} = \text{Accuracy} \approx \text{Margin Loss} = min(0, \psi)\) does not obtain the global optimum, since its gradient is constant for \(\psi > 0\) and therefore violates (b) and (c).
\end{corollary}

We argue that the surrogate is certainly not well suited if it even does not work in such a basic scenario. More generally, we state the subsequent conjectures a well-suited, monotonically decreasing surrogate loss \(\mathcal{L'}(y, \vp)\) should obey:
\begin{conjecture}\label{conjecture:saturate}
  A proper surrogate loss \(\mathcal{L}'(y, \vp)\) should saturate for low confidence values of the correct class: \(\lim_{\psi \to -1^+} \mathcal{L}(y, \vp) = k < \infty\).
\end{conjecture}
\begin{conjecture}\label{conjecture:maxgrad}
  A proper surrogate loss \(\mathcal{L}'(y, \vp)\) should favour points close to the decision boundary: \(\nicefrac{\partial \mathcal{L}(y, \vp)}{\partial \evp_{c^*}} |_{\psi > 0}  > \nicefrac{\partial \mathcal{L}(y, \vp)}{\partial \evp_{c^*}} |_{\psi \to 0^+}\).
\end{conjecture}

In this work we will study several choices for surrogate loss functions:
\begin{enumerate}
  \item Cross Entropy: \(\text{CE} = \log(\evp_{c^*})\)
  \item Carlini-Wagner~\cite{Carlini2017}: \(\text{CW} = (\min_{c \ne c^*} \evz_{c^*} - \evz_{c})+\)
  \item Second-most-likely CE: \(\text{SCE} = \log(\arg\max_{c \ne c^*} \evp_{c})\)
  \item Masked CE: \(\text{MCE} = \frac{1}{|\sV^+|} \sum_{n \in \sV^+} \log(\evp_{c^*}^{(n)})\)
  \item \(\text{tanh Margin} = \tanh(\min_{c \ne c^*} \evz_{c^*} - \evz_{c})\)
\end{enumerate}
Note that \(\sV^+\) is the set of correctly classified nodes, \(\vp\) is the vector of confidence scores, and \(\vz\) is the vector with logits. To simplify notation, we define the losses for a single node (except for \(\text{MCE}\)) and denote the correct class with \(c^*\). 

We choose the losses (1-3) since they are alleged natural choices or have been used in the literature~\citep{Chen2018, Wu2019, Xu2018, Zugner2019a}. Loss (1) violates Conjecture~\ref{conjecture:saturate} and losses (2) and (3) violate Conjecture~\ref{conjecture:saturate}. (4) and (5) are both losses that obeys both conjectures. Note that (5) would not pass the test of Theorem~\ref{theorem:goodsurrogate}. We study it nevertheless since a surrogate loss that saturates already for small negative values could negatively impact the learning dynamics for projected gradient descent.

\todo{Should we already teaser that thos losses are much better?}

\section{Attacks}\label{sec:prbcd}

\todo{Explain chapter}

\subsection{Related Work}\label{sec:related} % Simon

\begin{algorithm}[b]
  \small
  \caption{R-BCD~\citep{Nesterov2012}}
  \label{algo:rbcd}
  \begin{algorithmic}[1]
    \STATE Choose \(\vx_0 \in \R^d\)
    \FOR{\(k \in \{1,2, \dots, K\}\)}
    \STATE Draw random indices \(\vi_k \in \{0, 1, \dots, n\}^b\)
    \STATE \(\vx_{k} \leftarrow \vx_{k-1} - \alpha_{k} \nabla_{\vi_{k}} \mathcal{L}(\vx_{k-1})\)
    \ENDFOR
  \end{algorithmic}
\end{algorithm}

\textbf{Large scale optimization.} In a big data setting, the cost to calculate the gradient towards all variables can be prohibitively high. For this reason, coordinate descent has gained importance in machine learning and large scale optimization~\citep{Wright2015}. \citet{Nesterov2012} proposed (and analyzed the convergence) of Randomized Block Coordinate Descent (R-BCD). For R-BCD's pseudo code see Algorithm~\ref{algo:rbcd}. In R-BCD only a subset of variables is optimized at a time and, hence, only the gradients towards those variables are required. In many cases, this allows for a lower memory footprint and in some settings even converges faster than standard methods~\citep{Nesterov2017}. Constrained optimization with first-order methods can be solved with methods such as Projected Gradient Descent (PGD) or Fast Gradient Sign Method (FGSM)~\cite{Goodfellow2015}. Similarly, one can extend R-BCD to constrained optimization~\citep{Nesterov2012}.

\textbf{Adversarial attacks.} Beginning with~\citep{Dai2018, Zugner2018}, many adversarial attacks on the graph structure have been proposed~\citep{Zugner2019a, Xu2019a, Bojchevski2019, Wu2019, Wang2019, Tang2020}. We limit the scope to an adversary with perfect knowledge about the graph, GNN and test labels. Even this white-box scenario has not been studied for large graphs and our primary goal is to asses adversarial robustness. Further, we make a distinction between local attacks that attack a single node or small group and global attacks that attack the whole graph. It is apparent that local attacks are much easier to scale than global attacks. 

First-order optimization attacks such as Metattack~\citep{Zugner2019a} or integrated gradients~\citep{Wu2019} rely on the gradient towards all possible entries in the \textit{dense} adjacency matrix \(\adj\) (quadratic space complexity) to solve the optimization problem for structure perturbations:
\begin{equation}\label{eq:attack}
  \max_{\adj} \mathcal{L}(f_{\theta}(\adj, \features))
\end{equation}
with the adjacency matrix \(\adj\), node features \(\features\), loss \(\mathcal{L}\), and the trained network \(f_{\theta}\).
%We will consider more sophisticated threat models in future work.

\citet{Dai2018} scale their local reinforcement learning approach to a very sparse, large-scale graph for financial transactions. In contrast to our work, they scale their local attack only using a very small budget \(\Delta\) (their time complexity scales linearly with \(\Delta\)). \citet{Wang2019} also scale their attack to a larger graph than PubMed but they do not attack GNNs. \todo{Adversarial attack on large scale graph} analyze adversarial attacks on mini-batch techniques such as Cluster-GCN~\todo{cite}. However, they only scale to a dataset with around 200k nodes and also only consider a local attack. We scale to much bigger datasets, consider a wider class of Graph Neural Networks and also propose a global attack that is scalable. \todo{Mention Fake Node Attacks} propose an attack based on GANs that inserts fake nodes and is the closest related work to our attack in Section~\ref{sec:attackkdd}.
\todo{MGA: Momentum Gradient Attack on Network}

\subsection{Adding and Removing Edges}\label{sec:prbcd}

In this section we discuss the case where the attack vector is to perturb the existing, binary graph structure:
%
\begin{equation}\label{eq:pgd}
  \max_{\mP\,\,\text{s.t.}\, \sum \mP \le \Delta} \mathcal{L}(f_{\theta}(\adj \oplus \mP, \features))\,.
\end{equation}
%
Here, \(\oplus\) stands for an element-wise exclusive or and \(\Delta\) denotes the edge budget (i.e.\ the number of altered entries in the perturbed adjacency matrix). In the following, we use set \(\sP\) and matrix notation \(\mP\) for the sparse perturbations \(\sP\) interchangeably. Naively, applying R-BCD to optimize towards the dense adjacency matrix would only save some computation on obtaining the respective gradient, but still has a space complexity of \(\mathcal{O}(n^2)\) (on top of the complexity of the attacked model; in the following we will neglect this fact). Note that in R-BCD we interpret each possible entry in the perturbation set \(\sP\) as one dimension of our optimization problem. To mitigate the quadratic complexity, we make use of the fact that the solution is going to be sparse (\(L_0\) perturbation). As in evolutionary algorithms, in each epoch, we keep that part of the search space which is promising and resample the rest. We can view the underlying problem as a combination of \(L_0\)-norm PGD and an adaptive version of Randomized Block Coordinate Descent (R-BCD). We give a formal definition of Projected and Randomized Block Coordinate Descent (PR-BCD) in Algorithm~\ref{algo:prbcd}. PR-BCD comes with a space complexity of \(\Theta(m)\), as we typically choose \(\Delta\) to be a fraction of \(m\).

As proposed by~\citet{Xu2019a}, \(\vp\) is interpreted as the probability for flipping each potential entry in the perturbation set \(\sP\) and is used in the end to sample the final edge additions/removals. In each epoch \(e \in \{1,2, \dots, E\}\), this probability mass \(\vp\) is used as edge weight. In this case we overload \(\oplus\) s.t.\ \(\adj_{ij} \oplus p_{ij} = \adj_{ij} + p_{ij}\) if \(\adj_{ij} = 1\) and \(\adj_{ij} - p_{ij}\) otherwise. The projection \(\Pi_{\E[\text{Bernoulli}(\vp)] = \Delta} (\vp)\) adjusts the probability mass such that \(\E[\text{Bernoulli}(\vp)] = \sum_{i \in b} \evp_i \approx \Delta\) and that \(\vp \in [0, 1]\) (line 8). If using an undirected graph, the potential edges are restricted to the the upper/lower triangular \(n \times n\) matrix. In the end we sample \(\sP \sim \text{Bernoulli}(\vp)\) (line 16).

Note that the projection of the perturbation set \(\Pi_{\E[\text{Bernoulli}(\vp)] = \Delta} (\vp)\) contains many zero elements, but is not guaranteed to be sparse. If \(\vp\) has more than 50\% non-zero entries, we remove the entries with the lowest probability mass such that 50\% of the search space is resampled. Otherwise, we resample all zero entries in \(\vp\). However, one also might apply a more sophisticated heuristic \(h(\vp)\) (see line 11).

We keep the random block fixed and run \(K_{\text{resample}}\) epochs. Thereafter, we decay the learning rate as in~\cite{Xu2019a}. We also employ early stopping for both stages (\(k \le K_{\text{resample}} \text{ and } k > K_{\text{resample}}\) with the epoch \(k\)) such that we take the result of the epoch with highest loss \(\mathcal{L}\).

With growing \(n\) it is unrealistic that each possible entry of the adjacency matrix was part of at least one random search space of (P)R-BCD. Apparently, with a constant search space size, the number of mutually exclusive chunks of the perturbation set grows with \(\Theta(n^2)\) and this would imply a quadratic runtime. However, as evident in randomized black-box attacks~\citep{Waniek2018}, it is not necessary to test every possible edge to obtain an effective attack. In Fig.~\ref{fig:randomblocksizeinfluence}, we analyze the influence of the random block size on the perturbed accuracy. For a sufficient block size \(b\) our method performs comparably to its dense equivalent. For larger graphs, we observe that depending on the block size we observe a greater influence of the block size \(b\). However, as shown in Fig.~\ref{fig:arxivrandomblocksizeinfluence}, one might increase the number of epochs for an even improved attack strength. We argue that this indicates that PRBCD successfully spots the right edges to keep.

\begin{figure}[t]
  \centering
  \resizebox{\linewidth}{!}{\input{assets/global_PRBCD_novel_loss_cora_ml_0.25_block_size_legend.pgf}}
  \makebox[\linewidth][c]{
    \(\begin{array}{cc}
      \subfloat[\(\epsilon=0.1\) (i.e. \(\Delta=798\)]{\resizebox{0.5\linewidth}{!}{%% Creator: Matplotlib, PGF backend
%%
%% To include the figure in your LaTeX document, write
%%   \input{<filename>.pgf}
%%
%% Make sure the required packages are loaded in your preamble
%%   \usepackage{pgf}
%%
%% and, on pdftex
%%   \usepackage[utf8]{inputenc}\DeclareUnicodeCharacter{2212}{-}
%%
%% or, on luatex and xetex
%%   \usepackage{unicode-math}
%%
%% Figures using additional raster images can only be included by \input if
%% they are in the same directory as the main LaTeX file. For loading figures
%% from other directories you can use the `import` package
%%   \usepackage{import}
%%
%% and then include the figures with
%%   \import{<path to file>}{<filename>.pgf}
%%
%% Matplotlib used the following preamble
%%   \usepackage[utf8]{inputenc}
%%   \usepackage[T1]{fontenc}
%%   \usepackage{amsmath}
%%   \newcommand*{\mat}[1]{\boldsymbol{#1}}
%%
\begingroup%
\makeatletter%
\begin{pgfpicture}%
\pgfpathrectangle{\pgfpointorigin}{\pgfqpoint{1.986956in}{1.814099in}}%
\pgfusepath{use as bounding box, clip}%
\begin{pgfscope}%
\pgfsetbuttcap%
\pgfsetmiterjoin%
\definecolor{currentfill}{rgb}{1.000000,1.000000,1.000000}%
\pgfsetfillcolor{currentfill}%
\pgfsetlinewidth{0.000000pt}%
\definecolor{currentstroke}{rgb}{1.000000,1.000000,1.000000}%
\pgfsetstrokecolor{currentstroke}%
\pgfsetstrokeopacity{0.000000}%
\pgfsetdash{}{0pt}%
\pgfpathmoveto{\pgfqpoint{0.000000in}{0.000000in}}%
\pgfpathlineto{\pgfqpoint{1.986956in}{0.000000in}}%
\pgfpathlineto{\pgfqpoint{1.986956in}{1.814099in}}%
\pgfpathlineto{\pgfqpoint{0.000000in}{1.814099in}}%
\pgfpathclose%
\pgfusepath{fill}%
\end{pgfscope}%
\begin{pgfscope}%
\pgfsetbuttcap%
\pgfsetmiterjoin%
\definecolor{currentfill}{rgb}{1.000000,1.000000,1.000000}%
\pgfsetfillcolor{currentfill}%
\pgfsetlinewidth{0.000000pt}%
\definecolor{currentstroke}{rgb}{0.000000,0.000000,0.000000}%
\pgfsetstrokecolor{currentstroke}%
\pgfsetstrokeopacity{0.000000}%
\pgfsetdash{}{0pt}%
\pgfpathmoveto{\pgfqpoint{0.578368in}{0.468349in}}%
\pgfpathlineto{\pgfqpoint{1.857118in}{0.468349in}}%
\pgfpathlineto{\pgfqpoint{1.857118in}{1.714099in}}%
\pgfpathlineto{\pgfqpoint{0.578368in}{1.714099in}}%
\pgfpathclose%
\pgfusepath{fill}%
\end{pgfscope}%
\begin{pgfscope}%
\pgfpathrectangle{\pgfqpoint{0.578368in}{0.468349in}}{\pgfqpoint{1.278750in}{1.245750in}}%
\pgfusepath{clip}%
\pgfsetroundcap%
\pgfsetroundjoin%
\pgfsetlinewidth{0.501875pt}%
\definecolor{currentstroke}{rgb}{0.800000,0.800000,0.800000}%
\pgfsetstrokecolor{currentstroke}%
\pgfsetdash{}{0pt}%
\pgfpathmoveto{\pgfqpoint{0.742505in}{0.468349in}}%
\pgfpathlineto{\pgfqpoint{0.742505in}{1.714099in}}%
\pgfusepath{stroke}%
\end{pgfscope}%
\begin{pgfscope}%
\definecolor{textcolor}{rgb}{0.150000,0.150000,0.150000}%
\pgfsetstrokecolor{textcolor}%
\pgfsetfillcolor{textcolor}%
\pgftext[x=0.742505in,y=0.378072in,,top]{\color{textcolor}\rmfamily\fontsize{8.000000}{9.600000}\selectfont \(\displaystyle {10^{4}}\)}%
\end{pgfscope}%
\begin{pgfscope}%
\pgfpathrectangle{\pgfqpoint{0.578368in}{0.468349in}}{\pgfqpoint{1.278750in}{1.245750in}}%
\pgfusepath{clip}%
\pgfsetroundcap%
\pgfsetroundjoin%
\pgfsetlinewidth{0.501875pt}%
\definecolor{currentstroke}{rgb}{0.800000,0.800000,0.800000}%
\pgfsetstrokecolor{currentstroke}%
\pgfsetdash{}{0pt}%
\pgfpathmoveto{\pgfqpoint{1.094667in}{0.468349in}}%
\pgfpathlineto{\pgfqpoint{1.094667in}{1.714099in}}%
\pgfusepath{stroke}%
\end{pgfscope}%
\begin{pgfscope}%
\definecolor{textcolor}{rgb}{0.150000,0.150000,0.150000}%
\pgfsetstrokecolor{textcolor}%
\pgfsetfillcolor{textcolor}%
\pgftext[x=1.094667in,y=0.378072in,,top]{\color{textcolor}\rmfamily\fontsize{8.000000}{9.600000}\selectfont \(\displaystyle {10^{5}}\)}%
\end{pgfscope}%
\begin{pgfscope}%
\pgfpathrectangle{\pgfqpoint{0.578368in}{0.468349in}}{\pgfqpoint{1.278750in}{1.245750in}}%
\pgfusepath{clip}%
\pgfsetroundcap%
\pgfsetroundjoin%
\pgfsetlinewidth{0.501875pt}%
\definecolor{currentstroke}{rgb}{0.800000,0.800000,0.800000}%
\pgfsetstrokecolor{currentstroke}%
\pgfsetdash{}{0pt}%
\pgfpathmoveto{\pgfqpoint{1.446830in}{0.468349in}}%
\pgfpathlineto{\pgfqpoint{1.446830in}{1.714099in}}%
\pgfusepath{stroke}%
\end{pgfscope}%
\begin{pgfscope}%
\definecolor{textcolor}{rgb}{0.150000,0.150000,0.150000}%
\pgfsetstrokecolor{textcolor}%
\pgfsetfillcolor{textcolor}%
\pgftext[x=1.446830in,y=0.378072in,,top]{\color{textcolor}\rmfamily\fontsize{8.000000}{9.600000}\selectfont \(\displaystyle {10^{6}}\)}%
\end{pgfscope}%
\begin{pgfscope}%
\pgfpathrectangle{\pgfqpoint{0.578368in}{0.468349in}}{\pgfqpoint{1.278750in}{1.245750in}}%
\pgfusepath{clip}%
\pgfsetroundcap%
\pgfsetroundjoin%
\pgfsetlinewidth{0.501875pt}%
\definecolor{currentstroke}{rgb}{0.800000,0.800000,0.800000}%
\pgfsetstrokecolor{currentstroke}%
\pgfsetdash{}{0pt}%
\pgfpathmoveto{\pgfqpoint{1.798993in}{0.468349in}}%
\pgfpathlineto{\pgfqpoint{1.798993in}{1.714099in}}%
\pgfusepath{stroke}%
\end{pgfscope}%
\begin{pgfscope}%
\definecolor{textcolor}{rgb}{0.150000,0.150000,0.150000}%
\pgfsetstrokecolor{textcolor}%
\pgfsetfillcolor{textcolor}%
\pgftext[x=1.798993in,y=0.378072in,,top]{\color{textcolor}\rmfamily\fontsize{8.000000}{9.600000}\selectfont \(\displaystyle {10^{7}}\)}%
\end{pgfscope}%
\begin{pgfscope}%
\definecolor{textcolor}{rgb}{0.150000,0.150000,0.150000}%
\pgfsetstrokecolor{textcolor}%
\pgfsetfillcolor{textcolor}%
\pgftext[x=1.217743in,y=0.222655in,,top]{\color{textcolor}\rmfamily\fontsize{10.000000}{12.000000}\selectfont Block size \(\displaystyle b\)}%
\end{pgfscope}%
\begin{pgfscope}%
\pgfpathrectangle{\pgfqpoint{0.578368in}{0.468349in}}{\pgfqpoint{1.278750in}{1.245750in}}%
\pgfusepath{clip}%
\pgfsetroundcap%
\pgfsetroundjoin%
\pgfsetlinewidth{0.501875pt}%
\definecolor{currentstroke}{rgb}{0.800000,0.800000,0.800000}%
\pgfsetstrokecolor{currentstroke}%
\pgfsetdash{}{0pt}%
\pgfpathmoveto{\pgfqpoint{0.578368in}{0.781918in}}%
\pgfpathlineto{\pgfqpoint{1.857118in}{0.781918in}}%
\pgfusepath{stroke}%
\end{pgfscope}%
\begin{pgfscope}%
\definecolor{textcolor}{rgb}{0.150000,0.150000,0.150000}%
\pgfsetstrokecolor{textcolor}%
\pgfsetfillcolor{textcolor}%
\pgftext[x=0.278211in, y=0.743656in, left, base]{\color{textcolor}\rmfamily\fontsize{8.000000}{9.600000}\selectfont \(\displaystyle {0.65}\)}%
\end{pgfscope}%
\begin{pgfscope}%
\pgfpathrectangle{\pgfqpoint{0.578368in}{0.468349in}}{\pgfqpoint{1.278750in}{1.245750in}}%
\pgfusepath{clip}%
\pgfsetroundcap%
\pgfsetroundjoin%
\pgfsetlinewidth{0.501875pt}%
\definecolor{currentstroke}{rgb}{0.800000,0.800000,0.800000}%
\pgfsetstrokecolor{currentstroke}%
\pgfsetdash{}{0pt}%
\pgfpathmoveto{\pgfqpoint{0.578368in}{1.196101in}}%
\pgfpathlineto{\pgfqpoint{1.857118in}{1.196101in}}%
\pgfusepath{stroke}%
\end{pgfscope}%
\begin{pgfscope}%
\definecolor{textcolor}{rgb}{0.150000,0.150000,0.150000}%
\pgfsetstrokecolor{textcolor}%
\pgfsetfillcolor{textcolor}%
\pgftext[x=0.278211in, y=1.157839in, left, base]{\color{textcolor}\rmfamily\fontsize{8.000000}{9.600000}\selectfont \(\displaystyle {0.70}\)}%
\end{pgfscope}%
\begin{pgfscope}%
\pgfpathrectangle{\pgfqpoint{0.578368in}{0.468349in}}{\pgfqpoint{1.278750in}{1.245750in}}%
\pgfusepath{clip}%
\pgfsetroundcap%
\pgfsetroundjoin%
\pgfsetlinewidth{0.501875pt}%
\definecolor{currentstroke}{rgb}{0.800000,0.800000,0.800000}%
\pgfsetstrokecolor{currentstroke}%
\pgfsetdash{}{0pt}%
\pgfpathmoveto{\pgfqpoint{0.578368in}{1.610284in}}%
\pgfpathlineto{\pgfqpoint{1.857118in}{1.610284in}}%
\pgfusepath{stroke}%
\end{pgfscope}%
\begin{pgfscope}%
\definecolor{textcolor}{rgb}{0.150000,0.150000,0.150000}%
\pgfsetstrokecolor{textcolor}%
\pgfsetfillcolor{textcolor}%
\pgftext[x=0.278211in, y=1.572022in, left, base]{\color{textcolor}\rmfamily\fontsize{8.000000}{9.600000}\selectfont \(\displaystyle {0.75}\)}%
\end{pgfscope}%
\begin{pgfscope}%
\definecolor{textcolor}{rgb}{0.150000,0.150000,0.150000}%
\pgfsetstrokecolor{textcolor}%
\pgfsetfillcolor{textcolor}%
\pgftext[x=0.222655in,y=1.091224in,,bottom,rotate=90.000000]{\color{textcolor}\rmfamily\fontsize{10.000000}{12.000000}\selectfont Accuracy}%
\end{pgfscope}%
\begin{pgfscope}%
\pgfpathrectangle{\pgfqpoint{0.578368in}{0.468349in}}{\pgfqpoint{1.278750in}{1.245750in}}%
\pgfusepath{clip}%
\pgfsetbuttcap%
\pgfsetroundjoin%
\pgfsetlinewidth{1.003750pt}%
\definecolor{currentstroke}{rgb}{0.298039,0.447059,0.690196}%
\pgfsetstrokecolor{currentstroke}%
\pgfsetdash{}{0pt}%
\pgfpathmoveto{\pgfqpoint{0.636493in}{0.625052in}}%
\pgfpathlineto{\pgfqpoint{0.636493in}{0.640835in}}%
\pgfusepath{stroke}%
\end{pgfscope}%
\begin{pgfscope}%
\pgfpathrectangle{\pgfqpoint{0.578368in}{0.468349in}}{\pgfqpoint{1.278750in}{1.245750in}}%
\pgfusepath{clip}%
\pgfsetbuttcap%
\pgfsetroundjoin%
\pgfsetlinewidth{1.003750pt}%
\definecolor{currentstroke}{rgb}{0.298039,0.447059,0.690196}%
\pgfsetstrokecolor{currentstroke}%
\pgfsetdash{}{0pt}%
\pgfpathmoveto{\pgfqpoint{0.742505in}{0.625052in}}%
\pgfpathlineto{\pgfqpoint{0.742505in}{0.640835in}}%
\pgfusepath{stroke}%
\end{pgfscope}%
\begin{pgfscope}%
\pgfpathrectangle{\pgfqpoint{0.578368in}{0.468349in}}{\pgfqpoint{1.278750in}{1.245750in}}%
\pgfusepath{clip}%
\pgfsetbuttcap%
\pgfsetroundjoin%
\pgfsetlinewidth{1.003750pt}%
\definecolor{currentstroke}{rgb}{0.298039,0.447059,0.690196}%
\pgfsetstrokecolor{currentstroke}%
\pgfsetdash{}{0pt}%
\pgfpathmoveto{\pgfqpoint{1.094667in}{0.625052in}}%
\pgfpathlineto{\pgfqpoint{1.094667in}{0.640835in}}%
\pgfusepath{stroke}%
\end{pgfscope}%
\begin{pgfscope}%
\pgfpathrectangle{\pgfqpoint{0.578368in}{0.468349in}}{\pgfqpoint{1.278750in}{1.245750in}}%
\pgfusepath{clip}%
\pgfsetbuttcap%
\pgfsetroundjoin%
\pgfsetlinewidth{1.003750pt}%
\definecolor{currentstroke}{rgb}{0.298039,0.447059,0.690196}%
\pgfsetstrokecolor{currentstroke}%
\pgfsetdash{}{0pt}%
\pgfpathmoveto{\pgfqpoint{1.446830in}{0.625052in}}%
\pgfpathlineto{\pgfqpoint{1.446830in}{0.640835in}}%
\pgfusepath{stroke}%
\end{pgfscope}%
\begin{pgfscope}%
\pgfpathrectangle{\pgfqpoint{0.578368in}{0.468349in}}{\pgfqpoint{1.278750in}{1.245750in}}%
\pgfusepath{clip}%
\pgfsetbuttcap%
\pgfsetroundjoin%
\pgfsetlinewidth{1.003750pt}%
\definecolor{currentstroke}{rgb}{0.298039,0.447059,0.690196}%
\pgfsetstrokecolor{currentstroke}%
\pgfsetdash{}{0pt}%
\pgfpathmoveto{\pgfqpoint{1.798993in}{0.625052in}}%
\pgfpathlineto{\pgfqpoint{1.798993in}{0.640835in}}%
\pgfusepath{stroke}%
\end{pgfscope}%
\begin{pgfscope}%
\pgfpathrectangle{\pgfqpoint{0.578368in}{0.468349in}}{\pgfqpoint{1.278750in}{1.245750in}}%
\pgfusepath{clip}%
\pgfsetbuttcap%
\pgfsetroundjoin%
\pgfsetlinewidth{1.003750pt}%
\definecolor{currentstroke}{rgb}{0.866667,0.517647,0.321569}%
\pgfsetstrokecolor{currentstroke}%
\pgfsetdash{}{0pt}%
\pgfpathmoveto{\pgfqpoint{0.636493in}{0.764691in}}%
\pgfpathlineto{\pgfqpoint{0.636493in}{0.781465in}}%
\pgfusepath{stroke}%
\end{pgfscope}%
\begin{pgfscope}%
\pgfpathrectangle{\pgfqpoint{0.578368in}{0.468349in}}{\pgfqpoint{1.278750in}{1.245750in}}%
\pgfusepath{clip}%
\pgfsetbuttcap%
\pgfsetroundjoin%
\pgfsetlinewidth{1.003750pt}%
\definecolor{currentstroke}{rgb}{0.866667,0.517647,0.321569}%
\pgfsetstrokecolor{currentstroke}%
\pgfsetdash{}{0pt}%
\pgfpathmoveto{\pgfqpoint{0.742505in}{0.764691in}}%
\pgfpathlineto{\pgfqpoint{0.742505in}{0.781465in}}%
\pgfusepath{stroke}%
\end{pgfscope}%
\begin{pgfscope}%
\pgfpathrectangle{\pgfqpoint{0.578368in}{0.468349in}}{\pgfqpoint{1.278750in}{1.245750in}}%
\pgfusepath{clip}%
\pgfsetbuttcap%
\pgfsetroundjoin%
\pgfsetlinewidth{1.003750pt}%
\definecolor{currentstroke}{rgb}{0.866667,0.517647,0.321569}%
\pgfsetstrokecolor{currentstroke}%
\pgfsetdash{}{0pt}%
\pgfpathmoveto{\pgfqpoint{1.094667in}{0.764691in}}%
\pgfpathlineto{\pgfqpoint{1.094667in}{0.781465in}}%
\pgfusepath{stroke}%
\end{pgfscope}%
\begin{pgfscope}%
\pgfpathrectangle{\pgfqpoint{0.578368in}{0.468349in}}{\pgfqpoint{1.278750in}{1.245750in}}%
\pgfusepath{clip}%
\pgfsetbuttcap%
\pgfsetroundjoin%
\pgfsetlinewidth{1.003750pt}%
\definecolor{currentstroke}{rgb}{0.866667,0.517647,0.321569}%
\pgfsetstrokecolor{currentstroke}%
\pgfsetdash{}{0pt}%
\pgfpathmoveto{\pgfqpoint{1.446830in}{0.764691in}}%
\pgfpathlineto{\pgfqpoint{1.446830in}{0.781465in}}%
\pgfusepath{stroke}%
\end{pgfscope}%
\begin{pgfscope}%
\pgfpathrectangle{\pgfqpoint{0.578368in}{0.468349in}}{\pgfqpoint{1.278750in}{1.245750in}}%
\pgfusepath{clip}%
\pgfsetbuttcap%
\pgfsetroundjoin%
\pgfsetlinewidth{1.003750pt}%
\definecolor{currentstroke}{rgb}{0.866667,0.517647,0.321569}%
\pgfsetstrokecolor{currentstroke}%
\pgfsetdash{}{0pt}%
\pgfpathmoveto{\pgfqpoint{1.798993in}{0.764691in}}%
\pgfpathlineto{\pgfqpoint{1.798993in}{0.781465in}}%
\pgfusepath{stroke}%
\end{pgfscope}%
\begin{pgfscope}%
\pgfpathrectangle{\pgfqpoint{0.578368in}{0.468349in}}{\pgfqpoint{1.278750in}{1.245750in}}%
\pgfusepath{clip}%
\pgfsetbuttcap%
\pgfsetroundjoin%
\pgfsetlinewidth{1.003750pt}%
\definecolor{currentstroke}{rgb}{0.333333,0.658824,0.407843}%
\pgfsetstrokecolor{currentstroke}%
\pgfsetdash{}{0pt}%
\pgfpathmoveto{\pgfqpoint{0.636493in}{1.608289in}}%
\pgfpathlineto{\pgfqpoint{0.636493in}{1.644366in}}%
\pgfusepath{stroke}%
\end{pgfscope}%
\begin{pgfscope}%
\pgfpathrectangle{\pgfqpoint{0.578368in}{0.468349in}}{\pgfqpoint{1.278750in}{1.245750in}}%
\pgfusepath{clip}%
\pgfsetbuttcap%
\pgfsetroundjoin%
\pgfsetlinewidth{1.003750pt}%
\definecolor{currentstroke}{rgb}{0.333333,0.658824,0.407843}%
\pgfsetstrokecolor{currentstroke}%
\pgfsetdash{}{0pt}%
\pgfpathmoveto{\pgfqpoint{0.742505in}{1.608289in}}%
\pgfpathlineto{\pgfqpoint{0.742505in}{1.644366in}}%
\pgfusepath{stroke}%
\end{pgfscope}%
\begin{pgfscope}%
\pgfpathrectangle{\pgfqpoint{0.578368in}{0.468349in}}{\pgfqpoint{1.278750in}{1.245750in}}%
\pgfusepath{clip}%
\pgfsetbuttcap%
\pgfsetroundjoin%
\pgfsetlinewidth{1.003750pt}%
\definecolor{currentstroke}{rgb}{0.333333,0.658824,0.407843}%
\pgfsetstrokecolor{currentstroke}%
\pgfsetdash{}{0pt}%
\pgfpathmoveto{\pgfqpoint{1.094667in}{1.608289in}}%
\pgfpathlineto{\pgfqpoint{1.094667in}{1.644366in}}%
\pgfusepath{stroke}%
\end{pgfscope}%
\begin{pgfscope}%
\pgfpathrectangle{\pgfqpoint{0.578368in}{0.468349in}}{\pgfqpoint{1.278750in}{1.245750in}}%
\pgfusepath{clip}%
\pgfsetbuttcap%
\pgfsetroundjoin%
\pgfsetlinewidth{1.003750pt}%
\definecolor{currentstroke}{rgb}{0.333333,0.658824,0.407843}%
\pgfsetstrokecolor{currentstroke}%
\pgfsetdash{}{0pt}%
\pgfpathmoveto{\pgfqpoint{1.446830in}{1.608289in}}%
\pgfpathlineto{\pgfqpoint{1.446830in}{1.644366in}}%
\pgfusepath{stroke}%
\end{pgfscope}%
\begin{pgfscope}%
\pgfpathrectangle{\pgfqpoint{0.578368in}{0.468349in}}{\pgfqpoint{1.278750in}{1.245750in}}%
\pgfusepath{clip}%
\pgfsetbuttcap%
\pgfsetroundjoin%
\pgfsetlinewidth{1.003750pt}%
\definecolor{currentstroke}{rgb}{0.333333,0.658824,0.407843}%
\pgfsetstrokecolor{currentstroke}%
\pgfsetdash{}{0pt}%
\pgfpathmoveto{\pgfqpoint{1.798993in}{1.608289in}}%
\pgfpathlineto{\pgfqpoint{1.798993in}{1.644366in}}%
\pgfusepath{stroke}%
\end{pgfscope}%
\begin{pgfscope}%
\pgfpathrectangle{\pgfqpoint{0.578368in}{0.468349in}}{\pgfqpoint{1.278750in}{1.245750in}}%
\pgfusepath{clip}%
\pgfsetbuttcap%
\pgfsetroundjoin%
\pgfsetlinewidth{1.003750pt}%
\definecolor{currentstroke}{rgb}{0.298039,0.447059,0.690196}%
\pgfsetstrokecolor{currentstroke}%
\pgfsetdash{}{0pt}%
\pgfpathmoveto{\pgfqpoint{0.636493in}{0.538067in}}%
\pgfpathlineto{\pgfqpoint{0.636493in}{0.667575in}}%
\pgfusepath{stroke}%
\end{pgfscope}%
\begin{pgfscope}%
\pgfpathrectangle{\pgfqpoint{0.578368in}{0.468349in}}{\pgfqpoint{1.278750in}{1.245750in}}%
\pgfusepath{clip}%
\pgfsetbuttcap%
\pgfsetroundjoin%
\pgfsetlinewidth{1.003750pt}%
\definecolor{currentstroke}{rgb}{0.298039,0.447059,0.690196}%
\pgfsetstrokecolor{currentstroke}%
\pgfsetdash{}{0pt}%
\pgfpathmoveto{\pgfqpoint{0.742505in}{0.524974in}}%
\pgfpathlineto{\pgfqpoint{0.742505in}{0.647926in}}%
\pgfusepath{stroke}%
\end{pgfscope}%
\begin{pgfscope}%
\pgfpathrectangle{\pgfqpoint{0.578368in}{0.468349in}}{\pgfqpoint{1.278750in}{1.245750in}}%
\pgfusepath{clip}%
\pgfsetbuttcap%
\pgfsetroundjoin%
\pgfsetlinewidth{1.003750pt}%
\definecolor{currentstroke}{rgb}{0.298039,0.447059,0.690196}%
\pgfsetstrokecolor{currentstroke}%
\pgfsetdash{}{0pt}%
\pgfpathmoveto{\pgfqpoint{1.094667in}{0.597287in}}%
\pgfpathlineto{\pgfqpoint{1.094667in}{0.707890in}}%
\pgfusepath{stroke}%
\end{pgfscope}%
\begin{pgfscope}%
\pgfpathrectangle{\pgfqpoint{0.578368in}{0.468349in}}{\pgfqpoint{1.278750in}{1.245750in}}%
\pgfusepath{clip}%
\pgfsetbuttcap%
\pgfsetroundjoin%
\pgfsetlinewidth{1.003750pt}%
\definecolor{currentstroke}{rgb}{0.298039,0.447059,0.690196}%
\pgfsetstrokecolor{currentstroke}%
\pgfsetdash{}{0pt}%
\pgfpathmoveto{\pgfqpoint{1.446830in}{0.638855in}}%
\pgfpathlineto{\pgfqpoint{1.446830in}{0.740973in}}%
\pgfusepath{stroke}%
\end{pgfscope}%
\begin{pgfscope}%
\pgfpathrectangle{\pgfqpoint{0.578368in}{0.468349in}}{\pgfqpoint{1.278750in}{1.245750in}}%
\pgfusepath{clip}%
\pgfsetbuttcap%
\pgfsetroundjoin%
\pgfsetlinewidth{1.003750pt}%
\definecolor{currentstroke}{rgb}{0.298039,0.447059,0.690196}%
\pgfsetstrokecolor{currentstroke}%
\pgfsetdash{}{0pt}%
\pgfpathmoveto{\pgfqpoint{1.798993in}{0.614827in}}%
\pgfpathlineto{\pgfqpoint{1.798993in}{0.719163in}}%
\pgfusepath{stroke}%
\end{pgfscope}%
\begin{pgfscope}%
\pgfpathrectangle{\pgfqpoint{0.578368in}{0.468349in}}{\pgfqpoint{1.278750in}{1.245750in}}%
\pgfusepath{clip}%
\pgfsetbuttcap%
\pgfsetroundjoin%
\pgfsetlinewidth{1.003750pt}%
\definecolor{currentstroke}{rgb}{0.866667,0.517647,0.321569}%
\pgfsetstrokecolor{currentstroke}%
\pgfsetdash{}{0pt}%
\pgfpathmoveto{\pgfqpoint{0.636493in}{0.691867in}}%
\pgfpathlineto{\pgfqpoint{0.636493in}{0.742967in}}%
\pgfusepath{stroke}%
\end{pgfscope}%
\begin{pgfscope}%
\pgfpathrectangle{\pgfqpoint{0.578368in}{0.468349in}}{\pgfqpoint{1.278750in}{1.245750in}}%
\pgfusepath{clip}%
\pgfsetbuttcap%
\pgfsetroundjoin%
\pgfsetlinewidth{1.003750pt}%
\definecolor{currentstroke}{rgb}{0.866667,0.517647,0.321569}%
\pgfsetstrokecolor{currentstroke}%
\pgfsetdash{}{0pt}%
\pgfpathmoveto{\pgfqpoint{0.742505in}{0.675290in}}%
\pgfpathlineto{\pgfqpoint{0.742505in}{0.734660in}}%
\pgfusepath{stroke}%
\end{pgfscope}%
\begin{pgfscope}%
\pgfpathrectangle{\pgfqpoint{0.578368in}{0.468349in}}{\pgfqpoint{1.278750in}{1.245750in}}%
\pgfusepath{clip}%
\pgfsetbuttcap%
\pgfsetroundjoin%
\pgfsetlinewidth{1.003750pt}%
\definecolor{currentstroke}{rgb}{0.866667,0.517647,0.321569}%
\pgfsetstrokecolor{currentstroke}%
\pgfsetdash{}{0pt}%
\pgfpathmoveto{\pgfqpoint{1.094667in}{0.775110in}}%
\pgfpathlineto{\pgfqpoint{1.094667in}{0.829982in}}%
\pgfusepath{stroke}%
\end{pgfscope}%
\begin{pgfscope}%
\pgfpathrectangle{\pgfqpoint{0.578368in}{0.468349in}}{\pgfqpoint{1.278750in}{1.245750in}}%
\pgfusepath{clip}%
\pgfsetbuttcap%
\pgfsetroundjoin%
\pgfsetlinewidth{1.003750pt}%
\definecolor{currentstroke}{rgb}{0.866667,0.517647,0.321569}%
\pgfsetstrokecolor{currentstroke}%
\pgfsetdash{}{0pt}%
\pgfpathmoveto{\pgfqpoint{1.446830in}{0.802791in}}%
\pgfpathlineto{\pgfqpoint{1.446830in}{0.850758in}}%
\pgfusepath{stroke}%
\end{pgfscope}%
\begin{pgfscope}%
\pgfpathrectangle{\pgfqpoint{0.578368in}{0.468349in}}{\pgfqpoint{1.278750in}{1.245750in}}%
\pgfusepath{clip}%
\pgfsetbuttcap%
\pgfsetroundjoin%
\pgfsetlinewidth{1.003750pt}%
\definecolor{currentstroke}{rgb}{0.866667,0.517647,0.321569}%
\pgfsetstrokecolor{currentstroke}%
\pgfsetdash{}{0pt}%
\pgfpathmoveto{\pgfqpoint{1.798993in}{0.774780in}}%
\pgfpathlineto{\pgfqpoint{1.798993in}{0.819834in}}%
\pgfusepath{stroke}%
\end{pgfscope}%
\begin{pgfscope}%
\pgfpathrectangle{\pgfqpoint{0.578368in}{0.468349in}}{\pgfqpoint{1.278750in}{1.245750in}}%
\pgfusepath{clip}%
\pgfsetbuttcap%
\pgfsetroundjoin%
\pgfsetlinewidth{1.003750pt}%
\definecolor{currentstroke}{rgb}{0.333333,0.658824,0.407843}%
\pgfsetstrokecolor{currentstroke}%
\pgfsetdash{}{0pt}%
\pgfpathmoveto{\pgfqpoint{0.636493in}{1.552231in}}%
\pgfpathlineto{\pgfqpoint{0.636493in}{1.629703in}}%
\pgfusepath{stroke}%
\end{pgfscope}%
\begin{pgfscope}%
\pgfpathrectangle{\pgfqpoint{0.578368in}{0.468349in}}{\pgfqpoint{1.278750in}{1.245750in}}%
\pgfusepath{clip}%
\pgfsetbuttcap%
\pgfsetroundjoin%
\pgfsetlinewidth{1.003750pt}%
\definecolor{currentstroke}{rgb}{0.333333,0.658824,0.407843}%
\pgfsetstrokecolor{currentstroke}%
\pgfsetdash{}{0pt}%
\pgfpathmoveto{\pgfqpoint{0.742505in}{1.547155in}}%
\pgfpathlineto{\pgfqpoint{0.742505in}{1.628230in}}%
\pgfusepath{stroke}%
\end{pgfscope}%
\begin{pgfscope}%
\pgfpathrectangle{\pgfqpoint{0.578368in}{0.468349in}}{\pgfqpoint{1.278750in}{1.245750in}}%
\pgfusepath{clip}%
\pgfsetbuttcap%
\pgfsetroundjoin%
\pgfsetlinewidth{1.003750pt}%
\definecolor{currentstroke}{rgb}{0.333333,0.658824,0.407843}%
\pgfsetstrokecolor{currentstroke}%
\pgfsetdash{}{0pt}%
\pgfpathmoveto{\pgfqpoint{1.094667in}{1.601729in}}%
\pgfpathlineto{\pgfqpoint{1.094667in}{1.657474in}}%
\pgfusepath{stroke}%
\end{pgfscope}%
\begin{pgfscope}%
\pgfpathrectangle{\pgfqpoint{0.578368in}{0.468349in}}{\pgfqpoint{1.278750in}{1.245750in}}%
\pgfusepath{clip}%
\pgfsetbuttcap%
\pgfsetroundjoin%
\pgfsetlinewidth{1.003750pt}%
\definecolor{currentstroke}{rgb}{0.333333,0.658824,0.407843}%
\pgfsetstrokecolor{currentstroke}%
\pgfsetdash{}{0pt}%
\pgfpathmoveto{\pgfqpoint{1.446830in}{1.554634in}}%
\pgfpathlineto{\pgfqpoint{1.446830in}{1.643015in}}%
\pgfusepath{stroke}%
\end{pgfscope}%
\begin{pgfscope}%
\pgfpathrectangle{\pgfqpoint{0.578368in}{0.468349in}}{\pgfqpoint{1.278750in}{1.245750in}}%
\pgfusepath{clip}%
\pgfsetbuttcap%
\pgfsetroundjoin%
\pgfsetlinewidth{1.003750pt}%
\definecolor{currentstroke}{rgb}{0.333333,0.658824,0.407843}%
\pgfsetstrokecolor{currentstroke}%
\pgfsetdash{}{0pt}%
\pgfpathmoveto{\pgfqpoint{1.798993in}{1.598360in}}%
\pgfpathlineto{\pgfqpoint{1.798993in}{1.650366in}}%
\pgfusepath{stroke}%
\end{pgfscope}%
\begin{pgfscope}%
\pgfpathrectangle{\pgfqpoint{0.578368in}{0.468349in}}{\pgfqpoint{1.278750in}{1.245750in}}%
\pgfusepath{clip}%
\pgfsetbuttcap%
\pgfsetroundjoin%
\pgfsetlinewidth{1.003750pt}%
\definecolor{currentstroke}{rgb}{0.298039,0.447059,0.690196}%
\pgfsetstrokecolor{currentstroke}%
\pgfsetdash{{6.400000pt}{1.600000pt}{1.000000pt}{1.600000pt}}{0.000000pt}%
\pgfpathmoveto{\pgfqpoint{0.636493in}{0.632943in}}%
\pgfpathlineto{\pgfqpoint{0.742505in}{0.632943in}}%
\pgfpathlineto{\pgfqpoint{1.094667in}{0.632943in}}%
\pgfpathlineto{\pgfqpoint{1.446830in}{0.632943in}}%
\pgfpathlineto{\pgfqpoint{1.798993in}{0.632943in}}%
\pgfusepath{stroke}%
\end{pgfscope}%
\begin{pgfscope}%
\pgfpathrectangle{\pgfqpoint{0.578368in}{0.468349in}}{\pgfqpoint{1.278750in}{1.245750in}}%
\pgfusepath{clip}%
\pgfsetbuttcap%
\pgfsetroundjoin%
\pgfsetlinewidth{1.003750pt}%
\definecolor{currentstroke}{rgb}{0.866667,0.517647,0.321569}%
\pgfsetstrokecolor{currentstroke}%
\pgfsetdash{{6.400000pt}{1.600000pt}{1.000000pt}{1.600000pt}}{0.000000pt}%
\pgfpathmoveto{\pgfqpoint{0.636493in}{0.773078in}}%
\pgfpathlineto{\pgfqpoint{0.742505in}{0.773078in}}%
\pgfpathlineto{\pgfqpoint{1.094667in}{0.773078in}}%
\pgfpathlineto{\pgfqpoint{1.446830in}{0.773078in}}%
\pgfpathlineto{\pgfqpoint{1.798993in}{0.773078in}}%
\pgfusepath{stroke}%
\end{pgfscope}%
\begin{pgfscope}%
\pgfpathrectangle{\pgfqpoint{0.578368in}{0.468349in}}{\pgfqpoint{1.278750in}{1.245750in}}%
\pgfusepath{clip}%
\pgfsetbuttcap%
\pgfsetroundjoin%
\pgfsetlinewidth{1.003750pt}%
\definecolor{currentstroke}{rgb}{0.333333,0.658824,0.407843}%
\pgfsetstrokecolor{currentstroke}%
\pgfsetdash{{6.400000pt}{1.600000pt}{1.000000pt}{1.600000pt}}{0.000000pt}%
\pgfpathmoveto{\pgfqpoint{0.636493in}{1.626328in}}%
\pgfpathlineto{\pgfqpoint{0.742505in}{1.626328in}}%
\pgfpathlineto{\pgfqpoint{1.094667in}{1.626328in}}%
\pgfpathlineto{\pgfqpoint{1.446830in}{1.626328in}}%
\pgfpathlineto{\pgfqpoint{1.798993in}{1.626328in}}%
\pgfusepath{stroke}%
\end{pgfscope}%
\begin{pgfscope}%
\pgfpathrectangle{\pgfqpoint{0.578368in}{0.468349in}}{\pgfqpoint{1.278750in}{1.245750in}}%
\pgfusepath{clip}%
\pgfsetroundcap%
\pgfsetroundjoin%
\pgfsetlinewidth{1.003750pt}%
\definecolor{currentstroke}{rgb}{0.298039,0.447059,0.690196}%
\pgfsetstrokecolor{currentstroke}%
\pgfsetdash{}{0pt}%
\pgfpathmoveto{\pgfqpoint{0.636493in}{0.602821in}}%
\pgfpathlineto{\pgfqpoint{0.742505in}{0.586450in}}%
\pgfpathlineto{\pgfqpoint{1.094667in}{0.652589in}}%
\pgfpathlineto{\pgfqpoint{1.446830in}{0.689914in}}%
\pgfpathlineto{\pgfqpoint{1.798993in}{0.666995in}}%
\pgfusepath{stroke}%
\end{pgfscope}%
\begin{pgfscope}%
\pgfpathrectangle{\pgfqpoint{0.578368in}{0.468349in}}{\pgfqpoint{1.278750in}{1.245750in}}%
\pgfusepath{clip}%
\pgfsetroundcap%
\pgfsetroundjoin%
\pgfsetlinewidth{1.003750pt}%
\definecolor{currentstroke}{rgb}{0.866667,0.517647,0.321569}%
\pgfsetstrokecolor{currentstroke}%
\pgfsetdash{}{0pt}%
\pgfpathmoveto{\pgfqpoint{0.636493in}{0.717417in}}%
\pgfpathlineto{\pgfqpoint{0.742505in}{0.704975in}}%
\pgfpathlineto{\pgfqpoint{1.094667in}{0.802546in}}%
\pgfpathlineto{\pgfqpoint{1.446830in}{0.826775in}}%
\pgfpathlineto{\pgfqpoint{1.798993in}{0.797307in}}%
\pgfusepath{stroke}%
\end{pgfscope}%
\begin{pgfscope}%
\pgfpathrectangle{\pgfqpoint{0.578368in}{0.468349in}}{\pgfqpoint{1.278750in}{1.245750in}}%
\pgfusepath{clip}%
\pgfsetroundcap%
\pgfsetroundjoin%
\pgfsetlinewidth{1.003750pt}%
\definecolor{currentstroke}{rgb}{0.333333,0.658824,0.407843}%
\pgfsetstrokecolor{currentstroke}%
\pgfsetdash{}{0pt}%
\pgfpathmoveto{\pgfqpoint{0.636493in}{1.590967in}}%
\pgfpathlineto{\pgfqpoint{0.742505in}{1.587693in}}%
\pgfpathlineto{\pgfqpoint{1.094667in}{1.629602in}}%
\pgfpathlineto{\pgfqpoint{1.446830in}{1.598825in}}%
\pgfpathlineto{\pgfqpoint{1.798993in}{1.624363in}}%
\pgfusepath{stroke}%
\end{pgfscope}%
\begin{pgfscope}%
\pgfsetrectcap%
\pgfsetmiterjoin%
\pgfsetlinewidth{0.752812pt}%
\definecolor{currentstroke}{rgb}{0.700000,0.700000,0.700000}%
\pgfsetstrokecolor{currentstroke}%
\pgfsetdash{}{0pt}%
\pgfpathmoveto{\pgfqpoint{0.578368in}{0.468349in}}%
\pgfpathlineto{\pgfqpoint{0.578368in}{1.714099in}}%
\pgfusepath{stroke}%
\end{pgfscope}%
\begin{pgfscope}%
\pgfsetrectcap%
\pgfsetmiterjoin%
\pgfsetlinewidth{0.752812pt}%
\definecolor{currentstroke}{rgb}{0.700000,0.700000,0.700000}%
\pgfsetstrokecolor{currentstroke}%
\pgfsetdash{}{0pt}%
\pgfpathmoveto{\pgfqpoint{1.857118in}{0.468349in}}%
\pgfpathlineto{\pgfqpoint{1.857118in}{1.714099in}}%
\pgfusepath{stroke}%
\end{pgfscope}%
\begin{pgfscope}%
\pgfsetrectcap%
\pgfsetmiterjoin%
\pgfsetlinewidth{0.752812pt}%
\definecolor{currentstroke}{rgb}{0.700000,0.700000,0.700000}%
\pgfsetstrokecolor{currentstroke}%
\pgfsetdash{}{0pt}%
\pgfpathmoveto{\pgfqpoint{0.578368in}{0.468349in}}%
\pgfpathlineto{\pgfqpoint{1.857118in}{0.468349in}}%
\pgfusepath{stroke}%
\end{pgfscope}%
\begin{pgfscope}%
\pgfsetrectcap%
\pgfsetmiterjoin%
\pgfsetlinewidth{0.752812pt}%
\definecolor{currentstroke}{rgb}{0.700000,0.700000,0.700000}%
\pgfsetstrokecolor{currentstroke}%
\pgfsetdash{}{0pt}%
\pgfpathmoveto{\pgfqpoint{0.578368in}{1.714099in}}%
\pgfpathlineto{\pgfqpoint{1.857118in}{1.714099in}}%
\pgfusepath{stroke}%
\end{pgfscope}%
\end{pgfpicture}%
\makeatother%
\endgroup%
}} &
      \subfloat[\(\epsilon=0.25\) (i.e. \(\Delta=1995\)]{\resizebox{0.48\linewidth}{!}{%% Creator: Matplotlib, PGF backend
%%
%% To include the figure in your LaTeX document, write
%%   \input{<filename>.pgf}
%%
%% Make sure the required packages are loaded in your preamble
%%   \usepackage{pgf}
%%
%% and, on pdftex
%%   \usepackage[utf8]{inputenc}\DeclareUnicodeCharacter{2212}{-}
%%
%% or, on luatex and xetex
%%   \usepackage{unicode-math}
%%
%% Figures using additional raster images can only be included by \input if
%% they are in the same directory as the main LaTeX file. For loading figures
%% from other directories you can use the `import` package
%%   \usepackage{import}
%%
%% and then include the figures with
%%   \import{<path to file>}{<filename>.pgf}
%%
%% Matplotlib used the following preamble
%%   \usepackage[utf8]{inputenc}
%%   \usepackage[T1]{fontenc}
%%   \usepackage{amsmath}
%%   \newcommand*{\mat}[1]{\boldsymbol{#1}}
%%
\begingroup%
\makeatletter%
\begin{pgfpicture}%
\pgfpathrectangle{\pgfpointorigin}{\pgfqpoint{1.927928in}{1.814099in}}%
\pgfusepath{use as bounding box, clip}%
\begin{pgfscope}%
\pgfsetbuttcap%
\pgfsetmiterjoin%
\definecolor{currentfill}{rgb}{1.000000,1.000000,1.000000}%
\pgfsetfillcolor{currentfill}%
\pgfsetlinewidth{0.000000pt}%
\definecolor{currentstroke}{rgb}{1.000000,1.000000,1.000000}%
\pgfsetstrokecolor{currentstroke}%
\pgfsetstrokeopacity{0.000000}%
\pgfsetdash{}{0pt}%
\pgfpathmoveto{\pgfqpoint{0.000000in}{0.000000in}}%
\pgfpathlineto{\pgfqpoint{1.927928in}{0.000000in}}%
\pgfpathlineto{\pgfqpoint{1.927928in}{1.814099in}}%
\pgfpathlineto{\pgfqpoint{0.000000in}{1.814099in}}%
\pgfpathclose%
\pgfusepath{fill}%
\end{pgfscope}%
\begin{pgfscope}%
\pgfsetbuttcap%
\pgfsetmiterjoin%
\definecolor{currentfill}{rgb}{1.000000,1.000000,1.000000}%
\pgfsetfillcolor{currentfill}%
\pgfsetlinewidth{0.000000pt}%
\definecolor{currentstroke}{rgb}{0.000000,0.000000,0.000000}%
\pgfsetstrokecolor{currentstroke}%
\pgfsetstrokeopacity{0.000000}%
\pgfsetdash{}{0pt}%
\pgfpathmoveto{\pgfqpoint{0.519339in}{0.468349in}}%
\pgfpathlineto{\pgfqpoint{1.798089in}{0.468349in}}%
\pgfpathlineto{\pgfqpoint{1.798089in}{1.714099in}}%
\pgfpathlineto{\pgfqpoint{0.519339in}{1.714099in}}%
\pgfpathclose%
\pgfusepath{fill}%
\end{pgfscope}%
\begin{pgfscope}%
\pgfpathrectangle{\pgfqpoint{0.519339in}{0.468349in}}{\pgfqpoint{1.278750in}{1.245750in}}%
\pgfusepath{clip}%
\pgfsetroundcap%
\pgfsetroundjoin%
\pgfsetlinewidth{0.501875pt}%
\definecolor{currentstroke}{rgb}{0.800000,0.800000,0.800000}%
\pgfsetstrokecolor{currentstroke}%
\pgfsetdash{}{0pt}%
\pgfpathmoveto{\pgfqpoint{0.683476in}{0.468349in}}%
\pgfpathlineto{\pgfqpoint{0.683476in}{1.714099in}}%
\pgfusepath{stroke}%
\end{pgfscope}%
\begin{pgfscope}%
\definecolor{textcolor}{rgb}{0.150000,0.150000,0.150000}%
\pgfsetstrokecolor{textcolor}%
\pgfsetfillcolor{textcolor}%
\pgftext[x=0.683476in,y=0.378072in,,top]{\color{textcolor}\rmfamily\fontsize{8.000000}{9.600000}\selectfont \(\displaystyle {10^{4}}\)}%
\end{pgfscope}%
\begin{pgfscope}%
\pgfpathrectangle{\pgfqpoint{0.519339in}{0.468349in}}{\pgfqpoint{1.278750in}{1.245750in}}%
\pgfusepath{clip}%
\pgfsetroundcap%
\pgfsetroundjoin%
\pgfsetlinewidth{0.501875pt}%
\definecolor{currentstroke}{rgb}{0.800000,0.800000,0.800000}%
\pgfsetstrokecolor{currentstroke}%
\pgfsetdash{}{0pt}%
\pgfpathmoveto{\pgfqpoint{1.035639in}{0.468349in}}%
\pgfpathlineto{\pgfqpoint{1.035639in}{1.714099in}}%
\pgfusepath{stroke}%
\end{pgfscope}%
\begin{pgfscope}%
\definecolor{textcolor}{rgb}{0.150000,0.150000,0.150000}%
\pgfsetstrokecolor{textcolor}%
\pgfsetfillcolor{textcolor}%
\pgftext[x=1.035639in,y=0.378072in,,top]{\color{textcolor}\rmfamily\fontsize{8.000000}{9.600000}\selectfont \(\displaystyle {10^{5}}\)}%
\end{pgfscope}%
\begin{pgfscope}%
\pgfpathrectangle{\pgfqpoint{0.519339in}{0.468349in}}{\pgfqpoint{1.278750in}{1.245750in}}%
\pgfusepath{clip}%
\pgfsetroundcap%
\pgfsetroundjoin%
\pgfsetlinewidth{0.501875pt}%
\definecolor{currentstroke}{rgb}{0.800000,0.800000,0.800000}%
\pgfsetstrokecolor{currentstroke}%
\pgfsetdash{}{0pt}%
\pgfpathmoveto{\pgfqpoint{1.387802in}{0.468349in}}%
\pgfpathlineto{\pgfqpoint{1.387802in}{1.714099in}}%
\pgfusepath{stroke}%
\end{pgfscope}%
\begin{pgfscope}%
\definecolor{textcolor}{rgb}{0.150000,0.150000,0.150000}%
\pgfsetstrokecolor{textcolor}%
\pgfsetfillcolor{textcolor}%
\pgftext[x=1.387802in,y=0.378072in,,top]{\color{textcolor}\rmfamily\fontsize{8.000000}{9.600000}\selectfont \(\displaystyle {10^{6}}\)}%
\end{pgfscope}%
\begin{pgfscope}%
\pgfpathrectangle{\pgfqpoint{0.519339in}{0.468349in}}{\pgfqpoint{1.278750in}{1.245750in}}%
\pgfusepath{clip}%
\pgfsetroundcap%
\pgfsetroundjoin%
\pgfsetlinewidth{0.501875pt}%
\definecolor{currentstroke}{rgb}{0.800000,0.800000,0.800000}%
\pgfsetstrokecolor{currentstroke}%
\pgfsetdash{}{0pt}%
\pgfpathmoveto{\pgfqpoint{1.739964in}{0.468349in}}%
\pgfpathlineto{\pgfqpoint{1.739964in}{1.714099in}}%
\pgfusepath{stroke}%
\end{pgfscope}%
\begin{pgfscope}%
\definecolor{textcolor}{rgb}{0.150000,0.150000,0.150000}%
\pgfsetstrokecolor{textcolor}%
\pgfsetfillcolor{textcolor}%
\pgftext[x=1.739964in,y=0.378072in,,top]{\color{textcolor}\rmfamily\fontsize{8.000000}{9.600000}\selectfont \(\displaystyle {10^{7}}\)}%
\end{pgfscope}%
\begin{pgfscope}%
\definecolor{textcolor}{rgb}{0.150000,0.150000,0.150000}%
\pgfsetstrokecolor{textcolor}%
\pgfsetfillcolor{textcolor}%
\pgftext[x=1.158714in,y=0.222655in,,top]{\color{textcolor}\rmfamily\fontsize{10.000000}{12.000000}\selectfont Block size \(\displaystyle b\)}%
\end{pgfscope}%
\begin{pgfscope}%
\pgfpathrectangle{\pgfqpoint{0.519339in}{0.468349in}}{\pgfqpoint{1.278750in}{1.245750in}}%
\pgfusepath{clip}%
\pgfsetroundcap%
\pgfsetroundjoin%
\pgfsetlinewidth{0.501875pt}%
\definecolor{currentstroke}{rgb}{0.800000,0.800000,0.800000}%
\pgfsetstrokecolor{currentstroke}%
\pgfsetdash{}{0pt}%
\pgfpathmoveto{\pgfqpoint{0.519339in}{0.626756in}}%
\pgfpathlineto{\pgfqpoint{1.798089in}{0.626756in}}%
\pgfusepath{stroke}%
\end{pgfscope}%
\begin{pgfscope}%
\definecolor{textcolor}{rgb}{0.150000,0.150000,0.150000}%
\pgfsetstrokecolor{textcolor}%
\pgfsetfillcolor{textcolor}%
\pgftext[x=0.278211in, y=0.588494in, left, base]{\color{textcolor}\rmfamily\fontsize{8.000000}{9.600000}\selectfont \(\displaystyle {0.5}\)}%
\end{pgfscope}%
\begin{pgfscope}%
\pgfpathrectangle{\pgfqpoint{0.519339in}{0.468349in}}{\pgfqpoint{1.278750in}{1.245750in}}%
\pgfusepath{clip}%
\pgfsetroundcap%
\pgfsetroundjoin%
\pgfsetlinewidth{0.501875pt}%
\definecolor{currentstroke}{rgb}{0.800000,0.800000,0.800000}%
\pgfsetstrokecolor{currentstroke}%
\pgfsetdash{}{0pt}%
\pgfpathmoveto{\pgfqpoint{0.519339in}{1.086245in}}%
\pgfpathlineto{\pgfqpoint{1.798089in}{1.086245in}}%
\pgfusepath{stroke}%
\end{pgfscope}%
\begin{pgfscope}%
\definecolor{textcolor}{rgb}{0.150000,0.150000,0.150000}%
\pgfsetstrokecolor{textcolor}%
\pgfsetfillcolor{textcolor}%
\pgftext[x=0.278211in, y=1.047982in, left, base]{\color{textcolor}\rmfamily\fontsize{8.000000}{9.600000}\selectfont \(\displaystyle {0.6}\)}%
\end{pgfscope}%
\begin{pgfscope}%
\pgfpathrectangle{\pgfqpoint{0.519339in}{0.468349in}}{\pgfqpoint{1.278750in}{1.245750in}}%
\pgfusepath{clip}%
\pgfsetroundcap%
\pgfsetroundjoin%
\pgfsetlinewidth{0.501875pt}%
\definecolor{currentstroke}{rgb}{0.800000,0.800000,0.800000}%
\pgfsetstrokecolor{currentstroke}%
\pgfsetdash{}{0pt}%
\pgfpathmoveto{\pgfqpoint{0.519339in}{1.545733in}}%
\pgfpathlineto{\pgfqpoint{1.798089in}{1.545733in}}%
\pgfusepath{stroke}%
\end{pgfscope}%
\begin{pgfscope}%
\definecolor{textcolor}{rgb}{0.150000,0.150000,0.150000}%
\pgfsetstrokecolor{textcolor}%
\pgfsetfillcolor{textcolor}%
\pgftext[x=0.278211in, y=1.507470in, left, base]{\color{textcolor}\rmfamily\fontsize{8.000000}{9.600000}\selectfont \(\displaystyle {0.7}\)}%
\end{pgfscope}%
\begin{pgfscope}%
\definecolor{textcolor}{rgb}{0.150000,0.150000,0.150000}%
\pgfsetstrokecolor{textcolor}%
\pgfsetfillcolor{textcolor}%
\pgftext[x=0.222655in,y=1.091224in,,bottom,rotate=90.000000]{\color{textcolor}\rmfamily\fontsize{10.000000}{12.000000}\selectfont Accuracy}%
\end{pgfscope}%
\begin{pgfscope}%
\pgfpathrectangle{\pgfqpoint{0.519339in}{0.468349in}}{\pgfqpoint{1.278750in}{1.245750in}}%
\pgfusepath{clip}%
\pgfsetbuttcap%
\pgfsetroundjoin%
\pgfsetlinewidth{1.003750pt}%
\definecolor{currentstroke}{rgb}{0.298039,0.447059,0.690196}%
\pgfsetstrokecolor{currentstroke}%
\pgfsetdash{}{0pt}%
\pgfpathmoveto{\pgfqpoint{0.577464in}{0.572305in}}%
\pgfpathlineto{\pgfqpoint{0.577464in}{0.583861in}}%
\pgfusepath{stroke}%
\end{pgfscope}%
\begin{pgfscope}%
\pgfpathrectangle{\pgfqpoint{0.519339in}{0.468349in}}{\pgfqpoint{1.278750in}{1.245750in}}%
\pgfusepath{clip}%
\pgfsetbuttcap%
\pgfsetroundjoin%
\pgfsetlinewidth{1.003750pt}%
\definecolor{currentstroke}{rgb}{0.298039,0.447059,0.690196}%
\pgfsetstrokecolor{currentstroke}%
\pgfsetdash{}{0pt}%
\pgfpathmoveto{\pgfqpoint{0.683476in}{0.572305in}}%
\pgfpathlineto{\pgfqpoint{0.683476in}{0.583861in}}%
\pgfusepath{stroke}%
\end{pgfscope}%
\begin{pgfscope}%
\pgfpathrectangle{\pgfqpoint{0.519339in}{0.468349in}}{\pgfqpoint{1.278750in}{1.245750in}}%
\pgfusepath{clip}%
\pgfsetbuttcap%
\pgfsetroundjoin%
\pgfsetlinewidth{1.003750pt}%
\definecolor{currentstroke}{rgb}{0.298039,0.447059,0.690196}%
\pgfsetstrokecolor{currentstroke}%
\pgfsetdash{}{0pt}%
\pgfpathmoveto{\pgfqpoint{1.035639in}{0.572305in}}%
\pgfpathlineto{\pgfqpoint{1.035639in}{0.583861in}}%
\pgfusepath{stroke}%
\end{pgfscope}%
\begin{pgfscope}%
\pgfpathrectangle{\pgfqpoint{0.519339in}{0.468349in}}{\pgfqpoint{1.278750in}{1.245750in}}%
\pgfusepath{clip}%
\pgfsetbuttcap%
\pgfsetroundjoin%
\pgfsetlinewidth{1.003750pt}%
\definecolor{currentstroke}{rgb}{0.298039,0.447059,0.690196}%
\pgfsetstrokecolor{currentstroke}%
\pgfsetdash{}{0pt}%
\pgfpathmoveto{\pgfqpoint{1.387802in}{0.572305in}}%
\pgfpathlineto{\pgfqpoint{1.387802in}{0.583861in}}%
\pgfusepath{stroke}%
\end{pgfscope}%
\begin{pgfscope}%
\pgfpathrectangle{\pgfqpoint{0.519339in}{0.468349in}}{\pgfqpoint{1.278750in}{1.245750in}}%
\pgfusepath{clip}%
\pgfsetbuttcap%
\pgfsetroundjoin%
\pgfsetlinewidth{1.003750pt}%
\definecolor{currentstroke}{rgb}{0.298039,0.447059,0.690196}%
\pgfsetstrokecolor{currentstroke}%
\pgfsetdash{}{0pt}%
\pgfpathmoveto{\pgfqpoint{1.739964in}{0.572305in}}%
\pgfpathlineto{\pgfqpoint{1.739964in}{0.583861in}}%
\pgfusepath{stroke}%
\end{pgfscope}%
\begin{pgfscope}%
\pgfpathrectangle{\pgfqpoint{0.519339in}{0.468349in}}{\pgfqpoint{1.278750in}{1.245750in}}%
\pgfusepath{clip}%
\pgfsetbuttcap%
\pgfsetroundjoin%
\pgfsetlinewidth{1.003750pt}%
\definecolor{currentstroke}{rgb}{0.866667,0.517647,0.321569}%
\pgfsetstrokecolor{currentstroke}%
\pgfsetdash{}{0pt}%
\pgfpathmoveto{\pgfqpoint{0.577464in}{0.761593in}}%
\pgfpathlineto{\pgfqpoint{0.577464in}{0.773061in}}%
\pgfusepath{stroke}%
\end{pgfscope}%
\begin{pgfscope}%
\pgfpathrectangle{\pgfqpoint{0.519339in}{0.468349in}}{\pgfqpoint{1.278750in}{1.245750in}}%
\pgfusepath{clip}%
\pgfsetbuttcap%
\pgfsetroundjoin%
\pgfsetlinewidth{1.003750pt}%
\definecolor{currentstroke}{rgb}{0.866667,0.517647,0.321569}%
\pgfsetstrokecolor{currentstroke}%
\pgfsetdash{}{0pt}%
\pgfpathmoveto{\pgfqpoint{0.683476in}{0.761593in}}%
\pgfpathlineto{\pgfqpoint{0.683476in}{0.773061in}}%
\pgfusepath{stroke}%
\end{pgfscope}%
\begin{pgfscope}%
\pgfpathrectangle{\pgfqpoint{0.519339in}{0.468349in}}{\pgfqpoint{1.278750in}{1.245750in}}%
\pgfusepath{clip}%
\pgfsetbuttcap%
\pgfsetroundjoin%
\pgfsetlinewidth{1.003750pt}%
\definecolor{currentstroke}{rgb}{0.866667,0.517647,0.321569}%
\pgfsetstrokecolor{currentstroke}%
\pgfsetdash{}{0pt}%
\pgfpathmoveto{\pgfqpoint{1.035639in}{0.761593in}}%
\pgfpathlineto{\pgfqpoint{1.035639in}{0.773061in}}%
\pgfusepath{stroke}%
\end{pgfscope}%
\begin{pgfscope}%
\pgfpathrectangle{\pgfqpoint{0.519339in}{0.468349in}}{\pgfqpoint{1.278750in}{1.245750in}}%
\pgfusepath{clip}%
\pgfsetbuttcap%
\pgfsetroundjoin%
\pgfsetlinewidth{1.003750pt}%
\definecolor{currentstroke}{rgb}{0.866667,0.517647,0.321569}%
\pgfsetstrokecolor{currentstroke}%
\pgfsetdash{}{0pt}%
\pgfpathmoveto{\pgfqpoint{1.387802in}{0.761593in}}%
\pgfpathlineto{\pgfqpoint{1.387802in}{0.773061in}}%
\pgfusepath{stroke}%
\end{pgfscope}%
\begin{pgfscope}%
\pgfpathrectangle{\pgfqpoint{0.519339in}{0.468349in}}{\pgfqpoint{1.278750in}{1.245750in}}%
\pgfusepath{clip}%
\pgfsetbuttcap%
\pgfsetroundjoin%
\pgfsetlinewidth{1.003750pt}%
\definecolor{currentstroke}{rgb}{0.866667,0.517647,0.321569}%
\pgfsetstrokecolor{currentstroke}%
\pgfsetdash{}{0pt}%
\pgfpathmoveto{\pgfqpoint{1.739964in}{0.761593in}}%
\pgfpathlineto{\pgfqpoint{1.739964in}{0.773061in}}%
\pgfusepath{stroke}%
\end{pgfscope}%
\begin{pgfscope}%
\pgfpathrectangle{\pgfqpoint{0.519339in}{0.468349in}}{\pgfqpoint{1.278750in}{1.245750in}}%
\pgfusepath{clip}%
\pgfsetbuttcap%
\pgfsetroundjoin%
\pgfsetlinewidth{1.003750pt}%
\definecolor{currentstroke}{rgb}{0.333333,0.658824,0.407843}%
\pgfsetstrokecolor{currentstroke}%
\pgfsetdash{}{0pt}%
\pgfpathmoveto{\pgfqpoint{0.577464in}{1.617700in}}%
\pgfpathlineto{\pgfqpoint{0.577464in}{1.643031in}}%
\pgfusepath{stroke}%
\end{pgfscope}%
\begin{pgfscope}%
\pgfpathrectangle{\pgfqpoint{0.519339in}{0.468349in}}{\pgfqpoint{1.278750in}{1.245750in}}%
\pgfusepath{clip}%
\pgfsetbuttcap%
\pgfsetroundjoin%
\pgfsetlinewidth{1.003750pt}%
\definecolor{currentstroke}{rgb}{0.333333,0.658824,0.407843}%
\pgfsetstrokecolor{currentstroke}%
\pgfsetdash{}{0pt}%
\pgfpathmoveto{\pgfqpoint{0.683476in}{1.617700in}}%
\pgfpathlineto{\pgfqpoint{0.683476in}{1.643031in}}%
\pgfusepath{stroke}%
\end{pgfscope}%
\begin{pgfscope}%
\pgfpathrectangle{\pgfqpoint{0.519339in}{0.468349in}}{\pgfqpoint{1.278750in}{1.245750in}}%
\pgfusepath{clip}%
\pgfsetbuttcap%
\pgfsetroundjoin%
\pgfsetlinewidth{1.003750pt}%
\definecolor{currentstroke}{rgb}{0.333333,0.658824,0.407843}%
\pgfsetstrokecolor{currentstroke}%
\pgfsetdash{}{0pt}%
\pgfpathmoveto{\pgfqpoint{1.035639in}{1.617700in}}%
\pgfpathlineto{\pgfqpoint{1.035639in}{1.643031in}}%
\pgfusepath{stroke}%
\end{pgfscope}%
\begin{pgfscope}%
\pgfpathrectangle{\pgfqpoint{0.519339in}{0.468349in}}{\pgfqpoint{1.278750in}{1.245750in}}%
\pgfusepath{clip}%
\pgfsetbuttcap%
\pgfsetroundjoin%
\pgfsetlinewidth{1.003750pt}%
\definecolor{currentstroke}{rgb}{0.333333,0.658824,0.407843}%
\pgfsetstrokecolor{currentstroke}%
\pgfsetdash{}{0pt}%
\pgfpathmoveto{\pgfqpoint{1.387802in}{1.617700in}}%
\pgfpathlineto{\pgfqpoint{1.387802in}{1.643031in}}%
\pgfusepath{stroke}%
\end{pgfscope}%
\begin{pgfscope}%
\pgfpathrectangle{\pgfqpoint{0.519339in}{0.468349in}}{\pgfqpoint{1.278750in}{1.245750in}}%
\pgfusepath{clip}%
\pgfsetbuttcap%
\pgfsetroundjoin%
\pgfsetlinewidth{1.003750pt}%
\definecolor{currentstroke}{rgb}{0.333333,0.658824,0.407843}%
\pgfsetstrokecolor{currentstroke}%
\pgfsetdash{}{0pt}%
\pgfpathmoveto{\pgfqpoint{1.739964in}{1.617700in}}%
\pgfpathlineto{\pgfqpoint{1.739964in}{1.643031in}}%
\pgfusepath{stroke}%
\end{pgfscope}%
\begin{pgfscope}%
\pgfpathrectangle{\pgfqpoint{0.519339in}{0.468349in}}{\pgfqpoint{1.278750in}{1.245750in}}%
\pgfusepath{clip}%
\pgfsetbuttcap%
\pgfsetroundjoin%
\pgfsetlinewidth{1.003750pt}%
\definecolor{currentstroke}{rgb}{0.298039,0.447059,0.690196}%
\pgfsetstrokecolor{currentstroke}%
\pgfsetdash{}{0pt}%
\pgfpathmoveto{\pgfqpoint{0.577464in}{0.861605in}}%
\pgfpathlineto{\pgfqpoint{0.577464in}{0.920046in}}%
\pgfusepath{stroke}%
\end{pgfscope}%
\begin{pgfscope}%
\pgfpathrectangle{\pgfqpoint{0.519339in}{0.468349in}}{\pgfqpoint{1.278750in}{1.245750in}}%
\pgfusepath{clip}%
\pgfsetbuttcap%
\pgfsetroundjoin%
\pgfsetlinewidth{1.003750pt}%
\definecolor{currentstroke}{rgb}{0.298039,0.447059,0.690196}%
\pgfsetstrokecolor{currentstroke}%
\pgfsetdash{}{0pt}%
\pgfpathmoveto{\pgfqpoint{0.683476in}{0.524974in}}%
\pgfpathlineto{\pgfqpoint{0.683476in}{0.596322in}}%
\pgfusepath{stroke}%
\end{pgfscope}%
\begin{pgfscope}%
\pgfpathrectangle{\pgfqpoint{0.519339in}{0.468349in}}{\pgfqpoint{1.278750in}{1.245750in}}%
\pgfusepath{clip}%
\pgfsetbuttcap%
\pgfsetroundjoin%
\pgfsetlinewidth{1.003750pt}%
\definecolor{currentstroke}{rgb}{0.298039,0.447059,0.690196}%
\pgfsetstrokecolor{currentstroke}%
\pgfsetdash{}{0pt}%
\pgfpathmoveto{\pgfqpoint{1.035639in}{0.565227in}}%
\pgfpathlineto{\pgfqpoint{1.035639in}{0.632348in}}%
\pgfusepath{stroke}%
\end{pgfscope}%
\begin{pgfscope}%
\pgfpathrectangle{\pgfqpoint{0.519339in}{0.468349in}}{\pgfqpoint{1.278750in}{1.245750in}}%
\pgfusepath{clip}%
\pgfsetbuttcap%
\pgfsetroundjoin%
\pgfsetlinewidth{1.003750pt}%
\definecolor{currentstroke}{rgb}{0.298039,0.447059,0.690196}%
\pgfsetstrokecolor{currentstroke}%
\pgfsetdash{}{0pt}%
\pgfpathmoveto{\pgfqpoint{1.387802in}{0.576213in}}%
\pgfpathlineto{\pgfqpoint{1.387802in}{0.644609in}}%
\pgfusepath{stroke}%
\end{pgfscope}%
\begin{pgfscope}%
\pgfpathrectangle{\pgfqpoint{0.519339in}{0.468349in}}{\pgfqpoint{1.278750in}{1.245750in}}%
\pgfusepath{clip}%
\pgfsetbuttcap%
\pgfsetroundjoin%
\pgfsetlinewidth{1.003750pt}%
\definecolor{currentstroke}{rgb}{0.298039,0.447059,0.690196}%
\pgfsetstrokecolor{currentstroke}%
\pgfsetdash{}{0pt}%
\pgfpathmoveto{\pgfqpoint{1.739964in}{0.595263in}}%
\pgfpathlineto{\pgfqpoint{1.739964in}{0.674232in}}%
\pgfusepath{stroke}%
\end{pgfscope}%
\begin{pgfscope}%
\pgfpathrectangle{\pgfqpoint{0.519339in}{0.468349in}}{\pgfqpoint{1.278750in}{1.245750in}}%
\pgfusepath{clip}%
\pgfsetbuttcap%
\pgfsetroundjoin%
\pgfsetlinewidth{1.003750pt}%
\definecolor{currentstroke}{rgb}{0.866667,0.517647,0.321569}%
\pgfsetstrokecolor{currentstroke}%
\pgfsetdash{}{0pt}%
\pgfpathmoveto{\pgfqpoint{0.577464in}{0.952718in}}%
\pgfpathlineto{\pgfqpoint{0.577464in}{0.980764in}}%
\pgfusepath{stroke}%
\end{pgfscope}%
\begin{pgfscope}%
\pgfpathrectangle{\pgfqpoint{0.519339in}{0.468349in}}{\pgfqpoint{1.278750in}{1.245750in}}%
\pgfusepath{clip}%
\pgfsetbuttcap%
\pgfsetroundjoin%
\pgfsetlinewidth{1.003750pt}%
\definecolor{currentstroke}{rgb}{0.866667,0.517647,0.321569}%
\pgfsetstrokecolor{currentstroke}%
\pgfsetdash{}{0pt}%
\pgfpathmoveto{\pgfqpoint{0.683476in}{0.675369in}}%
\pgfpathlineto{\pgfqpoint{0.683476in}{0.702369in}}%
\pgfusepath{stroke}%
\end{pgfscope}%
\begin{pgfscope}%
\pgfpathrectangle{\pgfqpoint{0.519339in}{0.468349in}}{\pgfqpoint{1.278750in}{1.245750in}}%
\pgfusepath{clip}%
\pgfsetbuttcap%
\pgfsetroundjoin%
\pgfsetlinewidth{1.003750pt}%
\definecolor{currentstroke}{rgb}{0.866667,0.517647,0.321569}%
\pgfsetstrokecolor{currentstroke}%
\pgfsetdash{}{0pt}%
\pgfpathmoveto{\pgfqpoint{1.035639in}{0.765863in}}%
\pgfpathlineto{\pgfqpoint{1.035639in}{0.806567in}}%
\pgfusepath{stroke}%
\end{pgfscope}%
\begin{pgfscope}%
\pgfpathrectangle{\pgfqpoint{0.519339in}{0.468349in}}{\pgfqpoint{1.278750in}{1.245750in}}%
\pgfusepath{clip}%
\pgfsetbuttcap%
\pgfsetroundjoin%
\pgfsetlinewidth{1.003750pt}%
\definecolor{currentstroke}{rgb}{0.866667,0.517647,0.321569}%
\pgfsetstrokecolor{currentstroke}%
\pgfsetdash{}{0pt}%
\pgfpathmoveto{\pgfqpoint{1.387802in}{0.784254in}}%
\pgfpathlineto{\pgfqpoint{1.387802in}{0.832491in}}%
\pgfusepath{stroke}%
\end{pgfscope}%
\begin{pgfscope}%
\pgfpathrectangle{\pgfqpoint{0.519339in}{0.468349in}}{\pgfqpoint{1.278750in}{1.245750in}}%
\pgfusepath{clip}%
\pgfsetbuttcap%
\pgfsetroundjoin%
\pgfsetlinewidth{1.003750pt}%
\definecolor{currentstroke}{rgb}{0.866667,0.517647,0.321569}%
\pgfsetstrokecolor{currentstroke}%
\pgfsetdash{}{0pt}%
\pgfpathmoveto{\pgfqpoint{1.739964in}{0.818506in}}%
\pgfpathlineto{\pgfqpoint{1.739964in}{0.862894in}}%
\pgfusepath{stroke}%
\end{pgfscope}%
\begin{pgfscope}%
\pgfpathrectangle{\pgfqpoint{0.519339in}{0.468349in}}{\pgfqpoint{1.278750in}{1.245750in}}%
\pgfusepath{clip}%
\pgfsetbuttcap%
\pgfsetroundjoin%
\pgfsetlinewidth{1.003750pt}%
\definecolor{currentstroke}{rgb}{0.333333,0.658824,0.407843}%
\pgfsetstrokecolor{currentstroke}%
\pgfsetdash{}{0pt}%
\pgfpathmoveto{\pgfqpoint{0.577464in}{1.577349in}}%
\pgfpathlineto{\pgfqpoint{0.577464in}{1.613641in}}%
\pgfusepath{stroke}%
\end{pgfscope}%
\begin{pgfscope}%
\pgfpathrectangle{\pgfqpoint{0.519339in}{0.468349in}}{\pgfqpoint{1.278750in}{1.245750in}}%
\pgfusepath{clip}%
\pgfsetbuttcap%
\pgfsetroundjoin%
\pgfsetlinewidth{1.003750pt}%
\definecolor{currentstroke}{rgb}{0.333333,0.658824,0.407843}%
\pgfsetstrokecolor{currentstroke}%
\pgfsetdash{}{0pt}%
\pgfpathmoveto{\pgfqpoint{0.683476in}{1.521733in}}%
\pgfpathlineto{\pgfqpoint{0.683476in}{1.574818in}}%
\pgfusepath{stroke}%
\end{pgfscope}%
\begin{pgfscope}%
\pgfpathrectangle{\pgfqpoint{0.519339in}{0.468349in}}{\pgfqpoint{1.278750in}{1.245750in}}%
\pgfusepath{clip}%
\pgfsetbuttcap%
\pgfsetroundjoin%
\pgfsetlinewidth{1.003750pt}%
\definecolor{currentstroke}{rgb}{0.333333,0.658824,0.407843}%
\pgfsetstrokecolor{currentstroke}%
\pgfsetdash{}{0pt}%
\pgfpathmoveto{\pgfqpoint{1.035639in}{1.565126in}}%
\pgfpathlineto{\pgfqpoint{1.035639in}{1.624412in}}%
\pgfusepath{stroke}%
\end{pgfscope}%
\begin{pgfscope}%
\pgfpathrectangle{\pgfqpoint{0.519339in}{0.468349in}}{\pgfqpoint{1.278750in}{1.245750in}}%
\pgfusepath{clip}%
\pgfsetbuttcap%
\pgfsetroundjoin%
\pgfsetlinewidth{1.003750pt}%
\definecolor{currentstroke}{rgb}{0.333333,0.658824,0.407843}%
\pgfsetstrokecolor{currentstroke}%
\pgfsetdash{}{0pt}%
\pgfpathmoveto{\pgfqpoint{1.387802in}{1.580736in}}%
\pgfpathlineto{\pgfqpoint{1.387802in}{1.657474in}}%
\pgfusepath{stroke}%
\end{pgfscope}%
\begin{pgfscope}%
\pgfpathrectangle{\pgfqpoint{0.519339in}{0.468349in}}{\pgfqpoint{1.278750in}{1.245750in}}%
\pgfusepath{clip}%
\pgfsetbuttcap%
\pgfsetroundjoin%
\pgfsetlinewidth{1.003750pt}%
\definecolor{currentstroke}{rgb}{0.333333,0.658824,0.407843}%
\pgfsetstrokecolor{currentstroke}%
\pgfsetdash{}{0pt}%
\pgfpathmoveto{\pgfqpoint{1.739964in}{1.584013in}}%
\pgfpathlineto{\pgfqpoint{1.739964in}{1.652019in}}%
\pgfusepath{stroke}%
\end{pgfscope}%
\begin{pgfscope}%
\pgfpathrectangle{\pgfqpoint{0.519339in}{0.468349in}}{\pgfqpoint{1.278750in}{1.245750in}}%
\pgfusepath{clip}%
\pgfsetbuttcap%
\pgfsetroundjoin%
\pgfsetlinewidth{1.003750pt}%
\definecolor{currentstroke}{rgb}{0.298039,0.447059,0.690196}%
\pgfsetstrokecolor{currentstroke}%
\pgfsetdash{{6.400000pt}{1.600000pt}{1.000000pt}{1.600000pt}}{0.000000pt}%
\pgfpathmoveto{\pgfqpoint{0.577464in}{0.578083in}}%
\pgfpathlineto{\pgfqpoint{0.683476in}{0.578083in}}%
\pgfpathlineto{\pgfqpoint{1.035639in}{0.578083in}}%
\pgfpathlineto{\pgfqpoint{1.387802in}{0.578083in}}%
\pgfpathlineto{\pgfqpoint{1.739964in}{0.578083in}}%
\pgfusepath{stroke}%
\end{pgfscope}%
\begin{pgfscope}%
\pgfpathrectangle{\pgfqpoint{0.519339in}{0.468349in}}{\pgfqpoint{1.278750in}{1.245750in}}%
\pgfusepath{clip}%
\pgfsetbuttcap%
\pgfsetroundjoin%
\pgfsetlinewidth{1.003750pt}%
\definecolor{currentstroke}{rgb}{0.866667,0.517647,0.321569}%
\pgfsetstrokecolor{currentstroke}%
\pgfsetdash{{6.400000pt}{1.600000pt}{1.000000pt}{1.600000pt}}{0.000000pt}%
\pgfpathmoveto{\pgfqpoint{0.577464in}{0.767327in}}%
\pgfpathlineto{\pgfqpoint{0.683476in}{0.767327in}}%
\pgfpathlineto{\pgfqpoint{1.035639in}{0.767327in}}%
\pgfpathlineto{\pgfqpoint{1.387802in}{0.767327in}}%
\pgfpathlineto{\pgfqpoint{1.739964in}{0.767327in}}%
\pgfusepath{stroke}%
\end{pgfscope}%
\begin{pgfscope}%
\pgfpathrectangle{\pgfqpoint{0.519339in}{0.468349in}}{\pgfqpoint{1.278750in}{1.245750in}}%
\pgfusepath{clip}%
\pgfsetbuttcap%
\pgfsetroundjoin%
\pgfsetlinewidth{1.003750pt}%
\definecolor{currentstroke}{rgb}{0.333333,0.658824,0.407843}%
\pgfsetstrokecolor{currentstroke}%
\pgfsetdash{{6.400000pt}{1.600000pt}{1.000000pt}{1.600000pt}}{0.000000pt}%
\pgfpathmoveto{\pgfqpoint{0.577464in}{1.630366in}}%
\pgfpathlineto{\pgfqpoint{0.683476in}{1.630366in}}%
\pgfpathlineto{\pgfqpoint{1.035639in}{1.630366in}}%
\pgfpathlineto{\pgfqpoint{1.387802in}{1.630366in}}%
\pgfpathlineto{\pgfqpoint{1.739964in}{1.630366in}}%
\pgfusepath{stroke}%
\end{pgfscope}%
\begin{pgfscope}%
\pgfpathrectangle{\pgfqpoint{0.519339in}{0.468349in}}{\pgfqpoint{1.278750in}{1.245750in}}%
\pgfusepath{clip}%
\pgfsetroundcap%
\pgfsetroundjoin%
\pgfsetlinewidth{1.003750pt}%
\definecolor{currentstroke}{rgb}{0.298039,0.447059,0.690196}%
\pgfsetstrokecolor{currentstroke}%
\pgfsetdash{}{0pt}%
\pgfpathmoveto{\pgfqpoint{0.577464in}{0.890826in}}%
\pgfpathlineto{\pgfqpoint{0.683476in}{0.560648in}}%
\pgfpathlineto{\pgfqpoint{1.035639in}{0.598788in}}%
\pgfpathlineto{\pgfqpoint{1.387802in}{0.610411in}}%
\pgfpathlineto{\pgfqpoint{1.739964in}{0.634747in}}%
\pgfusepath{stroke}%
\end{pgfscope}%
\begin{pgfscope}%
\pgfpathrectangle{\pgfqpoint{0.519339in}{0.468349in}}{\pgfqpoint{1.278750in}{1.245750in}}%
\pgfusepath{clip}%
\pgfsetroundcap%
\pgfsetroundjoin%
\pgfsetlinewidth{1.003750pt}%
\definecolor{currentstroke}{rgb}{0.866667,0.517647,0.321569}%
\pgfsetstrokecolor{currentstroke}%
\pgfsetdash{}{0pt}%
\pgfpathmoveto{\pgfqpoint{0.577464in}{0.966741in}}%
\pgfpathlineto{\pgfqpoint{0.683476in}{0.688869in}}%
\pgfpathlineto{\pgfqpoint{1.035639in}{0.786215in}}%
\pgfpathlineto{\pgfqpoint{1.387802in}{0.808372in}}%
\pgfpathlineto{\pgfqpoint{1.739964in}{0.840700in}}%
\pgfusepath{stroke}%
\end{pgfscope}%
\begin{pgfscope}%
\pgfpathrectangle{\pgfqpoint{0.519339in}{0.468349in}}{\pgfqpoint{1.278750in}{1.245750in}}%
\pgfusepath{clip}%
\pgfsetroundcap%
\pgfsetroundjoin%
\pgfsetlinewidth{1.003750pt}%
\definecolor{currentstroke}{rgb}{0.333333,0.658824,0.407843}%
\pgfsetstrokecolor{currentstroke}%
\pgfsetdash{}{0pt}%
\pgfpathmoveto{\pgfqpoint{0.577464in}{1.595495in}}%
\pgfpathlineto{\pgfqpoint{0.683476in}{1.548275in}}%
\pgfpathlineto{\pgfqpoint{1.035639in}{1.594769in}}%
\pgfpathlineto{\pgfqpoint{1.387802in}{1.619105in}}%
\pgfpathlineto{\pgfqpoint{1.739964in}{1.618016in}}%
\pgfusepath{stroke}%
\end{pgfscope}%
\begin{pgfscope}%
\pgfsetrectcap%
\pgfsetmiterjoin%
\pgfsetlinewidth{0.752812pt}%
\definecolor{currentstroke}{rgb}{0.700000,0.700000,0.700000}%
\pgfsetstrokecolor{currentstroke}%
\pgfsetdash{}{0pt}%
\pgfpathmoveto{\pgfqpoint{0.519339in}{0.468349in}}%
\pgfpathlineto{\pgfqpoint{0.519339in}{1.714099in}}%
\pgfusepath{stroke}%
\end{pgfscope}%
\begin{pgfscope}%
\pgfsetrectcap%
\pgfsetmiterjoin%
\pgfsetlinewidth{0.752812pt}%
\definecolor{currentstroke}{rgb}{0.700000,0.700000,0.700000}%
\pgfsetstrokecolor{currentstroke}%
\pgfsetdash{}{0pt}%
\pgfpathmoveto{\pgfqpoint{1.798089in}{0.468349in}}%
\pgfpathlineto{\pgfqpoint{1.798089in}{1.714099in}}%
\pgfusepath{stroke}%
\end{pgfscope}%
\begin{pgfscope}%
\pgfsetrectcap%
\pgfsetmiterjoin%
\pgfsetlinewidth{0.752812pt}%
\definecolor{currentstroke}{rgb}{0.700000,0.700000,0.700000}%
\pgfsetstrokecolor{currentstroke}%
\pgfsetdash{}{0pt}%
\pgfpathmoveto{\pgfqpoint{0.519339in}{0.468349in}}%
\pgfpathlineto{\pgfqpoint{1.798089in}{0.468349in}}%
\pgfusepath{stroke}%
\end{pgfscope}%
\begin{pgfscope}%
\pgfsetrectcap%
\pgfsetmiterjoin%
\pgfsetlinewidth{0.752812pt}%
\definecolor{currentstroke}{rgb}{0.700000,0.700000,0.700000}%
\pgfsetstrokecolor{currentstroke}%
\pgfsetdash{}{0pt}%
\pgfpathmoveto{\pgfqpoint{0.519339in}{1.714099in}}%
\pgfpathlineto{\pgfqpoint{1.798089in}{1.714099in}}%
\pgfusepath{stroke}%
\end{pgfscope}%
\end{pgfpicture}%
\makeatother%
\endgroup%
}}  \\
    \end{array}\)
  }
  \caption{We perform the proposed PR-BCD (solid lines) to obtain a perturbed adjacency matrix with the fraction of perturbed edges \(\epsilon=0.1\) and \(\epsilon=0.25\) with loss (5) of Sec.~\ref{sec:ceisbad}. We run 50 epochs where we resample the search space and subsequently fine-tune for 250 epochs. The dashed lines show the performance of vanilla PGD~\citep{Xu2019a}. We report the mean and its three-sigma error over five random seeds. \label{fig:randomblocksizeinfluence}}
\end{figure}

\begin{figure}[t]
  \centering
  \resizebox{\linewidth}{!}{\input{assets/global_prbcd_arxiv_0_1_block_size_cmp_epochs_loss_legend.pgf}}
  \makebox[\linewidth][c]{
    \(\begin{array}{cc}
      \subfloat[]{\resizebox{0.5\linewidth}{!}{%% Creator: Matplotlib, PGF backend
%%
%% To include the figure in your LaTeX document, write
%%   \input{<filename>.pgf}
%%
%% Make sure the required packages are loaded in your preamble
%%   \usepackage{pgf}
%%
%% and, on pdftex
%%   \usepackage[utf8]{inputenc}\DeclareUnicodeCharacter{2212}{-}
%%
%% or, on luatex and xetex
%%   \usepackage{unicode-math}
%%
%% Figures using additional raster images can only be included by \input if
%% they are in the same directory as the main LaTeX file. For loading figures
%% from other directories you can use the `import` package
%%   \usepackage{import}
%%
%% and then include the figures with
%%   \import{<path to file>}{<filename>.pgf}
%%
%% Matplotlib used the following preamble
%%   \usepackage[utf8]{inputenc}
%%   \usepackage[T1]{fontenc}
%%   \usepackage{amsmath}
%%   \newcommand*{\mat}[1]{\boldsymbol{#1}}
%%
\begingroup%
\makeatletter%
\begin{pgfpicture}%
\pgfpathrectangle{\pgfpointorigin}{\pgfqpoint{2.051008in}{1.813158in}}%
\pgfusepath{use as bounding box, clip}%
\begin{pgfscope}%
\pgfsetbuttcap%
\pgfsetmiterjoin%
\definecolor{currentfill}{rgb}{1.000000,1.000000,1.000000}%
\pgfsetfillcolor{currentfill}%
\pgfsetlinewidth{0.000000pt}%
\definecolor{currentstroke}{rgb}{1.000000,1.000000,1.000000}%
\pgfsetstrokecolor{currentstroke}%
\pgfsetstrokeopacity{0.000000}%
\pgfsetdash{}{0pt}%
\pgfpathmoveto{\pgfqpoint{0.000000in}{0.000000in}}%
\pgfpathlineto{\pgfqpoint{2.051008in}{0.000000in}}%
\pgfpathlineto{\pgfqpoint{2.051008in}{1.813158in}}%
\pgfpathlineto{\pgfqpoint{0.000000in}{1.813158in}}%
\pgfpathclose%
\pgfusepath{fill}%
\end{pgfscope}%
\begin{pgfscope}%
\pgfsetbuttcap%
\pgfsetmiterjoin%
\definecolor{currentfill}{rgb}{1.000000,1.000000,1.000000}%
\pgfsetfillcolor{currentfill}%
\pgfsetlinewidth{0.000000pt}%
\definecolor{currentstroke}{rgb}{0.000000,0.000000,0.000000}%
\pgfsetstrokecolor{currentstroke}%
\pgfsetstrokeopacity{0.000000}%
\pgfsetdash{}{0pt}%
\pgfpathmoveto{\pgfqpoint{0.611162in}{0.467408in}}%
\pgfpathlineto{\pgfqpoint{1.889912in}{0.467408in}}%
\pgfpathlineto{\pgfqpoint{1.889912in}{1.713158in}}%
\pgfpathlineto{\pgfqpoint{0.611162in}{1.713158in}}%
\pgfpathclose%
\pgfusepath{fill}%
\end{pgfscope}%
\begin{pgfscope}%
\pgfpathrectangle{\pgfqpoint{0.611162in}{0.467408in}}{\pgfqpoint{1.278750in}{1.245750in}}%
\pgfusepath{clip}%
\pgfsetroundcap%
\pgfsetroundjoin%
\pgfsetlinewidth{0.501875pt}%
\definecolor{currentstroke}{rgb}{0.800000,0.800000,0.800000}%
\pgfsetstrokecolor{currentstroke}%
\pgfsetdash{}{0pt}%
\pgfpathmoveto{\pgfqpoint{0.669287in}{0.467408in}}%
\pgfpathlineto{\pgfqpoint{0.669287in}{1.713158in}}%
\pgfusepath{stroke}%
\end{pgfscope}%
\begin{pgfscope}%
\definecolor{textcolor}{rgb}{0.150000,0.150000,0.150000}%
\pgfsetstrokecolor{textcolor}%
\pgfsetfillcolor{textcolor}%
\pgftext[x=0.669287in,y=0.377130in,,top]{\color{textcolor}\rmfamily\fontsize{8.000000}{9.600000}\selectfont \(\displaystyle {0}\)}%
\end{pgfscope}%
\begin{pgfscope}%
\pgfpathrectangle{\pgfqpoint{0.611162in}{0.467408in}}{\pgfqpoint{1.278750in}{1.245750in}}%
\pgfusepath{clip}%
\pgfsetroundcap%
\pgfsetroundjoin%
\pgfsetlinewidth{0.501875pt}%
\definecolor{currentstroke}{rgb}{0.800000,0.800000,0.800000}%
\pgfsetstrokecolor{currentstroke}%
\pgfsetdash{}{0pt}%
\pgfpathmoveto{\pgfqpoint{1.251118in}{0.467408in}}%
\pgfpathlineto{\pgfqpoint{1.251118in}{1.713158in}}%
\pgfusepath{stroke}%
\end{pgfscope}%
\begin{pgfscope}%
\definecolor{textcolor}{rgb}{0.150000,0.150000,0.150000}%
\pgfsetstrokecolor{textcolor}%
\pgfsetfillcolor{textcolor}%
\pgftext[x=1.251118in,y=0.377130in,,top]{\color{textcolor}\rmfamily\fontsize{8.000000}{9.600000}\selectfont \(\displaystyle {500}\)}%
\end{pgfscope}%
\begin{pgfscope}%
\pgfpathrectangle{\pgfqpoint{0.611162in}{0.467408in}}{\pgfqpoint{1.278750in}{1.245750in}}%
\pgfusepath{clip}%
\pgfsetroundcap%
\pgfsetroundjoin%
\pgfsetlinewidth{0.501875pt}%
\definecolor{currentstroke}{rgb}{0.800000,0.800000,0.800000}%
\pgfsetstrokecolor{currentstroke}%
\pgfsetdash{}{0pt}%
\pgfpathmoveto{\pgfqpoint{1.832950in}{0.467408in}}%
\pgfpathlineto{\pgfqpoint{1.832950in}{1.713158in}}%
\pgfusepath{stroke}%
\end{pgfscope}%
\begin{pgfscope}%
\definecolor{textcolor}{rgb}{0.150000,0.150000,0.150000}%
\pgfsetstrokecolor{textcolor}%
\pgfsetfillcolor{textcolor}%
\pgftext[x=1.832950in,y=0.377130in,,top]{\color{textcolor}\rmfamily\fontsize{8.000000}{9.600000}\selectfont \(\displaystyle {1000}\)}%
\end{pgfscope}%
\begin{pgfscope}%
\definecolor{textcolor}{rgb}{0.150000,0.150000,0.150000}%
\pgfsetstrokecolor{textcolor}%
\pgfsetfillcolor{textcolor}%
\pgftext[x=1.250537in,y=0.223450in,,top]{\color{textcolor}\rmfamily\fontsize{10.000000}{12.000000}\selectfont Epochs \(\displaystyle k\)}%
\end{pgfscope}%
\begin{pgfscope}%
\pgfpathrectangle{\pgfqpoint{0.611162in}{0.467408in}}{\pgfqpoint{1.278750in}{1.245750in}}%
\pgfusepath{clip}%
\pgfsetroundcap%
\pgfsetroundjoin%
\pgfsetlinewidth{0.501875pt}%
\definecolor{currentstroke}{rgb}{0.800000,0.800000,0.800000}%
\pgfsetstrokecolor{currentstroke}%
\pgfsetdash{}{0pt}%
\pgfpathmoveto{\pgfqpoint{0.611162in}{0.493043in}}%
\pgfpathlineto{\pgfqpoint{1.889912in}{0.493043in}}%
\pgfusepath{stroke}%
\end{pgfscope}%
\begin{pgfscope}%
\definecolor{textcolor}{rgb}{0.150000,0.150000,0.150000}%
\pgfsetstrokecolor{textcolor}%
\pgfsetfillcolor{textcolor}%
\pgftext[x=0.278211in, y=0.454780in, left, base]{\color{textcolor}\rmfamily\fontsize{8.000000}{9.600000}\selectfont \(\displaystyle {-0.4}\)}%
\end{pgfscope}%
\begin{pgfscope}%
\pgfpathrectangle{\pgfqpoint{0.611162in}{0.467408in}}{\pgfqpoint{1.278750in}{1.245750in}}%
\pgfusepath{clip}%
\pgfsetroundcap%
\pgfsetroundjoin%
\pgfsetlinewidth{0.501875pt}%
\definecolor{currentstroke}{rgb}{0.800000,0.800000,0.800000}%
\pgfsetstrokecolor{currentstroke}%
\pgfsetdash{}{0pt}%
\pgfpathmoveto{\pgfqpoint{0.611162in}{0.956343in}}%
\pgfpathlineto{\pgfqpoint{1.889912in}{0.956343in}}%
\pgfusepath{stroke}%
\end{pgfscope}%
\begin{pgfscope}%
\definecolor{textcolor}{rgb}{0.150000,0.150000,0.150000}%
\pgfsetstrokecolor{textcolor}%
\pgfsetfillcolor{textcolor}%
\pgftext[x=0.278211in, y=0.918081in, left, base]{\color{textcolor}\rmfamily\fontsize{8.000000}{9.600000}\selectfont \(\displaystyle {-0.2}\)}%
\end{pgfscope}%
\begin{pgfscope}%
\pgfpathrectangle{\pgfqpoint{0.611162in}{0.467408in}}{\pgfqpoint{1.278750in}{1.245750in}}%
\pgfusepath{clip}%
\pgfsetroundcap%
\pgfsetroundjoin%
\pgfsetlinewidth{0.501875pt}%
\definecolor{currentstroke}{rgb}{0.800000,0.800000,0.800000}%
\pgfsetstrokecolor{currentstroke}%
\pgfsetdash{}{0pt}%
\pgfpathmoveto{\pgfqpoint{0.611162in}{1.419644in}}%
\pgfpathlineto{\pgfqpoint{1.889912in}{1.419644in}}%
\pgfusepath{stroke}%
\end{pgfscope}%
\begin{pgfscope}%
\definecolor{textcolor}{rgb}{0.150000,0.150000,0.150000}%
\pgfsetstrokecolor{textcolor}%
\pgfsetfillcolor{textcolor}%
\pgftext[x=0.370033in, y=1.381382in, left, base]{\color{textcolor}\rmfamily\fontsize{8.000000}{9.600000}\selectfont \(\displaystyle {0.0}\)}%
\end{pgfscope}%
\begin{pgfscope}%
\definecolor{textcolor}{rgb}{0.150000,0.150000,0.150000}%
\pgfsetstrokecolor{textcolor}%
\pgfsetfillcolor{textcolor}%
\pgftext[x=0.222655in,y=1.090283in,,bottom,rotate=90.000000]{\color{textcolor}\rmfamily\fontsize{10.000000}{12.000000}\selectfont Loss}%
\end{pgfscope}%
\begin{pgfscope}%
\pgfpathrectangle{\pgfqpoint{0.611162in}{0.467408in}}{\pgfqpoint{1.278750in}{1.245750in}}%
\pgfusepath{clip}%
\pgfsetroundcap%
\pgfsetroundjoin%
\pgfsetlinewidth{1.003750pt}%
\definecolor{currentstroke}{rgb}{0.298039,0.447059,0.690196}%
\pgfsetstrokecolor{currentstroke}%
\pgfsetdash{}{0pt}%
\pgfpathmoveto{\pgfqpoint{0.669287in}{0.524033in}}%
\pgfpathlineto{\pgfqpoint{0.671614in}{0.557289in}}%
\pgfpathlineto{\pgfqpoint{0.673941in}{0.601695in}}%
\pgfpathlineto{\pgfqpoint{0.677432in}{0.680188in}}%
\pgfpathlineto{\pgfqpoint{0.686742in}{0.896093in}}%
\pgfpathlineto{\pgfqpoint{0.690233in}{0.963695in}}%
\pgfpathlineto{\pgfqpoint{0.692560in}{1.002701in}}%
\pgfpathlineto{\pgfqpoint{0.696051in}{1.053866in}}%
\pgfpathlineto{\pgfqpoint{0.698378in}{1.084092in}}%
\pgfpathlineto{\pgfqpoint{0.699542in}{1.096441in}}%
\pgfpathlineto{\pgfqpoint{0.704197in}{1.130889in}}%
\pgfpathlineto{\pgfqpoint{0.708851in}{1.162301in}}%
\pgfpathlineto{\pgfqpoint{0.713506in}{1.190768in}}%
\pgfpathlineto{\pgfqpoint{0.716997in}{1.209607in}}%
\pgfpathlineto{\pgfqpoint{0.721652in}{1.232765in}}%
\pgfpathlineto{\pgfqpoint{0.725143in}{1.248461in}}%
\pgfpathlineto{\pgfqpoint{0.729797in}{1.267587in}}%
\pgfpathlineto{\pgfqpoint{0.735615in}{1.288967in}}%
\pgfpathlineto{\pgfqpoint{0.740270in}{1.304689in}}%
\pgfpathlineto{\pgfqpoint{0.746088in}{1.322769in}}%
\pgfpathlineto{\pgfqpoint{0.750743in}{1.335727in}}%
\pgfpathlineto{\pgfqpoint{0.758889in}{1.356261in}}%
\pgfpathlineto{\pgfqpoint{0.763543in}{1.366750in}}%
\pgfpathlineto{\pgfqpoint{0.767034in}{1.374074in}}%
\pgfpathlineto{\pgfqpoint{0.770525in}{1.380934in}}%
\pgfpathlineto{\pgfqpoint{0.777507in}{1.393763in}}%
\pgfpathlineto{\pgfqpoint{0.787980in}{1.410586in}}%
\pgfpathlineto{\pgfqpoint{0.791471in}{1.415688in}}%
\pgfpathlineto{\pgfqpoint{0.798453in}{1.425161in}}%
\pgfpathlineto{\pgfqpoint{0.805435in}{1.433792in}}%
\pgfpathlineto{\pgfqpoint{0.811254in}{1.440531in}}%
\pgfpathlineto{\pgfqpoint{0.818236in}{1.447829in}}%
\pgfpathlineto{\pgfqpoint{0.825218in}{1.454795in}}%
\pgfpathlineto{\pgfqpoint{0.835691in}{1.464434in}}%
\pgfpathlineto{\pgfqpoint{0.849655in}{1.475950in}}%
\pgfpathlineto{\pgfqpoint{0.857800in}{1.482128in}}%
\pgfpathlineto{\pgfqpoint{0.869437in}{1.490139in}}%
\pgfpathlineto{\pgfqpoint{0.876419in}{1.494784in}}%
\pgfpathlineto{\pgfqpoint{0.878746in}{1.496263in}}%
\pgfpathlineto{\pgfqpoint{0.882237in}{1.498496in}}%
\pgfpathlineto{\pgfqpoint{0.885728in}{1.500664in}}%
\pgfpathlineto{\pgfqpoint{0.892710in}{1.504829in}}%
\pgfpathlineto{\pgfqpoint{0.913656in}{1.515807in}}%
\pgfpathlineto{\pgfqpoint{0.927620in}{1.522354in}}%
\pgfpathlineto{\pgfqpoint{0.947402in}{1.530677in}}%
\pgfpathlineto{\pgfqpoint{0.949730in}{1.531609in}}%
\pgfpathlineto{\pgfqpoint{0.970676in}{1.539516in}}%
\pgfpathlineto{\pgfqpoint{0.973003in}{1.540316in}}%
\pgfpathlineto{\pgfqpoint{0.976494in}{1.541536in}}%
\pgfpathlineto{\pgfqpoint{0.982312in}{1.543600in}}%
\pgfpathlineto{\pgfqpoint{1.003258in}{1.550865in}}%
\pgfpathlineto{\pgfqpoint{1.043986in}{1.564158in}}%
\pgfpathlineto{\pgfqpoint{1.052132in}{1.566523in}}%
\pgfpathlineto{\pgfqpoint{1.056787in}{1.567866in}}%
\pgfpathlineto{\pgfqpoint{1.075405in}{1.572937in}}%
\pgfpathlineto{\pgfqpoint{1.078896in}{1.573791in}}%
\pgfpathlineto{\pgfqpoint{1.082387in}{1.574642in}}%
\pgfpathlineto{\pgfqpoint{1.088206in}{1.576002in}}%
\pgfpathlineto{\pgfqpoint{1.091697in}{1.576873in}}%
\pgfpathlineto{\pgfqpoint{1.098679in}{1.578426in}}%
\pgfpathlineto{\pgfqpoint{1.106824in}{1.580353in}}%
\pgfpathlineto{\pgfqpoint{1.138243in}{1.586572in}}%
\pgfpathlineto{\pgfqpoint{1.142898in}{1.587426in}}%
\pgfpathlineto{\pgfqpoint{1.145225in}{1.587934in}}%
\pgfpathlineto{\pgfqpoint{1.147552in}{1.588404in}}%
\pgfpathlineto{\pgfqpoint{1.155698in}{1.589915in}}%
\pgfpathlineto{\pgfqpoint{1.160353in}{1.590727in}}%
\pgfpathlineto{\pgfqpoint{1.167335in}{1.592019in}}%
\pgfpathlineto{\pgfqpoint{1.171989in}{1.592877in}}%
\pgfpathlineto{\pgfqpoint{1.189444in}{1.596055in}}%
\pgfpathlineto{\pgfqpoint{1.194099in}{1.596769in}}%
\pgfpathlineto{\pgfqpoint{1.199917in}{1.597677in}}%
\pgfpathlineto{\pgfqpoint{1.213881in}{1.600100in}}%
\pgfpathlineto{\pgfqpoint{1.220863in}{1.601305in}}%
\pgfpathlineto{\pgfqpoint{1.234827in}{1.603901in}}%
\pgfpathlineto{\pgfqpoint{1.238318in}{1.604387in}}%
\pgfpathlineto{\pgfqpoint{1.242973in}{1.605176in}}%
\pgfpathlineto{\pgfqpoint{1.266246in}{1.608515in}}%
\pgfpathlineto{\pgfqpoint{1.269737in}{1.609084in}}%
\pgfpathlineto{\pgfqpoint{1.276719in}{1.610097in}}%
\pgfpathlineto{\pgfqpoint{1.322102in}{1.616031in}}%
\pgfpathlineto{\pgfqpoint{1.331411in}{1.617125in}}%
\pgfpathlineto{\pgfqpoint{1.334902in}{1.617635in}}%
\pgfpathlineto{\pgfqpoint{1.340721in}{1.618318in}}%
\pgfpathlineto{\pgfqpoint{1.352357in}{1.619763in}}%
\pgfpathlineto{\pgfqpoint{1.359339in}{1.620720in}}%
\pgfpathlineto{\pgfqpoint{1.402395in}{1.625481in}}%
\pgfpathlineto{\pgfqpoint{1.405886in}{1.625779in}}%
\pgfpathlineto{\pgfqpoint{1.509452in}{1.634917in}}%
\pgfpathlineto{\pgfqpoint{1.521088in}{1.635919in}}%
\pgfpathlineto{\pgfqpoint{1.526907in}{1.636476in}}%
\pgfpathlineto{\pgfqpoint{1.530398in}{1.636708in}}%
\pgfpathlineto{\pgfqpoint{1.630473in}{1.644019in}}%
\pgfpathlineto{\pgfqpoint{1.633964in}{1.644184in}}%
\pgfpathlineto{\pgfqpoint{1.663055in}{1.646171in}}%
\pgfpathlineto{\pgfqpoint{1.665383in}{1.646314in}}%
\pgfpathlineto{\pgfqpoint{1.672365in}{1.646838in}}%
\pgfpathlineto{\pgfqpoint{1.677019in}{1.647211in}}%
\pgfpathlineto{\pgfqpoint{1.681674in}{1.647505in}}%
\pgfpathlineto{\pgfqpoint{1.716584in}{1.649655in}}%
\pgfpathlineto{\pgfqpoint{1.723566in}{1.650155in}}%
\pgfpathlineto{\pgfqpoint{1.754985in}{1.652257in}}%
\pgfpathlineto{\pgfqpoint{1.760803in}{1.652611in}}%
\pgfpathlineto{\pgfqpoint{1.782913in}{1.653905in}}%
\pgfpathlineto{\pgfqpoint{1.788731in}{1.654272in}}%
\pgfpathlineto{\pgfqpoint{1.795713in}{1.654573in}}%
\pgfpathlineto{\pgfqpoint{1.807350in}{1.655295in}}%
\pgfpathlineto{\pgfqpoint{1.815495in}{1.655680in}}%
\pgfpathlineto{\pgfqpoint{1.830623in}{1.656521in}}%
\pgfpathlineto{\pgfqpoint{1.831787in}{1.656533in}}%
\pgfpathlineto{\pgfqpoint{1.831787in}{1.656533in}}%
\pgfusepath{stroke}%
\end{pgfscope}%
\begin{pgfscope}%
\pgfpathrectangle{\pgfqpoint{0.611162in}{0.467408in}}{\pgfqpoint{1.278750in}{1.245750in}}%
\pgfusepath{clip}%
\pgfsetroundcap%
\pgfsetroundjoin%
\pgfsetlinewidth{1.003750pt}%
\definecolor{currentstroke}{rgb}{0.866667,0.517647,0.321569}%
\pgfsetstrokecolor{currentstroke}%
\pgfsetdash{}{0pt}%
\pgfpathmoveto{\pgfqpoint{0.669287in}{0.524033in}}%
\pgfpathlineto{\pgfqpoint{0.673941in}{0.649989in}}%
\pgfpathlineto{\pgfqpoint{0.684414in}{0.971259in}}%
\pgfpathlineto{\pgfqpoint{0.687905in}{1.058327in}}%
\pgfpathlineto{\pgfqpoint{0.689069in}{1.077648in}}%
\pgfpathlineto{\pgfqpoint{0.692560in}{1.123760in}}%
\pgfpathlineto{\pgfqpoint{0.694887in}{1.150284in}}%
\pgfpathlineto{\pgfqpoint{0.697215in}{1.173905in}}%
\pgfpathlineto{\pgfqpoint{0.700706in}{1.205018in}}%
\pgfpathlineto{\pgfqpoint{0.704197in}{1.232138in}}%
\pgfpathlineto{\pgfqpoint{0.707688in}{1.256040in}}%
\pgfpathlineto{\pgfqpoint{0.711179in}{1.277361in}}%
\pgfpathlineto{\pgfqpoint{0.714670in}{1.296066in}}%
\pgfpathlineto{\pgfqpoint{0.718161in}{1.312988in}}%
\pgfpathlineto{\pgfqpoint{0.721652in}{1.328317in}}%
\pgfpathlineto{\pgfqpoint{0.725143in}{1.342334in}}%
\pgfpathlineto{\pgfqpoint{0.729797in}{1.358720in}}%
\pgfpathlineto{\pgfqpoint{0.734452in}{1.373583in}}%
\pgfpathlineto{\pgfqpoint{0.737943in}{1.383834in}}%
\pgfpathlineto{\pgfqpoint{0.742597in}{1.396240in}}%
\pgfpathlineto{\pgfqpoint{0.747252in}{1.407544in}}%
\pgfpathlineto{\pgfqpoint{0.751907in}{1.417969in}}%
\pgfpathlineto{\pgfqpoint{0.755398in}{1.425334in}}%
\pgfpathlineto{\pgfqpoint{0.760052in}{1.434225in}}%
\pgfpathlineto{\pgfqpoint{0.770525in}{1.451655in}}%
\pgfpathlineto{\pgfqpoint{0.774016in}{1.456949in}}%
\pgfpathlineto{\pgfqpoint{0.782162in}{1.468132in}}%
\pgfpathlineto{\pgfqpoint{0.786817in}{1.474052in}}%
\pgfpathlineto{\pgfqpoint{0.793799in}{1.482175in}}%
\pgfpathlineto{\pgfqpoint{0.797290in}{1.486104in}}%
\pgfpathlineto{\pgfqpoint{0.803108in}{1.492504in}}%
\pgfpathlineto{\pgfqpoint{0.811254in}{1.500344in}}%
\pgfpathlineto{\pgfqpoint{0.825218in}{1.511861in}}%
\pgfpathlineto{\pgfqpoint{0.833363in}{1.517784in}}%
\pgfpathlineto{\pgfqpoint{0.838018in}{1.520882in}}%
\pgfpathlineto{\pgfqpoint{0.843836in}{1.524805in}}%
\pgfpathlineto{\pgfqpoint{0.855473in}{1.531756in}}%
\pgfpathlineto{\pgfqpoint{0.862455in}{1.535816in}}%
\pgfpathlineto{\pgfqpoint{0.869437in}{1.539850in}}%
\pgfpathlineto{\pgfqpoint{0.875255in}{1.543042in}}%
\pgfpathlineto{\pgfqpoint{0.879910in}{1.545775in}}%
\pgfpathlineto{\pgfqpoint{0.893874in}{1.552636in}}%
\pgfpathlineto{\pgfqpoint{0.900856in}{1.555567in}}%
\pgfpathlineto{\pgfqpoint{0.904347in}{1.557047in}}%
\pgfpathlineto{\pgfqpoint{0.907838in}{1.558451in}}%
\pgfpathlineto{\pgfqpoint{0.915983in}{1.561903in}}%
\pgfpathlineto{\pgfqpoint{0.927620in}{1.566328in}}%
\pgfpathlineto{\pgfqpoint{0.933438in}{1.568560in}}%
\pgfpathlineto{\pgfqpoint{0.954384in}{1.575966in}}%
\pgfpathlineto{\pgfqpoint{0.961366in}{1.578236in}}%
\pgfpathlineto{\pgfqpoint{0.964857in}{1.579451in}}%
\pgfpathlineto{\pgfqpoint{0.969512in}{1.580751in}}%
\pgfpathlineto{\pgfqpoint{0.973003in}{1.581894in}}%
\pgfpathlineto{\pgfqpoint{0.996276in}{1.588821in}}%
\pgfpathlineto{\pgfqpoint{0.999767in}{1.589887in}}%
\pgfpathlineto{\pgfqpoint{1.019549in}{1.594818in}}%
\pgfpathlineto{\pgfqpoint{1.024204in}{1.595838in}}%
\pgfpathlineto{\pgfqpoint{1.057950in}{1.603857in}}%
\pgfpathlineto{\pgfqpoint{1.063769in}{1.605000in}}%
\pgfpathlineto{\pgfqpoint{1.067260in}{1.605780in}}%
\pgfpathlineto{\pgfqpoint{1.069587in}{1.606202in}}%
\pgfpathlineto{\pgfqpoint{1.101006in}{1.612439in}}%
\pgfpathlineto{\pgfqpoint{1.105661in}{1.613407in}}%
\pgfpathlineto{\pgfqpoint{1.148716in}{1.620909in}}%
\pgfpathlineto{\pgfqpoint{1.154534in}{1.621702in}}%
\pgfpathlineto{\pgfqpoint{1.226682in}{1.632426in}}%
\pgfpathlineto{\pgfqpoint{1.232500in}{1.633200in}}%
\pgfpathlineto{\pgfqpoint{1.238318in}{1.633961in}}%
\pgfpathlineto{\pgfqpoint{1.247627in}{1.635139in}}%
\pgfpathlineto{\pgfqpoint{1.249955in}{1.635445in}}%
\pgfpathlineto{\pgfqpoint{1.249955in}{1.635445in}}%
\pgfusepath{stroke}%
\end{pgfscope}%
\begin{pgfscope}%
\pgfpathrectangle{\pgfqpoint{0.611162in}{0.467408in}}{\pgfqpoint{1.278750in}{1.245750in}}%
\pgfusepath{clip}%
\pgfsetroundcap%
\pgfsetroundjoin%
\pgfsetlinewidth{1.003750pt}%
\definecolor{currentstroke}{rgb}{0.333333,0.658824,0.407843}%
\pgfsetstrokecolor{currentstroke}%
\pgfsetdash{}{0pt}%
\pgfpathmoveto{\pgfqpoint{0.669287in}{0.524033in}}%
\pgfpathlineto{\pgfqpoint{0.670450in}{0.630159in}}%
\pgfpathlineto{\pgfqpoint{0.671614in}{0.707486in}}%
\pgfpathlineto{\pgfqpoint{0.672778in}{0.792341in}}%
\pgfpathlineto{\pgfqpoint{0.673941in}{0.872028in}}%
\pgfpathlineto{\pgfqpoint{0.675105in}{0.947598in}}%
\pgfpathlineto{\pgfqpoint{0.676269in}{1.017732in}}%
\pgfpathlineto{\pgfqpoint{0.677432in}{1.064808in}}%
\pgfpathlineto{\pgfqpoint{0.678596in}{1.104548in}}%
\pgfpathlineto{\pgfqpoint{0.679760in}{1.139854in}}%
\pgfpathlineto{\pgfqpoint{0.680923in}{1.169946in}}%
\pgfpathlineto{\pgfqpoint{0.682087in}{1.195966in}}%
\pgfpathlineto{\pgfqpoint{0.683251in}{1.219735in}}%
\pgfpathlineto{\pgfqpoint{0.684414in}{1.241106in}}%
\pgfpathlineto{\pgfqpoint{0.685578in}{1.260131in}}%
\pgfpathlineto{\pgfqpoint{0.686742in}{1.277195in}}%
\pgfpathlineto{\pgfqpoint{0.687905in}{1.293339in}}%
\pgfpathlineto{\pgfqpoint{0.689069in}{1.307119in}}%
\pgfpathlineto{\pgfqpoint{0.690233in}{1.320336in}}%
\pgfpathlineto{\pgfqpoint{0.691396in}{1.332635in}}%
\pgfpathlineto{\pgfqpoint{0.692560in}{1.343365in}}%
\pgfpathlineto{\pgfqpoint{0.693724in}{1.353507in}}%
\pgfpathlineto{\pgfqpoint{0.694887in}{1.363193in}}%
\pgfpathlineto{\pgfqpoint{0.696051in}{1.372217in}}%
\pgfpathlineto{\pgfqpoint{0.697215in}{1.380470in}}%
\pgfpathlineto{\pgfqpoint{0.698378in}{1.388660in}}%
\pgfpathlineto{\pgfqpoint{0.699542in}{1.395918in}}%
\pgfpathlineto{\pgfqpoint{0.700706in}{1.403257in}}%
\pgfpathlineto{\pgfqpoint{0.701869in}{1.409897in}}%
\pgfpathlineto{\pgfqpoint{0.703033in}{1.416265in}}%
\pgfpathlineto{\pgfqpoint{0.704197in}{1.421886in}}%
\pgfpathlineto{\pgfqpoint{0.705360in}{1.427677in}}%
\pgfpathlineto{\pgfqpoint{0.706524in}{1.432883in}}%
\pgfpathlineto{\pgfqpoint{0.707688in}{1.437905in}}%
\pgfpathlineto{\pgfqpoint{0.708851in}{1.442789in}}%
\pgfpathlineto{\pgfqpoint{0.710015in}{1.447374in}}%
\pgfpathlineto{\pgfqpoint{0.711179in}{1.451631in}}%
\pgfpathlineto{\pgfqpoint{0.712342in}{1.455880in}}%
\pgfpathlineto{\pgfqpoint{0.713506in}{1.460148in}}%
\pgfpathlineto{\pgfqpoint{0.714670in}{1.463916in}}%
\pgfpathlineto{\pgfqpoint{0.715833in}{1.467879in}}%
\pgfpathlineto{\pgfqpoint{0.716997in}{1.471408in}}%
\pgfpathlineto{\pgfqpoint{0.718161in}{1.475241in}}%
\pgfpathlineto{\pgfqpoint{0.719324in}{1.478544in}}%
\pgfpathlineto{\pgfqpoint{0.720488in}{1.481575in}}%
\pgfpathlineto{\pgfqpoint{0.721652in}{1.484745in}}%
\pgfpathlineto{\pgfqpoint{0.722815in}{1.487588in}}%
\pgfpathlineto{\pgfqpoint{0.723979in}{1.490293in}}%
\pgfpathlineto{\pgfqpoint{0.725143in}{1.493165in}}%
\pgfpathlineto{\pgfqpoint{0.726306in}{1.496031in}}%
\pgfpathlineto{\pgfqpoint{0.727470in}{1.498840in}}%
\pgfpathlineto{\pgfqpoint{0.728634in}{1.501465in}}%
\pgfpathlineto{\pgfqpoint{0.729797in}{1.503889in}}%
\pgfpathlineto{\pgfqpoint{0.730961in}{1.506182in}}%
\pgfpathlineto{\pgfqpoint{0.732125in}{1.508594in}}%
\pgfpathlineto{\pgfqpoint{0.733288in}{1.510888in}}%
\pgfpathlineto{\pgfqpoint{0.734452in}{1.513157in}}%
\pgfpathlineto{\pgfqpoint{0.735615in}{1.515365in}}%
\pgfpathlineto{\pgfqpoint{0.736779in}{1.517213in}}%
\pgfpathlineto{\pgfqpoint{0.737943in}{1.519587in}}%
\pgfpathlineto{\pgfqpoint{0.739106in}{1.521252in}}%
\pgfpathlineto{\pgfqpoint{0.740270in}{1.523200in}}%
\pgfpathlineto{\pgfqpoint{0.741434in}{1.525236in}}%
\pgfpathlineto{\pgfqpoint{0.742597in}{1.527253in}}%
\pgfpathlineto{\pgfqpoint{0.743761in}{1.529039in}}%
\pgfpathlineto{\pgfqpoint{0.744925in}{1.530769in}}%
\pgfpathlineto{\pgfqpoint{0.746088in}{1.532613in}}%
\pgfpathlineto{\pgfqpoint{0.747252in}{1.534329in}}%
\pgfpathlineto{\pgfqpoint{0.748416in}{1.535793in}}%
\pgfpathlineto{\pgfqpoint{0.749579in}{1.537501in}}%
\pgfpathlineto{\pgfqpoint{0.750743in}{1.538760in}}%
\pgfpathlineto{\pgfqpoint{0.751907in}{1.540145in}}%
\pgfpathlineto{\pgfqpoint{0.753070in}{1.541610in}}%
\pgfpathlineto{\pgfqpoint{0.754234in}{1.542815in}}%
\pgfpathlineto{\pgfqpoint{0.755398in}{1.544296in}}%
\pgfpathlineto{\pgfqpoint{0.756561in}{1.545986in}}%
\pgfpathlineto{\pgfqpoint{0.757725in}{1.547148in}}%
\pgfpathlineto{\pgfqpoint{0.758889in}{1.548560in}}%
\pgfpathlineto{\pgfqpoint{0.760052in}{1.549996in}}%
\pgfpathlineto{\pgfqpoint{0.761216in}{1.551008in}}%
\pgfpathlineto{\pgfqpoint{0.762380in}{1.552318in}}%
\pgfpathlineto{\pgfqpoint{0.763543in}{1.553529in}}%
\pgfpathlineto{\pgfqpoint{0.764707in}{1.554675in}}%
\pgfpathlineto{\pgfqpoint{0.765871in}{1.556258in}}%
\pgfpathlineto{\pgfqpoint{0.767034in}{1.557342in}}%
\pgfpathlineto{\pgfqpoint{0.768198in}{1.558634in}}%
\pgfpathlineto{\pgfqpoint{0.769362in}{1.559470in}}%
\pgfpathlineto{\pgfqpoint{0.770525in}{1.560607in}}%
\pgfpathlineto{\pgfqpoint{0.771689in}{1.561835in}}%
\pgfpathlineto{\pgfqpoint{0.772853in}{1.562895in}}%
\pgfpathlineto{\pgfqpoint{0.774016in}{1.564027in}}%
\pgfpathlineto{\pgfqpoint{0.775180in}{1.564960in}}%
\pgfpathlineto{\pgfqpoint{0.776344in}{1.566141in}}%
\pgfpathlineto{\pgfqpoint{0.777507in}{1.567141in}}%
\pgfpathlineto{\pgfqpoint{0.778671in}{1.568026in}}%
\pgfpathlineto{\pgfqpoint{0.779835in}{1.569064in}}%
\pgfpathlineto{\pgfqpoint{0.780998in}{1.569898in}}%
\pgfpathlineto{\pgfqpoint{0.782162in}{1.570776in}}%
\pgfpathlineto{\pgfqpoint{0.783326in}{1.571650in}}%
\pgfpathlineto{\pgfqpoint{0.784489in}{1.572608in}}%
\pgfusepath{stroke}%
\end{pgfscope}%
\begin{pgfscope}%
\pgfsetrectcap%
\pgfsetmiterjoin%
\pgfsetlinewidth{0.752812pt}%
\definecolor{currentstroke}{rgb}{0.700000,0.700000,0.700000}%
\pgfsetstrokecolor{currentstroke}%
\pgfsetdash{}{0pt}%
\pgfpathmoveto{\pgfqpoint{0.611162in}{0.467408in}}%
\pgfpathlineto{\pgfqpoint{0.611162in}{1.713158in}}%
\pgfusepath{stroke}%
\end{pgfscope}%
\begin{pgfscope}%
\pgfsetrectcap%
\pgfsetmiterjoin%
\pgfsetlinewidth{0.752812pt}%
\definecolor{currentstroke}{rgb}{0.700000,0.700000,0.700000}%
\pgfsetstrokecolor{currentstroke}%
\pgfsetdash{}{0pt}%
\pgfpathmoveto{\pgfqpoint{1.889912in}{0.467408in}}%
\pgfpathlineto{\pgfqpoint{1.889912in}{1.713158in}}%
\pgfusepath{stroke}%
\end{pgfscope}%
\begin{pgfscope}%
\pgfsetrectcap%
\pgfsetmiterjoin%
\pgfsetlinewidth{0.752812pt}%
\definecolor{currentstroke}{rgb}{0.700000,0.700000,0.700000}%
\pgfsetstrokecolor{currentstroke}%
\pgfsetdash{}{0pt}%
\pgfpathmoveto{\pgfqpoint{0.611162in}{0.467408in}}%
\pgfpathlineto{\pgfqpoint{1.889912in}{0.467408in}}%
\pgfusepath{stroke}%
\end{pgfscope}%
\begin{pgfscope}%
\pgfsetrectcap%
\pgfsetmiterjoin%
\pgfsetlinewidth{0.752812pt}%
\definecolor{currentstroke}{rgb}{0.700000,0.700000,0.700000}%
\pgfsetstrokecolor{currentstroke}%
\pgfsetdash{}{0pt}%
\pgfpathmoveto{\pgfqpoint{0.611162in}{1.713158in}}%
\pgfpathlineto{\pgfqpoint{1.889912in}{1.713158in}}%
\pgfusepath{stroke}%
\end{pgfscope}%
\end{pgfpicture}%
\makeatother%
\endgroup%
}} &
      \subfloat[]{\resizebox{0.48\linewidth}{!}{%% Creator: Matplotlib, PGF backend
%%
%% To include the figure in your LaTeX document, write
%%   \input{<filename>.pgf}
%%
%% Make sure the required packages are loaded in your preamble
%%   \usepackage{pgf}
%%
%% and, on pdftex
%%   \usepackage[utf8]{inputenc}\DeclareUnicodeCharacter{2212}{-}
%%
%% or, on luatex and xetex
%%   \usepackage{unicode-math}
%%
%% Figures using additional raster images can only be included by \input if
%% they are in the same directory as the main LaTeX file. For loading figures
%% from other directories you can use the `import` package
%%   \usepackage{import}
%%
%% and then include the figures with
%%   \import{<path to file>}{<filename>.pgf}
%%
%% Matplotlib used the following preamble
%%   \usepackage[utf8]{inputenc}
%%   \usepackage[T1]{fontenc}
%%   \usepackage{amsmath}
%%   \newcommand*{\mat}[1]{\boldsymbol{#1}}
%%
\begingroup%
\makeatletter%
\begin{pgfpicture}%
\pgfpathrectangle{\pgfpointorigin}{\pgfqpoint{1.959185in}{1.813158in}}%
\pgfusepath{use as bounding box, clip}%
\begin{pgfscope}%
\pgfsetbuttcap%
\pgfsetmiterjoin%
\definecolor{currentfill}{rgb}{1.000000,1.000000,1.000000}%
\pgfsetfillcolor{currentfill}%
\pgfsetlinewidth{0.000000pt}%
\definecolor{currentstroke}{rgb}{1.000000,1.000000,1.000000}%
\pgfsetstrokecolor{currentstroke}%
\pgfsetstrokeopacity{0.000000}%
\pgfsetdash{}{0pt}%
\pgfpathmoveto{\pgfqpoint{0.000000in}{0.000000in}}%
\pgfpathlineto{\pgfqpoint{1.959185in}{0.000000in}}%
\pgfpathlineto{\pgfqpoint{1.959185in}{1.813158in}}%
\pgfpathlineto{\pgfqpoint{0.000000in}{1.813158in}}%
\pgfpathclose%
\pgfusepath{fill}%
\end{pgfscope}%
\begin{pgfscope}%
\pgfsetbuttcap%
\pgfsetmiterjoin%
\definecolor{currentfill}{rgb}{1.000000,1.000000,1.000000}%
\pgfsetfillcolor{currentfill}%
\pgfsetlinewidth{0.000000pt}%
\definecolor{currentstroke}{rgb}{0.000000,0.000000,0.000000}%
\pgfsetstrokecolor{currentstroke}%
\pgfsetstrokeopacity{0.000000}%
\pgfsetdash{}{0pt}%
\pgfpathmoveto{\pgfqpoint{0.519339in}{0.467408in}}%
\pgfpathlineto{\pgfqpoint{1.798089in}{0.467408in}}%
\pgfpathlineto{\pgfqpoint{1.798089in}{1.713158in}}%
\pgfpathlineto{\pgfqpoint{0.519339in}{1.713158in}}%
\pgfpathclose%
\pgfusepath{fill}%
\end{pgfscope}%
\begin{pgfscope}%
\pgfpathrectangle{\pgfqpoint{0.519339in}{0.467408in}}{\pgfqpoint{1.278750in}{1.245750in}}%
\pgfusepath{clip}%
\pgfsetroundcap%
\pgfsetroundjoin%
\pgfsetlinewidth{0.501875pt}%
\definecolor{currentstroke}{rgb}{0.800000,0.800000,0.800000}%
\pgfsetstrokecolor{currentstroke}%
\pgfsetdash{}{0pt}%
\pgfpathmoveto{\pgfqpoint{0.577464in}{0.467408in}}%
\pgfpathlineto{\pgfqpoint{0.577464in}{1.713158in}}%
\pgfusepath{stroke}%
\end{pgfscope}%
\begin{pgfscope}%
\definecolor{textcolor}{rgb}{0.150000,0.150000,0.150000}%
\pgfsetstrokecolor{textcolor}%
\pgfsetfillcolor{textcolor}%
\pgftext[x=0.577464in,y=0.377130in,,top]{\color{textcolor}\rmfamily\fontsize{8.000000}{9.600000}\selectfont \(\displaystyle {0}\)}%
\end{pgfscope}%
\begin{pgfscope}%
\pgfpathrectangle{\pgfqpoint{0.519339in}{0.467408in}}{\pgfqpoint{1.278750in}{1.245750in}}%
\pgfusepath{clip}%
\pgfsetroundcap%
\pgfsetroundjoin%
\pgfsetlinewidth{0.501875pt}%
\definecolor{currentstroke}{rgb}{0.800000,0.800000,0.800000}%
\pgfsetstrokecolor{currentstroke}%
\pgfsetdash{}{0pt}%
\pgfpathmoveto{\pgfqpoint{1.159296in}{0.467408in}}%
\pgfpathlineto{\pgfqpoint{1.159296in}{1.713158in}}%
\pgfusepath{stroke}%
\end{pgfscope}%
\begin{pgfscope}%
\definecolor{textcolor}{rgb}{0.150000,0.150000,0.150000}%
\pgfsetstrokecolor{textcolor}%
\pgfsetfillcolor{textcolor}%
\pgftext[x=1.159296in,y=0.377130in,,top]{\color{textcolor}\rmfamily\fontsize{8.000000}{9.600000}\selectfont \(\displaystyle {500}\)}%
\end{pgfscope}%
\begin{pgfscope}%
\pgfpathrectangle{\pgfqpoint{0.519339in}{0.467408in}}{\pgfqpoint{1.278750in}{1.245750in}}%
\pgfusepath{clip}%
\pgfsetroundcap%
\pgfsetroundjoin%
\pgfsetlinewidth{0.501875pt}%
\definecolor{currentstroke}{rgb}{0.800000,0.800000,0.800000}%
\pgfsetstrokecolor{currentstroke}%
\pgfsetdash{}{0pt}%
\pgfpathmoveto{\pgfqpoint{1.741128in}{0.467408in}}%
\pgfpathlineto{\pgfqpoint{1.741128in}{1.713158in}}%
\pgfusepath{stroke}%
\end{pgfscope}%
\begin{pgfscope}%
\definecolor{textcolor}{rgb}{0.150000,0.150000,0.150000}%
\pgfsetstrokecolor{textcolor}%
\pgfsetfillcolor{textcolor}%
\pgftext[x=1.741128in,y=0.377130in,,top]{\color{textcolor}\rmfamily\fontsize{8.000000}{9.600000}\selectfont \(\displaystyle {1000}\)}%
\end{pgfscope}%
\begin{pgfscope}%
\definecolor{textcolor}{rgb}{0.150000,0.150000,0.150000}%
\pgfsetstrokecolor{textcolor}%
\pgfsetfillcolor{textcolor}%
\pgftext[x=1.158714in,y=0.223450in,,top]{\color{textcolor}\rmfamily\fontsize{10.000000}{12.000000}\selectfont Epochs \(\displaystyle k\)}%
\end{pgfscope}%
\begin{pgfscope}%
\pgfpathrectangle{\pgfqpoint{0.519339in}{0.467408in}}{\pgfqpoint{1.278750in}{1.245750in}}%
\pgfusepath{clip}%
\pgfsetroundcap%
\pgfsetroundjoin%
\pgfsetlinewidth{0.501875pt}%
\definecolor{currentstroke}{rgb}{0.800000,0.800000,0.800000}%
\pgfsetstrokecolor{currentstroke}%
\pgfsetdash{}{0pt}%
\pgfpathmoveto{\pgfqpoint{0.519339in}{0.857638in}}%
\pgfpathlineto{\pgfqpoint{1.798089in}{0.857638in}}%
\pgfusepath{stroke}%
\end{pgfscope}%
\begin{pgfscope}%
\definecolor{textcolor}{rgb}{0.150000,0.150000,0.150000}%
\pgfsetstrokecolor{textcolor}%
\pgfsetfillcolor{textcolor}%
\pgftext[x=0.278211in, y=0.819375in, left, base]{\color{textcolor}\rmfamily\fontsize{8.000000}{9.600000}\selectfont \(\displaystyle {0.5}\)}%
\end{pgfscope}%
\begin{pgfscope}%
\pgfpathrectangle{\pgfqpoint{0.519339in}{0.467408in}}{\pgfqpoint{1.278750in}{1.245750in}}%
\pgfusepath{clip}%
\pgfsetroundcap%
\pgfsetroundjoin%
\pgfsetlinewidth{0.501875pt}%
\definecolor{currentstroke}{rgb}{0.800000,0.800000,0.800000}%
\pgfsetstrokecolor{currentstroke}%
\pgfsetdash{}{0pt}%
\pgfpathmoveto{\pgfqpoint{0.519339in}{1.251138in}}%
\pgfpathlineto{\pgfqpoint{1.798089in}{1.251138in}}%
\pgfusepath{stroke}%
\end{pgfscope}%
\begin{pgfscope}%
\definecolor{textcolor}{rgb}{0.150000,0.150000,0.150000}%
\pgfsetstrokecolor{textcolor}%
\pgfsetfillcolor{textcolor}%
\pgftext[x=0.278211in, y=1.212876in, left, base]{\color{textcolor}\rmfamily\fontsize{8.000000}{9.600000}\selectfont \(\displaystyle {0.6}\)}%
\end{pgfscope}%
\begin{pgfscope}%
\pgfpathrectangle{\pgfqpoint{0.519339in}{0.467408in}}{\pgfqpoint{1.278750in}{1.245750in}}%
\pgfusepath{clip}%
\pgfsetroundcap%
\pgfsetroundjoin%
\pgfsetlinewidth{0.501875pt}%
\definecolor{currentstroke}{rgb}{0.800000,0.800000,0.800000}%
\pgfsetstrokecolor{currentstroke}%
\pgfsetdash{}{0pt}%
\pgfpathmoveto{\pgfqpoint{0.519339in}{1.644639in}}%
\pgfpathlineto{\pgfqpoint{1.798089in}{1.644639in}}%
\pgfusepath{stroke}%
\end{pgfscope}%
\begin{pgfscope}%
\definecolor{textcolor}{rgb}{0.150000,0.150000,0.150000}%
\pgfsetstrokecolor{textcolor}%
\pgfsetfillcolor{textcolor}%
\pgftext[x=0.278211in, y=1.606377in, left, base]{\color{textcolor}\rmfamily\fontsize{8.000000}{9.600000}\selectfont \(\displaystyle {0.7}\)}%
\end{pgfscope}%
\begin{pgfscope}%
\definecolor{textcolor}{rgb}{0.150000,0.150000,0.150000}%
\pgfsetstrokecolor{textcolor}%
\pgfsetfillcolor{textcolor}%
\pgftext[x=0.222655in,y=1.090283in,,bottom,rotate=90.000000]{\color{textcolor}\rmfamily\fontsize{10.000000}{12.000000}\selectfont Accuracy}%
\end{pgfscope}%
\begin{pgfscope}%
\pgfpathrectangle{\pgfqpoint{0.519339in}{0.467408in}}{\pgfqpoint{1.278750in}{1.245750in}}%
\pgfusepath{clip}%
\pgfsetroundcap%
\pgfsetroundjoin%
\pgfsetlinewidth{1.003750pt}%
\definecolor{currentstroke}{rgb}{0.298039,0.447059,0.690196}%
\pgfsetstrokecolor{currentstroke}%
\pgfsetdash{}{0pt}%
\pgfpathmoveto{\pgfqpoint{0.577464in}{1.656533in}}%
\pgfpathlineto{\pgfqpoint{0.578628in}{1.637345in}}%
\pgfpathlineto{\pgfqpoint{0.580955in}{1.584881in}}%
\pgfpathlineto{\pgfqpoint{0.583283in}{1.522459in}}%
\pgfpathlineto{\pgfqpoint{0.592592in}{1.290421in}}%
\pgfpathlineto{\pgfqpoint{0.596083in}{1.211078in}}%
\pgfpathlineto{\pgfqpoint{0.598410in}{1.169140in}}%
\pgfpathlineto{\pgfqpoint{0.601901in}{1.116595in}}%
\pgfpathlineto{\pgfqpoint{0.605392in}{1.070123in}}%
\pgfpathlineto{\pgfqpoint{0.606556in}{1.058222in}}%
\pgfpathlineto{\pgfqpoint{0.608883in}{1.039438in}}%
\pgfpathlineto{\pgfqpoint{0.610047in}{1.031018in}}%
\pgfpathlineto{\pgfqpoint{0.611211in}{1.020331in}}%
\pgfpathlineto{\pgfqpoint{0.618193in}{0.971430in}}%
\pgfpathlineto{\pgfqpoint{0.620520in}{0.956047in}}%
\pgfpathlineto{\pgfqpoint{0.621684in}{0.950299in}}%
\pgfpathlineto{\pgfqpoint{0.624011in}{0.936373in}}%
\pgfpathlineto{\pgfqpoint{0.625175in}{0.929168in}}%
\pgfpathlineto{\pgfqpoint{0.627502in}{0.916781in}}%
\pgfpathlineto{\pgfqpoint{0.629829in}{0.905041in}}%
\pgfpathlineto{\pgfqpoint{0.630993in}{0.899293in}}%
\pgfpathlineto{\pgfqpoint{0.632157in}{0.895245in}}%
\pgfpathlineto{\pgfqpoint{0.636811in}{0.872089in}}%
\pgfpathlineto{\pgfqpoint{0.639139in}{0.864398in}}%
\pgfpathlineto{\pgfqpoint{0.641466in}{0.855897in}}%
\pgfpathlineto{\pgfqpoint{0.650775in}{0.821326in}}%
\pgfpathlineto{\pgfqpoint{0.654266in}{0.810234in}}%
\pgfpathlineto{\pgfqpoint{0.658921in}{0.797685in}}%
\pgfpathlineto{\pgfqpoint{0.660085in}{0.794609in}}%
\pgfpathlineto{\pgfqpoint{0.661248in}{0.790156in}}%
\pgfpathlineto{\pgfqpoint{0.668230in}{0.771777in}}%
\pgfpathlineto{\pgfqpoint{0.669394in}{0.769186in}}%
\pgfpathlineto{\pgfqpoint{0.670558in}{0.765705in}}%
\pgfpathlineto{\pgfqpoint{0.671721in}{0.763438in}}%
\pgfpathlineto{\pgfqpoint{0.674049in}{0.758257in}}%
\pgfpathlineto{\pgfqpoint{0.676376in}{0.753966in}}%
\pgfpathlineto{\pgfqpoint{0.678703in}{0.748703in}}%
\pgfpathlineto{\pgfqpoint{0.679867in}{0.746112in}}%
\pgfpathlineto{\pgfqpoint{0.681031in}{0.744007in}}%
\pgfpathlineto{\pgfqpoint{0.682194in}{0.741497in}}%
\pgfpathlineto{\pgfqpoint{0.683358in}{0.739554in}}%
\pgfpathlineto{\pgfqpoint{0.688012in}{0.730567in}}%
\pgfpathlineto{\pgfqpoint{0.689176in}{0.728624in}}%
\pgfpathlineto{\pgfqpoint{0.691503in}{0.724171in}}%
\pgfpathlineto{\pgfqpoint{0.698485in}{0.713241in}}%
\pgfpathlineto{\pgfqpoint{0.699649in}{0.712270in}}%
\pgfpathlineto{\pgfqpoint{0.705467in}{0.706117in}}%
\pgfpathlineto{\pgfqpoint{0.706631in}{0.704093in}}%
\pgfpathlineto{\pgfqpoint{0.708958in}{0.701421in}}%
\pgfpathlineto{\pgfqpoint{0.711286in}{0.699478in}}%
\pgfpathlineto{\pgfqpoint{0.713613in}{0.696482in}}%
\pgfpathlineto{\pgfqpoint{0.715940in}{0.693973in}}%
\pgfpathlineto{\pgfqpoint{0.717104in}{0.693487in}}%
\pgfpathlineto{\pgfqpoint{0.719431in}{0.690491in}}%
\pgfpathlineto{\pgfqpoint{0.720595in}{0.689762in}}%
\pgfpathlineto{\pgfqpoint{0.724086in}{0.686119in}}%
\pgfpathlineto{\pgfqpoint{0.726413in}{0.684257in}}%
\pgfpathlineto{\pgfqpoint{0.729904in}{0.681747in}}%
\pgfpathlineto{\pgfqpoint{0.732232in}{0.679237in}}%
\pgfpathlineto{\pgfqpoint{0.735723in}{0.676080in}}%
\pgfpathlineto{\pgfqpoint{0.736886in}{0.675594in}}%
\pgfpathlineto{\pgfqpoint{0.738050in}{0.674380in}}%
\pgfpathlineto{\pgfqpoint{0.740377in}{0.672679in}}%
\pgfpathlineto{\pgfqpoint{0.741541in}{0.671303in}}%
\pgfpathlineto{\pgfqpoint{0.742705in}{0.670331in}}%
\pgfpathlineto{\pgfqpoint{0.743868in}{0.669684in}}%
\pgfpathlineto{\pgfqpoint{0.747359in}{0.666607in}}%
\pgfpathlineto{\pgfqpoint{0.754341in}{0.661588in}}%
\pgfpathlineto{\pgfqpoint{0.755505in}{0.660292in}}%
\pgfpathlineto{\pgfqpoint{0.756669in}{0.659564in}}%
\pgfpathlineto{\pgfqpoint{0.758996in}{0.658673in}}%
\pgfpathlineto{\pgfqpoint{0.761323in}{0.657135in}}%
\pgfpathlineto{\pgfqpoint{0.762487in}{0.655839in}}%
\pgfpathlineto{\pgfqpoint{0.764814in}{0.654301in}}%
\pgfpathlineto{\pgfqpoint{0.765978in}{0.653410in}}%
\pgfpathlineto{\pgfqpoint{0.770633in}{0.652115in}}%
\pgfpathlineto{\pgfqpoint{0.772960in}{0.650901in}}%
\pgfpathlineto{\pgfqpoint{0.774124in}{0.650415in}}%
\pgfpathlineto{\pgfqpoint{0.776451in}{0.649119in}}%
\pgfpathlineto{\pgfqpoint{0.777615in}{0.647986in}}%
\pgfpathlineto{\pgfqpoint{0.778778in}{0.647581in}}%
\pgfpathlineto{\pgfqpoint{0.782269in}{0.645395in}}%
\pgfpathlineto{\pgfqpoint{0.784597in}{0.644505in}}%
\pgfpathlineto{\pgfqpoint{0.788088in}{0.642157in}}%
\pgfpathlineto{\pgfqpoint{0.789251in}{0.641995in}}%
\pgfpathlineto{\pgfqpoint{0.790415in}{0.641266in}}%
\pgfpathlineto{\pgfqpoint{0.791579in}{0.640861in}}%
\pgfpathlineto{\pgfqpoint{0.793906in}{0.639566in}}%
\pgfpathlineto{\pgfqpoint{0.796233in}{0.638675in}}%
\pgfpathlineto{\pgfqpoint{0.798561in}{0.637461in}}%
\pgfpathlineto{\pgfqpoint{0.800888in}{0.636084in}}%
\pgfpathlineto{\pgfqpoint{0.803215in}{0.635113in}}%
\pgfpathlineto{\pgfqpoint{0.814852in}{0.631065in}}%
\pgfpathlineto{\pgfqpoint{0.817179in}{0.630822in}}%
\pgfpathlineto{\pgfqpoint{0.824161in}{0.628069in}}%
\pgfpathlineto{\pgfqpoint{0.825325in}{0.627664in}}%
\pgfpathlineto{\pgfqpoint{0.826488in}{0.627017in}}%
\pgfpathlineto{\pgfqpoint{0.828816in}{0.626693in}}%
\pgfpathlineto{\pgfqpoint{0.829979in}{0.626288in}}%
\pgfpathlineto{\pgfqpoint{0.831143in}{0.625640in}}%
\pgfpathlineto{\pgfqpoint{0.832307in}{0.625559in}}%
\pgfpathlineto{\pgfqpoint{0.839289in}{0.623131in}}%
\pgfpathlineto{\pgfqpoint{0.840452in}{0.623050in}}%
\pgfpathlineto{\pgfqpoint{0.841616in}{0.622483in}}%
\pgfpathlineto{\pgfqpoint{0.843943in}{0.621997in}}%
\pgfpathlineto{\pgfqpoint{0.845107in}{0.621349in}}%
\pgfpathlineto{\pgfqpoint{0.846271in}{0.621268in}}%
\pgfpathlineto{\pgfqpoint{0.848598in}{0.620540in}}%
\pgfpathlineto{\pgfqpoint{0.853253in}{0.619406in}}%
\pgfpathlineto{\pgfqpoint{0.855580in}{0.618759in}}%
\pgfpathlineto{\pgfqpoint{0.860235in}{0.617706in}}%
\pgfpathlineto{\pgfqpoint{0.862562in}{0.617058in}}%
\pgfpathlineto{\pgfqpoint{0.867217in}{0.615925in}}%
\pgfpathlineto{\pgfqpoint{0.868380in}{0.615196in}}%
\pgfpathlineto{\pgfqpoint{0.869544in}{0.615034in}}%
\pgfpathlineto{\pgfqpoint{0.871871in}{0.614306in}}%
\pgfpathlineto{\pgfqpoint{0.874199in}{0.614144in}}%
\pgfpathlineto{\pgfqpoint{0.877690in}{0.612605in}}%
\pgfpathlineto{\pgfqpoint{0.886999in}{0.610500in}}%
\pgfpathlineto{\pgfqpoint{0.892817in}{0.608152in}}%
\pgfpathlineto{\pgfqpoint{0.895145in}{0.607829in}}%
\pgfpathlineto{\pgfqpoint{0.897472in}{0.607343in}}%
\pgfpathlineto{\pgfqpoint{0.900963in}{0.606290in}}%
\pgfpathlineto{\pgfqpoint{0.902127in}{0.606047in}}%
\pgfpathlineto{\pgfqpoint{0.904454in}{0.605076in}}%
\pgfpathlineto{\pgfqpoint{0.907945in}{0.604590in}}%
\pgfpathlineto{\pgfqpoint{0.911436in}{0.603538in}}%
\pgfpathlineto{\pgfqpoint{0.916091in}{0.602404in}}%
\pgfpathlineto{\pgfqpoint{0.919582in}{0.601028in}}%
\pgfpathlineto{\pgfqpoint{0.921909in}{0.600704in}}%
\pgfpathlineto{\pgfqpoint{0.925400in}{0.599651in}}%
\pgfpathlineto{\pgfqpoint{0.926564in}{0.599651in}}%
\pgfpathlineto{\pgfqpoint{0.933546in}{0.597385in}}%
\pgfpathlineto{\pgfqpoint{0.937037in}{0.596737in}}%
\pgfpathlineto{\pgfqpoint{0.944018in}{0.595280in}}%
\pgfpathlineto{\pgfqpoint{0.945182in}{0.594632in}}%
\pgfpathlineto{\pgfqpoint{0.946346in}{0.594551in}}%
\pgfpathlineto{\pgfqpoint{0.948673in}{0.593822in}}%
\pgfpathlineto{\pgfqpoint{0.949837in}{0.593741in}}%
\pgfpathlineto{\pgfqpoint{0.952164in}{0.592932in}}%
\pgfpathlineto{\pgfqpoint{0.954491in}{0.592608in}}%
\pgfpathlineto{\pgfqpoint{0.956819in}{0.591636in}}%
\pgfpathlineto{\pgfqpoint{0.976601in}{0.587345in}}%
\pgfpathlineto{\pgfqpoint{0.977765in}{0.587021in}}%
\pgfpathlineto{\pgfqpoint{0.981256in}{0.586617in}}%
\pgfpathlineto{\pgfqpoint{0.982419in}{0.586293in}}%
\pgfpathlineto{\pgfqpoint{0.985910in}{0.585888in}}%
\pgfpathlineto{\pgfqpoint{0.988238in}{0.585240in}}%
\pgfpathlineto{\pgfqpoint{0.991729in}{0.584997in}}%
\pgfpathlineto{\pgfqpoint{0.992892in}{0.584674in}}%
\pgfpathlineto{\pgfqpoint{0.996383in}{0.584431in}}%
\pgfpathlineto{\pgfqpoint{0.998711in}{0.583945in}}%
\pgfpathlineto{\pgfqpoint{1.008020in}{0.582568in}}%
\pgfpathlineto{\pgfqpoint{1.012675in}{0.581273in}}%
\pgfpathlineto{\pgfqpoint{1.015002in}{0.580949in}}%
\pgfpathlineto{\pgfqpoint{1.026639in}{0.579249in}}%
\pgfpathlineto{\pgfqpoint{1.028966in}{0.578520in}}%
\pgfpathlineto{\pgfqpoint{1.031293in}{0.578035in}}%
\pgfpathlineto{\pgfqpoint{1.034784in}{0.577468in}}%
\pgfpathlineto{\pgfqpoint{1.035948in}{0.577468in}}%
\pgfpathlineto{\pgfqpoint{1.038275in}{0.577144in}}%
\pgfpathlineto{\pgfqpoint{1.041766in}{0.576739in}}%
\pgfpathlineto{\pgfqpoint{1.042930in}{0.576415in}}%
\pgfpathlineto{\pgfqpoint{1.045257in}{0.576091in}}%
\pgfpathlineto{\pgfqpoint{1.048748in}{0.575444in}}%
\pgfpathlineto{\pgfqpoint{1.051076in}{0.575120in}}%
\pgfpathlineto{\pgfqpoint{1.053403in}{0.574958in}}%
\pgfpathlineto{\pgfqpoint{1.055730in}{0.574391in}}%
\pgfpathlineto{\pgfqpoint{1.062712in}{0.573744in}}%
\pgfpathlineto{\pgfqpoint{1.068530in}{0.572691in}}%
\pgfpathlineto{\pgfqpoint{1.069694in}{0.572610in}}%
\pgfpathlineto{\pgfqpoint{1.070858in}{0.572286in}}%
\pgfpathlineto{\pgfqpoint{1.072021in}{0.572205in}}%
\pgfpathlineto{\pgfqpoint{1.074349in}{0.571719in}}%
\pgfpathlineto{\pgfqpoint{1.076676in}{0.571639in}}%
\pgfpathlineto{\pgfqpoint{1.077840in}{0.571153in}}%
\pgfpathlineto{\pgfqpoint{1.079003in}{0.571153in}}%
\pgfpathlineto{\pgfqpoint{1.081331in}{0.570829in}}%
\pgfpathlineto{\pgfqpoint{1.084822in}{0.570424in}}%
\pgfpathlineto{\pgfqpoint{1.089476in}{0.569857in}}%
\pgfpathlineto{\pgfqpoint{1.090640in}{0.569857in}}%
\pgfpathlineto{\pgfqpoint{1.091804in}{0.569452in}}%
\pgfpathlineto{\pgfqpoint{1.094131in}{0.569129in}}%
\pgfpathlineto{\pgfqpoint{1.096458in}{0.568562in}}%
\pgfpathlineto{\pgfqpoint{1.104604in}{0.567833in}}%
\pgfpathlineto{\pgfqpoint{1.106931in}{0.567671in}}%
\pgfpathlineto{\pgfqpoint{1.111586in}{0.567186in}}%
\pgfpathlineto{\pgfqpoint{1.116241in}{0.566700in}}%
\pgfpathlineto{\pgfqpoint{1.117404in}{0.566619in}}%
\pgfpathlineto{\pgfqpoint{1.119732in}{0.566133in}}%
\pgfpathlineto{\pgfqpoint{1.126714in}{0.565323in}}%
\pgfpathlineto{\pgfqpoint{1.129041in}{0.565243in}}%
\pgfpathlineto{\pgfqpoint{1.132532in}{0.564433in}}%
\pgfpathlineto{\pgfqpoint{1.137187in}{0.563785in}}%
\pgfpathlineto{\pgfqpoint{1.138350in}{0.563380in}}%
\pgfpathlineto{\pgfqpoint{1.145332in}{0.562733in}}%
\pgfpathlineto{\pgfqpoint{1.147660in}{0.562571in}}%
\pgfpathlineto{\pgfqpoint{1.148823in}{0.562571in}}%
\pgfpathlineto{\pgfqpoint{1.149987in}{0.562166in}}%
\pgfpathlineto{\pgfqpoint{1.154642in}{0.561680in}}%
\pgfpathlineto{\pgfqpoint{1.158133in}{0.561599in}}%
\pgfpathlineto{\pgfqpoint{1.161624in}{0.561032in}}%
\pgfpathlineto{\pgfqpoint{1.162787in}{0.560870in}}%
\pgfpathlineto{\pgfqpoint{1.163951in}{0.560466in}}%
\pgfpathlineto{\pgfqpoint{1.173260in}{0.559656in}}%
\pgfpathlineto{\pgfqpoint{1.175588in}{0.559089in}}%
\pgfpathlineto{\pgfqpoint{1.179079in}{0.558685in}}%
\pgfpathlineto{\pgfqpoint{1.180242in}{0.558442in}}%
\pgfpathlineto{\pgfqpoint{1.182570in}{0.558442in}}%
\pgfpathlineto{\pgfqpoint{1.183733in}{0.558118in}}%
\pgfpathlineto{\pgfqpoint{1.187224in}{0.557956in}}%
\pgfpathlineto{\pgfqpoint{1.191879in}{0.557551in}}%
\pgfpathlineto{\pgfqpoint{1.193043in}{0.557308in}}%
\pgfpathlineto{\pgfqpoint{1.194206in}{0.557308in}}%
\pgfpathlineto{\pgfqpoint{1.195370in}{0.557065in}}%
\pgfpathlineto{\pgfqpoint{1.198861in}{0.556903in}}%
\pgfpathlineto{\pgfqpoint{1.200024in}{0.556741in}}%
\pgfpathlineto{\pgfqpoint{1.201188in}{0.556337in}}%
\pgfpathlineto{\pgfqpoint{1.207006in}{0.556175in}}%
\pgfpathlineto{\pgfqpoint{1.218643in}{0.554717in}}%
\pgfpathlineto{\pgfqpoint{1.222134in}{0.554313in}}%
\pgfpathlineto{\pgfqpoint{1.224461in}{0.554070in}}%
\pgfpathlineto{\pgfqpoint{1.226789in}{0.554070in}}%
\pgfpathlineto{\pgfqpoint{1.229116in}{0.553665in}}%
\pgfpathlineto{\pgfqpoint{1.236098in}{0.553341in}}%
\pgfpathlineto{\pgfqpoint{1.239589in}{0.553260in}}%
\pgfpathlineto{\pgfqpoint{1.241916in}{0.552936in}}%
\pgfpathlineto{\pgfqpoint{1.247735in}{0.552208in}}%
\pgfpathlineto{\pgfqpoint{1.251226in}{0.552046in}}%
\pgfpathlineto{\pgfqpoint{1.260535in}{0.551317in}}%
\pgfpathlineto{\pgfqpoint{1.261699in}{0.550993in}}%
\pgfpathlineto{\pgfqpoint{1.262862in}{0.550993in}}%
\pgfpathlineto{\pgfqpoint{1.267517in}{0.550103in}}%
\pgfpathlineto{\pgfqpoint{1.269844in}{0.549941in}}%
\pgfpathlineto{\pgfqpoint{1.305918in}{0.547269in}}%
\pgfpathlineto{\pgfqpoint{1.308245in}{0.546864in}}%
\pgfpathlineto{\pgfqpoint{1.330355in}{0.545488in}}%
\pgfpathlineto{\pgfqpoint{1.337337in}{0.545326in}}%
\pgfpathlineto{\pgfqpoint{1.339664in}{0.545083in}}%
\pgfpathlineto{\pgfqpoint{1.343155in}{0.545083in}}%
\pgfpathlineto{\pgfqpoint{1.345482in}{0.544516in}}%
\pgfpathlineto{\pgfqpoint{1.350137in}{0.544030in}}%
\pgfpathlineto{\pgfqpoint{1.355955in}{0.543868in}}%
\pgfpathlineto{\pgfqpoint{1.358283in}{0.543707in}}%
\pgfpathlineto{\pgfqpoint{1.361774in}{0.543626in}}%
\pgfpathlineto{\pgfqpoint{1.364101in}{0.543545in}}%
\pgfpathlineto{\pgfqpoint{1.367592in}{0.543221in}}%
\pgfpathlineto{\pgfqpoint{1.372247in}{0.542735in}}%
\pgfpathlineto{\pgfqpoint{1.373410in}{0.542492in}}%
\pgfpathlineto{\pgfqpoint{1.375738in}{0.542411in}}%
\pgfpathlineto{\pgfqpoint{1.376901in}{0.542168in}}%
\pgfpathlineto{\pgfqpoint{1.381556in}{0.542087in}}%
\pgfpathlineto{\pgfqpoint{1.383883in}{0.541683in}}%
\pgfpathlineto{\pgfqpoint{1.388538in}{0.541521in}}%
\pgfpathlineto{\pgfqpoint{1.390865in}{0.541521in}}%
\pgfpathlineto{\pgfqpoint{1.392029in}{0.541278in}}%
\pgfpathlineto{\pgfqpoint{1.402502in}{0.541116in}}%
\pgfpathlineto{\pgfqpoint{1.405993in}{0.541035in}}%
\pgfpathlineto{\pgfqpoint{1.409484in}{0.540873in}}%
\pgfpathlineto{\pgfqpoint{1.414139in}{0.540468in}}%
\pgfpathlineto{\pgfqpoint{1.415302in}{0.540225in}}%
\pgfpathlineto{\pgfqpoint{1.418793in}{0.540063in}}%
\pgfpathlineto{\pgfqpoint{1.439739in}{0.538768in}}%
\pgfpathlineto{\pgfqpoint{1.445558in}{0.538768in}}%
\pgfpathlineto{\pgfqpoint{1.447885in}{0.538444in}}%
\pgfpathlineto{\pgfqpoint{1.451376in}{0.538120in}}%
\pgfpathlineto{\pgfqpoint{1.459521in}{0.537958in}}%
\pgfpathlineto{\pgfqpoint{1.480467in}{0.536663in}}%
\pgfpathlineto{\pgfqpoint{1.485122in}{0.536663in}}%
\pgfpathlineto{\pgfqpoint{1.487449in}{0.536096in}}%
\pgfpathlineto{\pgfqpoint{1.494431in}{0.536015in}}%
\pgfpathlineto{\pgfqpoint{1.496759in}{0.535772in}}%
\pgfpathlineto{\pgfqpoint{1.503741in}{0.535691in}}%
\pgfpathlineto{\pgfqpoint{1.506068in}{0.535529in}}%
\pgfpathlineto{\pgfqpoint{1.510723in}{0.535044in}}%
\pgfpathlineto{\pgfqpoint{1.511886in}{0.534801in}}%
\pgfpathlineto{\pgfqpoint{1.518868in}{0.534639in}}%
\pgfpathlineto{\pgfqpoint{1.523523in}{0.534477in}}%
\pgfpathlineto{\pgfqpoint{1.528178in}{0.534315in}}%
\pgfpathlineto{\pgfqpoint{1.532832in}{0.534153in}}%
\pgfpathlineto{\pgfqpoint{1.535160in}{0.534153in}}%
\pgfpathlineto{\pgfqpoint{1.536323in}{0.533910in}}%
\pgfpathlineto{\pgfqpoint{1.542142in}{0.533748in}}%
\pgfpathlineto{\pgfqpoint{1.544469in}{0.533586in}}%
\pgfpathlineto{\pgfqpoint{1.549124in}{0.533424in}}%
\pgfpathlineto{\pgfqpoint{1.552615in}{0.533424in}}%
\pgfpathlineto{\pgfqpoint{1.554942in}{0.533019in}}%
\pgfpathlineto{\pgfqpoint{1.566579in}{0.532372in}}%
\pgfpathlineto{\pgfqpoint{1.573561in}{0.532210in}}%
\pgfpathlineto{\pgfqpoint{1.574724in}{0.531967in}}%
\pgfpathlineto{\pgfqpoint{1.579379in}{0.531886in}}%
\pgfpathlineto{\pgfqpoint{1.582870in}{0.531562in}}%
\pgfpathlineto{\pgfqpoint{1.585197in}{0.531319in}}%
\pgfpathlineto{\pgfqpoint{1.586361in}{0.530915in}}%
\pgfpathlineto{\pgfqpoint{1.594506in}{0.530753in}}%
\pgfpathlineto{\pgfqpoint{1.601488in}{0.530429in}}%
\pgfpathlineto{\pgfqpoint{1.606143in}{0.530267in}}%
\pgfpathlineto{\pgfqpoint{1.609634in}{0.529943in}}%
\pgfpathlineto{\pgfqpoint{1.614289in}{0.529781in}}%
\pgfpathlineto{\pgfqpoint{1.616616in}{0.529457in}}%
\pgfpathlineto{\pgfqpoint{1.620107in}{0.529376in}}%
\pgfpathlineto{\pgfqpoint{1.622434in}{0.529214in}}%
\pgfpathlineto{\pgfqpoint{1.635235in}{0.528486in}}%
\pgfpathlineto{\pgfqpoint{1.637562in}{0.528324in}}%
\pgfpathlineto{\pgfqpoint{1.644544in}{0.528000in}}%
\pgfpathlineto{\pgfqpoint{1.651526in}{0.527838in}}%
\pgfpathlineto{\pgfqpoint{1.655017in}{0.527514in}}%
\pgfpathlineto{\pgfqpoint{1.665490in}{0.526947in}}%
\pgfpathlineto{\pgfqpoint{1.678290in}{0.526785in}}%
\pgfpathlineto{\pgfqpoint{1.679454in}{0.526543in}}%
\pgfpathlineto{\pgfqpoint{1.682945in}{0.526381in}}%
\pgfpathlineto{\pgfqpoint{1.695745in}{0.525652in}}%
\pgfpathlineto{\pgfqpoint{1.698073in}{0.525490in}}%
\pgfpathlineto{\pgfqpoint{1.701564in}{0.525409in}}%
\pgfpathlineto{\pgfqpoint{1.709709in}{0.525247in}}%
\pgfpathlineto{\pgfqpoint{1.717855in}{0.524842in}}%
\pgfpathlineto{\pgfqpoint{1.727164in}{0.524680in}}%
\pgfpathlineto{\pgfqpoint{1.738801in}{0.524033in}}%
\pgfpathlineto{\pgfqpoint{1.739964in}{0.524033in}}%
\pgfpathlineto{\pgfqpoint{1.739964in}{0.524033in}}%
\pgfusepath{stroke}%
\end{pgfscope}%
\begin{pgfscope}%
\pgfpathrectangle{\pgfqpoint{0.519339in}{0.467408in}}{\pgfqpoint{1.278750in}{1.245750in}}%
\pgfusepath{clip}%
\pgfsetroundcap%
\pgfsetroundjoin%
\pgfsetlinewidth{1.003750pt}%
\definecolor{currentstroke}{rgb}{0.866667,0.517647,0.321569}%
\pgfsetstrokecolor{currentstroke}%
\pgfsetdash{}{0pt}%
\pgfpathmoveto{\pgfqpoint{0.577464in}{1.641150in}}%
\pgfpathlineto{\pgfqpoint{0.578628in}{1.608198in}}%
\pgfpathlineto{\pgfqpoint{0.590265in}{1.218608in}}%
\pgfpathlineto{\pgfqpoint{0.592592in}{1.155214in}}%
\pgfpathlineto{\pgfqpoint{0.594919in}{1.097893in}}%
\pgfpathlineto{\pgfqpoint{0.596083in}{1.075709in}}%
\pgfpathlineto{\pgfqpoint{0.599574in}{1.023732in}}%
\pgfpathlineto{\pgfqpoint{0.601901in}{0.994343in}}%
\pgfpathlineto{\pgfqpoint{0.606556in}{0.946089in}}%
\pgfpathlineto{\pgfqpoint{0.612374in}{0.891682in}}%
\pgfpathlineto{\pgfqpoint{0.615865in}{0.868122in}}%
\pgfpathlineto{\pgfqpoint{0.618193in}{0.852901in}}%
\pgfpathlineto{\pgfqpoint{0.620520in}{0.839381in}}%
\pgfpathlineto{\pgfqpoint{0.624011in}{0.820921in}}%
\pgfpathlineto{\pgfqpoint{0.629829in}{0.795823in}}%
\pgfpathlineto{\pgfqpoint{0.630993in}{0.789508in}}%
\pgfpathlineto{\pgfqpoint{0.632157in}{0.784326in}}%
\pgfpathlineto{\pgfqpoint{0.633320in}{0.780035in}}%
\pgfpathlineto{\pgfqpoint{0.634484in}{0.776635in}}%
\pgfpathlineto{\pgfqpoint{0.635648in}{0.772263in}}%
\pgfpathlineto{\pgfqpoint{0.637975in}{0.761333in}}%
\pgfpathlineto{\pgfqpoint{0.640302in}{0.753885in}}%
\pgfpathlineto{\pgfqpoint{0.641466in}{0.751375in}}%
\pgfpathlineto{\pgfqpoint{0.642630in}{0.747731in}}%
\pgfpathlineto{\pgfqpoint{0.643793in}{0.743440in}}%
\pgfpathlineto{\pgfqpoint{0.651939in}{0.721985in}}%
\pgfpathlineto{\pgfqpoint{0.653103in}{0.720123in}}%
\pgfpathlineto{\pgfqpoint{0.655430in}{0.715104in}}%
\pgfpathlineto{\pgfqpoint{0.657757in}{0.710165in}}%
\pgfpathlineto{\pgfqpoint{0.658921in}{0.707493in}}%
\pgfpathlineto{\pgfqpoint{0.661248in}{0.703283in}}%
\pgfpathlineto{\pgfqpoint{0.664739in}{0.696644in}}%
\pgfpathlineto{\pgfqpoint{0.665903in}{0.695025in}}%
\pgfpathlineto{\pgfqpoint{0.667067in}{0.691948in}}%
\pgfpathlineto{\pgfqpoint{0.668230in}{0.690329in}}%
\pgfpathlineto{\pgfqpoint{0.669394in}{0.689439in}}%
\pgfpathlineto{\pgfqpoint{0.670558in}{0.687253in}}%
\pgfpathlineto{\pgfqpoint{0.671721in}{0.684662in}}%
\pgfpathlineto{\pgfqpoint{0.672885in}{0.682800in}}%
\pgfpathlineto{\pgfqpoint{0.674049in}{0.681504in}}%
\pgfpathlineto{\pgfqpoint{0.675212in}{0.679804in}}%
\pgfpathlineto{\pgfqpoint{0.676376in}{0.678671in}}%
\pgfpathlineto{\pgfqpoint{0.677540in}{0.677132in}}%
\pgfpathlineto{\pgfqpoint{0.678703in}{0.676080in}}%
\pgfpathlineto{\pgfqpoint{0.679867in}{0.674380in}}%
\pgfpathlineto{\pgfqpoint{0.682194in}{0.671789in}}%
\pgfpathlineto{\pgfqpoint{0.683358in}{0.669684in}}%
\pgfpathlineto{\pgfqpoint{0.684521in}{0.668146in}}%
\pgfpathlineto{\pgfqpoint{0.688012in}{0.664664in}}%
\pgfpathlineto{\pgfqpoint{0.689176in}{0.663935in}}%
\pgfpathlineto{\pgfqpoint{0.690340in}{0.662883in}}%
\pgfpathlineto{\pgfqpoint{0.691503in}{0.661021in}}%
\pgfpathlineto{\pgfqpoint{0.692667in}{0.659968in}}%
\pgfpathlineto{\pgfqpoint{0.693831in}{0.659240in}}%
\pgfpathlineto{\pgfqpoint{0.694994in}{0.658268in}}%
\pgfpathlineto{\pgfqpoint{0.696158in}{0.656730in}}%
\pgfpathlineto{\pgfqpoint{0.697322in}{0.656082in}}%
\pgfpathlineto{\pgfqpoint{0.698485in}{0.654949in}}%
\pgfpathlineto{\pgfqpoint{0.699649in}{0.653491in}}%
\pgfpathlineto{\pgfqpoint{0.701976in}{0.651710in}}%
\pgfpathlineto{\pgfqpoint{0.704304in}{0.648957in}}%
\pgfpathlineto{\pgfqpoint{0.705467in}{0.648472in}}%
\pgfpathlineto{\pgfqpoint{0.706631in}{0.647581in}}%
\pgfpathlineto{\pgfqpoint{0.707795in}{0.647014in}}%
\pgfpathlineto{\pgfqpoint{0.712449in}{0.642076in}}%
\pgfpathlineto{\pgfqpoint{0.713613in}{0.641266in}}%
\pgfpathlineto{\pgfqpoint{0.714777in}{0.640699in}}%
\pgfpathlineto{\pgfqpoint{0.719431in}{0.637461in}}%
\pgfpathlineto{\pgfqpoint{0.721759in}{0.636327in}}%
\pgfpathlineto{\pgfqpoint{0.722922in}{0.635518in}}%
\pgfpathlineto{\pgfqpoint{0.724086in}{0.634465in}}%
\pgfpathlineto{\pgfqpoint{0.725250in}{0.634141in}}%
\pgfpathlineto{\pgfqpoint{0.734559in}{0.629527in}}%
\pgfpathlineto{\pgfqpoint{0.735723in}{0.628636in}}%
\pgfpathlineto{\pgfqpoint{0.736886in}{0.627988in}}%
\pgfpathlineto{\pgfqpoint{0.738050in}{0.627745in}}%
\pgfpathlineto{\pgfqpoint{0.739214in}{0.627098in}}%
\pgfpathlineto{\pgfqpoint{0.741541in}{0.626450in}}%
\pgfpathlineto{\pgfqpoint{0.743868in}{0.625316in}}%
\pgfpathlineto{\pgfqpoint{0.745032in}{0.624993in}}%
\pgfpathlineto{\pgfqpoint{0.746196in}{0.624183in}}%
\pgfpathlineto{\pgfqpoint{0.747359in}{0.623778in}}%
\pgfpathlineto{\pgfqpoint{0.750850in}{0.621673in}}%
\pgfpathlineto{\pgfqpoint{0.752014in}{0.621430in}}%
\pgfpathlineto{\pgfqpoint{0.753178in}{0.620945in}}%
\pgfpathlineto{\pgfqpoint{0.754341in}{0.620216in}}%
\pgfpathlineto{\pgfqpoint{0.755505in}{0.619730in}}%
\pgfpathlineto{\pgfqpoint{0.757832in}{0.619244in}}%
\pgfpathlineto{\pgfqpoint{0.758996in}{0.618516in}}%
\pgfpathlineto{\pgfqpoint{0.760160in}{0.617544in}}%
\pgfpathlineto{\pgfqpoint{0.762487in}{0.617058in}}%
\pgfpathlineto{\pgfqpoint{0.765978in}{0.615520in}}%
\pgfpathlineto{\pgfqpoint{0.768305in}{0.614629in}}%
\pgfpathlineto{\pgfqpoint{0.771796in}{0.612363in}}%
\pgfpathlineto{\pgfqpoint{0.776451in}{0.610177in}}%
\pgfpathlineto{\pgfqpoint{0.777615in}{0.609853in}}%
\pgfpathlineto{\pgfqpoint{0.781106in}{0.608233in}}%
\pgfpathlineto{\pgfqpoint{0.782269in}{0.608072in}}%
\pgfpathlineto{\pgfqpoint{0.788088in}{0.605400in}}%
\pgfpathlineto{\pgfqpoint{0.789251in}{0.605076in}}%
\pgfpathlineto{\pgfqpoint{0.795070in}{0.602323in}}%
\pgfpathlineto{\pgfqpoint{0.796233in}{0.601433in}}%
\pgfpathlineto{\pgfqpoint{0.797397in}{0.600947in}}%
\pgfpathlineto{\pgfqpoint{0.798561in}{0.600866in}}%
\pgfpathlineto{\pgfqpoint{0.802052in}{0.599813in}}%
\pgfpathlineto{\pgfqpoint{0.807870in}{0.599085in}}%
\pgfpathlineto{\pgfqpoint{0.809034in}{0.598599in}}%
\pgfpathlineto{\pgfqpoint{0.814852in}{0.597304in}}%
\pgfpathlineto{\pgfqpoint{0.820670in}{0.595441in}}%
\pgfpathlineto{\pgfqpoint{0.826488in}{0.594146in}}%
\pgfpathlineto{\pgfqpoint{0.827652in}{0.593417in}}%
\pgfpathlineto{\pgfqpoint{0.838125in}{0.590665in}}%
\pgfpathlineto{\pgfqpoint{0.840452in}{0.590179in}}%
\pgfpathlineto{\pgfqpoint{0.841616in}{0.589936in}}%
\pgfpathlineto{\pgfqpoint{0.843943in}{0.589126in}}%
\pgfpathlineto{\pgfqpoint{0.852089in}{0.587021in}}%
\pgfpathlineto{\pgfqpoint{0.853253in}{0.586859in}}%
\pgfpathlineto{\pgfqpoint{0.854416in}{0.586293in}}%
\pgfpathlineto{\pgfqpoint{0.856744in}{0.585807in}}%
\pgfpathlineto{\pgfqpoint{0.859071in}{0.585321in}}%
\pgfpathlineto{\pgfqpoint{0.860235in}{0.584835in}}%
\pgfpathlineto{\pgfqpoint{0.862562in}{0.584512in}}%
\pgfpathlineto{\pgfqpoint{0.863726in}{0.584350in}}%
\pgfpathlineto{\pgfqpoint{0.869544in}{0.582407in}}%
\pgfpathlineto{\pgfqpoint{0.871871in}{0.582083in}}%
\pgfpathlineto{\pgfqpoint{0.883508in}{0.579573in}}%
\pgfpathlineto{\pgfqpoint{0.884672in}{0.579168in}}%
\pgfpathlineto{\pgfqpoint{0.889326in}{0.578682in}}%
\pgfpathlineto{\pgfqpoint{0.890490in}{0.578601in}}%
\pgfpathlineto{\pgfqpoint{0.891654in}{0.578196in}}%
\pgfpathlineto{\pgfqpoint{0.892817in}{0.578196in}}%
\pgfpathlineto{\pgfqpoint{0.895145in}{0.577549in}}%
\pgfpathlineto{\pgfqpoint{0.896308in}{0.577306in}}%
\pgfpathlineto{\pgfqpoint{0.898636in}{0.576334in}}%
\pgfpathlineto{\pgfqpoint{0.900963in}{0.575849in}}%
\pgfpathlineto{\pgfqpoint{0.907945in}{0.573501in}}%
\pgfpathlineto{\pgfqpoint{0.914927in}{0.572205in}}%
\pgfpathlineto{\pgfqpoint{0.917254in}{0.572043in}}%
\pgfpathlineto{\pgfqpoint{0.924236in}{0.571153in}}%
\pgfpathlineto{\pgfqpoint{0.926564in}{0.570991in}}%
\pgfpathlineto{\pgfqpoint{0.928891in}{0.570667in}}%
\pgfpathlineto{\pgfqpoint{0.938200in}{0.569291in}}%
\pgfpathlineto{\pgfqpoint{0.939364in}{0.568967in}}%
\pgfpathlineto{\pgfqpoint{0.941691in}{0.568724in}}%
\pgfpathlineto{\pgfqpoint{0.953328in}{0.566214in}}%
\pgfpathlineto{\pgfqpoint{0.954491in}{0.565809in}}%
\pgfpathlineto{\pgfqpoint{0.959146in}{0.565000in}}%
\pgfpathlineto{\pgfqpoint{0.962637in}{0.564109in}}%
\pgfpathlineto{\pgfqpoint{0.964964in}{0.563623in}}%
\pgfpathlineto{\pgfqpoint{0.966128in}{0.563380in}}%
\pgfpathlineto{\pgfqpoint{0.968455in}{0.563299in}}%
\pgfpathlineto{\pgfqpoint{0.984747in}{0.561194in}}%
\pgfpathlineto{\pgfqpoint{0.985910in}{0.561113in}}%
\pgfpathlineto{\pgfqpoint{0.988238in}{0.560547in}}%
\pgfpathlineto{\pgfqpoint{0.996383in}{0.559494in}}%
\pgfpathlineto{\pgfqpoint{0.998711in}{0.559413in}}%
\pgfpathlineto{\pgfqpoint{1.001038in}{0.558766in}}%
\pgfpathlineto{\pgfqpoint{1.003365in}{0.558361in}}%
\pgfpathlineto{\pgfqpoint{1.005693in}{0.557794in}}%
\pgfpathlineto{\pgfqpoint{1.008020in}{0.557551in}}%
\pgfpathlineto{\pgfqpoint{1.009184in}{0.557227in}}%
\pgfpathlineto{\pgfqpoint{1.011511in}{0.556903in}}%
\pgfpathlineto{\pgfqpoint{1.016166in}{0.556418in}}%
\pgfpathlineto{\pgfqpoint{1.021984in}{0.555932in}}%
\pgfpathlineto{\pgfqpoint{1.024311in}{0.555527in}}%
\pgfpathlineto{\pgfqpoint{1.027802in}{0.555203in}}%
\pgfpathlineto{\pgfqpoint{1.030130in}{0.554798in}}%
\pgfpathlineto{\pgfqpoint{1.031293in}{0.554717in}}%
\pgfpathlineto{\pgfqpoint{1.032457in}{0.554394in}}%
\pgfpathlineto{\pgfqpoint{1.033621in}{0.554313in}}%
\pgfpathlineto{\pgfqpoint{1.034784in}{0.553989in}}%
\pgfpathlineto{\pgfqpoint{1.038275in}{0.553584in}}%
\pgfpathlineto{\pgfqpoint{1.042930in}{0.552774in}}%
\pgfpathlineto{\pgfqpoint{1.045257in}{0.552370in}}%
\pgfpathlineto{\pgfqpoint{1.066203in}{0.550750in}}%
\pgfpathlineto{\pgfqpoint{1.068530in}{0.550588in}}%
\pgfpathlineto{\pgfqpoint{1.073185in}{0.550103in}}%
\pgfpathlineto{\pgfqpoint{1.076676in}{0.549536in}}%
\pgfpathlineto{\pgfqpoint{1.077840in}{0.549536in}}%
\pgfpathlineto{\pgfqpoint{1.079003in}{0.549131in}}%
\pgfpathlineto{\pgfqpoint{1.091804in}{0.547512in}}%
\pgfpathlineto{\pgfqpoint{1.101113in}{0.546702in}}%
\pgfpathlineto{\pgfqpoint{1.105768in}{0.546216in}}%
\pgfpathlineto{\pgfqpoint{1.109259in}{0.545812in}}%
\pgfpathlineto{\pgfqpoint{1.113913in}{0.544921in}}%
\pgfpathlineto{\pgfqpoint{1.117404in}{0.544759in}}%
\pgfpathlineto{\pgfqpoint{1.119732in}{0.544435in}}%
\pgfpathlineto{\pgfqpoint{1.125550in}{0.543788in}}%
\pgfpathlineto{\pgfqpoint{1.126714in}{0.543788in}}%
\pgfpathlineto{\pgfqpoint{1.127877in}{0.543464in}}%
\pgfpathlineto{\pgfqpoint{1.130205in}{0.543302in}}%
\pgfpathlineto{\pgfqpoint{1.132532in}{0.542816in}}%
\pgfpathlineto{\pgfqpoint{1.134859in}{0.542735in}}%
\pgfpathlineto{\pgfqpoint{1.143005in}{0.541683in}}%
\pgfpathlineto{\pgfqpoint{1.147660in}{0.541521in}}%
\pgfpathlineto{\pgfqpoint{1.153478in}{0.540873in}}%
\pgfpathlineto{\pgfqpoint{1.158133in}{0.540630in}}%
\pgfpathlineto{\pgfqpoint{1.158133in}{0.540630in}}%
\pgfusepath{stroke}%
\end{pgfscope}%
\begin{pgfscope}%
\pgfpathrectangle{\pgfqpoint{0.519339in}{0.467408in}}{\pgfqpoint{1.278750in}{1.245750in}}%
\pgfusepath{clip}%
\pgfsetroundcap%
\pgfsetroundjoin%
\pgfsetlinewidth{1.003750pt}%
\definecolor{currentstroke}{rgb}{0.333333,0.658824,0.407843}%
\pgfsetstrokecolor{currentstroke}%
\pgfsetdash{}{0pt}%
\pgfpathmoveto{\pgfqpoint{0.577464in}{1.556139in}}%
\pgfpathlineto{\pgfqpoint{0.578628in}{1.468781in}}%
\pgfpathlineto{\pgfqpoint{0.579792in}{1.381747in}}%
\pgfpathlineto{\pgfqpoint{0.580955in}{1.298680in}}%
\pgfpathlineto{\pgfqpoint{0.582119in}{1.219337in}}%
\pgfpathlineto{\pgfqpoint{0.583283in}{1.146552in}}%
\pgfpathlineto{\pgfqpoint{0.584446in}{1.094655in}}%
\pgfpathlineto{\pgfqpoint{0.585610in}{1.049640in}}%
\pgfpathlineto{\pgfqpoint{0.586774in}{1.008754in}}%
\pgfpathlineto{\pgfqpoint{0.587937in}{0.975559in}}%
\pgfpathlineto{\pgfqpoint{0.589101in}{0.947708in}}%
\pgfpathlineto{\pgfqpoint{0.590265in}{0.920181in}}%
\pgfpathlineto{\pgfqpoint{0.591428in}{0.895164in}}%
\pgfpathlineto{\pgfqpoint{0.592592in}{0.873385in}}%
\pgfpathlineto{\pgfqpoint{0.593756in}{0.854683in}}%
\pgfpathlineto{\pgfqpoint{0.594919in}{0.838085in}}%
\pgfpathlineto{\pgfqpoint{0.596083in}{0.820436in}}%
\pgfpathlineto{\pgfqpoint{0.597247in}{0.809101in}}%
\pgfpathlineto{\pgfqpoint{0.598410in}{0.796228in}}%
\pgfpathlineto{\pgfqpoint{0.599574in}{0.784731in}}%
\pgfpathlineto{\pgfqpoint{0.600738in}{0.773396in}}%
\pgfpathlineto{\pgfqpoint{0.601901in}{0.763438in}}%
\pgfpathlineto{\pgfqpoint{0.603065in}{0.753966in}}%
\pgfpathlineto{\pgfqpoint{0.604229in}{0.746841in}}%
\pgfpathlineto{\pgfqpoint{0.605392in}{0.737692in}}%
\pgfpathlineto{\pgfqpoint{0.606556in}{0.730406in}}%
\pgfpathlineto{\pgfqpoint{0.607720in}{0.722714in}}%
\pgfpathlineto{\pgfqpoint{0.608883in}{0.715751in}}%
\pgfpathlineto{\pgfqpoint{0.610047in}{0.709517in}}%
\pgfpathlineto{\pgfqpoint{0.611211in}{0.703850in}}%
\pgfpathlineto{\pgfqpoint{0.612374in}{0.697697in}}%
\pgfpathlineto{\pgfqpoint{0.613538in}{0.693163in}}%
\pgfpathlineto{\pgfqpoint{0.614702in}{0.688953in}}%
\pgfpathlineto{\pgfqpoint{0.615865in}{0.684905in}}%
\pgfpathlineto{\pgfqpoint{0.617029in}{0.681180in}}%
\pgfpathlineto{\pgfqpoint{0.618193in}{0.677618in}}%
\pgfpathlineto{\pgfqpoint{0.619356in}{0.674461in}}%
\pgfpathlineto{\pgfqpoint{0.620520in}{0.671222in}}%
\pgfpathlineto{\pgfqpoint{0.621684in}{0.667984in}}%
\pgfpathlineto{\pgfqpoint{0.622847in}{0.665069in}}%
\pgfpathlineto{\pgfqpoint{0.624011in}{0.661345in}}%
\pgfpathlineto{\pgfqpoint{0.625175in}{0.658106in}}%
\pgfpathlineto{\pgfqpoint{0.626338in}{0.655515in}}%
\pgfpathlineto{\pgfqpoint{0.627502in}{0.652358in}}%
\pgfpathlineto{\pgfqpoint{0.628666in}{0.650091in}}%
\pgfpathlineto{\pgfqpoint{0.629829in}{0.647338in}}%
\pgfpathlineto{\pgfqpoint{0.630993in}{0.644990in}}%
\pgfpathlineto{\pgfqpoint{0.632157in}{0.643290in}}%
\pgfpathlineto{\pgfqpoint{0.633320in}{0.641671in}}%
\pgfpathlineto{\pgfqpoint{0.634484in}{0.639647in}}%
\pgfpathlineto{\pgfqpoint{0.635648in}{0.638351in}}%
\pgfpathlineto{\pgfqpoint{0.636811in}{0.636408in}}%
\pgfpathlineto{\pgfqpoint{0.637975in}{0.633898in}}%
\pgfpathlineto{\pgfqpoint{0.639139in}{0.632360in}}%
\pgfpathlineto{\pgfqpoint{0.640302in}{0.631065in}}%
\pgfpathlineto{\pgfqpoint{0.641466in}{0.629446in}}%
\pgfpathlineto{\pgfqpoint{0.642630in}{0.627341in}}%
\pgfpathlineto{\pgfqpoint{0.643793in}{0.625559in}}%
\pgfpathlineto{\pgfqpoint{0.644957in}{0.624669in}}%
\pgfpathlineto{\pgfqpoint{0.646121in}{0.623535in}}%
\pgfpathlineto{\pgfqpoint{0.647284in}{0.622159in}}%
\pgfpathlineto{\pgfqpoint{0.648448in}{0.621430in}}%
\pgfpathlineto{\pgfqpoint{0.649612in}{0.619406in}}%
\pgfpathlineto{\pgfqpoint{0.650775in}{0.618192in}}%
\pgfpathlineto{\pgfqpoint{0.651939in}{0.616816in}}%
\pgfpathlineto{\pgfqpoint{0.653103in}{0.615520in}}%
\pgfpathlineto{\pgfqpoint{0.654266in}{0.613982in}}%
\pgfpathlineto{\pgfqpoint{0.655430in}{0.612767in}}%
\pgfpathlineto{\pgfqpoint{0.656594in}{0.611391in}}%
\pgfpathlineto{\pgfqpoint{0.657757in}{0.610096in}}%
\pgfpathlineto{\pgfqpoint{0.658921in}{0.609286in}}%
\pgfpathlineto{\pgfqpoint{0.660085in}{0.608395in}}%
\pgfpathlineto{\pgfqpoint{0.661248in}{0.607991in}}%
\pgfpathlineto{\pgfqpoint{0.662412in}{0.606533in}}%
\pgfpathlineto{\pgfqpoint{0.663576in}{0.605400in}}%
\pgfpathlineto{\pgfqpoint{0.664739in}{0.604995in}}%
\pgfpathlineto{\pgfqpoint{0.665903in}{0.604266in}}%
\pgfpathlineto{\pgfqpoint{0.667067in}{0.603376in}}%
\pgfpathlineto{\pgfqpoint{0.668230in}{0.602323in}}%
\pgfpathlineto{\pgfqpoint{0.669394in}{0.601837in}}%
\pgfpathlineto{\pgfqpoint{0.670558in}{0.601190in}}%
\pgfpathlineto{\pgfqpoint{0.671721in}{0.600785in}}%
\pgfpathlineto{\pgfqpoint{0.672885in}{0.599732in}}%
\pgfpathlineto{\pgfqpoint{0.674049in}{0.599085in}}%
\pgfpathlineto{\pgfqpoint{0.675212in}{0.598032in}}%
\pgfpathlineto{\pgfqpoint{0.676376in}{0.596899in}}%
\pgfpathlineto{\pgfqpoint{0.677540in}{0.596089in}}%
\pgfpathlineto{\pgfqpoint{0.678703in}{0.595360in}}%
\pgfpathlineto{\pgfqpoint{0.679867in}{0.594956in}}%
\pgfpathlineto{\pgfqpoint{0.681031in}{0.593984in}}%
\pgfpathlineto{\pgfqpoint{0.682194in}{0.593579in}}%
\pgfpathlineto{\pgfqpoint{0.683358in}{0.592608in}}%
\pgfpathlineto{\pgfqpoint{0.684521in}{0.591879in}}%
\pgfpathlineto{\pgfqpoint{0.685685in}{0.591636in}}%
\pgfpathlineto{\pgfqpoint{0.686849in}{0.590989in}}%
\pgfpathlineto{\pgfqpoint{0.688012in}{0.590098in}}%
\pgfpathlineto{\pgfqpoint{0.689176in}{0.589450in}}%
\pgfpathlineto{\pgfqpoint{0.690340in}{0.588883in}}%
\pgfpathlineto{\pgfqpoint{0.691503in}{0.588479in}}%
\pgfpathlineto{\pgfqpoint{0.692667in}{0.587750in}}%
\pgfusepath{stroke}%
\end{pgfscope}%
\begin{pgfscope}%
\pgfsetrectcap%
\pgfsetmiterjoin%
\pgfsetlinewidth{0.752812pt}%
\definecolor{currentstroke}{rgb}{0.700000,0.700000,0.700000}%
\pgfsetstrokecolor{currentstroke}%
\pgfsetdash{}{0pt}%
\pgfpathmoveto{\pgfqpoint{0.519339in}{0.467408in}}%
\pgfpathlineto{\pgfqpoint{0.519339in}{1.713158in}}%
\pgfusepath{stroke}%
\end{pgfscope}%
\begin{pgfscope}%
\pgfsetrectcap%
\pgfsetmiterjoin%
\pgfsetlinewidth{0.752812pt}%
\definecolor{currentstroke}{rgb}{0.700000,0.700000,0.700000}%
\pgfsetstrokecolor{currentstroke}%
\pgfsetdash{}{0pt}%
\pgfpathmoveto{\pgfqpoint{1.798089in}{0.467408in}}%
\pgfpathlineto{\pgfqpoint{1.798089in}{1.713158in}}%
\pgfusepath{stroke}%
\end{pgfscope}%
\begin{pgfscope}%
\pgfsetrectcap%
\pgfsetmiterjoin%
\pgfsetlinewidth{0.752812pt}%
\definecolor{currentstroke}{rgb}{0.700000,0.700000,0.700000}%
\pgfsetstrokecolor{currentstroke}%
\pgfsetdash{}{0pt}%
\pgfpathmoveto{\pgfqpoint{0.519339in}{0.467408in}}%
\pgfpathlineto{\pgfqpoint{1.798089in}{0.467408in}}%
\pgfusepath{stroke}%
\end{pgfscope}%
\begin{pgfscope}%
\pgfsetrectcap%
\pgfsetmiterjoin%
\pgfsetlinewidth{0.752812pt}%
\definecolor{currentstroke}{rgb}{0.700000,0.700000,0.700000}%
\pgfsetstrokecolor{currentstroke}%
\pgfsetdash{}{0pt}%
\pgfpathmoveto{\pgfqpoint{0.519339in}{1.713158in}}%
\pgfpathlineto{\pgfqpoint{1.798089in}{1.713158in}}%
\pgfusepath{stroke}%
\end{pgfscope}%
\end{pgfpicture}%
\makeatother%
\endgroup%
}}  \\
    \end{array}\)
  }
  \caption{The perturbed loss and perturbed accuracy over the epochs \(k\) on the arXiv dataset (see Tab.~\ref{tab:datasets}) using PR-BCD for different block sizes \(b\) with loss (5) of Sec.~\ref{sec:ceisbad}. The number of epochs \(k\) is chosen such that \(k b = \text{const.}\). Hence, approximately the same amount of edges has been ``visited''. At \(k=100\) the accuracy varies in the range of around 5\% despite the very different choices of \(b\). Furthermore, it seems like that the attacks that ran more epochs are slightly stronger.\label{fig:arxivrandomblocksizeinfluence}}
\end{figure}

\begin{algorithm}[h]
  \small
  \caption{Projected and Randomized Block Coordinate Descent (PR-BCD)}
  \label{algo:prbcd}
  \begin{algorithmic}[1]
    \STATE {\bfseries Input:} Adj.\ \(\adj\), feat.\ \(\features\), labels\ \(\vy\), GNN \(f_{\theta}(\adj, \features)\), loss \(\mathcal{L}\)
    \STATE {\bfseries Parameter:} budget \(\Delta\), block size \(b\), epochs \(K\), heur.\ \(h(\dots)\)
    \STATE Draw random indices \(\vi_0 \in \{0, 1, \dots, N\}^b\)
    \STATE Initialize zeros for \(\vp_0 \in \R^b\)
    \FOR{\(k \in \{1,2, \dots, K\}\)}
    \STATE \(\hat{\vy} \leftarrow f_{\theta}(\adj \oplus \vp_{k-1}, \features)\)
    \STATE \(\vp_{k} \leftarrow \vp_{k-1} + \alpha_{k-1} \nabla_{\vi_{k-1}} \mathcal{L}(\hat{\vy}, \vy)\)
    %\STATE \(\vp_{k} \leftarrow \vp_{k-1} + \alpha_{k-1} \nabla_{\vi_{k-1}} \mathcal{L}(\vp_{k-1})\)
    \STATE Projection \(\vp_{k} \leftarrow \Pi_{\E[\text{Bernoulli}(\vp_k)] = \Delta} (\vp_{k})\)
    \STATE \(\vi_{k} \leftarrow \vi_{k-1}\)
    \IF{\(k \le K_{\text{resample}}\)}
    \STATE \(\text{mask}_{\text{resample}} \leftarrow h(\vp_{k})\)
    \STATE \(\vp_k[\text{mask}_{\text{resample}}] \leftarrow \mathbf{0}\)
    \STATE Resample \(\vi_{k}[\text{mask}_{\text{resample}}] \in \{0, 1, \dots, N\}^{|\text{mask}_{\text{resample}}|}\)
    \ENDIF
    \ENDFOR
    \STATE \(\mP \sim \text{Bernoulli}(\vp_{k})\) s.t.\ \(\sum \mP \le \Delta\)
    \STATE Return \(\adj \oplus \mP\)
    %\STATE Return \(\tilde{\sA}\) via sampling the edge add / rem.~w.r.t.~\(\text{Bernoulli}(\vp_{K})\) 
  \end{algorithmic}
\end{algorithm}

We also compare this approach to a Greedy R-BCD (GR-BCD), that greedily flips the entries with largest gradient in the random search space, such that after \(K\) iterations the budget requirements are met.

\todo{Local version}
%If we performed a targeted instead of a global attack, we could further reduce the space requirements since we only need to consider the node's ``receptive field''. Such an attack could be very similar to adding one node in the introduced GANG attack with further constraints (see Section~\ref{sec:attackkdd}).

\subsection{Adding Adversarial Nodes}\label{sec:attackkdd}

\begin{figure}[t]
  \centering
  \hbox{\hspace{45pt} \resizebox{0.7\linewidth}{!}{%% Creator: Matplotlib, PGF backend
%%
%% To include the figure in your LaTeX document, write
%%   \input{<filename>.pgf}
%%
%% Make sure the required packages are loaded in your preamble
%%   \usepackage{pgf}
%%
%% and, on pdftex
%%   \usepackage[utf8]{inputenc}\DeclareUnicodeCharacter{2212}{-}
%%
%% or, on luatex and xetex
%%   \usepackage{unicode-math}
%%
%% Figures using additional raster images can only be included by \input if
%% they are in the same directory as the main LaTeX file. For loading figures
%% from other directories you can use the `import` package
%%   \usepackage{import}
%%
%% and then include the figures with
%%   \import{<path to file>}{<filename>.pgf}
%%
%% Matplotlib used the following preamble
%%   \usepackage[utf8]{inputenc}
%%   \usepackage[T1]{fontenc}
%%   \usepackage{amsmath}
%%   \newcommand*{\mat}[1]{\boldsymbol{#1}}
%%
\begingroup%
\makeatletter%
\begin{pgfpicture}%
\pgfpathrectangle{\pgfpointorigin}{\pgfqpoint{3.196757in}{0.554906in}}%
\pgfusepath{use as bounding box, clip}%
\begin{pgfscope}%
\pgfsetbuttcap%
\pgfsetmiterjoin%
\definecolor{currentfill}{rgb}{1.000000,1.000000,1.000000}%
\pgfsetfillcolor{currentfill}%
\pgfsetlinewidth{0.000000pt}%
\definecolor{currentstroke}{rgb}{1.000000,1.000000,1.000000}%
\pgfsetstrokecolor{currentstroke}%
\pgfsetstrokeopacity{0.000000}%
\pgfsetdash{}{0pt}%
\pgfpathmoveto{\pgfqpoint{0.000000in}{0.000000in}}%
\pgfpathlineto{\pgfqpoint{3.196757in}{0.000000in}}%
\pgfpathlineto{\pgfqpoint{3.196757in}{0.554906in}}%
\pgfpathlineto{\pgfqpoint{0.000000in}{0.554906in}}%
\pgfpathclose%
\pgfusepath{fill}%
\end{pgfscope}%
\begin{pgfscope}%
\pgfsetbuttcap%
\pgfsetmiterjoin%
\definecolor{currentfill}{rgb}{1.000000,1.000000,1.000000}%
\pgfsetfillcolor{currentfill}%
\pgfsetfillopacity{0.800000}%
\pgfsetlinewidth{1.003750pt}%
\definecolor{currentstroke}{rgb}{0.800000,0.800000,0.800000}%
\pgfsetstrokecolor{currentstroke}%
\pgfsetstrokeopacity{0.800000}%
\pgfsetdash{}{0pt}%
\pgfpathmoveto{\pgfqpoint{0.122222in}{0.100000in}}%
\pgfpathlineto{\pgfqpoint{3.074535in}{0.100000in}}%
\pgfpathquadraticcurveto{\pgfqpoint{3.096757in}{0.100000in}}{\pgfqpoint{3.096757in}{0.122222in}}%
\pgfpathlineto{\pgfqpoint{3.096757in}{0.432684in}}%
\pgfpathquadraticcurveto{\pgfqpoint{3.096757in}{0.454906in}}{\pgfqpoint{3.074535in}{0.454906in}}%
\pgfpathlineto{\pgfqpoint{0.122222in}{0.454906in}}%
\pgfpathquadraticcurveto{\pgfqpoint{0.100000in}{0.454906in}}{\pgfqpoint{0.100000in}{0.432684in}}%
\pgfpathlineto{\pgfqpoint{0.100000in}{0.122222in}}%
\pgfpathquadraticcurveto{\pgfqpoint{0.100000in}{0.100000in}}{\pgfqpoint{0.122222in}{0.100000in}}%
\pgfpathclose%
\pgfusepath{stroke,fill}%
\end{pgfscope}%
\begin{pgfscope}%
\pgfsetbuttcap%
\pgfsetroundjoin%
\pgfsetlinewidth{1.003750pt}%
\definecolor{currentstroke}{rgb}{0.298039,0.447059,0.690196}%
\pgfsetstrokecolor{currentstroke}%
\pgfsetdash{}{0pt}%
\pgfpathmoveto{\pgfqpoint{0.255556in}{0.310482in}}%
\pgfpathlineto{\pgfqpoint{0.255556in}{0.421593in}}%
\pgfusepath{stroke}%
\end{pgfscope}%
\begin{pgfscope}%
\pgfsetroundcap%
\pgfsetroundjoin%
\pgfsetlinewidth{1.003750pt}%
\definecolor{currentstroke}{rgb}{0.298039,0.447059,0.690196}%
\pgfsetstrokecolor{currentstroke}%
\pgfsetdash{}{0pt}%
\pgfpathmoveto{\pgfqpoint{0.144444in}{0.366037in}}%
\pgfpathlineto{\pgfqpoint{0.366667in}{0.366037in}}%
\pgfusepath{stroke}%
\end{pgfscope}%
\begin{pgfscope}%
\definecolor{textcolor}{rgb}{0.150000,0.150000,0.150000}%
\pgfsetstrokecolor{textcolor}%
\pgfsetfillcolor{textcolor}%
\pgftext[x=0.455556in,y=0.327148in,left,base]{\color{textcolor}\rmfamily\fontsize{8.000000}{9.600000}\selectfont Vanilla GCN}%
\end{pgfscope}%
\begin{pgfscope}%
\pgfsetbuttcap%
\pgfsetroundjoin%
\pgfsetlinewidth{1.003750pt}%
\definecolor{currentstroke}{rgb}{0.866667,0.517647,0.321569}%
\pgfsetstrokecolor{currentstroke}%
\pgfsetdash{}{0pt}%
\pgfpathmoveto{\pgfqpoint{0.255556in}{0.155549in}}%
\pgfpathlineto{\pgfqpoint{0.255556in}{0.266660in}}%
\pgfusepath{stroke}%
\end{pgfscope}%
\begin{pgfscope}%
\pgfsetbuttcap%
\pgfsetroundjoin%
\pgfsetlinewidth{1.003750pt}%
\definecolor{currentstroke}{rgb}{0.866667,0.517647,0.321569}%
\pgfsetstrokecolor{currentstroke}%
\pgfsetdash{{3.700000pt}{1.600000pt}}{0.000000pt}%
\pgfpathmoveto{\pgfqpoint{0.144444in}{0.211104in}}%
\pgfpathlineto{\pgfqpoint{0.366667in}{0.211104in}}%
\pgfusepath{stroke}%
\end{pgfscope}%
\begin{pgfscope}%
\definecolor{textcolor}{rgb}{0.150000,0.150000,0.150000}%
\pgfsetstrokecolor{textcolor}%
\pgfsetfillcolor{textcolor}%
\pgftext[x=0.455556in,y=0.172215in,left,base]{\color{textcolor}\rmfamily\fontsize{8.000000}{9.600000}\selectfont Vanilla GDC}%
\end{pgfscope}%
\begin{pgfscope}%
\pgfsetbuttcap%
\pgfsetroundjoin%
\pgfsetlinewidth{1.003750pt}%
\definecolor{currentstroke}{rgb}{0.333333,0.658824,0.407843}%
\pgfsetstrokecolor{currentstroke}%
\pgfsetdash{}{0pt}%
\pgfpathmoveto{\pgfqpoint{1.456685in}{0.310482in}}%
\pgfpathlineto{\pgfqpoint{1.456685in}{0.421593in}}%
\pgfusepath{stroke}%
\end{pgfscope}%
\begin{pgfscope}%
\pgfsetbuttcap%
\pgfsetroundjoin%
\pgfsetlinewidth{1.003750pt}%
\definecolor{currentstroke}{rgb}{0.333333,0.658824,0.407843}%
\pgfsetstrokecolor{currentstroke}%
\pgfsetdash{{3.700000pt}{1.600000pt}}{0.000000pt}%
\pgfpathmoveto{\pgfqpoint{1.345574in}{0.366037in}}%
\pgfpathlineto{\pgfqpoint{1.567796in}{0.366037in}}%
\pgfusepath{stroke}%
\end{pgfscope}%
\begin{pgfscope}%
\definecolor{textcolor}{rgb}{0.150000,0.150000,0.150000}%
\pgfsetstrokecolor{textcolor}%
\pgfsetfillcolor{textcolor}%
\pgftext[x=1.656685in,y=0.327148in,left,base]{\color{textcolor}\rmfamily\fontsize{8.000000}{9.600000}\selectfont Soft Medoid GDC (T=1.0)}%
\end{pgfscope}%
\begin{pgfscope}%
\pgfsetbuttcap%
\pgfsetroundjoin%
\pgfsetlinewidth{1.003750pt}%
\definecolor{currentstroke}{rgb}{0.768627,0.305882,0.321569}%
\pgfsetstrokecolor{currentstroke}%
\pgfsetdash{}{0pt}%
\pgfpathmoveto{\pgfqpoint{1.456685in}{0.149378in}}%
\pgfpathlineto{\pgfqpoint{1.456685in}{0.260489in}}%
\pgfusepath{stroke}%
\end{pgfscope}%
\begin{pgfscope}%
\pgfsetbuttcap%
\pgfsetroundjoin%
\pgfsetlinewidth{1.003750pt}%
\definecolor{currentstroke}{rgb}{0.768627,0.305882,0.321569}%
\pgfsetstrokecolor{currentstroke}%
\pgfsetdash{{3.700000pt}{1.600000pt}}{0.000000pt}%
\pgfpathmoveto{\pgfqpoint{1.345574in}{0.204933in}}%
\pgfpathlineto{\pgfqpoint{1.567796in}{0.204933in}}%
\pgfusepath{stroke}%
\end{pgfscope}%
\begin{pgfscope}%
\definecolor{textcolor}{rgb}{0.150000,0.150000,0.150000}%
\pgfsetstrokecolor{textcolor}%
\pgfsetfillcolor{textcolor}%
\pgftext[x=1.656685in,y=0.166044in,left,base]{\color{textcolor}\rmfamily\fontsize{8.000000}{9.600000}\selectfont SVD GCN}%
\end{pgfscope}%
\end{pgfpicture}%
\makeatother%
\endgroup%
}}
  \vspace{-14pt}
  \makebox[\linewidth][c]{
    \(\begin{array}{cc}
      \subfloat[]{\resizebox{0.5\linewidth}{!}{%% Creator: Matplotlib, PGF backend
%%
%% To include the figure in your LaTeX document, write
%%   \input{<filename>.pgf}
%%
%% Make sure the required packages are loaded in your preamble
%%   \usepackage{pgf}
%%
%% Figures using additional raster images can only be included by \input if
%% they are in the same directory as the main LaTeX file. For loading figures
%% from other directories you can use the `import` package
%%   \usepackage{import}
%% and then include the figures with
%%   \import{<path to file>}{<filename>.pgf}
%%
%% Matplotlib used the following preamble
%%   \usepackage[utf8]{inputenc}
%%   \usepackage[T1]{fontenc}
%%   \usepackage{amsmath}
%%   \newcommand*{\mat}[1]{\boldsymbol{#1}}
%%
\begingroup%
\makeatletter%
\begin{pgfpicture}%
\pgfpathrectangle{\pgfpointorigin}{\pgfqpoint{2.151259in}{1.814099in}}%
\pgfusepath{use as bounding box, clip}%
\begin{pgfscope}%
\pgfsetbuttcap%
\pgfsetmiterjoin%
\definecolor{currentfill}{rgb}{1.000000,1.000000,1.000000}%
\pgfsetfillcolor{currentfill}%
\pgfsetlinewidth{0.000000pt}%
\definecolor{currentstroke}{rgb}{1.000000,1.000000,1.000000}%
\pgfsetstrokecolor{currentstroke}%
\pgfsetdash{}{0pt}%
\pgfpathmoveto{\pgfqpoint{0.000000in}{0.000000in}}%
\pgfpathlineto{\pgfqpoint{2.151259in}{0.000000in}}%
\pgfpathlineto{\pgfqpoint{2.151259in}{1.814099in}}%
\pgfpathlineto{\pgfqpoint{0.000000in}{1.814099in}}%
\pgfpathclose%
\pgfusepath{fill}%
\end{pgfscope}%
\begin{pgfscope}%
\pgfsetbuttcap%
\pgfsetmiterjoin%
\definecolor{currentfill}{rgb}{1.000000,1.000000,1.000000}%
\pgfsetfillcolor{currentfill}%
\pgfsetlinewidth{0.000000pt}%
\definecolor{currentstroke}{rgb}{0.000000,0.000000,0.000000}%
\pgfsetstrokecolor{currentstroke}%
\pgfsetstrokeopacity{0.000000}%
\pgfsetdash{}{0pt}%
\pgfpathmoveto{\pgfqpoint{0.578368in}{0.468349in}}%
\pgfpathlineto{\pgfqpoint{1.857118in}{0.468349in}}%
\pgfpathlineto{\pgfqpoint{1.857118in}{1.714099in}}%
\pgfpathlineto{\pgfqpoint{0.578368in}{1.714099in}}%
\pgfpathclose%
\pgfusepath{fill}%
\end{pgfscope}%
\begin{pgfscope}%
\pgfpathrectangle{\pgfqpoint{0.578368in}{0.468349in}}{\pgfqpoint{1.278750in}{1.245750in}}%
\pgfusepath{clip}%
\pgfsetroundcap%
\pgfsetroundjoin%
\pgfsetlinewidth{0.501875pt}%
\definecolor{currentstroke}{rgb}{0.800000,0.800000,0.800000}%
\pgfsetstrokecolor{currentstroke}%
\pgfsetdash{}{0pt}%
\pgfpathmoveto{\pgfqpoint{0.854929in}{0.468349in}}%
\pgfpathlineto{\pgfqpoint{0.854929in}{1.714099in}}%
\pgfusepath{stroke}%
\end{pgfscope}%
\begin{pgfscope}%
\definecolor{textcolor}{rgb}{0.150000,0.150000,0.150000}%
\pgfsetstrokecolor{textcolor}%
\pgfsetfillcolor{textcolor}%
\pgftext[x=0.854929in,y=0.378072in,,top]{\color{textcolor}\rmfamily\fontsize{8.000000}{9.600000}\selectfont \(\displaystyle 10^{1}\)}%
\end{pgfscope}%
\begin{pgfscope}%
\pgfpathrectangle{\pgfqpoint{0.578368in}{0.468349in}}{\pgfqpoint{1.278750in}{1.245750in}}%
\pgfusepath{clip}%
\pgfsetroundcap%
\pgfsetroundjoin%
\pgfsetlinewidth{0.501875pt}%
\definecolor{currentstroke}{rgb}{0.800000,0.800000,0.800000}%
\pgfsetstrokecolor{currentstroke}%
\pgfsetdash{}{0pt}%
\pgfpathmoveto{\pgfqpoint{1.580557in}{0.468349in}}%
\pgfpathlineto{\pgfqpoint{1.580557in}{1.714099in}}%
\pgfusepath{stroke}%
\end{pgfscope}%
\begin{pgfscope}%
\definecolor{textcolor}{rgb}{0.150000,0.150000,0.150000}%
\pgfsetstrokecolor{textcolor}%
\pgfsetfillcolor{textcolor}%
\pgftext[x=1.580557in,y=0.378072in,,top]{\color{textcolor}\rmfamily\fontsize{8.000000}{9.600000}\selectfont \(\displaystyle 10^{2}\)}%
\end{pgfscope}%
\begin{pgfscope}%
\definecolor{textcolor}{rgb}{0.150000,0.150000,0.150000}%
\pgfsetstrokecolor{textcolor}%
\pgfsetfillcolor{textcolor}%
\pgftext[x=1.217743in,y=0.222655in,,top]{\color{textcolor}\rmfamily\fontsize{10.000000}{12.000000}\selectfont Degree of adversarial nodes}%
\end{pgfscope}%
\begin{pgfscope}%
\pgfpathrectangle{\pgfqpoint{0.578368in}{0.468349in}}{\pgfqpoint{1.278750in}{1.245750in}}%
\pgfusepath{clip}%
\pgfsetroundcap%
\pgfsetroundjoin%
\pgfsetlinewidth{0.501875pt}%
\definecolor{currentstroke}{rgb}{0.800000,0.800000,0.800000}%
\pgfsetstrokecolor{currentstroke}%
\pgfsetdash{}{0pt}%
\pgfpathmoveto{\pgfqpoint{0.578368in}{0.642486in}}%
\pgfpathlineto{\pgfqpoint{1.857118in}{0.642486in}}%
\pgfusepath{stroke}%
\end{pgfscope}%
\begin{pgfscope}%
\definecolor{textcolor}{rgb}{0.150000,0.150000,0.150000}%
\pgfsetstrokecolor{textcolor}%
\pgfsetfillcolor{textcolor}%
\pgftext[x=0.278211in,y=0.604223in,left,base]{\color{textcolor}\rmfamily\fontsize{8.000000}{9.600000}\selectfont \(\displaystyle 0.70\)}%
\end{pgfscope}%
\begin{pgfscope}%
\pgfpathrectangle{\pgfqpoint{0.578368in}{0.468349in}}{\pgfqpoint{1.278750in}{1.245750in}}%
\pgfusepath{clip}%
\pgfsetroundcap%
\pgfsetroundjoin%
\pgfsetlinewidth{0.501875pt}%
\definecolor{currentstroke}{rgb}{0.800000,0.800000,0.800000}%
\pgfsetstrokecolor{currentstroke}%
\pgfsetdash{}{0pt}%
\pgfpathmoveto{\pgfqpoint{0.578368in}{1.055491in}}%
\pgfpathlineto{\pgfqpoint{1.857118in}{1.055491in}}%
\pgfusepath{stroke}%
\end{pgfscope}%
\begin{pgfscope}%
\definecolor{textcolor}{rgb}{0.150000,0.150000,0.150000}%
\pgfsetstrokecolor{textcolor}%
\pgfsetfillcolor{textcolor}%
\pgftext[x=0.278211in,y=1.017229in,left,base]{\color{textcolor}\rmfamily\fontsize{8.000000}{9.600000}\selectfont \(\displaystyle 0.75\)}%
\end{pgfscope}%
\begin{pgfscope}%
\pgfpathrectangle{\pgfqpoint{0.578368in}{0.468349in}}{\pgfqpoint{1.278750in}{1.245750in}}%
\pgfusepath{clip}%
\pgfsetroundcap%
\pgfsetroundjoin%
\pgfsetlinewidth{0.501875pt}%
\definecolor{currentstroke}{rgb}{0.800000,0.800000,0.800000}%
\pgfsetstrokecolor{currentstroke}%
\pgfsetdash{}{0pt}%
\pgfpathmoveto{\pgfqpoint{0.578368in}{1.468497in}}%
\pgfpathlineto{\pgfqpoint{1.857118in}{1.468497in}}%
\pgfusepath{stroke}%
\end{pgfscope}%
\begin{pgfscope}%
\definecolor{textcolor}{rgb}{0.150000,0.150000,0.150000}%
\pgfsetstrokecolor{textcolor}%
\pgfsetfillcolor{textcolor}%
\pgftext[x=0.278211in,y=1.430235in,left,base]{\color{textcolor}\rmfamily\fontsize{8.000000}{9.600000}\selectfont \(\displaystyle 0.80\)}%
\end{pgfscope}%
\begin{pgfscope}%
\definecolor{textcolor}{rgb}{0.150000,0.150000,0.150000}%
\pgfsetstrokecolor{textcolor}%
\pgfsetfillcolor{textcolor}%
\pgftext[x=0.222655in,y=1.091224in,,bottom,rotate=90.000000]{\color{textcolor}\rmfamily\fontsize{10.000000}{12.000000}\selectfont Accuracy}%
\end{pgfscope}%
\begin{pgfscope}%
\pgfpathrectangle{\pgfqpoint{0.578368in}{0.468349in}}{\pgfqpoint{1.278750in}{1.245750in}}%
\pgfusepath{clip}%
\pgfsetbuttcap%
\pgfsetroundjoin%
\pgfsetlinewidth{1.003750pt}%
\definecolor{currentstroke}{rgb}{0.298039,0.447059,0.690196}%
\pgfsetstrokecolor{currentstroke}%
\pgfsetdash{}{0pt}%
\pgfpathmoveto{\pgfqpoint{0.636493in}{1.375983in}}%
\pgfpathlineto{\pgfqpoint{0.636493in}{1.415181in}}%
\pgfusepath{stroke}%
\end{pgfscope}%
\begin{pgfscope}%
\pgfpathrectangle{\pgfqpoint{0.578368in}{0.468349in}}{\pgfqpoint{1.278750in}{1.245750in}}%
\pgfusepath{clip}%
\pgfsetbuttcap%
\pgfsetroundjoin%
\pgfsetlinewidth{1.003750pt}%
\definecolor{currentstroke}{rgb}{0.298039,0.447059,0.690196}%
\pgfsetstrokecolor{currentstroke}%
\pgfsetdash{}{0pt}%
\pgfpathmoveto{\pgfqpoint{0.854929in}{1.446159in}}%
\pgfpathlineto{\pgfqpoint{0.854929in}{1.477776in}}%
\pgfusepath{stroke}%
\end{pgfscope}%
\begin{pgfscope}%
\pgfpathrectangle{\pgfqpoint{0.578368in}{0.468349in}}{\pgfqpoint{1.278750in}{1.245750in}}%
\pgfusepath{clip}%
\pgfsetbuttcap%
\pgfsetroundjoin%
\pgfsetlinewidth{1.003750pt}%
\definecolor{currentstroke}{rgb}{0.298039,0.447059,0.690196}%
\pgfsetstrokecolor{currentstroke}%
\pgfsetdash{}{0pt}%
\pgfpathmoveto{\pgfqpoint{1.073365in}{1.466677in}}%
\pgfpathlineto{\pgfqpoint{1.073365in}{1.509496in}}%
\pgfusepath{stroke}%
\end{pgfscope}%
\begin{pgfscope}%
\pgfpathrectangle{\pgfqpoint{0.578368in}{0.468349in}}{\pgfqpoint{1.278750in}{1.245750in}}%
\pgfusepath{clip}%
\pgfsetbuttcap%
\pgfsetroundjoin%
\pgfsetlinewidth{1.003750pt}%
\definecolor{currentstroke}{rgb}{0.298039,0.447059,0.690196}%
\pgfsetstrokecolor{currentstroke}%
\pgfsetdash{}{0pt}%
\pgfpathmoveto{\pgfqpoint{1.362121in}{1.476710in}}%
\pgfpathlineto{\pgfqpoint{1.362121in}{1.505993in}}%
\pgfusepath{stroke}%
\end{pgfscope}%
\begin{pgfscope}%
\pgfpathrectangle{\pgfqpoint{0.578368in}{0.468349in}}{\pgfqpoint{1.278750in}{1.245750in}}%
\pgfusepath{clip}%
\pgfsetbuttcap%
\pgfsetroundjoin%
\pgfsetlinewidth{1.003750pt}%
\definecolor{currentstroke}{rgb}{0.298039,0.447059,0.690196}%
\pgfsetstrokecolor{currentstroke}%
\pgfsetdash{}{0pt}%
\pgfpathmoveto{\pgfqpoint{1.580557in}{1.492857in}}%
\pgfpathlineto{\pgfqpoint{1.580557in}{1.531201in}}%
\pgfusepath{stroke}%
\end{pgfscope}%
\begin{pgfscope}%
\pgfpathrectangle{\pgfqpoint{0.578368in}{0.468349in}}{\pgfqpoint{1.278750in}{1.245750in}}%
\pgfusepath{clip}%
\pgfsetbuttcap%
\pgfsetroundjoin%
\pgfsetlinewidth{1.003750pt}%
\definecolor{currentstroke}{rgb}{0.298039,0.447059,0.690196}%
\pgfsetstrokecolor{currentstroke}%
\pgfsetdash{}{0pt}%
\pgfpathmoveto{\pgfqpoint{1.798993in}{1.502234in}}%
\pgfpathlineto{\pgfqpoint{1.798993in}{1.539236in}}%
\pgfusepath{stroke}%
\end{pgfscope}%
\begin{pgfscope}%
\pgfpathrectangle{\pgfqpoint{0.578368in}{0.468349in}}{\pgfqpoint{1.278750in}{1.245750in}}%
\pgfusepath{clip}%
\pgfsetbuttcap%
\pgfsetroundjoin%
\pgfsetlinewidth{1.003750pt}%
\definecolor{currentstroke}{rgb}{0.866667,0.517647,0.321569}%
\pgfsetstrokecolor{currentstroke}%
\pgfsetdash{}{0pt}%
\pgfpathmoveto{\pgfqpoint{0.636493in}{0.957893in}}%
\pgfpathlineto{\pgfqpoint{0.636493in}{1.008347in}}%
\pgfusepath{stroke}%
\end{pgfscope}%
\begin{pgfscope}%
\pgfpathrectangle{\pgfqpoint{0.578368in}{0.468349in}}{\pgfqpoint{1.278750in}{1.245750in}}%
\pgfusepath{clip}%
\pgfsetbuttcap%
\pgfsetroundjoin%
\pgfsetlinewidth{1.003750pt}%
\definecolor{currentstroke}{rgb}{0.866667,0.517647,0.321569}%
\pgfsetstrokecolor{currentstroke}%
\pgfsetdash{}{0pt}%
\pgfpathmoveto{\pgfqpoint{0.854929in}{0.784220in}}%
\pgfpathlineto{\pgfqpoint{0.854929in}{0.842474in}}%
\pgfusepath{stroke}%
\end{pgfscope}%
\begin{pgfscope}%
\pgfpathrectangle{\pgfqpoint{0.578368in}{0.468349in}}{\pgfqpoint{1.278750in}{1.245750in}}%
\pgfusepath{clip}%
\pgfsetbuttcap%
\pgfsetroundjoin%
\pgfsetlinewidth{1.003750pt}%
\definecolor{currentstroke}{rgb}{0.866667,0.517647,0.321569}%
\pgfsetstrokecolor{currentstroke}%
\pgfsetdash{}{0pt}%
\pgfpathmoveto{\pgfqpoint{1.073365in}{0.890560in}}%
\pgfpathlineto{\pgfqpoint{1.073365in}{0.918967in}}%
\pgfusepath{stroke}%
\end{pgfscope}%
\begin{pgfscope}%
\pgfpathrectangle{\pgfqpoint{0.578368in}{0.468349in}}{\pgfqpoint{1.278750in}{1.245750in}}%
\pgfusepath{clip}%
\pgfsetbuttcap%
\pgfsetroundjoin%
\pgfsetlinewidth{1.003750pt}%
\definecolor{currentstroke}{rgb}{0.866667,0.517647,0.321569}%
\pgfsetstrokecolor{currentstroke}%
\pgfsetdash{}{0pt}%
\pgfpathmoveto{\pgfqpoint{1.362121in}{0.895569in}}%
\pgfpathlineto{\pgfqpoint{1.362121in}{0.935723in}}%
\pgfusepath{stroke}%
\end{pgfscope}%
\begin{pgfscope}%
\pgfpathrectangle{\pgfqpoint{0.578368in}{0.468349in}}{\pgfqpoint{1.278750in}{1.245750in}}%
\pgfusepath{clip}%
\pgfsetbuttcap%
\pgfsetroundjoin%
\pgfsetlinewidth{1.003750pt}%
\definecolor{currentstroke}{rgb}{0.866667,0.517647,0.321569}%
\pgfsetstrokecolor{currentstroke}%
\pgfsetdash{}{0pt}%
\pgfpathmoveto{\pgfqpoint{1.580557in}{1.030459in}}%
\pgfpathlineto{\pgfqpoint{1.580557in}{1.094671in}}%
\pgfusepath{stroke}%
\end{pgfscope}%
\begin{pgfscope}%
\pgfpathrectangle{\pgfqpoint{0.578368in}{0.468349in}}{\pgfqpoint{1.278750in}{1.245750in}}%
\pgfusepath{clip}%
\pgfsetbuttcap%
\pgfsetroundjoin%
\pgfsetlinewidth{1.003750pt}%
\definecolor{currentstroke}{rgb}{0.866667,0.517647,0.321569}%
\pgfsetstrokecolor{currentstroke}%
\pgfsetdash{}{0pt}%
\pgfpathmoveto{\pgfqpoint{1.798993in}{1.220272in}}%
\pgfpathlineto{\pgfqpoint{1.798993in}{1.261818in}}%
\pgfusepath{stroke}%
\end{pgfscope}%
\begin{pgfscope}%
\pgfpathrectangle{\pgfqpoint{0.578368in}{0.468349in}}{\pgfqpoint{1.278750in}{1.245750in}}%
\pgfusepath{clip}%
\pgfsetbuttcap%
\pgfsetroundjoin%
\pgfsetlinewidth{1.003750pt}%
\definecolor{currentstroke}{rgb}{0.333333,0.658824,0.407843}%
\pgfsetstrokecolor{currentstroke}%
\pgfsetdash{}{0pt}%
\pgfpathmoveto{\pgfqpoint{0.636493in}{1.098889in}}%
\pgfpathlineto{\pgfqpoint{0.636493in}{1.135071in}}%
\pgfusepath{stroke}%
\end{pgfscope}%
\begin{pgfscope}%
\pgfpathrectangle{\pgfqpoint{0.578368in}{0.468349in}}{\pgfqpoint{1.278750in}{1.245750in}}%
\pgfusepath{clip}%
\pgfsetbuttcap%
\pgfsetroundjoin%
\pgfsetlinewidth{1.003750pt}%
\definecolor{currentstroke}{rgb}{0.333333,0.658824,0.407843}%
\pgfsetstrokecolor{currentstroke}%
\pgfsetdash{}{0pt}%
\pgfpathmoveto{\pgfqpoint{0.854929in}{1.049890in}}%
\pgfpathlineto{\pgfqpoint{0.854929in}{1.131832in}}%
\pgfusepath{stroke}%
\end{pgfscope}%
\begin{pgfscope}%
\pgfpathrectangle{\pgfqpoint{0.578368in}{0.468349in}}{\pgfqpoint{1.278750in}{1.245750in}}%
\pgfusepath{clip}%
\pgfsetbuttcap%
\pgfsetroundjoin%
\pgfsetlinewidth{1.003750pt}%
\definecolor{currentstroke}{rgb}{0.333333,0.658824,0.407843}%
\pgfsetstrokecolor{currentstroke}%
\pgfsetdash{}{0pt}%
\pgfpathmoveto{\pgfqpoint{1.073365in}{1.165229in}}%
\pgfpathlineto{\pgfqpoint{1.073365in}{1.231974in}}%
\pgfusepath{stroke}%
\end{pgfscope}%
\begin{pgfscope}%
\pgfpathrectangle{\pgfqpoint{0.578368in}{0.468349in}}{\pgfqpoint{1.278750in}{1.245750in}}%
\pgfusepath{clip}%
\pgfsetbuttcap%
\pgfsetroundjoin%
\pgfsetlinewidth{1.003750pt}%
\definecolor{currentstroke}{rgb}{0.333333,0.658824,0.407843}%
\pgfsetstrokecolor{currentstroke}%
\pgfsetdash{}{0pt}%
\pgfpathmoveto{\pgfqpoint{1.362121in}{1.330436in}}%
\pgfpathlineto{\pgfqpoint{1.362121in}{1.382371in}}%
\pgfusepath{stroke}%
\end{pgfscope}%
\begin{pgfscope}%
\pgfpathrectangle{\pgfqpoint{0.578368in}{0.468349in}}{\pgfqpoint{1.278750in}{1.245750in}}%
\pgfusepath{clip}%
\pgfsetbuttcap%
\pgfsetroundjoin%
\pgfsetlinewidth{1.003750pt}%
\definecolor{currentstroke}{rgb}{0.333333,0.658824,0.407843}%
\pgfsetstrokecolor{currentstroke}%
\pgfsetdash{}{0pt}%
\pgfpathmoveto{\pgfqpoint{1.580557in}{1.502381in}}%
\pgfpathlineto{\pgfqpoint{1.580557in}{1.558679in}}%
\pgfusepath{stroke}%
\end{pgfscope}%
\begin{pgfscope}%
\pgfpathrectangle{\pgfqpoint{0.578368in}{0.468349in}}{\pgfqpoint{1.278750in}{1.245750in}}%
\pgfusepath{clip}%
\pgfsetbuttcap%
\pgfsetroundjoin%
\pgfsetlinewidth{1.003750pt}%
\definecolor{currentstroke}{rgb}{0.333333,0.658824,0.407843}%
\pgfsetstrokecolor{currentstroke}%
\pgfsetdash{}{0pt}%
\pgfpathmoveto{\pgfqpoint{1.798993in}{1.582065in}}%
\pgfpathlineto{\pgfqpoint{1.798993in}{1.657474in}}%
\pgfusepath{stroke}%
\end{pgfscope}%
\begin{pgfscope}%
\pgfpathrectangle{\pgfqpoint{0.578368in}{0.468349in}}{\pgfqpoint{1.278750in}{1.245750in}}%
\pgfusepath{clip}%
\pgfsetbuttcap%
\pgfsetroundjoin%
\pgfsetlinewidth{1.003750pt}%
\definecolor{currentstroke}{rgb}{0.768627,0.305882,0.321569}%
\pgfsetstrokecolor{currentstroke}%
\pgfsetdash{}{0pt}%
\pgfpathmoveto{\pgfqpoint{0.636493in}{1.170134in}}%
\pgfpathlineto{\pgfqpoint{0.636493in}{1.261895in}}%
\pgfusepath{stroke}%
\end{pgfscope}%
\begin{pgfscope}%
\pgfpathrectangle{\pgfqpoint{0.578368in}{0.468349in}}{\pgfqpoint{1.278750in}{1.245750in}}%
\pgfusepath{clip}%
\pgfsetbuttcap%
\pgfsetroundjoin%
\pgfsetlinewidth{1.003750pt}%
\definecolor{currentstroke}{rgb}{0.768627,0.305882,0.321569}%
\pgfsetstrokecolor{currentstroke}%
\pgfsetdash{}{0pt}%
\pgfpathmoveto{\pgfqpoint{0.854929in}{1.173295in}}%
\pgfpathlineto{\pgfqpoint{0.854929in}{1.265263in}}%
\pgfusepath{stroke}%
\end{pgfscope}%
\begin{pgfscope}%
\pgfpathrectangle{\pgfqpoint{0.578368in}{0.468349in}}{\pgfqpoint{1.278750in}{1.245750in}}%
\pgfusepath{clip}%
\pgfsetbuttcap%
\pgfsetroundjoin%
\pgfsetlinewidth{1.003750pt}%
\definecolor{currentstroke}{rgb}{0.768627,0.305882,0.321569}%
\pgfsetstrokecolor{currentstroke}%
\pgfsetdash{}{0pt}%
\pgfpathmoveto{\pgfqpoint{1.073365in}{1.125541in}}%
\pgfpathlineto{\pgfqpoint{1.073365in}{1.223777in}}%
\pgfusepath{stroke}%
\end{pgfscope}%
\begin{pgfscope}%
\pgfpathrectangle{\pgfqpoint{0.578368in}{0.468349in}}{\pgfqpoint{1.278750in}{1.245750in}}%
\pgfusepath{clip}%
\pgfsetbuttcap%
\pgfsetroundjoin%
\pgfsetlinewidth{1.003750pt}%
\definecolor{currentstroke}{rgb}{0.768627,0.305882,0.321569}%
\pgfsetstrokecolor{currentstroke}%
\pgfsetdash{}{0pt}%
\pgfpathmoveto{\pgfqpoint{1.362121in}{0.524974in}}%
\pgfpathlineto{\pgfqpoint{1.362121in}{0.622872in}}%
\pgfusepath{stroke}%
\end{pgfscope}%
\begin{pgfscope}%
\pgfpathrectangle{\pgfqpoint{0.578368in}{0.468349in}}{\pgfqpoint{1.278750in}{1.245750in}}%
\pgfusepath{clip}%
\pgfsetbuttcap%
\pgfsetroundjoin%
\pgfsetlinewidth{1.003750pt}%
\definecolor{currentstroke}{rgb}{0.768627,0.305882,0.321569}%
\pgfsetstrokecolor{currentstroke}%
\pgfsetdash{}{0pt}%
\pgfpathmoveto{\pgfqpoint{1.580557in}{0.622715in}}%
\pgfpathlineto{\pgfqpoint{1.580557in}{0.703611in}}%
\pgfusepath{stroke}%
\end{pgfscope}%
\begin{pgfscope}%
\pgfpathrectangle{\pgfqpoint{0.578368in}{0.468349in}}{\pgfqpoint{1.278750in}{1.245750in}}%
\pgfusepath{clip}%
\pgfsetbuttcap%
\pgfsetroundjoin%
\pgfsetlinewidth{1.003750pt}%
\definecolor{currentstroke}{rgb}{0.768627,0.305882,0.321569}%
\pgfsetstrokecolor{currentstroke}%
\pgfsetdash{}{0pt}%
\pgfpathmoveto{\pgfqpoint{1.798993in}{0.628338in}}%
\pgfpathlineto{\pgfqpoint{1.798993in}{0.695812in}}%
\pgfusepath{stroke}%
\end{pgfscope}%
\begin{pgfscope}%
\pgfpathrectangle{\pgfqpoint{0.578368in}{0.468349in}}{\pgfqpoint{1.278750in}{1.245750in}}%
\pgfusepath{clip}%
\pgfsetbuttcap%
\pgfsetroundjoin%
\pgfsetlinewidth{1.003750pt}%
\definecolor{currentstroke}{rgb}{0.298039,0.447059,0.690196}%
\pgfsetstrokecolor{currentstroke}%
\pgfsetdash{{3.700000pt}{1.600000pt}}{0.000000pt}%
\pgfpathmoveto{\pgfqpoint{0.636493in}{1.395582in}}%
\pgfpathlineto{\pgfqpoint{0.854929in}{1.461968in}}%
\pgfpathlineto{\pgfqpoint{1.073365in}{1.488087in}}%
\pgfpathlineto{\pgfqpoint{1.362121in}{1.491352in}}%
\pgfpathlineto{\pgfqpoint{1.580557in}{1.512029in}}%
\pgfpathlineto{\pgfqpoint{1.798993in}{1.520735in}}%
\pgfusepath{stroke}%
\end{pgfscope}%
\begin{pgfscope}%
\pgfpathrectangle{\pgfqpoint{0.578368in}{0.468349in}}{\pgfqpoint{1.278750in}{1.245750in}}%
\pgfusepath{clip}%
\pgfsetroundcap%
\pgfsetroundjoin%
\pgfsetlinewidth{1.003750pt}%
\definecolor{currentstroke}{rgb}{0.866667,0.517647,0.321569}%
\pgfsetstrokecolor{currentstroke}%
\pgfsetdash{}{0pt}%
\pgfpathmoveto{\pgfqpoint{0.636493in}{0.983120in}}%
\pgfpathlineto{\pgfqpoint{0.854929in}{0.813347in}}%
\pgfpathlineto{\pgfqpoint{1.073365in}{0.904763in}}%
\pgfpathlineto{\pgfqpoint{1.362121in}{0.915646in}}%
\pgfpathlineto{\pgfqpoint{1.580557in}{1.062565in}}%
\pgfpathlineto{\pgfqpoint{1.798993in}{1.241045in}}%
\pgfusepath{stroke}%
\end{pgfscope}%
\begin{pgfscope}%
\pgfpathrectangle{\pgfqpoint{0.578368in}{0.468349in}}{\pgfqpoint{1.278750in}{1.245750in}}%
\pgfusepath{clip}%
\pgfsetbuttcap%
\pgfsetroundjoin%
\pgfsetlinewidth{1.003750pt}%
\definecolor{currentstroke}{rgb}{0.333333,0.658824,0.407843}%
\pgfsetstrokecolor{currentstroke}%
\pgfsetdash{{3.700000pt}{1.600000pt}}{0.000000pt}%
\pgfpathmoveto{\pgfqpoint{0.636493in}{1.116980in}}%
\pgfpathlineto{\pgfqpoint{0.854929in}{1.090861in}}%
\pgfpathlineto{\pgfqpoint{1.073365in}{1.198602in}}%
\pgfpathlineto{\pgfqpoint{1.362121in}{1.356404in}}%
\pgfpathlineto{\pgfqpoint{1.580557in}{1.530530in}}%
\pgfpathlineto{\pgfqpoint{1.798993in}{1.619770in}}%
\pgfusepath{stroke}%
\end{pgfscope}%
\begin{pgfscope}%
\pgfpathrectangle{\pgfqpoint{0.578368in}{0.468349in}}{\pgfqpoint{1.278750in}{1.245750in}}%
\pgfusepath{clip}%
\pgfsetbuttcap%
\pgfsetroundjoin%
\pgfsetlinewidth{1.003750pt}%
\definecolor{currentstroke}{rgb}{0.768627,0.305882,0.321569}%
\pgfsetstrokecolor{currentstroke}%
\pgfsetdash{{3.700000pt}{1.600000pt}}{0.000000pt}%
\pgfpathmoveto{\pgfqpoint{0.636493in}{1.216014in}}%
\pgfpathlineto{\pgfqpoint{0.854929in}{1.219279in}}%
\pgfpathlineto{\pgfqpoint{1.073365in}{1.174659in}}%
\pgfpathlineto{\pgfqpoint{1.362121in}{0.573923in}}%
\pgfpathlineto{\pgfqpoint{1.580557in}{0.663163in}}%
\pgfpathlineto{\pgfqpoint{1.798993in}{0.662075in}}%
\pgfusepath{stroke}%
\end{pgfscope}%
\begin{pgfscope}%
\pgfsetrectcap%
\pgfsetmiterjoin%
\pgfsetlinewidth{0.752812pt}%
\definecolor{currentstroke}{rgb}{0.700000,0.700000,0.700000}%
\pgfsetstrokecolor{currentstroke}%
\pgfsetdash{}{0pt}%
\pgfpathmoveto{\pgfqpoint{0.578368in}{0.468349in}}%
\pgfpathlineto{\pgfqpoint{0.578368in}{1.714099in}}%
\pgfusepath{stroke}%
\end{pgfscope}%
\begin{pgfscope}%
\pgfsetrectcap%
\pgfsetmiterjoin%
\pgfsetlinewidth{0.752812pt}%
\definecolor{currentstroke}{rgb}{0.700000,0.700000,0.700000}%
\pgfsetstrokecolor{currentstroke}%
\pgfsetdash{}{0pt}%
\pgfpathmoveto{\pgfqpoint{1.857118in}{0.468349in}}%
\pgfpathlineto{\pgfqpoint{1.857118in}{1.714099in}}%
\pgfusepath{stroke}%
\end{pgfscope}%
\begin{pgfscope}%
\pgfsetrectcap%
\pgfsetmiterjoin%
\pgfsetlinewidth{0.752812pt}%
\definecolor{currentstroke}{rgb}{0.700000,0.700000,0.700000}%
\pgfsetstrokecolor{currentstroke}%
\pgfsetdash{}{0pt}%
\pgfpathmoveto{\pgfqpoint{0.578368in}{0.468349in}}%
\pgfpathlineto{\pgfqpoint{1.857118in}{0.468349in}}%
\pgfusepath{stroke}%
\end{pgfscope}%
\begin{pgfscope}%
\pgfsetrectcap%
\pgfsetmiterjoin%
\pgfsetlinewidth{0.752812pt}%
\definecolor{currentstroke}{rgb}{0.700000,0.700000,0.700000}%
\pgfsetstrokecolor{currentstroke}%
\pgfsetdash{}{0pt}%
\pgfpathmoveto{\pgfqpoint{0.578368in}{1.714099in}}%
\pgfpathlineto{\pgfqpoint{1.857118in}{1.714099in}}%
\pgfusepath{stroke}%
\end{pgfscope}%
\end{pgfpicture}%
\makeatother%
\endgroup%
}} &
      \subfloat[]{\resizebox{0.5\linewidth}{!}{%% Creator: Matplotlib, PGF backend
%%
%% To include the figure in your LaTeX document, write
%%   \input{<filename>.pgf}
%%
%% Make sure the required packages are loaded in your preamble
%%   \usepackage{pgf}
%%
%% Figures using additional raster images can only be included by \input if
%% they are in the same directory as the main LaTeX file. For loading figures
%% from other directories you can use the `import` package
%%   \usepackage{import}
%% and then include the figures with
%%   \import{<path to file>}{<filename>.pgf}
%%
%% Matplotlib used the following preamble
%%   \usepackage[utf8]{inputenc}
%%   \usepackage[T1]{fontenc}
%%   \usepackage{amsmath}
%%   \newcommand*{\mat}[1]{\boldsymbol{#1}}
%%
\begingroup%
\makeatletter%
\begin{pgfpicture}%
\pgfpathrectangle{\pgfpointorigin}{\pgfqpoint{2.151259in}{1.814099in}}%
\pgfusepath{use as bounding box, clip}%
\begin{pgfscope}%
\pgfsetbuttcap%
\pgfsetmiterjoin%
\definecolor{currentfill}{rgb}{1.000000,1.000000,1.000000}%
\pgfsetfillcolor{currentfill}%
\pgfsetlinewidth{0.000000pt}%
\definecolor{currentstroke}{rgb}{1.000000,1.000000,1.000000}%
\pgfsetstrokecolor{currentstroke}%
\pgfsetdash{}{0pt}%
\pgfpathmoveto{\pgfqpoint{0.000000in}{0.000000in}}%
\pgfpathlineto{\pgfqpoint{2.151259in}{0.000000in}}%
\pgfpathlineto{\pgfqpoint{2.151259in}{1.814099in}}%
\pgfpathlineto{\pgfqpoint{0.000000in}{1.814099in}}%
\pgfpathclose%
\pgfusepath{fill}%
\end{pgfscope}%
\begin{pgfscope}%
\pgfsetbuttcap%
\pgfsetmiterjoin%
\definecolor{currentfill}{rgb}{1.000000,1.000000,1.000000}%
\pgfsetfillcolor{currentfill}%
\pgfsetlinewidth{0.000000pt}%
\definecolor{currentstroke}{rgb}{0.000000,0.000000,0.000000}%
\pgfsetstrokecolor{currentstroke}%
\pgfsetstrokeopacity{0.000000}%
\pgfsetdash{}{0pt}%
\pgfpathmoveto{\pgfqpoint{0.578368in}{0.468349in}}%
\pgfpathlineto{\pgfqpoint{1.857118in}{0.468349in}}%
\pgfpathlineto{\pgfqpoint{1.857118in}{1.714099in}}%
\pgfpathlineto{\pgfqpoint{0.578368in}{1.714099in}}%
\pgfpathclose%
\pgfusepath{fill}%
\end{pgfscope}%
\begin{pgfscope}%
\pgfpathrectangle{\pgfqpoint{0.578368in}{0.468349in}}{\pgfqpoint{1.278750in}{1.245750in}}%
\pgfusepath{clip}%
\pgfsetroundcap%
\pgfsetroundjoin%
\pgfsetlinewidth{0.501875pt}%
\definecolor{currentstroke}{rgb}{0.800000,0.800000,0.800000}%
\pgfsetstrokecolor{currentstroke}%
\pgfsetdash{}{0pt}%
\pgfpathmoveto{\pgfqpoint{0.854929in}{0.468349in}}%
\pgfpathlineto{\pgfqpoint{0.854929in}{1.714099in}}%
\pgfusepath{stroke}%
\end{pgfscope}%
\begin{pgfscope}%
\definecolor{textcolor}{rgb}{0.150000,0.150000,0.150000}%
\pgfsetstrokecolor{textcolor}%
\pgfsetfillcolor{textcolor}%
\pgftext[x=0.854929in,y=0.378072in,,top]{\color{textcolor}\rmfamily\fontsize{8.000000}{9.600000}\selectfont \(\displaystyle 10^{1}\)}%
\end{pgfscope}%
\begin{pgfscope}%
\pgfpathrectangle{\pgfqpoint{0.578368in}{0.468349in}}{\pgfqpoint{1.278750in}{1.245750in}}%
\pgfusepath{clip}%
\pgfsetroundcap%
\pgfsetroundjoin%
\pgfsetlinewidth{0.501875pt}%
\definecolor{currentstroke}{rgb}{0.800000,0.800000,0.800000}%
\pgfsetstrokecolor{currentstroke}%
\pgfsetdash{}{0pt}%
\pgfpathmoveto{\pgfqpoint{1.580557in}{0.468349in}}%
\pgfpathlineto{\pgfqpoint{1.580557in}{1.714099in}}%
\pgfusepath{stroke}%
\end{pgfscope}%
\begin{pgfscope}%
\definecolor{textcolor}{rgb}{0.150000,0.150000,0.150000}%
\pgfsetstrokecolor{textcolor}%
\pgfsetfillcolor{textcolor}%
\pgftext[x=1.580557in,y=0.378072in,,top]{\color{textcolor}\rmfamily\fontsize{8.000000}{9.600000}\selectfont \(\displaystyle 10^{2}\)}%
\end{pgfscope}%
\begin{pgfscope}%
\definecolor{textcolor}{rgb}{0.150000,0.150000,0.150000}%
\pgfsetstrokecolor{textcolor}%
\pgfsetfillcolor{textcolor}%
\pgftext[x=1.217743in,y=0.222655in,,top]{\color{textcolor}\rmfamily\fontsize{10.000000}{12.000000}\selectfont Degree of adversarial nodes}%
\end{pgfscope}%
\begin{pgfscope}%
\pgfpathrectangle{\pgfqpoint{0.578368in}{0.468349in}}{\pgfqpoint{1.278750in}{1.245750in}}%
\pgfusepath{clip}%
\pgfsetroundcap%
\pgfsetroundjoin%
\pgfsetlinewidth{0.501875pt}%
\definecolor{currentstroke}{rgb}{0.800000,0.800000,0.800000}%
\pgfsetstrokecolor{currentstroke}%
\pgfsetdash{}{0pt}%
\pgfpathmoveto{\pgfqpoint{0.578368in}{0.602481in}}%
\pgfpathlineto{\pgfqpoint{1.857118in}{0.602481in}}%
\pgfusepath{stroke}%
\end{pgfscope}%
\begin{pgfscope}%
\definecolor{textcolor}{rgb}{0.150000,0.150000,0.150000}%
\pgfsetstrokecolor{textcolor}%
\pgfsetfillcolor{textcolor}%
\pgftext[x=0.278211in,y=0.564219in,left,base]{\color{textcolor}\rmfamily\fontsize{8.000000}{9.600000}\selectfont \(\displaystyle 0.65\)}%
\end{pgfscope}%
\begin{pgfscope}%
\pgfpathrectangle{\pgfqpoint{0.578368in}{0.468349in}}{\pgfqpoint{1.278750in}{1.245750in}}%
\pgfusepath{clip}%
\pgfsetroundcap%
\pgfsetroundjoin%
\pgfsetlinewidth{0.501875pt}%
\definecolor{currentstroke}{rgb}{0.800000,0.800000,0.800000}%
\pgfsetstrokecolor{currentstroke}%
\pgfsetdash{}{0pt}%
\pgfpathmoveto{\pgfqpoint{0.578368in}{0.924044in}}%
\pgfpathlineto{\pgfqpoint{1.857118in}{0.924044in}}%
\pgfusepath{stroke}%
\end{pgfscope}%
\begin{pgfscope}%
\definecolor{textcolor}{rgb}{0.150000,0.150000,0.150000}%
\pgfsetstrokecolor{textcolor}%
\pgfsetfillcolor{textcolor}%
\pgftext[x=0.278211in,y=0.885781in,left,base]{\color{textcolor}\rmfamily\fontsize{8.000000}{9.600000}\selectfont \(\displaystyle 0.70\)}%
\end{pgfscope}%
\begin{pgfscope}%
\pgfpathrectangle{\pgfqpoint{0.578368in}{0.468349in}}{\pgfqpoint{1.278750in}{1.245750in}}%
\pgfusepath{clip}%
\pgfsetroundcap%
\pgfsetroundjoin%
\pgfsetlinewidth{0.501875pt}%
\definecolor{currentstroke}{rgb}{0.800000,0.800000,0.800000}%
\pgfsetstrokecolor{currentstroke}%
\pgfsetdash{}{0pt}%
\pgfpathmoveto{\pgfqpoint{0.578368in}{1.245606in}}%
\pgfpathlineto{\pgfqpoint{1.857118in}{1.245606in}}%
\pgfusepath{stroke}%
\end{pgfscope}%
\begin{pgfscope}%
\definecolor{textcolor}{rgb}{0.150000,0.150000,0.150000}%
\pgfsetstrokecolor{textcolor}%
\pgfsetfillcolor{textcolor}%
\pgftext[x=0.278211in,y=1.207344in,left,base]{\color{textcolor}\rmfamily\fontsize{8.000000}{9.600000}\selectfont \(\displaystyle 0.75\)}%
\end{pgfscope}%
\begin{pgfscope}%
\pgfpathrectangle{\pgfqpoint{0.578368in}{0.468349in}}{\pgfqpoint{1.278750in}{1.245750in}}%
\pgfusepath{clip}%
\pgfsetroundcap%
\pgfsetroundjoin%
\pgfsetlinewidth{0.501875pt}%
\definecolor{currentstroke}{rgb}{0.800000,0.800000,0.800000}%
\pgfsetstrokecolor{currentstroke}%
\pgfsetdash{}{0pt}%
\pgfpathmoveto{\pgfqpoint{0.578368in}{1.567168in}}%
\pgfpathlineto{\pgfqpoint{1.857118in}{1.567168in}}%
\pgfusepath{stroke}%
\end{pgfscope}%
\begin{pgfscope}%
\definecolor{textcolor}{rgb}{0.150000,0.150000,0.150000}%
\pgfsetstrokecolor{textcolor}%
\pgfsetfillcolor{textcolor}%
\pgftext[x=0.278211in,y=1.528906in,left,base]{\color{textcolor}\rmfamily\fontsize{8.000000}{9.600000}\selectfont \(\displaystyle 0.80\)}%
\end{pgfscope}%
\begin{pgfscope}%
\definecolor{textcolor}{rgb}{0.150000,0.150000,0.150000}%
\pgfsetstrokecolor{textcolor}%
\pgfsetfillcolor{textcolor}%
\pgftext[x=0.222655in,y=1.091224in,,bottom,rotate=90.000000]{\color{textcolor}\rmfamily\fontsize{10.000000}{12.000000}\selectfont Accuracy}%
\end{pgfscope}%
\begin{pgfscope}%
\pgfpathrectangle{\pgfqpoint{0.578368in}{0.468349in}}{\pgfqpoint{1.278750in}{1.245750in}}%
\pgfusepath{clip}%
\pgfsetbuttcap%
\pgfsetroundjoin%
\pgfsetlinewidth{1.003750pt}%
\definecolor{currentstroke}{rgb}{0.298039,0.447059,0.690196}%
\pgfsetstrokecolor{currentstroke}%
\pgfsetdash{}{0pt}%
\pgfpathmoveto{\pgfqpoint{0.636493in}{1.329379in}}%
\pgfpathlineto{\pgfqpoint{0.636493in}{1.374513in}}%
\pgfusepath{stroke}%
\end{pgfscope}%
\begin{pgfscope}%
\pgfpathrectangle{\pgfqpoint{0.578368in}{0.468349in}}{\pgfqpoint{1.278750in}{1.245750in}}%
\pgfusepath{clip}%
\pgfsetbuttcap%
\pgfsetroundjoin%
\pgfsetlinewidth{1.003750pt}%
\definecolor{currentstroke}{rgb}{0.298039,0.447059,0.690196}%
\pgfsetstrokecolor{currentstroke}%
\pgfsetdash{}{0pt}%
\pgfpathmoveto{\pgfqpoint{0.854929in}{1.447782in}}%
\pgfpathlineto{\pgfqpoint{0.854929in}{1.469638in}}%
\pgfusepath{stroke}%
\end{pgfscope}%
\begin{pgfscope}%
\pgfpathrectangle{\pgfqpoint{0.578368in}{0.468349in}}{\pgfqpoint{1.278750in}{1.245750in}}%
\pgfusepath{clip}%
\pgfsetbuttcap%
\pgfsetroundjoin%
\pgfsetlinewidth{1.003750pt}%
\definecolor{currentstroke}{rgb}{0.298039,0.447059,0.690196}%
\pgfsetstrokecolor{currentstroke}%
\pgfsetdash{}{0pt}%
\pgfpathmoveto{\pgfqpoint{1.073365in}{1.503161in}}%
\pgfpathlineto{\pgfqpoint{1.073365in}{1.519328in}}%
\pgfusepath{stroke}%
\end{pgfscope}%
\begin{pgfscope}%
\pgfpathrectangle{\pgfqpoint{0.578368in}{0.468349in}}{\pgfqpoint{1.278750in}{1.245750in}}%
\pgfusepath{clip}%
\pgfsetbuttcap%
\pgfsetroundjoin%
\pgfsetlinewidth{1.003750pt}%
\definecolor{currentstroke}{rgb}{0.298039,0.447059,0.690196}%
\pgfsetstrokecolor{currentstroke}%
\pgfsetdash{}{0pt}%
\pgfpathmoveto{\pgfqpoint{1.362121in}{1.512513in}}%
\pgfpathlineto{\pgfqpoint{1.362121in}{1.538785in}}%
\pgfusepath{stroke}%
\end{pgfscope}%
\begin{pgfscope}%
\pgfpathrectangle{\pgfqpoint{0.578368in}{0.468349in}}{\pgfqpoint{1.278750in}{1.245750in}}%
\pgfusepath{clip}%
\pgfsetbuttcap%
\pgfsetroundjoin%
\pgfsetlinewidth{1.003750pt}%
\definecolor{currentstroke}{rgb}{0.298039,0.447059,0.690196}%
\pgfsetstrokecolor{currentstroke}%
\pgfsetdash{}{0pt}%
\pgfpathmoveto{\pgfqpoint{1.580557in}{1.520118in}}%
\pgfpathlineto{\pgfqpoint{1.580557in}{1.543043in}}%
\pgfusepath{stroke}%
\end{pgfscope}%
\begin{pgfscope}%
\pgfpathrectangle{\pgfqpoint{0.578368in}{0.468349in}}{\pgfqpoint{1.278750in}{1.245750in}}%
\pgfusepath{clip}%
\pgfsetbuttcap%
\pgfsetroundjoin%
\pgfsetlinewidth{1.003750pt}%
\definecolor{currentstroke}{rgb}{0.298039,0.447059,0.690196}%
\pgfsetstrokecolor{currentstroke}%
\pgfsetdash{}{0pt}%
\pgfpathmoveto{\pgfqpoint{1.798993in}{1.527901in}}%
\pgfpathlineto{\pgfqpoint{1.798993in}{1.548817in}}%
\pgfusepath{stroke}%
\end{pgfscope}%
\begin{pgfscope}%
\pgfpathrectangle{\pgfqpoint{0.578368in}{0.468349in}}{\pgfqpoint{1.278750in}{1.245750in}}%
\pgfusepath{clip}%
\pgfsetbuttcap%
\pgfsetroundjoin%
\pgfsetlinewidth{1.003750pt}%
\definecolor{currentstroke}{rgb}{0.866667,0.517647,0.321569}%
\pgfsetstrokecolor{currentstroke}%
\pgfsetdash{}{0pt}%
\pgfpathmoveto{\pgfqpoint{0.636493in}{0.642116in}}%
\pgfpathlineto{\pgfqpoint{0.636493in}{0.714519in}}%
\pgfusepath{stroke}%
\end{pgfscope}%
\begin{pgfscope}%
\pgfpathrectangle{\pgfqpoint{0.578368in}{0.468349in}}{\pgfqpoint{1.278750in}{1.245750in}}%
\pgfusepath{clip}%
\pgfsetbuttcap%
\pgfsetroundjoin%
\pgfsetlinewidth{1.003750pt}%
\definecolor{currentstroke}{rgb}{0.866667,0.517647,0.321569}%
\pgfsetstrokecolor{currentstroke}%
\pgfsetdash{}{0pt}%
\pgfpathmoveto{\pgfqpoint{0.854929in}{0.527980in}}%
\pgfpathlineto{\pgfqpoint{0.854929in}{0.560899in}}%
\pgfusepath{stroke}%
\end{pgfscope}%
\begin{pgfscope}%
\pgfpathrectangle{\pgfqpoint{0.578368in}{0.468349in}}{\pgfqpoint{1.278750in}{1.245750in}}%
\pgfusepath{clip}%
\pgfsetbuttcap%
\pgfsetroundjoin%
\pgfsetlinewidth{1.003750pt}%
\definecolor{currentstroke}{rgb}{0.866667,0.517647,0.321569}%
\pgfsetstrokecolor{currentstroke}%
\pgfsetdash{}{0pt}%
\pgfpathmoveto{\pgfqpoint{1.073365in}{0.649835in}}%
\pgfpathlineto{\pgfqpoint{1.073365in}{0.700021in}}%
\pgfusepath{stroke}%
\end{pgfscope}%
\begin{pgfscope}%
\pgfpathrectangle{\pgfqpoint{0.578368in}{0.468349in}}{\pgfqpoint{1.278750in}{1.245750in}}%
\pgfusepath{clip}%
\pgfsetbuttcap%
\pgfsetroundjoin%
\pgfsetlinewidth{1.003750pt}%
\definecolor{currentstroke}{rgb}{0.866667,0.517647,0.321569}%
\pgfsetstrokecolor{currentstroke}%
\pgfsetdash{}{0pt}%
\pgfpathmoveto{\pgfqpoint{1.362121in}{0.682292in}}%
\pgfpathlineto{\pgfqpoint{1.362121in}{0.686206in}}%
\pgfusepath{stroke}%
\end{pgfscope}%
\begin{pgfscope}%
\pgfpathrectangle{\pgfqpoint{0.578368in}{0.468349in}}{\pgfqpoint{1.278750in}{1.245750in}}%
\pgfusepath{clip}%
\pgfsetbuttcap%
\pgfsetroundjoin%
\pgfsetlinewidth{1.003750pt}%
\definecolor{currentstroke}{rgb}{0.866667,0.517647,0.321569}%
\pgfsetstrokecolor{currentstroke}%
\pgfsetdash{}{0pt}%
\pgfpathmoveto{\pgfqpoint{1.580557in}{0.794336in}}%
\pgfpathlineto{\pgfqpoint{1.580557in}{0.852087in}}%
\pgfusepath{stroke}%
\end{pgfscope}%
\begin{pgfscope}%
\pgfpathrectangle{\pgfqpoint{0.578368in}{0.468349in}}{\pgfqpoint{1.278750in}{1.245750in}}%
\pgfusepath{clip}%
\pgfsetbuttcap%
\pgfsetroundjoin%
\pgfsetlinewidth{1.003750pt}%
\definecolor{currentstroke}{rgb}{0.866667,0.517647,0.321569}%
\pgfsetstrokecolor{currentstroke}%
\pgfsetdash{}{0pt}%
\pgfpathmoveto{\pgfqpoint{1.798993in}{1.002655in}}%
\pgfpathlineto{\pgfqpoint{1.798993in}{1.026761in}}%
\pgfusepath{stroke}%
\end{pgfscope}%
\begin{pgfscope}%
\pgfpathrectangle{\pgfqpoint{0.578368in}{0.468349in}}{\pgfqpoint{1.278750in}{1.245750in}}%
\pgfusepath{clip}%
\pgfsetbuttcap%
\pgfsetroundjoin%
\pgfsetlinewidth{1.003750pt}%
\definecolor{currentstroke}{rgb}{0.333333,0.658824,0.407843}%
\pgfsetstrokecolor{currentstroke}%
\pgfsetdash{}{0pt}%
\pgfpathmoveto{\pgfqpoint{0.636493in}{0.862385in}}%
\pgfpathlineto{\pgfqpoint{0.636493in}{0.875549in}}%
\pgfusepath{stroke}%
\end{pgfscope}%
\begin{pgfscope}%
\pgfpathrectangle{\pgfqpoint{0.578368in}{0.468349in}}{\pgfqpoint{1.278750in}{1.245750in}}%
\pgfusepath{clip}%
\pgfsetbuttcap%
\pgfsetroundjoin%
\pgfsetlinewidth{1.003750pt}%
\definecolor{currentstroke}{rgb}{0.333333,0.658824,0.407843}%
\pgfsetstrokecolor{currentstroke}%
\pgfsetdash{}{0pt}%
\pgfpathmoveto{\pgfqpoint{0.854929in}{0.913833in}}%
\pgfpathlineto{\pgfqpoint{0.854929in}{0.964758in}}%
\pgfusepath{stroke}%
\end{pgfscope}%
\begin{pgfscope}%
\pgfpathrectangle{\pgfqpoint{0.578368in}{0.468349in}}{\pgfqpoint{1.278750in}{1.245750in}}%
\pgfusepath{clip}%
\pgfsetbuttcap%
\pgfsetroundjoin%
\pgfsetlinewidth{1.003750pt}%
\definecolor{currentstroke}{rgb}{0.333333,0.658824,0.407843}%
\pgfsetstrokecolor{currentstroke}%
\pgfsetdash{}{0pt}%
\pgfpathmoveto{\pgfqpoint{1.073365in}{1.126431in}}%
\pgfpathlineto{\pgfqpoint{1.073365in}{1.157185in}}%
\pgfusepath{stroke}%
\end{pgfscope}%
\begin{pgfscope}%
\pgfpathrectangle{\pgfqpoint{0.578368in}{0.468349in}}{\pgfqpoint{1.278750in}{1.245750in}}%
\pgfusepath{clip}%
\pgfsetbuttcap%
\pgfsetroundjoin%
\pgfsetlinewidth{1.003750pt}%
\definecolor{currentstroke}{rgb}{0.333333,0.658824,0.407843}%
\pgfsetstrokecolor{currentstroke}%
\pgfsetdash{}{0pt}%
\pgfpathmoveto{\pgfqpoint{1.362121in}{1.370755in}}%
\pgfpathlineto{\pgfqpoint{1.362121in}{1.399230in}}%
\pgfusepath{stroke}%
\end{pgfscope}%
\begin{pgfscope}%
\pgfpathrectangle{\pgfqpoint{0.578368in}{0.468349in}}{\pgfqpoint{1.278750in}{1.245750in}}%
\pgfusepath{clip}%
\pgfsetbuttcap%
\pgfsetroundjoin%
\pgfsetlinewidth{1.003750pt}%
\definecolor{currentstroke}{rgb}{0.333333,0.658824,0.407843}%
\pgfsetstrokecolor{currentstroke}%
\pgfsetdash{}{0pt}%
\pgfpathmoveto{\pgfqpoint{1.580557in}{1.544883in}}%
\pgfpathlineto{\pgfqpoint{1.580557in}{1.591149in}}%
\pgfusepath{stroke}%
\end{pgfscope}%
\begin{pgfscope}%
\pgfpathrectangle{\pgfqpoint{0.578368in}{0.468349in}}{\pgfqpoint{1.278750in}{1.245750in}}%
\pgfusepath{clip}%
\pgfsetbuttcap%
\pgfsetroundjoin%
\pgfsetlinewidth{1.003750pt}%
\definecolor{currentstroke}{rgb}{0.333333,0.658824,0.407843}%
\pgfsetstrokecolor{currentstroke}%
\pgfsetdash{}{0pt}%
\pgfpathmoveto{\pgfqpoint{1.798993in}{1.612435in}}%
\pgfpathlineto{\pgfqpoint{1.798993in}{1.657474in}}%
\pgfusepath{stroke}%
\end{pgfscope}%
\begin{pgfscope}%
\pgfpathrectangle{\pgfqpoint{0.578368in}{0.468349in}}{\pgfqpoint{1.278750in}{1.245750in}}%
\pgfusepath{clip}%
\pgfsetbuttcap%
\pgfsetroundjoin%
\pgfsetlinewidth{1.003750pt}%
\definecolor{currentstroke}{rgb}{0.768627,0.305882,0.321569}%
\pgfsetstrokecolor{currentstroke}%
\pgfsetdash{}{0pt}%
\pgfpathmoveto{\pgfqpoint{0.636493in}{1.280415in}}%
\pgfpathlineto{\pgfqpoint{0.636493in}{1.384500in}}%
\pgfusepath{stroke}%
\end{pgfscope}%
\begin{pgfscope}%
\pgfpathrectangle{\pgfqpoint{0.578368in}{0.468349in}}{\pgfqpoint{1.278750in}{1.245750in}}%
\pgfusepath{clip}%
\pgfsetbuttcap%
\pgfsetroundjoin%
\pgfsetlinewidth{1.003750pt}%
\definecolor{currentstroke}{rgb}{0.768627,0.305882,0.321569}%
\pgfsetstrokecolor{currentstroke}%
\pgfsetdash{}{0pt}%
\pgfpathmoveto{\pgfqpoint{0.854929in}{1.225023in}}%
\pgfpathlineto{\pgfqpoint{0.854929in}{1.312792in}}%
\pgfusepath{stroke}%
\end{pgfscope}%
\begin{pgfscope}%
\pgfpathrectangle{\pgfqpoint{0.578368in}{0.468349in}}{\pgfqpoint{1.278750in}{1.245750in}}%
\pgfusepath{clip}%
\pgfsetbuttcap%
\pgfsetroundjoin%
\pgfsetlinewidth{1.003750pt}%
\definecolor{currentstroke}{rgb}{0.768627,0.305882,0.321569}%
\pgfsetstrokecolor{currentstroke}%
\pgfsetdash{}{0pt}%
\pgfpathmoveto{\pgfqpoint{1.073365in}{0.914655in}}%
\pgfpathlineto{\pgfqpoint{1.073365in}{1.002913in}}%
\pgfusepath{stroke}%
\end{pgfscope}%
\begin{pgfscope}%
\pgfpathrectangle{\pgfqpoint{0.578368in}{0.468349in}}{\pgfqpoint{1.278750in}{1.245750in}}%
\pgfusepath{clip}%
\pgfsetbuttcap%
\pgfsetroundjoin%
\pgfsetlinewidth{1.003750pt}%
\definecolor{currentstroke}{rgb}{0.768627,0.305882,0.321569}%
\pgfsetstrokecolor{currentstroke}%
\pgfsetdash{}{0pt}%
\pgfpathmoveto{\pgfqpoint{1.362121in}{0.524974in}}%
\pgfpathlineto{\pgfqpoint{1.362121in}{0.604576in}}%
\pgfusepath{stroke}%
\end{pgfscope}%
\begin{pgfscope}%
\pgfpathrectangle{\pgfqpoint{0.578368in}{0.468349in}}{\pgfqpoint{1.278750in}{1.245750in}}%
\pgfusepath{clip}%
\pgfsetbuttcap%
\pgfsetroundjoin%
\pgfsetlinewidth{1.003750pt}%
\definecolor{currentstroke}{rgb}{0.768627,0.305882,0.321569}%
\pgfsetstrokecolor{currentstroke}%
\pgfsetdash{}{0pt}%
\pgfpathmoveto{\pgfqpoint{1.580557in}{0.532494in}}%
\pgfpathlineto{\pgfqpoint{1.580557in}{0.605529in}}%
\pgfusepath{stroke}%
\end{pgfscope}%
\begin{pgfscope}%
\pgfpathrectangle{\pgfqpoint{0.578368in}{0.468349in}}{\pgfqpoint{1.278750in}{1.245750in}}%
\pgfusepath{clip}%
\pgfsetbuttcap%
\pgfsetroundjoin%
\pgfsetlinewidth{1.003750pt}%
\definecolor{currentstroke}{rgb}{0.768627,0.305882,0.321569}%
\pgfsetstrokecolor{currentstroke}%
\pgfsetdash{}{0pt}%
\pgfpathmoveto{\pgfqpoint{1.798993in}{0.617103in}}%
\pgfpathlineto{\pgfqpoint{1.798993in}{0.709028in}}%
\pgfusepath{stroke}%
\end{pgfscope}%
\begin{pgfscope}%
\pgfpathrectangle{\pgfqpoint{0.578368in}{0.468349in}}{\pgfqpoint{1.278750in}{1.245750in}}%
\pgfusepath{clip}%
\pgfsetbuttcap%
\pgfsetroundjoin%
\pgfsetlinewidth{1.003750pt}%
\definecolor{currentstroke}{rgb}{0.298039,0.447059,0.690196}%
\pgfsetstrokecolor{currentstroke}%
\pgfsetdash{{3.700000pt}{1.600000pt}}{0.000000pt}%
\pgfpathmoveto{\pgfqpoint{0.636493in}{1.351946in}}%
\pgfpathlineto{\pgfqpoint{0.854929in}{1.458710in}}%
\pgfpathlineto{\pgfqpoint{1.073365in}{1.511245in}}%
\pgfpathlineto{\pgfqpoint{1.362121in}{1.525649in}}%
\pgfpathlineto{\pgfqpoint{1.580557in}{1.531580in}}%
\pgfpathlineto{\pgfqpoint{1.798993in}{1.538359in}}%
\pgfusepath{stroke}%
\end{pgfscope}%
\begin{pgfscope}%
\pgfpathrectangle{\pgfqpoint{0.578368in}{0.468349in}}{\pgfqpoint{1.278750in}{1.245750in}}%
\pgfusepath{clip}%
\pgfsetroundcap%
\pgfsetroundjoin%
\pgfsetlinewidth{1.003750pt}%
\definecolor{currentstroke}{rgb}{0.866667,0.517647,0.321569}%
\pgfsetstrokecolor{currentstroke}%
\pgfsetdash{}{0pt}%
\pgfpathmoveto{\pgfqpoint{0.636493in}{0.678317in}}%
\pgfpathlineto{\pgfqpoint{0.854929in}{0.544439in}}%
\pgfpathlineto{\pgfqpoint{1.073365in}{0.674928in}}%
\pgfpathlineto{\pgfqpoint{1.362121in}{0.684249in}}%
\pgfpathlineto{\pgfqpoint{1.580557in}{0.823211in}}%
\pgfpathlineto{\pgfqpoint{1.798993in}{1.014708in}}%
\pgfusepath{stroke}%
\end{pgfscope}%
\begin{pgfscope}%
\pgfpathrectangle{\pgfqpoint{0.578368in}{0.468349in}}{\pgfqpoint{1.278750in}{1.245750in}}%
\pgfusepath{clip}%
\pgfsetbuttcap%
\pgfsetroundjoin%
\pgfsetlinewidth{1.003750pt}%
\definecolor{currentstroke}{rgb}{0.333333,0.658824,0.407843}%
\pgfsetstrokecolor{currentstroke}%
\pgfsetdash{{3.700000pt}{1.600000pt}}{0.000000pt}%
\pgfpathmoveto{\pgfqpoint{0.636493in}{0.868967in}}%
\pgfpathlineto{\pgfqpoint{0.854929in}{0.939296in}}%
\pgfpathlineto{\pgfqpoint{1.073365in}{1.141808in}}%
\pgfpathlineto{\pgfqpoint{1.362121in}{1.384992in}}%
\pgfpathlineto{\pgfqpoint{1.580557in}{1.568016in}}%
\pgfpathlineto{\pgfqpoint{1.798993in}{1.634955in}}%
\pgfusepath{stroke}%
\end{pgfscope}%
\begin{pgfscope}%
\pgfpathrectangle{\pgfqpoint{0.578368in}{0.468349in}}{\pgfqpoint{1.278750in}{1.245750in}}%
\pgfusepath{clip}%
\pgfsetbuttcap%
\pgfsetroundjoin%
\pgfsetlinewidth{1.003750pt}%
\definecolor{currentstroke}{rgb}{0.768627,0.305882,0.321569}%
\pgfsetstrokecolor{currentstroke}%
\pgfsetdash{{3.700000pt}{1.600000pt}}{0.000000pt}%
\pgfpathmoveto{\pgfqpoint{0.636493in}{1.332458in}}%
\pgfpathlineto{\pgfqpoint{0.854929in}{1.268908in}}%
\pgfpathlineto{\pgfqpoint{1.073365in}{0.958784in}}%
\pgfpathlineto{\pgfqpoint{1.362121in}{0.564775in}}%
\pgfpathlineto{\pgfqpoint{1.580557in}{0.569012in}}%
\pgfpathlineto{\pgfqpoint{1.798993in}{0.663065in}}%
\pgfusepath{stroke}%
\end{pgfscope}%
\begin{pgfscope}%
\pgfsetrectcap%
\pgfsetmiterjoin%
\pgfsetlinewidth{0.752812pt}%
\definecolor{currentstroke}{rgb}{0.700000,0.700000,0.700000}%
\pgfsetstrokecolor{currentstroke}%
\pgfsetdash{}{0pt}%
\pgfpathmoveto{\pgfqpoint{0.578368in}{0.468349in}}%
\pgfpathlineto{\pgfqpoint{0.578368in}{1.714099in}}%
\pgfusepath{stroke}%
\end{pgfscope}%
\begin{pgfscope}%
\pgfsetrectcap%
\pgfsetmiterjoin%
\pgfsetlinewidth{0.752812pt}%
\definecolor{currentstroke}{rgb}{0.700000,0.700000,0.700000}%
\pgfsetstrokecolor{currentstroke}%
\pgfsetdash{}{0pt}%
\pgfpathmoveto{\pgfqpoint{1.857118in}{0.468349in}}%
\pgfpathlineto{\pgfqpoint{1.857118in}{1.714099in}}%
\pgfusepath{stroke}%
\end{pgfscope}%
\begin{pgfscope}%
\pgfsetrectcap%
\pgfsetmiterjoin%
\pgfsetlinewidth{0.752812pt}%
\definecolor{currentstroke}{rgb}{0.700000,0.700000,0.700000}%
\pgfsetstrokecolor{currentstroke}%
\pgfsetdash{}{0pt}%
\pgfpathmoveto{\pgfqpoint{0.578368in}{0.468349in}}%
\pgfpathlineto{\pgfqpoint{1.857118in}{0.468349in}}%
\pgfusepath{stroke}%
\end{pgfscope}%
\begin{pgfscope}%
\pgfsetrectcap%
\pgfsetmiterjoin%
\pgfsetlinewidth{0.752812pt}%
\definecolor{currentstroke}{rgb}{0.700000,0.700000,0.700000}%
\pgfsetstrokecolor{currentstroke}%
\pgfsetdash{}{0pt}%
\pgfpathmoveto{\pgfqpoint{0.578368in}{1.714099in}}%
\pgfpathlineto{\pgfqpoint{1.857118in}{1.714099in}}%
\pgfusepath{stroke}%
\end{pgfscope}%
\end{pgfpicture}%
\makeatother%
\endgroup%
}}  \\
    \end{array}\)
  }
  \caption{Influence of the degree of the added nodes on the perturbed accuracy on Cora ML. (a) shows the perturbed accuracy with a budget of changing \(\epsilon=0.1\) edges, and (b) for \(\epsilon=0.25\). The number of nodes is determined as \(\Delta_n = \nicefrac{\epsilon}{\Delta_e}\). We report the mean perturbed accuracy and its three-sigma error over five random seeds. \label{fig:gangnodeeffectiveness}}
\end{figure}

In this section we discuss how solve the attack optimization problem via adding nodes
\begin{equation}\label{eq:gang}
  \max_{\adj^\prime, \features^\prime} \mathcal{L}(f_{\theta}(\adj | \adj^\prime, \features | \features^\prime))
\end{equation}
with the space complexity of \(\mathcal{O}(m)\). With \(\adj | \adj^\prime\) we denote the addition of rows \& columns and with \(\features | \features^\prime\) the concatenation of the respective attributes (\(\adj\) \& \(\features\) are const.). For imperceptibility, we further limit the number of nodes \(\Delta_n\) and their degree \(\Delta_e\).

The attacks add one node (or a small group of nodes) at a time to the sparse adjacency matrix and connect it to every other node with edge weight zero. Subsequently, we perform a constrained gradient-based optimization to determine the best edges with a given budget. We decide to add nodes in an greedy manner. For each new node, we determine the edges in \(s_e\) steps via a greedy FGSM-like procedure (PGD/PRBCD is an alternative). Then, the initial features (randomly sampled) are optimized via PGD (\(s_x\) epochs). In Algorithm~\ref{algo:gang}, we give a formal definition of GANG.

\begin{algorithm}[t]
  \small
  \caption{Greedy Adversarial Node Generation (GANG)}
  \label{algo:gang}
  \begin{algorithmic}[1]
    \STATE {\bfseries Input:} Adj.\ \(\adj\), feat.\ \(\features\), labels\ \(\vy\), GNN \(f_{\theta}(\adj, \features)\), loss \(\mathcal{L}\)
    \STATE {\bfseries Parameter:} budgets \(\Delta_n\) \& \(\Delta_e\), step size \(s_e\), steps features \(s_x\)
    \STATE Initialize empty \(\adj^\prime\) and \(\features^\prime\)
    \FOR{\(k \in \{1, \dots, \Delta_n\}\)}
    \STATE \(\adj^\prime \leftarrow\) concatenate new node to \(\adj^\prime\) (empty row and column)
    \STATE \(\features^\prime \leftarrow\) concatenate \(\features^\prime\) vector \(\tilde{\vx}_k \sim \Pi(\mathcal{N}(0, \sigma_n^2))\)
    \FOR{\(j \in \{0, \dots, \Delta_e / s_e\}\)}
    \STATE \(\hat{\vy} \leftarrow f_{\theta}(\adj | \adj^\prime, \features | \features^\prime)\)
    \STATE \(g \leftarrow \nabla_{\adj^\prime_k} \mathcal{L}(\hat{\vy}, \vy)\) for all nodes where \(\hat{\vy} = \vy\)
    \STATE \(\adj^\prime \leftarrow\) add the top \(s_e\) edges to \(\adj^\prime\) w.r.t.~\(g\)
    \ENDFOR

    %\STATE \(l \leftarrow \nabla \mathcal{L}(f_{\theta}(\tilde{\adj}, \tilde{\features}), \vy)\) for \(i\)-th added node
    %\STATE \(\tilde{\adj} \leftarrow\) remove between \(s_e\) and \(2s_e\) from \(\tilde{\adj}\) according to the lowest \(s_e\) values of \(l\)

    \FOR{\(j \in \{1, \dots, s_x\}\)}
    %\STATE \(\features^\prime \leftarrow \Pi_{\|\features^\prime\|_\infty \le \max(\mX)}(\tilde{\features} + \alpha_x \nabla_{\features^\prime} \mathcal{L}(f_{\theta}(\adj | \adj^\prime, \features | \features^\prime))\)
    \STATE \(\features^\prime \leftarrow \Pi(\tilde{\features} + \alpha_x \nabla_{\features^\prime} \mathcal{L}(f_{\theta}(\adj | \adj^\prime, \features | \features^\prime))\)
    \ENDFOR
    \ENDFOR
  \end{algorithmic}
\end{algorithm}

In Fig.~\ref{fig:gangnodeeffectiveness} we analyze the influence of the degree of the added nodes via GANG. Interestingly, low degree nodes seem to be more effective than high degree nodes. This could be due to the normalization by the square root of the inverse node degree for a GCN~\cite{Kipf2017}. Surprisingly, we find that especially GDC~\cite{Klicpera2019a} (i.e.~personalized PageRank) and a low-rank SVD approximation~\cite{Entezari2020} are effective defenses (prepossessing techniques of the adjacency matrix). SVD is a strong defense against low-degree nodes and personalized page rank is particularly strong against high degree nodes. The recent defense Soft Medoid GDC~\citep{Geisler2020}, seems to be effective regardless of the node degree.

\section{Defense}

We build upon the very recent defence using a robust message passing aggregation that they call Soft Medoid~\citet{Geisler2020}. Our method \emph{Soft Median} performs similarly to Soft Medoid with better complexity w.r.t. the neighborhood size lower memory footprint and enables us to scale bigger graphs. For this we rely on the recent advancements in differentiable sorting~\citet{Prillo2020}.

\textbf{Related work.} Many defenses have been proposed, often observing specific characteristics of some attacks. We can classify those defenses into categories such as (1) preprocessing~\citep{Entezari2020,Wu2019}, (2) robust training~\citep{Xu2019a, Zugner2019a}, and (3) modifications of the architecture~\citep{Zhu2019, Zhang2019a,Geisler2020}. In this section, we improve the Soft Medoid of~\cite{Geisler2020}. They suggest to interpret a GNNs such as:
\begin{equation}\label{eq:mean-how-powerfull}
  \mathbf{h}^{(l)}_v = \sigma^{(l)} \left[ \text{AGG}^{(l)} \left \{ \left( \adj_{vu}, \mathbf{h}^{(l-1)}_u \weight^{(l)} \right), \forall \, u\in \neighbors'(v) \right \} \right]
\end{equation}
with the neighborhood \(\neighbors'(v) = \neighbors(v) \cup v\) including the node itself, some message passing aggregation \(\text{AGG}^{(l)}\) of the \(l\)-th layer, the embeddings \(\mathbf{h}^{(l)}_v\), the normalized message passing matrix \(\adj\), the weights \(\weight^{(l)}\), and activations \( \sigma^{(l)}\). And they propose to use the differentiable robust aggregation for \(\text{AGG}^{(l)}\) which they call Soft Medoid (for simplicity we only present the unweighted version):
\begin{equation}\label{eq:softoutsoftmedoid}
  t_{\text{SoftMedoid}}(\features)
  = \text{s}\left(-\frac{1}{T} \vd \right)^\top \features \text{,~with}~\evd_{v} = \sum_{u\in \neighbors'(v)} \|\features_{u,:} - \features_{v,:}\| 
\end{equation}
where \(s(\vz)_i = \nicefrac{\exp{\left(-T^{-1} \evz_i \right )}}{\sum_{j=1}^n \exp{\left(-T^{-1} \evz_j \right )}}\) is the \(i\)-th element of the softmax with temperature. Note that calculating \(\vd\) is equivalent to a row/column sum over the distance matrix with respect to a nodes neighborhood. Hence this operation has a quadratic complexity w.r.t.\ the neighborhood size and comes with a recognizable memory overhead during training and inference. \todo{Scalable defenses}

\textbf{Our novel, robust, differentiable aggregation.} For improving the previous aggregation we leverage two key facts. First, the Medoid is a multivariate generalization of the median Median and here we look into an alternative that is the dimension-wise Median. Second, we do not need to sort all inputs to obtain the median. This principle can be generalized to soft sorting which is a differentiable relaxation of the sort operation. In summary, we propose a differentiable relaxation of the Median in the space of distances to the dimension-wise Median \(\bar{\vx}\):
\begin{equation}\label{eq:softmedian}
  \begin{aligned}
  t_{\text{SoftMedian}}(\features)
  &= \text{s}\left(-\frac{1}{T} \vd \right)^\top \features \text{,~with}~\evd_{v} = \|\bar{\vx} - \features_{v,:}\| \\
  &= \softout^\top\features \approx \argmin_{\vx' \in \featset} \| \bar{\vx} - \vx' \|,
  \end{aligned}
\end{equation}
Intuitively, our proposed aggregation relies on a robust version of the Mahalanobis distance on a spherical Gaussian. To recover the Mahalanobis distance, we would simply need to replace dimension-wise Median with the sample mean. Equivalently, we use the standardized Euclidean distance. Due to the weighting of the samples with the softmax with temperature this also has connections to the density of a bell shaped distribution.

\textbf{The temperature hyperparameter}. Temperature parameter $T$ controls the steepness of the weight distribution $\hat{\softout}$ between the neighbors and corresponds to the twice the standard deviation in the interpretation as Mahalanobis distance. In the extreme case as $T \to 0$ we recover the point which is closes to the dimension-wise Median (i.e. \(\argmin_{\vx' \in \featset} \| \bar{\vx} - \vx' \|\)). In the other extreme temperature as $T\to\infty$, the Soft Median is equivalent to the sample mean. We observe a similar behavior as~\citet{Geisler2020} and by grid search decide for a temperature value of \(T=0.2\) which is a good compromise between clean accuracy and robustness (similar to \(T=0.5\) for the Soft Medoid).

\textbf{Robustness.} Naturally, the question arises if this estimator is robust since in one extreme it recovers the sample mean which is non to be non-robust. Many metrics have been proposed that capture robustness with different flavours. One of the most widely used properties is the break down point. The (finite-sample) breakdown point captures the minimal fraction \(\epsilon = \nicefrac{m}{n}\) so that the result of the location estimator \(t(\features)\) can be arbitrarily placed~\citep{Donoho1983} (here \(m\) denotes the number of perturbed examples):
%
\begin{equation}\label{eq:breakdown}
  \epsilon^*(t, \features) = \min_{1 \le m \le n} \left \{ \frac{m}{n}: \sup_{\pertm} \|t(\features)-t(\pertm)\| = \infty \right \}
\end{equation}
%
Following Theorem 1 of~\citet{Geisler2020}, our proposed Soft Median comes with the best possible breakdown point as we state formally in \autoref{theorem:softmedianbreakdown}.

\begin{theorem}\label{theorem:softmedianbreakdown}
  Let \(\featset = \{ \mathbf{\mathbf{x}}_1, \dots, \mathbf{\mathbf{x}}_n\} \) be a collection of points in \(\mathbb{R}^d\) with finite coordinates and temperature \(T \in [0, \infty) \). Then the Soft Median location estimator (\autoref{eq:softmedian}) has the finite sample breakdown point of \(\epsilon^*(t_{\text{Soft Median}}, \features) = \nicefrac{1}{n} \lfloor \nicefrac{(n+1)}{2}\rfloor \) (asymptotically \( \lim_{n \to \infty} \epsilon^*(t_{\text{SoftMedian}}, \features) = 0.5 \)).
\end{theorem}

\begin{proof}\label{proof:actual_soft_median}
  Let \( \pertmset \) be decomposable such that \(\pertmset = \pertmset^{(\text{c})} \cup \pertmset^{(\text{p})} \). We now have to find the minimal fraction of outliers \(\epsilon\) for which \newline\(\lim_{\tilde{\evd_{v}} \to \infty} \|t_{\text{SoftMedian}}(\pertm)\| < \infty\) does not hold anymore. According to Eq.~\ref{eq:breakdown}, if we now want to arbitrarily perturb the Soft Median, 
  %the distance to the median \(\tilde{\evd_{v}} = \|\bar{\vx} - \tilde{\features}_{v,:}\| = \|\bar{\vx} - \tilde{\vx}_{v}\|\) 
  we must \(\tilde{\vx_{v}} \to \infty,\,\exists v \in \pertmset^{(\text{p})}\). Next we analyze the influence of this point on Eq.~\ref{eq:softmedian}:
  \[
    \begin{aligned}
      \hat{\softout}_{v} \vx_{v}
      &= \frac{\exp \left\{-\frac{1}{T} \|\bar{\vx} - \tilde{\vx}_{v}\| \right\} \vx_{v}}{\sum\limits_{i \in \pertmset^{(\text{c})}} \exp \left \{-\frac{1}{T} \|\bar{\vx} - \vx_{i}\| \right\} + \sum\limits_{j \in \pertmset^{(\text{p})}} \exp \left \{-\frac{1}{T} \|\bar{\vx} - \vx_{j}\| \right\}} \\
    \end{aligned}
  \]
  Instead of \(\lim_{\|\tilde{\vx}_{v}\| \to \infty} \hat{\softout}_{v} \vx_{v}\), we can equivalently derive the limit for the enumerator and the denominator independently (as long as the denominator does not approach 0 and it is easy to show that the denominator is \(> 0\) and \(\le |\pertmset|\)):
  \[
    \begin{aligned}
      \lim_{\|\tilde{\vx}_{v}\| \to \infty} \exp \left\{-\frac{1}{T} \|\bar{\vx} - \tilde{\vx}_{v}\| \right\} \|\vx_{v}\| = 
      \begin{cases}
        0 \text{, if } \lim_{\|\tilde{\vx}_{v}\| \to \infty} \|\bar{\vx} - \tilde{\vx}_{v}\| = 0\\
        \infty\text{, otherwise}
      \end{cases}
    \end{aligned}
  \]
  Please note that \(\lim_{x \to \infty} x e^{-x/a} = 0\) for \(a \in [0, \infty)\). 
  
  As long as \(\epsilon < 0.5\), we know that for each dimension the perturbed dimension-wise Median must be still within the range of the clean points. Or in other words, the perturbed Median lays within the smallest possible hypercube around the original clean data \(\featset\). As long as \(\epsilon < 0.5\) we have that \(\lim_{\tilde{\vx}_{v}\| \to \infty} \|\bar{\vx} - \tilde{\vx}_{v}\| = 0\). Consequently, \(\|t(\features)-t(\pertm)\| = \infty\) can only be true if \(m \ge n\) for \(T \in [0, \infty)\).
\end{proof}

\todo{Add weighted version}

\textbf{Empirical robustness.} The optimal breakdown point does not necessarily imply that the proposed aggregation is more robust for finite perturbations. In Fig.~\ref{fig:empbiascurve}, we analyze the \(L_2\) distance in the latent space after the first message passing operation for a clean vs.\ perturbed graph. Empirically the Soft Median has a 20\% lower error than the weighted sum (we call it sum since the weights do not sum up to 1). At least in the latent space, the Soft Medoid seems to be more robust. However, this is not consistent with the perturbed accuracy values in Tab.\~ref{tab:losscompare} and~ref{tab:global}. An important and interesting fact is, that similarily to the Soft Medoid

\begin{figure}
  \centering
  \hbox{\hspace{15pt} \resizebox{0.9\linewidth}{!}{%% Creator: Matplotlib, PGF backend
%%
%% To include the figure in your LaTeX document, write
%%   \input{<filename>.pgf}
%%
%% Make sure the required packages are loaded in your preamble
%%   \usepackage{pgf}
%%
%% and, on pdftex
%%   \usepackage[utf8]{inputenc}\DeclareUnicodeCharacter{2212}{-}
%%
%% or, on luatex and xetex
%%   \usepackage{unicode-math}
%%
%% Figures using additional raster images can only be included by \input if
%% they are in the same directory as the main LaTeX file. For loading figures
%% from other directories you can use the `import` package
%%   \usepackage{import}
%%
%% and then include the figures with
%%   \import{<path to file>}{<filename>.pgf}
%%
%% Matplotlib used the following preamble
%%   \usepackage[utf8]{inputenc}
%%   \usepackage[T1]{fontenc}
%%   \usepackage{amsmath}
%%   \newcommand*{\mat}[1]{\boldsymbol{#1}}
%%
\begingroup%
\makeatletter%
\begin{pgfpicture}%
\pgfpathrectangle{\pgfpointorigin}{\pgfqpoint{3.657037in}{0.388266in}}%
\pgfusepath{use as bounding box, clip}%
\begin{pgfscope}%
\pgfsetbuttcap%
\pgfsetmiterjoin%
\definecolor{currentfill}{rgb}{1.000000,1.000000,1.000000}%
\pgfsetfillcolor{currentfill}%
\pgfsetlinewidth{0.000000pt}%
\definecolor{currentstroke}{rgb}{1.000000,1.000000,1.000000}%
\pgfsetstrokecolor{currentstroke}%
\pgfsetstrokeopacity{0.000000}%
\pgfsetdash{}{0pt}%
\pgfpathmoveto{\pgfqpoint{0.000000in}{0.000000in}}%
\pgfpathlineto{\pgfqpoint{3.657037in}{0.000000in}}%
\pgfpathlineto{\pgfqpoint{3.657037in}{0.388266in}}%
\pgfpathlineto{\pgfqpoint{0.000000in}{0.388266in}}%
\pgfpathclose%
\pgfusepath{fill}%
\end{pgfscope}%
\begin{pgfscope}%
\pgfsetbuttcap%
\pgfsetmiterjoin%
\definecolor{currentfill}{rgb}{1.000000,1.000000,1.000000}%
\pgfsetfillcolor{currentfill}%
\pgfsetfillopacity{0.800000}%
\pgfsetlinewidth{1.003750pt}%
\definecolor{currentstroke}{rgb}{0.800000,0.800000,0.800000}%
\pgfsetstrokecolor{currentstroke}%
\pgfsetstrokeopacity{0.800000}%
\pgfsetdash{}{0pt}%
\pgfpathmoveto{\pgfqpoint{0.122222in}{0.100000in}}%
\pgfpathlineto{\pgfqpoint{3.534815in}{0.100000in}}%
\pgfpathquadraticcurveto{\pgfqpoint{3.557037in}{0.100000in}}{\pgfqpoint{3.557037in}{0.122222in}}%
\pgfpathlineto{\pgfqpoint{3.557037in}{0.266044in}}%
\pgfpathquadraticcurveto{\pgfqpoint{3.557037in}{0.288266in}}{\pgfqpoint{3.534815in}{0.288266in}}%
\pgfpathlineto{\pgfqpoint{0.122222in}{0.288266in}}%
\pgfpathquadraticcurveto{\pgfqpoint{0.100000in}{0.288266in}}{\pgfqpoint{0.100000in}{0.266044in}}%
\pgfpathlineto{\pgfqpoint{0.100000in}{0.122222in}}%
\pgfpathquadraticcurveto{\pgfqpoint{0.100000in}{0.100000in}}{\pgfqpoint{0.122222in}{0.100000in}}%
\pgfpathclose%
\pgfusepath{stroke,fill}%
\end{pgfscope}%
\begin{pgfscope}%
\pgfsetroundcap%
\pgfsetroundjoin%
\pgfsetlinewidth{1.003750pt}%
\definecolor{currentstroke}{rgb}{0.298039,0.447059,0.690196}%
\pgfsetstrokecolor{currentstroke}%
\pgfsetdash{}{0pt}%
\pgfpathmoveto{\pgfqpoint{0.144444in}{0.204933in}}%
\pgfpathlineto{\pgfqpoint{0.366667in}{0.204933in}}%
\pgfusepath{stroke}%
\end{pgfscope}%
\begin{pgfscope}%
\definecolor{textcolor}{rgb}{0.150000,0.150000,0.150000}%
\pgfsetstrokecolor{textcolor}%
\pgfsetfillcolor{textcolor}%
\pgftext[x=0.455556in,y=0.166044in,left,base]{\color{textcolor}\rmfamily\fontsize{8.000000}{9.600000}\selectfont Weighted sum}%
\end{pgfscope}%
\begin{pgfscope}%
\pgfsetroundcap%
\pgfsetroundjoin%
\pgfsetlinewidth{1.003750pt}%
\definecolor{currentstroke}{rgb}{0.866667,0.517647,0.321569}%
\pgfsetstrokecolor{currentstroke}%
\pgfsetdash{}{0pt}%
\pgfpathmoveto{\pgfqpoint{1.409428in}{0.204933in}}%
\pgfpathlineto{\pgfqpoint{1.631650in}{0.204933in}}%
\pgfusepath{stroke}%
\end{pgfscope}%
\begin{pgfscope}%
\definecolor{textcolor}{rgb}{0.150000,0.150000,0.150000}%
\pgfsetstrokecolor{textcolor}%
\pgfsetfillcolor{textcolor}%
\pgftext[x=1.720539in,y=0.166044in,left,base]{\color{textcolor}\rmfamily\fontsize{8.000000}{9.600000}\selectfont Soft Medoid}%
\end{pgfscope}%
\begin{pgfscope}%
\pgfsetroundcap%
\pgfsetroundjoin%
\pgfsetlinewidth{1.003750pt}%
\definecolor{currentstroke}{rgb}{0.333333,0.658824,0.407843}%
\pgfsetstrokecolor{currentstroke}%
\pgfsetdash{}{0pt}%
\pgfpathmoveto{\pgfqpoint{2.572121in}{0.204933in}}%
\pgfpathlineto{\pgfqpoint{2.794343in}{0.204933in}}%
\pgfusepath{stroke}%
\end{pgfscope}%
\begin{pgfscope}%
\definecolor{textcolor}{rgb}{0.150000,0.150000,0.150000}%
\pgfsetstrokecolor{textcolor}%
\pgfsetfillcolor{textcolor}%
\pgftext[x=2.883232in,y=0.166044in,left,base]{\color{textcolor}\rmfamily\fontsize{8.000000}{9.600000}\selectfont Soft Median}%
\end{pgfscope}%
\end{pgfpicture}%
\makeatother%
\endgroup%
}}
  \vspace{-14pt}
  \makebox[\linewidth][c]{
  \(\begin{array}{cc}
    \subfloat[]{\resizebox{0.50\linewidth}{!}{%% Creator: Matplotlib, PGF backend
%%
%% To include the figure in your LaTeX document, write
%%   \input{<filename>.pgf}
%%
%% Make sure the required packages are loaded in your preamble
%%   \usepackage{pgf}
%%
%% and, on pdftex
%%   \usepackage[utf8]{inputenc}\DeclareUnicodeCharacter{2212}{-}
%%
%% or, on luatex and xetex
%%   \usepackage{unicode-math}
%%
%% Figures using additional raster images can only be included by \input if
%% they are in the same directory as the main LaTeX file. For loading figures
%% from other directories you can use the `import` package
%%   \usepackage{import}
%%
%% and then include the figures with
%%   \import{<path to file>}{<filename>.pgf}
%%
%% Matplotlib used the following preamble
%%   \usepackage[utf8]{inputenc}
%%   \usepackage[T1]{fontenc}
%%   \usepackage{amsmath}
%%   \newcommand*{\mat}[1]{\boldsymbol{#1}}
%%
\begingroup%
\makeatletter%
\begin{pgfpicture}%
\pgfpathrectangle{\pgfpointorigin}{\pgfqpoint{1.957118in}{1.812363in}}%
\pgfusepath{use as bounding box, clip}%
\begin{pgfscope}%
\pgfsetbuttcap%
\pgfsetmiterjoin%
\definecolor{currentfill}{rgb}{1.000000,1.000000,1.000000}%
\pgfsetfillcolor{currentfill}%
\pgfsetlinewidth{0.000000pt}%
\definecolor{currentstroke}{rgb}{1.000000,1.000000,1.000000}%
\pgfsetstrokecolor{currentstroke}%
\pgfsetstrokeopacity{0.000000}%
\pgfsetdash{}{0pt}%
\pgfpathmoveto{\pgfqpoint{0.000000in}{-0.000000in}}%
\pgfpathlineto{\pgfqpoint{1.957118in}{-0.000000in}}%
\pgfpathlineto{\pgfqpoint{1.957118in}{1.812363in}}%
\pgfpathlineto{\pgfqpoint{0.000000in}{1.812363in}}%
\pgfpathclose%
\pgfusepath{fill}%
\end{pgfscope}%
\begin{pgfscope}%
\pgfsetbuttcap%
\pgfsetmiterjoin%
\definecolor{currentfill}{rgb}{1.000000,1.000000,1.000000}%
\pgfsetfillcolor{currentfill}%
\pgfsetlinewidth{0.000000pt}%
\definecolor{currentstroke}{rgb}{0.000000,0.000000,0.000000}%
\pgfsetstrokecolor{currentstroke}%
\pgfsetstrokeopacity{0.000000}%
\pgfsetdash{}{0pt}%
\pgfpathmoveto{\pgfqpoint{0.578368in}{0.466613in}}%
\pgfpathlineto{\pgfqpoint{1.857118in}{0.466613in}}%
\pgfpathlineto{\pgfqpoint{1.857118in}{1.712363in}}%
\pgfpathlineto{\pgfqpoint{0.578368in}{1.712363in}}%
\pgfpathclose%
\pgfusepath{fill}%
\end{pgfscope}%
\begin{pgfscope}%
\pgfpathrectangle{\pgfqpoint{0.578368in}{0.466613in}}{\pgfqpoint{1.278750in}{1.245750in}}%
\pgfusepath{clip}%
\pgfsetroundcap%
\pgfsetroundjoin%
\pgfsetlinewidth{0.501875pt}%
\definecolor{currentstroke}{rgb}{0.800000,0.800000,0.800000}%
\pgfsetstrokecolor{currentstroke}%
\pgfsetdash{}{0pt}%
\pgfpathmoveto{\pgfqpoint{0.588056in}{0.466613in}}%
\pgfpathlineto{\pgfqpoint{0.588056in}{1.712363in}}%
\pgfusepath{stroke}%
\end{pgfscope}%
\begin{pgfscope}%
\definecolor{textcolor}{rgb}{0.150000,0.150000,0.150000}%
\pgfsetstrokecolor{textcolor}%
\pgfsetfillcolor{textcolor}%
\pgftext[x=0.588056in,y=0.376335in,,top]{\color{textcolor}\rmfamily\fontsize{8.000000}{9.600000}\selectfont \(\displaystyle {0.0}\)}%
\end{pgfscope}%
\begin{pgfscope}%
\pgfpathrectangle{\pgfqpoint{0.578368in}{0.466613in}}{\pgfqpoint{1.278750in}{1.245750in}}%
\pgfusepath{clip}%
\pgfsetroundcap%
\pgfsetroundjoin%
\pgfsetlinewidth{0.501875pt}%
\definecolor{currentstroke}{rgb}{0.800000,0.800000,0.800000}%
\pgfsetstrokecolor{currentstroke}%
\pgfsetdash{}{0pt}%
\pgfpathmoveto{\pgfqpoint{1.072431in}{0.466613in}}%
\pgfpathlineto{\pgfqpoint{1.072431in}{1.712363in}}%
\pgfusepath{stroke}%
\end{pgfscope}%
\begin{pgfscope}%
\definecolor{textcolor}{rgb}{0.150000,0.150000,0.150000}%
\pgfsetstrokecolor{textcolor}%
\pgfsetfillcolor{textcolor}%
\pgftext[x=1.072431in,y=0.376335in,,top]{\color{textcolor}\rmfamily\fontsize{8.000000}{9.600000}\selectfont \(\displaystyle {0.1}\)}%
\end{pgfscope}%
\begin{pgfscope}%
\pgfpathrectangle{\pgfqpoint{0.578368in}{0.466613in}}{\pgfqpoint{1.278750in}{1.245750in}}%
\pgfusepath{clip}%
\pgfsetroundcap%
\pgfsetroundjoin%
\pgfsetlinewidth{0.501875pt}%
\definecolor{currentstroke}{rgb}{0.800000,0.800000,0.800000}%
\pgfsetstrokecolor{currentstroke}%
\pgfsetdash{}{0pt}%
\pgfpathmoveto{\pgfqpoint{1.556806in}{0.466613in}}%
\pgfpathlineto{\pgfqpoint{1.556806in}{1.712363in}}%
\pgfusepath{stroke}%
\end{pgfscope}%
\begin{pgfscope}%
\definecolor{textcolor}{rgb}{0.150000,0.150000,0.150000}%
\pgfsetstrokecolor{textcolor}%
\pgfsetfillcolor{textcolor}%
\pgftext[x=1.556806in,y=0.376335in,,top]{\color{textcolor}\rmfamily\fontsize{8.000000}{9.600000}\selectfont \(\displaystyle {0.2}\)}%
\end{pgfscope}%
\begin{pgfscope}%
\definecolor{textcolor}{rgb}{0.150000,0.150000,0.150000}%
\pgfsetstrokecolor{textcolor}%
\pgfsetfillcolor{textcolor}%
\pgftext[x=1.217743in,y=0.222655in,,top]{\color{textcolor}\rmfamily\fontsize{10.000000}{12.000000}\selectfont Fract. pert. edges \(\displaystyle \epsilon\)}%
\end{pgfscope}%
\begin{pgfscope}%
\pgfpathrectangle{\pgfqpoint{0.578368in}{0.466613in}}{\pgfqpoint{1.278750in}{1.245750in}}%
\pgfusepath{clip}%
\pgfsetroundcap%
\pgfsetroundjoin%
\pgfsetlinewidth{0.501875pt}%
\definecolor{currentstroke}{rgb}{0.800000,0.800000,0.800000}%
\pgfsetstrokecolor{currentstroke}%
\pgfsetdash{}{0pt}%
\pgfpathmoveto{\pgfqpoint{0.578368in}{0.770799in}}%
\pgfpathlineto{\pgfqpoint{1.857118in}{0.770799in}}%
\pgfusepath{stroke}%
\end{pgfscope}%
\begin{pgfscope}%
\definecolor{textcolor}{rgb}{0.150000,0.150000,0.150000}%
\pgfsetstrokecolor{textcolor}%
\pgfsetfillcolor{textcolor}%
\pgftext[x=0.278211in, y=0.732536in, left, base]{\color{textcolor}\rmfamily\fontsize{8.000000}{9.600000}\selectfont \(\displaystyle {0.05}\)}%
\end{pgfscope}%
\begin{pgfscope}%
\pgfpathrectangle{\pgfqpoint{0.578368in}{0.466613in}}{\pgfqpoint{1.278750in}{1.245750in}}%
\pgfusepath{clip}%
\pgfsetroundcap%
\pgfsetroundjoin%
\pgfsetlinewidth{0.501875pt}%
\definecolor{currentstroke}{rgb}{0.800000,0.800000,0.800000}%
\pgfsetstrokecolor{currentstroke}%
\pgfsetdash{}{0pt}%
\pgfpathmoveto{\pgfqpoint{0.578368in}{1.114587in}}%
\pgfpathlineto{\pgfqpoint{1.857118in}{1.114587in}}%
\pgfusepath{stroke}%
\end{pgfscope}%
\begin{pgfscope}%
\definecolor{textcolor}{rgb}{0.150000,0.150000,0.150000}%
\pgfsetstrokecolor{textcolor}%
\pgfsetfillcolor{textcolor}%
\pgftext[x=0.278211in, y=1.076325in, left, base]{\color{textcolor}\rmfamily\fontsize{8.000000}{9.600000}\selectfont \(\displaystyle {0.10}\)}%
\end{pgfscope}%
\begin{pgfscope}%
\pgfpathrectangle{\pgfqpoint{0.578368in}{0.466613in}}{\pgfqpoint{1.278750in}{1.245750in}}%
\pgfusepath{clip}%
\pgfsetroundcap%
\pgfsetroundjoin%
\pgfsetlinewidth{0.501875pt}%
\definecolor{currentstroke}{rgb}{0.800000,0.800000,0.800000}%
\pgfsetstrokecolor{currentstroke}%
\pgfsetdash{}{0pt}%
\pgfpathmoveto{\pgfqpoint{0.578368in}{1.458375in}}%
\pgfpathlineto{\pgfqpoint{1.857118in}{1.458375in}}%
\pgfusepath{stroke}%
\end{pgfscope}%
\begin{pgfscope}%
\definecolor{textcolor}{rgb}{0.150000,0.150000,0.150000}%
\pgfsetstrokecolor{textcolor}%
\pgfsetfillcolor{textcolor}%
\pgftext[x=0.278211in, y=1.420113in, left, base]{\color{textcolor}\rmfamily\fontsize{8.000000}{9.600000}\selectfont \(\displaystyle {0.15}\)}%
\end{pgfscope}%
\begin{pgfscope}%
\definecolor{textcolor}{rgb}{0.150000,0.150000,0.150000}%
\pgfsetstrokecolor{textcolor}%
\pgfsetfillcolor{textcolor}%
\pgftext[x=0.222655in,y=1.089488in,,bottom,rotate=90.000000]{\color{textcolor}\rmfamily\fontsize{10.000000}{12.000000}\selectfont \(\displaystyle L_2\) error}%
\end{pgfscope}%
\begin{pgfscope}%
\pgfpathrectangle{\pgfqpoint{0.578368in}{0.466613in}}{\pgfqpoint{1.278750in}{1.245750in}}%
\pgfusepath{clip}%
\pgfsetroundcap%
\pgfsetroundjoin%
\pgfsetlinewidth{1.003750pt}%
\definecolor{currentstroke}{rgb}{0.298039,0.447059,0.690196}%
\pgfsetstrokecolor{currentstroke}%
\pgfsetdash{}{0pt}%
\pgfpathmoveto{\pgfqpoint{0.636493in}{0.570370in}}%
\pgfpathlineto{\pgfqpoint{0.636493in}{0.575482in}}%
\pgfusepath{stroke}%
\end{pgfscope}%
\begin{pgfscope}%
\pgfpathrectangle{\pgfqpoint{0.578368in}{0.466613in}}{\pgfqpoint{1.278750in}{1.245750in}}%
\pgfusepath{clip}%
\pgfsetroundcap%
\pgfsetroundjoin%
\pgfsetlinewidth{1.003750pt}%
\definecolor{currentstroke}{rgb}{0.298039,0.447059,0.690196}%
\pgfsetstrokecolor{currentstroke}%
\pgfsetdash{}{0pt}%
\pgfpathmoveto{\pgfqpoint{0.830243in}{0.863039in}}%
\pgfpathlineto{\pgfqpoint{0.830243in}{0.885826in}}%
\pgfusepath{stroke}%
\end{pgfscope}%
\begin{pgfscope}%
\pgfpathrectangle{\pgfqpoint{0.578368in}{0.466613in}}{\pgfqpoint{1.278750in}{1.245750in}}%
\pgfusepath{clip}%
\pgfsetroundcap%
\pgfsetroundjoin%
\pgfsetlinewidth{1.003750pt}%
\definecolor{currentstroke}{rgb}{0.298039,0.447059,0.690196}%
\pgfsetstrokecolor{currentstroke}%
\pgfsetdash{}{0pt}%
\pgfpathmoveto{\pgfqpoint{1.072431in}{1.135105in}}%
\pgfpathlineto{\pgfqpoint{1.072431in}{1.139504in}}%
\pgfusepath{stroke}%
\end{pgfscope}%
\begin{pgfscope}%
\pgfpathrectangle{\pgfqpoint{0.578368in}{0.466613in}}{\pgfqpoint{1.278750in}{1.245750in}}%
\pgfusepath{clip}%
\pgfsetroundcap%
\pgfsetroundjoin%
\pgfsetlinewidth{1.003750pt}%
\definecolor{currentstroke}{rgb}{0.298039,0.447059,0.690196}%
\pgfsetstrokecolor{currentstroke}%
\pgfsetdash{}{0pt}%
\pgfpathmoveto{\pgfqpoint{1.798993in}{1.622282in}}%
\pgfpathlineto{\pgfqpoint{1.798993in}{1.655738in}}%
\pgfusepath{stroke}%
\end{pgfscope}%
\begin{pgfscope}%
\pgfpathrectangle{\pgfqpoint{0.578368in}{0.466613in}}{\pgfqpoint{1.278750in}{1.245750in}}%
\pgfusepath{clip}%
\pgfsetroundcap%
\pgfsetroundjoin%
\pgfsetlinewidth{1.003750pt}%
\definecolor{currentstroke}{rgb}{0.866667,0.517647,0.321569}%
\pgfsetstrokecolor{currentstroke}%
\pgfsetdash{}{0pt}%
\pgfpathmoveto{\pgfqpoint{0.636493in}{0.523238in}}%
\pgfpathlineto{\pgfqpoint{0.636493in}{0.524527in}}%
\pgfusepath{stroke}%
\end{pgfscope}%
\begin{pgfscope}%
\pgfpathrectangle{\pgfqpoint{0.578368in}{0.466613in}}{\pgfqpoint{1.278750in}{1.245750in}}%
\pgfusepath{clip}%
\pgfsetroundcap%
\pgfsetroundjoin%
\pgfsetlinewidth{1.003750pt}%
\definecolor{currentstroke}{rgb}{0.866667,0.517647,0.321569}%
\pgfsetstrokecolor{currentstroke}%
\pgfsetdash{}{0pt}%
\pgfpathmoveto{\pgfqpoint{0.830243in}{0.722936in}}%
\pgfpathlineto{\pgfqpoint{0.830243in}{0.740342in}}%
\pgfusepath{stroke}%
\end{pgfscope}%
\begin{pgfscope}%
\pgfpathrectangle{\pgfqpoint{0.578368in}{0.466613in}}{\pgfqpoint{1.278750in}{1.245750in}}%
\pgfusepath{clip}%
\pgfsetroundcap%
\pgfsetroundjoin%
\pgfsetlinewidth{1.003750pt}%
\definecolor{currentstroke}{rgb}{0.866667,0.517647,0.321569}%
\pgfsetstrokecolor{currentstroke}%
\pgfsetdash{}{0pt}%
\pgfpathmoveto{\pgfqpoint{1.072431in}{0.899463in}}%
\pgfpathlineto{\pgfqpoint{1.072431in}{0.920713in}}%
\pgfusepath{stroke}%
\end{pgfscope}%
\begin{pgfscope}%
\pgfpathrectangle{\pgfqpoint{0.578368in}{0.466613in}}{\pgfqpoint{1.278750in}{1.245750in}}%
\pgfusepath{clip}%
\pgfsetroundcap%
\pgfsetroundjoin%
\pgfsetlinewidth{1.003750pt}%
\definecolor{currentstroke}{rgb}{0.866667,0.517647,0.321569}%
\pgfsetstrokecolor{currentstroke}%
\pgfsetdash{}{0pt}%
\pgfpathmoveto{\pgfqpoint{1.798993in}{1.212782in}}%
\pgfpathlineto{\pgfqpoint{1.798993in}{1.261488in}}%
\pgfusepath{stroke}%
\end{pgfscope}%
\begin{pgfscope}%
\pgfpathrectangle{\pgfqpoint{0.578368in}{0.466613in}}{\pgfqpoint{1.278750in}{1.245750in}}%
\pgfusepath{clip}%
\pgfsetroundcap%
\pgfsetroundjoin%
\pgfsetlinewidth{1.003750pt}%
\definecolor{currentstroke}{rgb}{0.333333,0.658824,0.407843}%
\pgfsetstrokecolor{currentstroke}%
\pgfsetdash{}{0pt}%
\pgfpathmoveto{\pgfqpoint{0.636493in}{0.537379in}}%
\pgfpathlineto{\pgfqpoint{0.636493in}{0.544782in}}%
\pgfusepath{stroke}%
\end{pgfscope}%
\begin{pgfscope}%
\pgfpathrectangle{\pgfqpoint{0.578368in}{0.466613in}}{\pgfqpoint{1.278750in}{1.245750in}}%
\pgfusepath{clip}%
\pgfsetroundcap%
\pgfsetroundjoin%
\pgfsetlinewidth{1.003750pt}%
\definecolor{currentstroke}{rgb}{0.333333,0.658824,0.407843}%
\pgfsetstrokecolor{currentstroke}%
\pgfsetdash{}{0pt}%
\pgfpathmoveto{\pgfqpoint{0.830243in}{0.775619in}}%
\pgfpathlineto{\pgfqpoint{0.830243in}{0.792146in}}%
\pgfusepath{stroke}%
\end{pgfscope}%
\begin{pgfscope}%
\pgfpathrectangle{\pgfqpoint{0.578368in}{0.466613in}}{\pgfqpoint{1.278750in}{1.245750in}}%
\pgfusepath{clip}%
\pgfsetroundcap%
\pgfsetroundjoin%
\pgfsetlinewidth{1.003750pt}%
\definecolor{currentstroke}{rgb}{0.333333,0.658824,0.407843}%
\pgfsetstrokecolor{currentstroke}%
\pgfsetdash{}{0pt}%
\pgfpathmoveto{\pgfqpoint{1.072431in}{0.981660in}}%
\pgfpathlineto{\pgfqpoint{1.072431in}{1.008607in}}%
\pgfusepath{stroke}%
\end{pgfscope}%
\begin{pgfscope}%
\pgfpathrectangle{\pgfqpoint{0.578368in}{0.466613in}}{\pgfqpoint{1.278750in}{1.245750in}}%
\pgfusepath{clip}%
\pgfsetroundcap%
\pgfsetroundjoin%
\pgfsetlinewidth{1.003750pt}%
\definecolor{currentstroke}{rgb}{0.333333,0.658824,0.407843}%
\pgfsetstrokecolor{currentstroke}%
\pgfsetdash{}{0pt}%
\pgfpathmoveto{\pgfqpoint{1.798993in}{1.344181in}}%
\pgfpathlineto{\pgfqpoint{1.798993in}{1.416183in}}%
\pgfusepath{stroke}%
\end{pgfscope}%
\begin{pgfscope}%
\pgfpathrectangle{\pgfqpoint{0.578368in}{0.466613in}}{\pgfqpoint{1.278750in}{1.245750in}}%
\pgfusepath{clip}%
\pgfsetroundcap%
\pgfsetroundjoin%
\pgfsetlinewidth{1.003750pt}%
\definecolor{currentstroke}{rgb}{0.298039,0.447059,0.690196}%
\pgfsetstrokecolor{currentstroke}%
\pgfsetdash{}{0pt}%
\pgfpathmoveto{\pgfqpoint{0.636493in}{0.572135in}}%
\pgfpathlineto{\pgfqpoint{0.830243in}{0.875764in}}%
\pgfpathlineto{\pgfqpoint{1.072431in}{1.137837in}}%
\pgfpathlineto{\pgfqpoint{1.798993in}{1.635409in}}%
\pgfusepath{stroke}%
\end{pgfscope}%
\begin{pgfscope}%
\pgfpathrectangle{\pgfqpoint{0.578368in}{0.466613in}}{\pgfqpoint{1.278750in}{1.245750in}}%
\pgfusepath{clip}%
\pgfsetroundcap%
\pgfsetroundjoin%
\pgfsetlinewidth{1.003750pt}%
\definecolor{currentstroke}{rgb}{0.866667,0.517647,0.321569}%
\pgfsetstrokecolor{currentstroke}%
\pgfsetdash{}{0pt}%
\pgfpathmoveto{\pgfqpoint{0.636493in}{0.523710in}}%
\pgfpathlineto{\pgfqpoint{0.830243in}{0.731651in}}%
\pgfpathlineto{\pgfqpoint{1.072431in}{0.913372in}}%
\pgfpathlineto{\pgfqpoint{1.798993in}{1.240219in}}%
\pgfusepath{stroke}%
\end{pgfscope}%
\begin{pgfscope}%
\pgfpathrectangle{\pgfqpoint{0.578368in}{0.466613in}}{\pgfqpoint{1.278750in}{1.245750in}}%
\pgfusepath{clip}%
\pgfsetroundcap%
\pgfsetroundjoin%
\pgfsetlinewidth{1.003750pt}%
\definecolor{currentstroke}{rgb}{0.333333,0.658824,0.407843}%
\pgfsetstrokecolor{currentstroke}%
\pgfsetdash{}{0pt}%
\pgfpathmoveto{\pgfqpoint{0.636493in}{0.540822in}}%
\pgfpathlineto{\pgfqpoint{0.830243in}{0.784954in}}%
\pgfpathlineto{\pgfqpoint{1.072431in}{0.996274in}}%
\pgfpathlineto{\pgfqpoint{1.798993in}{1.379264in}}%
\pgfusepath{stroke}%
\end{pgfscope}%
\begin{pgfscope}%
\pgfsetrectcap%
\pgfsetmiterjoin%
\pgfsetlinewidth{0.752812pt}%
\definecolor{currentstroke}{rgb}{0.700000,0.700000,0.700000}%
\pgfsetstrokecolor{currentstroke}%
\pgfsetdash{}{0pt}%
\pgfpathmoveto{\pgfqpoint{0.578368in}{0.466613in}}%
\pgfpathlineto{\pgfqpoint{0.578368in}{1.712363in}}%
\pgfusepath{stroke}%
\end{pgfscope}%
\begin{pgfscope}%
\pgfsetrectcap%
\pgfsetmiterjoin%
\pgfsetlinewidth{0.752812pt}%
\definecolor{currentstroke}{rgb}{0.700000,0.700000,0.700000}%
\pgfsetstrokecolor{currentstroke}%
\pgfsetdash{}{0pt}%
\pgfpathmoveto{\pgfqpoint{1.857118in}{0.466613in}}%
\pgfpathlineto{\pgfqpoint{1.857118in}{1.712363in}}%
\pgfusepath{stroke}%
\end{pgfscope}%
\begin{pgfscope}%
\pgfsetrectcap%
\pgfsetmiterjoin%
\pgfsetlinewidth{0.752812pt}%
\definecolor{currentstroke}{rgb}{0.700000,0.700000,0.700000}%
\pgfsetstrokecolor{currentstroke}%
\pgfsetdash{}{0pt}%
\pgfpathmoveto{\pgfqpoint{0.578368in}{0.466613in}}%
\pgfpathlineto{\pgfqpoint{1.857118in}{0.466613in}}%
\pgfusepath{stroke}%
\end{pgfscope}%
\begin{pgfscope}%
\pgfsetrectcap%
\pgfsetmiterjoin%
\pgfsetlinewidth{0.752812pt}%
\definecolor{currentstroke}{rgb}{0.700000,0.700000,0.700000}%
\pgfsetstrokecolor{currentstroke}%
\pgfsetdash{}{0pt}%
\pgfpathmoveto{\pgfqpoint{0.578368in}{1.712363in}}%
\pgfpathlineto{\pgfqpoint{1.857118in}{1.712363in}}%
\pgfusepath{stroke}%
\end{pgfscope}%
\end{pgfpicture}%
\makeatother%
\endgroup%
}} & 
    \subfloat[]{\resizebox{0.5\linewidth}{!}{%% Creator: Matplotlib, PGF backend
%%
%% To include the figure in your LaTeX document, write
%%   \input{<filename>.pgf}
%%
%% Make sure the required packages are loaded in your preamble
%%   \usepackage{pgf}
%%
%% and, on pdftex
%%   \usepackage[utf8]{inputenc}\DeclareUnicodeCharacter{2212}{-}
%%
%% or, on luatex and xetex
%%   \usepackage{unicode-math}
%%
%% Figures using additional raster images can only be included by \input if
%% they are in the same directory as the main LaTeX file. For loading figures
%% from other directories you can use the `import` package
%%   \usepackage{import}
%%
%% and then include the figures with
%%   \import{<path to file>}{<filename>.pgf}
%%
%% Matplotlib used the following preamble
%%   \usepackage[utf8]{inputenc}
%%   \usepackage[T1]{fontenc}
%%   \usepackage{amsmath}
%%   \newcommand*{\mat}[1]{\boldsymbol{#1}}
%%
\begingroup%
\makeatletter%
\begin{pgfpicture}%
\pgfpathrectangle{\pgfpointorigin}{\pgfqpoint{1.898089in}{1.812363in}}%
\pgfusepath{use as bounding box, clip}%
\begin{pgfscope}%
\pgfsetbuttcap%
\pgfsetmiterjoin%
\definecolor{currentfill}{rgb}{1.000000,1.000000,1.000000}%
\pgfsetfillcolor{currentfill}%
\pgfsetlinewidth{0.000000pt}%
\definecolor{currentstroke}{rgb}{1.000000,1.000000,1.000000}%
\pgfsetstrokecolor{currentstroke}%
\pgfsetstrokeopacity{0.000000}%
\pgfsetdash{}{0pt}%
\pgfpathmoveto{\pgfqpoint{0.000000in}{-0.000000in}}%
\pgfpathlineto{\pgfqpoint{1.898089in}{-0.000000in}}%
\pgfpathlineto{\pgfqpoint{1.898089in}{1.812363in}}%
\pgfpathlineto{\pgfqpoint{0.000000in}{1.812363in}}%
\pgfpathclose%
\pgfusepath{fill}%
\end{pgfscope}%
\begin{pgfscope}%
\pgfsetbuttcap%
\pgfsetmiterjoin%
\definecolor{currentfill}{rgb}{1.000000,1.000000,1.000000}%
\pgfsetfillcolor{currentfill}%
\pgfsetlinewidth{0.000000pt}%
\definecolor{currentstroke}{rgb}{0.000000,0.000000,0.000000}%
\pgfsetstrokecolor{currentstroke}%
\pgfsetstrokeopacity{0.000000}%
\pgfsetdash{}{0pt}%
\pgfpathmoveto{\pgfqpoint{0.519339in}{0.466613in}}%
\pgfpathlineto{\pgfqpoint{1.798089in}{0.466613in}}%
\pgfpathlineto{\pgfqpoint{1.798089in}{1.712363in}}%
\pgfpathlineto{\pgfqpoint{0.519339in}{1.712363in}}%
\pgfpathclose%
\pgfusepath{fill}%
\end{pgfscope}%
\begin{pgfscope}%
\pgfpathrectangle{\pgfqpoint{0.519339in}{0.466613in}}{\pgfqpoint{1.278750in}{1.245750in}}%
\pgfusepath{clip}%
\pgfsetroundcap%
\pgfsetroundjoin%
\pgfsetlinewidth{0.501875pt}%
\definecolor{currentstroke}{rgb}{0.800000,0.800000,0.800000}%
\pgfsetstrokecolor{currentstroke}%
\pgfsetdash{}{0pt}%
\pgfpathmoveto{\pgfqpoint{0.529027in}{0.466613in}}%
\pgfpathlineto{\pgfqpoint{0.529027in}{1.712363in}}%
\pgfusepath{stroke}%
\end{pgfscope}%
\begin{pgfscope}%
\definecolor{textcolor}{rgb}{0.150000,0.150000,0.150000}%
\pgfsetstrokecolor{textcolor}%
\pgfsetfillcolor{textcolor}%
\pgftext[x=0.529027in,y=0.376335in,,top]{\color{textcolor}\rmfamily\fontsize{8.000000}{9.600000}\selectfont \(\displaystyle {0.0}\)}%
\end{pgfscope}%
\begin{pgfscope}%
\pgfpathrectangle{\pgfqpoint{0.519339in}{0.466613in}}{\pgfqpoint{1.278750in}{1.245750in}}%
\pgfusepath{clip}%
\pgfsetroundcap%
\pgfsetroundjoin%
\pgfsetlinewidth{0.501875pt}%
\definecolor{currentstroke}{rgb}{0.800000,0.800000,0.800000}%
\pgfsetstrokecolor{currentstroke}%
\pgfsetdash{}{0pt}%
\pgfpathmoveto{\pgfqpoint{1.013402in}{0.466613in}}%
\pgfpathlineto{\pgfqpoint{1.013402in}{1.712363in}}%
\pgfusepath{stroke}%
\end{pgfscope}%
\begin{pgfscope}%
\definecolor{textcolor}{rgb}{0.150000,0.150000,0.150000}%
\pgfsetstrokecolor{textcolor}%
\pgfsetfillcolor{textcolor}%
\pgftext[x=1.013402in,y=0.376335in,,top]{\color{textcolor}\rmfamily\fontsize{8.000000}{9.600000}\selectfont \(\displaystyle {0.1}\)}%
\end{pgfscope}%
\begin{pgfscope}%
\pgfpathrectangle{\pgfqpoint{0.519339in}{0.466613in}}{\pgfqpoint{1.278750in}{1.245750in}}%
\pgfusepath{clip}%
\pgfsetroundcap%
\pgfsetroundjoin%
\pgfsetlinewidth{0.501875pt}%
\definecolor{currentstroke}{rgb}{0.800000,0.800000,0.800000}%
\pgfsetstrokecolor{currentstroke}%
\pgfsetdash{}{0pt}%
\pgfpathmoveto{\pgfqpoint{1.497777in}{0.466613in}}%
\pgfpathlineto{\pgfqpoint{1.497777in}{1.712363in}}%
\pgfusepath{stroke}%
\end{pgfscope}%
\begin{pgfscope}%
\definecolor{textcolor}{rgb}{0.150000,0.150000,0.150000}%
\pgfsetstrokecolor{textcolor}%
\pgfsetfillcolor{textcolor}%
\pgftext[x=1.497777in,y=0.376335in,,top]{\color{textcolor}\rmfamily\fontsize{8.000000}{9.600000}\selectfont \(\displaystyle {0.2}\)}%
\end{pgfscope}%
\begin{pgfscope}%
\definecolor{textcolor}{rgb}{0.150000,0.150000,0.150000}%
\pgfsetstrokecolor{textcolor}%
\pgfsetfillcolor{textcolor}%
\pgftext[x=1.158714in,y=0.222655in,,top]{\color{textcolor}\rmfamily\fontsize{10.000000}{12.000000}\selectfont Fract. pert. edges \(\displaystyle \epsilon\)}%
\end{pgfscope}%
\begin{pgfscope}%
\pgfpathrectangle{\pgfqpoint{0.519339in}{0.466613in}}{\pgfqpoint{1.278750in}{1.245750in}}%
\pgfusepath{clip}%
\pgfsetroundcap%
\pgfsetroundjoin%
\pgfsetlinewidth{0.501875pt}%
\definecolor{currentstroke}{rgb}{0.800000,0.800000,0.800000}%
\pgfsetstrokecolor{currentstroke}%
\pgfsetdash{}{0pt}%
\pgfpathmoveto{\pgfqpoint{0.519339in}{0.690210in}}%
\pgfpathlineto{\pgfqpoint{1.798089in}{0.690210in}}%
\pgfusepath{stroke}%
\end{pgfscope}%
\begin{pgfscope}%
\definecolor{textcolor}{rgb}{0.150000,0.150000,0.150000}%
\pgfsetstrokecolor{textcolor}%
\pgfsetfillcolor{textcolor}%
\pgftext[x=0.278211in, y=0.651948in, left, base]{\color{textcolor}\rmfamily\fontsize{8.000000}{9.600000}\selectfont \(\displaystyle {0.7}\)}%
\end{pgfscope}%
\begin{pgfscope}%
\pgfpathrectangle{\pgfqpoint{0.519339in}{0.466613in}}{\pgfqpoint{1.278750in}{1.245750in}}%
\pgfusepath{clip}%
\pgfsetroundcap%
\pgfsetroundjoin%
\pgfsetlinewidth{0.501875pt}%
\definecolor{currentstroke}{rgb}{0.800000,0.800000,0.800000}%
\pgfsetstrokecolor{currentstroke}%
\pgfsetdash{}{0pt}%
\pgfpathmoveto{\pgfqpoint{0.519339in}{1.012053in}}%
\pgfpathlineto{\pgfqpoint{1.798089in}{1.012053in}}%
\pgfusepath{stroke}%
\end{pgfscope}%
\begin{pgfscope}%
\definecolor{textcolor}{rgb}{0.150000,0.150000,0.150000}%
\pgfsetstrokecolor{textcolor}%
\pgfsetfillcolor{textcolor}%
\pgftext[x=0.278211in, y=0.973790in, left, base]{\color{textcolor}\rmfamily\fontsize{8.000000}{9.600000}\selectfont \(\displaystyle {0.8}\)}%
\end{pgfscope}%
\begin{pgfscope}%
\pgfpathrectangle{\pgfqpoint{0.519339in}{0.466613in}}{\pgfqpoint{1.278750in}{1.245750in}}%
\pgfusepath{clip}%
\pgfsetroundcap%
\pgfsetroundjoin%
\pgfsetlinewidth{0.501875pt}%
\definecolor{currentstroke}{rgb}{0.800000,0.800000,0.800000}%
\pgfsetstrokecolor{currentstroke}%
\pgfsetdash{}{0pt}%
\pgfpathmoveto{\pgfqpoint{0.519339in}{1.333895in}}%
\pgfpathlineto{\pgfqpoint{1.798089in}{1.333895in}}%
\pgfusepath{stroke}%
\end{pgfscope}%
\begin{pgfscope}%
\definecolor{textcolor}{rgb}{0.150000,0.150000,0.150000}%
\pgfsetstrokecolor{textcolor}%
\pgfsetfillcolor{textcolor}%
\pgftext[x=0.278211in, y=1.295633in, left, base]{\color{textcolor}\rmfamily\fontsize{8.000000}{9.600000}\selectfont \(\displaystyle {0.9}\)}%
\end{pgfscope}%
\begin{pgfscope}%
\pgfpathrectangle{\pgfqpoint{0.519339in}{0.466613in}}{\pgfqpoint{1.278750in}{1.245750in}}%
\pgfusepath{clip}%
\pgfsetroundcap%
\pgfsetroundjoin%
\pgfsetlinewidth{0.501875pt}%
\definecolor{currentstroke}{rgb}{0.800000,0.800000,0.800000}%
\pgfsetstrokecolor{currentstroke}%
\pgfsetdash{}{0pt}%
\pgfpathmoveto{\pgfqpoint{0.519339in}{1.655738in}}%
\pgfpathlineto{\pgfqpoint{1.798089in}{1.655738in}}%
\pgfusepath{stroke}%
\end{pgfscope}%
\begin{pgfscope}%
\definecolor{textcolor}{rgb}{0.150000,0.150000,0.150000}%
\pgfsetstrokecolor{textcolor}%
\pgfsetfillcolor{textcolor}%
\pgftext[x=0.278211in, y=1.617475in, left, base]{\color{textcolor}\rmfamily\fontsize{8.000000}{9.600000}\selectfont \(\displaystyle {1.0}\)}%
\end{pgfscope}%
\begin{pgfscope}%
\definecolor{textcolor}{rgb}{0.150000,0.150000,0.150000}%
\pgfsetstrokecolor{textcolor}%
\pgfsetfillcolor{textcolor}%
\pgftext[x=0.222655in,y=1.089488in,,bottom,rotate=90.000000]{\color{textcolor}\rmfamily\fontsize{10.000000}{12.000000}\selectfont Relative \(\displaystyle L_2\) error}%
\end{pgfscope}%
\begin{pgfscope}%
\pgfpathrectangle{\pgfqpoint{0.519339in}{0.466613in}}{\pgfqpoint{1.278750in}{1.245750in}}%
\pgfusepath{clip}%
\pgfsetroundcap%
\pgfsetroundjoin%
\pgfsetlinewidth{1.003750pt}%
\definecolor{currentstroke}{rgb}{0.298039,0.447059,0.690196}%
\pgfsetstrokecolor{currentstroke}%
\pgfsetdash{}{0pt}%
\pgfpathmoveto{\pgfqpoint{0.577464in}{1.655738in}}%
\pgfpathlineto{\pgfqpoint{0.577464in}{1.655738in}}%
\pgfusepath{stroke}%
\end{pgfscope}%
\begin{pgfscope}%
\pgfpathrectangle{\pgfqpoint{0.519339in}{0.466613in}}{\pgfqpoint{1.278750in}{1.245750in}}%
\pgfusepath{clip}%
\pgfsetroundcap%
\pgfsetroundjoin%
\pgfsetlinewidth{1.003750pt}%
\definecolor{currentstroke}{rgb}{0.298039,0.447059,0.690196}%
\pgfsetstrokecolor{currentstroke}%
\pgfsetdash{}{0pt}%
\pgfpathmoveto{\pgfqpoint{0.771214in}{1.655738in}}%
\pgfpathlineto{\pgfqpoint{0.771214in}{1.655738in}}%
\pgfusepath{stroke}%
\end{pgfscope}%
\begin{pgfscope}%
\pgfpathrectangle{\pgfqpoint{0.519339in}{0.466613in}}{\pgfqpoint{1.278750in}{1.245750in}}%
\pgfusepath{clip}%
\pgfsetroundcap%
\pgfsetroundjoin%
\pgfsetlinewidth{1.003750pt}%
\definecolor{currentstroke}{rgb}{0.298039,0.447059,0.690196}%
\pgfsetstrokecolor{currentstroke}%
\pgfsetdash{}{0pt}%
\pgfpathmoveto{\pgfqpoint{1.013402in}{1.655738in}}%
\pgfpathlineto{\pgfqpoint{1.013402in}{1.655738in}}%
\pgfusepath{stroke}%
\end{pgfscope}%
\begin{pgfscope}%
\pgfpathrectangle{\pgfqpoint{0.519339in}{0.466613in}}{\pgfqpoint{1.278750in}{1.245750in}}%
\pgfusepath{clip}%
\pgfsetroundcap%
\pgfsetroundjoin%
\pgfsetlinewidth{1.003750pt}%
\definecolor{currentstroke}{rgb}{0.298039,0.447059,0.690196}%
\pgfsetstrokecolor{currentstroke}%
\pgfsetdash{}{0pt}%
\pgfpathmoveto{\pgfqpoint{1.739964in}{1.655738in}}%
\pgfpathlineto{\pgfqpoint{1.739964in}{1.655738in}}%
\pgfusepath{stroke}%
\end{pgfscope}%
\begin{pgfscope}%
\pgfpathrectangle{\pgfqpoint{0.519339in}{0.466613in}}{\pgfqpoint{1.278750in}{1.245750in}}%
\pgfusepath{clip}%
\pgfsetroundcap%
\pgfsetroundjoin%
\pgfsetlinewidth{1.003750pt}%
\definecolor{currentstroke}{rgb}{0.866667,0.517647,0.321569}%
\pgfsetstrokecolor{currentstroke}%
\pgfsetdash{}{0pt}%
\pgfpathmoveto{\pgfqpoint{0.577464in}{0.523238in}}%
\pgfpathlineto{\pgfqpoint{0.577464in}{0.629377in}}%
\pgfusepath{stroke}%
\end{pgfscope}%
\begin{pgfscope}%
\pgfpathrectangle{\pgfqpoint{0.519339in}{0.466613in}}{\pgfqpoint{1.278750in}{1.245750in}}%
\pgfusepath{clip}%
\pgfsetroundcap%
\pgfsetroundjoin%
\pgfsetlinewidth{1.003750pt}%
\definecolor{currentstroke}{rgb}{0.866667,0.517647,0.321569}%
\pgfsetstrokecolor{currentstroke}%
\pgfsetdash{}{0pt}%
\pgfpathmoveto{\pgfqpoint{0.771214in}{0.547147in}}%
\pgfpathlineto{\pgfqpoint{0.771214in}{0.686118in}}%
\pgfusepath{stroke}%
\end{pgfscope}%
\begin{pgfscope}%
\pgfpathrectangle{\pgfqpoint{0.519339in}{0.466613in}}{\pgfqpoint{1.278750in}{1.245750in}}%
\pgfusepath{clip}%
\pgfsetroundcap%
\pgfsetroundjoin%
\pgfsetlinewidth{1.003750pt}%
\definecolor{currentstroke}{rgb}{0.866667,0.517647,0.321569}%
\pgfsetstrokecolor{currentstroke}%
\pgfsetdash{}{0pt}%
\pgfpathmoveto{\pgfqpoint{1.013402in}{0.573246in}}%
\pgfpathlineto{\pgfqpoint{1.013402in}{0.677772in}}%
\pgfusepath{stroke}%
\end{pgfscope}%
\begin{pgfscope}%
\pgfpathrectangle{\pgfqpoint{0.519339in}{0.466613in}}{\pgfqpoint{1.278750in}{1.245750in}}%
\pgfusepath{clip}%
\pgfsetroundcap%
\pgfsetroundjoin%
\pgfsetlinewidth{1.003750pt}%
\definecolor{currentstroke}{rgb}{0.866667,0.517647,0.321569}%
\pgfsetstrokecolor{currentstroke}%
\pgfsetdash{}{0pt}%
\pgfpathmoveto{\pgfqpoint{1.739964in}{0.553107in}}%
\pgfpathlineto{\pgfqpoint{1.739964in}{0.632703in}}%
\pgfusepath{stroke}%
\end{pgfscope}%
\begin{pgfscope}%
\pgfpathrectangle{\pgfqpoint{0.519339in}{0.466613in}}{\pgfqpoint{1.278750in}{1.245750in}}%
\pgfusepath{clip}%
\pgfsetroundcap%
\pgfsetroundjoin%
\pgfsetlinewidth{1.003750pt}%
\definecolor{currentstroke}{rgb}{0.333333,0.658824,0.407843}%
\pgfsetstrokecolor{currentstroke}%
\pgfsetdash{}{0pt}%
\pgfpathmoveto{\pgfqpoint{0.577464in}{0.893214in}}%
\pgfpathlineto{\pgfqpoint{0.577464in}{1.084683in}}%
\pgfusepath{stroke}%
\end{pgfscope}%
\begin{pgfscope}%
\pgfpathrectangle{\pgfqpoint{0.519339in}{0.466613in}}{\pgfqpoint{1.278750in}{1.245750in}}%
\pgfusepath{clip}%
\pgfsetroundcap%
\pgfsetroundjoin%
\pgfsetlinewidth{1.003750pt}%
\definecolor{currentstroke}{rgb}{0.333333,0.658824,0.407843}%
\pgfsetstrokecolor{currentstroke}%
\pgfsetdash{}{0pt}%
\pgfpathmoveto{\pgfqpoint{0.771214in}{0.922756in}}%
\pgfpathlineto{\pgfqpoint{0.771214in}{1.132461in}}%
\pgfusepath{stroke}%
\end{pgfscope}%
\begin{pgfscope}%
\pgfpathrectangle{\pgfqpoint{0.519339in}{0.466613in}}{\pgfqpoint{1.278750in}{1.245750in}}%
\pgfusepath{clip}%
\pgfsetroundcap%
\pgfsetroundjoin%
\pgfsetlinewidth{1.003750pt}%
\definecolor{currentstroke}{rgb}{0.333333,0.658824,0.407843}%
\pgfsetstrokecolor{currentstroke}%
\pgfsetdash{}{0pt}%
\pgfpathmoveto{\pgfqpoint{1.013402in}{0.944855in}}%
\pgfpathlineto{\pgfqpoint{1.013402in}{1.126460in}}%
\pgfusepath{stroke}%
\end{pgfscope}%
\begin{pgfscope}%
\pgfpathrectangle{\pgfqpoint{0.519339in}{0.466613in}}{\pgfqpoint{1.278750in}{1.245750in}}%
\pgfusepath{clip}%
\pgfsetroundcap%
\pgfsetroundjoin%
\pgfsetlinewidth{1.003750pt}%
\definecolor{currentstroke}{rgb}{0.333333,0.658824,0.407843}%
\pgfsetstrokecolor{currentstroke}%
\pgfsetdash{}{0pt}%
\pgfpathmoveto{\pgfqpoint{1.739964in}{0.906916in}}%
\pgfpathlineto{\pgfqpoint{1.739964in}{1.087649in}}%
\pgfusepath{stroke}%
\end{pgfscope}%
\begin{pgfscope}%
\pgfpathrectangle{\pgfqpoint{0.519339in}{0.466613in}}{\pgfqpoint{1.278750in}{1.245750in}}%
\pgfusepath{clip}%
\pgfsetroundcap%
\pgfsetroundjoin%
\pgfsetlinewidth{1.003750pt}%
\definecolor{currentstroke}{rgb}{0.298039,0.447059,0.690196}%
\pgfsetstrokecolor{currentstroke}%
\pgfsetdash{}{0pt}%
\pgfpathmoveto{\pgfqpoint{0.577464in}{1.655738in}}%
\pgfpathlineto{\pgfqpoint{0.771214in}{1.655738in}}%
\pgfpathlineto{\pgfqpoint{1.013402in}{1.655738in}}%
\pgfpathlineto{\pgfqpoint{1.739964in}{1.655738in}}%
\pgfusepath{stroke}%
\end{pgfscope}%
\begin{pgfscope}%
\pgfpathrectangle{\pgfqpoint{0.519339in}{0.466613in}}{\pgfqpoint{1.278750in}{1.245750in}}%
\pgfusepath{clip}%
\pgfsetroundcap%
\pgfsetroundjoin%
\pgfsetlinewidth{1.003750pt}%
\definecolor{currentstroke}{rgb}{0.866667,0.517647,0.321569}%
\pgfsetstrokecolor{currentstroke}%
\pgfsetdash{}{0pt}%
\pgfpathmoveto{\pgfqpoint{0.577464in}{0.582526in}}%
\pgfpathlineto{\pgfqpoint{0.771214in}{0.622829in}}%
\pgfpathlineto{\pgfqpoint{1.013402in}{0.639483in}}%
\pgfpathlineto{\pgfqpoint{1.739964in}{0.602961in}}%
\pgfusepath{stroke}%
\end{pgfscope}%
\begin{pgfscope}%
\pgfpathrectangle{\pgfqpoint{0.519339in}{0.466613in}}{\pgfqpoint{1.278750in}{1.245750in}}%
\pgfusepath{clip}%
\pgfsetroundcap%
\pgfsetroundjoin%
\pgfsetlinewidth{1.003750pt}%
\definecolor{currentstroke}{rgb}{0.333333,0.658824,0.407843}%
\pgfsetstrokecolor{currentstroke}%
\pgfsetdash{}{0pt}%
\pgfpathmoveto{\pgfqpoint{0.577464in}{0.962211in}}%
\pgfpathlineto{\pgfqpoint{0.771214in}{1.006136in}}%
\pgfpathlineto{\pgfqpoint{1.013402in}{1.014992in}}%
\pgfpathlineto{\pgfqpoint{1.739964in}{0.973774in}}%
\pgfusepath{stroke}%
\end{pgfscope}%
\begin{pgfscope}%
\pgfsetrectcap%
\pgfsetmiterjoin%
\pgfsetlinewidth{0.752812pt}%
\definecolor{currentstroke}{rgb}{0.700000,0.700000,0.700000}%
\pgfsetstrokecolor{currentstroke}%
\pgfsetdash{}{0pt}%
\pgfpathmoveto{\pgfqpoint{0.519339in}{0.466613in}}%
\pgfpathlineto{\pgfqpoint{0.519339in}{1.712363in}}%
\pgfusepath{stroke}%
\end{pgfscope}%
\begin{pgfscope}%
\pgfsetrectcap%
\pgfsetmiterjoin%
\pgfsetlinewidth{0.752812pt}%
\definecolor{currentstroke}{rgb}{0.700000,0.700000,0.700000}%
\pgfsetstrokecolor{currentstroke}%
\pgfsetdash{}{0pt}%
\pgfpathmoveto{\pgfqpoint{1.798089in}{0.466613in}}%
\pgfpathlineto{\pgfqpoint{1.798089in}{1.712363in}}%
\pgfusepath{stroke}%
\end{pgfscope}%
\begin{pgfscope}%
\pgfsetrectcap%
\pgfsetmiterjoin%
\pgfsetlinewidth{0.752812pt}%
\definecolor{currentstroke}{rgb}{0.700000,0.700000,0.700000}%
\pgfsetstrokecolor{currentstroke}%
\pgfsetdash{}{0pt}%
\pgfpathmoveto{\pgfqpoint{0.519339in}{0.466613in}}%
\pgfpathlineto{\pgfqpoint{1.798089in}{0.466613in}}%
\pgfusepath{stroke}%
\end{pgfscope}%
\begin{pgfscope}%
\pgfsetrectcap%
\pgfsetmiterjoin%
\pgfsetlinewidth{0.752812pt}%
\definecolor{currentstroke}{rgb}{0.700000,0.700000,0.700000}%
\pgfsetstrokecolor{currentstroke}%
\pgfsetdash{}{0pt}%
\pgfpathmoveto{\pgfqpoint{0.519339in}{1.712363in}}%
\pgfpathlineto{\pgfqpoint{1.798089in}{1.712363in}}%
\pgfusepath{stroke}%
\end{pgfscope}%
\end{pgfpicture}%
\makeatother%
\endgroup%
}} \\
  \end{array}\)
  }
  \caption{Empirical bias \(B(\epsilon)\) for the second layer of a GDC~\citep{Klicpera2019a} network. (a) shows the absolute bias for a PGD attack, with loss (5) of Sec.~\ref{sec:ceisbad}, and a budget of changing \(\epsilon=0.25\) edges. (b) shows the relative bias over the weighted mean of a GDC. We use for all estimator a temperature of \(T=0.2\)\label{fig:empbiascurve}}
\end{figure}

\section{Empirical Evaluation}\label{sec:empirical}

\begin{table*}
  \centering
  \caption{Perturbed accuracy comparing the conventional losses with our loss. We report the mean over three different seeds. \(\epsilon\) denotes the fraction of edges perturbed (relative to the clean graph). We use random split with 20 nodes per class. For each architecture and budget we embolden the better loss. For details about the set up we refer to Section~\ref{sec:empirical}.}
  \label{tab:losscompare}
  \resizebox{\linewidth}{!}{
    \begin{tabular}{lcl|ccccccc|ccccccc}
    \toprule
                               &      &     & \multicolumn{7}{c|}{\textbf{Cora ML~\citep{Bojchevski2018}}} & \multicolumn{7}{c}{\textbf{Citeseer~\citep{McCallum2000}}} \\
                               \rotatebox{90}{\textbf{Attack}} & \makecell{\textbf{Frac.}\\\textbf{edges}\\\(\boldsymbol{\epsilon}\)} & \textbf{Loss} &                 \makecell{Vanilla\\GCN} & \makecell{Vanilla\\GDC} & \makecell{SVD\\GCN} & \makecell{Jaccard\\GCN} &  \makecell{RGCN} & \makecell{Soft\\Medoid\\GDC} & \makecell{Soft\\Median\\GDC} &                \makecell{Vanilla\\GCN} & \makecell{Vanilla\\GDC} & \makecell{SVD\\GCN} & \makecell{Jaccard\\GCN} &  \makecell{RGCN} & \makecell{Soft\\Medoid\\GDC} & \makecell{Soft\\Median\\GDC} \\
    \midrule
    \midrule
    \multirow{20}{*}{\rotatebox{90}{\textbf{greedy FGSM}}} & \multirow{5}{*}{0.01} & CE &                  0.8087 &                  0.8144 &              0.7576 &                  0.8066 &           0.7864 &                       0.8061 &                       0.8092 &                  0.7052 &                  0.7000 &              0.6401 &                  0.7091 &           0.6389 &                       0.7045 &                       0.7061 \\
                                 &      & CW &                  0.8079 &                  0.8145 &              0.7573 &                  0.8041 &           0.7843 &                       0.8149 &                       0.8167 &                  0.6966 &                  0.6916 &              0.6394 &                  0.7018 &           0.6312 &                       0.7070 &                       0.7077 \\
                                 &      & MCE &         \textbf{0.7859} &         \textbf{0.7953} &              0.7573 &         \textbf{0.7871} &  \textbf{0.7730} &                       0.8070 &                       0.8078 &         \textbf{0.6850} &         \textbf{0.6832} &     \textbf{0.6376} &         \textbf{0.6959} &  \textbf{0.6305} &                       0.7037 &                       0.7046 \\
                                 &      & SCE &                  0.8124 &                  0.8204 &              0.7580 &                  0.8084 &           0.7872 &                       0.8157 &                       0.8170 &                  0.7009 &                  0.6975 &              0.6387 &                  0.7046 &           0.6328 &                       0.7075 &                       0.7089 \\
                                 &      & tanh Margin &                  0.7920 &                  0.7953 &     \textbf{0.7567} &                  0.7905 &           0.7756 &              \textbf{0.8033} &              \textbf{0.8025} &                  0.6895 &                  0.6856 &              0.6383 &                  0.6970 &           0.6335 &              \textbf{0.7029} &              \textbf{0.7021} \\
    \cline{2-17}
                                 & \multirow{5}{*}{0.05} & CE &                  0.7577 &                  0.7586 &              0.7414 &                  0.7605 &           0.7428 &                       0.7722 &                       0.7722 &                  0.6693 &                  0.6594 &              0.6244 &                  0.6799 &           0.6077 &              \textbf{0.6852} &                       0.6831 \\
                                 &      & CW &                  0.7378 &                  0.7531 &              0.7445 &                  0.7465 &           0.7318 &                       0.8054 &                       0.8016 &                  0.6332 &                  0.6385 &              0.6225 &                  0.6560 &           0.5818 &                       0.7029 &                       0.7039 \\
                                 &      & MCE &         \textbf{0.6908} &                  0.7045 &              0.7426 &                  0.7116 &           0.7004 &                       0.7885 &                       0.7850 &         \textbf{0.6064} &         \textbf{0.6036} &              0.6212 &                  0.6410 &  \textbf{0.5815} &                       0.6952 &                       0.6911 \\
                                 &      & SCE &                  0.7623 &                  0.7750 &              0.7444 &                  0.7647 &           0.7466 &                       0.8083 &                       0.8062 &                  0.6519 &                  0.6565 &              0.6205 &                  0.6693 &           0.5879 &                       0.7061 &                       0.7059 \\
                                 &      & tanh Margin &                  0.7011 &         \textbf{0.7007} &     \textbf{0.7389} &         \textbf{0.7071} &  \textbf{0.6993} &              \textbf{0.7713} &              \textbf{0.7639} &                  0.6082 &                  0.6094 &     \textbf{0.6168} &         \textbf{0.6348} &           0.5927 &                       0.6889 &              \textbf{0.6802} \\
    \cline{2-17}
                                 & \multirow{5}{*}{0.10} & CE &                  0.7188 &                  0.7179 &              0.7153 &                  0.7221 &           0.7080 &                       0.7426 &                       0.7410 &                  0.6316 &                  0.6223 &              0.6025 &                  0.6476 &           0.5766 &              \textbf{0.6665} &              \textbf{0.6590} \\
                                 &      & CW &                  0.6751 &                  0.7001 &              0.7265 &                  0.7005 &           0.6874 &                       0.7924 &                       0.7874 &                  0.5692 &                  0.5763 &              0.5884 &                  0.6112 &           0.5348 &                       0.6984 &                       0.6968 \\
                                 &      & MCE &         \textbf{0.6086} &                  0.6385 &              0.7215 &                  0.6441 &           0.6387 &                       0.7773 &                       0.7693 &                  0.5335 &         \textbf{0.5346} &              0.5991 &                  0.5922 &  \textbf{0.5323} &                       0.6893 &                       0.6791 \\
                                 &      & SCE &                  0.7042 &                  0.7235 &              0.7271 &                  0.7166 &           0.7065 &                       0.7947 &                       0.7881 &                  0.6004 &                  0.6096 &              0.5866 &                  0.6298 &           0.5455 &                       0.7012 &                       0.6998 \\
                                 &      & tanh Margin &                  0.6337 &         \textbf{0.6350} &     \textbf{0.7107} &         \textbf{0.6430} &  \textbf{0.6370} &              \textbf{0.7415} &              \textbf{0.7320} &         \textbf{0.5323} &                  0.5417 &     \textbf{0.5848} &         \textbf{0.5763} &           0.5492 &                       0.6797 &                       0.6668 \\
    \cline{2-17}
                                 & \multirow{5}{*}{0.25} & CE &                  0.6353 &                  0.6391 &              0.6399 &                  0.6427 &           0.6312 &              \textbf{0.6785} &                       0.6746 &                  0.5401 &                  0.5330 &     \textbf{0.3626} &                  0.5729 &           0.5130 &              \textbf{0.6201} &              \textbf{0.6073} \\
                                 &      & CW &                  0.4987 &                  0.5788 &              0.6344 &                  0.5635 &           0.5613 &                       0.7738 &                       0.7635 &                  0.4431 &                  0.4804 &              0.4998 &                  0.5355 &           0.4369 &                       0.6895 &                       0.6807 \\
                                 &      & MCE &         \textbf{0.4599} &                  0.5275 &              0.6374 &                  0.5245 &  \textbf{0.5101} &                       0.7632 &                       0.7524 &                  0.3898 &                  0.4128 &              0.5036 &                  0.4973 &  \textbf{0.4244} &                       0.6820 &                       0.6674 \\
                                 &      & SCE &                  0.5364 &                  0.5931 &     \textbf{0.4644} &                  0.5833 &           0.5785 &                       0.7679 &                       0.7569 &                  0.4558 &                  0.4783 &              0.4854 &                  0.5373 &           0.4449 &                       0.6879 &                       0.6768 \\
                                 &      & tanh Margin &                  0.5128 &         \textbf{0.5165} &              0.6195 &         \textbf{0.5233} &           0.5173 &                       0.6841 &              \textbf{0.6744} &         \textbf{0.3897} &         \textbf{0.4116} &              0.4973 &         \textbf{0.4590} &           0.4360 &                       0.6560 &                       0.6369 \\
    \cline{1-17}
    \cline{2-17}
    \multirow{20}{*}{\rotatebox{90}{\textbf{PGD}}} & \multirow{5}{*}{0.01} & CE &                  0.8047 &                  0.8107 &              0.7568 &                  0.8020 &           0.7848 &                       0.8053 &                       0.8069 &                  0.7032 &                  0.6980 &     \textbf{0.6378} &                  0.7057 &           0.6373 &              \textbf{0.7030} &              \textbf{0.7041} \\
                                 &      & CW &                  0.8124 &                  0.8204 &     \textbf{0.7563} &                  0.8087 &           0.7904 &                       0.8142 &                       0.8161 &                  0.7011 &                  0.6975 &              0.6378 &                  0.7053 &  \textbf{0.6339} &                       0.7064 &                       0.7086 \\
                                 &      & MCE &                  0.8245 &                  0.8314 &              0.7606 &                  0.8191 &           0.8000 &                       0.8167 &                       0.8195 &                  0.7121 &                  0.7094 &              0.6408 &                  0.7137 &           0.6456 &                       0.7073 &                       0.7093 \\
                                 &      & SCE &                  0.8245 &                  0.8314 &              0.7606 &                  0.8191 &           0.8000 &                       0.8167 &                       0.8195 &                  0.7121 &                  0.7094 &              0.6408 &                  0.7137 &           0.6456 &                       0.7073 &                       0.7093 \\
                                 &      & tanh Margin &         \textbf{0.7892} &         \textbf{0.7960} &              0.7567 &         \textbf{0.7903} &  \textbf{0.7758} &              \textbf{0.8050} &              \textbf{0.8053} &         \textbf{0.6873} &         \textbf{0.6840} &              0.6383 &         \textbf{0.6957} &           0.6355 &                       0.7045 &                       0.7045 \\
    \cline{2-17}
                                 & \multirow{5}{*}{0.05} & CE &                  0.7547 &                  0.7555 &              0.7348 &                  0.7564 &           0.7390 &              \textbf{0.7735} &              \textbf{0.7742} &                  0.6595 &                  0.6551 &              0.6144 &                  0.6718 &           0.6068 &              \textbf{0.6852} &              \textbf{0.6829} \\
                                 &      & CW &                  0.7719 &                  0.7838 &              0.7369 &                  0.7764 &           0.7617 &                       0.8083 &                       0.8074 &                  0.6677 &                  0.6613 &              0.4257 &                  0.6797 &  \textbf{0.5998} &                       0.7039 &                       0.7050 \\
                                 &      & MCE &                  0.8245 &                  0.8314 &              0.7606 &                  0.8191 &           0.8000 &                       0.8167 &                       0.8195 &                  0.7121 &                  0.7094 &              0.6408 &                  0.7137 &           0.6456 &                       0.7073 &                       0.7093 \\
                                 &      & SCE &                  0.8245 &                  0.8314 &              0.7606 &                  0.8191 &           0.8000 &                       0.8167 &                       0.8195 &                  0.7121 &                  0.7094 &              0.6408 &                  0.7137 &           0.6456 &                       0.7073 &                       0.7093 \\
                                 &      & tanh Margin &         \textbf{0.7065} &         \textbf{0.7165} &     \textbf{0.7253} &         \textbf{0.7153} &  \textbf{0.7155} &                       0.7816 &                       0.7762 &         \textbf{0.6264} &         \textbf{0.6205} &     \textbf{0.4137} &         \textbf{0.6419} &           0.6020 &                       0.6884 &                       0.6868 \\
    \cline{2-17}
                                 & \multirow{5}{*}{0.10} & CE &                  0.7053 &                  0.7045 &              0.6957 &                  0.7080 &           0.6991 &              \textbf{0.7432} &              \textbf{0.7402} &                  0.6184 &                  0.6102 &              0.5925 &                  0.6392 &           0.5836 &              \textbf{0.6708} &              \textbf{0.6617} \\
                                 &      & CW &                  0.7411 &                  0.7540 &              0.7043 &                  0.7472 &           0.7325 &                       0.7988 &                       0.7950 &                  0.6440 &                  0.6376 &              0.4009 &                  0.6640 &           0.5733 &                       0.7012 &                       0.6989 \\
                                 &      & MCE &                  0.8245 &                  0.8314 &              0.7606 &                  0.8191 &           0.8000 &                       0.8167 &                       0.8195 &                  0.7121 &                  0.7094 &              0.6408 &                  0.7137 &           0.6456 &                       0.7073 &                       0.7093 \\
                                 &      & SCE &                  0.8245 &                  0.8314 &              0.7606 &                  0.8191 &           0.8000 &                       0.8167 &                       0.8195 &                  0.7121 &                  0.7094 &              0.6408 &                  0.7137 &           0.6456 &                       0.7073 &                       0.7093 \\
                                 &      & tanh Margin &         \textbf{0.6379} &         \textbf{0.6481} &     \textbf{0.6905} &         \textbf{0.6577} &  \textbf{0.6560} &                       0.7603 &                       0.7532 &         \textbf{0.5592} &         \textbf{0.5620} &     \textbf{0.3947} &         \textbf{0.5932} &  \textbf{0.5554} &                       0.6838 &                       0.6713 \\
    \cline{2-17}
                                 & \multirow{5}{*}{0.25} & CE &                  0.5901 &                  0.5953 &              0.6028 &                  0.6011 &           0.5942 &              \textbf{0.6830} &              \textbf{0.6775} &                  0.5260 &                  0.5175 &              0.4904 &                  0.5542 &           0.5039 &              \textbf{0.6283} &              \textbf{0.6086} \\
                                 &      & CW &                  0.6819 &                  0.7198 &              0.6144 &                  0.7078 &           0.6856 &                       0.7848 &                       0.7846 &                  0.5982 &                  0.5907 &              0.4902 &                  0.6357 &           0.5217 &                       0.6900 &                       0.6836 \\
                                 &      & MCE &                  0.8245 &                  0.8314 &              0.7606 &                  0.8191 &           0.8000 &                       0.8167 &                       0.8195 &                  0.7121 &                  0.7094 &              0.6408 &                  0.7137 &           0.6456 &                       0.7073 &                       0.7093 \\
                                 &      & SCE &                  0.8245 &                  0.8314 &              0.7606 &                  0.8191 &           0.8000 &                       0.8167 &                       0.8195 &                  0.7121 &                  0.7094 &              0.6408 &                  0.7137 &           0.6456 &                       0.7073 &                       0.7093 \\
                                 &      & tanh Margin &         \textbf{0.4993} &         \textbf{0.5254} &     \textbf{0.5791} &         \textbf{0.5332} &  \textbf{0.5311} &                       0.7246 &                       0.7175 &         \textbf{0.4283} &         \textbf{0.4342} &     \textbf{0.4713} &         \textbf{0.5043} &  \textbf{0.4599} &                       0.6674 &                       0.6487 \\
    \bottomrule
  \end{tabular}
  }
\end{table*}


\begin{table*}
  \centering
  \caption{Perturbed accuracy for the proposed attacks (see Sections~\ref{sec:attackkdd}-\ref{sec:prbcd}) and baselines on all datasets (see Table~\ref{tab:datasets}). \(\epsilon\) denotes the fraction of edges perturbed (relative to the clean graph). The last column contains the clean accuracy. As this a work-in-progress report, the experiments for the defenses on the large datasets are due and on Products we did not optimize the hyperparameters for GANG. For each architecture we italicize the strongest attack where \(\epsilon=0.05\), underline where \(\epsilon=0.1\), and embolden where \(\epsilon=0.25\). From an attack perspective, a lower perturbed accuracy is better. We rerun the experiments with three different seeds. For OGB we use the provided data splits and otherwise we use random split with 20 nodes per class.}
  \label{tab:global}
  \resizebox{\linewidth}{!}{
    \begin{tabular}{ll|cccc|cccc|cccc|cccc|cccc|cccc|c}
      \toprule
                                         & \textbf{Attack}                       & \multicolumn{4}{c|}{\textbf{DICE}} & \multicolumn{4}{c|}{\textbf{GANG (ours)}} & \multicolumn{4}{c|}{\textbf{greedy FGSM}} & \multicolumn{4}{c|}{\textbf{GR-BCD (ours)}} & \multicolumn{4}{c|}{\textbf{PGD}} & \multicolumn{4}{c|}{\textbf{PR-BCD (ours)}} & \textbf{Accuracy}                                                                                                                                                                                                                                                                                   \\
                                         & Frac. edges \(\boldsymbol{\epsilon}\) & 0.01                              & 0.05                                     & 0.1                                      & 0.25                                       & 0.01                             & 0.05                                       & 0.1               & 0.25  & 0.01  & 0.05           & 0.1               & 0.25           & 0.01  & 0.05           & 0.1               & 0.25           & 0.01  & 0.05           & 0.1               & 0.25           & 0.01  & 0.05           & 0.1               & 0.25 &         \\
                                         \toprule
                                         & \textbf{Attack} & \multicolumn{4}{l}{\textbf{DICE}} & \multicolumn{4}{l}{\textbf{GANG (ours)}} & \multicolumn{4}{l}{\textbf{greedy FGSM}} & \multicolumn{4}{l}{\textbf{GR-BCD (ours)}} & \multicolumn{4}{l}{\textbf{PGD}} & \multicolumn{4}{l}{\textbf{PR-BCD (ours)}} & \textbf{Accuracy} \\
                                         & Frac. edges \(\boldsymbol{\epsilon}\) &          0.01 &   0.05 &    0.1 &   0.25 &                 0.01 &            0.05 &                0.1 &   0.25 &                 0.01 &            0.05 &                0.1 &            0.25 &                   0.01 &            0.05 &                0.1 &            0.25 &         0.01 &   0.05 &    0.1 &   0.25 &                   0.01 &            0.05 &                0.1 & \multicolumn{2}{l}{0.25} \\
           & \textbf{Architecture} &               &        &        &        &                      &                 &                    &        &                      &                 &                    &                 &                        &                 &                    &                 &              &        &        &        &                        &                 &                    &                 &                   \\
       \midrule
       \multirow{7}{*}{\rotatebox{90}{\textbf{Cora ML}}} & Vanilla GCN &         0.822 &  0.813 &  0.803 &  0.765 &                0.809 &           0.766 &              0.732 &  0.658 &                0.792 &           0.701 &              0.634 &           0.513 &                  0.790 &  \textit{0.699} &  \underline{0.627} &           0.506 &        0.825 &  0.825 &  0.825 &  0.825 &                  0.790 &           0.711 &              0.641 &  \textbf{0.498} &             0.825 \\
                                         & Vanilla GDC &         0.829 &  0.820 &  0.807 &  0.774 &                0.822 &           0.788 &              0.762 &  0.712 &                0.795 &  \textit{0.701} &  \underline{0.635} &  \textbf{0.516} &                  0.798 &           0.709 &              0.640 &           0.542 &        0.831 &  0.831 &  0.831 &  0.831 &                  0.794 &           0.717 &              0.649 &           0.526 &             0.831 \\
                                         & SVD GCN &         0.758 &  0.754 &  0.741 &  0.696 &                0.770 &           0.764 &              0.760 &  0.722 &                0.757 &           0.739 &              0.711 &           0.619 &                  0.757 &           0.743 &              0.722 &           0.633 &        0.761 &  0.761 &  0.761 &  0.761 &                  0.757 &  \textit{0.733} &  \underline{0.691} &  \textbf{0.570} &             0.761 \\
                                         & Jaccard GCN &         0.817 &  0.810 &  0.801 &  0.769 &                0.808 &           0.788 &              0.768 &  0.737 &                0.791 &  \textit{0.707} &  \underline{0.643} &  \textbf{0.523} &                  0.789 &           0.716 &              0.655 &           0.557 &        0.819 &  0.819 &  0.819 &  0.819 &                  0.790 &           0.724 &              0.660 &           0.532 &             0.819 \\
                                         & RGCN &         0.799 &  0.794 &  0.785 &  0.756 &                0.732 &           0.701 &              0.674 &  0.603 &                0.776 &  \textit{0.699} &  \underline{0.637} &  \textbf{0.517} &                  0.774 &           0.706 &              0.643 &           0.529 &        0.800 &  0.800 &  0.800 &  0.800 &                  0.777 &           0.712 &              0.658 &           0.528 &             0.800 \\
                                         & Soft Medoid GDC &         0.816 &  0.813 &  0.806 &  0.793 &                0.772 &  \textit{0.769} &              0.765 &  0.755 &                0.803 &           0.771 &  \underline{0.742} &  \textbf{0.684} &                  0.806 &           0.788 &              0.775 &           0.755 &        0.817 &  0.817 &  0.817 &  0.817 &                  0.806 &           0.780 &              0.758 &           0.725 &             0.817 \\
                                         & Soft Median GDC &         0.819 &  0.814 &  0.811 &  0.797 &                0.791 &           0.789 &              0.785 &  0.773 &                0.803 &  \textit{0.764} &  \underline{0.732} &  \textbf{0.674} &                  0.808 &           0.782 &              0.767 &           0.742 &        0.819 &  0.819 &  0.819 &  0.819 &                  0.805 &           0.776 &              0.750 &           0.711 &             0.819 \\
       \cline{1-27}
       \multirow{7}{*}{\rotatebox{90}{\textbf{Citeseer}}} & Vanilla GCN &         0.710 &  0.702 &  0.691 &  0.663 &                0.700 &           0.675 &              0.644 &  0.593 &                0.689 &           0.608 &              0.532 &           0.390 &                  0.682 &  \textit{0.602} &  \underline{0.528} &  \textbf{0.368} &        0.712 &  0.712 &  0.712 &  0.712 &                  0.685 &           0.608 &              0.544 &           0.410 &             0.712 \\
                                         & Vanilla GDC &         0.706 &  0.694 &  0.682 &  0.649 &                0.701 &           0.686 &              0.662 &  0.630 &                0.686 &           0.609 &              0.542 &           0.412 &                  0.681 &  \textit{0.602} &  \underline{0.537} &           0.407 &        0.709 &  0.709 &  0.709 &  0.709 &                  0.679 &           0.611 &              0.539 &  \textbf{0.405} &             0.709 \\
                                         & SVD GCN &         0.637 &  0.625 &  0.606 &  0.566 &                0.639 &           0.627 &  \underline{0.420} &  0.539 &                0.638 &           0.617 &              0.585 &           0.497 &                  0.639 &           0.624 &              0.593 &           0.484 &        0.641 &  0.641 &  0.641 &  0.641 &                  0.635 &  \textit{0.608} &              0.562 &  \textbf{0.464} &             0.641 \\
                                         & Jaccard GCN &         0.712 &  0.707 &  0.699 &  0.676 &                0.710 &           0.705 &              0.698 &  0.691 &                0.697 &  \textit{0.635} &  \underline{0.576} &  \textbf{0.459} &                  0.694 &           0.636 &              0.589 &           0.494 &        0.714 &  0.714 &  0.714 &  0.714 &                  0.693 &           0.637 &              0.590 &           0.481 &             0.714 \\
                                         & RGCN &         0.643 &  0.634 &  0.624 &  0.597 &                0.641 &           0.624 &              0.601 &  0.543 &                0.634 &           0.593 &  \underline{0.549} &  \textbf{0.436} &                  0.634 &  \textit{0.590} &              0.550 &           0.452 &        0.646 &  0.646 &  0.646 &  0.646 &                  0.637 &           0.599 &              0.560 &           0.462 &             0.646 \\
                                         & Soft Medoid GDC &         0.707 &  0.703 &  0.701 &  0.694 &                0.706 &           0.704 &              0.700 &  0.695 &                0.703 &  \textit{0.689} &  \underline{0.680} &  \textbf{0.656} &                  0.702 &           0.694 &              0.689 &           0.678 &        0.707 &  0.707 &  0.707 &  0.707 &                  0.702 &           0.691 &              0.682 &           0.661 &             0.707 \\
                                         & Soft Median GDC &         0.709 &  0.708 &  0.701 &  0.693 &                0.710 &           0.707 &              0.703 &  0.697 &                0.702 &  \textit{0.680} &  \underline{0.667} &  \textbf{0.637} &                  0.704 &           0.695 &              0.684 &           0.663 &        0.709 &  0.709 &  0.709 &  0.709 &                  0.702 &           0.685 &              0.669 &           0.643 &             0.709 \\
       \cline{1-27}
       \multirow{4}{*}{\rotatebox{90}{\textbf{PubMed}}} & Vanilla GCN &         0.780 &  0.767 &  0.753 &  0.712 &                0.758 &           0.695 &              0.631 &  0.493 &                    - &               - &                  - &               - &                  0.752 &  \textit{0.654} &  \underline{0.575} &  \textbf{0.457} &            - &      - &      - &      - &                  0.755 &           0.678 &              0.607 &           0.459 &             0.783 \\
                                         & Vanilla GDC &         0.781 &  0.767 &  0.754 &  0.713 &                0.761 &           0.722 &              0.685 &  0.623 &                    - &               - &                  - &               - &                  0.753 &  \textit{0.661} &  \underline{0.588} &  \textbf{0.496} &            - &      - &      - &      - &                  0.756 &           0.686 &              0.633 &           0.550 &             0.784 \\
                                         & Soft Medoid GDC &         0.770 &  0.763 &  0.757 &  0.734 &                0.726 &           0.724 &              0.723 &  0.719 &                    - &               - &                  - &               - &                  0.756 &  \textit{0.717} &  \underline{0.687} &  \textbf{0.660} &            - &      - &      - &      - &                  0.758 &           0.728 &              0.708 &           0.679 &             0.770 \\
                                         & Soft Median GDC &         0.770 &  0.762 &  0.757 &  0.736 &                0.727 &           0.725 &              0.724 &  0.720 &                    - &               - &                  - &               - &                  0.755 &  \textit{0.714} &  \underline{0.683} &  \textbf{0.653} &            - &      - &      - &      - &                  0.758 &           0.725 &              0.703 &           0.671 &             0.772 \\
       \cline{1-27}
       \multirow{4}{*}{\rotatebox{90}{\textbf{arXiv}}} & Vanilla GCN &         0.699 &  0.685 &  0.661 &  0.613 &                0.618 &           0.489 &              0.377 &  0.180 &                    - &               - &                  - &               - &                  0.599 &           0.473 &              0.394 &           0.297 &            - &      - &      - &      - &                  0.597 &  \textit{0.421} &  \underline{0.304} &  \textbf{0.175} &             0.686 \\
                                         & Vanilla GDC &         0.635 &  0.634 &  0.609 &  0.552 &                0.697 &           0.689 &              0.663 &  0.646 &                    - &               - &                  - &               - &                  0.497 &           0.381 &              0.305 &           0.247 &            - &      - &      - &      - &                  0.559 &  \textit{0.379} &  \underline{0.277} &  \textbf{0.176} &             0.677 \\
                                         & Soft Medoid GDC &         0.574 &  0.562 &  0.552 &  0.530 &                0.564 &           0.557 &              0.553 &  0.548 &                    - &               - &                  - &               - &                  0.502 &  \textit{0.446} &              0.429 &           0.428 &            - &      - &      - &      - &                  0.527 &           0.456 &  \underline{0.420} &  \textbf{0.373} &             0.585 \\
                                         & Soft Median GDC &         0.656 &  0.641 &  0.628 &  0.593 &                0.658 &           0.650 &              0.644 &  0.634 &                    - &               - &                  - &               - &                  0.572 &           0.465 &              0.422 &           0.414 &            - &      - &      - &      - &                  0.584 &  \textit{0.463} &  \underline{0.389} &  \textbf{0.298} &             0.665 \\
       \cline{1-27}
       \multirow{3}{*}{\rotatebox{90}{\textbf{Products}}} & Vanilla GCN &         0.709 &  0.674 &  0.636 &  0.545 &                0.708 &           0.659 &              0.594 &  0.390 &                    - &               - &                  - &               - &                  0.604 &  \textit{0.508} &  \underline{0.441} &  \textbf{0.321} &            - &      - &      - &      - &                  0.619 &           0.516 &              0.533 &           0.480 &             0.719 \\
                                         & Vanilla GDC &         0.701 &  0.677 &  0.657 &  0.618 &                0.706 &           0.700 &              0.695 &  0.685 &                    - &               - &                  - &               - &                  0.610 &           0.575 &  \underline{0.560} &  \textbf{0.528} &            - &      - &      - &      - &                  0.628 &  \textit{0.564} &              0.572 &           0.539 &             0.709 \\
                                         & Soft Median GDC &             - &      - &      - &      - &                    - &               - &                  - &      - &                    - &               - &                  - &               - &                      - &               - &                  - &               - &            - &      - &      - &      - &                  0.382 &  \textit{0.376} &  \underline{0.373} &  \textbf{0.174} &             0.391 \\
       \bottomrule
    \end{tabular}
  }
\end{table*}

In the following, we present our experiments to show the effectiveness and scalability of our proposed attacks and the defense. We first describe our setup and then discuss the results over wide range of graphs of different scale. %Once the experiments have been finalized, we will open source the code with configuration to reproduce the results.

\textbf{Defenses.} We also report the results on state of the art defenses of~\citep{Entezari2020, Geisler2020, Wu2019, Zhu2019}. For the Soft Medoid GDC, we use the temperature \(T=0.5\) as it is a good compromise between accuracy and robustness. The SVD GCN~\citep{Entezari2020} uses a (dense) low-rank approximation (here rank 50) of the adjacency matrix to filter adversarial perturbations. RGCN~\citep{Zhu2019} models the neighborhood aggregation via Gaussian distribution to filter outliers, and Jaccard GCN~\citep{Wu2019} filters edges based on attribute dissimilarity (here threshold 0.01).

%\todo{Hyperparams}

\textbf{Attacks.} We compare our GANG, PR-BCD, and GR-BCD attacks (see Sections~\ref{sec:attackkdd}-\ref{sec:prbcd}) with the global DICE~\citep{Waniek2018}, PGD~\citep{Xu2019a}, and greedy FGSM attacks~\citet{Geisler2020}. The greedy FGSM-like attack is the dense equivalent of our GR-BCD attack with the exception of flipping one edge at a time. DICE is a greedy, randomized black-box attack that flips one randomly determined entry in the adjacency matrix at a time. An edge is deleted if both nodes share the same label and an edge is added if the labels of the nodes differ. We ensure that a single node does not become disconnected. Moreover, we use 60\% of the budget to add new edges and otherwise remove edges.

\begin{table}[t]
  \centering
  \caption{Statistics of the used datasets. For the dense adjacency matrix we assume that each element is represented by 4 bytes. In the sparse case we use two 8 byte integer pointers and a 4 bytes float value.}
  \label{tab:datasets}
  \resizebox{\linewidth}{!}{
    \begin{tabular}{lrrrrr}
    \toprule
    {} & \textbf{\#Nodes $n$} & \textbf{\#Edges $e$} & \textbf{\#Features $d$} & \textbf{Size (dense)} & \textbf{Size (sparse)} \\
    \midrule
    \textbf{Cora ML~\citep{Bojchevski2018}} &                2,995 &                8,416 &                   2,879 &              35.88 MB &              168.32 kB \\
    \textbf{Citeseer~\citep{McCallum2000}}  &                3,312 &                4,715 &                   3,703 &              43.88 MB &               94.30 kB \\
    \textbf{PubMed~\citep{Sen2008}}         &               19,717 &               88,648 &                     500 &               1.56 GB &                1.77 MB \\
    \textbf{arXiv~\citep{Hu2020}}           &              169,343 &            1,166,243 &                     128 &             114.71 GB &               23.32 MB \\
    \textbf{Products~\citep{Hu2020}}        &            2,449,029 &          123,718,280 &                     100 &              23.99 TB &                2.47 GB \\
    \textbf{Papers 100M~\citep{Hu2020}}     &          111,059,956 &        1,615,685,872 &                     128 &              49.34 PB &               32.31 GB \\
    \bottomrule
    \end{tabular}
  }
\end{table}

\textbf{Datasets.} We use the common Cora ML~\citep{Bojchevski2018}, Citeseer~\citep{McCallum2000}, and PubMed~\citep{Sen2008} for comparing against the other state of the art attacks and defenses. For large scale experiments, we use two graphs of the recent Open Graph Benchmark~\citep{Hu2020}. In comparison to Pubmed, we scale the global attack by more than 100 times (number of nodes), or by factor 15,000 if counting the possible adjacency matrix entries (see Table~\ref{tab:datasets}). For Cora, Citeseer, PubMed, arXiv as well as Papers 100M we use an 11GB GeForce GTX 1080 Ti. The only exception are the full-batch experiments on Products where we use a 32GB Tesla V100. We exclusively perform the calculations on GPU, but for products where we coalesce the adjacency matrix on CPU.

\textbf{Checkpointing.} Empirically, almost 30 GB are required to train a three-layer GCN on Products (our largest dataset) using sparse matrices. However, obtaining the gradient, e.g.\ towards the perturbation set, requires extra memory. We notice that most operations in modern GNNs only depend on the neighborhood size (i.e.~a row in the adjacency matrix). As proposed by~\citet{Chen2016}, the gradient is obtainable with sublinear memory cost via checkpointing. The idea is to discard some intermediate results in the forward phase and recompute them in the backward phase. Specifically, we chunk some operations (e.g. matrix multiplication) within the message passing step to successfully scale to larger graphs. This allows us to attack a three-layer GCN on Products with full GPU acceleration.

\textbf{Hyperparameters.} We use the same setup as~\citet{Geisler2020} in their evaluation and for models on OGB we follow~\citet{Hu2020}, For the attacks GR-BCD and PR-BCD, we run the attack for 500 epochs (100 epochs fine-tuning with PR-BCD). We choose the search space size to be at least twice the edge perturbation ratio \(\epsilon\) (depends on the dataset). Since the edge budget \(\Delta_e\) in GANG influences the perturbed accuracy (see Figure~\ref{fig:gangnodeeffectiveness}), we decide for a relatively low value of 250. As these are preliminary results, the only exception is Products on which we report the results with the budget of \(\Delta_e=25,000\). Moreover with GANG we binarize the attributes on Cora ML, Citeseer and PubMed and use \(L_0\)-norm PGD analogously to PR-BCD.

\textbf{Evaluation of Losses.} In Table~\ref{tab:losscompare}, we compare the convectional CE loss with our newly proposed losses. For the admittedly large budget of \(\epsilon=0.25\), we see gains of up to 40\% on the perturbed accuracy. Moreover, if we compare the accuracy drop (i.e.\ clean minus perturbed accuracy) we also achieve for low budgets like \(\epsilon=0.01\) an improvements of more then 100\%. And those are only the numbers for the small datasets. On larger graphs such as arXiv we even observe gains of more than 100\% directly on the perturbed accuracy.

\textbf{Results overview.} In~Table~\ref{tab:global} we present the preliminary experimental results for our proposed attacks. We do not observe that sampling the search space harms the attack strength. Similarly to Figure~\ref{fig:randomblocksizeinfluence}, we even outperform the dense PGD on Cora ML. We conclude that our attacks are as effective as the other state of the art attacks.

\textbf{GNNs' fragility on large graphs.} In the following, we analyze the results of PR-BCD, with a budget of \(\epsilon=0.25\), and the GCN. We observe a relative drop in the perturbed accuracy by 20\% on Cora, 25\% on PubMed, 33 \% on arXiv. On products with the lower budget of \(\epsilon=0.1\), we already see a drop of the perturbed accuracy of 31\%. We conclude that there is likely some relationship between the fragility and the graph size. This relationship is similar for GR-BCD but much stronger for GANG. However, for GANG we have to consider that we may choose the attributes as well. arXiv as well as Products have \textit{dense} continuous features, and all the other datasets have \textit{sparse} continuous features. This relationship seems to persist for architectures other than GCN as well. Please note that further experiments are required to confirm this hypothesis. For example, on arXiv and products we use a three-layer GCN (to achieve state of the art accuracy) and for the other datasets we use just two layers. Moreover, the datasets have a different number of classes.

\todo{Dedicated Experiment}

\iffalse
\begin{figure}[t]
  \centering
  \hbox{\hspace{45pt} \resizebox{0.7\linewidth}{!}{%% Creator: Matplotlib, PGF backend
%%
%% To include the figure in your LaTeX document, write
%%   \input{<filename>.pgf}
%%
%% Make sure the required packages are loaded in your preamble
%%   \usepackage{pgf}
%%
%% and, on pdftex
%%   \usepackage[utf8]{inputenc}\DeclareUnicodeCharacter{2212}{-}
%%
%% or, on luatex and xetex
%%   \usepackage{unicode-math}
%%
%% Figures using additional raster images can only be included by \input if
%% they are in the same directory as the main LaTeX file. For loading figures
%% from other directories you can use the `import` package
%%   \usepackage{import}
%%
%% and then include the figures with
%%   \import{<path to file>}{<filename>.pgf}
%%
%% Matplotlib used the following preamble
%%   \usepackage[utf8]{inputenc}
%%   \usepackage[T1]{fontenc}
%%   \usepackage{amsmath}
%%   \newcommand*{\mat}[1]{\boldsymbol{#1}}
%%
\begingroup%
\makeatletter%
\begin{pgfpicture}%
\pgfpathrectangle{\pgfpointorigin}{\pgfqpoint{3.196757in}{0.554906in}}%
\pgfusepath{use as bounding box, clip}%
\begin{pgfscope}%
\pgfsetbuttcap%
\pgfsetmiterjoin%
\definecolor{currentfill}{rgb}{1.000000,1.000000,1.000000}%
\pgfsetfillcolor{currentfill}%
\pgfsetlinewidth{0.000000pt}%
\definecolor{currentstroke}{rgb}{1.000000,1.000000,1.000000}%
\pgfsetstrokecolor{currentstroke}%
\pgfsetstrokeopacity{0.000000}%
\pgfsetdash{}{0pt}%
\pgfpathmoveto{\pgfqpoint{0.000000in}{0.000000in}}%
\pgfpathlineto{\pgfqpoint{3.196757in}{0.000000in}}%
\pgfpathlineto{\pgfqpoint{3.196757in}{0.554906in}}%
\pgfpathlineto{\pgfqpoint{0.000000in}{0.554906in}}%
\pgfpathclose%
\pgfusepath{fill}%
\end{pgfscope}%
\begin{pgfscope}%
\pgfsetbuttcap%
\pgfsetmiterjoin%
\definecolor{currentfill}{rgb}{1.000000,1.000000,1.000000}%
\pgfsetfillcolor{currentfill}%
\pgfsetfillopacity{0.800000}%
\pgfsetlinewidth{1.003750pt}%
\definecolor{currentstroke}{rgb}{0.800000,0.800000,0.800000}%
\pgfsetstrokecolor{currentstroke}%
\pgfsetstrokeopacity{0.800000}%
\pgfsetdash{}{0pt}%
\pgfpathmoveto{\pgfqpoint{0.122222in}{0.100000in}}%
\pgfpathlineto{\pgfqpoint{3.074535in}{0.100000in}}%
\pgfpathquadraticcurveto{\pgfqpoint{3.096757in}{0.100000in}}{\pgfqpoint{3.096757in}{0.122222in}}%
\pgfpathlineto{\pgfqpoint{3.096757in}{0.432684in}}%
\pgfpathquadraticcurveto{\pgfqpoint{3.096757in}{0.454906in}}{\pgfqpoint{3.074535in}{0.454906in}}%
\pgfpathlineto{\pgfqpoint{0.122222in}{0.454906in}}%
\pgfpathquadraticcurveto{\pgfqpoint{0.100000in}{0.454906in}}{\pgfqpoint{0.100000in}{0.432684in}}%
\pgfpathlineto{\pgfqpoint{0.100000in}{0.122222in}}%
\pgfpathquadraticcurveto{\pgfqpoint{0.100000in}{0.100000in}}{\pgfqpoint{0.122222in}{0.100000in}}%
\pgfpathclose%
\pgfusepath{stroke,fill}%
\end{pgfscope}%
\begin{pgfscope}%
\pgfsetbuttcap%
\pgfsetroundjoin%
\pgfsetlinewidth{1.003750pt}%
\definecolor{currentstroke}{rgb}{0.298039,0.447059,0.690196}%
\pgfsetstrokecolor{currentstroke}%
\pgfsetdash{}{0pt}%
\pgfpathmoveto{\pgfqpoint{0.255556in}{0.310482in}}%
\pgfpathlineto{\pgfqpoint{0.255556in}{0.421593in}}%
\pgfusepath{stroke}%
\end{pgfscope}%
\begin{pgfscope}%
\pgfsetroundcap%
\pgfsetroundjoin%
\pgfsetlinewidth{1.003750pt}%
\definecolor{currentstroke}{rgb}{0.298039,0.447059,0.690196}%
\pgfsetstrokecolor{currentstroke}%
\pgfsetdash{}{0pt}%
\pgfpathmoveto{\pgfqpoint{0.144444in}{0.366037in}}%
\pgfpathlineto{\pgfqpoint{0.366667in}{0.366037in}}%
\pgfusepath{stroke}%
\end{pgfscope}%
\begin{pgfscope}%
\definecolor{textcolor}{rgb}{0.150000,0.150000,0.150000}%
\pgfsetstrokecolor{textcolor}%
\pgfsetfillcolor{textcolor}%
\pgftext[x=0.455556in,y=0.327148in,left,base]{\color{textcolor}\rmfamily\fontsize{8.000000}{9.600000}\selectfont Vanilla GCN}%
\end{pgfscope}%
\begin{pgfscope}%
\pgfsetbuttcap%
\pgfsetroundjoin%
\pgfsetlinewidth{1.003750pt}%
\definecolor{currentstroke}{rgb}{0.866667,0.517647,0.321569}%
\pgfsetstrokecolor{currentstroke}%
\pgfsetdash{}{0pt}%
\pgfpathmoveto{\pgfqpoint{0.255556in}{0.155549in}}%
\pgfpathlineto{\pgfqpoint{0.255556in}{0.266660in}}%
\pgfusepath{stroke}%
\end{pgfscope}%
\begin{pgfscope}%
\pgfsetbuttcap%
\pgfsetroundjoin%
\pgfsetlinewidth{1.003750pt}%
\definecolor{currentstroke}{rgb}{0.866667,0.517647,0.321569}%
\pgfsetstrokecolor{currentstroke}%
\pgfsetdash{{3.700000pt}{1.600000pt}}{0.000000pt}%
\pgfpathmoveto{\pgfqpoint{0.144444in}{0.211104in}}%
\pgfpathlineto{\pgfqpoint{0.366667in}{0.211104in}}%
\pgfusepath{stroke}%
\end{pgfscope}%
\begin{pgfscope}%
\definecolor{textcolor}{rgb}{0.150000,0.150000,0.150000}%
\pgfsetstrokecolor{textcolor}%
\pgfsetfillcolor{textcolor}%
\pgftext[x=0.455556in,y=0.172215in,left,base]{\color{textcolor}\rmfamily\fontsize{8.000000}{9.600000}\selectfont Vanilla GDC}%
\end{pgfscope}%
\begin{pgfscope}%
\pgfsetbuttcap%
\pgfsetroundjoin%
\pgfsetlinewidth{1.003750pt}%
\definecolor{currentstroke}{rgb}{0.333333,0.658824,0.407843}%
\pgfsetstrokecolor{currentstroke}%
\pgfsetdash{}{0pt}%
\pgfpathmoveto{\pgfqpoint{1.456685in}{0.310482in}}%
\pgfpathlineto{\pgfqpoint{1.456685in}{0.421593in}}%
\pgfusepath{stroke}%
\end{pgfscope}%
\begin{pgfscope}%
\pgfsetbuttcap%
\pgfsetroundjoin%
\pgfsetlinewidth{1.003750pt}%
\definecolor{currentstroke}{rgb}{0.333333,0.658824,0.407843}%
\pgfsetstrokecolor{currentstroke}%
\pgfsetdash{{3.700000pt}{1.600000pt}}{0.000000pt}%
\pgfpathmoveto{\pgfqpoint{1.345574in}{0.366037in}}%
\pgfpathlineto{\pgfqpoint{1.567796in}{0.366037in}}%
\pgfusepath{stroke}%
\end{pgfscope}%
\begin{pgfscope}%
\definecolor{textcolor}{rgb}{0.150000,0.150000,0.150000}%
\pgfsetstrokecolor{textcolor}%
\pgfsetfillcolor{textcolor}%
\pgftext[x=1.656685in,y=0.327148in,left,base]{\color{textcolor}\rmfamily\fontsize{8.000000}{9.600000}\selectfont Soft Medoid GDC (T=1.0)}%
\end{pgfscope}%
\begin{pgfscope}%
\pgfsetbuttcap%
\pgfsetroundjoin%
\pgfsetlinewidth{1.003750pt}%
\definecolor{currentstroke}{rgb}{0.768627,0.305882,0.321569}%
\pgfsetstrokecolor{currentstroke}%
\pgfsetdash{}{0pt}%
\pgfpathmoveto{\pgfqpoint{1.456685in}{0.149378in}}%
\pgfpathlineto{\pgfqpoint{1.456685in}{0.260489in}}%
\pgfusepath{stroke}%
\end{pgfscope}%
\begin{pgfscope}%
\pgfsetbuttcap%
\pgfsetroundjoin%
\pgfsetlinewidth{1.003750pt}%
\definecolor{currentstroke}{rgb}{0.768627,0.305882,0.321569}%
\pgfsetstrokecolor{currentstroke}%
\pgfsetdash{{3.700000pt}{1.600000pt}}{0.000000pt}%
\pgfpathmoveto{\pgfqpoint{1.345574in}{0.204933in}}%
\pgfpathlineto{\pgfqpoint{1.567796in}{0.204933in}}%
\pgfusepath{stroke}%
\end{pgfscope}%
\begin{pgfscope}%
\definecolor{textcolor}{rgb}{0.150000,0.150000,0.150000}%
\pgfsetstrokecolor{textcolor}%
\pgfsetfillcolor{textcolor}%
\pgftext[x=1.656685in,y=0.166044in,left,base]{\color{textcolor}\rmfamily\fontsize{8.000000}{9.600000}\selectfont SVD GCN}%
\end{pgfscope}%
\end{pgfpicture}%
\makeatother%
\endgroup%
}}
  \vspace{-14pt}
  \makebox[\linewidth][c]{
    \(\begin{array}{cc}
      \subfloat[Clean graph]{\resizebox{0.5\linewidth}{!}{%% Creator: Matplotlib, PGF backend
%%
%% To include the figure in your LaTeX document, write
%%   \input{<filename>.pgf}
%%
%% Make sure the required packages are loaded in your preamble
%%   \usepackage{pgf}
%%
%% and, on pdftex
%%   \usepackage[utf8]{inputenc}\DeclareUnicodeCharacter{2212}{-}
%%
%% or, on luatex and xetex
%%   \usepackage{unicode-math}
%%
%% Figures using additional raster images can only be included by \input if
%% they are in the same directory as the main LaTeX file. For loading figures
%% from other directories you can use the `import` package
%%   \usepackage{import}
%%
%% and then include the figures with
%%   \import{<path to file>}{<filename>.pgf}
%%
%% Matplotlib used the following preamble
%%   \usepackage[utf8]{inputenc}
%%   \usepackage[T1]{fontenc}
%%   \usepackage{amsmath}
%%   \newcommand*{\mat}[1]{\boldsymbol{#1}}
%%
\begingroup%
\makeatletter%
\begin{pgfpicture}%
\pgfpathrectangle{\pgfpointorigin}{\pgfqpoint{1.898089in}{1.812363in}}%
\pgfusepath{use as bounding box, clip}%
\begin{pgfscope}%
\pgfsetbuttcap%
\pgfsetmiterjoin%
\definecolor{currentfill}{rgb}{1.000000,1.000000,1.000000}%
\pgfsetfillcolor{currentfill}%
\pgfsetlinewidth{0.000000pt}%
\definecolor{currentstroke}{rgb}{1.000000,1.000000,1.000000}%
\pgfsetstrokecolor{currentstroke}%
\pgfsetstrokeopacity{0.000000}%
\pgfsetdash{}{0pt}%
\pgfpathmoveto{\pgfqpoint{0.000000in}{-0.000000in}}%
\pgfpathlineto{\pgfqpoint{1.898089in}{-0.000000in}}%
\pgfpathlineto{\pgfqpoint{1.898089in}{1.812363in}}%
\pgfpathlineto{\pgfqpoint{0.000000in}{1.812363in}}%
\pgfpathclose%
\pgfusepath{fill}%
\end{pgfscope}%
\begin{pgfscope}%
\pgfsetbuttcap%
\pgfsetmiterjoin%
\definecolor{currentfill}{rgb}{1.000000,1.000000,1.000000}%
\pgfsetfillcolor{currentfill}%
\pgfsetlinewidth{0.000000pt}%
\definecolor{currentstroke}{rgb}{0.000000,0.000000,0.000000}%
\pgfsetstrokecolor{currentstroke}%
\pgfsetstrokeopacity{0.000000}%
\pgfsetdash{}{0pt}%
\pgfpathmoveto{\pgfqpoint{0.519339in}{0.466613in}}%
\pgfpathlineto{\pgfqpoint{1.798089in}{0.466613in}}%
\pgfpathlineto{\pgfqpoint{1.798089in}{1.712363in}}%
\pgfpathlineto{\pgfqpoint{0.519339in}{1.712363in}}%
\pgfpathclose%
\pgfusepath{fill}%
\end{pgfscope}%
\begin{pgfscope}%
\pgfpathrectangle{\pgfqpoint{0.519339in}{0.466613in}}{\pgfqpoint{1.278750in}{1.245750in}}%
\pgfusepath{clip}%
\pgfsetroundcap%
\pgfsetroundjoin%
\pgfsetlinewidth{0.501875pt}%
\definecolor{currentstroke}{rgb}{0.800000,0.800000,0.800000}%
\pgfsetstrokecolor{currentstroke}%
\pgfsetdash{}{0pt}%
\pgfpathmoveto{\pgfqpoint{0.793368in}{0.466613in}}%
\pgfpathlineto{\pgfqpoint{0.793368in}{1.712363in}}%
\pgfusepath{stroke}%
\end{pgfscope}%
\begin{pgfscope}%
\definecolor{textcolor}{rgb}{0.150000,0.150000,0.150000}%
\pgfsetstrokecolor{textcolor}%
\pgfsetfillcolor{textcolor}%
\pgftext[x=0.793368in,y=0.376335in,,top]{\color{textcolor}\rmfamily\fontsize{8.000000}{9.600000}\selectfont \(\displaystyle {25000}\)}%
\end{pgfscope}%
\begin{pgfscope}%
\pgfpathrectangle{\pgfqpoint{0.519339in}{0.466613in}}{\pgfqpoint{1.278750in}{1.245750in}}%
\pgfusepath{clip}%
\pgfsetroundcap%
\pgfsetroundjoin%
\pgfsetlinewidth{0.501875pt}%
\definecolor{currentstroke}{rgb}{0.800000,0.800000,0.800000}%
\pgfsetstrokecolor{currentstroke}%
\pgfsetdash{}{0pt}%
\pgfpathmoveto{\pgfqpoint{1.151649in}{0.466613in}}%
\pgfpathlineto{\pgfqpoint{1.151649in}{1.712363in}}%
\pgfusepath{stroke}%
\end{pgfscope}%
\begin{pgfscope}%
\definecolor{textcolor}{rgb}{0.150000,0.150000,0.150000}%
\pgfsetstrokecolor{textcolor}%
\pgfsetfillcolor{textcolor}%
\pgftext[x=1.151649in,y=0.376335in,,top]{\color{textcolor}\rmfamily\fontsize{8.000000}{9.600000}\selectfont \(\displaystyle {50000}\)}%
\end{pgfscope}%
\begin{pgfscope}%
\pgfpathrectangle{\pgfqpoint{0.519339in}{0.466613in}}{\pgfqpoint{1.278750in}{1.245750in}}%
\pgfusepath{clip}%
\pgfsetroundcap%
\pgfsetroundjoin%
\pgfsetlinewidth{0.501875pt}%
\definecolor{currentstroke}{rgb}{0.800000,0.800000,0.800000}%
\pgfsetstrokecolor{currentstroke}%
\pgfsetdash{}{0pt}%
\pgfpathmoveto{\pgfqpoint{1.509931in}{0.466613in}}%
\pgfpathlineto{\pgfqpoint{1.509931in}{1.712363in}}%
\pgfusepath{stroke}%
\end{pgfscope}%
\begin{pgfscope}%
\definecolor{textcolor}{rgb}{0.150000,0.150000,0.150000}%
\pgfsetstrokecolor{textcolor}%
\pgfsetfillcolor{textcolor}%
\pgftext[x=1.509931in,y=0.376335in,,top]{\color{textcolor}\rmfamily\fontsize{8.000000}{9.600000}\selectfont \(\displaystyle {75000}\)}%
\end{pgfscope}%
\begin{pgfscope}%
\definecolor{textcolor}{rgb}{0.150000,0.150000,0.150000}%
\pgfsetstrokecolor{textcolor}%
\pgfsetfillcolor{textcolor}%
\pgftext[x=1.158714in,y=0.222655in,,top]{\color{textcolor}\rmfamily\fontsize{10.000000}{12.000000}\selectfont Number of nodes}%
\end{pgfscope}%
\begin{pgfscope}%
\pgfpathrectangle{\pgfqpoint{0.519339in}{0.466613in}}{\pgfqpoint{1.278750in}{1.245750in}}%
\pgfusepath{clip}%
\pgfsetroundcap%
\pgfsetroundjoin%
\pgfsetlinewidth{0.501875pt}%
\definecolor{currentstroke}{rgb}{0.800000,0.800000,0.800000}%
\pgfsetstrokecolor{currentstroke}%
\pgfsetdash{}{0pt}%
\pgfpathmoveto{\pgfqpoint{0.519339in}{0.805931in}}%
\pgfpathlineto{\pgfqpoint{1.798089in}{0.805931in}}%
\pgfusepath{stroke}%
\end{pgfscope}%
\begin{pgfscope}%
\definecolor{textcolor}{rgb}{0.150000,0.150000,0.150000}%
\pgfsetstrokecolor{textcolor}%
\pgfsetfillcolor{textcolor}%
\pgftext[x=0.278211in, y=0.767669in, left, base]{\color{textcolor}\rmfamily\fontsize{8.000000}{9.600000}\selectfont \(\displaystyle {0.8}\)}%
\end{pgfscope}%
\begin{pgfscope}%
\pgfpathrectangle{\pgfqpoint{0.519339in}{0.466613in}}{\pgfqpoint{1.278750in}{1.245750in}}%
\pgfusepath{clip}%
\pgfsetroundcap%
\pgfsetroundjoin%
\pgfsetlinewidth{0.501875pt}%
\definecolor{currentstroke}{rgb}{0.800000,0.800000,0.800000}%
\pgfsetstrokecolor{currentstroke}%
\pgfsetdash{}{0pt}%
\pgfpathmoveto{\pgfqpoint{0.519339in}{1.271422in}}%
\pgfpathlineto{\pgfqpoint{1.798089in}{1.271422in}}%
\pgfusepath{stroke}%
\end{pgfscope}%
\begin{pgfscope}%
\definecolor{textcolor}{rgb}{0.150000,0.150000,0.150000}%
\pgfsetstrokecolor{textcolor}%
\pgfsetfillcolor{textcolor}%
\pgftext[x=0.278211in, y=1.233159in, left, base]{\color{textcolor}\rmfamily\fontsize{8.000000}{9.600000}\selectfont \(\displaystyle {0.9}\)}%
\end{pgfscope}%
\begin{pgfscope}%
\definecolor{textcolor}{rgb}{0.150000,0.150000,0.150000}%
\pgfsetstrokecolor{textcolor}%
\pgfsetfillcolor{textcolor}%
\pgftext[x=0.222655in,y=1.089488in,,bottom,rotate=90.000000]{\color{textcolor}\rmfamily\fontsize{10.000000}{12.000000}\selectfont Accuracy}%
\end{pgfscope}%
\begin{pgfscope}%
\pgfpathrectangle{\pgfqpoint{0.519339in}{0.466613in}}{\pgfqpoint{1.278750in}{1.245750in}}%
\pgfusepath{clip}%
\pgfsetbuttcap%
\pgfsetroundjoin%
\definecolor{currentfill}{rgb}{0.298039,0.447059,0.690196}%
\pgfsetfillcolor{currentfill}%
\pgfsetfillopacity{0.800000}%
\pgfsetlinewidth{1.003750pt}%
\definecolor{currentstroke}{rgb}{0.298039,0.447059,0.690196}%
\pgfsetstrokecolor{currentstroke}%
\pgfsetstrokeopacity{0.800000}%
\pgfsetdash{}{0pt}%
\pgfsys@defobject{currentmarker}{\pgfqpoint{-0.017010in}{-0.017010in}}{\pgfqpoint{0.017010in}{0.017010in}}{%
\pgfpathmoveto{\pgfqpoint{0.000000in}{-0.017010in}}%
\pgfpathcurveto{\pgfqpoint{0.004511in}{-0.017010in}}{\pgfqpoint{0.008838in}{-0.015218in}}{\pgfqpoint{0.012028in}{-0.012028in}}%
\pgfpathcurveto{\pgfqpoint{0.015218in}{-0.008838in}}{\pgfqpoint{0.017010in}{-0.004511in}}{\pgfqpoint{0.017010in}{0.000000in}}%
\pgfpathcurveto{\pgfqpoint{0.017010in}{0.004511in}}{\pgfqpoint{0.015218in}{0.008838in}}{\pgfqpoint{0.012028in}{0.012028in}}%
\pgfpathcurveto{\pgfqpoint{0.008838in}{0.015218in}}{\pgfqpoint{0.004511in}{0.017010in}}{\pgfqpoint{0.000000in}{0.017010in}}%
\pgfpathcurveto{\pgfqpoint{-0.004511in}{0.017010in}}{\pgfqpoint{-0.008838in}{0.015218in}}{\pgfqpoint{-0.012028in}{0.012028in}}%
\pgfpathcurveto{\pgfqpoint{-0.015218in}{0.008838in}}{\pgfqpoint{-0.017010in}{0.004511in}}{\pgfqpoint{-0.017010in}{0.000000in}}%
\pgfpathcurveto{\pgfqpoint{-0.017010in}{-0.004511in}}{\pgfqpoint{-0.015218in}{-0.008838in}}{\pgfqpoint{-0.012028in}{-0.012028in}}%
\pgfpathcurveto{\pgfqpoint{-0.008838in}{-0.015218in}}{\pgfqpoint{-0.004511in}{-0.017010in}}{\pgfqpoint{0.000000in}{-0.017010in}}%
\pgfpathclose%
\pgfusepath{stroke,fill}%
}%
\begin{pgfscope}%
\pgfsys@transformshift{0.993862in}{1.480281in}%
\pgfsys@useobject{currentmarker}{}%
\end{pgfscope}%
\begin{pgfscope}%
\pgfsys@transformshift{0.751836in}{1.497697in}%
\pgfsys@useobject{currentmarker}{}%
\end{pgfscope}%
\begin{pgfscope}%
\pgfsys@transformshift{0.862430in}{1.412639in}%
\pgfsys@useobject{currentmarker}{}%
\end{pgfscope}%
\begin{pgfscope}%
\pgfsys@transformshift{0.678660in}{1.291838in}%
\pgfsys@useobject{currentmarker}{}%
\end{pgfscope}%
\begin{pgfscope}%
\pgfsys@transformshift{0.743265in}{0.952657in}%
\pgfsys@useobject{currentmarker}{}%
\end{pgfscope}%
\begin{pgfscope}%
\pgfsys@transformshift{1.093779in}{1.470028in}%
\pgfsys@useobject{currentmarker}{}%
\end{pgfscope}%
\begin{pgfscope}%
\pgfsys@transformshift{0.607477in}{0.809490in}%
\pgfsys@useobject{currentmarker}{}%
\end{pgfscope}%
\begin{pgfscope}%
\pgfsys@transformshift{0.910325in}{1.420259in}%
\pgfsys@useobject{currentmarker}{}%
\end{pgfscope}%
\begin{pgfscope}%
\pgfsys@transformshift{1.274138in}{1.501017in}%
\pgfsys@useobject{currentmarker}{}%
\end{pgfscope}%
\begin{pgfscope}%
\pgfsys@transformshift{1.124291in}{1.253276in}%
\pgfsys@useobject{currentmarker}{}%
\end{pgfscope}%
\begin{pgfscope}%
\pgfsys@transformshift{1.036268in}{1.396345in}%
\pgfsys@useobject{currentmarker}{}%
\end{pgfscope}%
\begin{pgfscope}%
\pgfsys@transformshift{0.629002in}{1.295560in}%
\pgfsys@useobject{currentmarker}{}%
\end{pgfscope}%
\begin{pgfscope}%
\pgfsys@transformshift{0.752409in}{0.895677in}%
\pgfsys@useobject{currentmarker}{}%
\end{pgfscope}%
\begin{pgfscope}%
\pgfsys@transformshift{0.941152in}{1.344266in}%
\pgfsys@useobject{currentmarker}{}%
\end{pgfscope}%
\begin{pgfscope}%
\pgfsys@transformshift{0.825613in}{1.603667in}%
\pgfsys@useobject{currentmarker}{}%
\end{pgfscope}%
\begin{pgfscope}%
\pgfsys@transformshift{1.317204in}{1.295537in}%
\pgfsys@useobject{currentmarker}{}%
\end{pgfscope}%
\begin{pgfscope}%
\pgfsys@transformshift{0.611360in}{1.294946in}%
\pgfsys@useobject{currentmarker}{}%
\end{pgfscope}%
\begin{pgfscope}%
\pgfsys@transformshift{1.105001in}{1.514262in}%
\pgfsys@useobject{currentmarker}{}%
\end{pgfscope}%
\begin{pgfscope}%
\pgfsys@transformshift{0.826946in}{1.372424in}%
\pgfsys@useobject{currentmarker}{}%
\end{pgfscope}%
\begin{pgfscope}%
\pgfsys@transformshift{1.038432in}{1.385368in}%
\pgfsys@useobject{currentmarker}{}%
\end{pgfscope}%
\begin{pgfscope}%
\pgfsys@transformshift{0.680824in}{1.156856in}%
\pgfsys@useobject{currentmarker}{}%
\end{pgfscope}%
\begin{pgfscope}%
\pgfsys@transformshift{0.924556in}{1.231672in}%
\pgfsys@useobject{currentmarker}{}%
\end{pgfscope}%
\begin{pgfscope}%
\pgfsys@transformshift{0.586238in}{1.341182in}%
\pgfsys@useobject{currentmarker}{}%
\end{pgfscope}%
\begin{pgfscope}%
\pgfsys@transformshift{1.038203in}{1.594625in}%
\pgfsys@useobject{currentmarker}{}%
\end{pgfscope}%
\begin{pgfscope}%
\pgfsys@transformshift{0.589419in}{1.194661in}%
\pgfsys@useobject{currentmarker}{}%
\end{pgfscope}%
\begin{pgfscope}%
\pgfsys@transformshift{0.794055in}{0.795705in}%
\pgfsys@useobject{currentmarker}{}%
\end{pgfscope}%
\begin{pgfscope}%
\pgfsys@transformshift{0.611131in}{1.082340in}%
\pgfsys@useobject{currentmarker}{}%
\end{pgfscope}%
\begin{pgfscope}%
\pgfsys@transformshift{0.978484in}{1.349115in}%
\pgfsys@useobject{currentmarker}{}%
\end{pgfscope}%
\begin{pgfscope}%
\pgfsys@transformshift{0.728863in}{1.221320in}%
\pgfsys@useobject{currentmarker}{}%
\end{pgfscope}%
\begin{pgfscope}%
\pgfsys@transformshift{0.831073in}{1.310369in}%
\pgfsys@useobject{currentmarker}{}%
\end{pgfscope}%
\begin{pgfscope}%
\pgfsys@transformshift{0.999064in}{1.371954in}%
\pgfsys@useobject{currentmarker}{}%
\end{pgfscope}%
\begin{pgfscope}%
\pgfsys@transformshift{0.954437in}{1.304791in}%
\pgfsys@useobject{currentmarker}{}%
\end{pgfscope}%
\begin{pgfscope}%
\pgfsys@transformshift{1.019773in}{1.258183in}%
\pgfsys@useobject{currentmarker}{}%
\end{pgfscope}%
\begin{pgfscope}%
\pgfsys@transformshift{1.049482in}{1.423495in}%
\pgfsys@useobject{currentmarker}{}%
\end{pgfscope}%
\begin{pgfscope}%
\pgfsys@transformshift{0.753971in}{1.253947in}%
\pgfsys@useobject{currentmarker}{}%
\end{pgfscope}%
\begin{pgfscope}%
\pgfsys@transformshift{1.338844in}{1.032392in}%
\pgfsys@useobject{currentmarker}{}%
\end{pgfscope}%
\begin{pgfscope}%
\pgfsys@transformshift{0.648966in}{1.412742in}%
\pgfsys@useobject{currentmarker}{}%
\end{pgfscope}%
\begin{pgfscope}%
\pgfsys@transformshift{1.096416in}{1.579619in}%
\pgfsys@useobject{currentmarker}{}%
\end{pgfscope}%
\begin{pgfscope}%
\pgfsys@transformshift{0.935591in}{1.299444in}%
\pgfsys@useobject{currentmarker}{}%
\end{pgfscope}%
\begin{pgfscope}%
\pgfsys@transformshift{1.243025in}{0.989001in}%
\pgfsys@useobject{currentmarker}{}%
\end{pgfscope}%
\begin{pgfscope}%
\pgfsys@transformshift{1.082787in}{1.383034in}%
\pgfsys@useobject{currentmarker}{}%
\end{pgfscope}%
\begin{pgfscope}%
\pgfsys@transformshift{0.608365in}{1.405664in}%
\pgfsys@useobject{currentmarker}{}%
\end{pgfscope}%
\begin{pgfscope}%
\pgfsys@transformshift{0.922951in}{1.459599in}%
\pgfsys@useobject{currentmarker}{}%
\end{pgfscope}%
\begin{pgfscope}%
\pgfsys@transformshift{0.616219in}{1.335986in}%
\pgfsys@useobject{currentmarker}{}%
\end{pgfscope}%
\begin{pgfscope}%
\pgfsys@transformshift{0.922965in}{1.342120in}%
\pgfsys@useobject{currentmarker}{}%
\end{pgfscope}%
\begin{pgfscope}%
\pgfsys@transformshift{0.659184in}{0.989179in}%
\pgfsys@useobject{currentmarker}{}%
\end{pgfscope}%
\begin{pgfscope}%
\pgfsys@transformshift{0.593332in}{1.489027in}%
\pgfsys@useobject{currentmarker}{}%
\end{pgfscope}%
\begin{pgfscope}%
\pgfsys@transformshift{0.977023in}{1.264297in}%
\pgfsys@useobject{currentmarker}{}%
\end{pgfscope}%
\begin{pgfscope}%
\pgfsys@transformshift{1.124190in}{0.904371in}%
\pgfsys@useobject{currentmarker}{}%
\end{pgfscope}%
\begin{pgfscope}%
\pgfsys@transformshift{0.846451in}{1.475064in}%
\pgfsys@useobject{currentmarker}{}%
\end{pgfscope}%
\begin{pgfscope}%
\pgfsys@transformshift{0.896997in}{1.333524in}%
\pgfsys@useobject{currentmarker}{}%
\end{pgfscope}%
\begin{pgfscope}%
\pgfsys@transformshift{0.600970in}{1.440893in}%
\pgfsys@useobject{currentmarker}{}%
\end{pgfscope}%
\begin{pgfscope}%
\pgfsys@transformshift{0.610071in}{1.500129in}%
\pgfsys@useobject{currentmarker}{}%
\end{pgfscope}%
\begin{pgfscope}%
\pgfsys@transformshift{0.710132in}{1.269801in}%
\pgfsys@useobject{currentmarker}{}%
\end{pgfscope}%
\begin{pgfscope}%
\pgfsys@transformshift{0.647690in}{1.015120in}%
\pgfsys@useobject{currentmarker}{}%
\end{pgfscope}%
\begin{pgfscope}%
\pgfsys@transformshift{0.572809in}{1.481849in}%
\pgfsys@useobject{currentmarker}{}%
\end{pgfscope}%
\begin{pgfscope}%
\pgfsys@transformshift{0.811726in}{1.145825in}%
\pgfsys@useobject{currentmarker}{}%
\end{pgfscope}%
\begin{pgfscope}%
\pgfsys@transformshift{1.086327in}{1.557445in}%
\pgfsys@useobject{currentmarker}{}%
\end{pgfscope}%
\begin{pgfscope}%
\pgfsys@transformshift{0.709544in}{1.159110in}%
\pgfsys@useobject{currentmarker}{}%
\end{pgfscope}%
\begin{pgfscope}%
\pgfsys@transformshift{0.627440in}{1.101350in}%
\pgfsys@useobject{currentmarker}{}%
\end{pgfscope}%
\begin{pgfscope}%
\pgfsys@transformshift{0.671709in}{1.089602in}%
\pgfsys@useobject{currentmarker}{}%
\end{pgfscope}%
\begin{pgfscope}%
\pgfsys@transformshift{0.909651in}{1.503553in}%
\pgfsys@useobject{currentmarker}{}%
\end{pgfscope}%
\begin{pgfscope}%
\pgfsys@transformshift{0.528741in}{1.200519in}%
\pgfsys@useobject{currentmarker}{}%
\end{pgfscope}%
\begin{pgfscope}%
\pgfsys@transformshift{0.949105in}{1.305580in}%
\pgfsys@useobject{currentmarker}{}%
\end{pgfscope}%
\begin{pgfscope}%
\pgfsys@transformshift{0.599079in}{1.312087in}%
\pgfsys@useobject{currentmarker}{}%
\end{pgfscope}%
\begin{pgfscope}%
\pgfsys@transformshift{0.725165in}{1.111831in}%
\pgfsys@useobject{currentmarker}{}%
\end{pgfscope}%
\begin{pgfscope}%
\pgfsys@transformshift{0.899791in}{1.343915in}%
\pgfsys@useobject{currentmarker}{}%
\end{pgfscope}%
\begin{pgfscope}%
\pgfsys@transformshift{0.849976in}{1.270339in}%
\pgfsys@useobject{currentmarker}{}%
\end{pgfscope}%
\begin{pgfscope}%
\pgfsys@transformshift{0.906699in}{1.345684in}%
\pgfsys@useobject{currentmarker}{}%
\end{pgfscope}%
\begin{pgfscope}%
\pgfsys@transformshift{1.294317in}{0.858475in}%
\pgfsys@useobject{currentmarker}{}%
\end{pgfscope}%
\begin{pgfscope}%
\pgfsys@transformshift{1.010529in}{1.374674in}%
\pgfsys@useobject{currentmarker}{}%
\end{pgfscope}%
\begin{pgfscope}%
\pgfsys@transformshift{0.652534in}{1.378174in}%
\pgfsys@useobject{currentmarker}{}%
\end{pgfscope}%
\begin{pgfscope}%
\pgfsys@transformshift{0.913292in}{1.310377in}%
\pgfsys@useobject{currentmarker}{}%
\end{pgfscope}%
\begin{pgfscope}%
\pgfsys@transformshift{0.825899in}{1.532711in}%
\pgfsys@useobject{currentmarker}{}%
\end{pgfscope}%
\begin{pgfscope}%
\pgfsys@transformshift{1.015301in}{1.428631in}%
\pgfsys@useobject{currentmarker}{}%
\end{pgfscope}%
\begin{pgfscope}%
\pgfsys@transformshift{0.978599in}{1.267723in}%
\pgfsys@useobject{currentmarker}{}%
\end{pgfscope}%
\begin{pgfscope}%
\pgfsys@transformshift{0.750976in}{1.089460in}%
\pgfsys@useobject{currentmarker}{}%
\end{pgfscope}%
\begin{pgfscope}%
\pgfsys@transformshift{1.215968in}{0.883844in}%
\pgfsys@useobject{currentmarker}{}%
\end{pgfscope}%
\begin{pgfscope}%
\pgfsys@transformshift{1.200848in}{1.449233in}%
\pgfsys@useobject{currentmarker}{}%
\end{pgfscope}%
\begin{pgfscope}%
\pgfsys@transformshift{1.207641in}{1.446168in}%
\pgfsys@useobject{currentmarker}{}%
\end{pgfscope}%
\begin{pgfscope}%
\pgfsys@transformshift{0.945580in}{1.427560in}%
\pgfsys@useobject{currentmarker}{}%
\end{pgfscope}%
\begin{pgfscope}%
\pgfsys@transformshift{0.714990in}{1.023652in}%
\pgfsys@useobject{currentmarker}{}%
\end{pgfscope}%
\begin{pgfscope}%
\pgfsys@transformshift{0.599236in}{1.302051in}%
\pgfsys@useobject{currentmarker}{}%
\end{pgfscope}%
\begin{pgfscope}%
\pgfsys@transformshift{1.287939in}{1.272436in}%
\pgfsys@useobject{currentmarker}{}%
\end{pgfscope}%
\begin{pgfscope}%
\pgfsys@transformshift{0.707409in}{1.207212in}%
\pgfsys@useobject{currentmarker}{}%
\end{pgfscope}%
\begin{pgfscope}%
\pgfsys@transformshift{0.953720in}{1.497545in}%
\pgfsys@useobject{currentmarker}{}%
\end{pgfscope}%
\begin{pgfscope}%
\pgfsys@transformshift{0.693063in}{1.551569in}%
\pgfsys@useobject{currentmarker}{}%
\end{pgfscope}%
\begin{pgfscope}%
\pgfsys@transformshift{1.269796in}{1.090696in}%
\pgfsys@useobject{currentmarker}{}%
\end{pgfscope}%
\begin{pgfscope}%
\pgfsys@transformshift{1.027669in}{1.244339in}%
\pgfsys@useobject{currentmarker}{}%
\end{pgfscope}%
\begin{pgfscope}%
\pgfsys@transformshift{0.664486in}{1.415611in}%
\pgfsys@useobject{currentmarker}{}%
\end{pgfscope}%
\begin{pgfscope}%
\pgfsys@transformshift{1.161538in}{1.306208in}%
\pgfsys@useobject{currentmarker}{}%
\end{pgfscope}%
\begin{pgfscope}%
\pgfsys@transformshift{1.377753in}{1.412407in}%
\pgfsys@useobject{currentmarker}{}%
\end{pgfscope}%
\begin{pgfscope}%
\pgfsys@transformshift{1.082544in}{1.280350in}%
\pgfsys@useobject{currentmarker}{}%
\end{pgfscope}%
\begin{pgfscope}%
\pgfsys@transformshift{0.836490in}{1.524640in}%
\pgfsys@useobject{currentmarker}{}%
\end{pgfscope}%
\begin{pgfscope}%
\pgfsys@transformshift{0.789527in}{1.458921in}%
\pgfsys@useobject{currentmarker}{}%
\end{pgfscope}%
\begin{pgfscope}%
\pgfsys@transformshift{0.971061in}{1.579036in}%
\pgfsys@useobject{currentmarker}{}%
\end{pgfscope}%
\begin{pgfscope}%
\pgfsys@transformshift{0.977610in}{1.255946in}%
\pgfsys@useobject{currentmarker}{}%
\end{pgfscope}%
\begin{pgfscope}%
\pgfsys@transformshift{0.986696in}{1.418045in}%
\pgfsys@useobject{currentmarker}{}%
\end{pgfscope}%
\begin{pgfscope}%
\pgfsys@transformshift{0.995166in}{1.524367in}%
\pgfsys@useobject{currentmarker}{}%
\end{pgfscope}%
\begin{pgfscope}%
\pgfsys@transformshift{0.926562in}{1.611529in}%
\pgfsys@useobject{currentmarker}{}%
\end{pgfscope}%
\begin{pgfscope}%
\pgfsys@transformshift{0.593145in}{1.225622in}%
\pgfsys@useobject{currentmarker}{}%
\end{pgfscope}%
\begin{pgfscope}%
\pgfsys@transformshift{0.822374in}{1.501846in}%
\pgfsys@useobject{currentmarker}{}%
\end{pgfscope}%
\begin{pgfscope}%
\pgfsys@transformshift{0.780011in}{1.305641in}%
\pgfsys@useobject{currentmarker}{}%
\end{pgfscope}%
\begin{pgfscope}%
\pgfsys@transformshift{0.632513in}{0.996764in}%
\pgfsys@useobject{currentmarker}{}%
\end{pgfscope}%
\begin{pgfscope}%
\pgfsys@transformshift{1.464630in}{0.902906in}%
\pgfsys@useobject{currentmarker}{}%
\end{pgfscope}%
\begin{pgfscope}%
\pgfsys@transformshift{0.519339in}{1.407866in}%
\pgfsys@useobject{currentmarker}{}%
\end{pgfscope}%
\begin{pgfscope}%
\pgfsys@transformshift{0.597617in}{1.216037in}%
\pgfsys@useobject{currentmarker}{}%
\end{pgfscope}%
\begin{pgfscope}%
\pgfsys@transformshift{1.095557in}{0.842274in}%
\pgfsys@useobject{currentmarker}{}%
\end{pgfscope}%
\begin{pgfscope}%
\pgfsys@transformshift{1.345895in}{0.937199in}%
\pgfsys@useobject{currentmarker}{}%
\end{pgfscope}%
\begin{pgfscope}%
\pgfsys@transformshift{0.617021in}{1.033899in}%
\pgfsys@useobject{currentmarker}{}%
\end{pgfscope}%
\begin{pgfscope}%
\pgfsys@transformshift{0.746060in}{1.238515in}%
\pgfsys@useobject{currentmarker}{}%
\end{pgfscope}%
\begin{pgfscope}%
\pgfsys@transformshift{0.768288in}{0.959364in}%
\pgfsys@useobject{currentmarker}{}%
\end{pgfscope}%
\begin{pgfscope}%
\pgfsys@transformshift{0.977768in}{1.115257in}%
\pgfsys@useobject{currentmarker}{}%
\end{pgfscope}%
\begin{pgfscope}%
\pgfsys@transformshift{0.590466in}{1.277673in}%
\pgfsys@useobject{currentmarker}{}%
\end{pgfscope}%
\begin{pgfscope}%
\pgfsys@transformshift{0.775769in}{1.464961in}%
\pgfsys@useobject{currentmarker}{}%
\end{pgfscope}%
\begin{pgfscope}%
\pgfsys@transformshift{0.956013in}{1.581524in}%
\pgfsys@useobject{currentmarker}{}%
\end{pgfscope}%
\begin{pgfscope}%
\pgfsys@transformshift{0.935247in}{1.442479in}%
\pgfsys@useobject{currentmarker}{}%
\end{pgfscope}%
\begin{pgfscope}%
\pgfsys@transformshift{1.280989in}{1.004151in}%
\pgfsys@useobject{currentmarker}{}%
\end{pgfscope}%
\begin{pgfscope}%
\pgfsys@transformshift{1.108369in}{1.477315in}%
\pgfsys@useobject{currentmarker}{}%
\end{pgfscope}%
\begin{pgfscope}%
\pgfsys@transformshift{0.714775in}{0.925376in}%
\pgfsys@useobject{currentmarker}{}%
\end{pgfscope}%
\begin{pgfscope}%
\pgfsys@transformshift{0.597889in}{1.470549in}%
\pgfsys@useobject{currentmarker}{}%
\end{pgfscope}%
\begin{pgfscope}%
\pgfsys@transformshift{0.941667in}{1.332477in}%
\pgfsys@useobject{currentmarker}{}%
\end{pgfscope}%
\begin{pgfscope}%
\pgfsys@transformshift{0.927795in}{1.433299in}%
\pgfsys@useobject{currentmarker}{}%
\end{pgfscope}%
\begin{pgfscope}%
\pgfsys@transformshift{0.816727in}{1.505751in}%
\pgfsys@useobject{currentmarker}{}%
\end{pgfscope}%
\begin{pgfscope}%
\pgfsys@transformshift{0.593403in}{1.408210in}%
\pgfsys@useobject{currentmarker}{}%
\end{pgfscope}%
\begin{pgfscope}%
\pgfsys@transformshift{0.759761in}{1.398067in}%
\pgfsys@useobject{currentmarker}{}%
\end{pgfscope}%
\begin{pgfscope}%
\pgfsys@transformshift{0.731399in}{1.467435in}%
\pgfsys@useobject{currentmarker}{}%
\end{pgfscope}%
\begin{pgfscope}%
\pgfsys@transformshift{0.853731in}{1.415037in}%
\pgfsys@useobject{currentmarker}{}%
\end{pgfscope}%
\begin{pgfscope}%
\pgfsys@transformshift{0.619787in}{1.435965in}%
\pgfsys@useobject{currentmarker}{}%
\end{pgfscope}%
\begin{pgfscope}%
\pgfsys@transformshift{0.599595in}{1.124956in}%
\pgfsys@useobject{currentmarker}{}%
\end{pgfscope}%
\begin{pgfscope}%
\pgfsys@transformshift{0.892511in}{1.052757in}%
\pgfsys@useobject{currentmarker}{}%
\end{pgfscope}%
\begin{pgfscope}%
\pgfsys@transformshift{1.086299in}{1.408226in}%
\pgfsys@useobject{currentmarker}{}%
\end{pgfscope}%
\begin{pgfscope}%
\pgfsys@transformshift{0.964139in}{1.385145in}%
\pgfsys@useobject{currentmarker}{}%
\end{pgfscope}%
\begin{pgfscope}%
\pgfsys@transformshift{1.064400in}{1.435308in}%
\pgfsys@useobject{currentmarker}{}%
\end{pgfscope}%
\begin{pgfscope}%
\pgfsys@transformshift{0.613037in}{1.389510in}%
\pgfsys@useobject{currentmarker}{}%
\end{pgfscope}%
\begin{pgfscope}%
\pgfsys@transformshift{0.812055in}{1.406697in}%
\pgfsys@useobject{currentmarker}{}%
\end{pgfscope}%
\begin{pgfscope}%
\pgfsys@transformshift{1.364124in}{1.337959in}%
\pgfsys@useobject{currentmarker}{}%
\end{pgfscope}%
\begin{pgfscope}%
\pgfsys@transformshift{1.076009in}{1.538229in}%
\pgfsys@useobject{currentmarker}{}%
\end{pgfscope}%
\begin{pgfscope}%
\pgfsys@transformshift{1.070978in}{1.433311in}%
\pgfsys@useobject{currentmarker}{}%
\end{pgfscope}%
\begin{pgfscope}%
\pgfsys@transformshift{1.031080in}{1.447576in}%
\pgfsys@useobject{currentmarker}{}%
\end{pgfscope}%
\begin{pgfscope}%
\pgfsys@transformshift{0.960140in}{1.462108in}%
\pgfsys@useobject{currentmarker}{}%
\end{pgfscope}%
\begin{pgfscope}%
\pgfsys@transformshift{0.987513in}{1.527189in}%
\pgfsys@useobject{currentmarker}{}%
\end{pgfscope}%
\begin{pgfscope}%
\pgfsys@transformshift{0.610114in}{1.272209in}%
\pgfsys@useobject{currentmarker}{}%
\end{pgfscope}%
\begin{pgfscope}%
\pgfsys@transformshift{0.674619in}{1.349599in}%
\pgfsys@useobject{currentmarker}{}%
\end{pgfscope}%
\begin{pgfscope}%
\pgfsys@transformshift{0.840159in}{1.319079in}%
\pgfsys@useobject{currentmarker}{}%
\end{pgfscope}%
\begin{pgfscope}%
\pgfsys@transformshift{0.643391in}{1.319244in}%
\pgfsys@useobject{currentmarker}{}%
\end{pgfscope}%
\begin{pgfscope}%
\pgfsys@transformshift{1.121897in}{1.438631in}%
\pgfsys@useobject{currentmarker}{}%
\end{pgfscope}%
\begin{pgfscope}%
\pgfsys@transformshift{0.748425in}{1.066560in}%
\pgfsys@useobject{currentmarker}{}%
\end{pgfscope}%
\begin{pgfscope}%
\pgfsys@transformshift{0.695299in}{1.161780in}%
\pgfsys@useobject{currentmarker}{}%
\end{pgfscope}%
\begin{pgfscope}%
\pgfsys@transformshift{1.006545in}{1.367023in}%
\pgfsys@useobject{currentmarker}{}%
\end{pgfscope}%
\begin{pgfscope}%
\pgfsys@transformshift{0.627197in}{1.211511in}%
\pgfsys@useobject{currentmarker}{}%
\end{pgfscope}%
\begin{pgfscope}%
\pgfsys@transformshift{0.671394in}{1.022731in}%
\pgfsys@useobject{currentmarker}{}%
\end{pgfscope}%
\begin{pgfscope}%
\pgfsys@transformshift{0.528139in}{0.944711in}%
\pgfsys@useobject{currentmarker}{}%
\end{pgfscope}%
\begin{pgfscope}%
\pgfsys@transformshift{0.852470in}{1.412527in}%
\pgfsys@useobject{currentmarker}{}%
\end{pgfscope}%
\begin{pgfscope}%
\pgfsys@transformshift{0.922363in}{1.530431in}%
\pgfsys@useobject{currentmarker}{}%
\end{pgfscope}%
\begin{pgfscope}%
\pgfsys@transformshift{0.740944in}{1.455598in}%
\pgfsys@useobject{currentmarker}{}%
\end{pgfscope}%
\begin{pgfscope}%
\pgfsys@transformshift{1.300379in}{1.286359in}%
\pgfsys@useobject{currentmarker}{}%
\end{pgfscope}%
\begin{pgfscope}%
\pgfsys@transformshift{0.714259in}{1.330817in}%
\pgfsys@useobject{currentmarker}{}%
\end{pgfscope}%
\begin{pgfscope}%
\pgfsys@transformshift{0.672283in}{1.479209in}%
\pgfsys@useobject{currentmarker}{}%
\end{pgfscope}%
\begin{pgfscope}%
\pgfsys@transformshift{0.863562in}{0.982765in}%
\pgfsys@useobject{currentmarker}{}%
\end{pgfscope}%
\begin{pgfscope}%
\pgfsys@transformshift{0.962792in}{1.601582in}%
\pgfsys@useobject{currentmarker}{}%
\end{pgfscope}%
\begin{pgfscope}%
\pgfsys@transformshift{0.918694in}{1.317277in}%
\pgfsys@useobject{currentmarker}{}%
\end{pgfscope}%
\begin{pgfscope}%
\pgfsys@transformshift{0.922564in}{1.478162in}%
\pgfsys@useobject{currentmarker}{}%
\end{pgfscope}%
\begin{pgfscope}%
\pgfsys@transformshift{1.326620in}{1.469995in}%
\pgfsys@useobject{currentmarker}{}%
\end{pgfscope}%
\begin{pgfscope}%
\pgfsys@transformshift{1.287037in}{1.159250in}%
\pgfsys@useobject{currentmarker}{}%
\end{pgfscope}%
\begin{pgfscope}%
\pgfsys@transformshift{0.679563in}{1.264290in}%
\pgfsys@useobject{currentmarker}{}%
\end{pgfscope}%
\begin{pgfscope}%
\pgfsys@transformshift{0.544433in}{1.342023in}%
\pgfsys@useobject{currentmarker}{}%
\end{pgfscope}%
\begin{pgfscope}%
\pgfsys@transformshift{0.576550in}{1.491404in}%
\pgfsys@useobject{currentmarker}{}%
\end{pgfscope}%
\begin{pgfscope}%
\pgfsys@transformshift{1.057937in}{1.432931in}%
\pgfsys@useobject{currentmarker}{}%
\end{pgfscope}%
\begin{pgfscope}%
\pgfsys@transformshift{0.796621in}{1.352644in}%
\pgfsys@useobject{currentmarker}{}%
\end{pgfscope}%
\begin{pgfscope}%
\pgfsys@transformshift{0.889416in}{1.325330in}%
\pgfsys@useobject{currentmarker}{}%
\end{pgfscope}%
\begin{pgfscope}%
\pgfsys@transformshift{0.772702in}{1.166147in}%
\pgfsys@useobject{currentmarker}{}%
\end{pgfscope}%
\begin{pgfscope}%
\pgfsys@transformshift{0.629318in}{1.417220in}%
\pgfsys@useobject{currentmarker}{}%
\end{pgfscope}%
\begin{pgfscope}%
\pgfsys@transformshift{0.527050in}{1.190577in}%
\pgfsys@useobject{currentmarker}{}%
\end{pgfscope}%
\begin{pgfscope}%
\pgfsys@transformshift{0.719691in}{1.325015in}%
\pgfsys@useobject{currentmarker}{}%
\end{pgfscope}%
\begin{pgfscope}%
\pgfsys@transformshift{1.172429in}{1.291421in}%
\pgfsys@useobject{currentmarker}{}%
\end{pgfscope}%
\begin{pgfscope}%
\pgfsys@transformshift{0.726870in}{1.084570in}%
\pgfsys@useobject{currentmarker}{}%
\end{pgfscope}%
\begin{pgfscope}%
\pgfsys@transformshift{0.602131in}{1.287178in}%
\pgfsys@useobject{currentmarker}{}%
\end{pgfscope}%
\begin{pgfscope}%
\pgfsys@transformshift{0.777116in}{1.170731in}%
\pgfsys@useobject{currentmarker}{}%
\end{pgfscope}%
\begin{pgfscope}%
\pgfsys@transformshift{0.911342in}{1.343180in}%
\pgfsys@useobject{currentmarker}{}%
\end{pgfscope}%
\begin{pgfscope}%
\pgfsys@transformshift{0.718945in}{1.232902in}%
\pgfsys@useobject{currentmarker}{}%
\end{pgfscope}%
\begin{pgfscope}%
\pgfsys@transformshift{0.882952in}{1.412151in}%
\pgfsys@useobject{currentmarker}{}%
\end{pgfscope}%
\begin{pgfscope}%
\pgfsys@transformshift{0.888026in}{1.203880in}%
\pgfsys@useobject{currentmarker}{}%
\end{pgfscope}%
\begin{pgfscope}%
\pgfsys@transformshift{0.832076in}{1.528407in}%
\pgfsys@useobject{currentmarker}{}%
\end{pgfscope}%
\begin{pgfscope}%
\pgfsys@transformshift{0.622611in}{1.383834in}%
\pgfsys@useobject{currentmarker}{}%
\end{pgfscope}%
\begin{pgfscope}%
\pgfsys@transformshift{0.939045in}{1.431661in}%
\pgfsys@useobject{currentmarker}{}%
\end{pgfscope}%
\begin{pgfscope}%
\pgfsys@transformshift{0.775067in}{1.514333in}%
\pgfsys@useobject{currentmarker}{}%
\end{pgfscope}%
\begin{pgfscope}%
\pgfsys@transformshift{1.199573in}{1.297907in}%
\pgfsys@useobject{currentmarker}{}%
\end{pgfscope}%
\begin{pgfscope}%
\pgfsys@transformshift{1.056447in}{1.358414in}%
\pgfsys@useobject{currentmarker}{}%
\end{pgfscope}%
\begin{pgfscope}%
\pgfsys@transformshift{0.885518in}{1.468721in}%
\pgfsys@useobject{currentmarker}{}%
\end{pgfscope}%
\begin{pgfscope}%
\pgfsys@transformshift{0.602489in}{1.282875in}%
\pgfsys@useobject{currentmarker}{}%
\end{pgfscope}%
\begin{pgfscope}%
\pgfsys@transformshift{0.991139in}{1.312246in}%
\pgfsys@useobject{currentmarker}{}%
\end{pgfscope}%
\begin{pgfscope}%
\pgfsys@transformshift{0.976578in}{1.081749in}%
\pgfsys@useobject{currentmarker}{}%
\end{pgfscope}%
\begin{pgfscope}%
\pgfsys@transformshift{0.574988in}{1.390463in}%
\pgfsys@useobject{currentmarker}{}%
\end{pgfscope}%
\begin{pgfscope}%
\pgfsys@transformshift{0.591813in}{1.318140in}%
\pgfsys@useobject{currentmarker}{}%
\end{pgfscope}%
\begin{pgfscope}%
\pgfsys@transformshift{0.904263in}{1.378781in}%
\pgfsys@useobject{currentmarker}{}%
\end{pgfscope}%
\begin{pgfscope}%
\pgfsys@transformshift{1.239170in}{1.319981in}%
\pgfsys@useobject{currentmarker}{}%
\end{pgfscope}%
\begin{pgfscope}%
\pgfsys@transformshift{0.560570in}{1.289617in}%
\pgfsys@useobject{currentmarker}{}%
\end{pgfscope}%
\begin{pgfscope}%
\pgfsys@transformshift{0.896065in}{1.419826in}%
\pgfsys@useobject{currentmarker}{}%
\end{pgfscope}%
\begin{pgfscope}%
\pgfsys@transformshift{0.972193in}{1.395932in}%
\pgfsys@useobject{currentmarker}{}%
\end{pgfscope}%
\begin{pgfscope}%
\pgfsys@transformshift{0.868406in}{1.473708in}%
\pgfsys@useobject{currentmarker}{}%
\end{pgfscope}%
\begin{pgfscope}%
\pgfsys@transformshift{1.121654in}{1.420262in}%
\pgfsys@useobject{currentmarker}{}%
\end{pgfscope}%
\begin{pgfscope}%
\pgfsys@transformshift{1.239614in}{1.452156in}%
\pgfsys@useobject{currentmarker}{}%
\end{pgfscope}%
\begin{pgfscope}%
\pgfsys@transformshift{0.642130in}{1.013048in}%
\pgfsys@useobject{currentmarker}{}%
\end{pgfscope}%
\begin{pgfscope}%
\pgfsys@transformshift{0.883855in}{0.913730in}%
\pgfsys@useobject{currentmarker}{}%
\end{pgfscope}%
\begin{pgfscope}%
\pgfsys@transformshift{1.091687in}{1.523273in}%
\pgfsys@useobject{currentmarker}{}%
\end{pgfscope}%
\begin{pgfscope}%
\pgfsys@transformshift{0.595754in}{1.276969in}%
\pgfsys@useobject{currentmarker}{}%
\end{pgfscope}%
\begin{pgfscope}%
\pgfsys@transformshift{0.708971in}{1.523461in}%
\pgfsys@useobject{currentmarker}{}%
\end{pgfscope}%
\begin{pgfscope}%
\pgfsys@transformshift{0.753570in}{1.264399in}%
\pgfsys@useobject{currentmarker}{}%
\end{pgfscope}%
\begin{pgfscope}%
\pgfsys@transformshift{0.684321in}{1.166576in}%
\pgfsys@useobject{currentmarker}{}%
\end{pgfscope}%
\begin{pgfscope}%
\pgfsys@transformshift{0.788452in}{0.852496in}%
\pgfsys@useobject{currentmarker}{}%
\end{pgfscope}%
\begin{pgfscope}%
\pgfsys@transformshift{0.599795in}{1.308485in}%
\pgfsys@useobject{currentmarker}{}%
\end{pgfscope}%
\begin{pgfscope}%
\pgfsys@transformshift{1.019816in}{1.612350in}%
\pgfsys@useobject{currentmarker}{}%
\end{pgfscope}%
\begin{pgfscope}%
\pgfsys@transformshift{0.613553in}{1.219700in}%
\pgfsys@useobject{currentmarker}{}%
\end{pgfscope}%
\begin{pgfscope}%
\pgfsys@transformshift{0.934230in}{1.175415in}%
\pgfsys@useobject{currentmarker}{}%
\end{pgfscope}%
\begin{pgfscope}%
\pgfsys@transformshift{0.960986in}{1.237354in}%
\pgfsys@useobject{currentmarker}{}%
\end{pgfscope}%
\begin{pgfscope}%
\pgfsys@transformshift{1.001730in}{1.212897in}%
\pgfsys@useobject{currentmarker}{}%
\end{pgfscope}%
\begin{pgfscope}%
\pgfsys@transformshift{0.636498in}{1.164132in}%
\pgfsys@useobject{currentmarker}{}%
\end{pgfscope}%
\begin{pgfscope}%
\pgfsys@transformshift{1.114373in}{1.438810in}%
\pgfsys@useobject{currentmarker}{}%
\end{pgfscope}%
\begin{pgfscope}%
\pgfsys@transformshift{1.248471in}{1.047130in}%
\pgfsys@useobject{currentmarker}{}%
\end{pgfscope}%
\begin{pgfscope}%
\pgfsys@transformshift{0.677442in}{1.335465in}%
\pgfsys@useobject{currentmarker}{}%
\end{pgfscope}%
\begin{pgfscope}%
\pgfsys@transformshift{0.574830in}{1.343718in}%
\pgfsys@useobject{currentmarker}{}%
\end{pgfscope}%
\begin{pgfscope}%
\pgfsys@transformshift{0.529902in}{1.190704in}%
\pgfsys@useobject{currentmarker}{}%
\end{pgfscope}%
\begin{pgfscope}%
\pgfsys@transformshift{0.717440in}{1.601429in}%
\pgfsys@useobject{currentmarker}{}%
\end{pgfscope}%
\begin{pgfscope}%
\pgfsys@transformshift{0.674088in}{1.320979in}%
\pgfsys@useobject{currentmarker}{}%
\end{pgfscope}%
\begin{pgfscope}%
\pgfsys@transformshift{0.697276in}{1.455141in}%
\pgfsys@useobject{currentmarker}{}%
\end{pgfscope}%
\begin{pgfscope}%
\pgfsys@transformshift{0.625591in}{1.390905in}%
\pgfsys@useobject{currentmarker}{}%
\end{pgfscope}%
\begin{pgfscope}%
\pgfsys@transformshift{0.939675in}{1.202823in}%
\pgfsys@useobject{currentmarker}{}%
\end{pgfscope}%
\begin{pgfscope}%
\pgfsys@transformshift{1.204202in}{0.957089in}%
\pgfsys@useobject{currentmarker}{}%
\end{pgfscope}%
\begin{pgfscope}%
\pgfsys@transformshift{0.947844in}{1.255752in}%
\pgfsys@useobject{currentmarker}{}%
\end{pgfscope}%
\begin{pgfscope}%
\pgfsys@transformshift{0.967134in}{0.933371in}%
\pgfsys@useobject{currentmarker}{}%
\end{pgfscope}%
\begin{pgfscope}%
\pgfsys@transformshift{0.985836in}{1.319014in}%
\pgfsys@useobject{currentmarker}{}%
\end{pgfscope}%
\begin{pgfscope}%
\pgfsys@transformshift{0.894346in}{1.490369in}%
\pgfsys@useobject{currentmarker}{}%
\end{pgfscope}%
\begin{pgfscope}%
\pgfsys@transformshift{0.644079in}{1.277096in}%
\pgfsys@useobject{currentmarker}{}%
\end{pgfscope}%
\begin{pgfscope}%
\pgfsys@transformshift{1.014972in}{1.329362in}%
\pgfsys@useobject{currentmarker}{}%
\end{pgfscope}%
\begin{pgfscope}%
\pgfsys@transformshift{0.529443in}{1.583971in}%
\pgfsys@useobject{currentmarker}{}%
\end{pgfscope}%
\begin{pgfscope}%
\pgfsys@transformshift{0.930632in}{1.414295in}%
\pgfsys@useobject{currentmarker}{}%
\end{pgfscope}%
\begin{pgfscope}%
\pgfsys@transformshift{0.700200in}{1.004049in}%
\pgfsys@useobject{currentmarker}{}%
\end{pgfscope}%
\begin{pgfscope}%
\pgfsys@transformshift{1.067123in}{1.064629in}%
\pgfsys@useobject{currentmarker}{}%
\end{pgfscope}%
\begin{pgfscope}%
\pgfsys@transformshift{0.543215in}{0.997949in}%
\pgfsys@useobject{currentmarker}{}%
\end{pgfscope}%
\begin{pgfscope}%
\pgfsys@transformshift{0.642832in}{1.297770in}%
\pgfsys@useobject{currentmarker}{}%
\end{pgfscope}%
\begin{pgfscope}%
\pgfsys@transformshift{0.726684in}{0.727400in}%
\pgfsys@useobject{currentmarker}{}%
\end{pgfscope}%
\begin{pgfscope}%
\pgfsys@transformshift{0.654541in}{1.095876in}%
\pgfsys@useobject{currentmarker}{}%
\end{pgfscope}%
\begin{pgfscope}%
\pgfsys@transformshift{0.809805in}{1.084929in}%
\pgfsys@useobject{currentmarker}{}%
\end{pgfscope}%
\begin{pgfscope}%
\pgfsys@transformshift{0.885690in}{1.193432in}%
\pgfsys@useobject{currentmarker}{}%
\end{pgfscope}%
\begin{pgfscope}%
\pgfsys@transformshift{1.269853in}{1.451278in}%
\pgfsys@useobject{currentmarker}{}%
\end{pgfscope}%
\begin{pgfscope}%
\pgfsys@transformshift{1.327924in}{0.806701in}%
\pgfsys@useobject{currentmarker}{}%
\end{pgfscope}%
\begin{pgfscope}%
\pgfsys@transformshift{0.832148in}{1.349396in}%
\pgfsys@useobject{currentmarker}{}%
\end{pgfscope}%
\begin{pgfscope}%
\pgfsys@transformshift{0.673200in}{1.272643in}%
\pgfsys@useobject{currentmarker}{}%
\end{pgfscope}%
\begin{pgfscope}%
\pgfsys@transformshift{0.625548in}{1.133681in}%
\pgfsys@useobject{currentmarker}{}%
\end{pgfscope}%
\begin{pgfscope}%
\pgfsys@transformshift{0.676324in}{1.454165in}%
\pgfsys@useobject{currentmarker}{}%
\end{pgfscope}%
\begin{pgfscope}%
\pgfsys@transformshift{0.577094in}{1.426414in}%
\pgfsys@useobject{currentmarker}{}%
\end{pgfscope}%
\begin{pgfscope}%
\pgfsys@transformshift{0.707337in}{1.307192in}%
\pgfsys@useobject{currentmarker}{}%
\end{pgfscope}%
\begin{pgfscope}%
\pgfsys@transformshift{0.636641in}{1.279970in}%
\pgfsys@useobject{currentmarker}{}%
\end{pgfscope}%
\begin{pgfscope}%
\pgfsys@transformshift{0.841893in}{1.379455in}%
\pgfsys@useobject{currentmarker}{}%
\end{pgfscope}%
\begin{pgfscope}%
\pgfsys@transformshift{0.911486in}{1.394868in}%
\pgfsys@useobject{currentmarker}{}%
\end{pgfscope}%
\begin{pgfscope}%
\pgfsys@transformshift{0.630908in}{1.413812in}%
\pgfsys@useobject{currentmarker}{}%
\end{pgfscope}%
\begin{pgfscope}%
\pgfsys@transformshift{0.668198in}{1.295549in}%
\pgfsys@useobject{currentmarker}{}%
\end{pgfscope}%
\begin{pgfscope}%
\pgfsys@transformshift{0.583572in}{1.467181in}%
\pgfsys@useobject{currentmarker}{}%
\end{pgfscope}%
\begin{pgfscope}%
\pgfsys@transformshift{0.672870in}{1.357969in}%
\pgfsys@useobject{currentmarker}{}%
\end{pgfscope}%
\begin{pgfscope}%
\pgfsys@transformshift{0.761624in}{0.988106in}%
\pgfsys@useobject{currentmarker}{}%
\end{pgfscope}%
\begin{pgfscope}%
\pgfsys@transformshift{0.981365in}{1.415152in}%
\pgfsys@useobject{currentmarker}{}%
\end{pgfscope}%
\begin{pgfscope}%
\pgfsys@transformshift{0.623298in}{1.171997in}%
\pgfsys@useobject{currentmarker}{}%
\end{pgfscope}%
\begin{pgfscope}%
\pgfsys@transformshift{1.154358in}{1.481198in}%
\pgfsys@useobject{currentmarker}{}%
\end{pgfscope}%
\begin{pgfscope}%
\pgfsys@transformshift{0.911400in}{1.246883in}%
\pgfsys@useobject{currentmarker}{}%
\end{pgfscope}%
\begin{pgfscope}%
\pgfsys@transformshift{1.297197in}{0.989553in}%
\pgfsys@useobject{currentmarker}{}%
\end{pgfscope}%
\begin{pgfscope}%
\pgfsys@transformshift{0.534158in}{1.453107in}%
\pgfsys@useobject{currentmarker}{}%
\end{pgfscope}%
\begin{pgfscope}%
\pgfsys@transformshift{1.296782in}{0.998132in}%
\pgfsys@useobject{currentmarker}{}%
\end{pgfscope}%
\begin{pgfscope}%
\pgfsys@transformshift{1.321145in}{1.025534in}%
\pgfsys@useobject{currentmarker}{}%
\end{pgfscope}%
\begin{pgfscope}%
\pgfsys@transformshift{0.945078in}{1.473831in}%
\pgfsys@useobject{currentmarker}{}%
\end{pgfscope}%
\begin{pgfscope}%
\pgfsys@transformshift{0.573325in}{1.459319in}%
\pgfsys@useobject{currentmarker}{}%
\end{pgfscope}%
\begin{pgfscope}%
\pgfsys@transformshift{1.105832in}{1.020117in}%
\pgfsys@useobject{currentmarker}{}%
\end{pgfscope}%
\begin{pgfscope}%
\pgfsys@transformshift{1.100959in}{1.317333in}%
\pgfsys@useobject{currentmarker}{}%
\end{pgfscope}%
\begin{pgfscope}%
\pgfsys@transformshift{0.820898in}{1.039757in}%
\pgfsys@useobject{currentmarker}{}%
\end{pgfscope}%
\begin{pgfscope}%
\pgfsys@transformshift{0.560986in}{1.290098in}%
\pgfsys@useobject{currentmarker}{}%
\end{pgfscope}%
\begin{pgfscope}%
\pgfsys@transformshift{0.626881in}{1.308734in}%
\pgfsys@useobject{currentmarker}{}%
\end{pgfscope}%
\begin{pgfscope}%
\pgfsys@transformshift{1.104700in}{1.420046in}%
\pgfsys@useobject{currentmarker}{}%
\end{pgfscope}%
\begin{pgfscope}%
\pgfsys@transformshift{0.620289in}{1.498688in}%
\pgfsys@useobject{currentmarker}{}%
\end{pgfscope}%
\begin{pgfscope}%
\pgfsys@transformshift{1.159632in}{1.357746in}%
\pgfsys@useobject{currentmarker}{}%
\end{pgfscope}%
\begin{pgfscope}%
\pgfsys@transformshift{0.641771in}{1.621150in}%
\pgfsys@useobject{currentmarker}{}%
\end{pgfscope}%
\begin{pgfscope}%
\pgfsys@transformshift{1.597036in}{1.026500in}%
\pgfsys@useobject{currentmarker}{}%
\end{pgfscope}%
\begin{pgfscope}%
\pgfsys@transformshift{1.214821in}{1.072494in}%
\pgfsys@useobject{currentmarker}{}%
\end{pgfscope}%
\begin{pgfscope}%
\pgfsys@transformshift{0.689810in}{1.301930in}%
\pgfsys@useobject{currentmarker}{}%
\end{pgfscope}%
\begin{pgfscope}%
\pgfsys@transformshift{0.752753in}{1.490161in}%
\pgfsys@useobject{currentmarker}{}%
\end{pgfscope}%
\begin{pgfscope}%
\pgfsys@transformshift{1.018827in}{1.230298in}%
\pgfsys@useobject{currentmarker}{}%
\end{pgfscope}%
\begin{pgfscope}%
\pgfsys@transformshift{0.983099in}{1.453252in}%
\pgfsys@useobject{currentmarker}{}%
\end{pgfscope}%
\begin{pgfscope}%
\pgfsys@transformshift{1.078775in}{1.321417in}%
\pgfsys@useobject{currentmarker}{}%
\end{pgfscope}%
\begin{pgfscope}%
\pgfsys@transformshift{0.676596in}{1.436107in}%
\pgfsys@useobject{currentmarker}{}%
\end{pgfscope}%
\begin{pgfscope}%
\pgfsys@transformshift{0.565472in}{1.444655in}%
\pgfsys@useobject{currentmarker}{}%
\end{pgfscope}%
\begin{pgfscope}%
\pgfsys@transformshift{0.917620in}{1.426535in}%
\pgfsys@useobject{currentmarker}{}%
\end{pgfscope}%
\begin{pgfscope}%
\pgfsys@transformshift{0.849661in}{1.457828in}%
\pgfsys@useobject{currentmarker}{}%
\end{pgfscope}%
\begin{pgfscope}%
\pgfsys@transformshift{0.945451in}{1.327224in}%
\pgfsys@useobject{currentmarker}{}%
\end{pgfscope}%
\begin{pgfscope}%
\pgfsys@transformshift{0.793396in}{1.589755in}%
\pgfsys@useobject{currentmarker}{}%
\end{pgfscope}%
\begin{pgfscope}%
\pgfsys@transformshift{1.059270in}{1.483299in}%
\pgfsys@useobject{currentmarker}{}%
\end{pgfscope}%
\begin{pgfscope}%
\pgfsys@transformshift{0.955296in}{1.421331in}%
\pgfsys@useobject{currentmarker}{}%
\end{pgfscope}%
\begin{pgfscope}%
\pgfsys@transformshift{0.751864in}{1.335741in}%
\pgfsys@useobject{currentmarker}{}%
\end{pgfscope}%
\begin{pgfscope}%
\pgfsys@transformshift{1.178535in}{1.343515in}%
\pgfsys@useobject{currentmarker}{}%
\end{pgfscope}%
\begin{pgfscope}%
\pgfsys@transformshift{0.574873in}{1.493929in}%
\pgfsys@useobject{currentmarker}{}%
\end{pgfscope}%
\begin{pgfscope}%
\pgfsys@transformshift{0.652147in}{1.089886in}%
\pgfsys@useobject{currentmarker}{}%
\end{pgfscope}%
\begin{pgfscope}%
\pgfsys@transformshift{1.092160in}{1.305093in}%
\pgfsys@useobject{currentmarker}{}%
\end{pgfscope}%
\begin{pgfscope}%
\pgfsys@transformshift{0.707452in}{1.080873in}%
\pgfsys@useobject{currentmarker}{}%
\end{pgfscope}%
\begin{pgfscope}%
\pgfsys@transformshift{0.703038in}{0.909606in}%
\pgfsys@useobject{currentmarker}{}%
\end{pgfscope}%
\begin{pgfscope}%
\pgfsys@transformshift{1.308132in}{1.009730in}%
\pgfsys@useobject{currentmarker}{}%
\end{pgfscope}%
\begin{pgfscope}%
\pgfsys@transformshift{0.717269in}{0.886070in}%
\pgfsys@useobject{currentmarker}{}%
\end{pgfscope}%
\begin{pgfscope}%
\pgfsys@transformshift{0.605155in}{1.359816in}%
\pgfsys@useobject{currentmarker}{}%
\end{pgfscope}%
\begin{pgfscope}%
\pgfsys@transformshift{0.931865in}{1.077613in}%
\pgfsys@useobject{currentmarker}{}%
\end{pgfscope}%
\begin{pgfscope}%
\pgfsys@transformshift{0.911428in}{1.502997in}%
\pgfsys@useobject{currentmarker}{}%
\end{pgfscope}%
\begin{pgfscope}%
\pgfsys@transformshift{0.641227in}{1.422358in}%
\pgfsys@useobject{currentmarker}{}%
\end{pgfscope}%
\begin{pgfscope}%
\pgfsys@transformshift{0.902385in}{1.320350in}%
\pgfsys@useobject{currentmarker}{}%
\end{pgfscope}%
\begin{pgfscope}%
\pgfsys@transformshift{0.547945in}{1.418209in}%
\pgfsys@useobject{currentmarker}{}%
\end{pgfscope}%
\begin{pgfscope}%
\pgfsys@transformshift{0.587413in}{1.394607in}%
\pgfsys@useobject{currentmarker}{}%
\end{pgfscope}%
\begin{pgfscope}%
\pgfsys@transformshift{0.715276in}{1.368904in}%
\pgfsys@useobject{currentmarker}{}%
\end{pgfscope}%
\begin{pgfscope}%
\pgfsys@transformshift{0.880315in}{1.369273in}%
\pgfsys@useobject{currentmarker}{}%
\end{pgfscope}%
\begin{pgfscope}%
\pgfsys@transformshift{1.423485in}{0.803820in}%
\pgfsys@useobject{currentmarker}{}%
\end{pgfscope}%
\begin{pgfscope}%
\pgfsys@transformshift{0.714675in}{1.074127in}%
\pgfsys@useobject{currentmarker}{}%
\end{pgfscope}%
\begin{pgfscope}%
\pgfsys@transformshift{1.149499in}{1.365638in}%
\pgfsys@useobject{currentmarker}{}%
\end{pgfscope}%
\begin{pgfscope}%
\pgfsys@transformshift{0.826874in}{1.334864in}%
\pgfsys@useobject{currentmarker}{}%
\end{pgfscope}%
\begin{pgfscope}%
\pgfsys@transformshift{0.723187in}{0.523238in}%
\pgfsys@useobject{currentmarker}{}%
\end{pgfscope}%
\begin{pgfscope}%
\pgfsys@transformshift{0.785328in}{0.814977in}%
\pgfsys@useobject{currentmarker}{}%
\end{pgfscope}%
\begin{pgfscope}%
\pgfsys@transformshift{0.978556in}{1.154445in}%
\pgfsys@useobject{currentmarker}{}%
\end{pgfscope}%
\begin{pgfscope}%
\pgfsys@transformshift{1.289717in}{1.480759in}%
\pgfsys@useobject{currentmarker}{}%
\end{pgfscope}%
\begin{pgfscope}%
\pgfsys@transformshift{1.266915in}{1.327805in}%
\pgfsys@useobject{currentmarker}{}%
\end{pgfscope}%
\begin{pgfscope}%
\pgfsys@transformshift{0.966360in}{1.655738in}%
\pgfsys@useobject{currentmarker}{}%
\end{pgfscope}%
\begin{pgfscope}%
\pgfsys@transformshift{0.652362in}{1.347967in}%
\pgfsys@useobject{currentmarker}{}%
\end{pgfscope}%
\begin{pgfscope}%
\pgfsys@transformshift{0.909164in}{1.500260in}%
\pgfsys@useobject{currentmarker}{}%
\end{pgfscope}%
\begin{pgfscope}%
\pgfsys@transformshift{0.833079in}{1.529438in}%
\pgfsys@useobject{currentmarker}{}%
\end{pgfscope}%
\begin{pgfscope}%
\pgfsys@transformshift{0.816985in}{1.141374in}%
\pgfsys@useobject{currentmarker}{}%
\end{pgfscope}%
\begin{pgfscope}%
\pgfsys@transformshift{0.580548in}{1.274045in}%
\pgfsys@useobject{currentmarker}{}%
\end{pgfscope}%
\begin{pgfscope}%
\pgfsys@transformshift{0.927852in}{1.402264in}%
\pgfsys@useobject{currentmarker}{}%
\end{pgfscope}%
\begin{pgfscope}%
\pgfsys@transformshift{0.962376in}{1.130914in}%
\pgfsys@useobject{currentmarker}{}%
\end{pgfscope}%
\begin{pgfscope}%
\pgfsys@transformshift{0.778836in}{1.414355in}%
\pgfsys@useobject{currentmarker}{}%
\end{pgfscope}%
\begin{pgfscope}%
\pgfsys@transformshift{0.562563in}{1.526187in}%
\pgfsys@useobject{currentmarker}{}%
\end{pgfscope}%
\begin{pgfscope}%
\pgfsys@transformshift{1.015502in}{1.116403in}%
\pgfsys@useobject{currentmarker}{}%
\end{pgfscope}%
\begin{pgfscope}%
\pgfsys@transformshift{1.104127in}{1.427795in}%
\pgfsys@useobject{currentmarker}{}%
\end{pgfscope}%
\begin{pgfscope}%
\pgfsys@transformshift{1.176657in}{1.516865in}%
\pgfsys@useobject{currentmarker}{}%
\end{pgfscope}%
\begin{pgfscope}%
\pgfsys@transformshift{1.175740in}{1.408818in}%
\pgfsys@useobject{currentmarker}{}%
\end{pgfscope}%
\begin{pgfscope}%
\pgfsys@transformshift{0.806337in}{1.526934in}%
\pgfsys@useobject{currentmarker}{}%
\end{pgfscope}%
\begin{pgfscope}%
\pgfsys@transformshift{0.942943in}{1.391970in}%
\pgfsys@useobject{currentmarker}{}%
\end{pgfscope}%
\begin{pgfscope}%
\pgfsys@transformshift{0.645498in}{1.366295in}%
\pgfsys@useobject{currentmarker}{}%
\end{pgfscope}%
\begin{pgfscope}%
\pgfsys@transformshift{0.976750in}{1.029329in}%
\pgfsys@useobject{currentmarker}{}%
\end{pgfscope}%
\begin{pgfscope}%
\pgfsys@transformshift{0.917204in}{1.409818in}%
\pgfsys@useobject{currentmarker}{}%
\end{pgfscope}%
\begin{pgfscope}%
\pgfsys@transformshift{0.698136in}{1.587613in}%
\pgfsys@useobject{currentmarker}{}%
\end{pgfscope}%
\begin{pgfscope}%
\pgfsys@transformshift{0.831546in}{1.491276in}%
\pgfsys@useobject{currentmarker}{}%
\end{pgfscope}%
\begin{pgfscope}%
\pgfsys@transformshift{0.684292in}{1.091890in}%
\pgfsys@useobject{currentmarker}{}%
\end{pgfscope}%
\begin{pgfscope}%
\pgfsys@transformshift{0.926132in}{1.066709in}%
\pgfsys@useobject{currentmarker}{}%
\end{pgfscope}%
\begin{pgfscope}%
\pgfsys@transformshift{0.912274in}{1.404058in}%
\pgfsys@useobject{currentmarker}{}%
\end{pgfscope}%
\begin{pgfscope}%
\pgfsys@transformshift{0.953820in}{1.183003in}%
\pgfsys@useobject{currentmarker}{}%
\end{pgfscope}%
\begin{pgfscope}%
\pgfsys@transformshift{1.009196in}{1.166322in}%
\pgfsys@useobject{currentmarker}{}%
\end{pgfscope}%
\begin{pgfscope}%
\pgfsys@transformshift{0.741574in}{1.409659in}%
\pgfsys@useobject{currentmarker}{}%
\end{pgfscope}%
\begin{pgfscope}%
\pgfsys@transformshift{1.381910in}{0.965699in}%
\pgfsys@useobject{currentmarker}{}%
\end{pgfscope}%
\begin{pgfscope}%
\pgfsys@transformshift{1.798089in}{0.952938in}%
\pgfsys@useobject{currentmarker}{}%
\end{pgfscope}%
\begin{pgfscope}%
\pgfsys@transformshift{0.649052in}{1.354652in}%
\pgfsys@useobject{currentmarker}{}%
\end{pgfscope}%
\begin{pgfscope}%
\pgfsys@transformshift{1.014943in}{1.132915in}%
\pgfsys@useobject{currentmarker}{}%
\end{pgfscope}%
\begin{pgfscope}%
\pgfsys@transformshift{0.610257in}{1.260581in}%
\pgfsys@useobject{currentmarker}{}%
\end{pgfscope}%
\begin{pgfscope}%
\pgfsys@transformshift{0.801880in}{1.279277in}%
\pgfsys@useobject{currentmarker}{}%
\end{pgfscope}%
\begin{pgfscope}%
\pgfsys@transformshift{0.723202in}{1.100704in}%
\pgfsys@useobject{currentmarker}{}%
\end{pgfscope}%
\begin{pgfscope}%
\pgfsys@transformshift{0.813145in}{1.175037in}%
\pgfsys@useobject{currentmarker}{}%
\end{pgfscope}%
\begin{pgfscope}%
\pgfsys@transformshift{1.018153in}{1.591807in}%
\pgfsys@useobject{currentmarker}{}%
\end{pgfscope}%
\begin{pgfscope}%
\pgfsys@transformshift{0.770796in}{1.469515in}%
\pgfsys@useobject{currentmarker}{}%
\end{pgfscope}%
\begin{pgfscope}%
\pgfsys@transformshift{1.282293in}{1.569884in}%
\pgfsys@useobject{currentmarker}{}%
\end{pgfscope}%
\begin{pgfscope}%
\pgfsys@transformshift{0.983715in}{1.288917in}%
\pgfsys@useobject{currentmarker}{}%
\end{pgfscope}%
\begin{pgfscope}%
\pgfsys@transformshift{0.726913in}{1.453408in}%
\pgfsys@useobject{currentmarker}{}%
\end{pgfscope}%
\begin{pgfscope}%
\pgfsys@transformshift{0.997731in}{1.081833in}%
\pgfsys@useobject{currentmarker}{}%
\end{pgfscope}%
\begin{pgfscope}%
\pgfsys@transformshift{0.629160in}{1.307755in}%
\pgfsys@useobject{currentmarker}{}%
\end{pgfscope}%
\begin{pgfscope}%
\pgfsys@transformshift{0.823349in}{1.171344in}%
\pgfsys@useobject{currentmarker}{}%
\end{pgfscope}%
\begin{pgfscope}%
\pgfsys@transformshift{1.015516in}{1.551902in}%
\pgfsys@useobject{currentmarker}{}%
\end{pgfscope}%
\begin{pgfscope}%
\pgfsys@transformshift{0.878839in}{1.584212in}%
\pgfsys@useobject{currentmarker}{}%
\end{pgfscope}%
\begin{pgfscope}%
\pgfsys@transformshift{0.796807in}{1.009854in}%
\pgfsys@useobject{currentmarker}{}%
\end{pgfscope}%
\begin{pgfscope}%
\pgfsys@transformshift{0.751033in}{1.127038in}%
\pgfsys@useobject{currentmarker}{}%
\end{pgfscope}%
\begin{pgfscope}%
\pgfsys@transformshift{0.750316in}{1.092810in}%
\pgfsys@useobject{currentmarker}{}%
\end{pgfscope}%
\begin{pgfscope}%
\pgfsys@transformshift{1.049782in}{1.482904in}%
\pgfsys@useobject{currentmarker}{}%
\end{pgfscope}%
\begin{pgfscope}%
\pgfsys@transformshift{1.068112in}{1.520934in}%
\pgfsys@useobject{currentmarker}{}%
\end{pgfscope}%
\begin{pgfscope}%
\pgfsys@transformshift{0.928655in}{1.303322in}%
\pgfsys@useobject{currentmarker}{}%
\end{pgfscope}%
\begin{pgfscope}%
\pgfsys@transformshift{0.980118in}{1.398608in}%
\pgfsys@useobject{currentmarker}{}%
\end{pgfscope}%
\begin{pgfscope}%
\pgfsys@transformshift{1.113256in}{1.087734in}%
\pgfsys@useobject{currentmarker}{}%
\end{pgfscope}%
\begin{pgfscope}%
\pgfsys@transformshift{0.575575in}{1.208008in}%
\pgfsys@useobject{currentmarker}{}%
\end{pgfscope}%
\begin{pgfscope}%
\pgfsys@transformshift{0.862645in}{1.480729in}%
\pgfsys@useobject{currentmarker}{}%
\end{pgfscope}%
\begin{pgfscope}%
\pgfsys@transformshift{0.943043in}{1.372656in}%
\pgfsys@useobject{currentmarker}{}%
\end{pgfscope}%
\begin{pgfscope}%
\pgfsys@transformshift{0.895377in}{1.423508in}%
\pgfsys@useobject{currentmarker}{}%
\end{pgfscope}%
\begin{pgfscope}%
\pgfsys@transformshift{0.871846in}{1.490347in}%
\pgfsys@useobject{currentmarker}{}%
\end{pgfscope}%
\begin{pgfscope}%
\pgfsys@transformshift{0.607362in}{1.388230in}%
\pgfsys@useobject{currentmarker}{}%
\end{pgfscope}%
\begin{pgfscope}%
\pgfsys@transformshift{0.853630in}{1.347465in}%
\pgfsys@useobject{currentmarker}{}%
\end{pgfscope}%
\begin{pgfscope}%
\pgfsys@transformshift{0.748955in}{0.915730in}%
\pgfsys@useobject{currentmarker}{}%
\end{pgfscope}%
\begin{pgfscope}%
\pgfsys@transformshift{0.669431in}{1.075185in}%
\pgfsys@useobject{currentmarker}{}%
\end{pgfscope}%
\begin{pgfscope}%
\pgfsys@transformshift{0.604926in}{1.096969in}%
\pgfsys@useobject{currentmarker}{}%
\end{pgfscope}%
\begin{pgfscope}%
\pgfsys@transformshift{0.770595in}{1.221378in}%
\pgfsys@useobject{currentmarker}{}%
\end{pgfscope}%
\begin{pgfscope}%
\pgfsys@transformshift{1.284371in}{1.411909in}%
\pgfsys@useobject{currentmarker}{}%
\end{pgfscope}%
\begin{pgfscope}%
\pgfsys@transformshift{0.863232in}{1.412654in}%
\pgfsys@useobject{currentmarker}{}%
\end{pgfscope}%
\begin{pgfscope}%
\pgfsys@transformshift{0.867661in}{1.403704in}%
\pgfsys@useobject{currentmarker}{}%
\end{pgfscope}%
\begin{pgfscope}%
\pgfsys@transformshift{0.792866in}{1.385716in}%
\pgfsys@useobject{currentmarker}{}%
\end{pgfscope}%
\begin{pgfscope}%
\pgfsys@transformshift{0.619372in}{0.830861in}%
\pgfsys@useobject{currentmarker}{}%
\end{pgfscope}%
\begin{pgfscope}%
\pgfsys@transformshift{0.655759in}{1.264926in}%
\pgfsys@useobject{currentmarker}{}%
\end{pgfscope}%
\begin{pgfscope}%
\pgfsys@transformshift{0.665017in}{0.973557in}%
\pgfsys@useobject{currentmarker}{}%
\end{pgfscope}%
\begin{pgfscope}%
\pgfsys@transformshift{1.205277in}{1.032600in}%
\pgfsys@useobject{currentmarker}{}%
\end{pgfscope}%
\begin{pgfscope}%
\pgfsys@transformshift{0.918121in}{1.387150in}%
\pgfsys@useobject{currentmarker}{}%
\end{pgfscope}%
\begin{pgfscope}%
\pgfsys@transformshift{0.801336in}{1.270228in}%
\pgfsys@useobject{currentmarker}{}%
\end{pgfscope}%
\begin{pgfscope}%
\pgfsys@transformshift{1.063970in}{1.221925in}%
\pgfsys@useobject{currentmarker}{}%
\end{pgfscope}%
\begin{pgfscope}%
\pgfsys@transformshift{1.012607in}{1.513542in}%
\pgfsys@useobject{currentmarker}{}%
\end{pgfscope}%
\begin{pgfscope}%
\pgfsys@transformshift{0.812457in}{1.552889in}%
\pgfsys@useobject{currentmarker}{}%
\end{pgfscope}%
\begin{pgfscope}%
\pgfsys@transformshift{0.858002in}{1.517773in}%
\pgfsys@useobject{currentmarker}{}%
\end{pgfscope}%
\begin{pgfscope}%
\pgfsys@transformshift{1.011690in}{1.278917in}%
\pgfsys@useobject{currentmarker}{}%
\end{pgfscope}%
\begin{pgfscope}%
\pgfsys@transformshift{0.660316in}{1.140889in}%
\pgfsys@useobject{currentmarker}{}%
\end{pgfscope}%
\begin{pgfscope}%
\pgfsys@transformshift{0.657034in}{1.387242in}%
\pgfsys@useobject{currentmarker}{}%
\end{pgfscope}%
\begin{pgfscope}%
\pgfsys@transformshift{1.005857in}{1.054041in}%
\pgfsys@useobject{currentmarker}{}%
\end{pgfscope}%
\begin{pgfscope}%
\pgfsys@transformshift{0.869553in}{1.031304in}%
\pgfsys@useobject{currentmarker}{}%
\end{pgfscope}%
\begin{pgfscope}%
\pgfsys@transformshift{1.235186in}{1.519307in}%
\pgfsys@useobject{currentmarker}{}%
\end{pgfscope}%
\begin{pgfscope}%
\pgfsys@transformshift{0.912690in}{1.429643in}%
\pgfsys@useobject{currentmarker}{}%
\end{pgfscope}%
\begin{pgfscope}%
\pgfsys@transformshift{0.588731in}{1.288192in}%
\pgfsys@useobject{currentmarker}{}%
\end{pgfscope}%
\begin{pgfscope}%
\pgfsys@transformshift{1.095170in}{1.170430in}%
\pgfsys@useobject{currentmarker}{}%
\end{pgfscope}%
\begin{pgfscope}%
\pgfsys@transformshift{0.660689in}{1.200180in}%
\pgfsys@useobject{currentmarker}{}%
\end{pgfscope}%
\begin{pgfscope}%
\pgfsys@transformshift{0.685367in}{0.923250in}%
\pgfsys@useobject{currentmarker}{}%
\end{pgfscope}%
\begin{pgfscope}%
\pgfsys@transformshift{0.587255in}{1.144274in}%
\pgfsys@useobject{currentmarker}{}%
\end{pgfscope}%
\begin{pgfscope}%
\pgfsys@transformshift{0.938644in}{1.364617in}%
\pgfsys@useobject{currentmarker}{}%
\end{pgfscope}%
\begin{pgfscope}%
\pgfsys@transformshift{0.713743in}{1.284397in}%
\pgfsys@useobject{currentmarker}{}%
\end{pgfscope}%
\begin{pgfscope}%
\pgfsys@transformshift{0.617738in}{1.243100in}%
\pgfsys@useobject{currentmarker}{}%
\end{pgfscope}%
\begin{pgfscope}%
\pgfsys@transformshift{0.679606in}{1.084224in}%
\pgfsys@useobject{currentmarker}{}%
\end{pgfscope}%
\begin{pgfscope}%
\pgfsys@transformshift{0.662910in}{1.413052in}%
\pgfsys@useobject{currentmarker}{}%
\end{pgfscope}%
\begin{pgfscope}%
\pgfsys@transformshift{0.950066in}{1.558025in}%
\pgfsys@useobject{currentmarker}{}%
\end{pgfscope}%
\begin{pgfscope}%
\pgfsys@transformshift{1.410758in}{0.889399in}%
\pgfsys@useobject{currentmarker}{}%
\end{pgfscope}%
\begin{pgfscope}%
\pgfsys@transformshift{0.651101in}{1.326612in}%
\pgfsys@useobject{currentmarker}{}%
\end{pgfscope}%
\begin{pgfscope}%
\pgfsys@transformshift{0.698036in}{1.209036in}%
\pgfsys@useobject{currentmarker}{}%
\end{pgfscope}%
\begin{pgfscope}%
\pgfsys@transformshift{0.942370in}{1.609854in}%
\pgfsys@useobject{currentmarker}{}%
\end{pgfscope}%
\begin{pgfscope}%
\pgfsys@transformshift{1.103138in}{1.585618in}%
\pgfsys@useobject{currentmarker}{}%
\end{pgfscope}%
\begin{pgfscope}%
\pgfsys@transformshift{0.754644in}{1.209004in}%
\pgfsys@useobject{currentmarker}{}%
\end{pgfscope}%
\begin{pgfscope}%
\pgfsys@transformshift{0.661104in}{0.954069in}%
\pgfsys@useobject{currentmarker}{}%
\end{pgfscope}%
\begin{pgfscope}%
\pgfsys@transformshift{1.128561in}{1.353549in}%
\pgfsys@useobject{currentmarker}{}%
\end{pgfscope}%
\begin{pgfscope}%
\pgfsys@transformshift{0.860381in}{1.461443in}%
\pgfsys@useobject{currentmarker}{}%
\end{pgfscope}%
\begin{pgfscope}%
\pgfsys@transformshift{0.869853in}{1.469268in}%
\pgfsys@useobject{currentmarker}{}%
\end{pgfscope}%
\begin{pgfscope}%
\pgfsys@transformshift{0.772215in}{0.957631in}%
\pgfsys@useobject{currentmarker}{}%
\end{pgfscope}%
\begin{pgfscope}%
\pgfsys@transformshift{0.694467in}{1.111015in}%
\pgfsys@useobject{currentmarker}{}%
\end{pgfscope}%
\begin{pgfscope}%
\pgfsys@transformshift{0.841793in}{1.423179in}%
\pgfsys@useobject{currentmarker}{}%
\end{pgfscope}%
\begin{pgfscope}%
\pgfsys@transformshift{0.690756in}{1.372684in}%
\pgfsys@useobject{currentmarker}{}%
\end{pgfscope}%
\begin{pgfscope}%
\pgfsys@transformshift{0.975991in}{1.550783in}%
\pgfsys@useobject{currentmarker}{}%
\end{pgfscope}%
\begin{pgfscope}%
\pgfsys@transformshift{0.650270in}{1.117811in}%
\pgfsys@useobject{currentmarker}{}%
\end{pgfscope}%
\begin{pgfscope}%
\pgfsys@transformshift{1.459384in}{1.373624in}%
\pgfsys@useobject{currentmarker}{}%
\end{pgfscope}%
\begin{pgfscope}%
\pgfsys@transformshift{0.645010in}{1.339317in}%
\pgfsys@useobject{currentmarker}{}%
\end{pgfscope}%
\begin{pgfscope}%
\pgfsys@transformshift{0.802883in}{1.499810in}%
\pgfsys@useobject{currentmarker}{}%
\end{pgfscope}%
\begin{pgfscope}%
\pgfsys@transformshift{0.739740in}{1.258993in}%
\pgfsys@useobject{currentmarker}{}%
\end{pgfscope}%
\begin{pgfscope}%
\pgfsys@transformshift{0.962591in}{1.587298in}%
\pgfsys@useobject{currentmarker}{}%
\end{pgfscope}%
\begin{pgfscope}%
\pgfsys@transformshift{0.741187in}{0.914799in}%
\pgfsys@useobject{currentmarker}{}%
\end{pgfscope}%
\begin{pgfscope}%
\pgfsys@transformshift{0.637028in}{1.405439in}%
\pgfsys@useobject{currentmarker}{}%
\end{pgfscope}%
\begin{pgfscope}%
\pgfsys@transformshift{1.142062in}{0.898589in}%
\pgfsys@useobject{currentmarker}{}%
\end{pgfscope}%
\begin{pgfscope}%
\pgfsys@transformshift{0.766468in}{0.930779in}%
\pgfsys@useobject{currentmarker}{}%
\end{pgfscope}%
\begin{pgfscope}%
\pgfsys@transformshift{1.005957in}{1.347497in}%
\pgfsys@useobject{currentmarker}{}%
\end{pgfscope}%
\begin{pgfscope}%
\pgfsys@transformshift{0.873694in}{1.647338in}%
\pgfsys@useobject{currentmarker}{}%
\end{pgfscope}%
\begin{pgfscope}%
\pgfsys@transformshift{0.682386in}{1.290926in}%
\pgfsys@useobject{currentmarker}{}%
\end{pgfscope}%
\begin{pgfscope}%
\pgfsys@transformshift{0.651316in}{1.014756in}%
\pgfsys@useobject{currentmarker}{}%
\end{pgfscope}%
\begin{pgfscope}%
\pgfsys@transformshift{0.572465in}{1.447274in}%
\pgfsys@useobject{currentmarker}{}%
\end{pgfscope}%
\begin{pgfscope}%
\pgfsys@transformshift{0.908534in}{1.351950in}%
\pgfsys@useobject{currentmarker}{}%
\end{pgfscope}%
\begin{pgfscope}%
\pgfsys@transformshift{1.159589in}{1.527196in}%
\pgfsys@useobject{currentmarker}{}%
\end{pgfscope}%
\begin{pgfscope}%
\pgfsys@transformshift{0.765536in}{1.424092in}%
\pgfsys@useobject{currentmarker}{}%
\end{pgfscope}%
\begin{pgfscope}%
\pgfsys@transformshift{0.806524in}{1.420047in}%
\pgfsys@useobject{currentmarker}{}%
\end{pgfscope}%
\begin{pgfscope}%
\pgfsys@transformshift{0.882422in}{1.092424in}%
\pgfsys@useobject{currentmarker}{}%
\end{pgfscope}%
\begin{pgfscope}%
\pgfsys@transformshift{0.929644in}{1.193153in}%
\pgfsys@useobject{currentmarker}{}%
\end{pgfscope}%
\begin{pgfscope}%
\pgfsys@transformshift{0.548618in}{1.249084in}%
\pgfsys@useobject{currentmarker}{}%
\end{pgfscope}%
\begin{pgfscope}%
\pgfsys@transformshift{0.619888in}{1.299074in}%
\pgfsys@useobject{currentmarker}{}%
\end{pgfscope}%
\begin{pgfscope}%
\pgfsys@transformshift{0.997846in}{1.202696in}%
\pgfsys@useobject{currentmarker}{}%
\end{pgfscope}%
\begin{pgfscope}%
\pgfsys@transformshift{0.598262in}{1.326678in}%
\pgfsys@useobject{currentmarker}{}%
\end{pgfscope}%
\begin{pgfscope}%
\pgfsys@transformshift{0.917204in}{1.277741in}%
\pgfsys@useobject{currentmarker}{}%
\end{pgfscope}%
\begin{pgfscope}%
\pgfsys@transformshift{0.860810in}{1.298810in}%
\pgfsys@useobject{currentmarker}{}%
\end{pgfscope}%
\begin{pgfscope}%
\pgfsys@transformshift{1.140886in}{1.431227in}%
\pgfsys@useobject{currentmarker}{}%
\end{pgfscope}%
\begin{pgfscope}%
\pgfsys@transformshift{0.549005in}{1.392854in}%
\pgfsys@useobject{currentmarker}{}%
\end{pgfscope}%
\begin{pgfscope}%
\pgfsys@transformshift{0.686528in}{1.492490in}%
\pgfsys@useobject{currentmarker}{}%
\end{pgfscope}%
\begin{pgfscope}%
\pgfsys@transformshift{0.599136in}{1.284140in}%
\pgfsys@useobject{currentmarker}{}%
\end{pgfscope}%
\begin{pgfscope}%
\pgfsys@transformshift{0.553061in}{1.149211in}%
\pgfsys@useobject{currentmarker}{}%
\end{pgfscope}%
\begin{pgfscope}%
\pgfsys@transformshift{1.250377in}{1.216059in}%
\pgfsys@useobject{currentmarker}{}%
\end{pgfscope}%
\begin{pgfscope}%
\pgfsys@transformshift{1.641979in}{0.867789in}%
\pgfsys@useobject{currentmarker}{}%
\end{pgfscope}%
\begin{pgfscope}%
\pgfsys@transformshift{0.897255in}{1.397274in}%
\pgfsys@useobject{currentmarker}{}%
\end{pgfscope}%
\begin{pgfscope}%
\pgfsys@transformshift{1.179466in}{1.303400in}%
\pgfsys@useobject{currentmarker}{}%
\end{pgfscope}%
\begin{pgfscope}%
\pgfsys@transformshift{0.621349in}{1.459037in}%
\pgfsys@useobject{currentmarker}{}%
\end{pgfscope}%
\begin{pgfscope}%
\pgfsys@transformshift{0.859048in}{1.450474in}%
\pgfsys@useobject{currentmarker}{}%
\end{pgfscope}%
\begin{pgfscope}%
\pgfsys@transformshift{0.665217in}{1.272777in}%
\pgfsys@useobject{currentmarker}{}%
\end{pgfscope}%
\begin{pgfscope}%
\pgfsys@transformshift{0.680580in}{1.584251in}%
\pgfsys@useobject{currentmarker}{}%
\end{pgfscope}%
\begin{pgfscope}%
\pgfsys@transformshift{0.999738in}{1.301247in}%
\pgfsys@useobject{currentmarker}{}%
\end{pgfscope}%
\begin{pgfscope}%
\pgfsys@transformshift{0.973168in}{1.361683in}%
\pgfsys@useobject{currentmarker}{}%
\end{pgfscope}%
\begin{pgfscope}%
\pgfsys@transformshift{0.583930in}{1.353382in}%
\pgfsys@useobject{currentmarker}{}%
\end{pgfscope}%
\begin{pgfscope}%
\pgfsys@transformshift{0.664630in}{1.200124in}%
\pgfsys@useobject{currentmarker}{}%
\end{pgfscope}%
\begin{pgfscope}%
\pgfsys@transformshift{0.595797in}{1.265281in}%
\pgfsys@useobject{currentmarker}{}%
\end{pgfscope}%
\begin{pgfscope}%
\pgfsys@transformshift{0.650485in}{1.308768in}%
\pgfsys@useobject{currentmarker}{}%
\end{pgfscope}%
\begin{pgfscope}%
\pgfsys@transformshift{0.599666in}{0.598158in}%
\pgfsys@useobject{currentmarker}{}%
\end{pgfscope}%
\begin{pgfscope}%
\pgfsys@transformshift{0.910927in}{1.209779in}%
\pgfsys@useobject{currentmarker}{}%
\end{pgfscope}%
\begin{pgfscope}%
\pgfsys@transformshift{0.866142in}{1.519143in}%
\pgfsys@useobject{currentmarker}{}%
\end{pgfscope}%
\begin{pgfscope}%
\pgfsys@transformshift{0.638303in}{1.232015in}%
\pgfsys@useobject{currentmarker}{}%
\end{pgfscope}%
\begin{pgfscope}%
\pgfsys@transformshift{1.224724in}{1.240730in}%
\pgfsys@useobject{currentmarker}{}%
\end{pgfscope}%
\begin{pgfscope}%
\pgfsys@transformshift{0.715706in}{1.207432in}%
\pgfsys@useobject{currentmarker}{}%
\end{pgfscope}%
\begin{pgfscope}%
\pgfsys@transformshift{1.087474in}{1.327998in}%
\pgfsys@useobject{currentmarker}{}%
\end{pgfscope}%
\begin{pgfscope}%
\pgfsys@transformshift{0.718186in}{1.350889in}%
\pgfsys@useobject{currentmarker}{}%
\end{pgfscope}%
\begin{pgfscope}%
\pgfsys@transformshift{0.597717in}{1.407462in}%
\pgfsys@useobject{currentmarker}{}%
\end{pgfscope}%
\begin{pgfscope}%
\pgfsys@transformshift{0.679205in}{0.981264in}%
\pgfsys@useobject{currentmarker}{}%
\end{pgfscope}%
\begin{pgfscope}%
\pgfsys@transformshift{1.057235in}{1.251973in}%
\pgfsys@useobject{currentmarker}{}%
\end{pgfscope}%
\begin{pgfscope}%
\pgfsys@transformshift{0.883984in}{1.160754in}%
\pgfsys@useobject{currentmarker}{}%
\end{pgfscope}%
\begin{pgfscope}%
\pgfsys@transformshift{0.653738in}{1.292088in}%
\pgfsys@useobject{currentmarker}{}%
\end{pgfscope}%
\begin{pgfscope}%
\pgfsys@transformshift{0.684665in}{1.168684in}%
\pgfsys@useobject{currentmarker}{}%
\end{pgfscope}%
\begin{pgfscope}%
\pgfsys@transformshift{0.546941in}{1.531171in}%
\pgfsys@useobject{currentmarker}{}%
\end{pgfscope}%
\begin{pgfscope}%
\pgfsys@transformshift{0.893142in}{1.407999in}%
\pgfsys@useobject{currentmarker}{}%
\end{pgfscope}%
\begin{pgfscope}%
\pgfsys@transformshift{0.676224in}{1.118239in}%
\pgfsys@useobject{currentmarker}{}%
\end{pgfscope}%
\begin{pgfscope}%
\pgfsys@transformshift{1.244788in}{1.535468in}%
\pgfsys@useobject{currentmarker}{}%
\end{pgfscope}%
\begin{pgfscope}%
\pgfsys@transformshift{0.605900in}{0.879811in}%
\pgfsys@useobject{currentmarker}{}%
\end{pgfscope}%
\begin{pgfscope}%
\pgfsys@transformshift{0.875873in}{1.499771in}%
\pgfsys@useobject{currentmarker}{}%
\end{pgfscope}%
\end{pgfscope}%
\begin{pgfscope}%
\pgfpathrectangle{\pgfqpoint{0.519339in}{0.466613in}}{\pgfqpoint{1.278750in}{1.245750in}}%
\pgfusepath{clip}%
\pgfsetbuttcap%
\pgfsetroundjoin%
\definecolor{currentfill}{rgb}{0.298039,0.447059,0.690196}%
\pgfsetfillcolor{currentfill}%
\pgfsetfillopacity{0.150000}%
\pgfsetlinewidth{1.003750pt}%
\definecolor{currentstroke}{rgb}{1.000000,1.000000,1.000000}%
\pgfsetstrokecolor{currentstroke}%
\pgfsetstrokeopacity{0.150000}%
\pgfsetdash{}{0pt}%
\pgfsys@defobject{currentmarker}{\pgfqpoint{0.519339in}{1.199220in}}{\pgfqpoint{1.798089in}{1.365653in}}{%
\pgfpathmoveto{\pgfqpoint{0.519339in}{1.329520in}}%
\pgfpathlineto{\pgfqpoint{0.519339in}{1.265558in}}%
\pgfpathlineto{\pgfqpoint{0.532256in}{1.266712in}}%
\pgfpathlineto{\pgfqpoint{0.545173in}{1.267733in}}%
\pgfpathlineto{\pgfqpoint{0.558089in}{1.268700in}}%
\pgfpathlineto{\pgfqpoint{0.571006in}{1.268719in}}%
\pgfpathlineto{\pgfqpoint{0.583923in}{1.269106in}}%
\pgfpathlineto{\pgfqpoint{0.596839in}{1.270076in}}%
\pgfpathlineto{\pgfqpoint{0.609756in}{1.270686in}}%
\pgfpathlineto{\pgfqpoint{0.622673in}{1.271326in}}%
\pgfpathlineto{\pgfqpoint{0.635589in}{1.271582in}}%
\pgfpathlineto{\pgfqpoint{0.648506in}{1.271692in}}%
\pgfpathlineto{\pgfqpoint{0.661423in}{1.272325in}}%
\pgfpathlineto{\pgfqpoint{0.674339in}{1.273656in}}%
\pgfpathlineto{\pgfqpoint{0.687256in}{1.274853in}}%
\pgfpathlineto{\pgfqpoint{0.700173in}{1.275652in}}%
\pgfpathlineto{\pgfqpoint{0.713089in}{1.276090in}}%
\pgfpathlineto{\pgfqpoint{0.726006in}{1.276532in}}%
\pgfpathlineto{\pgfqpoint{0.738923in}{1.276766in}}%
\pgfpathlineto{\pgfqpoint{0.751839in}{1.277299in}}%
\pgfpathlineto{\pgfqpoint{0.764756in}{1.277596in}}%
\pgfpathlineto{\pgfqpoint{0.777673in}{1.277692in}}%
\pgfpathlineto{\pgfqpoint{0.790589in}{1.277771in}}%
\pgfpathlineto{\pgfqpoint{0.803506in}{1.277967in}}%
\pgfpathlineto{\pgfqpoint{0.816423in}{1.278088in}}%
\pgfpathlineto{\pgfqpoint{0.829339in}{1.278130in}}%
\pgfpathlineto{\pgfqpoint{0.842256in}{1.277479in}}%
\pgfpathlineto{\pgfqpoint{0.855173in}{1.276979in}}%
\pgfpathlineto{\pgfqpoint{0.868089in}{1.276470in}}%
\pgfpathlineto{\pgfqpoint{0.881006in}{1.275912in}}%
\pgfpathlineto{\pgfqpoint{0.893923in}{1.275543in}}%
\pgfpathlineto{\pgfqpoint{0.906839in}{1.275178in}}%
\pgfpathlineto{\pgfqpoint{0.919756in}{1.274796in}}%
\pgfpathlineto{\pgfqpoint{0.932673in}{1.274079in}}%
\pgfpathlineto{\pgfqpoint{0.945589in}{1.273623in}}%
\pgfpathlineto{\pgfqpoint{0.958506in}{1.272824in}}%
\pgfpathlineto{\pgfqpoint{0.971423in}{1.272108in}}%
\pgfpathlineto{\pgfqpoint{0.984339in}{1.271317in}}%
\pgfpathlineto{\pgfqpoint{0.997256in}{1.270682in}}%
\pgfpathlineto{\pgfqpoint{1.010173in}{1.269567in}}%
\pgfpathlineto{\pgfqpoint{1.023089in}{1.268360in}}%
\pgfpathlineto{\pgfqpoint{1.036006in}{1.267468in}}%
\pgfpathlineto{\pgfqpoint{1.048923in}{1.267001in}}%
\pgfpathlineto{\pgfqpoint{1.061839in}{1.266329in}}%
\pgfpathlineto{\pgfqpoint{1.074756in}{1.265759in}}%
\pgfpathlineto{\pgfqpoint{1.087673in}{1.264805in}}%
\pgfpathlineto{\pgfqpoint{1.100589in}{1.263843in}}%
\pgfpathlineto{\pgfqpoint{1.113506in}{1.262918in}}%
\pgfpathlineto{\pgfqpoint{1.126423in}{1.261970in}}%
\pgfpathlineto{\pgfqpoint{1.139339in}{1.260926in}}%
\pgfpathlineto{\pgfqpoint{1.152256in}{1.259842in}}%
\pgfpathlineto{\pgfqpoint{1.165173in}{1.258407in}}%
\pgfpathlineto{\pgfqpoint{1.178089in}{1.256973in}}%
\pgfpathlineto{\pgfqpoint{1.191006in}{1.255770in}}%
\pgfpathlineto{\pgfqpoint{1.203923in}{1.254706in}}%
\pgfpathlineto{\pgfqpoint{1.216839in}{1.253812in}}%
\pgfpathlineto{\pgfqpoint{1.229756in}{1.252576in}}%
\pgfpathlineto{\pgfqpoint{1.242673in}{1.251682in}}%
\pgfpathlineto{\pgfqpoint{1.255589in}{1.251331in}}%
\pgfpathlineto{\pgfqpoint{1.268506in}{1.250280in}}%
\pgfpathlineto{\pgfqpoint{1.281423in}{1.248767in}}%
\pgfpathlineto{\pgfqpoint{1.294339in}{1.247256in}}%
\pgfpathlineto{\pgfqpoint{1.307256in}{1.245972in}}%
\pgfpathlineto{\pgfqpoint{1.320173in}{1.245036in}}%
\pgfpathlineto{\pgfqpoint{1.333089in}{1.244103in}}%
\pgfpathlineto{\pgfqpoint{1.346006in}{1.242948in}}%
\pgfpathlineto{\pgfqpoint{1.358923in}{1.241685in}}%
\pgfpathlineto{\pgfqpoint{1.371839in}{1.240229in}}%
\pgfpathlineto{\pgfqpoint{1.384756in}{1.239104in}}%
\pgfpathlineto{\pgfqpoint{1.397673in}{1.237951in}}%
\pgfpathlineto{\pgfqpoint{1.410589in}{1.236697in}}%
\pgfpathlineto{\pgfqpoint{1.423506in}{1.235145in}}%
\pgfpathlineto{\pgfqpoint{1.436423in}{1.233426in}}%
\pgfpathlineto{\pgfqpoint{1.449339in}{1.231709in}}%
\pgfpathlineto{\pgfqpoint{1.462256in}{1.230296in}}%
\pgfpathlineto{\pgfqpoint{1.475173in}{1.229184in}}%
\pgfpathlineto{\pgfqpoint{1.488089in}{1.228072in}}%
\pgfpathlineto{\pgfqpoint{1.501006in}{1.226960in}}%
\pgfpathlineto{\pgfqpoint{1.513923in}{1.225848in}}%
\pgfpathlineto{\pgfqpoint{1.526839in}{1.224736in}}%
\pgfpathlineto{\pgfqpoint{1.539756in}{1.223624in}}%
\pgfpathlineto{\pgfqpoint{1.552673in}{1.222512in}}%
\pgfpathlineto{\pgfqpoint{1.565589in}{1.221403in}}%
\pgfpathlineto{\pgfqpoint{1.578506in}{1.220300in}}%
\pgfpathlineto{\pgfqpoint{1.591423in}{1.218971in}}%
\pgfpathlineto{\pgfqpoint{1.604339in}{1.217808in}}%
\pgfpathlineto{\pgfqpoint{1.617256in}{1.216722in}}%
\pgfpathlineto{\pgfqpoint{1.630173in}{1.215472in}}%
\pgfpathlineto{\pgfqpoint{1.643089in}{1.214227in}}%
\pgfpathlineto{\pgfqpoint{1.656006in}{1.212983in}}%
\pgfpathlineto{\pgfqpoint{1.668923in}{1.211738in}}%
\pgfpathlineto{\pgfqpoint{1.681839in}{1.210556in}}%
\pgfpathlineto{\pgfqpoint{1.694756in}{1.209463in}}%
\pgfpathlineto{\pgfqpoint{1.707673in}{1.208116in}}%
\pgfpathlineto{\pgfqpoint{1.720589in}{1.206740in}}%
\pgfpathlineto{\pgfqpoint{1.733506in}{1.205485in}}%
\pgfpathlineto{\pgfqpoint{1.746423in}{1.204229in}}%
\pgfpathlineto{\pgfqpoint{1.759339in}{1.202973in}}%
\pgfpathlineto{\pgfqpoint{1.772256in}{1.201717in}}%
\pgfpathlineto{\pgfqpoint{1.785173in}{1.200469in}}%
\pgfpathlineto{\pgfqpoint{1.798089in}{1.199220in}}%
\pgfpathlineto{\pgfqpoint{1.798089in}{1.365653in}}%
\pgfpathlineto{\pgfqpoint{1.798089in}{1.365653in}}%
\pgfpathlineto{\pgfqpoint{1.785173in}{1.364659in}}%
\pgfpathlineto{\pgfqpoint{1.772256in}{1.363663in}}%
\pgfpathlineto{\pgfqpoint{1.759339in}{1.362663in}}%
\pgfpathlineto{\pgfqpoint{1.746423in}{1.361663in}}%
\pgfpathlineto{\pgfqpoint{1.733506in}{1.360663in}}%
\pgfpathlineto{\pgfqpoint{1.720589in}{1.359663in}}%
\pgfpathlineto{\pgfqpoint{1.707673in}{1.358663in}}%
\pgfpathlineto{\pgfqpoint{1.694756in}{1.357664in}}%
\pgfpathlineto{\pgfqpoint{1.681839in}{1.356664in}}%
\pgfpathlineto{\pgfqpoint{1.668923in}{1.355664in}}%
\pgfpathlineto{\pgfqpoint{1.656006in}{1.354725in}}%
\pgfpathlineto{\pgfqpoint{1.643089in}{1.354082in}}%
\pgfpathlineto{\pgfqpoint{1.630173in}{1.353041in}}%
\pgfpathlineto{\pgfqpoint{1.617256in}{1.352011in}}%
\pgfpathlineto{\pgfqpoint{1.604339in}{1.351141in}}%
\pgfpathlineto{\pgfqpoint{1.591423in}{1.350272in}}%
\pgfpathlineto{\pgfqpoint{1.578506in}{1.349402in}}%
\pgfpathlineto{\pgfqpoint{1.565589in}{1.348533in}}%
\pgfpathlineto{\pgfqpoint{1.552673in}{1.347661in}}%
\pgfpathlineto{\pgfqpoint{1.539756in}{1.346783in}}%
\pgfpathlineto{\pgfqpoint{1.526839in}{1.345905in}}%
\pgfpathlineto{\pgfqpoint{1.513923in}{1.345027in}}%
\pgfpathlineto{\pgfqpoint{1.501006in}{1.344148in}}%
\pgfpathlineto{\pgfqpoint{1.488089in}{1.343270in}}%
\pgfpathlineto{\pgfqpoint{1.475173in}{1.342392in}}%
\pgfpathlineto{\pgfqpoint{1.462256in}{1.341514in}}%
\pgfpathlineto{\pgfqpoint{1.449339in}{1.340636in}}%
\pgfpathlineto{\pgfqpoint{1.436423in}{1.339758in}}%
\pgfpathlineto{\pgfqpoint{1.423506in}{1.338880in}}%
\pgfpathlineto{\pgfqpoint{1.410589in}{1.338002in}}%
\pgfpathlineto{\pgfqpoint{1.397673in}{1.337124in}}%
\pgfpathlineto{\pgfqpoint{1.384756in}{1.336383in}}%
\pgfpathlineto{\pgfqpoint{1.371839in}{1.335657in}}%
\pgfpathlineto{\pgfqpoint{1.358923in}{1.334931in}}%
\pgfpathlineto{\pgfqpoint{1.346006in}{1.334045in}}%
\pgfpathlineto{\pgfqpoint{1.333089in}{1.333080in}}%
\pgfpathlineto{\pgfqpoint{1.320173in}{1.332112in}}%
\pgfpathlineto{\pgfqpoint{1.307256in}{1.331141in}}%
\pgfpathlineto{\pgfqpoint{1.294339in}{1.330169in}}%
\pgfpathlineto{\pgfqpoint{1.281423in}{1.329234in}}%
\pgfpathlineto{\pgfqpoint{1.268506in}{1.328355in}}%
\pgfpathlineto{\pgfqpoint{1.255589in}{1.327475in}}%
\pgfpathlineto{\pgfqpoint{1.242673in}{1.326642in}}%
\pgfpathlineto{\pgfqpoint{1.229756in}{1.325722in}}%
\pgfpathlineto{\pgfqpoint{1.216839in}{1.324844in}}%
\pgfpathlineto{\pgfqpoint{1.203923in}{1.323861in}}%
\pgfpathlineto{\pgfqpoint{1.191006in}{1.322923in}}%
\pgfpathlineto{\pgfqpoint{1.178089in}{1.321946in}}%
\pgfpathlineto{\pgfqpoint{1.165173in}{1.321023in}}%
\pgfpathlineto{\pgfqpoint{1.152256in}{1.320459in}}%
\pgfpathlineto{\pgfqpoint{1.139339in}{1.319597in}}%
\pgfpathlineto{\pgfqpoint{1.126423in}{1.318719in}}%
\pgfpathlineto{\pgfqpoint{1.113506in}{1.317877in}}%
\pgfpathlineto{\pgfqpoint{1.100589in}{1.317059in}}%
\pgfpathlineto{\pgfqpoint{1.087673in}{1.316135in}}%
\pgfpathlineto{\pgfqpoint{1.074756in}{1.315999in}}%
\pgfpathlineto{\pgfqpoint{1.061839in}{1.314865in}}%
\pgfpathlineto{\pgfqpoint{1.048923in}{1.314094in}}%
\pgfpathlineto{\pgfqpoint{1.036006in}{1.313038in}}%
\pgfpathlineto{\pgfqpoint{1.023089in}{1.312542in}}%
\pgfpathlineto{\pgfqpoint{1.010173in}{1.312084in}}%
\pgfpathlineto{\pgfqpoint{0.997256in}{1.311574in}}%
\pgfpathlineto{\pgfqpoint{0.984339in}{1.311223in}}%
\pgfpathlineto{\pgfqpoint{0.971423in}{1.310689in}}%
\pgfpathlineto{\pgfqpoint{0.958506in}{1.310293in}}%
\pgfpathlineto{\pgfqpoint{0.945589in}{1.309923in}}%
\pgfpathlineto{\pgfqpoint{0.932673in}{1.309967in}}%
\pgfpathlineto{\pgfqpoint{0.919756in}{1.309535in}}%
\pgfpathlineto{\pgfqpoint{0.906839in}{1.309522in}}%
\pgfpathlineto{\pgfqpoint{0.893923in}{1.309331in}}%
\pgfpathlineto{\pgfqpoint{0.881006in}{1.309021in}}%
\pgfpathlineto{\pgfqpoint{0.868089in}{1.308782in}}%
\pgfpathlineto{\pgfqpoint{0.855173in}{1.308702in}}%
\pgfpathlineto{\pgfqpoint{0.842256in}{1.308780in}}%
\pgfpathlineto{\pgfqpoint{0.829339in}{1.309030in}}%
\pgfpathlineto{\pgfqpoint{0.816423in}{1.309694in}}%
\pgfpathlineto{\pgfqpoint{0.803506in}{1.310323in}}%
\pgfpathlineto{\pgfqpoint{0.790589in}{1.310726in}}%
\pgfpathlineto{\pgfqpoint{0.777673in}{1.311267in}}%
\pgfpathlineto{\pgfqpoint{0.764756in}{1.311392in}}%
\pgfpathlineto{\pgfqpoint{0.751839in}{1.312424in}}%
\pgfpathlineto{\pgfqpoint{0.738923in}{1.313065in}}%
\pgfpathlineto{\pgfqpoint{0.726006in}{1.313999in}}%
\pgfpathlineto{\pgfqpoint{0.713089in}{1.314830in}}%
\pgfpathlineto{\pgfqpoint{0.700173in}{1.315347in}}%
\pgfpathlineto{\pgfqpoint{0.687256in}{1.315717in}}%
\pgfpathlineto{\pgfqpoint{0.674339in}{1.316773in}}%
\pgfpathlineto{\pgfqpoint{0.661423in}{1.317769in}}%
\pgfpathlineto{\pgfqpoint{0.648506in}{1.319262in}}%
\pgfpathlineto{\pgfqpoint{0.635589in}{1.320288in}}%
\pgfpathlineto{\pgfqpoint{0.622673in}{1.321192in}}%
\pgfpathlineto{\pgfqpoint{0.609756in}{1.322203in}}%
\pgfpathlineto{\pgfqpoint{0.596839in}{1.323434in}}%
\pgfpathlineto{\pgfqpoint{0.583923in}{1.324485in}}%
\pgfpathlineto{\pgfqpoint{0.571006in}{1.325602in}}%
\pgfpathlineto{\pgfqpoint{0.558089in}{1.326565in}}%
\pgfpathlineto{\pgfqpoint{0.545173in}{1.327525in}}%
\pgfpathlineto{\pgfqpoint{0.532256in}{1.328561in}}%
\pgfpathlineto{\pgfqpoint{0.519339in}{1.329520in}}%
\pgfpathclose%
\pgfusepath{stroke,fill}%
}%
\begin{pgfscope}%
\pgfsys@transformshift{0.000000in}{0.000000in}%
\pgfsys@useobject{currentmarker}{}%
\end{pgfscope}%
\end{pgfscope}%
\begin{pgfscope}%
\pgfpathrectangle{\pgfqpoint{0.519339in}{0.466613in}}{\pgfqpoint{1.278750in}{1.245750in}}%
\pgfusepath{clip}%
\pgfsetroundcap%
\pgfsetroundjoin%
\pgfsetlinewidth{1.505625pt}%
\definecolor{currentstroke}{rgb}{0.298039,0.447059,0.690196}%
\pgfsetstrokecolor{currentstroke}%
\pgfsetdash{}{0pt}%
\pgfpathmoveto{\pgfqpoint{0.519339in}{1.297393in}}%
\pgfpathlineto{\pgfqpoint{0.532256in}{1.297223in}}%
\pgfpathlineto{\pgfqpoint{0.545173in}{1.297053in}}%
\pgfpathlineto{\pgfqpoint{0.558089in}{1.296883in}}%
\pgfpathlineto{\pgfqpoint{0.571006in}{1.296713in}}%
\pgfpathlineto{\pgfqpoint{0.583923in}{1.296543in}}%
\pgfpathlineto{\pgfqpoint{0.596839in}{1.296373in}}%
\pgfpathlineto{\pgfqpoint{0.609756in}{1.296203in}}%
\pgfpathlineto{\pgfqpoint{0.622673in}{1.296033in}}%
\pgfpathlineto{\pgfqpoint{0.635589in}{1.295862in}}%
\pgfpathlineto{\pgfqpoint{0.648506in}{1.295692in}}%
\pgfpathlineto{\pgfqpoint{0.661423in}{1.295522in}}%
\pgfpathlineto{\pgfqpoint{0.674339in}{1.295352in}}%
\pgfpathlineto{\pgfqpoint{0.687256in}{1.295182in}}%
\pgfpathlineto{\pgfqpoint{0.700173in}{1.295012in}}%
\pgfpathlineto{\pgfqpoint{0.713089in}{1.294842in}}%
\pgfpathlineto{\pgfqpoint{0.726006in}{1.294672in}}%
\pgfpathlineto{\pgfqpoint{0.738923in}{1.294502in}}%
\pgfpathlineto{\pgfqpoint{0.751839in}{1.294332in}}%
\pgfpathlineto{\pgfqpoint{0.764756in}{1.294162in}}%
\pgfpathlineto{\pgfqpoint{0.777673in}{1.293991in}}%
\pgfpathlineto{\pgfqpoint{0.790589in}{1.293821in}}%
\pgfpathlineto{\pgfqpoint{0.803506in}{1.293651in}}%
\pgfpathlineto{\pgfqpoint{0.816423in}{1.293481in}}%
\pgfpathlineto{\pgfqpoint{0.829339in}{1.293311in}}%
\pgfpathlineto{\pgfqpoint{0.842256in}{1.293141in}}%
\pgfpathlineto{\pgfqpoint{0.855173in}{1.292971in}}%
\pgfpathlineto{\pgfqpoint{0.868089in}{1.292801in}}%
\pgfpathlineto{\pgfqpoint{0.881006in}{1.292631in}}%
\pgfpathlineto{\pgfqpoint{0.893923in}{1.292461in}}%
\pgfpathlineto{\pgfqpoint{0.906839in}{1.292291in}}%
\pgfpathlineto{\pgfqpoint{0.919756in}{1.292120in}}%
\pgfpathlineto{\pgfqpoint{0.932673in}{1.291950in}}%
\pgfpathlineto{\pgfqpoint{0.945589in}{1.291780in}}%
\pgfpathlineto{\pgfqpoint{0.958506in}{1.291610in}}%
\pgfpathlineto{\pgfqpoint{0.971423in}{1.291440in}}%
\pgfpathlineto{\pgfqpoint{0.984339in}{1.291270in}}%
\pgfpathlineto{\pgfqpoint{0.997256in}{1.291100in}}%
\pgfpathlineto{\pgfqpoint{1.010173in}{1.290930in}}%
\pgfpathlineto{\pgfqpoint{1.023089in}{1.290760in}}%
\pgfpathlineto{\pgfqpoint{1.036006in}{1.290590in}}%
\pgfpathlineto{\pgfqpoint{1.048923in}{1.290420in}}%
\pgfpathlineto{\pgfqpoint{1.061839in}{1.290249in}}%
\pgfpathlineto{\pgfqpoint{1.074756in}{1.290079in}}%
\pgfpathlineto{\pgfqpoint{1.087673in}{1.289909in}}%
\pgfpathlineto{\pgfqpoint{1.100589in}{1.289739in}}%
\pgfpathlineto{\pgfqpoint{1.113506in}{1.289569in}}%
\pgfpathlineto{\pgfqpoint{1.126423in}{1.289399in}}%
\pgfpathlineto{\pgfqpoint{1.139339in}{1.289229in}}%
\pgfpathlineto{\pgfqpoint{1.152256in}{1.289059in}}%
\pgfpathlineto{\pgfqpoint{1.165173in}{1.288889in}}%
\pgfpathlineto{\pgfqpoint{1.178089in}{1.288719in}}%
\pgfpathlineto{\pgfqpoint{1.191006in}{1.288549in}}%
\pgfpathlineto{\pgfqpoint{1.203923in}{1.288378in}}%
\pgfpathlineto{\pgfqpoint{1.216839in}{1.288208in}}%
\pgfpathlineto{\pgfqpoint{1.229756in}{1.288038in}}%
\pgfpathlineto{\pgfqpoint{1.242673in}{1.287868in}}%
\pgfpathlineto{\pgfqpoint{1.255589in}{1.287698in}}%
\pgfpathlineto{\pgfqpoint{1.268506in}{1.287528in}}%
\pgfpathlineto{\pgfqpoint{1.281423in}{1.287358in}}%
\pgfpathlineto{\pgfqpoint{1.294339in}{1.287188in}}%
\pgfpathlineto{\pgfqpoint{1.307256in}{1.287018in}}%
\pgfpathlineto{\pgfqpoint{1.320173in}{1.286848in}}%
\pgfpathlineto{\pgfqpoint{1.333089in}{1.286678in}}%
\pgfpathlineto{\pgfqpoint{1.346006in}{1.286507in}}%
\pgfpathlineto{\pgfqpoint{1.358923in}{1.286337in}}%
\pgfpathlineto{\pgfqpoint{1.371839in}{1.286167in}}%
\pgfpathlineto{\pgfqpoint{1.384756in}{1.285997in}}%
\pgfpathlineto{\pgfqpoint{1.397673in}{1.285827in}}%
\pgfpathlineto{\pgfqpoint{1.410589in}{1.285657in}}%
\pgfpathlineto{\pgfqpoint{1.423506in}{1.285487in}}%
\pgfpathlineto{\pgfqpoint{1.436423in}{1.285317in}}%
\pgfpathlineto{\pgfqpoint{1.449339in}{1.285147in}}%
\pgfpathlineto{\pgfqpoint{1.462256in}{1.284977in}}%
\pgfpathlineto{\pgfqpoint{1.475173in}{1.284807in}}%
\pgfpathlineto{\pgfqpoint{1.488089in}{1.284636in}}%
\pgfpathlineto{\pgfqpoint{1.501006in}{1.284466in}}%
\pgfpathlineto{\pgfqpoint{1.513923in}{1.284296in}}%
\pgfpathlineto{\pgfqpoint{1.526839in}{1.284126in}}%
\pgfpathlineto{\pgfqpoint{1.539756in}{1.283956in}}%
\pgfpathlineto{\pgfqpoint{1.552673in}{1.283786in}}%
\pgfpathlineto{\pgfqpoint{1.565589in}{1.283616in}}%
\pgfpathlineto{\pgfqpoint{1.578506in}{1.283446in}}%
\pgfpathlineto{\pgfqpoint{1.591423in}{1.283276in}}%
\pgfpathlineto{\pgfqpoint{1.604339in}{1.283106in}}%
\pgfpathlineto{\pgfqpoint{1.617256in}{1.282936in}}%
\pgfpathlineto{\pgfqpoint{1.630173in}{1.282765in}}%
\pgfpathlineto{\pgfqpoint{1.643089in}{1.282595in}}%
\pgfpathlineto{\pgfqpoint{1.656006in}{1.282425in}}%
\pgfpathlineto{\pgfqpoint{1.668923in}{1.282255in}}%
\pgfpathlineto{\pgfqpoint{1.681839in}{1.282085in}}%
\pgfpathlineto{\pgfqpoint{1.694756in}{1.281915in}}%
\pgfpathlineto{\pgfqpoint{1.707673in}{1.281745in}}%
\pgfpathlineto{\pgfqpoint{1.720589in}{1.281575in}}%
\pgfpathlineto{\pgfqpoint{1.733506in}{1.281405in}}%
\pgfpathlineto{\pgfqpoint{1.746423in}{1.281235in}}%
\pgfpathlineto{\pgfqpoint{1.759339in}{1.281065in}}%
\pgfpathlineto{\pgfqpoint{1.772256in}{1.280894in}}%
\pgfpathlineto{\pgfqpoint{1.785173in}{1.280724in}}%
\pgfpathlineto{\pgfqpoint{1.798089in}{1.280554in}}%
\pgfusepath{stroke}%
\end{pgfscope}%
\begin{pgfscope}%
\pgfsetrectcap%
\pgfsetmiterjoin%
\pgfsetlinewidth{0.752812pt}%
\definecolor{currentstroke}{rgb}{0.700000,0.700000,0.700000}%
\pgfsetstrokecolor{currentstroke}%
\pgfsetdash{}{0pt}%
\pgfpathmoveto{\pgfqpoint{0.519339in}{0.466613in}}%
\pgfpathlineto{\pgfqpoint{0.519339in}{1.712363in}}%
\pgfusepath{stroke}%
\end{pgfscope}%
\begin{pgfscope}%
\pgfsetrectcap%
\pgfsetmiterjoin%
\pgfsetlinewidth{0.752812pt}%
\definecolor{currentstroke}{rgb}{0.700000,0.700000,0.700000}%
\pgfsetstrokecolor{currentstroke}%
\pgfsetdash{}{0pt}%
\pgfpathmoveto{\pgfqpoint{1.798089in}{0.466613in}}%
\pgfpathlineto{\pgfqpoint{1.798089in}{1.712363in}}%
\pgfusepath{stroke}%
\end{pgfscope}%
\begin{pgfscope}%
\pgfsetrectcap%
\pgfsetmiterjoin%
\pgfsetlinewidth{0.752812pt}%
\definecolor{currentstroke}{rgb}{0.700000,0.700000,0.700000}%
\pgfsetstrokecolor{currentstroke}%
\pgfsetdash{}{0pt}%
\pgfpathmoveto{\pgfqpoint{0.519339in}{0.466613in}}%
\pgfpathlineto{\pgfqpoint{1.798089in}{0.466613in}}%
\pgfusepath{stroke}%
\end{pgfscope}%
\begin{pgfscope}%
\pgfsetrectcap%
\pgfsetmiterjoin%
\pgfsetlinewidth{0.752812pt}%
\definecolor{currentstroke}{rgb}{0.700000,0.700000,0.700000}%
\pgfsetstrokecolor{currentstroke}%
\pgfsetdash{}{0pt}%
\pgfpathmoveto{\pgfqpoint{0.519339in}{1.712363in}}%
\pgfpathlineto{\pgfqpoint{1.798089in}{1.712363in}}%
\pgfusepath{stroke}%
\end{pgfscope}%
\end{pgfpicture}%
\makeatother%
\endgroup%
}} &
      \subfloat[\(\epsilon=0.01\)]{\resizebox{0.5\linewidth}{!}{%% Creator: Matplotlib, PGF backend
%%
%% To include the figure in your LaTeX document, write
%%   \input{<filename>.pgf}
%%
%% Make sure the required packages are loaded in your preamble
%%   \usepackage{pgf}
%%
%% and, on pdftex
%%   \usepackage[utf8]{inputenc}\DeclareUnicodeCharacter{2212}{-}
%%
%% or, on luatex and xetex
%%   \usepackage{unicode-math}
%%
%% Figures using additional raster images can only be included by \input if
%% they are in the same directory as the main LaTeX file. For loading figures
%% from other directories you can use the `import` package
%%   \usepackage{import}
%%
%% and then include the figures with
%%   \import{<path to file>}{<filename>.pgf}
%%
%% Matplotlib used the following preamble
%%   \usepackage[utf8]{inputenc}
%%   \usepackage[T1]{fontenc}
%%   \usepackage{amsmath}
%%   \newcommand*{\mat}[1]{\boldsymbol{#1}}
%%
\begingroup%
\makeatletter%
\begin{pgfpicture}%
\pgfpathrectangle{\pgfpointorigin}{\pgfqpoint{1.957118in}{1.812363in}}%
\pgfusepath{use as bounding box, clip}%
\begin{pgfscope}%
\pgfsetbuttcap%
\pgfsetmiterjoin%
\definecolor{currentfill}{rgb}{1.000000,1.000000,1.000000}%
\pgfsetfillcolor{currentfill}%
\pgfsetlinewidth{0.000000pt}%
\definecolor{currentstroke}{rgb}{1.000000,1.000000,1.000000}%
\pgfsetstrokecolor{currentstroke}%
\pgfsetstrokeopacity{0.000000}%
\pgfsetdash{}{0pt}%
\pgfpathmoveto{\pgfqpoint{0.000000in}{-0.000000in}}%
\pgfpathlineto{\pgfqpoint{1.957118in}{-0.000000in}}%
\pgfpathlineto{\pgfqpoint{1.957118in}{1.812363in}}%
\pgfpathlineto{\pgfqpoint{0.000000in}{1.812363in}}%
\pgfpathclose%
\pgfusepath{fill}%
\end{pgfscope}%
\begin{pgfscope}%
\pgfsetbuttcap%
\pgfsetmiterjoin%
\definecolor{currentfill}{rgb}{1.000000,1.000000,1.000000}%
\pgfsetfillcolor{currentfill}%
\pgfsetlinewidth{0.000000pt}%
\definecolor{currentstroke}{rgb}{0.000000,0.000000,0.000000}%
\pgfsetstrokecolor{currentstroke}%
\pgfsetstrokeopacity{0.000000}%
\pgfsetdash{}{0pt}%
\pgfpathmoveto{\pgfqpoint{0.578368in}{0.466613in}}%
\pgfpathlineto{\pgfqpoint{1.857118in}{0.466613in}}%
\pgfpathlineto{\pgfqpoint{1.857118in}{1.712363in}}%
\pgfpathlineto{\pgfqpoint{0.578368in}{1.712363in}}%
\pgfpathclose%
\pgfusepath{fill}%
\end{pgfscope}%
\begin{pgfscope}%
\pgfpathrectangle{\pgfqpoint{0.578368in}{0.466613in}}{\pgfqpoint{1.278750in}{1.245750in}}%
\pgfusepath{clip}%
\pgfsetroundcap%
\pgfsetroundjoin%
\pgfsetlinewidth{0.501875pt}%
\definecolor{currentstroke}{rgb}{0.800000,0.800000,0.800000}%
\pgfsetstrokecolor{currentstroke}%
\pgfsetdash{}{0pt}%
\pgfpathmoveto{\pgfqpoint{0.827742in}{0.466613in}}%
\pgfpathlineto{\pgfqpoint{0.827742in}{1.712363in}}%
\pgfusepath{stroke}%
\end{pgfscope}%
\begin{pgfscope}%
\definecolor{textcolor}{rgb}{0.150000,0.150000,0.150000}%
\pgfsetstrokecolor{textcolor}%
\pgfsetfillcolor{textcolor}%
\pgftext[x=0.827742in,y=0.376335in,,top]{\color{textcolor}\rmfamily\fontsize{8.000000}{9.600000}\selectfont \(\displaystyle {25000}\)}%
\end{pgfscope}%
\begin{pgfscope}%
\pgfpathrectangle{\pgfqpoint{0.578368in}{0.466613in}}{\pgfqpoint{1.278750in}{1.245750in}}%
\pgfusepath{clip}%
\pgfsetroundcap%
\pgfsetroundjoin%
\pgfsetlinewidth{0.501875pt}%
\definecolor{currentstroke}{rgb}{0.800000,0.800000,0.800000}%
\pgfsetstrokecolor{currentstroke}%
\pgfsetdash{}{0pt}%
\pgfpathmoveto{\pgfqpoint{1.181806in}{0.466613in}}%
\pgfpathlineto{\pgfqpoint{1.181806in}{1.712363in}}%
\pgfusepath{stroke}%
\end{pgfscope}%
\begin{pgfscope}%
\definecolor{textcolor}{rgb}{0.150000,0.150000,0.150000}%
\pgfsetstrokecolor{textcolor}%
\pgfsetfillcolor{textcolor}%
\pgftext[x=1.181806in,y=0.376335in,,top]{\color{textcolor}\rmfamily\fontsize{8.000000}{9.600000}\selectfont \(\displaystyle {50000}\)}%
\end{pgfscope}%
\begin{pgfscope}%
\pgfpathrectangle{\pgfqpoint{0.578368in}{0.466613in}}{\pgfqpoint{1.278750in}{1.245750in}}%
\pgfusepath{clip}%
\pgfsetroundcap%
\pgfsetroundjoin%
\pgfsetlinewidth{0.501875pt}%
\definecolor{currentstroke}{rgb}{0.800000,0.800000,0.800000}%
\pgfsetstrokecolor{currentstroke}%
\pgfsetdash{}{0pt}%
\pgfpathmoveto{\pgfqpoint{1.535869in}{0.466613in}}%
\pgfpathlineto{\pgfqpoint{1.535869in}{1.712363in}}%
\pgfusepath{stroke}%
\end{pgfscope}%
\begin{pgfscope}%
\definecolor{textcolor}{rgb}{0.150000,0.150000,0.150000}%
\pgfsetstrokecolor{textcolor}%
\pgfsetfillcolor{textcolor}%
\pgftext[x=1.535869in,y=0.376335in,,top]{\color{textcolor}\rmfamily\fontsize{8.000000}{9.600000}\selectfont \(\displaystyle {75000}\)}%
\end{pgfscope}%
\begin{pgfscope}%
\definecolor{textcolor}{rgb}{0.150000,0.150000,0.150000}%
\pgfsetstrokecolor{textcolor}%
\pgfsetfillcolor{textcolor}%
\pgftext[x=1.217743in,y=0.222655in,,top]{\color{textcolor}\rmfamily\fontsize{10.000000}{12.000000}\selectfont Number of nodes}%
\end{pgfscope}%
\begin{pgfscope}%
\pgfpathrectangle{\pgfqpoint{0.578368in}{0.466613in}}{\pgfqpoint{1.278750in}{1.245750in}}%
\pgfusepath{clip}%
\pgfsetroundcap%
\pgfsetroundjoin%
\pgfsetlinewidth{0.501875pt}%
\definecolor{currentstroke}{rgb}{0.800000,0.800000,0.800000}%
\pgfsetstrokecolor{currentstroke}%
\pgfsetdash{}{0pt}%
\pgfpathmoveto{\pgfqpoint{0.578368in}{0.497096in}}%
\pgfpathlineto{\pgfqpoint{1.857118in}{0.497096in}}%
\pgfusepath{stroke}%
\end{pgfscope}%
\begin{pgfscope}%
\definecolor{textcolor}{rgb}{0.150000,0.150000,0.150000}%
\pgfsetstrokecolor{textcolor}%
\pgfsetfillcolor{textcolor}%
\pgftext[x=0.278211in, y=0.458834in, left, base]{\color{textcolor}\rmfamily\fontsize{8.000000}{9.600000}\selectfont \(\displaystyle {0.05}\)}%
\end{pgfscope}%
\begin{pgfscope}%
\pgfpathrectangle{\pgfqpoint{0.578368in}{0.466613in}}{\pgfqpoint{1.278750in}{1.245750in}}%
\pgfusepath{clip}%
\pgfsetroundcap%
\pgfsetroundjoin%
\pgfsetlinewidth{0.501875pt}%
\definecolor{currentstroke}{rgb}{0.800000,0.800000,0.800000}%
\pgfsetstrokecolor{currentstroke}%
\pgfsetdash{}{0pt}%
\pgfpathmoveto{\pgfqpoint{0.578368in}{0.918770in}}%
\pgfpathlineto{\pgfqpoint{1.857118in}{0.918770in}}%
\pgfusepath{stroke}%
\end{pgfscope}%
\begin{pgfscope}%
\definecolor{textcolor}{rgb}{0.150000,0.150000,0.150000}%
\pgfsetstrokecolor{textcolor}%
\pgfsetfillcolor{textcolor}%
\pgftext[x=0.278211in, y=0.880508in, left, base]{\color{textcolor}\rmfamily\fontsize{8.000000}{9.600000}\selectfont \(\displaystyle {0.10}\)}%
\end{pgfscope}%
\begin{pgfscope}%
\pgfpathrectangle{\pgfqpoint{0.578368in}{0.466613in}}{\pgfqpoint{1.278750in}{1.245750in}}%
\pgfusepath{clip}%
\pgfsetroundcap%
\pgfsetroundjoin%
\pgfsetlinewidth{0.501875pt}%
\definecolor{currentstroke}{rgb}{0.800000,0.800000,0.800000}%
\pgfsetstrokecolor{currentstroke}%
\pgfsetdash{}{0pt}%
\pgfpathmoveto{\pgfqpoint{0.578368in}{1.340444in}}%
\pgfpathlineto{\pgfqpoint{1.857118in}{1.340444in}}%
\pgfusepath{stroke}%
\end{pgfscope}%
\begin{pgfscope}%
\definecolor{textcolor}{rgb}{0.150000,0.150000,0.150000}%
\pgfsetstrokecolor{textcolor}%
\pgfsetfillcolor{textcolor}%
\pgftext[x=0.278211in, y=1.302182in, left, base]{\color{textcolor}\rmfamily\fontsize{8.000000}{9.600000}\selectfont \(\displaystyle {0.15}\)}%
\end{pgfscope}%
\begin{pgfscope}%
\definecolor{textcolor}{rgb}{0.150000,0.150000,0.150000}%
\pgfsetstrokecolor{textcolor}%
\pgfsetfillcolor{textcolor}%
\pgftext[x=0.222655in,y=1.089488in,,bottom,rotate=90.000000]{\color{textcolor}\rmfamily\fontsize{10.000000}{12.000000}\selectfont Accuracy drop}%
\end{pgfscope}%
\begin{pgfscope}%
\pgfpathrectangle{\pgfqpoint{0.578368in}{0.466613in}}{\pgfqpoint{1.278750in}{1.245750in}}%
\pgfusepath{clip}%
\pgfsetbuttcap%
\pgfsetroundjoin%
\definecolor{currentfill}{rgb}{0.298039,0.447059,0.690196}%
\pgfsetfillcolor{currentfill}%
\pgfsetfillopacity{0.800000}%
\pgfsetlinewidth{1.003750pt}%
\definecolor{currentstroke}{rgb}{0.298039,0.447059,0.690196}%
\pgfsetstrokecolor{currentstroke}%
\pgfsetstrokeopacity{0.800000}%
\pgfsetdash{}{0pt}%
\pgfsys@defobject{currentmarker}{\pgfqpoint{-0.017010in}{-0.017010in}}{\pgfqpoint{0.017010in}{0.017010in}}{%
\pgfpathmoveto{\pgfqpoint{0.000000in}{-0.017010in}}%
\pgfpathcurveto{\pgfqpoint{0.004511in}{-0.017010in}}{\pgfqpoint{0.008838in}{-0.015218in}}{\pgfqpoint{0.012028in}{-0.012028in}}%
\pgfpathcurveto{\pgfqpoint{0.015218in}{-0.008838in}}{\pgfqpoint{0.017010in}{-0.004511in}}{\pgfqpoint{0.017010in}{0.000000in}}%
\pgfpathcurveto{\pgfqpoint{0.017010in}{0.004511in}}{\pgfqpoint{0.015218in}{0.008838in}}{\pgfqpoint{0.012028in}{0.012028in}}%
\pgfpathcurveto{\pgfqpoint{0.008838in}{0.015218in}}{\pgfqpoint{0.004511in}{0.017010in}}{\pgfqpoint{0.000000in}{0.017010in}}%
\pgfpathcurveto{\pgfqpoint{-0.004511in}{0.017010in}}{\pgfqpoint{-0.008838in}{0.015218in}}{\pgfqpoint{-0.012028in}{0.012028in}}%
\pgfpathcurveto{\pgfqpoint{-0.015218in}{0.008838in}}{\pgfqpoint{-0.017010in}{0.004511in}}{\pgfqpoint{-0.017010in}{0.000000in}}%
\pgfpathcurveto{\pgfqpoint{-0.017010in}{-0.004511in}}{\pgfqpoint{-0.015218in}{-0.008838in}}{\pgfqpoint{-0.012028in}{-0.012028in}}%
\pgfpathcurveto{\pgfqpoint{-0.008838in}{-0.015218in}}{\pgfqpoint{-0.004511in}{-0.017010in}}{\pgfqpoint{0.000000in}{-0.017010in}}%
\pgfpathclose%
\pgfusepath{stroke,fill}%
}%
\begin{pgfscope}%
\pgfsys@transformshift{1.073122in}{0.971890in}%
\pgfsys@useobject{currentmarker}{}%
\end{pgfscope}%
\begin{pgfscope}%
\pgfsys@transformshift{0.823451in}{0.919718in}%
\pgfsys@useobject{currentmarker}{}%
\end{pgfscope}%
\begin{pgfscope}%
\pgfsys@transformshift{0.946084in}{0.995221in}%
\pgfsys@useobject{currentmarker}{}%
\end{pgfscope}%
\begin{pgfscope}%
\pgfsys@transformshift{0.760484in}{1.024805in}%
\pgfsys@useobject{currentmarker}{}%
\end{pgfscope}%
\begin{pgfscope}%
\pgfsys@transformshift{0.833535in}{0.914410in}%
\pgfsys@useobject{currentmarker}{}%
\end{pgfscope}%
\begin{pgfscope}%
\pgfsys@transformshift{1.160222in}{0.971114in}%
\pgfsys@useobject{currentmarker}{}%
\end{pgfscope}%
\begin{pgfscope}%
\pgfsys@transformshift{0.687873in}{1.021233in}%
\pgfsys@useobject{currentmarker}{}%
\end{pgfscope}%
\begin{pgfscope}%
\pgfsys@transformshift{0.993968in}{0.713664in}%
\pgfsys@useobject{currentmarker}{}%
\end{pgfscope}%
\begin{pgfscope}%
\pgfsys@transformshift{1.351360in}{1.020625in}%
\pgfsys@useobject{currentmarker}{}%
\end{pgfscope}%
\begin{pgfscope}%
\pgfsys@transformshift{1.206817in}{0.939109in}%
\pgfsys@useobject{currentmarker}{}%
\end{pgfscope}%
\begin{pgfscope}%
\pgfsys@transformshift{1.105469in}{1.147414in}%
\pgfsys@useobject{currentmarker}{}%
\end{pgfscope}%
\begin{pgfscope}%
\pgfsys@transformshift{0.714371in}{0.916441in}%
\pgfsys@useobject{currentmarker}{}%
\end{pgfscope}%
\begin{pgfscope}%
\pgfsys@transformshift{0.850728in}{0.946062in}%
\pgfsys@useobject{currentmarker}{}%
\end{pgfscope}%
\begin{pgfscope}%
\pgfsys@transformshift{1.009561in}{1.005008in}%
\pgfsys@useobject{currentmarker}{}%
\end{pgfscope}%
\begin{pgfscope}%
\pgfsys@transformshift{0.893640in}{1.372729in}%
\pgfsys@useobject{currentmarker}{}%
\end{pgfscope}%
\begin{pgfscope}%
\pgfsys@transformshift{1.393309in}{1.061178in}%
\pgfsys@useobject{currentmarker}{}%
\end{pgfscope}%
\begin{pgfscope}%
\pgfsys@transformshift{0.686768in}{0.837406in}%
\pgfsys@useobject{currentmarker}{}%
\end{pgfscope}%
\begin{pgfscope}%
\pgfsys@transformshift{1.175702in}{0.957856in}%
\pgfsys@useobject{currentmarker}{}%
\end{pgfscope}%
\begin{pgfscope}%
\pgfsys@transformshift{0.897748in}{0.930704in}%
\pgfsys@useobject{currentmarker}{}%
\end{pgfscope}%
\begin{pgfscope}%
\pgfsys@transformshift{1.122040in}{1.062220in}%
\pgfsys@useobject{currentmarker}{}%
\end{pgfscope}%
\begin{pgfscope}%
\pgfsys@transformshift{0.759946in}{0.880191in}%
\pgfsys@useobject{currentmarker}{}%
\end{pgfscope}%
\begin{pgfscope}%
\pgfsys@transformshift{1.014716in}{1.011010in}%
\pgfsys@useobject{currentmarker}{}%
\end{pgfscope}%
\begin{pgfscope}%
\pgfsys@transformshift{0.650724in}{0.985411in}%
\pgfsys@useobject{currentmarker}{}%
\end{pgfscope}%
\begin{pgfscope}%
\pgfsys@transformshift{1.102439in}{0.882339in}%
\pgfsys@useobject{currentmarker}{}%
\end{pgfscope}%
\begin{pgfscope}%
\pgfsys@transformshift{0.649889in}{1.134392in}%
\pgfsys@useobject{currentmarker}{}%
\end{pgfscope}%
\begin{pgfscope}%
\pgfsys@transformshift{0.868714in}{1.266432in}%
\pgfsys@useobject{currentmarker}{}%
\end{pgfscope}%
\begin{pgfscope}%
\pgfsys@transformshift{0.687306in}{0.994438in}%
\pgfsys@useobject{currentmarker}{}%
\end{pgfscope}%
\begin{pgfscope}%
\pgfsys@transformshift{1.057147in}{1.311008in}%
\pgfsys@useobject{currentmarker}{}%
\end{pgfscope}%
\begin{pgfscope}%
\pgfsys@transformshift{0.815633in}{0.804893in}%
\pgfsys@useobject{currentmarker}{}%
\end{pgfscope}%
\begin{pgfscope}%
\pgfsys@transformshift{0.910055in}{1.023840in}%
\pgfsys@useobject{currentmarker}{}%
\end{pgfscope}%
\begin{pgfscope}%
\pgfsys@transformshift{1.092497in}{0.800680in}%
\pgfsys@useobject{currentmarker}{}%
\end{pgfscope}%
\begin{pgfscope}%
\pgfsys@transformshift{1.040364in}{0.782510in}%
\pgfsys@useobject{currentmarker}{}%
\end{pgfscope}%
\begin{pgfscope}%
\pgfsys@transformshift{1.102821in}{1.544800in}%
\pgfsys@useobject{currentmarker}{}%
\end{pgfscope}%
\begin{pgfscope}%
\pgfsys@transformshift{1.121714in}{0.989591in}%
\pgfsys@useobject{currentmarker}{}%
\end{pgfscope}%
\begin{pgfscope}%
\pgfsys@transformshift{0.836310in}{0.909066in}%
\pgfsys@useobject{currentmarker}{}%
\end{pgfscope}%
\begin{pgfscope}%
\pgfsys@transformshift{1.416507in}{1.187417in}%
\pgfsys@useobject{currentmarker}{}%
\end{pgfscope}%
\begin{pgfscope}%
\pgfsys@transformshift{0.723605in}{0.769330in}%
\pgfsys@useobject{currentmarker}{}%
\end{pgfscope}%
\begin{pgfscope}%
\pgfsys@transformshift{1.160576in}{0.902315in}%
\pgfsys@useobject{currentmarker}{}%
\end{pgfscope}%
\begin{pgfscope}%
\pgfsys@transformshift{1.019560in}{0.807736in}%
\pgfsys@useobject{currentmarker}{}%
\end{pgfscope}%
\begin{pgfscope}%
\pgfsys@transformshift{1.317752in}{1.174038in}%
\pgfsys@useobject{currentmarker}{}%
\end{pgfscope}%
\begin{pgfscope}%
\pgfsys@transformshift{1.163111in}{0.974582in}%
\pgfsys@useobject{currentmarker}{}%
\end{pgfscope}%
\begin{pgfscope}%
\pgfsys@transformshift{0.688425in}{0.826468in}%
\pgfsys@useobject{currentmarker}{}%
\end{pgfscope}%
\begin{pgfscope}%
\pgfsys@transformshift{0.993670in}{1.457858in}%
\pgfsys@useobject{currentmarker}{}%
\end{pgfscope}%
\begin{pgfscope}%
\pgfsys@transformshift{0.702021in}{0.833240in}%
\pgfsys@useobject{currentmarker}{}%
\end{pgfscope}%
\begin{pgfscope}%
\pgfsys@transformshift{1.002352in}{1.273864in}%
\pgfsys@useobject{currentmarker}{}%
\end{pgfscope}%
\begin{pgfscope}%
\pgfsys@transformshift{0.740827in}{1.124837in}%
\pgfsys@useobject{currentmarker}{}%
\end{pgfscope}%
\begin{pgfscope}%
\pgfsys@transformshift{0.673073in}{0.746235in}%
\pgfsys@useobject{currentmarker}{}%
\end{pgfscope}%
\begin{pgfscope}%
\pgfsys@transformshift{1.073136in}{0.804350in}%
\pgfsys@useobject{currentmarker}{}%
\end{pgfscope}%
\begin{pgfscope}%
\pgfsys@transformshift{1.206732in}{0.998255in}%
\pgfsys@useobject{currentmarker}{}%
\end{pgfscope}%
\begin{pgfscope}%
\pgfsys@transformshift{0.922150in}{0.679525in}%
\pgfsys@useobject{currentmarker}{}%
\end{pgfscope}%
\begin{pgfscope}%
\pgfsys@transformshift{0.978318in}{1.020830in}%
\pgfsys@useobject{currentmarker}{}%
\end{pgfscope}%
\begin{pgfscope}%
\pgfsys@transformshift{0.668116in}{0.740447in}%
\pgfsys@useobject{currentmarker}{}%
\end{pgfscope}%
\begin{pgfscope}%
\pgfsys@transformshift{0.692830in}{0.793731in}%
\pgfsys@useobject{currentmarker}{}%
\end{pgfscope}%
\begin{pgfscope}%
\pgfsys@transformshift{0.778768in}{0.994464in}%
\pgfsys@useobject{currentmarker}{}%
\end{pgfscope}%
\begin{pgfscope}%
\pgfsys@transformshift{0.732881in}{0.857665in}%
\pgfsys@useobject{currentmarker}{}%
\end{pgfscope}%
\begin{pgfscope}%
\pgfsys@transformshift{0.643275in}{0.813019in}%
\pgfsys@useobject{currentmarker}{}%
\end{pgfscope}%
\begin{pgfscope}%
\pgfsys@transformshift{0.892734in}{1.018322in}%
\pgfsys@useobject{currentmarker}{}%
\end{pgfscope}%
\begin{pgfscope}%
\pgfsys@transformshift{1.156837in}{0.904641in}%
\pgfsys@useobject{currentmarker}{}%
\end{pgfscope}%
\begin{pgfscope}%
\pgfsys@transformshift{0.802674in}{1.000330in}%
\pgfsys@useobject{currentmarker}{}%
\end{pgfscope}%
\begin{pgfscope}%
\pgfsys@transformshift{0.706851in}{0.893715in}%
\pgfsys@useobject{currentmarker}{}%
\end{pgfscope}%
\begin{pgfscope}%
\pgfsys@transformshift{0.747554in}{0.895315in}%
\pgfsys@useobject{currentmarker}{}%
\end{pgfscope}%
\begin{pgfscope}%
\pgfsys@transformshift{0.986702in}{0.895962in}%
\pgfsys@useobject{currentmarker}{}%
\end{pgfscope}%
\begin{pgfscope}%
\pgfsys@transformshift{0.598989in}{0.734463in}%
\pgfsys@useobject{currentmarker}{}%
\end{pgfscope}%
\begin{pgfscope}%
\pgfsys@transformshift{1.018157in}{1.317995in}%
\pgfsys@useobject{currentmarker}{}%
\end{pgfscope}%
\begin{pgfscope}%
\pgfsys@transformshift{0.674872in}{1.087934in}%
\pgfsys@useobject{currentmarker}{}%
\end{pgfscope}%
\begin{pgfscope}%
\pgfsys@transformshift{0.809543in}{1.035752in}%
\pgfsys@useobject{currentmarker}{}%
\end{pgfscope}%
\begin{pgfscope}%
\pgfsys@transformshift{0.963391in}{1.224282in}%
\pgfsys@useobject{currentmarker}{}%
\end{pgfscope}%
\begin{pgfscope}%
\pgfsys@transformshift{0.930180in}{1.002269in}%
\pgfsys@useobject{currentmarker}{}%
\end{pgfscope}%
\begin{pgfscope}%
\pgfsys@transformshift{0.973050in}{1.000722in}%
\pgfsys@useobject{currentmarker}{}%
\end{pgfscope}%
\begin{pgfscope}%
\pgfsys@transformshift{1.359659in}{1.267206in}%
\pgfsys@useobject{currentmarker}{}%
\end{pgfscope}%
\begin{pgfscope}%
\pgfsys@transformshift{1.084042in}{0.975652in}%
\pgfsys@useobject{currentmarker}{}%
\end{pgfscope}%
\begin{pgfscope}%
\pgfsys@transformshift{0.732853in}{0.668247in}%
\pgfsys@useobject{currentmarker}{}%
\end{pgfscope}%
\begin{pgfscope}%
\pgfsys@transformshift{0.990838in}{1.065210in}%
\pgfsys@useobject{currentmarker}{}%
\end{pgfscope}%
\begin{pgfscope}%
\pgfsys@transformshift{0.890425in}{1.344423in}%
\pgfsys@useobject{currentmarker}{}%
\end{pgfscope}%
\begin{pgfscope}%
\pgfsys@transformshift{1.093162in}{0.982118in}%
\pgfsys@useobject{currentmarker}{}%
\end{pgfscope}%
\begin{pgfscope}%
\pgfsys@transformshift{1.066253in}{0.828271in}%
\pgfsys@useobject{currentmarker}{}%
\end{pgfscope}%
\begin{pgfscope}%
\pgfsys@transformshift{0.839908in}{1.027753in}%
\pgfsys@useobject{currentmarker}{}%
\end{pgfscope}%
\begin{pgfscope}%
\pgfsys@transformshift{1.284229in}{1.192737in}%
\pgfsys@useobject{currentmarker}{}%
\end{pgfscope}%
\begin{pgfscope}%
\pgfsys@transformshift{1.283818in}{0.972887in}%
\pgfsys@useobject{currentmarker}{}%
\end{pgfscope}%
\begin{pgfscope}%
\pgfsys@transformshift{1.283889in}{0.968942in}%
\pgfsys@useobject{currentmarker}{}%
\end{pgfscope}%
\begin{pgfscope}%
\pgfsys@transformshift{1.008286in}{0.942605in}%
\pgfsys@useobject{currentmarker}{}%
\end{pgfscope}%
\begin{pgfscope}%
\pgfsys@transformshift{0.796599in}{1.078506in}%
\pgfsys@useobject{currentmarker}{}%
\end{pgfscope}%
\begin{pgfscope}%
\pgfsys@transformshift{0.671288in}{0.969467in}%
\pgfsys@useobject{currentmarker}{}%
\end{pgfscope}%
\begin{pgfscope}%
\pgfsys@transformshift{1.369530in}{1.097753in}%
\pgfsys@useobject{currentmarker}{}%
\end{pgfscope}%
\begin{pgfscope}%
\pgfsys@transformshift{0.788979in}{0.960504in}%
\pgfsys@useobject{currentmarker}{}%
\end{pgfscope}%
\begin{pgfscope}%
\pgfsys@transformshift{1.035634in}{1.235298in}%
\pgfsys@useobject{currentmarker}{}%
\end{pgfscope}%
\begin{pgfscope}%
\pgfsys@transformshift{0.763288in}{0.856960in}%
\pgfsys@useobject{currentmarker}{}%
\end{pgfscope}%
\begin{pgfscope}%
\pgfsys@transformshift{1.352223in}{1.183744in}%
\pgfsys@useobject{currentmarker}{}%
\end{pgfscope}%
\begin{pgfscope}%
\pgfsys@transformshift{1.098587in}{1.062719in}%
\pgfsys@useobject{currentmarker}{}%
\end{pgfscope}%
\begin{pgfscope}%
\pgfsys@transformshift{0.735048in}{1.110610in}%
\pgfsys@useobject{currentmarker}{}%
\end{pgfscope}%
\begin{pgfscope}%
\pgfsys@transformshift{1.236643in}{1.057545in}%
\pgfsys@useobject{currentmarker}{}%
\end{pgfscope}%
\begin{pgfscope}%
\pgfsys@transformshift{1.460581in}{1.186957in}%
\pgfsys@useobject{currentmarker}{}%
\end{pgfscope}%
\begin{pgfscope}%
\pgfsys@transformshift{1.154868in}{1.203676in}%
\pgfsys@useobject{currentmarker}{}%
\end{pgfscope}%
\begin{pgfscope}%
\pgfsys@transformshift{0.908199in}{0.653295in}%
\pgfsys@useobject{currentmarker}{}%
\end{pgfscope}%
\begin{pgfscope}%
\pgfsys@transformshift{0.867369in}{0.919121in}%
\pgfsys@useobject{currentmarker}{}%
\end{pgfscope}%
\begin{pgfscope}%
\pgfsys@transformshift{1.037560in}{0.864925in}%
\pgfsys@useobject{currentmarker}{}%
\end{pgfscope}%
\begin{pgfscope}%
\pgfsys@transformshift{1.063690in}{1.159536in}%
\pgfsys@useobject{currentmarker}{}%
\end{pgfscope}%
\begin{pgfscope}%
\pgfsys@transformshift{1.073986in}{0.728431in}%
\pgfsys@useobject{currentmarker}{}%
\end{pgfscope}%
\begin{pgfscope}%
\pgfsys@transformshift{1.069114in}{0.955595in}%
\pgfsys@useobject{currentmarker}{}%
\end{pgfscope}%
\begin{pgfscope}%
\pgfsys@transformshift{1.001177in}{0.868058in}%
\pgfsys@useobject{currentmarker}{}%
\end{pgfscope}%
\begin{pgfscope}%
\pgfsys@transformshift{0.673781in}{0.727169in}%
\pgfsys@useobject{currentmarker}{}%
\end{pgfscope}%
\begin{pgfscope}%
\pgfsys@transformshift{0.898569in}{0.951978in}%
\pgfsys@useobject{currentmarker}{}%
\end{pgfscope}%
\begin{pgfscope}%
\pgfsys@transformshift{0.851875in}{0.963584in}%
\pgfsys@useobject{currentmarker}{}%
\end{pgfscope}%
\begin{pgfscope}%
\pgfsys@transformshift{0.709981in}{0.942466in}%
\pgfsys@useobject{currentmarker}{}%
\end{pgfscope}%
\begin{pgfscope}%
\pgfsys@transformshift{1.526168in}{1.247016in}%
\pgfsys@useobject{currentmarker}{}%
\end{pgfscope}%
\begin{pgfscope}%
\pgfsys@transformshift{0.578368in}{0.965667in}%
\pgfsys@useobject{currentmarker}{}%
\end{pgfscope}%
\begin{pgfscope}%
\pgfsys@transformshift{0.674532in}{0.995221in}%
\pgfsys@useobject{currentmarker}{}%
\end{pgfscope}%
\begin{pgfscope}%
\pgfsys@transformshift{1.173322in}{0.976232in}%
\pgfsys@useobject{currentmarker}{}%
\end{pgfscope}%
\begin{pgfscope}%
\pgfsys@transformshift{1.427228in}{1.223169in}%
\pgfsys@useobject{currentmarker}{}%
\end{pgfscope}%
\begin{pgfscope}%
\pgfsys@transformshift{0.692136in}{0.866834in}%
\pgfsys@useobject{currentmarker}{}%
\end{pgfscope}%
\begin{pgfscope}%
\pgfsys@transformshift{0.833605in}{0.837646in}%
\pgfsys@useobject{currentmarker}{}%
\end{pgfscope}%
\begin{pgfscope}%
\pgfsys@transformshift{0.842854in}{1.092391in}%
\pgfsys@useobject{currentmarker}{}%
\end{pgfscope}%
\begin{pgfscope}%
\pgfsys@transformshift{1.049018in}{1.586648in}%
\pgfsys@useobject{currentmarker}{}%
\end{pgfscope}%
\begin{pgfscope}%
\pgfsys@transformshift{0.662522in}{1.051197in}%
\pgfsys@useobject{currentmarker}{}%
\end{pgfscope}%
\begin{pgfscope}%
\pgfsys@transformshift{0.841905in}{0.992793in}%
\pgfsys@useobject{currentmarker}{}%
\end{pgfscope}%
\begin{pgfscope}%
\pgfsys@transformshift{1.022364in}{0.894154in}%
\pgfsys@useobject{currentmarker}{}%
\end{pgfscope}%
\begin{pgfscope}%
\pgfsys@transformshift{1.013455in}{0.688602in}%
\pgfsys@useobject{currentmarker}{}%
\end{pgfscope}%
\begin{pgfscope}%
\pgfsys@transformshift{1.360976in}{1.183930in}%
\pgfsys@useobject{currentmarker}{}%
\end{pgfscope}%
\begin{pgfscope}%
\pgfsys@transformshift{1.189708in}{0.943756in}%
\pgfsys@useobject{currentmarker}{}%
\end{pgfscope}%
\begin{pgfscope}%
\pgfsys@transformshift{0.801669in}{0.912385in}%
\pgfsys@useobject{currentmarker}{}%
\end{pgfscope}%
\begin{pgfscope}%
\pgfsys@transformshift{0.670920in}{0.861561in}%
\pgfsys@useobject{currentmarker}{}%
\end{pgfscope}%
\begin{pgfscope}%
\pgfsys@transformshift{1.030720in}{0.782443in}%
\pgfsys@useobject{currentmarker}{}%
\end{pgfscope}%
\begin{pgfscope}%
\pgfsys@transformshift{0.993274in}{1.142639in}%
\pgfsys@useobject{currentmarker}{}%
\end{pgfscope}%
\begin{pgfscope}%
\pgfsys@transformshift{0.890695in}{1.467892in}%
\pgfsys@useobject{currentmarker}{}%
\end{pgfscope}%
\begin{pgfscope}%
\pgfsys@transformshift{0.660992in}{0.929867in}%
\pgfsys@useobject{currentmarker}{}%
\end{pgfscope}%
\begin{pgfscope}%
\pgfsys@transformshift{0.830759in}{0.889251in}%
\pgfsys@useobject{currentmarker}{}%
\end{pgfscope}%
\begin{pgfscope}%
\pgfsys@transformshift{0.823507in}{0.797991in}%
\pgfsys@useobject{currentmarker}{}%
\end{pgfscope}%
\begin{pgfscope}%
\pgfsys@transformshift{0.923750in}{0.989105in}%
\pgfsys@useobject{currentmarker}{}%
\end{pgfscope}%
\begin{pgfscope}%
\pgfsys@transformshift{0.685833in}{1.052601in}%
\pgfsys@useobject{currentmarker}{}%
\end{pgfscope}%
\begin{pgfscope}%
\pgfsys@transformshift{0.680834in}{0.881147in}%
\pgfsys@useobject{currentmarker}{}%
\end{pgfscope}%
\begin{pgfscope}%
\pgfsys@transformshift{0.971888in}{1.100187in}%
\pgfsys@useobject{currentmarker}{}%
\end{pgfscope}%
\begin{pgfscope}%
\pgfsys@transformshift{1.162530in}{1.041088in}%
\pgfsys@useobject{currentmarker}{}%
\end{pgfscope}%
\begin{pgfscope}%
\pgfsys@transformshift{1.030748in}{1.348762in}%
\pgfsys@useobject{currentmarker}{}%
\end{pgfscope}%
\begin{pgfscope}%
\pgfsys@transformshift{1.138341in}{1.055629in}%
\pgfsys@useobject{currentmarker}{}%
\end{pgfscope}%
\begin{pgfscope}%
\pgfsys@transformshift{0.689629in}{0.763196in}%
\pgfsys@useobject{currentmarker}{}%
\end{pgfscope}%
\begin{pgfscope}%
\pgfsys@transformshift{0.879690in}{0.920697in}%
\pgfsys@useobject{currentmarker}{}%
\end{pgfscope}%
\begin{pgfscope}%
\pgfsys@transformshift{1.440838in}{1.145149in}%
\pgfsys@useobject{currentmarker}{}%
\end{pgfscope}%
\begin{pgfscope}%
\pgfsys@transformshift{1.150251in}{1.017083in}%
\pgfsys@useobject{currentmarker}{}%
\end{pgfscope}%
\begin{pgfscope}%
\pgfsys@transformshift{1.137746in}{0.967140in}%
\pgfsys@useobject{currentmarker}{}%
\end{pgfscope}%
\begin{pgfscope}%
\pgfsys@transformshift{1.109293in}{1.047760in}%
\pgfsys@useobject{currentmarker}{}%
\end{pgfscope}%
\begin{pgfscope}%
\pgfsys@transformshift{1.047035in}{0.715003in}%
\pgfsys@useobject{currentmarker}{}%
\end{pgfscope}%
\begin{pgfscope}%
\pgfsys@transformshift{1.062826in}{0.926197in}%
\pgfsys@useobject{currentmarker}{}%
\end{pgfscope}%
\begin{pgfscope}%
\pgfsys@transformshift{0.683893in}{0.992260in}%
\pgfsys@useobject{currentmarker}{}%
\end{pgfscope}%
\begin{pgfscope}%
\pgfsys@transformshift{0.760243in}{0.960976in}%
\pgfsys@useobject{currentmarker}{}%
\end{pgfscope}%
\begin{pgfscope}%
\pgfsys@transformshift{0.919105in}{1.028810in}%
\pgfsys@useobject{currentmarker}{}%
\end{pgfscope}%
\begin{pgfscope}%
\pgfsys@transformshift{0.728618in}{1.051545in}%
\pgfsys@useobject{currentmarker}{}%
\end{pgfscope}%
\begin{pgfscope}%
\pgfsys@transformshift{1.189057in}{0.957364in}%
\pgfsys@useobject{currentmarker}{}%
\end{pgfscope}%
\begin{pgfscope}%
\pgfsys@transformshift{0.837415in}{0.790515in}%
\pgfsys@useobject{currentmarker}{}%
\end{pgfscope}%
\begin{pgfscope}%
\pgfsys@transformshift{0.783597in}{0.924375in}%
\pgfsys@useobject{currentmarker}{}%
\end{pgfscope}%
\begin{pgfscope}%
\pgfsys@transformshift{1.081747in}{0.999735in}%
\pgfsys@useobject{currentmarker}{}%
\end{pgfscope}%
\begin{pgfscope}%
\pgfsys@transformshift{0.699090in}{1.273846in}%
\pgfsys@useobject{currentmarker}{}%
\end{pgfscope}%
\begin{pgfscope}%
\pgfsys@transformshift{0.746506in}{0.961496in}%
\pgfsys@useobject{currentmarker}{}%
\end{pgfscope}%
\begin{pgfscope}%
\pgfsys@transformshift{0.609851in}{0.751148in}%
\pgfsys@useobject{currentmarker}{}%
\end{pgfscope}%
\begin{pgfscope}%
\pgfsys@transformshift{0.925747in}{0.982804in}%
\pgfsys@useobject{currentmarker}{}%
\end{pgfscope}%
\begin{pgfscope}%
\pgfsys@transformshift{0.992226in}{1.119707in}%
\pgfsys@useobject{currentmarker}{}%
\end{pgfscope}%
\begin{pgfscope}%
\pgfsys@transformshift{0.827190in}{0.889524in}%
\pgfsys@useobject{currentmarker}{}%
\end{pgfscope}%
\begin{pgfscope}%
\pgfsys@transformshift{1.369190in}{1.049429in}%
\pgfsys@useobject{currentmarker}{}%
\end{pgfscope}%
\begin{pgfscope}%
\pgfsys@transformshift{0.785424in}{1.003856in}%
\pgfsys@useobject{currentmarker}{}%
\end{pgfscope}%
\begin{pgfscope}%
\pgfsys@transformshift{0.757312in}{0.732602in}%
\pgfsys@useobject{currentmarker}{}%
\end{pgfscope}%
\begin{pgfscope}%
\pgfsys@transformshift{0.948322in}{1.018919in}%
\pgfsys@useobject{currentmarker}{}%
\end{pgfscope}%
\begin{pgfscope}%
\pgfsys@transformshift{1.035804in}{0.825354in}%
\pgfsys@useobject{currentmarker}{}%
\end{pgfscope}%
\begin{pgfscope}%
\pgfsys@transformshift{1.007422in}{0.777357in}%
\pgfsys@useobject{currentmarker}{}%
\end{pgfscope}%
\begin{pgfscope}%
\pgfsys@transformshift{1.000383in}{0.922201in}%
\pgfsys@useobject{currentmarker}{}%
\end{pgfscope}%
\begin{pgfscope}%
\pgfsys@transformshift{1.395943in}{1.077058in}%
\pgfsys@useobject{currentmarker}{}%
\end{pgfscope}%
\begin{pgfscope}%
\pgfsys@transformshift{1.364786in}{1.157423in}%
\pgfsys@useobject{currentmarker}{}%
\end{pgfscope}%
\begin{pgfscope}%
\pgfsys@transformshift{0.746931in}{0.848473in}%
\pgfsys@useobject{currentmarker}{}%
\end{pgfscope}%
\begin{pgfscope}%
\pgfsys@transformshift{0.603577in}{1.166644in}%
\pgfsys@useobject{currentmarker}{}%
\end{pgfscope}%
\begin{pgfscope}%
\pgfsys@transformshift{0.651079in}{0.753207in}%
\pgfsys@useobject{currentmarker}{}%
\end{pgfscope}%
\begin{pgfscope}%
\pgfsys@transformshift{1.146229in}{1.082940in}%
\pgfsys@useobject{currentmarker}{}%
\end{pgfscope}%
\begin{pgfscope}%
\pgfsys@transformshift{0.881800in}{1.059448in}%
\pgfsys@useobject{currentmarker}{}%
\end{pgfscope}%
\begin{pgfscope}%
\pgfsys@transformshift{0.962909in}{1.010201in}%
\pgfsys@useobject{currentmarker}{}%
\end{pgfscope}%
\begin{pgfscope}%
\pgfsys@transformshift{0.853829in}{1.220568in}%
\pgfsys@useobject{currentmarker}{}%
\end{pgfscope}%
\begin{pgfscope}%
\pgfsys@transformshift{0.707516in}{0.654620in}%
\pgfsys@useobject{currentmarker}{}%
\end{pgfscope}%
\begin{pgfscope}%
\pgfsys@transformshift{0.593479in}{0.767723in}%
\pgfsys@useobject{currentmarker}{}%
\end{pgfscope}%
\begin{pgfscope}%
\pgfsys@transformshift{0.804133in}{0.849120in}%
\pgfsys@useobject{currentmarker}{}%
\end{pgfscope}%
\begin{pgfscope}%
\pgfsys@transformshift{1.253581in}{1.088332in}%
\pgfsys@useobject{currentmarker}{}%
\end{pgfscope}%
\begin{pgfscope}%
\pgfsys@transformshift{0.804827in}{1.152374in}%
\pgfsys@useobject{currentmarker}{}%
\end{pgfscope}%
\begin{pgfscope}%
\pgfsys@transformshift{0.678072in}{0.790453in}%
\pgfsys@useobject{currentmarker}{}%
\end{pgfscope}%
\begin{pgfscope}%
\pgfsys@transformshift{0.851932in}{1.069397in}%
\pgfsys@useobject{currentmarker}{}%
\end{pgfscope}%
\begin{pgfscope}%
\pgfsys@transformshift{0.995568in}{1.349633in}%
\pgfsys@useobject{currentmarker}{}%
\end{pgfscope}%
\begin{pgfscope}%
\pgfsys@transformshift{0.800748in}{0.898285in}%
\pgfsys@useobject{currentmarker}{}%
\end{pgfscope}%
\begin{pgfscope}%
\pgfsys@transformshift{0.947472in}{1.412328in}%
\pgfsys@useobject{currentmarker}{}%
\end{pgfscope}%
\begin{pgfscope}%
\pgfsys@transformshift{0.949214in}{1.591795in}%
\pgfsys@useobject{currentmarker}{}%
\end{pgfscope}%
\begin{pgfscope}%
\pgfsys@transformshift{0.905877in}{0.834658in}%
\pgfsys@useobject{currentmarker}{}%
\end{pgfscope}%
\begin{pgfscope}%
\pgfsys@transformshift{0.686527in}{1.154244in}%
\pgfsys@useobject{currentmarker}{}%
\end{pgfscope}%
\begin{pgfscope}%
\pgfsys@transformshift{1.028085in}{0.738034in}%
\pgfsys@useobject{currentmarker}{}%
\end{pgfscope}%
\begin{pgfscope}%
\pgfsys@transformshift{0.846961in}{0.966198in}%
\pgfsys@useobject{currentmarker}{}%
\end{pgfscope}%
\begin{pgfscope}%
\pgfsys@transformshift{1.278238in}{1.065252in}%
\pgfsys@useobject{currentmarker}{}%
\end{pgfscope}%
\begin{pgfscope}%
\pgfsys@transformshift{1.139502in}{1.058535in}%
\pgfsys@useobject{currentmarker}{}%
\end{pgfscope}%
\begin{pgfscope}%
\pgfsys@transformshift{0.948095in}{1.392092in}%
\pgfsys@useobject{currentmarker}{}%
\end{pgfscope}%
\begin{pgfscope}%
\pgfsys@transformshift{0.675877in}{0.920252in}%
\pgfsys@useobject{currentmarker}{}%
\end{pgfscope}%
\begin{pgfscope}%
\pgfsys@transformshift{1.076365in}{0.836306in}%
\pgfsys@useobject{currentmarker}{}%
\end{pgfscope}%
\begin{pgfscope}%
\pgfsys@transformshift{1.042432in}{1.655738in}%
\pgfsys@useobject{currentmarker}{}%
\end{pgfscope}%
\begin{pgfscope}%
\pgfsys@transformshift{0.643105in}{0.673904in}%
\pgfsys@useobject{currentmarker}{}%
\end{pgfscope}%
\begin{pgfscope}%
\pgfsys@transformshift{0.674079in}{0.886756in}%
\pgfsys@useobject{currentmarker}{}%
\end{pgfscope}%
\begin{pgfscope}%
\pgfsys@transformshift{0.972710in}{1.359983in}%
\pgfsys@useobject{currentmarker}{}%
\end{pgfscope}%
\begin{pgfscope}%
\pgfsys@transformshift{1.319947in}{1.064157in}%
\pgfsys@useobject{currentmarker}{}%
\end{pgfscope}%
\begin{pgfscope}%
\pgfsys@transformshift{0.638870in}{0.635823in}%
\pgfsys@useobject{currentmarker}{}%
\end{pgfscope}%
\begin{pgfscope}%
\pgfsys@transformshift{0.967498in}{1.326250in}%
\pgfsys@useobject{currentmarker}{}%
\end{pgfscope}%
\begin{pgfscope}%
\pgfsys@transformshift{1.043126in}{1.318599in}%
\pgfsys@useobject{currentmarker}{}%
\end{pgfscope}%
\begin{pgfscope}%
\pgfsys@transformshift{0.933975in}{1.276563in}%
\pgfsys@useobject{currentmarker}{}%
\end{pgfscope}%
\begin{pgfscope}%
\pgfsys@transformshift{1.197569in}{0.979184in}%
\pgfsys@useobject{currentmarker}{}%
\end{pgfscope}%
\begin{pgfscope}%
\pgfsys@transformshift{1.311903in}{0.979203in}%
\pgfsys@useobject{currentmarker}{}%
\end{pgfscope}%
\begin{pgfscope}%
\pgfsys@transformshift{0.721183in}{1.140505in}%
\pgfsys@useobject{currentmarker}{}%
\end{pgfscope}%
\begin{pgfscope}%
\pgfsys@transformshift{0.973517in}{1.172753in}%
\pgfsys@useobject{currentmarker}{}%
\end{pgfscope}%
\begin{pgfscope}%
\pgfsys@transformshift{1.160293in}{0.944424in}%
\pgfsys@useobject{currentmarker}{}%
\end{pgfscope}%
\begin{pgfscope}%
\pgfsys@transformshift{0.675481in}{0.950870in}%
\pgfsys@useobject{currentmarker}{}%
\end{pgfscope}%
\begin{pgfscope}%
\pgfsys@transformshift{0.777677in}{0.884640in}%
\pgfsys@useobject{currentmarker}{}%
\end{pgfscope}%
\begin{pgfscope}%
\pgfsys@transformshift{0.834356in}{0.876517in}%
\pgfsys@useobject{currentmarker}{}%
\end{pgfscope}%
\begin{pgfscope}%
\pgfsys@transformshift{0.763005in}{1.108723in}%
\pgfsys@useobject{currentmarker}{}%
\end{pgfscope}%
\begin{pgfscope}%
\pgfsys@transformshift{0.861831in}{0.930575in}%
\pgfsys@useobject{currentmarker}{}%
\end{pgfscope}%
\begin{pgfscope}%
\pgfsys@transformshift{0.673824in}{0.963537in}%
\pgfsys@useobject{currentmarker}{}%
\end{pgfscope}%
\begin{pgfscope}%
\pgfsys@transformshift{1.084877in}{1.007403in}%
\pgfsys@useobject{currentmarker}{}%
\end{pgfscope}%
\begin{pgfscope}%
\pgfsys@transformshift{0.692377in}{0.949806in}%
\pgfsys@useobject{currentmarker}{}%
\end{pgfscope}%
\begin{pgfscope}%
\pgfsys@transformshift{1.001049in}{1.638648in}%
\pgfsys@useobject{currentmarker}{}%
\end{pgfscope}%
\begin{pgfscope}%
\pgfsys@transformshift{1.054513in}{0.783543in}%
\pgfsys@useobject{currentmarker}{}%
\end{pgfscope}%
\begin{pgfscope}%
\pgfsys@transformshift{1.083999in}{1.374369in}%
\pgfsys@useobject{currentmarker}{}%
\end{pgfscope}%
\begin{pgfscope}%
\pgfsys@transformshift{0.724143in}{0.756916in}%
\pgfsys@useobject{currentmarker}{}%
\end{pgfscope}%
\begin{pgfscope}%
\pgfsys@transformshift{1.202780in}{1.068646in}%
\pgfsys@useobject{currentmarker}{}%
\end{pgfscope}%
\begin{pgfscope}%
\pgfsys@transformshift{1.325230in}{1.156998in}%
\pgfsys@useobject{currentmarker}{}%
\end{pgfscope}%
\begin{pgfscope}%
\pgfsys@transformshift{0.758501in}{0.953555in}%
\pgfsys@useobject{currentmarker}{}%
\end{pgfscope}%
\begin{pgfscope}%
\pgfsys@transformshift{0.647255in}{0.826086in}%
\pgfsys@useobject{currentmarker}{}%
\end{pgfscope}%
\begin{pgfscope}%
\pgfsys@transformshift{0.595335in}{0.696663in}%
\pgfsys@useobject{currentmarker}{}%
\end{pgfscope}%
\begin{pgfscope}%
\pgfsys@transformshift{0.783753in}{0.848061in}%
\pgfsys@useobject{currentmarker}{}%
\end{pgfscope}%
\begin{pgfscope}%
\pgfsys@transformshift{0.754508in}{0.902279in}%
\pgfsys@useobject{currentmarker}{}%
\end{pgfscope}%
\begin{pgfscope}%
\pgfsys@transformshift{0.763600in}{0.896331in}%
\pgfsys@useobject{currentmarker}{}%
\end{pgfscope}%
\begin{pgfscope}%
\pgfsys@transformshift{0.697758in}{0.887581in}%
\pgfsys@useobject{currentmarker}{}%
\end{pgfscope}%
\begin{pgfscope}%
\pgfsys@transformshift{1.027901in}{0.842963in}%
\pgfsys@useobject{currentmarker}{}%
\end{pgfscope}%
\begin{pgfscope}%
\pgfsys@transformshift{1.265011in}{1.193393in}%
\pgfsys@useobject{currentmarker}{}%
\end{pgfscope}%
\begin{pgfscope}%
\pgfsys@transformshift{1.031909in}{0.795479in}%
\pgfsys@useobject{currentmarker}{}%
\end{pgfscope}%
\begin{pgfscope}%
\pgfsys@transformshift{1.046837in}{0.980138in}%
\pgfsys@useobject{currentmarker}{}%
\end{pgfscope}%
\begin{pgfscope}%
\pgfsys@transformshift{1.056680in}{1.406044in}%
\pgfsys@useobject{currentmarker}{}%
\end{pgfscope}%
\begin{pgfscope}%
\pgfsys@transformshift{0.978601in}{1.357449in}%
\pgfsys@useobject{currentmarker}{}%
\end{pgfscope}%
\begin{pgfscope}%
\pgfsys@transformshift{0.718039in}{0.832036in}%
\pgfsys@useobject{currentmarker}{}%
\end{pgfscope}%
\begin{pgfscope}%
\pgfsys@transformshift{1.094097in}{0.845486in}%
\pgfsys@useobject{currentmarker}{}%
\end{pgfscope}%
\begin{pgfscope}%
\pgfsys@transformshift{0.595901in}{0.821433in}%
\pgfsys@useobject{currentmarker}{}%
\end{pgfscope}%
\begin{pgfscope}%
\pgfsys@transformshift{1.006275in}{1.307387in}%
\pgfsys@useobject{currentmarker}{}%
\end{pgfscope}%
\begin{pgfscope}%
\pgfsys@transformshift{0.778796in}{1.091158in}%
\pgfsys@useobject{currentmarker}{}%
\end{pgfscope}%
\begin{pgfscope}%
\pgfsys@transformshift{1.131344in}{1.496783in}%
\pgfsys@useobject{currentmarker}{}%
\end{pgfscope}%
\begin{pgfscope}%
\pgfsys@transformshift{0.623433in}{0.886821in}%
\pgfsys@useobject{currentmarker}{}%
\end{pgfscope}%
\begin{pgfscope}%
\pgfsys@transformshift{0.725163in}{0.738036in}%
\pgfsys@useobject{currentmarker}{}%
\end{pgfscope}%
\begin{pgfscope}%
\pgfsys@transformshift{0.811668in}{1.218267in}%
\pgfsys@useobject{currentmarker}{}%
\end{pgfscope}%
\begin{pgfscope}%
\pgfsys@transformshift{0.736252in}{0.837026in}%
\pgfsys@useobject{currentmarker}{}%
\end{pgfscope}%
\begin{pgfscope}%
\pgfsys@transformshift{0.895666in}{1.071619in}%
\pgfsys@useobject{currentmarker}{}%
\end{pgfscope}%
\begin{pgfscope}%
\pgfsys@transformshift{0.961635in}{1.126038in}%
\pgfsys@useobject{currentmarker}{}%
\end{pgfscope}%
\begin{pgfscope}%
\pgfsys@transformshift{1.340794in}{1.013184in}%
\pgfsys@useobject{currentmarker}{}%
\end{pgfscope}%
\begin{pgfscope}%
\pgfsys@transformshift{1.397657in}{1.301730in}%
\pgfsys@useobject{currentmarker}{}%
\end{pgfscope}%
\begin{pgfscope}%
\pgfsys@transformshift{0.906500in}{0.899890in}%
\pgfsys@useobject{currentmarker}{}%
\end{pgfscope}%
\begin{pgfscope}%
\pgfsys@transformshift{0.763232in}{0.881141in}%
\pgfsys@useobject{currentmarker}{}%
\end{pgfscope}%
\begin{pgfscope}%
\pgfsys@transformshift{0.688397in}{1.264440in}%
\pgfsys@useobject{currentmarker}{}%
\end{pgfscope}%
\begin{pgfscope}%
\pgfsys@transformshift{0.742668in}{1.059508in}%
\pgfsys@useobject{currentmarker}{}%
\end{pgfscope}%
\begin{pgfscope}%
\pgfsys@transformshift{0.649252in}{0.991296in}%
\pgfsys@useobject{currentmarker}{}%
\end{pgfscope}%
\begin{pgfscope}%
\pgfsys@transformshift{0.786600in}{0.742330in}%
\pgfsys@useobject{currentmarker}{}%
\end{pgfscope}%
\begin{pgfscope}%
\pgfsys@transformshift{0.715985in}{1.032812in}%
\pgfsys@useobject{currentmarker}{}%
\end{pgfscope}%
\begin{pgfscope}%
\pgfsys@transformshift{0.917207in}{0.913293in}%
\pgfsys@useobject{currentmarker}{}%
\end{pgfscope}%
\begin{pgfscope}%
\pgfsys@transformshift{0.998641in}{0.745309in}%
\pgfsys@useobject{currentmarker}{}%
\end{pgfscope}%
\begin{pgfscope}%
\pgfsys@transformshift{0.703352in}{0.683388in}%
\pgfsys@useobject{currentmarker}{}%
\end{pgfscope}%
\begin{pgfscope}%
\pgfsys@transformshift{0.739609in}{0.933696in}%
\pgfsys@useobject{currentmarker}{}%
\end{pgfscope}%
\begin{pgfscope}%
\pgfsys@transformshift{0.655228in}{0.863978in}%
\pgfsys@useobject{currentmarker}{}%
\end{pgfscope}%
\begin{pgfscope}%
\pgfsys@transformshift{0.740982in}{0.907075in}%
\pgfsys@useobject{currentmarker}{}%
\end{pgfscope}%
\begin{pgfscope}%
\pgfsys@transformshift{0.838208in}{0.969319in}%
\pgfsys@useobject{currentmarker}{}%
\end{pgfscope}%
\begin{pgfscope}%
\pgfsys@transformshift{1.072485in}{0.745509in}%
\pgfsys@useobject{currentmarker}{}%
\end{pgfscope}%
\begin{pgfscope}%
\pgfsys@transformshift{0.708536in}{1.064787in}%
\pgfsys@useobject{currentmarker}{}%
\end{pgfscope}%
\begin{pgfscope}%
\pgfsys@transformshift{1.229519in}{1.006833in}%
\pgfsys@useobject{currentmarker}{}%
\end{pgfscope}%
\begin{pgfscope}%
\pgfsys@transformshift{0.988416in}{1.392813in}%
\pgfsys@useobject{currentmarker}{}%
\end{pgfscope}%
\begin{pgfscope}%
\pgfsys@transformshift{1.368610in}{1.196272in}%
\pgfsys@useobject{currentmarker}{}%
\end{pgfscope}%
\begin{pgfscope}%
\pgfsys@transformshift{0.604739in}{0.626330in}%
\pgfsys@useobject{currentmarker}{}%
\end{pgfscope}%
\begin{pgfscope}%
\pgfsys@transformshift{1.367108in}{1.189392in}%
\pgfsys@useobject{currentmarker}{}%
\end{pgfscope}%
\begin{pgfscope}%
\pgfsys@transformshift{1.392275in}{1.183279in}%
\pgfsys@useobject{currentmarker}{}%
\end{pgfscope}%
\begin{pgfscope}%
\pgfsys@transformshift{1.017435in}{0.773638in}%
\pgfsys@useobject{currentmarker}{}%
\end{pgfscope}%
\begin{pgfscope}%
\pgfsys@transformshift{0.640966in}{0.613500in}%
\pgfsys@useobject{currentmarker}{}%
\end{pgfscope}%
\begin{pgfscope}%
\pgfsys@transformshift{1.169300in}{1.460577in}%
\pgfsys@useobject{currentmarker}{}%
\end{pgfscope}%
\begin{pgfscope}%
\pgfsys@transformshift{1.186196in}{1.165488in}%
\pgfsys@useobject{currentmarker}{}%
\end{pgfscope}%
\begin{pgfscope}%
\pgfsys@transformshift{0.905778in}{1.070860in}%
\pgfsys@useobject{currentmarker}{}%
\end{pgfscope}%
\begin{pgfscope}%
\pgfsys@transformshift{0.647269in}{0.889217in}%
\pgfsys@useobject{currentmarker}{}%
\end{pgfscope}%
\begin{pgfscope}%
\pgfsys@transformshift{0.700095in}{0.995237in}%
\pgfsys@useobject{currentmarker}{}%
\end{pgfscope}%
\begin{pgfscope}%
\pgfsys@transformshift{1.187329in}{1.092610in}%
\pgfsys@useobject{currentmarker}{}%
\end{pgfscope}%
\begin{pgfscope}%
\pgfsys@transformshift{0.689799in}{0.742965in}%
\pgfsys@useobject{currentmarker}{}%
\end{pgfscope}%
\begin{pgfscope}%
\pgfsys@transformshift{1.231771in}{1.148043in}%
\pgfsys@useobject{currentmarker}{}%
\end{pgfscope}%
\begin{pgfscope}%
\pgfsys@transformshift{0.699132in}{0.702977in}%
\pgfsys@useobject{currentmarker}{}%
\end{pgfscope}%
\begin{pgfscope}%
\pgfsys@transformshift{1.665301in}{1.248899in}%
\pgfsys@useobject{currentmarker}{}%
\end{pgfscope}%
\begin{pgfscope}%
\pgfsys@transformshift{1.284314in}{1.149850in}%
\pgfsys@useobject{currentmarker}{}%
\end{pgfscope}%
\begin{pgfscope}%
\pgfsys@transformshift{0.761362in}{0.874293in}%
\pgfsys@useobject{currentmarker}{}%
\end{pgfscope}%
\begin{pgfscope}%
\pgfsys@transformshift{0.831736in}{0.769894in}%
\pgfsys@useobject{currentmarker}{}%
\end{pgfscope}%
\begin{pgfscope}%
\pgfsys@transformshift{1.105937in}{1.457837in}%
\pgfsys@useobject{currentmarker}{}%
\end{pgfscope}%
\begin{pgfscope}%
\pgfsys@transformshift{1.058719in}{0.923714in}%
\pgfsys@useobject{currentmarker}{}%
\end{pgfscope}%
\begin{pgfscope}%
\pgfsys@transformshift{1.161242in}{1.040594in}%
\pgfsys@useobject{currentmarker}{}%
\end{pgfscope}%
\begin{pgfscope}%
\pgfsys@transformshift{0.759507in}{0.728370in}%
\pgfsys@useobject{currentmarker}{}%
\end{pgfscope}%
\begin{pgfscope}%
\pgfsys@transformshift{0.642581in}{0.912365in}%
\pgfsys@useobject{currentmarker}{}%
\end{pgfscope}%
\begin{pgfscope}%
\pgfsys@transformshift{0.977653in}{0.884785in}%
\pgfsys@useobject{currentmarker}{}%
\end{pgfscope}%
\begin{pgfscope}%
\pgfsys@transformshift{0.927999in}{0.851081in}%
\pgfsys@useobject{currentmarker}{}%
\end{pgfscope}%
\begin{pgfscope}%
\pgfsys@transformshift{1.027377in}{1.554232in}%
\pgfsys@useobject{currentmarker}{}%
\end{pgfscope}%
\begin{pgfscope}%
\pgfsys@transformshift{0.870669in}{0.876447in}%
\pgfsys@useobject{currentmarker}{}%
\end{pgfscope}%
\begin{pgfscope}%
\pgfsys@transformshift{1.131274in}{0.925924in}%
\pgfsys@useobject{currentmarker}{}%
\end{pgfscope}%
\begin{pgfscope}%
\pgfsys@transformshift{1.042078in}{0.753168in}%
\pgfsys@useobject{currentmarker}{}%
\end{pgfscope}%
\begin{pgfscope}%
\pgfsys@transformshift{0.831325in}{0.847290in}%
\pgfsys@useobject{currentmarker}{}%
\end{pgfscope}%
\begin{pgfscope}%
\pgfsys@transformshift{1.256570in}{0.989348in}%
\pgfsys@useobject{currentmarker}{}%
\end{pgfscope}%
\begin{pgfscope}%
\pgfsys@transformshift{0.638007in}{0.891566in}%
\pgfsys@useobject{currentmarker}{}%
\end{pgfscope}%
\begin{pgfscope}%
\pgfsys@transformshift{0.735105in}{1.058772in}%
\pgfsys@useobject{currentmarker}{}%
\end{pgfscope}%
\begin{pgfscope}%
\pgfsys@transformshift{1.160930in}{0.958284in}%
\pgfsys@useobject{currentmarker}{}%
\end{pgfscope}%
\begin{pgfscope}%
\pgfsys@transformshift{0.777394in}{0.998586in}%
\pgfsys@useobject{currentmarker}{}%
\end{pgfscope}%
\begin{pgfscope}%
\pgfsys@transformshift{0.790296in}{1.046838in}%
\pgfsys@useobject{currentmarker}{}%
\end{pgfscope}%
\begin{pgfscope}%
\pgfsys@transformshift{1.388607in}{1.167456in}%
\pgfsys@useobject{currentmarker}{}%
\end{pgfscope}%
\begin{pgfscope}%
\pgfsys@transformshift{0.801216in}{0.986062in}%
\pgfsys@useobject{currentmarker}{}%
\end{pgfscope}%
\begin{pgfscope}%
\pgfsys@transformshift{0.675325in}{0.890854in}%
\pgfsys@useobject{currentmarker}{}%
\end{pgfscope}%
\begin{pgfscope}%
\pgfsys@transformshift{1.011147in}{0.963866in}%
\pgfsys@useobject{currentmarker}{}%
\end{pgfscope}%
\begin{pgfscope}%
\pgfsys@transformshift{0.980060in}{0.928477in}%
\pgfsys@useobject{currentmarker}{}%
\end{pgfscope}%
\begin{pgfscope}%
\pgfsys@transformshift{0.722189in}{0.802990in}%
\pgfsys@useobject{currentmarker}{}%
\end{pgfscope}%
\begin{pgfscope}%
\pgfsys@transformshift{0.979409in}{0.745433in}%
\pgfsys@useobject{currentmarker}{}%
\end{pgfscope}%
\begin{pgfscope}%
\pgfsys@transformshift{0.616890in}{0.913773in}%
\pgfsys@useobject{currentmarker}{}%
\end{pgfscope}%
\begin{pgfscope}%
\pgfsys@transformshift{0.660454in}{0.887354in}%
\pgfsys@useobject{currentmarker}{}%
\end{pgfscope}%
\begin{pgfscope}%
\pgfsys@transformshift{0.796910in}{0.965797in}%
\pgfsys@useobject{currentmarker}{}%
\end{pgfscope}%
\begin{pgfscope}%
\pgfsys@transformshift{0.962343in}{0.725120in}%
\pgfsys@useobject{currentmarker}{}%
\end{pgfscope}%
\begin{pgfscope}%
\pgfsys@transformshift{1.496441in}{1.287599in}%
\pgfsys@useobject{currentmarker}{}%
\end{pgfscope}%
\begin{pgfscope}%
\pgfsys@transformshift{0.787577in}{1.176149in}%
\pgfsys@useobject{currentmarker}{}%
\end{pgfscope}%
\begin{pgfscope}%
\pgfsys@transformshift{1.229604in}{1.047859in}%
\pgfsys@useobject{currentmarker}{}%
\end{pgfscope}%
\begin{pgfscope}%
\pgfsys@transformshift{0.905565in}{0.868028in}%
\pgfsys@useobject{currentmarker}{}%
\end{pgfscope}%
\begin{pgfscope}%
\pgfsys@transformshift{0.807560in}{1.153010in}%
\pgfsys@useobject{currentmarker}{}%
\end{pgfscope}%
\begin{pgfscope}%
\pgfsys@transformshift{0.867369in}{1.278462in}%
\pgfsys@useobject{currentmarker}{}%
\end{pgfscope}%
\begin{pgfscope}%
\pgfsys@transformshift{1.056722in}{1.501436in}%
\pgfsys@useobject{currentmarker}{}%
\end{pgfscope}%
\begin{pgfscope}%
\pgfsys@transformshift{1.356231in}{1.053512in}%
\pgfsys@useobject{currentmarker}{}%
\end{pgfscope}%
\begin{pgfscope}%
\pgfsys@transformshift{1.339109in}{1.037804in}%
\pgfsys@useobject{currentmarker}{}%
\end{pgfscope}%
\begin{pgfscope}%
\pgfsys@transformshift{1.033014in}{0.746243in}%
\pgfsys@useobject{currentmarker}{}%
\end{pgfscope}%
\begin{pgfscope}%
\pgfsys@transformshift{0.723860in}{1.027849in}%
\pgfsys@useobject{currentmarker}{}%
\end{pgfscope}%
\begin{pgfscope}%
\pgfsys@transformshift{0.980683in}{1.244142in}%
\pgfsys@useobject{currentmarker}{}%
\end{pgfscope}%
\begin{pgfscope}%
\pgfsys@transformshift{0.908185in}{0.899320in}%
\pgfsys@useobject{currentmarker}{}%
\end{pgfscope}%
\begin{pgfscope}%
\pgfsys@transformshift{0.896784in}{0.982195in}%
\pgfsys@useobject{currentmarker}{}%
\end{pgfscope}%
\begin{pgfscope}%
\pgfsys@transformshift{0.651857in}{0.869880in}%
\pgfsys@useobject{currentmarker}{}%
\end{pgfscope}%
\begin{pgfscope}%
\pgfsys@transformshift{1.010736in}{0.728747in}%
\pgfsys@useobject{currentmarker}{}%
\end{pgfscope}%
\begin{pgfscope}%
\pgfsys@transformshift{1.033340in}{1.631897in}%
\pgfsys@useobject{currentmarker}{}%
\end{pgfscope}%
\begin{pgfscope}%
\pgfsys@transformshift{0.854566in}{0.952557in}%
\pgfsys@useobject{currentmarker}{}%
\end{pgfscope}%
\begin{pgfscope}%
\pgfsys@transformshift{0.631917in}{0.650238in}%
\pgfsys@useobject{currentmarker}{}%
\end{pgfscope}%
\begin{pgfscope}%
\pgfsys@transformshift{1.083872in}{1.434216in}%
\pgfsys@useobject{currentmarker}{}%
\end{pgfscope}%
\begin{pgfscope}%
\pgfsys@transformshift{1.184440in}{1.117845in}%
\pgfsys@useobject{currentmarker}{}%
\end{pgfscope}%
\begin{pgfscope}%
\pgfsys@transformshift{1.253199in}{1.007554in}%
\pgfsys@useobject{currentmarker}{}%
\end{pgfscope}%
\begin{pgfscope}%
\pgfsys@transformshift{1.254289in}{0.956104in}%
\pgfsys@useobject{currentmarker}{}%
\end{pgfscope}%
\begin{pgfscope}%
\pgfsys@transformshift{0.869847in}{1.404775in}%
\pgfsys@useobject{currentmarker}{}%
\end{pgfscope}%
\begin{pgfscope}%
\pgfsys@transformshift{1.013824in}{0.812200in}%
\pgfsys@useobject{currentmarker}{}%
\end{pgfscope}%
\begin{pgfscope}%
\pgfsys@transformshift{0.719087in}{0.676521in}%
\pgfsys@useobject{currentmarker}{}%
\end{pgfscope}%
\begin{pgfscope}%
\pgfsys@transformshift{1.069440in}{1.267545in}%
\pgfsys@useobject{currentmarker}{}%
\end{pgfscope}%
\begin{pgfscope}%
\pgfsys@transformshift{0.999024in}{0.726808in}%
\pgfsys@useobject{currentmarker}{}%
\end{pgfscope}%
\begin{pgfscope}%
\pgfsys@transformshift{0.760612in}{0.748475in}%
\pgfsys@useobject{currentmarker}{}%
\end{pgfscope}%
\begin{pgfscope}%
\pgfsys@transformshift{0.914997in}{0.627846in}%
\pgfsys@useobject{currentmarker}{}%
\end{pgfscope}%
\begin{pgfscope}%
\pgfsys@transformshift{0.752426in}{1.308029in}%
\pgfsys@useobject{currentmarker}{}%
\end{pgfscope}%
\begin{pgfscope}%
\pgfsys@transformshift{0.994860in}{1.525336in}%
\pgfsys@useobject{currentmarker}{}%
\end{pgfscope}%
\begin{pgfscope}%
\pgfsys@transformshift{0.990569in}{1.191253in}%
\pgfsys@useobject{currentmarker}{}%
\end{pgfscope}%
\begin{pgfscope}%
\pgfsys@transformshift{1.027349in}{1.638407in}%
\pgfsys@useobject{currentmarker}{}%
\end{pgfscope}%
\begin{pgfscope}%
\pgfsys@transformshift{1.080897in}{1.295178in}%
\pgfsys@useobject{currentmarker}{}%
\end{pgfscope}%
\begin{pgfscope}%
\pgfsys@transformshift{0.819514in}{0.781821in}%
\pgfsys@useobject{currentmarker}{}%
\end{pgfscope}%
\begin{pgfscope}%
\pgfsys@transformshift{1.441872in}{1.202347in}%
\pgfsys@useobject{currentmarker}{}%
\end{pgfscope}%
\begin{pgfscope}%
\pgfsys@transformshift{1.857118in}{1.251939in}%
\pgfsys@useobject{currentmarker}{}%
\end{pgfscope}%
\begin{pgfscope}%
\pgfsys@transformshift{0.726098in}{0.817097in}%
\pgfsys@useobject{currentmarker}{}%
\end{pgfscope}%
\begin{pgfscope}%
\pgfsys@transformshift{1.093828in}{1.495955in}%
\pgfsys@useobject{currentmarker}{}%
\end{pgfscope}%
\begin{pgfscope}%
\pgfsys@transformshift{0.682902in}{0.941387in}%
\pgfsys@useobject{currentmarker}{}%
\end{pgfscope}%
\begin{pgfscope}%
\pgfsys@transformshift{0.880073in}{0.990386in}%
\pgfsys@useobject{currentmarker}{}%
\end{pgfscope}%
\begin{pgfscope}%
\pgfsys@transformshift{0.802717in}{1.172630in}%
\pgfsys@useobject{currentmarker}{}%
\end{pgfscope}%
\begin{pgfscope}%
\pgfsys@transformshift{0.886956in}{0.976512in}%
\pgfsys@useobject{currentmarker}{}%
\end{pgfscope}%
\begin{pgfscope}%
\pgfsys@transformshift{1.083985in}{0.962055in}%
\pgfsys@useobject{currentmarker}{}%
\end{pgfscope}%
\begin{pgfscope}%
\pgfsys@transformshift{0.846720in}{0.907419in}%
\pgfsys@useobject{currentmarker}{}%
\end{pgfscope}%
\begin{pgfscope}%
\pgfsys@transformshift{1.353838in}{1.039761in}%
\pgfsys@useobject{currentmarker}{}%
\end{pgfscope}%
\begin{pgfscope}%
\pgfsys@transformshift{1.059073in}{1.389687in}%
\pgfsys@useobject{currentmarker}{}%
\end{pgfscope}%
\begin{pgfscope}%
\pgfsys@transformshift{0.795055in}{0.905540in}%
\pgfsys@useobject{currentmarker}{}%
\end{pgfscope}%
\begin{pgfscope}%
\pgfsys@transformshift{1.076153in}{1.408468in}%
\pgfsys@useobject{currentmarker}{}%
\end{pgfscope}%
\begin{pgfscope}%
\pgfsys@transformshift{0.707927in}{0.812397in}%
\pgfsys@useobject{currentmarker}{}%
\end{pgfscope}%
\begin{pgfscope}%
\pgfsys@transformshift{0.909517in}{0.863913in}%
\pgfsys@useobject{currentmarker}{}%
\end{pgfscope}%
\begin{pgfscope}%
\pgfsys@transformshift{1.081422in}{1.097662in}%
\pgfsys@useobject{currentmarker}{}%
\end{pgfscope}%
\begin{pgfscope}%
\pgfsys@transformshift{0.945702in}{1.374803in}%
\pgfsys@useobject{currentmarker}{}%
\end{pgfscope}%
\begin{pgfscope}%
\pgfsys@transformshift{0.881645in}{1.116670in}%
\pgfsys@useobject{currentmarker}{}%
\end{pgfscope}%
\begin{pgfscope}%
\pgfsys@transformshift{0.844171in}{0.868300in}%
\pgfsys@useobject{currentmarker}{}%
\end{pgfscope}%
\begin{pgfscope}%
\pgfsys@transformshift{0.837684in}{0.973533in}%
\pgfsys@useobject{currentmarker}{}%
\end{pgfscope}%
\begin{pgfscope}%
\pgfsys@transformshift{1.121487in}{0.980877in}%
\pgfsys@useobject{currentmarker}{}%
\end{pgfscope}%
\begin{pgfscope}%
\pgfsys@transformshift{1.140437in}{1.038375in}%
\pgfsys@useobject{currentmarker}{}%
\end{pgfscope}%
\begin{pgfscope}%
\pgfsys@transformshift{1.007111in}{0.979732in}%
\pgfsys@useobject{currentmarker}{}%
\end{pgfscope}%
\begin{pgfscope}%
\pgfsys@transformshift{1.052700in}{0.907148in}%
\pgfsys@useobject{currentmarker}{}%
\end{pgfscope}%
\begin{pgfscope}%
\pgfsys@transformshift{1.183576in}{1.520537in}%
\pgfsys@useobject{currentmarker}{}%
\end{pgfscope}%
\begin{pgfscope}%
\pgfsys@transformshift{0.637638in}{1.028642in}%
\pgfsys@useobject{currentmarker}{}%
\end{pgfscope}%
\begin{pgfscope}%
\pgfsys@transformshift{0.933239in}{1.450059in}%
\pgfsys@useobject{currentmarker}{}%
\end{pgfscope}%
\begin{pgfscope}%
\pgfsys@transformshift{1.016231in}{1.155650in}%
\pgfsys@useobject{currentmarker}{}%
\end{pgfscope}%
\begin{pgfscope}%
\pgfsys@transformshift{0.978771in}{0.720362in}%
\pgfsys@useobject{currentmarker}{}%
\end{pgfscope}%
\begin{pgfscope}%
\pgfsys@transformshift{0.943563in}{1.473197in}%
\pgfsys@useobject{currentmarker}{}%
\end{pgfscope}%
\begin{pgfscope}%
\pgfsys@transformshift{0.681854in}{0.991418in}%
\pgfsys@useobject{currentmarker}{}%
\end{pgfscope}%
\begin{pgfscope}%
\pgfsys@transformshift{0.920323in}{1.040234in}%
\pgfsys@useobject{currentmarker}{}%
\end{pgfscope}%
\begin{pgfscope}%
\pgfsys@transformshift{0.840205in}{1.001517in}%
\pgfsys@useobject{currentmarker}{}%
\end{pgfscope}%
\begin{pgfscope}%
\pgfsys@transformshift{0.745656in}{1.222624in}%
\pgfsys@useobject{currentmarker}{}%
\end{pgfscope}%
\begin{pgfscope}%
\pgfsys@transformshift{0.686244in}{0.864808in}%
\pgfsys@useobject{currentmarker}{}%
\end{pgfscope}%
\begin{pgfscope}%
\pgfsys@transformshift{0.844964in}{0.965876in}%
\pgfsys@useobject{currentmarker}{}%
\end{pgfscope}%
\begin{pgfscope}%
\pgfsys@transformshift{1.358866in}{1.067369in}%
\pgfsys@useobject{currentmarker}{}%
\end{pgfscope}%
\begin{pgfscope}%
\pgfsys@transformshift{0.946042in}{0.992280in}%
\pgfsys@useobject{currentmarker}{}%
\end{pgfscope}%
\begin{pgfscope}%
\pgfsys@transformshift{0.946707in}{0.748274in}%
\pgfsys@useobject{currentmarker}{}%
\end{pgfscope}%
\begin{pgfscope}%
\pgfsys@transformshift{0.866817in}{0.701956in}%
\pgfsys@useobject{currentmarker}{}%
\end{pgfscope}%
\begin{pgfscope}%
\pgfsys@transformshift{0.691102in}{0.921467in}%
\pgfsys@useobject{currentmarker}{}%
\end{pgfscope}%
\begin{pgfscope}%
\pgfsys@transformshift{0.732003in}{0.909618in}%
\pgfsys@useobject{currentmarker}{}%
\end{pgfscope}%
\begin{pgfscope}%
\pgfsys@transformshift{0.750939in}{1.020943in}%
\pgfsys@useobject{currentmarker}{}%
\end{pgfscope}%
\begin{pgfscope}%
\pgfsys@transformshift{1.289214in}{1.059440in}%
\pgfsys@useobject{currentmarker}{}%
\end{pgfscope}%
\begin{pgfscope}%
\pgfsys@transformshift{1.006006in}{0.750567in}%
\pgfsys@useobject{currentmarker}{}%
\end{pgfscope}%
\begin{pgfscope}%
\pgfsys@transformshift{0.870329in}{0.984724in}%
\pgfsys@useobject{currentmarker}{}%
\end{pgfscope}%
\begin{pgfscope}%
\pgfsys@transformshift{1.151045in}{1.384274in}%
\pgfsys@useobject{currentmarker}{}%
\end{pgfscope}%
\begin{pgfscope}%
\pgfsys@transformshift{1.082271in}{1.069507in}%
\pgfsys@useobject{currentmarker}{}%
\end{pgfscope}%
\begin{pgfscope}%
\pgfsys@transformshift{0.884038in}{0.827958in}%
\pgfsys@useobject{currentmarker}{}%
\end{pgfscope}%
\begin{pgfscope}%
\pgfsys@transformshift{0.936921in}{0.664459in}%
\pgfsys@useobject{currentmarker}{}%
\end{pgfscope}%
\begin{pgfscope}%
\pgfsys@transformshift{1.089650in}{1.390852in}%
\pgfsys@useobject{currentmarker}{}%
\end{pgfscope}%
\begin{pgfscope}%
\pgfsys@transformshift{0.742979in}{1.189993in}%
\pgfsys@useobject{currentmarker}{}%
\end{pgfscope}%
\begin{pgfscope}%
\pgfsys@transformshift{0.736337in}{0.765759in}%
\pgfsys@useobject{currentmarker}{}%
\end{pgfscope}%
\begin{pgfscope}%
\pgfsys@transformshift{1.073236in}{1.521096in}%
\pgfsys@useobject{currentmarker}{}%
\end{pgfscope}%
\begin{pgfscope}%
\pgfsys@transformshift{0.954539in}{1.133014in}%
\pgfsys@useobject{currentmarker}{}%
\end{pgfscope}%
\begin{pgfscope}%
\pgfsys@transformshift{1.309268in}{0.976630in}%
\pgfsys@useobject{currentmarker}{}%
\end{pgfscope}%
\begin{pgfscope}%
\pgfsys@transformshift{0.994534in}{0.768648in}%
\pgfsys@useobject{currentmarker}{}%
\end{pgfscope}%
\begin{pgfscope}%
\pgfsys@transformshift{0.660964in}{0.781634in}%
\pgfsys@useobject{currentmarker}{}%
\end{pgfscope}%
\begin{pgfscope}%
\pgfsys@transformshift{1.176580in}{0.927183in}%
\pgfsys@useobject{currentmarker}{}%
\end{pgfscope}%
\begin{pgfscope}%
\pgfsys@transformshift{0.737626in}{1.176330in}%
\pgfsys@useobject{currentmarker}{}%
\end{pgfscope}%
\begin{pgfscope}%
\pgfsys@transformshift{0.767126in}{1.272859in}%
\pgfsys@useobject{currentmarker}{}%
\end{pgfscope}%
\begin{pgfscope}%
\pgfsys@transformshift{0.661431in}{0.937641in}%
\pgfsys@useobject{currentmarker}{}%
\end{pgfscope}%
\begin{pgfscope}%
\pgfsys@transformshift{1.015991in}{0.968586in}%
\pgfsys@useobject{currentmarker}{}%
\end{pgfscope}%
\begin{pgfscope}%
\pgfsys@transformshift{0.788059in}{0.972973in}%
\pgfsys@useobject{currentmarker}{}%
\end{pgfscope}%
\begin{pgfscope}%
\pgfsys@transformshift{0.700492in}{0.973371in}%
\pgfsys@useobject{currentmarker}{}%
\end{pgfscope}%
\begin{pgfscope}%
\pgfsys@transformshift{0.762849in}{0.802697in}%
\pgfsys@useobject{currentmarker}{}%
\end{pgfscope}%
\begin{pgfscope}%
\pgfsys@transformshift{0.740458in}{0.770594in}%
\pgfsys@useobject{currentmarker}{}%
\end{pgfscope}%
\begin{pgfscope}%
\pgfsys@transformshift{1.014858in}{0.814920in}%
\pgfsys@useobject{currentmarker}{}%
\end{pgfscope}%
\begin{pgfscope}%
\pgfsys@transformshift{1.487320in}{1.265311in}%
\pgfsys@useobject{currentmarker}{}%
\end{pgfscope}%
\begin{pgfscope}%
\pgfsys@transformshift{0.723095in}{1.039154in}%
\pgfsys@useobject{currentmarker}{}%
\end{pgfscope}%
\begin{pgfscope}%
\pgfsys@transformshift{0.785184in}{0.974493in}%
\pgfsys@useobject{currentmarker}{}%
\end{pgfscope}%
\begin{pgfscope}%
\pgfsys@transformshift{1.007380in}{0.802434in}%
\pgfsys@useobject{currentmarker}{}%
\end{pgfscope}%
\begin{pgfscope}%
\pgfsys@transformshift{1.174724in}{0.876807in}%
\pgfsys@useobject{currentmarker}{}%
\end{pgfscope}%
\begin{pgfscope}%
\pgfsys@transformshift{0.843378in}{1.050607in}%
\pgfsys@useobject{currentmarker}{}%
\end{pgfscope}%
\begin{pgfscope}%
\pgfsys@transformshift{0.736295in}{1.218251in}%
\pgfsys@useobject{currentmarker}{}%
\end{pgfscope}%
\begin{pgfscope}%
\pgfsys@transformshift{1.214507in}{1.056895in}%
\pgfsys@useobject{currentmarker}{}%
\end{pgfscope}%
\begin{pgfscope}%
\pgfsys@transformshift{0.941368in}{0.980195in}%
\pgfsys@useobject{currentmarker}{}%
\end{pgfscope}%
\begin{pgfscope}%
\pgfsys@transformshift{0.951084in}{0.694374in}%
\pgfsys@useobject{currentmarker}{}%
\end{pgfscope}%
\begin{pgfscope}%
\pgfsys@transformshift{0.850841in}{1.120195in}%
\pgfsys@useobject{currentmarker}{}%
\end{pgfscope}%
\begin{pgfscope}%
\pgfsys@transformshift{0.784221in}{1.044811in}%
\pgfsys@useobject{currentmarker}{}%
\end{pgfscope}%
\begin{pgfscope}%
\pgfsys@transformshift{0.912023in}{0.921492in}%
\pgfsys@useobject{currentmarker}{}%
\end{pgfscope}%
\begin{pgfscope}%
\pgfsys@transformshift{0.759464in}{0.906921in}%
\pgfsys@useobject{currentmarker}{}%
\end{pgfscope}%
\begin{pgfscope}%
\pgfsys@transformshift{1.042418in}{0.984093in}%
\pgfsys@useobject{currentmarker}{}%
\end{pgfscope}%
\begin{pgfscope}%
\pgfsys@transformshift{0.724228in}{1.245315in}%
\pgfsys@useobject{currentmarker}{}%
\end{pgfscope}%
\begin{pgfscope}%
\pgfsys@transformshift{1.520715in}{1.085299in}%
\pgfsys@useobject{currentmarker}{}%
\end{pgfscope}%
\begin{pgfscope}%
\pgfsys@transformshift{0.729921in}{0.703259in}%
\pgfsys@useobject{currentmarker}{}%
\end{pgfscope}%
\begin{pgfscope}%
\pgfsys@transformshift{0.877920in}{0.637899in}%
\pgfsys@useobject{currentmarker}{}%
\end{pgfscope}%
\begin{pgfscope}%
\pgfsys@transformshift{0.816894in}{1.014977in}%
\pgfsys@useobject{currentmarker}{}%
\end{pgfscope}%
\begin{pgfscope}%
\pgfsys@transformshift{1.025125in}{0.981338in}%
\pgfsys@useobject{currentmarker}{}%
\end{pgfscope}%
\begin{pgfscope}%
\pgfsys@transformshift{0.829654in}{1.094096in}%
\pgfsys@useobject{currentmarker}{}%
\end{pgfscope}%
\begin{pgfscope}%
\pgfsys@transformshift{0.708154in}{0.831277in}%
\pgfsys@useobject{currentmarker}{}%
\end{pgfscope}%
\begin{pgfscope}%
\pgfsys@transformshift{1.216475in}{1.078308in}%
\pgfsys@useobject{currentmarker}{}%
\end{pgfscope}%
\begin{pgfscope}%
\pgfsys@transformshift{0.844652in}{1.243435in}%
\pgfsys@useobject{currentmarker}{}%
\end{pgfscope}%
\begin{pgfscope}%
\pgfsys@transformshift{1.091406in}{0.805774in}%
\pgfsys@useobject{currentmarker}{}%
\end{pgfscope}%
\begin{pgfscope}%
\pgfsys@transformshift{0.933848in}{0.523238in}%
\pgfsys@useobject{currentmarker}{}%
\end{pgfscope}%
\begin{pgfscope}%
\pgfsys@transformshift{0.767849in}{0.890186in}%
\pgfsys@useobject{currentmarker}{}%
\end{pgfscope}%
\begin{pgfscope}%
\pgfsys@transformshift{0.723690in}{1.087176in}%
\pgfsys@useobject{currentmarker}{}%
\end{pgfscope}%
\begin{pgfscope}%
\pgfsys@transformshift{0.639125in}{0.993735in}%
\pgfsys@useobject{currentmarker}{}%
\end{pgfscope}%
\begin{pgfscope}%
\pgfsys@transformshift{0.989252in}{1.212860in}%
\pgfsys@useobject{currentmarker}{}%
\end{pgfscope}%
\begin{pgfscope}%
\pgfsys@transformshift{1.235085in}{1.102638in}%
\pgfsys@useobject{currentmarker}{}%
\end{pgfscope}%
\begin{pgfscope}%
\pgfsys@transformshift{0.849708in}{0.828782in}%
\pgfsys@useobject{currentmarker}{}%
\end{pgfscope}%
\begin{pgfscope}%
\pgfsys@transformshift{0.884024in}{0.885434in}%
\pgfsys@useobject{currentmarker}{}%
\end{pgfscope}%
\begin{pgfscope}%
\pgfsys@transformshift{0.963023in}{1.069463in}%
\pgfsys@useobject{currentmarker}{}%
\end{pgfscope}%
\begin{pgfscope}%
\pgfsys@transformshift{1.015778in}{1.082133in}%
\pgfsys@useobject{currentmarker}{}%
\end{pgfscope}%
\begin{pgfscope}%
\pgfsys@transformshift{0.615233in}{0.866206in}%
\pgfsys@useobject{currentmarker}{}%
\end{pgfscope}%
\begin{pgfscope}%
\pgfsys@transformshift{0.696739in}{0.883406in}%
\pgfsys@useobject{currentmarker}{}%
\end{pgfscope}%
\begin{pgfscope}%
\pgfsys@transformshift{1.073321in}{1.213578in}%
\pgfsys@useobject{currentmarker}{}%
\end{pgfscope}%
\begin{pgfscope}%
\pgfsys@transformshift{0.676359in}{0.947590in}%
\pgfsys@useobject{currentmarker}{}%
\end{pgfscope}%
\begin{pgfscope}%
\pgfsys@transformshift{0.994846in}{1.121110in}%
\pgfsys@useobject{currentmarker}{}%
\end{pgfscope}%
\begin{pgfscope}%
\pgfsys@transformshift{0.932998in}{0.980722in}%
\pgfsys@useobject{currentmarker}{}%
\end{pgfscope}%
\begin{pgfscope}%
\pgfsys@transformshift{1.222650in}{0.940770in}%
\pgfsys@useobject{currentmarker}{}%
\end{pgfscope}%
\begin{pgfscope}%
\pgfsys@transformshift{0.622442in}{0.909603in}%
\pgfsys@useobject{currentmarker}{}%
\end{pgfscope}%
\begin{pgfscope}%
\pgfsys@transformshift{0.761730in}{0.851469in}%
\pgfsys@useobject{currentmarker}{}%
\end{pgfscope}%
\begin{pgfscope}%
\pgfsys@transformshift{0.680183in}{0.766691in}%
\pgfsys@useobject{currentmarker}{}%
\end{pgfscope}%
\begin{pgfscope}%
\pgfsys@transformshift{0.613732in}{1.238904in}%
\pgfsys@useobject{currentmarker}{}%
\end{pgfscope}%
\begin{pgfscope}%
\pgfsys@transformshift{1.325598in}{1.092375in}%
\pgfsys@useobject{currentmarker}{}%
\end{pgfscope}%
\begin{pgfscope}%
\pgfsys@transformshift{1.708029in}{1.328218in}%
\pgfsys@useobject{currentmarker}{}%
\end{pgfscope}%
\begin{pgfscope}%
\pgfsys@transformshift{0.981760in}{0.751163in}%
\pgfsys@useobject{currentmarker}{}%
\end{pgfscope}%
\begin{pgfscope}%
\pgfsys@transformshift{1.253907in}{1.011749in}%
\pgfsys@useobject{currentmarker}{}%
\end{pgfscope}%
\begin{pgfscope}%
\pgfsys@transformshift{0.697390in}{0.625943in}%
\pgfsys@useobject{currentmarker}{}%
\end{pgfscope}%
\begin{pgfscope}%
\pgfsys@transformshift{0.938380in}{0.672251in}%
\pgfsys@useobject{currentmarker}{}%
\end{pgfscope}%
\begin{pgfscope}%
\pgfsys@transformshift{0.740444in}{0.881534in}%
\pgfsys@useobject{currentmarker}{}%
\end{pgfscope}%
\begin{pgfscope}%
\pgfsys@transformshift{0.739099in}{0.855829in}%
\pgfsys@useobject{currentmarker}{}%
\end{pgfscope}%
\begin{pgfscope}%
\pgfsys@transformshift{1.065305in}{1.431965in}%
\pgfsys@useobject{currentmarker}{}%
\end{pgfscope}%
\begin{pgfscope}%
\pgfsys@transformshift{1.040973in}{0.834153in}%
\pgfsys@useobject{currentmarker}{}%
\end{pgfscope}%
\begin{pgfscope}%
\pgfsys@transformshift{0.652353in}{0.922578in}%
\pgfsys@useobject{currentmarker}{}%
\end{pgfscope}%
\begin{pgfscope}%
\pgfsys@transformshift{0.750174in}{0.905364in}%
\pgfsys@useobject{currentmarker}{}%
\end{pgfscope}%
\begin{pgfscope}%
\pgfsys@transformshift{0.668711in}{1.032255in}%
\pgfsys@useobject{currentmarker}{}%
\end{pgfscope}%
\begin{pgfscope}%
\pgfsys@transformshift{0.731239in}{0.821599in}%
\pgfsys@useobject{currentmarker}{}%
\end{pgfscope}%
\begin{pgfscope}%
\pgfsys@transformshift{0.661006in}{1.247241in}%
\pgfsys@useobject{currentmarker}{}%
\end{pgfscope}%
\begin{pgfscope}%
\pgfsys@transformshift{0.980584in}{1.301826in}%
\pgfsys@useobject{currentmarker}{}%
\end{pgfscope}%
\begin{pgfscope}%
\pgfsys@transformshift{0.949823in}{0.675067in}%
\pgfsys@useobject{currentmarker}{}%
\end{pgfscope}%
\begin{pgfscope}%
\pgfsys@transformshift{0.714980in}{1.165464in}%
\pgfsys@useobject{currentmarker}{}%
\end{pgfscope}%
\begin{pgfscope}%
\pgfsys@transformshift{1.301791in}{1.080686in}%
\pgfsys@useobject{currentmarker}{}%
\end{pgfscope}%
\begin{pgfscope}%
\pgfsys@transformshift{0.789305in}{1.109421in}%
\pgfsys@useobject{currentmarker}{}%
\end{pgfscope}%
\begin{pgfscope}%
\pgfsys@transformshift{1.149883in}{1.137107in}%
\pgfsys@useobject{currentmarker}{}%
\end{pgfscope}%
\begin{pgfscope}%
\pgfsys@transformshift{0.809657in}{1.025854in}%
\pgfsys@useobject{currentmarker}{}%
\end{pgfscope}%
\begin{pgfscope}%
\pgfsys@transformshift{0.673654in}{0.708319in}%
\pgfsys@useobject{currentmarker}{}%
\end{pgfscope}%
\begin{pgfscope}%
\pgfsys@transformshift{0.763515in}{1.069696in}%
\pgfsys@useobject{currentmarker}{}%
\end{pgfscope}%
\begin{pgfscope}%
\pgfsys@transformshift{1.143170in}{0.826951in}%
\pgfsys@useobject{currentmarker}{}%
\end{pgfscope}%
\begin{pgfscope}%
\pgfsys@transformshift{0.963108in}{1.049918in}%
\pgfsys@useobject{currentmarker}{}%
\end{pgfscope}%
\begin{pgfscope}%
\pgfsys@transformshift{0.734935in}{0.863064in}%
\pgfsys@useobject{currentmarker}{}%
\end{pgfscope}%
\begin{pgfscope}%
\pgfsys@transformshift{0.760937in}{0.964729in}%
\pgfsys@useobject{currentmarker}{}%
\end{pgfscope}%
\begin{pgfscope}%
\pgfsys@transformshift{0.615785in}{0.576306in}%
\pgfsys@useobject{currentmarker}{}%
\end{pgfscope}%
\begin{pgfscope}%
\pgfsys@transformshift{0.964807in}{1.353270in}%
\pgfsys@useobject{currentmarker}{}%
\end{pgfscope}%
\begin{pgfscope}%
\pgfsys@transformshift{0.752851in}{0.963801in}%
\pgfsys@useobject{currentmarker}{}%
\end{pgfscope}%
\begin{pgfscope}%
\pgfsys@transformshift{1.305841in}{0.991927in}%
\pgfsys@useobject{currentmarker}{}%
\end{pgfscope}%
\begin{pgfscope}%
\pgfsys@transformshift{0.673965in}{1.116485in}%
\pgfsys@useobject{currentmarker}{}%
\end{pgfscope}%
\begin{pgfscope}%
\pgfsys@transformshift{0.962909in}{0.689490in}%
\pgfsys@useobject{currentmarker}{}%
\end{pgfscope}%
\end{pgfscope}%
\begin{pgfscope}%
\pgfpathrectangle{\pgfqpoint{0.578368in}{0.466613in}}{\pgfqpoint{1.278750in}{1.245750in}}%
\pgfusepath{clip}%
\pgfsetbuttcap%
\pgfsetroundjoin%
\definecolor{currentfill}{rgb}{0.298039,0.447059,0.690196}%
\pgfsetfillcolor{currentfill}%
\pgfsetfillopacity{0.150000}%
\pgfsetlinewidth{1.003750pt}%
\definecolor{currentstroke}{rgb}{1.000000,1.000000,1.000000}%
\pgfsetstrokecolor{currentstroke}%
\pgfsetstrokeopacity{0.150000}%
\pgfsetdash{}{0pt}%
\pgfsys@defobject{currentmarker}{\pgfqpoint{0.578368in}{0.859530in}}{\pgfqpoint{1.857118in}{1.366607in}}{%
\pgfpathmoveto{\pgfqpoint{0.578368in}{0.909300in}}%
\pgfpathlineto{\pgfqpoint{0.578368in}{0.859530in}}%
\pgfpathlineto{\pgfqpoint{0.591285in}{0.864538in}}%
\pgfpathlineto{\pgfqpoint{0.604201in}{0.869547in}}%
\pgfpathlineto{\pgfqpoint{0.617118in}{0.874273in}}%
\pgfpathlineto{\pgfqpoint{0.630035in}{0.879152in}}%
\pgfpathlineto{\pgfqpoint{0.642951in}{0.883753in}}%
\pgfpathlineto{\pgfqpoint{0.655868in}{0.888415in}}%
\pgfpathlineto{\pgfqpoint{0.668785in}{0.893334in}}%
\pgfpathlineto{\pgfqpoint{0.681701in}{0.898155in}}%
\pgfpathlineto{\pgfqpoint{0.694618in}{0.902563in}}%
\pgfpathlineto{\pgfqpoint{0.707535in}{0.907279in}}%
\pgfpathlineto{\pgfqpoint{0.720451in}{0.911994in}}%
\pgfpathlineto{\pgfqpoint{0.733368in}{0.916689in}}%
\pgfpathlineto{\pgfqpoint{0.746285in}{0.921421in}}%
\pgfpathlineto{\pgfqpoint{0.759201in}{0.925975in}}%
\pgfpathlineto{\pgfqpoint{0.772118in}{0.930670in}}%
\pgfpathlineto{\pgfqpoint{0.785035in}{0.935157in}}%
\pgfpathlineto{\pgfqpoint{0.797951in}{0.939598in}}%
\pgfpathlineto{\pgfqpoint{0.810868in}{0.944054in}}%
\pgfpathlineto{\pgfqpoint{0.823785in}{0.948699in}}%
\pgfpathlineto{\pgfqpoint{0.836701in}{0.953376in}}%
\pgfpathlineto{\pgfqpoint{0.849618in}{0.957994in}}%
\pgfpathlineto{\pgfqpoint{0.862535in}{0.962088in}}%
\pgfpathlineto{\pgfqpoint{0.875451in}{0.966382in}}%
\pgfpathlineto{\pgfqpoint{0.888368in}{0.970888in}}%
\pgfpathlineto{\pgfqpoint{0.901285in}{0.975133in}}%
\pgfpathlineto{\pgfqpoint{0.914201in}{0.979374in}}%
\pgfpathlineto{\pgfqpoint{0.927118in}{0.983622in}}%
\pgfpathlineto{\pgfqpoint{0.940035in}{0.987975in}}%
\pgfpathlineto{\pgfqpoint{0.952951in}{0.992323in}}%
\pgfpathlineto{\pgfqpoint{0.965868in}{0.996325in}}%
\pgfpathlineto{\pgfqpoint{0.978785in}{1.000212in}}%
\pgfpathlineto{\pgfqpoint{0.991701in}{1.004177in}}%
\pgfpathlineto{\pgfqpoint{1.004618in}{1.008142in}}%
\pgfpathlineto{\pgfqpoint{1.017535in}{1.012427in}}%
\pgfpathlineto{\pgfqpoint{1.030451in}{1.016847in}}%
\pgfpathlineto{\pgfqpoint{1.043368in}{1.021276in}}%
\pgfpathlineto{\pgfqpoint{1.056285in}{1.025248in}}%
\pgfpathlineto{\pgfqpoint{1.069201in}{1.029221in}}%
\pgfpathlineto{\pgfqpoint{1.082118in}{1.033193in}}%
\pgfpathlineto{\pgfqpoint{1.095035in}{1.037161in}}%
\pgfpathlineto{\pgfqpoint{1.107951in}{1.041017in}}%
\pgfpathlineto{\pgfqpoint{1.120868in}{1.045092in}}%
\pgfpathlineto{\pgfqpoint{1.133785in}{1.048915in}}%
\pgfpathlineto{\pgfqpoint{1.146701in}{1.052877in}}%
\pgfpathlineto{\pgfqpoint{1.159618in}{1.056898in}}%
\pgfpathlineto{\pgfqpoint{1.172535in}{1.060919in}}%
\pgfpathlineto{\pgfqpoint{1.185451in}{1.064934in}}%
\pgfpathlineto{\pgfqpoint{1.198368in}{1.068998in}}%
\pgfpathlineto{\pgfqpoint{1.211285in}{1.072902in}}%
\pgfpathlineto{\pgfqpoint{1.224201in}{1.076943in}}%
\pgfpathlineto{\pgfqpoint{1.237118in}{1.080986in}}%
\pgfpathlineto{\pgfqpoint{1.250035in}{1.084982in}}%
\pgfpathlineto{\pgfqpoint{1.262951in}{1.088877in}}%
\pgfpathlineto{\pgfqpoint{1.275868in}{1.092705in}}%
\pgfpathlineto{\pgfqpoint{1.288785in}{1.096447in}}%
\pgfpathlineto{\pgfqpoint{1.301701in}{1.100513in}}%
\pgfpathlineto{\pgfqpoint{1.314618in}{1.104542in}}%
\pgfpathlineto{\pgfqpoint{1.327535in}{1.108497in}}%
\pgfpathlineto{\pgfqpoint{1.340451in}{1.112187in}}%
\pgfpathlineto{\pgfqpoint{1.353368in}{1.116314in}}%
\pgfpathlineto{\pgfqpoint{1.366285in}{1.120290in}}%
\pgfpathlineto{\pgfqpoint{1.379201in}{1.123811in}}%
\pgfpathlineto{\pgfqpoint{1.392118in}{1.127578in}}%
\pgfpathlineto{\pgfqpoint{1.405035in}{1.131108in}}%
\pgfpathlineto{\pgfqpoint{1.417951in}{1.134643in}}%
\pgfpathlineto{\pgfqpoint{1.430868in}{1.138181in}}%
\pgfpathlineto{\pgfqpoint{1.443785in}{1.141720in}}%
\pgfpathlineto{\pgfqpoint{1.456701in}{1.145258in}}%
\pgfpathlineto{\pgfqpoint{1.469618in}{1.148796in}}%
\pgfpathlineto{\pgfqpoint{1.482535in}{1.152335in}}%
\pgfpathlineto{\pgfqpoint{1.495451in}{1.155876in}}%
\pgfpathlineto{\pgfqpoint{1.508368in}{1.159539in}}%
\pgfpathlineto{\pgfqpoint{1.521285in}{1.163299in}}%
\pgfpathlineto{\pgfqpoint{1.534201in}{1.167177in}}%
\pgfpathlineto{\pgfqpoint{1.547118in}{1.170879in}}%
\pgfpathlineto{\pgfqpoint{1.560035in}{1.174427in}}%
\pgfpathlineto{\pgfqpoint{1.572951in}{1.177975in}}%
\pgfpathlineto{\pgfqpoint{1.585868in}{1.181540in}}%
\pgfpathlineto{\pgfqpoint{1.598785in}{1.185205in}}%
\pgfpathlineto{\pgfqpoint{1.611701in}{1.188874in}}%
\pgfpathlineto{\pgfqpoint{1.624618in}{1.192603in}}%
\pgfpathlineto{\pgfqpoint{1.637535in}{1.196346in}}%
\pgfpathlineto{\pgfqpoint{1.650451in}{1.200087in}}%
\pgfpathlineto{\pgfqpoint{1.663368in}{1.203773in}}%
\pgfpathlineto{\pgfqpoint{1.676285in}{1.207319in}}%
\pgfpathlineto{\pgfqpoint{1.689201in}{1.210873in}}%
\pgfpathlineto{\pgfqpoint{1.702118in}{1.214536in}}%
\pgfpathlineto{\pgfqpoint{1.715035in}{1.218199in}}%
\pgfpathlineto{\pgfqpoint{1.727951in}{1.221863in}}%
\pgfpathlineto{\pgfqpoint{1.740868in}{1.225526in}}%
\pgfpathlineto{\pgfqpoint{1.753785in}{1.229189in}}%
\pgfpathlineto{\pgfqpoint{1.766701in}{1.232853in}}%
\pgfpathlineto{\pgfqpoint{1.779618in}{1.236525in}}%
\pgfpathlineto{\pgfqpoint{1.792535in}{1.240198in}}%
\pgfpathlineto{\pgfqpoint{1.805451in}{1.243880in}}%
\pgfpathlineto{\pgfqpoint{1.818368in}{1.247578in}}%
\pgfpathlineto{\pgfqpoint{1.831285in}{1.251198in}}%
\pgfpathlineto{\pgfqpoint{1.844201in}{1.254867in}}%
\pgfpathlineto{\pgfqpoint{1.857118in}{1.258606in}}%
\pgfpathlineto{\pgfqpoint{1.857118in}{1.366607in}}%
\pgfpathlineto{\pgfqpoint{1.857118in}{1.366607in}}%
\pgfpathlineto{\pgfqpoint{1.844201in}{1.361479in}}%
\pgfpathlineto{\pgfqpoint{1.831285in}{1.356352in}}%
\pgfpathlineto{\pgfqpoint{1.818368in}{1.351224in}}%
\pgfpathlineto{\pgfqpoint{1.805451in}{1.346096in}}%
\pgfpathlineto{\pgfqpoint{1.792535in}{1.340968in}}%
\pgfpathlineto{\pgfqpoint{1.779618in}{1.335991in}}%
\pgfpathlineto{\pgfqpoint{1.766701in}{1.331116in}}%
\pgfpathlineto{\pgfqpoint{1.753785in}{1.326240in}}%
\pgfpathlineto{\pgfqpoint{1.740868in}{1.321365in}}%
\pgfpathlineto{\pgfqpoint{1.727951in}{1.316490in}}%
\pgfpathlineto{\pgfqpoint{1.715035in}{1.311614in}}%
\pgfpathlineto{\pgfqpoint{1.702118in}{1.306739in}}%
\pgfpathlineto{\pgfqpoint{1.689201in}{1.301863in}}%
\pgfpathlineto{\pgfqpoint{1.676285in}{1.296988in}}%
\pgfpathlineto{\pgfqpoint{1.663368in}{1.292112in}}%
\pgfpathlineto{\pgfqpoint{1.650451in}{1.287237in}}%
\pgfpathlineto{\pgfqpoint{1.637535in}{1.282274in}}%
\pgfpathlineto{\pgfqpoint{1.624618in}{1.277263in}}%
\pgfpathlineto{\pgfqpoint{1.611701in}{1.272246in}}%
\pgfpathlineto{\pgfqpoint{1.598785in}{1.267235in}}%
\pgfpathlineto{\pgfqpoint{1.585868in}{1.262224in}}%
\pgfpathlineto{\pgfqpoint{1.572951in}{1.257214in}}%
\pgfpathlineto{\pgfqpoint{1.560035in}{1.252370in}}%
\pgfpathlineto{\pgfqpoint{1.547118in}{1.247563in}}%
\pgfpathlineto{\pgfqpoint{1.534201in}{1.242755in}}%
\pgfpathlineto{\pgfqpoint{1.521285in}{1.237925in}}%
\pgfpathlineto{\pgfqpoint{1.508368in}{1.232915in}}%
\pgfpathlineto{\pgfqpoint{1.495451in}{1.228055in}}%
\pgfpathlineto{\pgfqpoint{1.482535in}{1.223395in}}%
\pgfpathlineto{\pgfqpoint{1.469618in}{1.218329in}}%
\pgfpathlineto{\pgfqpoint{1.456701in}{1.213261in}}%
\pgfpathlineto{\pgfqpoint{1.443785in}{1.208189in}}%
\pgfpathlineto{\pgfqpoint{1.430868in}{1.203117in}}%
\pgfpathlineto{\pgfqpoint{1.417951in}{1.197934in}}%
\pgfpathlineto{\pgfqpoint{1.405035in}{1.192790in}}%
\pgfpathlineto{\pgfqpoint{1.392118in}{1.187815in}}%
\pgfpathlineto{\pgfqpoint{1.379201in}{1.182837in}}%
\pgfpathlineto{\pgfqpoint{1.366285in}{1.177771in}}%
\pgfpathlineto{\pgfqpoint{1.353368in}{1.172732in}}%
\pgfpathlineto{\pgfqpoint{1.340451in}{1.168051in}}%
\pgfpathlineto{\pgfqpoint{1.327535in}{1.162971in}}%
\pgfpathlineto{\pgfqpoint{1.314618in}{1.157972in}}%
\pgfpathlineto{\pgfqpoint{1.301701in}{1.152972in}}%
\pgfpathlineto{\pgfqpoint{1.288785in}{1.148082in}}%
\pgfpathlineto{\pgfqpoint{1.275868in}{1.143445in}}%
\pgfpathlineto{\pgfqpoint{1.262951in}{1.138694in}}%
\pgfpathlineto{\pgfqpoint{1.250035in}{1.134153in}}%
\pgfpathlineto{\pgfqpoint{1.237118in}{1.129358in}}%
\pgfpathlineto{\pgfqpoint{1.224201in}{1.124475in}}%
\pgfpathlineto{\pgfqpoint{1.211285in}{1.119249in}}%
\pgfpathlineto{\pgfqpoint{1.198368in}{1.114060in}}%
\pgfpathlineto{\pgfqpoint{1.185451in}{1.109626in}}%
\pgfpathlineto{\pgfqpoint{1.172535in}{1.104433in}}%
\pgfpathlineto{\pgfqpoint{1.159618in}{1.099831in}}%
\pgfpathlineto{\pgfqpoint{1.146701in}{1.095270in}}%
\pgfpathlineto{\pgfqpoint{1.133785in}{1.090807in}}%
\pgfpathlineto{\pgfqpoint{1.120868in}{1.086209in}}%
\pgfpathlineto{\pgfqpoint{1.107951in}{1.081333in}}%
\pgfpathlineto{\pgfqpoint{1.095035in}{1.076679in}}%
\pgfpathlineto{\pgfqpoint{1.082118in}{1.071792in}}%
\pgfpathlineto{\pgfqpoint{1.069201in}{1.067001in}}%
\pgfpathlineto{\pgfqpoint{1.056285in}{1.062278in}}%
\pgfpathlineto{\pgfqpoint{1.043368in}{1.057820in}}%
\pgfpathlineto{\pgfqpoint{1.030451in}{1.053266in}}%
\pgfpathlineto{\pgfqpoint{1.017535in}{1.048514in}}%
\pgfpathlineto{\pgfqpoint{1.004618in}{1.043792in}}%
\pgfpathlineto{\pgfqpoint{0.991701in}{1.039221in}}%
\pgfpathlineto{\pgfqpoint{0.978785in}{1.034845in}}%
\pgfpathlineto{\pgfqpoint{0.965868in}{1.030421in}}%
\pgfpathlineto{\pgfqpoint{0.952951in}{1.026189in}}%
\pgfpathlineto{\pgfqpoint{0.940035in}{1.021704in}}%
\pgfpathlineto{\pgfqpoint{0.927118in}{1.017523in}}%
\pgfpathlineto{\pgfqpoint{0.914201in}{1.013235in}}%
\pgfpathlineto{\pgfqpoint{0.901285in}{1.009018in}}%
\pgfpathlineto{\pgfqpoint{0.888368in}{1.004698in}}%
\pgfpathlineto{\pgfqpoint{0.875451in}{1.000338in}}%
\pgfpathlineto{\pgfqpoint{0.862535in}{0.996026in}}%
\pgfpathlineto{\pgfqpoint{0.849618in}{0.991909in}}%
\pgfpathlineto{\pgfqpoint{0.836701in}{0.987876in}}%
\pgfpathlineto{\pgfqpoint{0.823785in}{0.984210in}}%
\pgfpathlineto{\pgfqpoint{0.810868in}{0.980015in}}%
\pgfpathlineto{\pgfqpoint{0.797951in}{0.975981in}}%
\pgfpathlineto{\pgfqpoint{0.785035in}{0.972342in}}%
\pgfpathlineto{\pgfqpoint{0.772118in}{0.968193in}}%
\pgfpathlineto{\pgfqpoint{0.759201in}{0.964275in}}%
\pgfpathlineto{\pgfqpoint{0.746285in}{0.960244in}}%
\pgfpathlineto{\pgfqpoint{0.733368in}{0.956209in}}%
\pgfpathlineto{\pgfqpoint{0.720451in}{0.952331in}}%
\pgfpathlineto{\pgfqpoint{0.707535in}{0.948708in}}%
\pgfpathlineto{\pgfqpoint{0.694618in}{0.944891in}}%
\pgfpathlineto{\pgfqpoint{0.681701in}{0.940856in}}%
\pgfpathlineto{\pgfqpoint{0.668785in}{0.936800in}}%
\pgfpathlineto{\pgfqpoint{0.655868in}{0.932954in}}%
\pgfpathlineto{\pgfqpoint{0.642951in}{0.929151in}}%
\pgfpathlineto{\pgfqpoint{0.630035in}{0.925073in}}%
\pgfpathlineto{\pgfqpoint{0.617118in}{0.921270in}}%
\pgfpathlineto{\pgfqpoint{0.604201in}{0.916982in}}%
\pgfpathlineto{\pgfqpoint{0.591285in}{0.912977in}}%
\pgfpathlineto{\pgfqpoint{0.578368in}{0.909300in}}%
\pgfpathclose%
\pgfusepath{stroke,fill}%
}%
\begin{pgfscope}%
\pgfsys@transformshift{0.000000in}{0.000000in}%
\pgfsys@useobject{currentmarker}{}%
\end{pgfscope}%
\end{pgfscope}%
\begin{pgfscope}%
\pgfpathrectangle{\pgfqpoint{0.578368in}{0.466613in}}{\pgfqpoint{1.278750in}{1.245750in}}%
\pgfusepath{clip}%
\pgfsetroundcap%
\pgfsetroundjoin%
\pgfsetlinewidth{1.505625pt}%
\definecolor{currentstroke}{rgb}{0.298039,0.447059,0.690196}%
\pgfsetstrokecolor{currentstroke}%
\pgfsetdash{}{0pt}%
\pgfpathmoveto{\pgfqpoint{0.578368in}{0.885026in}}%
\pgfpathlineto{\pgfqpoint{0.591285in}{0.889319in}}%
\pgfpathlineto{\pgfqpoint{0.604201in}{0.893611in}}%
\pgfpathlineto{\pgfqpoint{0.617118in}{0.897904in}}%
\pgfpathlineto{\pgfqpoint{0.630035in}{0.902196in}}%
\pgfpathlineto{\pgfqpoint{0.642951in}{0.906489in}}%
\pgfpathlineto{\pgfqpoint{0.655868in}{0.910782in}}%
\pgfpathlineto{\pgfqpoint{0.668785in}{0.915074in}}%
\pgfpathlineto{\pgfqpoint{0.681701in}{0.919367in}}%
\pgfpathlineto{\pgfqpoint{0.694618in}{0.923659in}}%
\pgfpathlineto{\pgfqpoint{0.707535in}{0.927952in}}%
\pgfpathlineto{\pgfqpoint{0.720451in}{0.932244in}}%
\pgfpathlineto{\pgfqpoint{0.733368in}{0.936537in}}%
\pgfpathlineto{\pgfqpoint{0.746285in}{0.940830in}}%
\pgfpathlineto{\pgfqpoint{0.759201in}{0.945122in}}%
\pgfpathlineto{\pgfqpoint{0.772118in}{0.949415in}}%
\pgfpathlineto{\pgfqpoint{0.785035in}{0.953707in}}%
\pgfpathlineto{\pgfqpoint{0.797951in}{0.958000in}}%
\pgfpathlineto{\pgfqpoint{0.810868in}{0.962293in}}%
\pgfpathlineto{\pgfqpoint{0.823785in}{0.966585in}}%
\pgfpathlineto{\pgfqpoint{0.836701in}{0.970878in}}%
\pgfpathlineto{\pgfqpoint{0.849618in}{0.975170in}}%
\pgfpathlineto{\pgfqpoint{0.862535in}{0.979463in}}%
\pgfpathlineto{\pgfqpoint{0.875451in}{0.983755in}}%
\pgfpathlineto{\pgfqpoint{0.888368in}{0.988048in}}%
\pgfpathlineto{\pgfqpoint{0.901285in}{0.992341in}}%
\pgfpathlineto{\pgfqpoint{0.914201in}{0.996633in}}%
\pgfpathlineto{\pgfqpoint{0.927118in}{1.000926in}}%
\pgfpathlineto{\pgfqpoint{0.940035in}{1.005218in}}%
\pgfpathlineto{\pgfqpoint{0.952951in}{1.009511in}}%
\pgfpathlineto{\pgfqpoint{0.965868in}{1.013804in}}%
\pgfpathlineto{\pgfqpoint{0.978785in}{1.018096in}}%
\pgfpathlineto{\pgfqpoint{0.991701in}{1.022389in}}%
\pgfpathlineto{\pgfqpoint{1.004618in}{1.026681in}}%
\pgfpathlineto{\pgfqpoint{1.017535in}{1.030974in}}%
\pgfpathlineto{\pgfqpoint{1.030451in}{1.035266in}}%
\pgfpathlineto{\pgfqpoint{1.043368in}{1.039559in}}%
\pgfpathlineto{\pgfqpoint{1.056285in}{1.043852in}}%
\pgfpathlineto{\pgfqpoint{1.069201in}{1.048144in}}%
\pgfpathlineto{\pgfqpoint{1.082118in}{1.052437in}}%
\pgfpathlineto{\pgfqpoint{1.095035in}{1.056729in}}%
\pgfpathlineto{\pgfqpoint{1.107951in}{1.061022in}}%
\pgfpathlineto{\pgfqpoint{1.120868in}{1.065315in}}%
\pgfpathlineto{\pgfqpoint{1.133785in}{1.069607in}}%
\pgfpathlineto{\pgfqpoint{1.146701in}{1.073900in}}%
\pgfpathlineto{\pgfqpoint{1.159618in}{1.078192in}}%
\pgfpathlineto{\pgfqpoint{1.172535in}{1.082485in}}%
\pgfpathlineto{\pgfqpoint{1.185451in}{1.086777in}}%
\pgfpathlineto{\pgfqpoint{1.198368in}{1.091070in}}%
\pgfpathlineto{\pgfqpoint{1.211285in}{1.095363in}}%
\pgfpathlineto{\pgfqpoint{1.224201in}{1.099655in}}%
\pgfpathlineto{\pgfqpoint{1.237118in}{1.103948in}}%
\pgfpathlineto{\pgfqpoint{1.250035in}{1.108240in}}%
\pgfpathlineto{\pgfqpoint{1.262951in}{1.112533in}}%
\pgfpathlineto{\pgfqpoint{1.275868in}{1.116826in}}%
\pgfpathlineto{\pgfqpoint{1.288785in}{1.121118in}}%
\pgfpathlineto{\pgfqpoint{1.301701in}{1.125411in}}%
\pgfpathlineto{\pgfqpoint{1.314618in}{1.129703in}}%
\pgfpathlineto{\pgfqpoint{1.327535in}{1.133996in}}%
\pgfpathlineto{\pgfqpoint{1.340451in}{1.138288in}}%
\pgfpathlineto{\pgfqpoint{1.353368in}{1.142581in}}%
\pgfpathlineto{\pgfqpoint{1.366285in}{1.146874in}}%
\pgfpathlineto{\pgfqpoint{1.379201in}{1.151166in}}%
\pgfpathlineto{\pgfqpoint{1.392118in}{1.155459in}}%
\pgfpathlineto{\pgfqpoint{1.405035in}{1.159751in}}%
\pgfpathlineto{\pgfqpoint{1.417951in}{1.164044in}}%
\pgfpathlineto{\pgfqpoint{1.430868in}{1.168337in}}%
\pgfpathlineto{\pgfqpoint{1.443785in}{1.172629in}}%
\pgfpathlineto{\pgfqpoint{1.456701in}{1.176922in}}%
\pgfpathlineto{\pgfqpoint{1.469618in}{1.181214in}}%
\pgfpathlineto{\pgfqpoint{1.482535in}{1.185507in}}%
\pgfpathlineto{\pgfqpoint{1.495451in}{1.189799in}}%
\pgfpathlineto{\pgfqpoint{1.508368in}{1.194092in}}%
\pgfpathlineto{\pgfqpoint{1.521285in}{1.198385in}}%
\pgfpathlineto{\pgfqpoint{1.534201in}{1.202677in}}%
\pgfpathlineto{\pgfqpoint{1.547118in}{1.206970in}}%
\pgfpathlineto{\pgfqpoint{1.560035in}{1.211262in}}%
\pgfpathlineto{\pgfqpoint{1.572951in}{1.215555in}}%
\pgfpathlineto{\pgfqpoint{1.585868in}{1.219847in}}%
\pgfpathlineto{\pgfqpoint{1.598785in}{1.224140in}}%
\pgfpathlineto{\pgfqpoint{1.611701in}{1.228433in}}%
\pgfpathlineto{\pgfqpoint{1.624618in}{1.232725in}}%
\pgfpathlineto{\pgfqpoint{1.637535in}{1.237018in}}%
\pgfpathlineto{\pgfqpoint{1.650451in}{1.241310in}}%
\pgfpathlineto{\pgfqpoint{1.663368in}{1.245603in}}%
\pgfpathlineto{\pgfqpoint{1.676285in}{1.249896in}}%
\pgfpathlineto{\pgfqpoint{1.689201in}{1.254188in}}%
\pgfpathlineto{\pgfqpoint{1.702118in}{1.258481in}}%
\pgfpathlineto{\pgfqpoint{1.715035in}{1.262773in}}%
\pgfpathlineto{\pgfqpoint{1.727951in}{1.267066in}}%
\pgfpathlineto{\pgfqpoint{1.740868in}{1.271358in}}%
\pgfpathlineto{\pgfqpoint{1.753785in}{1.275651in}}%
\pgfpathlineto{\pgfqpoint{1.766701in}{1.279944in}}%
\pgfpathlineto{\pgfqpoint{1.779618in}{1.284236in}}%
\pgfpathlineto{\pgfqpoint{1.792535in}{1.288529in}}%
\pgfpathlineto{\pgfqpoint{1.805451in}{1.292821in}}%
\pgfpathlineto{\pgfqpoint{1.818368in}{1.297114in}}%
\pgfpathlineto{\pgfqpoint{1.831285in}{1.301407in}}%
\pgfpathlineto{\pgfqpoint{1.844201in}{1.305699in}}%
\pgfpathlineto{\pgfqpoint{1.857118in}{1.309992in}}%
\pgfusepath{stroke}%
\end{pgfscope}%
\begin{pgfscope}%
\pgfsetrectcap%
\pgfsetmiterjoin%
\pgfsetlinewidth{0.752812pt}%
\definecolor{currentstroke}{rgb}{0.700000,0.700000,0.700000}%
\pgfsetstrokecolor{currentstroke}%
\pgfsetdash{}{0pt}%
\pgfpathmoveto{\pgfqpoint{0.578368in}{0.466613in}}%
\pgfpathlineto{\pgfqpoint{0.578368in}{1.712363in}}%
\pgfusepath{stroke}%
\end{pgfscope}%
\begin{pgfscope}%
\pgfsetrectcap%
\pgfsetmiterjoin%
\pgfsetlinewidth{0.752812pt}%
\definecolor{currentstroke}{rgb}{0.700000,0.700000,0.700000}%
\pgfsetstrokecolor{currentstroke}%
\pgfsetdash{}{0pt}%
\pgfpathmoveto{\pgfqpoint{1.857118in}{0.466613in}}%
\pgfpathlineto{\pgfqpoint{1.857118in}{1.712363in}}%
\pgfusepath{stroke}%
\end{pgfscope}%
\begin{pgfscope}%
\pgfsetrectcap%
\pgfsetmiterjoin%
\pgfsetlinewidth{0.752812pt}%
\definecolor{currentstroke}{rgb}{0.700000,0.700000,0.700000}%
\pgfsetstrokecolor{currentstroke}%
\pgfsetdash{}{0pt}%
\pgfpathmoveto{\pgfqpoint{0.578368in}{0.466613in}}%
\pgfpathlineto{\pgfqpoint{1.857118in}{0.466613in}}%
\pgfusepath{stroke}%
\end{pgfscope}%
\begin{pgfscope}%
\pgfsetrectcap%
\pgfsetmiterjoin%
\pgfsetlinewidth{0.752812pt}%
\definecolor{currentstroke}{rgb}{0.700000,0.700000,0.700000}%
\pgfsetstrokecolor{currentstroke}%
\pgfsetdash{}{0pt}%
\pgfpathmoveto{\pgfqpoint{0.578368in}{1.712363in}}%
\pgfpathlineto{\pgfqpoint{1.857118in}{1.712363in}}%
\pgfusepath{stroke}%
\end{pgfscope}%
\end{pgfpicture}%
\makeatother%
\endgroup%
}}  \\
      \subfloat[\(\epsilon=0.1\)]{\resizebox{0.5\linewidth}{!}{%% Creator: Matplotlib, PGF backend
%%
%% To include the figure in your LaTeX document, write
%%   \input{<filename>.pgf}
%%
%% Make sure the required packages are loaded in your preamble
%%   \usepackage{pgf}
%%
%% and, on pdftex
%%   \usepackage[utf8]{inputenc}\DeclareUnicodeCharacter{2212}{-}
%%
%% or, on luatex and xetex
%%   \usepackage{unicode-math}
%%
%% Figures using additional raster images can only be included by \input if
%% they are in the same directory as the main LaTeX file. For loading figures
%% from other directories you can use the `import` package
%%   \usepackage{import}
%%
%% and then include the figures with
%%   \import{<path to file>}{<filename>.pgf}
%%
%% Matplotlib used the following preamble
%%   \usepackage[utf8]{inputenc}
%%   \usepackage[T1]{fontenc}
%%   \usepackage{amsmath}
%%   \newcommand*{\mat}[1]{\boldsymbol{#1}}
%%
\begingroup%
\makeatletter%
\begin{pgfpicture}%
\pgfpathrectangle{\pgfpointorigin}{\pgfqpoint{1.898089in}{1.842293in}}%
\pgfusepath{use as bounding box, clip}%
\begin{pgfscope}%
\pgfsetbuttcap%
\pgfsetmiterjoin%
\definecolor{currentfill}{rgb}{1.000000,1.000000,1.000000}%
\pgfsetfillcolor{currentfill}%
\pgfsetlinewidth{0.000000pt}%
\definecolor{currentstroke}{rgb}{1.000000,1.000000,1.000000}%
\pgfsetstrokecolor{currentstroke}%
\pgfsetstrokeopacity{0.000000}%
\pgfsetdash{}{0pt}%
\pgfpathmoveto{\pgfqpoint{0.000000in}{-0.000000in}}%
\pgfpathlineto{\pgfqpoint{1.898089in}{-0.000000in}}%
\pgfpathlineto{\pgfqpoint{1.898089in}{1.842293in}}%
\pgfpathlineto{\pgfqpoint{0.000000in}{1.842293in}}%
\pgfpathclose%
\pgfusepath{fill}%
\end{pgfscope}%
\begin{pgfscope}%
\pgfsetbuttcap%
\pgfsetmiterjoin%
\definecolor{currentfill}{rgb}{1.000000,1.000000,1.000000}%
\pgfsetfillcolor{currentfill}%
\pgfsetlinewidth{0.000000pt}%
\definecolor{currentstroke}{rgb}{0.000000,0.000000,0.000000}%
\pgfsetstrokecolor{currentstroke}%
\pgfsetstrokeopacity{0.000000}%
\pgfsetdash{}{0pt}%
\pgfpathmoveto{\pgfqpoint{0.519339in}{0.466613in}}%
\pgfpathlineto{\pgfqpoint{1.798089in}{0.466613in}}%
\pgfpathlineto{\pgfqpoint{1.798089in}{1.712363in}}%
\pgfpathlineto{\pgfqpoint{0.519339in}{1.712363in}}%
\pgfpathclose%
\pgfusepath{fill}%
\end{pgfscope}%
\begin{pgfscope}%
\pgfpathrectangle{\pgfqpoint{0.519339in}{0.466613in}}{\pgfqpoint{1.278750in}{1.245750in}}%
\pgfusepath{clip}%
\pgfsetroundcap%
\pgfsetroundjoin%
\pgfsetlinewidth{0.501875pt}%
\definecolor{currentstroke}{rgb}{0.800000,0.800000,0.800000}%
\pgfsetstrokecolor{currentstroke}%
\pgfsetdash{}{0pt}%
\pgfpathmoveto{\pgfqpoint{0.768713in}{0.466613in}}%
\pgfpathlineto{\pgfqpoint{0.768713in}{1.712363in}}%
\pgfusepath{stroke}%
\end{pgfscope}%
\begin{pgfscope}%
\definecolor{textcolor}{rgb}{0.150000,0.150000,0.150000}%
\pgfsetstrokecolor{textcolor}%
\pgfsetfillcolor{textcolor}%
\pgftext[x=0.768713in,y=0.376335in,,top]{\color{textcolor}\rmfamily\fontsize{8.000000}{9.600000}\selectfont \(\displaystyle {25000}\)}%
\end{pgfscope}%
\begin{pgfscope}%
\pgfpathrectangle{\pgfqpoint{0.519339in}{0.466613in}}{\pgfqpoint{1.278750in}{1.245750in}}%
\pgfusepath{clip}%
\pgfsetroundcap%
\pgfsetroundjoin%
\pgfsetlinewidth{0.501875pt}%
\definecolor{currentstroke}{rgb}{0.800000,0.800000,0.800000}%
\pgfsetstrokecolor{currentstroke}%
\pgfsetdash{}{0pt}%
\pgfpathmoveto{\pgfqpoint{1.122777in}{0.466613in}}%
\pgfpathlineto{\pgfqpoint{1.122777in}{1.712363in}}%
\pgfusepath{stroke}%
\end{pgfscope}%
\begin{pgfscope}%
\definecolor{textcolor}{rgb}{0.150000,0.150000,0.150000}%
\pgfsetstrokecolor{textcolor}%
\pgfsetfillcolor{textcolor}%
\pgftext[x=1.122777in,y=0.376335in,,top]{\color{textcolor}\rmfamily\fontsize{8.000000}{9.600000}\selectfont \(\displaystyle {50000}\)}%
\end{pgfscope}%
\begin{pgfscope}%
\pgfpathrectangle{\pgfqpoint{0.519339in}{0.466613in}}{\pgfqpoint{1.278750in}{1.245750in}}%
\pgfusepath{clip}%
\pgfsetroundcap%
\pgfsetroundjoin%
\pgfsetlinewidth{0.501875pt}%
\definecolor{currentstroke}{rgb}{0.800000,0.800000,0.800000}%
\pgfsetstrokecolor{currentstroke}%
\pgfsetdash{}{0pt}%
\pgfpathmoveto{\pgfqpoint{1.476841in}{0.466613in}}%
\pgfpathlineto{\pgfqpoint{1.476841in}{1.712363in}}%
\pgfusepath{stroke}%
\end{pgfscope}%
\begin{pgfscope}%
\definecolor{textcolor}{rgb}{0.150000,0.150000,0.150000}%
\pgfsetstrokecolor{textcolor}%
\pgfsetfillcolor{textcolor}%
\pgftext[x=1.476841in,y=0.376335in,,top]{\color{textcolor}\rmfamily\fontsize{8.000000}{9.600000}\selectfont \(\displaystyle {75000}\)}%
\end{pgfscope}%
\begin{pgfscope}%
\definecolor{textcolor}{rgb}{0.150000,0.150000,0.150000}%
\pgfsetstrokecolor{textcolor}%
\pgfsetfillcolor{textcolor}%
\pgftext[x=1.158714in,y=0.222655in,,top]{\color{textcolor}\rmfamily\fontsize{10.000000}{12.000000}\selectfont Number of nodes}%
\end{pgfscope}%
\begin{pgfscope}%
\pgfpathrectangle{\pgfqpoint{0.519339in}{0.466613in}}{\pgfqpoint{1.278750in}{1.245750in}}%
\pgfusepath{clip}%
\pgfsetroundcap%
\pgfsetroundjoin%
\pgfsetlinewidth{0.501875pt}%
\definecolor{currentstroke}{rgb}{0.800000,0.800000,0.800000}%
\pgfsetstrokecolor{currentstroke}%
\pgfsetdash{}{0pt}%
\pgfpathmoveto{\pgfqpoint{0.519339in}{0.750866in}}%
\pgfpathlineto{\pgfqpoint{1.798089in}{0.750866in}}%
\pgfusepath{stroke}%
\end{pgfscope}%
\begin{pgfscope}%
\definecolor{textcolor}{rgb}{0.150000,0.150000,0.150000}%
\pgfsetstrokecolor{textcolor}%
\pgfsetfillcolor{textcolor}%
\pgftext[x=0.278211in, y=0.712603in, left, base]{\color{textcolor}\rmfamily\fontsize{8.000000}{9.600000}\selectfont \(\displaystyle {0.2}\)}%
\end{pgfscope}%
\begin{pgfscope}%
\pgfpathrectangle{\pgfqpoint{0.519339in}{0.466613in}}{\pgfqpoint{1.278750in}{1.245750in}}%
\pgfusepath{clip}%
\pgfsetroundcap%
\pgfsetroundjoin%
\pgfsetlinewidth{0.501875pt}%
\definecolor{currentstroke}{rgb}{0.800000,0.800000,0.800000}%
\pgfsetstrokecolor{currentstroke}%
\pgfsetdash{}{0pt}%
\pgfpathmoveto{\pgfqpoint{0.519339in}{1.227448in}}%
\pgfpathlineto{\pgfqpoint{1.798089in}{1.227448in}}%
\pgfusepath{stroke}%
\end{pgfscope}%
\begin{pgfscope}%
\definecolor{textcolor}{rgb}{0.150000,0.150000,0.150000}%
\pgfsetstrokecolor{textcolor}%
\pgfsetfillcolor{textcolor}%
\pgftext[x=0.278211in, y=1.189186in, left, base]{\color{textcolor}\rmfamily\fontsize{8.000000}{9.600000}\selectfont \(\displaystyle {0.4}\)}%
\end{pgfscope}%
\begin{pgfscope}%
\pgfpathrectangle{\pgfqpoint{0.519339in}{0.466613in}}{\pgfqpoint{1.278750in}{1.245750in}}%
\pgfusepath{clip}%
\pgfsetroundcap%
\pgfsetroundjoin%
\pgfsetlinewidth{0.501875pt}%
\definecolor{currentstroke}{rgb}{0.800000,0.800000,0.800000}%
\pgfsetstrokecolor{currentstroke}%
\pgfsetdash{}{0pt}%
\pgfpathmoveto{\pgfqpoint{0.519339in}{1.704030in}}%
\pgfpathlineto{\pgfqpoint{1.798089in}{1.704030in}}%
\pgfusepath{stroke}%
\end{pgfscope}%
\begin{pgfscope}%
\definecolor{textcolor}{rgb}{0.150000,0.150000,0.150000}%
\pgfsetstrokecolor{textcolor}%
\pgfsetfillcolor{textcolor}%
\pgftext[x=0.278211in, y=1.665768in, left, base]{\color{textcolor}\rmfamily\fontsize{8.000000}{9.600000}\selectfont \(\displaystyle {0.6}\)}%
\end{pgfscope}%
\begin{pgfscope}%
\definecolor{textcolor}{rgb}{0.150000,0.150000,0.150000}%
\pgfsetstrokecolor{textcolor}%
\pgfsetfillcolor{textcolor}%
\pgftext[x=0.222655in,y=1.089488in,,bottom,rotate=90.000000]{\color{textcolor}\rmfamily\fontsize{10.000000}{12.000000}\selectfont Accuracy drop}%
\end{pgfscope}%
\begin{pgfscope}%
\pgfpathrectangle{\pgfqpoint{0.519339in}{0.466613in}}{\pgfqpoint{1.278750in}{1.245750in}}%
\pgfusepath{clip}%
\pgfsetbuttcap%
\pgfsetroundjoin%
\definecolor{currentfill}{rgb}{0.298039,0.447059,0.690196}%
\pgfsetfillcolor{currentfill}%
\pgfsetfillopacity{0.800000}%
\pgfsetlinewidth{1.003750pt}%
\definecolor{currentstroke}{rgb}{0.298039,0.447059,0.690196}%
\pgfsetstrokecolor{currentstroke}%
\pgfsetstrokeopacity{0.800000}%
\pgfsetdash{}{0pt}%
\pgfsys@defobject{currentmarker}{\pgfqpoint{-0.017010in}{-0.017010in}}{\pgfqpoint{0.017010in}{0.017010in}}{%
\pgfpathmoveto{\pgfqpoint{0.000000in}{-0.017010in}}%
\pgfpathcurveto{\pgfqpoint{0.004511in}{-0.017010in}}{\pgfqpoint{0.008838in}{-0.015218in}}{\pgfqpoint{0.012028in}{-0.012028in}}%
\pgfpathcurveto{\pgfqpoint{0.015218in}{-0.008838in}}{\pgfqpoint{0.017010in}{-0.004511in}}{\pgfqpoint{0.017010in}{0.000000in}}%
\pgfpathcurveto{\pgfqpoint{0.017010in}{0.004511in}}{\pgfqpoint{0.015218in}{0.008838in}}{\pgfqpoint{0.012028in}{0.012028in}}%
\pgfpathcurveto{\pgfqpoint{0.008838in}{0.015218in}}{\pgfqpoint{0.004511in}{0.017010in}}{\pgfqpoint{0.000000in}{0.017010in}}%
\pgfpathcurveto{\pgfqpoint{-0.004511in}{0.017010in}}{\pgfqpoint{-0.008838in}{0.015218in}}{\pgfqpoint{-0.012028in}{0.012028in}}%
\pgfpathcurveto{\pgfqpoint{-0.015218in}{0.008838in}}{\pgfqpoint{-0.017010in}{0.004511in}}{\pgfqpoint{-0.017010in}{0.000000in}}%
\pgfpathcurveto{\pgfqpoint{-0.017010in}{-0.004511in}}{\pgfqpoint{-0.015218in}{-0.008838in}}{\pgfqpoint{-0.012028in}{-0.012028in}}%
\pgfpathcurveto{\pgfqpoint{-0.008838in}{-0.015218in}}{\pgfqpoint{-0.004511in}{-0.017010in}}{\pgfqpoint{0.000000in}{-0.017010in}}%
\pgfpathclose%
\pgfusepath{stroke,fill}%
}%
\begin{pgfscope}%
\pgfsys@transformshift{1.014094in}{1.037333in}%
\pgfsys@useobject{currentmarker}{}%
\end{pgfscope}%
\begin{pgfscope}%
\pgfsys@transformshift{0.764422in}{1.130151in}%
\pgfsys@useobject{currentmarker}{}%
\end{pgfscope}%
\begin{pgfscope}%
\pgfsys@transformshift{0.887056in}{1.311706in}%
\pgfsys@useobject{currentmarker}{}%
\end{pgfscope}%
\begin{pgfscope}%
\pgfsys@transformshift{0.701456in}{1.407908in}%
\pgfsys@useobject{currentmarker}{}%
\end{pgfscope}%
\begin{pgfscope}%
\pgfsys@transformshift{0.774506in}{1.280723in}%
\pgfsys@useobject{currentmarker}{}%
\end{pgfscope}%
\begin{pgfscope}%
\pgfsys@transformshift{1.101193in}{1.072159in}%
\pgfsys@useobject{currentmarker}{}%
\end{pgfscope}%
\begin{pgfscope}%
\pgfsys@transformshift{0.628844in}{1.246074in}%
\pgfsys@useobject{currentmarker}{}%
\end{pgfscope}%
\begin{pgfscope}%
\pgfsys@transformshift{0.934939in}{0.958784in}%
\pgfsys@useobject{currentmarker}{}%
\end{pgfscope}%
\begin{pgfscope}%
\pgfsys@transformshift{1.292331in}{1.062874in}%
\pgfsys@useobject{currentmarker}{}%
\end{pgfscope}%
\begin{pgfscope}%
\pgfsys@transformshift{1.147788in}{1.144816in}%
\pgfsys@useobject{currentmarker}{}%
\end{pgfscope}%
\begin{pgfscope}%
\pgfsys@transformshift{1.046441in}{1.373160in}%
\pgfsys@useobject{currentmarker}{}%
\end{pgfscope}%
\begin{pgfscope}%
\pgfsys@transformshift{0.655342in}{1.271611in}%
\pgfsys@useobject{currentmarker}{}%
\end{pgfscope}%
\begin{pgfscope}%
\pgfsys@transformshift{0.791699in}{1.315857in}%
\pgfsys@useobject{currentmarker}{}%
\end{pgfscope}%
\begin{pgfscope}%
\pgfsys@transformshift{0.950532in}{1.183844in}%
\pgfsys@useobject{currentmarker}{}%
\end{pgfscope}%
\begin{pgfscope}%
\pgfsys@transformshift{0.834612in}{1.382833in}%
\pgfsys@useobject{currentmarker}{}%
\end{pgfscope}%
\begin{pgfscope}%
\pgfsys@transformshift{1.334280in}{1.303920in}%
\pgfsys@useobject{currentmarker}{}%
\end{pgfscope}%
\begin{pgfscope}%
\pgfsys@transformshift{0.627740in}{1.191139in}%
\pgfsys@useobject{currentmarker}{}%
\end{pgfscope}%
\begin{pgfscope}%
\pgfsys@transformshift{1.116673in}{1.098179in}%
\pgfsys@useobject{currentmarker}{}%
\end{pgfscope}%
\begin{pgfscope}%
\pgfsys@transformshift{0.838719in}{1.179115in}%
\pgfsys@useobject{currentmarker}{}%
\end{pgfscope}%
\begin{pgfscope}%
\pgfsys@transformshift{1.063011in}{1.064694in}%
\pgfsys@useobject{currentmarker}{}%
\end{pgfscope}%
\begin{pgfscope}%
\pgfsys@transformshift{0.700917in}{1.251913in}%
\pgfsys@useobject{currentmarker}{}%
\end{pgfscope}%
\begin{pgfscope}%
\pgfsys@transformshift{0.955687in}{1.301387in}%
\pgfsys@useobject{currentmarker}{}%
\end{pgfscope}%
\begin{pgfscope}%
\pgfsys@transformshift{0.591696in}{1.365099in}%
\pgfsys@useobject{currentmarker}{}%
\end{pgfscope}%
\begin{pgfscope}%
\pgfsys@transformshift{1.043410in}{1.044057in}%
\pgfsys@useobject{currentmarker}{}%
\end{pgfscope}%
\begin{pgfscope}%
\pgfsys@transformshift{0.590860in}{1.242829in}%
\pgfsys@useobject{currentmarker}{}%
\end{pgfscope}%
\begin{pgfscope}%
\pgfsys@transformshift{0.809686in}{1.470814in}%
\pgfsys@useobject{currentmarker}{}%
\end{pgfscope}%
\begin{pgfscope}%
\pgfsys@transformshift{0.628278in}{1.305971in}%
\pgfsys@useobject{currentmarker}{}%
\end{pgfscope}%
\begin{pgfscope}%
\pgfsys@transformshift{0.998118in}{1.532483in}%
\pgfsys@useobject{currentmarker}{}%
\end{pgfscope}%
\begin{pgfscope}%
\pgfsys@transformshift{0.756604in}{1.216256in}%
\pgfsys@useobject{currentmarker}{}%
\end{pgfscope}%
\begin{pgfscope}%
\pgfsys@transformshift{0.851026in}{1.152075in}%
\pgfsys@useobject{currentmarker}{}%
\end{pgfscope}%
\begin{pgfscope}%
\pgfsys@transformshift{1.033468in}{1.072609in}%
\pgfsys@useobject{currentmarker}{}%
\end{pgfscope}%
\begin{pgfscope}%
\pgfsys@transformshift{0.981336in}{0.973900in}%
\pgfsys@useobject{currentmarker}{}%
\end{pgfscope}%
\begin{pgfscope}%
\pgfsys@transformshift{1.043792in}{1.505785in}%
\pgfsys@useobject{currentmarker}{}%
\end{pgfscope}%
\begin{pgfscope}%
\pgfsys@transformshift{1.062685in}{1.017003in}%
\pgfsys@useobject{currentmarker}{}%
\end{pgfscope}%
\begin{pgfscope}%
\pgfsys@transformshift{0.777282in}{1.304136in}%
\pgfsys@useobject{currentmarker}{}%
\end{pgfscope}%
\begin{pgfscope}%
\pgfsys@transformshift{1.357479in}{1.309689in}%
\pgfsys@useobject{currentmarker}{}%
\end{pgfscope}%
\begin{pgfscope}%
\pgfsys@transformshift{0.664576in}{1.116423in}%
\pgfsys@useobject{currentmarker}{}%
\end{pgfscope}%
\begin{pgfscope}%
\pgfsys@transformshift{1.101547in}{1.096503in}%
\pgfsys@useobject{currentmarker}{}%
\end{pgfscope}%
\begin{pgfscope}%
\pgfsys@transformshift{0.960531in}{1.036939in}%
\pgfsys@useobject{currentmarker}{}%
\end{pgfscope}%
\begin{pgfscope}%
\pgfsys@transformshift{1.258723in}{1.289076in}%
\pgfsys@useobject{currentmarker}{}%
\end{pgfscope}%
\begin{pgfscope}%
\pgfsys@transformshift{1.104082in}{1.078942in}%
\pgfsys@useobject{currentmarker}{}%
\end{pgfscope}%
\begin{pgfscope}%
\pgfsys@transformshift{0.629397in}{1.227250in}%
\pgfsys@useobject{currentmarker}{}%
\end{pgfscope}%
\begin{pgfscope}%
\pgfsys@transformshift{0.934642in}{1.552371in}%
\pgfsys@useobject{currentmarker}{}%
\end{pgfscope}%
\begin{pgfscope}%
\pgfsys@transformshift{0.642993in}{1.234734in}%
\pgfsys@useobject{currentmarker}{}%
\end{pgfscope}%
\begin{pgfscope}%
\pgfsys@transformshift{0.943323in}{1.532031in}%
\pgfsys@useobject{currentmarker}{}%
\end{pgfscope}%
\begin{pgfscope}%
\pgfsys@transformshift{0.681798in}{1.380250in}%
\pgfsys@useobject{currentmarker}{}%
\end{pgfscope}%
\begin{pgfscope}%
\pgfsys@transformshift{0.614044in}{1.040477in}%
\pgfsys@useobject{currentmarker}{}%
\end{pgfscope}%
\begin{pgfscope}%
\pgfsys@transformshift{1.014108in}{1.079149in}%
\pgfsys@useobject{currentmarker}{}%
\end{pgfscope}%
\begin{pgfscope}%
\pgfsys@transformshift{1.147703in}{1.276289in}%
\pgfsys@useobject{currentmarker}{}%
\end{pgfscope}%
\begin{pgfscope}%
\pgfsys@transformshift{0.863121in}{0.911240in}%
\pgfsys@useobject{currentmarker}{}%
\end{pgfscope}%
\begin{pgfscope}%
\pgfsys@transformshift{0.919290in}{1.181640in}%
\pgfsys@useobject{currentmarker}{}%
\end{pgfscope}%
\begin{pgfscope}%
\pgfsys@transformshift{0.609087in}{1.081973in}%
\pgfsys@useobject{currentmarker}{}%
\end{pgfscope}%
\begin{pgfscope}%
\pgfsys@transformshift{0.633801in}{1.183849in}%
\pgfsys@useobject{currentmarker}{}%
\end{pgfscope}%
\begin{pgfscope}%
\pgfsys@transformshift{0.719739in}{1.133747in}%
\pgfsys@useobject{currentmarker}{}%
\end{pgfscope}%
\begin{pgfscope}%
\pgfsys@transformshift{0.673853in}{1.288717in}%
\pgfsys@useobject{currentmarker}{}%
\end{pgfscope}%
\begin{pgfscope}%
\pgfsys@transformshift{0.584246in}{1.124247in}%
\pgfsys@useobject{currentmarker}{}%
\end{pgfscope}%
\begin{pgfscope}%
\pgfsys@transformshift{0.833705in}{1.413902in}%
\pgfsys@useobject{currentmarker}{}%
\end{pgfscope}%
\begin{pgfscope}%
\pgfsys@transformshift{1.097808in}{1.190560in}%
\pgfsys@useobject{currentmarker}{}%
\end{pgfscope}%
\begin{pgfscope}%
\pgfsys@transformshift{0.743646in}{1.334477in}%
\pgfsys@useobject{currentmarker}{}%
\end{pgfscope}%
\begin{pgfscope}%
\pgfsys@transformshift{0.647822in}{1.288753in}%
\pgfsys@useobject{currentmarker}{}%
\end{pgfscope}%
\begin{pgfscope}%
\pgfsys@transformshift{0.688525in}{1.253723in}%
\pgfsys@useobject{currentmarker}{}%
\end{pgfscope}%
\begin{pgfscope}%
\pgfsys@transformshift{0.927674in}{0.806745in}%
\pgfsys@useobject{currentmarker}{}%
\end{pgfscope}%
\begin{pgfscope}%
\pgfsys@transformshift{0.539960in}{1.154856in}%
\pgfsys@useobject{currentmarker}{}%
\end{pgfscope}%
\begin{pgfscope}%
\pgfsys@transformshift{0.959129in}{1.563113in}%
\pgfsys@useobject{currentmarker}{}%
\end{pgfscope}%
\begin{pgfscope}%
\pgfsys@transformshift{0.615843in}{1.486091in}%
\pgfsys@useobject{currentmarker}{}%
\end{pgfscope}%
\begin{pgfscope}%
\pgfsys@transformshift{0.750515in}{1.451674in}%
\pgfsys@useobject{currentmarker}{}%
\end{pgfscope}%
\begin{pgfscope}%
\pgfsys@transformshift{0.904362in}{1.481770in}%
\pgfsys@useobject{currentmarker}{}%
\end{pgfscope}%
\begin{pgfscope}%
\pgfsys@transformshift{0.871151in}{1.156465in}%
\pgfsys@useobject{currentmarker}{}%
\end{pgfscope}%
\begin{pgfscope}%
\pgfsys@transformshift{0.914021in}{1.121752in}%
\pgfsys@useobject{currentmarker}{}%
\end{pgfscope}%
\begin{pgfscope}%
\pgfsys@transformshift{1.300630in}{1.333124in}%
\pgfsys@useobject{currentmarker}{}%
\end{pgfscope}%
\begin{pgfscope}%
\pgfsys@transformshift{1.025013in}{1.067298in}%
\pgfsys@useobject{currentmarker}{}%
\end{pgfscope}%
\begin{pgfscope}%
\pgfsys@transformshift{0.673824in}{0.919261in}%
\pgfsys@useobject{currentmarker}{}%
\end{pgfscope}%
\begin{pgfscope}%
\pgfsys@transformshift{0.931809in}{1.223396in}%
\pgfsys@useobject{currentmarker}{}%
\end{pgfscope}%
\begin{pgfscope}%
\pgfsys@transformshift{0.831397in}{1.303319in}%
\pgfsys@useobject{currentmarker}{}%
\end{pgfscope}%
\begin{pgfscope}%
\pgfsys@transformshift{1.034134in}{1.055835in}%
\pgfsys@useobject{currentmarker}{}%
\end{pgfscope}%
\begin{pgfscope}%
\pgfsys@transformshift{1.007225in}{1.012578in}%
\pgfsys@useobject{currentmarker}{}%
\end{pgfscope}%
\begin{pgfscope}%
\pgfsys@transformshift{0.780879in}{1.399067in}%
\pgfsys@useobject{currentmarker}{}%
\end{pgfscope}%
\begin{pgfscope}%
\pgfsys@transformshift{1.225200in}{1.250326in}%
\pgfsys@useobject{currentmarker}{}%
\end{pgfscope}%
\begin{pgfscope}%
\pgfsys@transformshift{1.224790in}{1.117457in}%
\pgfsys@useobject{currentmarker}{}%
\end{pgfscope}%
\begin{pgfscope}%
\pgfsys@transformshift{1.224861in}{1.234798in}%
\pgfsys@useobject{currentmarker}{}%
\end{pgfscope}%
\begin{pgfscope}%
\pgfsys@transformshift{0.949258in}{0.975252in}%
\pgfsys@useobject{currentmarker}{}%
\end{pgfscope}%
\begin{pgfscope}%
\pgfsys@transformshift{0.737570in}{1.414182in}%
\pgfsys@useobject{currentmarker}{}%
\end{pgfscope}%
\begin{pgfscope}%
\pgfsys@transformshift{0.612260in}{1.298297in}%
\pgfsys@useobject{currentmarker}{}%
\end{pgfscope}%
\begin{pgfscope}%
\pgfsys@transformshift{1.310501in}{1.279971in}%
\pgfsys@useobject{currentmarker}{}%
\end{pgfscope}%
\begin{pgfscope}%
\pgfsys@transformshift{0.729951in}{1.294555in}%
\pgfsys@useobject{currentmarker}{}%
\end{pgfscope}%
\begin{pgfscope}%
\pgfsys@transformshift{0.976605in}{1.453957in}%
\pgfsys@useobject{currentmarker}{}%
\end{pgfscope}%
\begin{pgfscope}%
\pgfsys@transformshift{0.704260in}{0.867912in}%
\pgfsys@useobject{currentmarker}{}%
\end{pgfscope}%
\begin{pgfscope}%
\pgfsys@transformshift{1.293195in}{1.326878in}%
\pgfsys@useobject{currentmarker}{}%
\end{pgfscope}%
\begin{pgfscope}%
\pgfsys@transformshift{1.039558in}{1.247881in}%
\pgfsys@useobject{currentmarker}{}%
\end{pgfscope}%
\begin{pgfscope}%
\pgfsys@transformshift{0.676020in}{1.315440in}%
\pgfsys@useobject{currentmarker}{}%
\end{pgfscope}%
\begin{pgfscope}%
\pgfsys@transformshift{1.177614in}{1.183294in}%
\pgfsys@useobject{currentmarker}{}%
\end{pgfscope}%
\begin{pgfscope}%
\pgfsys@transformshift{1.401552in}{1.172075in}%
\pgfsys@useobject{currentmarker}{}%
\end{pgfscope}%
\begin{pgfscope}%
\pgfsys@transformshift{1.095840in}{1.334512in}%
\pgfsys@useobject{currentmarker}{}%
\end{pgfscope}%
\begin{pgfscope}%
\pgfsys@transformshift{0.849171in}{0.921189in}%
\pgfsys@useobject{currentmarker}{}%
\end{pgfscope}%
\begin{pgfscope}%
\pgfsys@transformshift{0.808340in}{1.122847in}%
\pgfsys@useobject{currentmarker}{}%
\end{pgfscope}%
\begin{pgfscope}%
\pgfsys@transformshift{0.978532in}{0.882270in}%
\pgfsys@useobject{currentmarker}{}%
\end{pgfscope}%
\begin{pgfscope}%
\pgfsys@transformshift{1.004661in}{1.417829in}%
\pgfsys@useobject{currentmarker}{}%
\end{pgfscope}%
\begin{pgfscope}%
\pgfsys@transformshift{1.014958in}{1.028679in}%
\pgfsys@useobject{currentmarker}{}%
\end{pgfscope}%
\begin{pgfscope}%
\pgfsys@transformshift{1.010086in}{0.963501in}%
\pgfsys@useobject{currentmarker}{}%
\end{pgfscope}%
\begin{pgfscope}%
\pgfsys@transformshift{0.942148in}{0.733592in}%
\pgfsys@useobject{currentmarker}{}%
\end{pgfscope}%
\begin{pgfscope}%
\pgfsys@transformshift{0.614752in}{1.145602in}%
\pgfsys@useobject{currentmarker}{}%
\end{pgfscope}%
\begin{pgfscope}%
\pgfsys@transformshift{0.839540in}{1.262847in}%
\pgfsys@useobject{currentmarker}{}%
\end{pgfscope}%
\begin{pgfscope}%
\pgfsys@transformshift{0.792846in}{1.129494in}%
\pgfsys@useobject{currentmarker}{}%
\end{pgfscope}%
\begin{pgfscope}%
\pgfsys@transformshift{0.650952in}{1.235969in}%
\pgfsys@useobject{currentmarker}{}%
\end{pgfscope}%
\begin{pgfscope}%
\pgfsys@transformshift{1.467139in}{1.335179in}%
\pgfsys@useobject{currentmarker}{}%
\end{pgfscope}%
\begin{pgfscope}%
\pgfsys@transformshift{0.519339in}{1.266976in}%
\pgfsys@useobject{currentmarker}{}%
\end{pgfscope}%
\begin{pgfscope}%
\pgfsys@transformshift{0.615503in}{1.384734in}%
\pgfsys@useobject{currentmarker}{}%
\end{pgfscope}%
\begin{pgfscope}%
\pgfsys@transformshift{1.114294in}{1.231043in}%
\pgfsys@useobject{currentmarker}{}%
\end{pgfscope}%
\begin{pgfscope}%
\pgfsys@transformshift{1.368200in}{1.348170in}%
\pgfsys@useobject{currentmarker}{}%
\end{pgfscope}%
\begin{pgfscope}%
\pgfsys@transformshift{0.633107in}{1.291912in}%
\pgfsys@useobject{currentmarker}{}%
\end{pgfscope}%
\begin{pgfscope}%
\pgfsys@transformshift{0.774577in}{1.217975in}%
\pgfsys@useobject{currentmarker}{}%
\end{pgfscope}%
\begin{pgfscope}%
\pgfsys@transformshift{0.783825in}{1.390913in}%
\pgfsys@useobject{currentmarker}{}%
\end{pgfscope}%
\begin{pgfscope}%
\pgfsys@transformshift{0.989989in}{1.583347in}%
\pgfsys@useobject{currentmarker}{}%
\end{pgfscope}%
\begin{pgfscope}%
\pgfsys@transformshift{0.603493in}{1.369733in}%
\pgfsys@useobject{currentmarker}{}%
\end{pgfscope}%
\begin{pgfscope}%
\pgfsys@transformshift{0.782876in}{1.166922in}%
\pgfsys@useobject{currentmarker}{}%
\end{pgfscope}%
\begin{pgfscope}%
\pgfsys@transformshift{0.963335in}{0.958491in}%
\pgfsys@useobject{currentmarker}{}%
\end{pgfscope}%
\begin{pgfscope}%
\pgfsys@transformshift{0.954427in}{0.958432in}%
\pgfsys@useobject{currentmarker}{}%
\end{pgfscope}%
\begin{pgfscope}%
\pgfsys@transformshift{1.301947in}{1.365657in}%
\pgfsys@useobject{currentmarker}{}%
\end{pgfscope}%
\begin{pgfscope}%
\pgfsys@transformshift{1.130680in}{1.072463in}%
\pgfsys@useobject{currentmarker}{}%
\end{pgfscope}%
\begin{pgfscope}%
\pgfsys@transformshift{0.742640in}{1.239248in}%
\pgfsys@useobject{currentmarker}{}%
\end{pgfscope}%
\begin{pgfscope}%
\pgfsys@transformshift{0.611892in}{1.184782in}%
\pgfsys@useobject{currentmarker}{}%
\end{pgfscope}%
\begin{pgfscope}%
\pgfsys@transformshift{0.971691in}{1.023972in}%
\pgfsys@useobject{currentmarker}{}%
\end{pgfscope}%
\begin{pgfscope}%
\pgfsys@transformshift{0.934245in}{1.399849in}%
\pgfsys@useobject{currentmarker}{}%
\end{pgfscope}%
\begin{pgfscope}%
\pgfsys@transformshift{0.831666in}{1.258804in}%
\pgfsys@useobject{currentmarker}{}%
\end{pgfscope}%
\begin{pgfscope}%
\pgfsys@transformshift{0.601964in}{1.181462in}%
\pgfsys@useobject{currentmarker}{}%
\end{pgfscope}%
\begin{pgfscope}%
\pgfsys@transformshift{0.771730in}{1.188768in}%
\pgfsys@useobject{currentmarker}{}%
\end{pgfscope}%
\begin{pgfscope}%
\pgfsys@transformshift{0.764479in}{1.158431in}%
\pgfsys@useobject{currentmarker}{}%
\end{pgfscope}%
\begin{pgfscope}%
\pgfsys@transformshift{0.864721in}{1.191223in}%
\pgfsys@useobject{currentmarker}{}%
\end{pgfscope}%
\begin{pgfscope}%
\pgfsys@transformshift{0.626805in}{1.445732in}%
\pgfsys@useobject{currentmarker}{}%
\end{pgfscope}%
\begin{pgfscope}%
\pgfsys@transformshift{0.621805in}{1.224437in}%
\pgfsys@useobject{currentmarker}{}%
\end{pgfscope}%
\begin{pgfscope}%
\pgfsys@transformshift{0.912860in}{1.347055in}%
\pgfsys@useobject{currentmarker}{}%
\end{pgfscope}%
\begin{pgfscope}%
\pgfsys@transformshift{1.103502in}{1.187365in}%
\pgfsys@useobject{currentmarker}{}%
\end{pgfscope}%
\begin{pgfscope}%
\pgfsys@transformshift{0.971719in}{1.576387in}%
\pgfsys@useobject{currentmarker}{}%
\end{pgfscope}%
\begin{pgfscope}%
\pgfsys@transformshift{1.079312in}{1.219657in}%
\pgfsys@useobject{currentmarker}{}%
\end{pgfscope}%
\begin{pgfscope}%
\pgfsys@transformshift{0.630600in}{1.122608in}%
\pgfsys@useobject{currentmarker}{}%
\end{pgfscope}%
\begin{pgfscope}%
\pgfsys@transformshift{0.820662in}{0.988549in}%
\pgfsys@useobject{currentmarker}{}%
\end{pgfscope}%
\begin{pgfscope}%
\pgfsys@transformshift{1.381810in}{1.265599in}%
\pgfsys@useobject{currentmarker}{}%
\end{pgfscope}%
\begin{pgfscope}%
\pgfsys@transformshift{1.091223in}{1.096740in}%
\pgfsys@useobject{currentmarker}{}%
\end{pgfscope}%
\begin{pgfscope}%
\pgfsys@transformshift{1.078717in}{1.028677in}%
\pgfsys@useobject{currentmarker}{}%
\end{pgfscope}%
\begin{pgfscope}%
\pgfsys@transformshift{1.050265in}{1.068408in}%
\pgfsys@useobject{currentmarker}{}%
\end{pgfscope}%
\begin{pgfscope}%
\pgfsys@transformshift{0.988006in}{1.011089in}%
\pgfsys@useobject{currentmarker}{}%
\end{pgfscope}%
\begin{pgfscope}%
\pgfsys@transformshift{1.003797in}{0.903456in}%
\pgfsys@useobject{currentmarker}{}%
\end{pgfscope}%
\begin{pgfscope}%
\pgfsys@transformshift{0.624865in}{1.287323in}%
\pgfsys@useobject{currentmarker}{}%
\end{pgfscope}%
\begin{pgfscope}%
\pgfsys@transformshift{0.701215in}{1.219546in}%
\pgfsys@useobject{currentmarker}{}%
\end{pgfscope}%
\begin{pgfscope}%
\pgfsys@transformshift{0.860076in}{1.338487in}%
\pgfsys@useobject{currentmarker}{}%
\end{pgfscope}%
\begin{pgfscope}%
\pgfsys@transformshift{0.669590in}{1.413280in}%
\pgfsys@useobject{currentmarker}{}%
\end{pgfscope}%
\begin{pgfscope}%
\pgfsys@transformshift{1.130028in}{1.112367in}%
\pgfsys@useobject{currentmarker}{}%
\end{pgfscope}%
\begin{pgfscope}%
\pgfsys@transformshift{0.778386in}{1.192406in}%
\pgfsys@useobject{currentmarker}{}%
\end{pgfscope}%
\begin{pgfscope}%
\pgfsys@transformshift{0.724569in}{1.207977in}%
\pgfsys@useobject{currentmarker}{}%
\end{pgfscope}%
\begin{pgfscope}%
\pgfsys@transformshift{1.022719in}{1.066427in}%
\pgfsys@useobject{currentmarker}{}%
\end{pgfscope}%
\begin{pgfscope}%
\pgfsys@transformshift{0.640061in}{1.479768in}%
\pgfsys@useobject{currentmarker}{}%
\end{pgfscope}%
\begin{pgfscope}%
\pgfsys@transformshift{0.687477in}{1.338157in}%
\pgfsys@useobject{currentmarker}{}%
\end{pgfscope}%
\begin{pgfscope}%
\pgfsys@transformshift{0.550823in}{1.132724in}%
\pgfsys@useobject{currentmarker}{}%
\end{pgfscope}%
\begin{pgfscope}%
\pgfsys@transformshift{0.866718in}{1.219682in}%
\pgfsys@useobject{currentmarker}{}%
\end{pgfscope}%
\begin{pgfscope}%
\pgfsys@transformshift{0.933197in}{1.464534in}%
\pgfsys@useobject{currentmarker}{}%
\end{pgfscope}%
\begin{pgfscope}%
\pgfsys@transformshift{0.768161in}{1.216140in}%
\pgfsys@useobject{currentmarker}{}%
\end{pgfscope}%
\begin{pgfscope}%
\pgfsys@transformshift{1.310162in}{1.263150in}%
\pgfsys@useobject{currentmarker}{}%
\end{pgfscope}%
\begin{pgfscope}%
\pgfsys@transformshift{0.726396in}{1.143657in}%
\pgfsys@useobject{currentmarker}{}%
\end{pgfscope}%
\begin{pgfscope}%
\pgfsys@transformshift{0.698283in}{1.089096in}%
\pgfsys@useobject{currentmarker}{}%
\end{pgfscope}%
\begin{pgfscope}%
\pgfsys@transformshift{0.889293in}{1.396959in}%
\pgfsys@useobject{currentmarker}{}%
\end{pgfscope}%
\begin{pgfscope}%
\pgfsys@transformshift{0.976775in}{0.765418in}%
\pgfsys@useobject{currentmarker}{}%
\end{pgfscope}%
\begin{pgfscope}%
\pgfsys@transformshift{0.948394in}{1.018354in}%
\pgfsys@useobject{currentmarker}{}%
\end{pgfscope}%
\begin{pgfscope}%
\pgfsys@transformshift{0.941355in}{0.792332in}%
\pgfsys@useobject{currentmarker}{}%
\end{pgfscope}%
\begin{pgfscope}%
\pgfsys@transformshift{1.336915in}{1.131004in}%
\pgfsys@useobject{currentmarker}{}%
\end{pgfscope}%
\begin{pgfscope}%
\pgfsys@transformshift{1.305757in}{1.313555in}%
\pgfsys@useobject{currentmarker}{}%
\end{pgfscope}%
\begin{pgfscope}%
\pgfsys@transformshift{0.687902in}{1.218538in}%
\pgfsys@useobject{currentmarker}{}%
\end{pgfscope}%
\begin{pgfscope}%
\pgfsys@transformshift{0.544549in}{1.411629in}%
\pgfsys@useobject{currentmarker}{}%
\end{pgfscope}%
\begin{pgfscope}%
\pgfsys@transformshift{0.592050in}{1.137079in}%
\pgfsys@useobject{currentmarker}{}%
\end{pgfscope}%
\begin{pgfscope}%
\pgfsys@transformshift{1.087201in}{1.086925in}%
\pgfsys@useobject{currentmarker}{}%
\end{pgfscope}%
\begin{pgfscope}%
\pgfsys@transformshift{0.822772in}{1.419992in}%
\pgfsys@useobject{currentmarker}{}%
\end{pgfscope}%
\begin{pgfscope}%
\pgfsys@transformshift{0.903881in}{1.235215in}%
\pgfsys@useobject{currentmarker}{}%
\end{pgfscope}%
\begin{pgfscope}%
\pgfsys@transformshift{0.794801in}{1.497122in}%
\pgfsys@useobject{currentmarker}{}%
\end{pgfscope}%
\begin{pgfscope}%
\pgfsys@transformshift{0.648488in}{1.016928in}%
\pgfsys@useobject{currentmarker}{}%
\end{pgfscope}%
\begin{pgfscope}%
\pgfsys@transformshift{0.534451in}{1.218395in}%
\pgfsys@useobject{currentmarker}{}%
\end{pgfscope}%
\begin{pgfscope}%
\pgfsys@transformshift{0.745105in}{1.213682in}%
\pgfsys@useobject{currentmarker}{}%
\end{pgfscope}%
\begin{pgfscope}%
\pgfsys@transformshift{1.194553in}{1.193965in}%
\pgfsys@useobject{currentmarker}{}%
\end{pgfscope}%
\begin{pgfscope}%
\pgfsys@transformshift{0.745798in}{1.469655in}%
\pgfsys@useobject{currentmarker}{}%
\end{pgfscope}%
\begin{pgfscope}%
\pgfsys@transformshift{0.619044in}{1.144285in}%
\pgfsys@useobject{currentmarker}{}%
\end{pgfscope}%
\begin{pgfscope}%
\pgfsys@transformshift{0.792903in}{1.322152in}%
\pgfsys@useobject{currentmarker}{}%
\end{pgfscope}%
\begin{pgfscope}%
\pgfsys@transformshift{0.936540in}{1.457085in}%
\pgfsys@useobject{currentmarker}{}%
\end{pgfscope}%
\begin{pgfscope}%
\pgfsys@transformshift{0.741720in}{1.328886in}%
\pgfsys@useobject{currentmarker}{}%
\end{pgfscope}%
\begin{pgfscope}%
\pgfsys@transformshift{0.888444in}{1.445149in}%
\pgfsys@useobject{currentmarker}{}%
\end{pgfscope}%
\begin{pgfscope}%
\pgfsys@transformshift{0.890186in}{1.535825in}%
\pgfsys@useobject{currentmarker}{}%
\end{pgfscope}%
\begin{pgfscope}%
\pgfsys@transformshift{0.846848in}{1.127126in}%
\pgfsys@useobject{currentmarker}{}%
\end{pgfscope}%
\begin{pgfscope}%
\pgfsys@transformshift{0.627499in}{1.371458in}%
\pgfsys@useobject{currentmarker}{}%
\end{pgfscope}%
\begin{pgfscope}%
\pgfsys@transformshift{0.969057in}{1.011886in}%
\pgfsys@useobject{currentmarker}{}%
\end{pgfscope}%
\begin{pgfscope}%
\pgfsys@transformshift{0.787932in}{1.246019in}%
\pgfsys@useobject{currentmarker}{}%
\end{pgfscope}%
\begin{pgfscope}%
\pgfsys@transformshift{1.219210in}{1.158517in}%
\pgfsys@useobject{currentmarker}{}%
\end{pgfscope}%
\begin{pgfscope}%
\pgfsys@transformshift{1.080473in}{1.127460in}%
\pgfsys@useobject{currentmarker}{}%
\end{pgfscope}%
\begin{pgfscope}%
\pgfsys@transformshift{0.889067in}{1.473420in}%
\pgfsys@useobject{currentmarker}{}%
\end{pgfscope}%
\begin{pgfscope}%
\pgfsys@transformshift{0.616849in}{1.271002in}%
\pgfsys@useobject{currentmarker}{}%
\end{pgfscope}%
\begin{pgfscope}%
\pgfsys@transformshift{1.017337in}{1.049656in}%
\pgfsys@useobject{currentmarker}{}%
\end{pgfscope}%
\begin{pgfscope}%
\pgfsys@transformshift{0.983403in}{1.654197in}%
\pgfsys@useobject{currentmarker}{}%
\end{pgfscope}%
\begin{pgfscope}%
\pgfsys@transformshift{0.584076in}{1.063088in}%
\pgfsys@useobject{currentmarker}{}%
\end{pgfscope}%
\begin{pgfscope}%
\pgfsys@transformshift{0.615050in}{1.259418in}%
\pgfsys@useobject{currentmarker}{}%
\end{pgfscope}%
\begin{pgfscope}%
\pgfsys@transformshift{0.913681in}{1.545664in}%
\pgfsys@useobject{currentmarker}{}%
\end{pgfscope}%
\begin{pgfscope}%
\pgfsys@transformshift{1.260918in}{1.284674in}%
\pgfsys@useobject{currentmarker}{}%
\end{pgfscope}%
\begin{pgfscope}%
\pgfsys@transformshift{0.579842in}{0.961214in}%
\pgfsys@useobject{currentmarker}{}%
\end{pgfscope}%
\begin{pgfscope}%
\pgfsys@transformshift{0.908469in}{1.390713in}%
\pgfsys@useobject{currentmarker}{}%
\end{pgfscope}%
\begin{pgfscope}%
\pgfsys@transformshift{0.984097in}{1.559494in}%
\pgfsys@useobject{currentmarker}{}%
\end{pgfscope}%
\begin{pgfscope}%
\pgfsys@transformshift{0.874947in}{1.536498in}%
\pgfsys@useobject{currentmarker}{}%
\end{pgfscope}%
\begin{pgfscope}%
\pgfsys@transformshift{1.138540in}{1.123644in}%
\pgfsys@useobject{currentmarker}{}%
\end{pgfscope}%
\begin{pgfscope}%
\pgfsys@transformshift{1.252874in}{0.935079in}%
\pgfsys@useobject{currentmarker}{}%
\end{pgfscope}%
\begin{pgfscope}%
\pgfsys@transformshift{0.662155in}{1.314197in}%
\pgfsys@useobject{currentmarker}{}%
\end{pgfscope}%
\begin{pgfscope}%
\pgfsys@transformshift{0.914488in}{1.477117in}%
\pgfsys@useobject{currentmarker}{}%
\end{pgfscope}%
\begin{pgfscope}%
\pgfsys@transformshift{1.101264in}{1.099558in}%
\pgfsys@useobject{currentmarker}{}%
\end{pgfscope}%
\begin{pgfscope}%
\pgfsys@transformshift{0.616452in}{1.292128in}%
\pgfsys@useobject{currentmarker}{}%
\end{pgfscope}%
\begin{pgfscope}%
\pgfsys@transformshift{0.718649in}{0.938061in}%
\pgfsys@useobject{currentmarker}{}%
\end{pgfscope}%
\begin{pgfscope}%
\pgfsys@transformshift{0.775327in}{1.159276in}%
\pgfsys@useobject{currentmarker}{}%
\end{pgfscope}%
\begin{pgfscope}%
\pgfsys@transformshift{0.703976in}{1.496606in}%
\pgfsys@useobject{currentmarker}{}%
\end{pgfscope}%
\begin{pgfscope}%
\pgfsys@transformshift{0.802803in}{1.305468in}%
\pgfsys@useobject{currentmarker}{}%
\end{pgfscope}%
\begin{pgfscope}%
\pgfsys@transformshift{0.614795in}{1.371892in}%
\pgfsys@useobject{currentmarker}{}%
\end{pgfscope}%
\begin{pgfscope}%
\pgfsys@transformshift{1.025849in}{1.183786in}%
\pgfsys@useobject{currentmarker}{}%
\end{pgfscope}%
\begin{pgfscope}%
\pgfsys@transformshift{0.633348in}{1.351906in}%
\pgfsys@useobject{currentmarker}{}%
\end{pgfscope}%
\begin{pgfscope}%
\pgfsys@transformshift{0.942020in}{1.624929in}%
\pgfsys@useobject{currentmarker}{}%
\end{pgfscope}%
\begin{pgfscope}%
\pgfsys@transformshift{0.995484in}{1.032122in}%
\pgfsys@useobject{currentmarker}{}%
\end{pgfscope}%
\begin{pgfscope}%
\pgfsys@transformshift{1.024970in}{1.515672in}%
\pgfsys@useobject{currentmarker}{}%
\end{pgfscope}%
\begin{pgfscope}%
\pgfsys@transformshift{0.665114in}{1.092110in}%
\pgfsys@useobject{currentmarker}{}%
\end{pgfscope}%
\begin{pgfscope}%
\pgfsys@transformshift{1.143752in}{1.080735in}%
\pgfsys@useobject{currentmarker}{}%
\end{pgfscope}%
\begin{pgfscope}%
\pgfsys@transformshift{1.266201in}{1.279840in}%
\pgfsys@useobject{currentmarker}{}%
\end{pgfscope}%
\begin{pgfscope}%
\pgfsys@transformshift{0.699473in}{1.260576in}%
\pgfsys@useobject{currentmarker}{}%
\end{pgfscope}%
\begin{pgfscope}%
\pgfsys@transformshift{0.588226in}{1.227664in}%
\pgfsys@useobject{currentmarker}{}%
\end{pgfscope}%
\begin{pgfscope}%
\pgfsys@transformshift{0.536306in}{0.877002in}%
\pgfsys@useobject{currentmarker}{}%
\end{pgfscope}%
\begin{pgfscope}%
\pgfsys@transformshift{0.724725in}{0.969020in}%
\pgfsys@useobject{currentmarker}{}%
\end{pgfscope}%
\begin{pgfscope}%
\pgfsys@transformshift{0.695479in}{1.257995in}%
\pgfsys@useobject{currentmarker}{}%
\end{pgfscope}%
\begin{pgfscope}%
\pgfsys@transformshift{0.704571in}{1.224278in}%
\pgfsys@useobject{currentmarker}{}%
\end{pgfscope}%
\begin{pgfscope}%
\pgfsys@transformshift{0.638730in}{1.220119in}%
\pgfsys@useobject{currentmarker}{}%
\end{pgfscope}%
\begin{pgfscope}%
\pgfsys@transformshift{0.968873in}{1.030302in}%
\pgfsys@useobject{currentmarker}{}%
\end{pgfscope}%
\begin{pgfscope}%
\pgfsys@transformshift{1.205982in}{1.251730in}%
\pgfsys@useobject{currentmarker}{}%
\end{pgfscope}%
\begin{pgfscope}%
\pgfsys@transformshift{0.972881in}{1.029880in}%
\pgfsys@useobject{currentmarker}{}%
\end{pgfscope}%
\begin{pgfscope}%
\pgfsys@transformshift{0.987808in}{1.256119in}%
\pgfsys@useobject{currentmarker}{}%
\end{pgfscope}%
\begin{pgfscope}%
\pgfsys@transformshift{0.997651in}{1.445061in}%
\pgfsys@useobject{currentmarker}{}%
\end{pgfscope}%
\begin{pgfscope}%
\pgfsys@transformshift{0.919573in}{1.422562in}%
\pgfsys@useobject{currentmarker}{}%
\end{pgfscope}%
\begin{pgfscope}%
\pgfsys@transformshift{0.659010in}{1.121971in}%
\pgfsys@useobject{currentmarker}{}%
\end{pgfscope}%
\begin{pgfscope}%
\pgfsys@transformshift{1.035068in}{1.094311in}%
\pgfsys@useobject{currentmarker}{}%
\end{pgfscope}%
\begin{pgfscope}%
\pgfsys@transformshift{0.536873in}{0.966873in}%
\pgfsys@useobject{currentmarker}{}%
\end{pgfscope}%
\begin{pgfscope}%
\pgfsys@transformshift{0.947246in}{1.582351in}%
\pgfsys@useobject{currentmarker}{}%
\end{pgfscope}%
\begin{pgfscope}%
\pgfsys@transformshift{0.719768in}{1.364758in}%
\pgfsys@useobject{currentmarker}{}%
\end{pgfscope}%
\begin{pgfscope}%
\pgfsys@transformshift{1.072316in}{1.615621in}%
\pgfsys@useobject{currentmarker}{}%
\end{pgfscope}%
\begin{pgfscope}%
\pgfsys@transformshift{0.564405in}{1.252954in}%
\pgfsys@useobject{currentmarker}{}%
\end{pgfscope}%
\begin{pgfscope}%
\pgfsys@transformshift{0.666134in}{1.201008in}%
\pgfsys@useobject{currentmarker}{}%
\end{pgfscope}%
\begin{pgfscope}%
\pgfsys@transformshift{0.752639in}{1.479714in}%
\pgfsys@useobject{currentmarker}{}%
\end{pgfscope}%
\begin{pgfscope}%
\pgfsys@transformshift{0.677223in}{1.239817in}%
\pgfsys@useobject{currentmarker}{}%
\end{pgfscope}%
\begin{pgfscope}%
\pgfsys@transformshift{0.836637in}{1.399063in}%
\pgfsys@useobject{currentmarker}{}%
\end{pgfscope}%
\begin{pgfscope}%
\pgfsys@transformshift{0.902606in}{1.381579in}%
\pgfsys@useobject{currentmarker}{}%
\end{pgfscope}%
\begin{pgfscope}%
\pgfsys@transformshift{1.281766in}{1.063933in}%
\pgfsys@useobject{currentmarker}{}%
\end{pgfscope}%
\begin{pgfscope}%
\pgfsys@transformshift{1.338628in}{1.358617in}%
\pgfsys@useobject{currentmarker}{}%
\end{pgfscope}%
\begin{pgfscope}%
\pgfsys@transformshift{0.847471in}{1.111747in}%
\pgfsys@useobject{currentmarker}{}%
\end{pgfscope}%
\begin{pgfscope}%
\pgfsys@transformshift{0.704203in}{1.265600in}%
\pgfsys@useobject{currentmarker}{}%
\end{pgfscope}%
\begin{pgfscope}%
\pgfsys@transformshift{0.629368in}{1.421984in}%
\pgfsys@useobject{currentmarker}{}%
\end{pgfscope}%
\begin{pgfscope}%
\pgfsys@transformshift{0.683639in}{1.396415in}%
\pgfsys@useobject{currentmarker}{}%
\end{pgfscope}%
\begin{pgfscope}%
\pgfsys@transformshift{0.590223in}{1.363277in}%
\pgfsys@useobject{currentmarker}{}%
\end{pgfscope}%
\begin{pgfscope}%
\pgfsys@transformshift{0.727571in}{1.130853in}%
\pgfsys@useobject{currentmarker}{}%
\end{pgfscope}%
\begin{pgfscope}%
\pgfsys@transformshift{0.656957in}{1.344406in}%
\pgfsys@useobject{currentmarker}{}%
\end{pgfscope}%
\begin{pgfscope}%
\pgfsys@transformshift{0.858178in}{1.150126in}%
\pgfsys@useobject{currentmarker}{}%
\end{pgfscope}%
\begin{pgfscope}%
\pgfsys@transformshift{0.939613in}{0.935342in}%
\pgfsys@useobject{currentmarker}{}%
\end{pgfscope}%
\begin{pgfscope}%
\pgfsys@transformshift{0.644324in}{1.007844in}%
\pgfsys@useobject{currentmarker}{}%
\end{pgfscope}%
\begin{pgfscope}%
\pgfsys@transformshift{0.680580in}{1.209674in}%
\pgfsys@useobject{currentmarker}{}%
\end{pgfscope}%
\begin{pgfscope}%
\pgfsys@transformshift{0.596200in}{1.181212in}%
\pgfsys@useobject{currentmarker}{}%
\end{pgfscope}%
\begin{pgfscope}%
\pgfsys@transformshift{0.681954in}{1.288887in}%
\pgfsys@useobject{currentmarker}{}%
\end{pgfscope}%
\begin{pgfscope}%
\pgfsys@transformshift{0.779180in}{1.379622in}%
\pgfsys@useobject{currentmarker}{}%
\end{pgfscope}%
\begin{pgfscope}%
\pgfsys@transformshift{1.013456in}{1.073738in}%
\pgfsys@useobject{currentmarker}{}%
\end{pgfscope}%
\begin{pgfscope}%
\pgfsys@transformshift{0.649507in}{1.380911in}%
\pgfsys@useobject{currentmarker}{}%
\end{pgfscope}%
\begin{pgfscope}%
\pgfsys@transformshift{1.170491in}{1.117369in}%
\pgfsys@useobject{currentmarker}{}%
\end{pgfscope}%
\begin{pgfscope}%
\pgfsys@transformshift{0.929387in}{1.545504in}%
\pgfsys@useobject{currentmarker}{}%
\end{pgfscope}%
\begin{pgfscope}%
\pgfsys@transformshift{1.309581in}{1.307990in}%
\pgfsys@useobject{currentmarker}{}%
\end{pgfscope}%
\begin{pgfscope}%
\pgfsys@transformshift{0.545710in}{1.027942in}%
\pgfsys@useobject{currentmarker}{}%
\end{pgfscope}%
\begin{pgfscope}%
\pgfsys@transformshift{1.308080in}{1.315370in}%
\pgfsys@useobject{currentmarker}{}%
\end{pgfscope}%
\begin{pgfscope}%
\pgfsys@transformshift{1.333246in}{1.302216in}%
\pgfsys@useobject{currentmarker}{}%
\end{pgfscope}%
\begin{pgfscope}%
\pgfsys@transformshift{0.958407in}{0.730513in}%
\pgfsys@useobject{currentmarker}{}%
\end{pgfscope}%
\begin{pgfscope}%
\pgfsys@transformshift{0.581938in}{0.950422in}%
\pgfsys@useobject{currentmarker}{}%
\end{pgfscope}%
\begin{pgfscope}%
\pgfsys@transformshift{1.110271in}{1.576752in}%
\pgfsys@useobject{currentmarker}{}%
\end{pgfscope}%
\begin{pgfscope}%
\pgfsys@transformshift{1.127167in}{1.143390in}%
\pgfsys@useobject{currentmarker}{}%
\end{pgfscope}%
\begin{pgfscope}%
\pgfsys@transformshift{0.846749in}{1.264243in}%
\pgfsys@useobject{currentmarker}{}%
\end{pgfscope}%
\begin{pgfscope}%
\pgfsys@transformshift{0.588240in}{1.263028in}%
\pgfsys@useobject{currentmarker}{}%
\end{pgfscope}%
\begin{pgfscope}%
\pgfsys@transformshift{0.641066in}{1.313872in}%
\pgfsys@useobject{currentmarker}{}%
\end{pgfscope}%
\begin{pgfscope}%
\pgfsys@transformshift{1.128300in}{1.137655in}%
\pgfsys@useobject{currentmarker}{}%
\end{pgfscope}%
\begin{pgfscope}%
\pgfsys@transformshift{0.630770in}{1.174278in}%
\pgfsys@useobject{currentmarker}{}%
\end{pgfscope}%
\begin{pgfscope}%
\pgfsys@transformshift{1.172742in}{1.252140in}%
\pgfsys@useobject{currentmarker}{}%
\end{pgfscope}%
\begin{pgfscope}%
\pgfsys@transformshift{0.640103in}{0.673827in}%
\pgfsys@useobject{currentmarker}{}%
\end{pgfscope}%
\begin{pgfscope}%
\pgfsys@transformshift{1.606272in}{1.300911in}%
\pgfsys@useobject{currentmarker}{}%
\end{pgfscope}%
\begin{pgfscope}%
\pgfsys@transformshift{1.225285in}{1.288872in}%
\pgfsys@useobject{currentmarker}{}%
\end{pgfscope}%
\begin{pgfscope}%
\pgfsys@transformshift{0.702334in}{1.290041in}%
\pgfsys@useobject{currentmarker}{}%
\end{pgfscope}%
\begin{pgfscope}%
\pgfsys@transformshift{0.772707in}{1.146972in}%
\pgfsys@useobject{currentmarker}{}%
\end{pgfscope}%
\begin{pgfscope}%
\pgfsys@transformshift{1.046908in}{1.520757in}%
\pgfsys@useobject{currentmarker}{}%
\end{pgfscope}%
\begin{pgfscope}%
\pgfsys@transformshift{0.999690in}{0.936918in}%
\pgfsys@useobject{currentmarker}{}%
\end{pgfscope}%
\begin{pgfscope}%
\pgfsys@transformshift{1.102213in}{1.134447in}%
\pgfsys@useobject{currentmarker}{}%
\end{pgfscope}%
\begin{pgfscope}%
\pgfsys@transformshift{0.700478in}{1.079079in}%
\pgfsys@useobject{currentmarker}{}%
\end{pgfscope}%
\begin{pgfscope}%
\pgfsys@transformshift{0.583552in}{1.240720in}%
\pgfsys@useobject{currentmarker}{}%
\end{pgfscope}%
\begin{pgfscope}%
\pgfsys@transformshift{0.918624in}{0.847547in}%
\pgfsys@useobject{currentmarker}{}%
\end{pgfscope}%
\begin{pgfscope}%
\pgfsys@transformshift{0.868970in}{1.112796in}%
\pgfsys@useobject{currentmarker}{}%
\end{pgfscope}%
\begin{pgfscope}%
\pgfsys@transformshift{0.968349in}{1.470761in}%
\pgfsys@useobject{currentmarker}{}%
\end{pgfscope}%
\begin{pgfscope}%
\pgfsys@transformshift{0.811640in}{1.079960in}%
\pgfsys@useobject{currentmarker}{}%
\end{pgfscope}%
\begin{pgfscope}%
\pgfsys@transformshift{1.072245in}{0.912619in}%
\pgfsys@useobject{currentmarker}{}%
\end{pgfscope}%
\begin{pgfscope}%
\pgfsys@transformshift{0.983049in}{1.021960in}%
\pgfsys@useobject{currentmarker}{}%
\end{pgfscope}%
\begin{pgfscope}%
\pgfsys@transformshift{0.772297in}{1.188666in}%
\pgfsys@useobject{currentmarker}{}%
\end{pgfscope}%
\begin{pgfscope}%
\pgfsys@transformshift{1.197541in}{1.278490in}%
\pgfsys@useobject{currentmarker}{}%
\end{pgfscope}%
\begin{pgfscope}%
\pgfsys@transformshift{0.578978in}{1.300123in}%
\pgfsys@useobject{currentmarker}{}%
\end{pgfscope}%
\begin{pgfscope}%
\pgfsys@transformshift{0.676076in}{1.382541in}%
\pgfsys@useobject{currentmarker}{}%
\end{pgfscope}%
\begin{pgfscope}%
\pgfsys@transformshift{1.101901in}{1.175430in}%
\pgfsys@useobject{currentmarker}{}%
\end{pgfscope}%
\begin{pgfscope}%
\pgfsys@transformshift{0.718366in}{1.177997in}%
\pgfsys@useobject{currentmarker}{}%
\end{pgfscope}%
\begin{pgfscope}%
\pgfsys@transformshift{0.731268in}{1.274624in}%
\pgfsys@useobject{currentmarker}{}%
\end{pgfscope}%
\begin{pgfscope}%
\pgfsys@transformshift{1.329578in}{1.335530in}%
\pgfsys@useobject{currentmarker}{}%
\end{pgfscope}%
\begin{pgfscope}%
\pgfsys@transformshift{0.742187in}{1.268885in}%
\pgfsys@useobject{currentmarker}{}%
\end{pgfscope}%
\begin{pgfscope}%
\pgfsys@transformshift{0.616296in}{1.303834in}%
\pgfsys@useobject{currentmarker}{}%
\end{pgfscope}%
\begin{pgfscope}%
\pgfsys@transformshift{0.952118in}{1.220515in}%
\pgfsys@useobject{currentmarker}{}%
\end{pgfscope}%
\begin{pgfscope}%
\pgfsys@transformshift{0.921032in}{0.769484in}%
\pgfsys@useobject{currentmarker}{}%
\end{pgfscope}%
\begin{pgfscope}%
\pgfsys@transformshift{0.663160in}{1.123959in}%
\pgfsys@useobject{currentmarker}{}%
\end{pgfscope}%
\begin{pgfscope}%
\pgfsys@transformshift{0.920380in}{0.959927in}%
\pgfsys@useobject{currentmarker}{}%
\end{pgfscope}%
\begin{pgfscope}%
\pgfsys@transformshift{0.557862in}{1.220232in}%
\pgfsys@useobject{currentmarker}{}%
\end{pgfscope}%
\begin{pgfscope}%
\pgfsys@transformshift{0.601426in}{1.232290in}%
\pgfsys@useobject{currentmarker}{}%
\end{pgfscope}%
\begin{pgfscope}%
\pgfsys@transformshift{0.737882in}{1.294181in}%
\pgfsys@useobject{currentmarker}{}%
\end{pgfscope}%
\begin{pgfscope}%
\pgfsys@transformshift{0.903314in}{0.962866in}%
\pgfsys@useobject{currentmarker}{}%
\end{pgfscope}%
\begin{pgfscope}%
\pgfsys@transformshift{1.437412in}{1.351167in}%
\pgfsys@useobject{currentmarker}{}%
\end{pgfscope}%
\begin{pgfscope}%
\pgfsys@transformshift{0.728548in}{1.541356in}%
\pgfsys@useobject{currentmarker}{}%
\end{pgfscope}%
\begin{pgfscope}%
\pgfsys@transformshift{1.170576in}{1.143345in}%
\pgfsys@useobject{currentmarker}{}%
\end{pgfscope}%
\begin{pgfscope}%
\pgfsys@transformshift{0.846537in}{1.169750in}%
\pgfsys@useobject{currentmarker}{}%
\end{pgfscope}%
\begin{pgfscope}%
\pgfsys@transformshift{0.748532in}{1.436349in}%
\pgfsys@useobject{currentmarker}{}%
\end{pgfscope}%
\begin{pgfscope}%
\pgfsys@transformshift{0.808340in}{1.471405in}%
\pgfsys@useobject{currentmarker}{}%
\end{pgfscope}%
\begin{pgfscope}%
\pgfsys@transformshift{0.997693in}{1.534199in}%
\pgfsys@useobject{currentmarker}{}%
\end{pgfscope}%
\begin{pgfscope}%
\pgfsys@transformshift{1.297203in}{1.080198in}%
\pgfsys@useobject{currentmarker}{}%
\end{pgfscope}%
\begin{pgfscope}%
\pgfsys@transformshift{1.280080in}{1.243748in}%
\pgfsys@useobject{currentmarker}{}%
\end{pgfscope}%
\begin{pgfscope}%
\pgfsys@transformshift{0.973985in}{0.696492in}%
\pgfsys@useobject{currentmarker}{}%
\end{pgfscope}%
\begin{pgfscope}%
\pgfsys@transformshift{0.664831in}{1.379823in}%
\pgfsys@useobject{currentmarker}{}%
\end{pgfscope}%
\begin{pgfscope}%
\pgfsys@transformshift{0.921655in}{1.486874in}%
\pgfsys@useobject{currentmarker}{}%
\end{pgfscope}%
\begin{pgfscope}%
\pgfsys@transformshift{0.849157in}{1.173973in}%
\pgfsys@useobject{currentmarker}{}%
\end{pgfscope}%
\begin{pgfscope}%
\pgfsys@transformshift{0.837756in}{1.278480in}%
\pgfsys@useobject{currentmarker}{}%
\end{pgfscope}%
\begin{pgfscope}%
\pgfsys@transformshift{0.592829in}{1.281553in}%
\pgfsys@useobject{currentmarker}{}%
\end{pgfscope}%
\begin{pgfscope}%
\pgfsys@transformshift{0.951708in}{0.969583in}%
\pgfsys@useobject{currentmarker}{}%
\end{pgfscope}%
\begin{pgfscope}%
\pgfsys@transformshift{0.974311in}{1.546305in}%
\pgfsys@useobject{currentmarker}{}%
\end{pgfscope}%
\begin{pgfscope}%
\pgfsys@transformshift{0.795537in}{1.191314in}%
\pgfsys@useobject{currentmarker}{}%
\end{pgfscope}%
\begin{pgfscope}%
\pgfsys@transformshift{0.572888in}{0.846376in}%
\pgfsys@useobject{currentmarker}{}%
\end{pgfscope}%
\begin{pgfscope}%
\pgfsys@transformshift{1.024843in}{1.562746in}%
\pgfsys@useobject{currentmarker}{}%
\end{pgfscope}%
\begin{pgfscope}%
\pgfsys@transformshift{1.125411in}{1.175683in}%
\pgfsys@useobject{currentmarker}{}%
\end{pgfscope}%
\begin{pgfscope}%
\pgfsys@transformshift{1.194170in}{1.133860in}%
\pgfsys@useobject{currentmarker}{}%
\end{pgfscope}%
\begin{pgfscope}%
\pgfsys@transformshift{1.195261in}{1.232516in}%
\pgfsys@useobject{currentmarker}{}%
\end{pgfscope}%
\begin{pgfscope}%
\pgfsys@transformshift{0.810819in}{1.270082in}%
\pgfsys@useobject{currentmarker}{}%
\end{pgfscope}%
\begin{pgfscope}%
\pgfsys@transformshift{0.954795in}{1.065644in}%
\pgfsys@useobject{currentmarker}{}%
\end{pgfscope}%
\begin{pgfscope}%
\pgfsys@transformshift{0.660058in}{0.932637in}%
\pgfsys@useobject{currentmarker}{}%
\end{pgfscope}%
\begin{pgfscope}%
\pgfsys@transformshift{1.010411in}{1.462823in}%
\pgfsys@useobject{currentmarker}{}%
\end{pgfscope}%
\begin{pgfscope}%
\pgfsys@transformshift{0.939995in}{0.982925in}%
\pgfsys@useobject{currentmarker}{}%
\end{pgfscope}%
\begin{pgfscope}%
\pgfsys@transformshift{0.701583in}{0.960344in}%
\pgfsys@useobject{currentmarker}{}%
\end{pgfscope}%
\begin{pgfscope}%
\pgfsys@transformshift{0.855969in}{0.755646in}%
\pgfsys@useobject{currentmarker}{}%
\end{pgfscope}%
\begin{pgfscope}%
\pgfsys@transformshift{0.693397in}{1.510158in}%
\pgfsys@useobject{currentmarker}{}%
\end{pgfscope}%
\begin{pgfscope}%
\pgfsys@transformshift{0.935831in}{1.578868in}%
\pgfsys@useobject{currentmarker}{}%
\end{pgfscope}%
\begin{pgfscope}%
\pgfsys@transformshift{0.931540in}{1.414547in}%
\pgfsys@useobject{currentmarker}{}%
\end{pgfscope}%
\begin{pgfscope}%
\pgfsys@transformshift{0.968320in}{1.627683in}%
\pgfsys@useobject{currentmarker}{}%
\end{pgfscope}%
\begin{pgfscope}%
\pgfsys@transformshift{1.021869in}{1.535569in}%
\pgfsys@useobject{currentmarker}{}%
\end{pgfscope}%
\begin{pgfscope}%
\pgfsys@transformshift{0.760485in}{1.174785in}%
\pgfsys@useobject{currentmarker}{}%
\end{pgfscope}%
\begin{pgfscope}%
\pgfsys@transformshift{1.382844in}{1.255435in}%
\pgfsys@useobject{currentmarker}{}%
\end{pgfscope}%
\begin{pgfscope}%
\pgfsys@transformshift{1.798089in}{1.293609in}%
\pgfsys@useobject{currentmarker}{}%
\end{pgfscope}%
\begin{pgfscope}%
\pgfsys@transformshift{0.667069in}{1.203438in}%
\pgfsys@useobject{currentmarker}{}%
\end{pgfscope}%
\begin{pgfscope}%
\pgfsys@transformshift{1.034799in}{1.563683in}%
\pgfsys@useobject{currentmarker}{}%
\end{pgfscope}%
\begin{pgfscope}%
\pgfsys@transformshift{0.623873in}{1.314390in}%
\pgfsys@useobject{currentmarker}{}%
\end{pgfscope}%
\begin{pgfscope}%
\pgfsys@transformshift{0.821044in}{1.248705in}%
\pgfsys@useobject{currentmarker}{}%
\end{pgfscope}%
\begin{pgfscope}%
\pgfsys@transformshift{0.743688in}{1.568535in}%
\pgfsys@useobject{currentmarker}{}%
\end{pgfscope}%
\begin{pgfscope}%
\pgfsys@transformshift{0.827927in}{1.182341in}%
\pgfsys@useobject{currentmarker}{}%
\end{pgfscope}%
\begin{pgfscope}%
\pgfsys@transformshift{1.024956in}{1.270219in}%
\pgfsys@useobject{currentmarker}{}%
\end{pgfscope}%
\begin{pgfscope}%
\pgfsys@transformshift{0.787691in}{1.139234in}%
\pgfsys@useobject{currentmarker}{}%
\end{pgfscope}%
\begin{pgfscope}%
\pgfsys@transformshift{1.294809in}{1.016343in}%
\pgfsys@useobject{currentmarker}{}%
\end{pgfscope}%
\begin{pgfscope}%
\pgfsys@transformshift{1.000044in}{1.524272in}%
\pgfsys@useobject{currentmarker}{}%
\end{pgfscope}%
\begin{pgfscope}%
\pgfsys@transformshift{0.736026in}{1.109878in}%
\pgfsys@useobject{currentmarker}{}%
\end{pgfscope}%
\begin{pgfscope}%
\pgfsys@transformshift{1.017124in}{1.565310in}%
\pgfsys@useobject{currentmarker}{}%
\end{pgfscope}%
\begin{pgfscope}%
\pgfsys@transformshift{0.648898in}{1.177310in}%
\pgfsys@useobject{currentmarker}{}%
\end{pgfscope}%
\begin{pgfscope}%
\pgfsys@transformshift{0.850488in}{1.274548in}%
\pgfsys@useobject{currentmarker}{}%
\end{pgfscope}%
\begin{pgfscope}%
\pgfsys@transformshift{1.022393in}{1.176117in}%
\pgfsys@useobject{currentmarker}{}%
\end{pgfscope}%
\begin{pgfscope}%
\pgfsys@transformshift{0.886673in}{1.363614in}%
\pgfsys@useobject{currentmarker}{}%
\end{pgfscope}%
\begin{pgfscope}%
\pgfsys@transformshift{0.822616in}{1.405923in}%
\pgfsys@useobject{currentmarker}{}%
\end{pgfscope}%
\begin{pgfscope}%
\pgfsys@transformshift{0.785142in}{1.224839in}%
\pgfsys@useobject{currentmarker}{}%
\end{pgfscope}%
\begin{pgfscope}%
\pgfsys@transformshift{0.778656in}{1.370366in}%
\pgfsys@useobject{currentmarker}{}%
\end{pgfscope}%
\begin{pgfscope}%
\pgfsys@transformshift{1.062459in}{1.068750in}%
\pgfsys@useobject{currentmarker}{}%
\end{pgfscope}%
\begin{pgfscope}%
\pgfsys@transformshift{1.081408in}{1.308210in}%
\pgfsys@useobject{currentmarker}{}%
\end{pgfscope}%
\begin{pgfscope}%
\pgfsys@transformshift{0.948082in}{1.151936in}%
\pgfsys@useobject{currentmarker}{}%
\end{pgfscope}%
\begin{pgfscope}%
\pgfsys@transformshift{0.993671in}{0.855656in}%
\pgfsys@useobject{currentmarker}{}%
\end{pgfscope}%
\begin{pgfscope}%
\pgfsys@transformshift{1.124547in}{1.649849in}%
\pgfsys@useobject{currentmarker}{}%
\end{pgfscope}%
\begin{pgfscope}%
\pgfsys@transformshift{0.578610in}{1.341703in}%
\pgfsys@useobject{currentmarker}{}%
\end{pgfscope}%
\begin{pgfscope}%
\pgfsys@transformshift{0.874210in}{1.349105in}%
\pgfsys@useobject{currentmarker}{}%
\end{pgfscope}%
\begin{pgfscope}%
\pgfsys@transformshift{0.957203in}{1.309385in}%
\pgfsys@useobject{currentmarker}{}%
\end{pgfscope}%
\begin{pgfscope}%
\pgfsys@transformshift{0.919743in}{1.003754in}%
\pgfsys@useobject{currentmarker}{}%
\end{pgfscope}%
\begin{pgfscope}%
\pgfsys@transformshift{0.884535in}{1.482026in}%
\pgfsys@useobject{currentmarker}{}%
\end{pgfscope}%
\begin{pgfscope}%
\pgfsys@transformshift{0.622825in}{1.288136in}%
\pgfsys@useobject{currentmarker}{}%
\end{pgfscope}%
\begin{pgfscope}%
\pgfsys@transformshift{0.861294in}{1.111906in}%
\pgfsys@useobject{currentmarker}{}%
\end{pgfscope}%
\begin{pgfscope}%
\pgfsys@transformshift{0.781176in}{1.343641in}%
\pgfsys@useobject{currentmarker}{}%
\end{pgfscope}%
\begin{pgfscope}%
\pgfsys@transformshift{0.686627in}{1.464279in}%
\pgfsys@useobject{currentmarker}{}%
\end{pgfscope}%
\begin{pgfscope}%
\pgfsys@transformshift{0.627216in}{1.268398in}%
\pgfsys@useobject{currentmarker}{}%
\end{pgfscope}%
\begin{pgfscope}%
\pgfsys@transformshift{0.785935in}{1.173636in}%
\pgfsys@useobject{currentmarker}{}%
\end{pgfscope}%
\begin{pgfscope}%
\pgfsys@transformshift{1.299837in}{1.156803in}%
\pgfsys@useobject{currentmarker}{}%
\end{pgfscope}%
\begin{pgfscope}%
\pgfsys@transformshift{0.887013in}{1.178019in}%
\pgfsys@useobject{currentmarker}{}%
\end{pgfscope}%
\begin{pgfscope}%
\pgfsys@transformshift{0.887679in}{0.936467in}%
\pgfsys@useobject{currentmarker}{}%
\end{pgfscope}%
\begin{pgfscope}%
\pgfsys@transformshift{0.807788in}{0.851112in}%
\pgfsys@useobject{currentmarker}{}%
\end{pgfscope}%
\begin{pgfscope}%
\pgfsys@transformshift{0.632073in}{1.201924in}%
\pgfsys@useobject{currentmarker}{}%
\end{pgfscope}%
\begin{pgfscope}%
\pgfsys@transformshift{0.672975in}{1.304292in}%
\pgfsys@useobject{currentmarker}{}%
\end{pgfscope}%
\begin{pgfscope}%
\pgfsys@transformshift{0.691910in}{1.378015in}%
\pgfsys@useobject{currentmarker}{}%
\end{pgfscope}%
\begin{pgfscope}%
\pgfsys@transformshift{1.230186in}{1.294785in}%
\pgfsys@useobject{currentmarker}{}%
\end{pgfscope}%
\begin{pgfscope}%
\pgfsys@transformshift{0.946977in}{1.013176in}%
\pgfsys@useobject{currentmarker}{}%
\end{pgfscope}%
\begin{pgfscope}%
\pgfsys@transformshift{0.811300in}{1.164820in}%
\pgfsys@useobject{currentmarker}{}%
\end{pgfscope}%
\begin{pgfscope}%
\pgfsys@transformshift{1.092016in}{1.518977in}%
\pgfsys@useobject{currentmarker}{}%
\end{pgfscope}%
\begin{pgfscope}%
\pgfsys@transformshift{1.023243in}{1.187742in}%
\pgfsys@useobject{currentmarker}{}%
\end{pgfscope}%
\begin{pgfscope}%
\pgfsys@transformshift{0.825010in}{1.061111in}%
\pgfsys@useobject{currentmarker}{}%
\end{pgfscope}%
\begin{pgfscope}%
\pgfsys@transformshift{0.877892in}{0.874753in}%
\pgfsys@useobject{currentmarker}{}%
\end{pgfscope}%
\begin{pgfscope}%
\pgfsys@transformshift{1.030621in}{1.511589in}%
\pgfsys@useobject{currentmarker}{}%
\end{pgfscope}%
\begin{pgfscope}%
\pgfsys@transformshift{0.683951in}{1.487597in}%
\pgfsys@useobject{currentmarker}{}%
\end{pgfscope}%
\begin{pgfscope}%
\pgfsys@transformshift{0.677308in}{1.168098in}%
\pgfsys@useobject{currentmarker}{}%
\end{pgfscope}%
\begin{pgfscope}%
\pgfsys@transformshift{1.014207in}{1.655738in}%
\pgfsys@useobject{currentmarker}{}%
\end{pgfscope}%
\begin{pgfscope}%
\pgfsys@transformshift{0.895511in}{1.450336in}%
\pgfsys@useobject{currentmarker}{}%
\end{pgfscope}%
\begin{pgfscope}%
\pgfsys@transformshift{1.250240in}{1.021885in}%
\pgfsys@useobject{currentmarker}{}%
\end{pgfscope}%
\begin{pgfscope}%
\pgfsys@transformshift{0.935506in}{1.014354in}%
\pgfsys@useobject{currentmarker}{}%
\end{pgfscope}%
\begin{pgfscope}%
\pgfsys@transformshift{0.601935in}{1.176866in}%
\pgfsys@useobject{currentmarker}{}%
\end{pgfscope}%
\begin{pgfscope}%
\pgfsys@transformshift{1.117551in}{1.225220in}%
\pgfsys@useobject{currentmarker}{}%
\end{pgfscope}%
\begin{pgfscope}%
\pgfsys@transformshift{0.678597in}{1.527622in}%
\pgfsys@useobject{currentmarker}{}%
\end{pgfscope}%
\begin{pgfscope}%
\pgfsys@transformshift{0.708098in}{1.406927in}%
\pgfsys@useobject{currentmarker}{}%
\end{pgfscope}%
\begin{pgfscope}%
\pgfsys@transformshift{0.602403in}{1.284631in}%
\pgfsys@useobject{currentmarker}{}%
\end{pgfscope}%
\begin{pgfscope}%
\pgfsys@transformshift{0.956962in}{1.165664in}%
\pgfsys@useobject{currentmarker}{}%
\end{pgfscope}%
\begin{pgfscope}%
\pgfsys@transformshift{0.729030in}{1.210761in}%
\pgfsys@useobject{currentmarker}{}%
\end{pgfscope}%
\begin{pgfscope}%
\pgfsys@transformshift{0.641463in}{1.310534in}%
\pgfsys@useobject{currentmarker}{}%
\end{pgfscope}%
\begin{pgfscope}%
\pgfsys@transformshift{0.703821in}{1.158267in}%
\pgfsys@useobject{currentmarker}{}%
\end{pgfscope}%
\begin{pgfscope}%
\pgfsys@transformshift{0.681430in}{1.146476in}%
\pgfsys@useobject{currentmarker}{}%
\end{pgfscope}%
\begin{pgfscope}%
\pgfsys@transformshift{0.955829in}{0.894431in}%
\pgfsys@useobject{currentmarker}{}%
\end{pgfscope}%
\begin{pgfscope}%
\pgfsys@transformshift{1.428291in}{1.365872in}%
\pgfsys@useobject{currentmarker}{}%
\end{pgfscope}%
\begin{pgfscope}%
\pgfsys@transformshift{0.664066in}{1.401612in}%
\pgfsys@useobject{currentmarker}{}%
\end{pgfscope}%
\begin{pgfscope}%
\pgfsys@transformshift{0.726155in}{1.314434in}%
\pgfsys@useobject{currentmarker}{}%
\end{pgfscope}%
\begin{pgfscope}%
\pgfsys@transformshift{0.948351in}{0.838315in}%
\pgfsys@useobject{currentmarker}{}%
\end{pgfscope}%
\begin{pgfscope}%
\pgfsys@transformshift{1.115696in}{1.063283in}%
\pgfsys@useobject{currentmarker}{}%
\end{pgfscope}%
\begin{pgfscope}%
\pgfsys@transformshift{0.784349in}{1.387580in}%
\pgfsys@useobject{currentmarker}{}%
\end{pgfscope}%
\begin{pgfscope}%
\pgfsys@transformshift{0.677266in}{1.404086in}%
\pgfsys@useobject{currentmarker}{}%
\end{pgfscope}%
\begin{pgfscope}%
\pgfsys@transformshift{1.155478in}{1.146220in}%
\pgfsys@useobject{currentmarker}{}%
\end{pgfscope}%
\begin{pgfscope}%
\pgfsys@transformshift{0.882340in}{1.176377in}%
\pgfsys@useobject{currentmarker}{}%
\end{pgfscope}%
\begin{pgfscope}%
\pgfsys@transformshift{0.892055in}{0.887258in}%
\pgfsys@useobject{currentmarker}{}%
\end{pgfscope}%
\begin{pgfscope}%
\pgfsys@transformshift{0.791813in}{1.391273in}%
\pgfsys@useobject{currentmarker}{}%
\end{pgfscope}%
\begin{pgfscope}%
\pgfsys@transformshift{0.725192in}{1.282569in}%
\pgfsys@useobject{currentmarker}{}%
\end{pgfscope}%
\begin{pgfscope}%
\pgfsys@transformshift{0.852995in}{1.078535in}%
\pgfsys@useobject{currentmarker}{}%
\end{pgfscope}%
\begin{pgfscope}%
\pgfsys@transformshift{0.700436in}{1.095718in}%
\pgfsys@useobject{currentmarker}{}%
\end{pgfscope}%
\begin{pgfscope}%
\pgfsys@transformshift{0.983389in}{1.126688in}%
\pgfsys@useobject{currentmarker}{}%
\end{pgfscope}%
\begin{pgfscope}%
\pgfsys@transformshift{0.665199in}{1.410788in}%
\pgfsys@useobject{currentmarker}{}%
\end{pgfscope}%
\begin{pgfscope}%
\pgfsys@transformshift{1.461687in}{1.188237in}%
\pgfsys@useobject{currentmarker}{}%
\end{pgfscope}%
\begin{pgfscope}%
\pgfsys@transformshift{0.670893in}{1.020021in}%
\pgfsys@useobject{currentmarker}{}%
\end{pgfscope}%
\begin{pgfscope}%
\pgfsys@transformshift{0.818891in}{0.789250in}%
\pgfsys@useobject{currentmarker}{}%
\end{pgfscope}%
\begin{pgfscope}%
\pgfsys@transformshift{0.757865in}{1.295118in}%
\pgfsys@useobject{currentmarker}{}%
\end{pgfscope}%
\begin{pgfscope}%
\pgfsys@transformshift{0.966097in}{1.100068in}%
\pgfsys@useobject{currentmarker}{}%
\end{pgfscope}%
\begin{pgfscope}%
\pgfsys@transformshift{0.770625in}{1.411843in}%
\pgfsys@useobject{currentmarker}{}%
\end{pgfscope}%
\begin{pgfscope}%
\pgfsys@transformshift{0.649125in}{1.219988in}%
\pgfsys@useobject{currentmarker}{}%
\end{pgfscope}%
\begin{pgfscope}%
\pgfsys@transformshift{1.157447in}{1.307672in}%
\pgfsys@useobject{currentmarker}{}%
\end{pgfscope}%
\begin{pgfscope}%
\pgfsys@transformshift{0.785624in}{1.489155in}%
\pgfsys@useobject{currentmarker}{}%
\end{pgfscope}%
\begin{pgfscope}%
\pgfsys@transformshift{1.032377in}{1.060985in}%
\pgfsys@useobject{currentmarker}{}%
\end{pgfscope}%
\begin{pgfscope}%
\pgfsys@transformshift{0.874819in}{0.523238in}%
\pgfsys@useobject{currentmarker}{}%
\end{pgfscope}%
\begin{pgfscope}%
\pgfsys@transformshift{0.708820in}{1.237115in}%
\pgfsys@useobject{currentmarker}{}%
\end{pgfscope}%
\begin{pgfscope}%
\pgfsys@transformshift{0.664661in}{1.337701in}%
\pgfsys@useobject{currentmarker}{}%
\end{pgfscope}%
\begin{pgfscope}%
\pgfsys@transformshift{0.580097in}{1.284373in}%
\pgfsys@useobject{currentmarker}{}%
\end{pgfscope}%
\begin{pgfscope}%
\pgfsys@transformshift{0.930223in}{1.392479in}%
\pgfsys@useobject{currentmarker}{}%
\end{pgfscope}%
\begin{pgfscope}%
\pgfsys@transformshift{1.176056in}{1.084210in}%
\pgfsys@useobject{currentmarker}{}%
\end{pgfscope}%
\begin{pgfscope}%
\pgfsys@transformshift{0.790680in}{1.195655in}%
\pgfsys@useobject{currentmarker}{}%
\end{pgfscope}%
\begin{pgfscope}%
\pgfsys@transformshift{0.824995in}{1.155494in}%
\pgfsys@useobject{currentmarker}{}%
\end{pgfscope}%
\begin{pgfscope}%
\pgfsys@transformshift{0.903994in}{1.311609in}%
\pgfsys@useobject{currentmarker}{}%
\end{pgfscope}%
\begin{pgfscope}%
\pgfsys@transformshift{0.956750in}{1.384434in}%
\pgfsys@useobject{currentmarker}{}%
\end{pgfscope}%
\begin{pgfscope}%
\pgfsys@transformshift{0.556205in}{1.202212in}%
\pgfsys@useobject{currentmarker}{}%
\end{pgfscope}%
\begin{pgfscope}%
\pgfsys@transformshift{0.637710in}{1.186090in}%
\pgfsys@useobject{currentmarker}{}%
\end{pgfscope}%
\begin{pgfscope}%
\pgfsys@transformshift{1.014292in}{1.425238in}%
\pgfsys@useobject{currentmarker}{}%
\end{pgfscope}%
\begin{pgfscope}%
\pgfsys@transformshift{0.617330in}{1.347879in}%
\pgfsys@useobject{currentmarker}{}%
\end{pgfscope}%
\begin{pgfscope}%
\pgfsys@transformshift{0.935817in}{1.408365in}%
\pgfsys@useobject{currentmarker}{}%
\end{pgfscope}%
\begin{pgfscope}%
\pgfsys@transformshift{0.873969in}{1.200126in}%
\pgfsys@useobject{currentmarker}{}%
\end{pgfscope}%
\begin{pgfscope}%
\pgfsys@transformshift{1.163622in}{1.216507in}%
\pgfsys@useobject{currentmarker}{}%
\end{pgfscope}%
\begin{pgfscope}%
\pgfsys@transformshift{0.563413in}{1.237808in}%
\pgfsys@useobject{currentmarker}{}%
\end{pgfscope}%
\begin{pgfscope}%
\pgfsys@transformshift{0.702702in}{1.225714in}%
\pgfsys@useobject{currentmarker}{}%
\end{pgfscope}%
\begin{pgfscope}%
\pgfsys@transformshift{0.621154in}{1.176231in}%
\pgfsys@useobject{currentmarker}{}%
\end{pgfscope}%
\begin{pgfscope}%
\pgfsys@transformshift{0.554703in}{1.401982in}%
\pgfsys@useobject{currentmarker}{}%
\end{pgfscope}%
\begin{pgfscope}%
\pgfsys@transformshift{1.266569in}{1.271818in}%
\pgfsys@useobject{currentmarker}{}%
\end{pgfscope}%
\begin{pgfscope}%
\pgfsys@transformshift{1.649000in}{1.348223in}%
\pgfsys@useobject{currentmarker}{}%
\end{pgfscope}%
\begin{pgfscope}%
\pgfsys@transformshift{0.922731in}{0.968807in}%
\pgfsys@useobject{currentmarker}{}%
\end{pgfscope}%
\begin{pgfscope}%
\pgfsys@transformshift{1.194878in}{1.131868in}%
\pgfsys@useobject{currentmarker}{}%
\end{pgfscope}%
\begin{pgfscope}%
\pgfsys@transformshift{0.638361in}{0.951753in}%
\pgfsys@useobject{currentmarker}{}%
\end{pgfscope}%
\begin{pgfscope}%
\pgfsys@transformshift{0.879351in}{0.936950in}%
\pgfsys@useobject{currentmarker}{}%
\end{pgfscope}%
\begin{pgfscope}%
\pgfsys@transformshift{0.681416in}{1.291154in}%
\pgfsys@useobject{currentmarker}{}%
\end{pgfscope}%
\begin{pgfscope}%
\pgfsys@transformshift{0.680070in}{0.960727in}%
\pgfsys@useobject{currentmarker}{}%
\end{pgfscope}%
\begin{pgfscope}%
\pgfsys@transformshift{1.006276in}{1.515300in}%
\pgfsys@useobject{currentmarker}{}%
\end{pgfscope}%
\begin{pgfscope}%
\pgfsys@transformshift{0.981945in}{1.103489in}%
\pgfsys@useobject{currentmarker}{}%
\end{pgfscope}%
\begin{pgfscope}%
\pgfsys@transformshift{0.593325in}{1.285542in}%
\pgfsys@useobject{currentmarker}{}%
\end{pgfscope}%
\begin{pgfscope}%
\pgfsys@transformshift{0.691145in}{1.231537in}%
\pgfsys@useobject{currentmarker}{}%
\end{pgfscope}%
\begin{pgfscope}%
\pgfsys@transformshift{0.609682in}{1.472655in}%
\pgfsys@useobject{currentmarker}{}%
\end{pgfscope}%
\begin{pgfscope}%
\pgfsys@transformshift{0.672210in}{1.156932in}%
\pgfsys@useobject{currentmarker}{}%
\end{pgfscope}%
\begin{pgfscope}%
\pgfsys@transformshift{0.601978in}{1.483636in}%
\pgfsys@useobject{currentmarker}{}%
\end{pgfscope}%
\begin{pgfscope}%
\pgfsys@transformshift{0.921556in}{1.412286in}%
\pgfsys@useobject{currentmarker}{}%
\end{pgfscope}%
\begin{pgfscope}%
\pgfsys@transformshift{0.890795in}{0.861459in}%
\pgfsys@useobject{currentmarker}{}%
\end{pgfscope}%
\begin{pgfscope}%
\pgfsys@transformshift{0.655951in}{1.380185in}%
\pgfsys@useobject{currentmarker}{}%
\end{pgfscope}%
\begin{pgfscope}%
\pgfsys@transformshift{1.242762in}{1.200767in}%
\pgfsys@useobject{currentmarker}{}%
\end{pgfscope}%
\begin{pgfscope}%
\pgfsys@transformshift{0.730276in}{1.325788in}%
\pgfsys@useobject{currentmarker}{}%
\end{pgfscope}%
\begin{pgfscope}%
\pgfsys@transformshift{1.090855in}{1.218736in}%
\pgfsys@useobject{currentmarker}{}%
\end{pgfscope}%
\begin{pgfscope}%
\pgfsys@transformshift{0.750628in}{1.382496in}%
\pgfsys@useobject{currentmarker}{}%
\end{pgfscope}%
\begin{pgfscope}%
\pgfsys@transformshift{0.614625in}{1.063014in}%
\pgfsys@useobject{currentmarker}{}%
\end{pgfscope}%
\begin{pgfscope}%
\pgfsys@transformshift{0.704486in}{1.389382in}%
\pgfsys@useobject{currentmarker}{}%
\end{pgfscope}%
\begin{pgfscope}%
\pgfsys@transformshift{1.084142in}{1.156666in}%
\pgfsys@useobject{currentmarker}{}%
\end{pgfscope}%
\begin{pgfscope}%
\pgfsys@transformshift{0.904079in}{1.301309in}%
\pgfsys@useobject{currentmarker}{}%
\end{pgfscope}%
\begin{pgfscope}%
\pgfsys@transformshift{0.675906in}{1.130299in}%
\pgfsys@useobject{currentmarker}{}%
\end{pgfscope}%
\begin{pgfscope}%
\pgfsys@transformshift{0.701909in}{1.259913in}%
\pgfsys@useobject{currentmarker}{}%
\end{pgfscope}%
\begin{pgfscope}%
\pgfsys@transformshift{0.556757in}{0.886468in}%
\pgfsys@useobject{currentmarker}{}%
\end{pgfscope}%
\begin{pgfscope}%
\pgfsys@transformshift{0.905779in}{1.489375in}%
\pgfsys@useobject{currentmarker}{}%
\end{pgfscope}%
\begin{pgfscope}%
\pgfsys@transformshift{0.693822in}{1.281800in}%
\pgfsys@useobject{currentmarker}{}%
\end{pgfscope}%
\begin{pgfscope}%
\pgfsys@transformshift{1.246813in}{1.101574in}%
\pgfsys@useobject{currentmarker}{}%
\end{pgfscope}%
\begin{pgfscope}%
\pgfsys@transformshift{0.614937in}{1.379223in}%
\pgfsys@useobject{currentmarker}{}%
\end{pgfscope}%
\begin{pgfscope}%
\pgfsys@transformshift{0.903881in}{0.868432in}%
\pgfsys@useobject{currentmarker}{}%
\end{pgfscope}%
\end{pgfscope}%
\begin{pgfscope}%
\pgfpathrectangle{\pgfqpoint{0.519339in}{0.466613in}}{\pgfqpoint{1.278750in}{1.245750in}}%
\pgfusepath{clip}%
\pgfsetbuttcap%
\pgfsetroundjoin%
\definecolor{currentfill}{rgb}{0.298039,0.447059,0.690196}%
\pgfsetfillcolor{currentfill}%
\pgfsetfillopacity{0.150000}%
\pgfsetlinewidth{1.003750pt}%
\definecolor{currentstroke}{rgb}{1.000000,1.000000,1.000000}%
\pgfsetstrokecolor{currentstroke}%
\pgfsetstrokeopacity{0.150000}%
\pgfsetdash{}{0pt}%
\pgfsys@defobject{currentmarker}{\pgfqpoint{0.519339in}{1.149793in}}{\pgfqpoint{1.798089in}{1.263231in}}{%
\pgfpathmoveto{\pgfqpoint{0.519339in}{1.257340in}}%
\pgfpathlineto{\pgfqpoint{0.519339in}{1.208710in}}%
\pgfpathlineto{\pgfqpoint{0.532256in}{1.208887in}}%
\pgfpathlineto{\pgfqpoint{0.545173in}{1.209226in}}%
\pgfpathlineto{\pgfqpoint{0.558089in}{1.209572in}}%
\pgfpathlineto{\pgfqpoint{0.571006in}{1.209929in}}%
\pgfpathlineto{\pgfqpoint{0.583923in}{1.210184in}}%
\pgfpathlineto{\pgfqpoint{0.596839in}{1.210409in}}%
\pgfpathlineto{\pgfqpoint{0.609756in}{1.210516in}}%
\pgfpathlineto{\pgfqpoint{0.622673in}{1.210790in}}%
\pgfpathlineto{\pgfqpoint{0.635589in}{1.210944in}}%
\pgfpathlineto{\pgfqpoint{0.648506in}{1.211069in}}%
\pgfpathlineto{\pgfqpoint{0.661423in}{1.211119in}}%
\pgfpathlineto{\pgfqpoint{0.674339in}{1.211444in}}%
\pgfpathlineto{\pgfqpoint{0.687256in}{1.211841in}}%
\pgfpathlineto{\pgfqpoint{0.700173in}{1.211904in}}%
\pgfpathlineto{\pgfqpoint{0.713089in}{1.212070in}}%
\pgfpathlineto{\pgfqpoint{0.726006in}{1.212121in}}%
\pgfpathlineto{\pgfqpoint{0.738923in}{1.212335in}}%
\pgfpathlineto{\pgfqpoint{0.751839in}{1.212118in}}%
\pgfpathlineto{\pgfqpoint{0.764756in}{1.211918in}}%
\pgfpathlineto{\pgfqpoint{0.777673in}{1.211658in}}%
\pgfpathlineto{\pgfqpoint{0.790589in}{1.211376in}}%
\pgfpathlineto{\pgfqpoint{0.803506in}{1.211122in}}%
\pgfpathlineto{\pgfqpoint{0.816423in}{1.210989in}}%
\pgfpathlineto{\pgfqpoint{0.829339in}{1.210489in}}%
\pgfpathlineto{\pgfqpoint{0.842256in}{1.210195in}}%
\pgfpathlineto{\pgfqpoint{0.855173in}{1.209846in}}%
\pgfpathlineto{\pgfqpoint{0.868089in}{1.209396in}}%
\pgfpathlineto{\pgfqpoint{0.881006in}{1.209079in}}%
\pgfpathlineto{\pgfqpoint{0.893923in}{1.208939in}}%
\pgfpathlineto{\pgfqpoint{0.906839in}{1.208424in}}%
\pgfpathlineto{\pgfqpoint{0.919756in}{1.207970in}}%
\pgfpathlineto{\pgfqpoint{0.932673in}{1.207473in}}%
\pgfpathlineto{\pgfqpoint{0.945589in}{1.207007in}}%
\pgfpathlineto{\pgfqpoint{0.958506in}{1.206749in}}%
\pgfpathlineto{\pgfqpoint{0.971423in}{1.206512in}}%
\pgfpathlineto{\pgfqpoint{0.984339in}{1.206039in}}%
\pgfpathlineto{\pgfqpoint{0.997256in}{1.205428in}}%
\pgfpathlineto{\pgfqpoint{1.010173in}{1.204748in}}%
\pgfpathlineto{\pgfqpoint{1.023089in}{1.203871in}}%
\pgfpathlineto{\pgfqpoint{1.036006in}{1.203144in}}%
\pgfpathlineto{\pgfqpoint{1.048923in}{1.202308in}}%
\pgfpathlineto{\pgfqpoint{1.061839in}{1.201467in}}%
\pgfpathlineto{\pgfqpoint{1.074756in}{1.200659in}}%
\pgfpathlineto{\pgfqpoint{1.087673in}{1.199901in}}%
\pgfpathlineto{\pgfqpoint{1.100589in}{1.199270in}}%
\pgfpathlineto{\pgfqpoint{1.113506in}{1.198387in}}%
\pgfpathlineto{\pgfqpoint{1.126423in}{1.197629in}}%
\pgfpathlineto{\pgfqpoint{1.139339in}{1.196717in}}%
\pgfpathlineto{\pgfqpoint{1.152256in}{1.195959in}}%
\pgfpathlineto{\pgfqpoint{1.165173in}{1.195171in}}%
\pgfpathlineto{\pgfqpoint{1.178089in}{1.194274in}}%
\pgfpathlineto{\pgfqpoint{1.191006in}{1.193383in}}%
\pgfpathlineto{\pgfqpoint{1.203923in}{1.192527in}}%
\pgfpathlineto{\pgfqpoint{1.216839in}{1.191708in}}%
\pgfpathlineto{\pgfqpoint{1.229756in}{1.190814in}}%
\pgfpathlineto{\pgfqpoint{1.242673in}{1.190037in}}%
\pgfpathlineto{\pgfqpoint{1.255589in}{1.189381in}}%
\pgfpathlineto{\pgfqpoint{1.268506in}{1.188251in}}%
\pgfpathlineto{\pgfqpoint{1.281423in}{1.187291in}}%
\pgfpathlineto{\pgfqpoint{1.294339in}{1.186455in}}%
\pgfpathlineto{\pgfqpoint{1.307256in}{1.185527in}}%
\pgfpathlineto{\pgfqpoint{1.320173in}{1.184550in}}%
\pgfpathlineto{\pgfqpoint{1.333089in}{1.183573in}}%
\pgfpathlineto{\pgfqpoint{1.346006in}{1.182758in}}%
\pgfpathlineto{\pgfqpoint{1.358923in}{1.181711in}}%
\pgfpathlineto{\pgfqpoint{1.371839in}{1.180809in}}%
\pgfpathlineto{\pgfqpoint{1.384756in}{1.179915in}}%
\pgfpathlineto{\pgfqpoint{1.397673in}{1.179152in}}%
\pgfpathlineto{\pgfqpoint{1.410589in}{1.178116in}}%
\pgfpathlineto{\pgfqpoint{1.423506in}{1.177211in}}%
\pgfpathlineto{\pgfqpoint{1.436423in}{1.176310in}}%
\pgfpathlineto{\pgfqpoint{1.449339in}{1.175413in}}%
\pgfpathlineto{\pgfqpoint{1.462256in}{1.174516in}}%
\pgfpathlineto{\pgfqpoint{1.475173in}{1.173618in}}%
\pgfpathlineto{\pgfqpoint{1.488089in}{1.172721in}}%
\pgfpathlineto{\pgfqpoint{1.501006in}{1.171798in}}%
\pgfpathlineto{\pgfqpoint{1.513923in}{1.170838in}}%
\pgfpathlineto{\pgfqpoint{1.526839in}{1.169878in}}%
\pgfpathlineto{\pgfqpoint{1.539756in}{1.168918in}}%
\pgfpathlineto{\pgfqpoint{1.552673in}{1.167958in}}%
\pgfpathlineto{\pgfqpoint{1.565589in}{1.167008in}}%
\pgfpathlineto{\pgfqpoint{1.578506in}{1.166128in}}%
\pgfpathlineto{\pgfqpoint{1.591423in}{1.165191in}}%
\pgfpathlineto{\pgfqpoint{1.604339in}{1.164234in}}%
\pgfpathlineto{\pgfqpoint{1.617256in}{1.163274in}}%
\pgfpathlineto{\pgfqpoint{1.630173in}{1.162315in}}%
\pgfpathlineto{\pgfqpoint{1.643089in}{1.161358in}}%
\pgfpathlineto{\pgfqpoint{1.656006in}{1.160506in}}%
\pgfpathlineto{\pgfqpoint{1.668923in}{1.159677in}}%
\pgfpathlineto{\pgfqpoint{1.681839in}{1.158829in}}%
\pgfpathlineto{\pgfqpoint{1.694756in}{1.157785in}}%
\pgfpathlineto{\pgfqpoint{1.707673in}{1.156647in}}%
\pgfpathlineto{\pgfqpoint{1.720589in}{1.155606in}}%
\pgfpathlineto{\pgfqpoint{1.733506in}{1.154647in}}%
\pgfpathlineto{\pgfqpoint{1.746423in}{1.153607in}}%
\pgfpathlineto{\pgfqpoint{1.759339in}{1.152563in}}%
\pgfpathlineto{\pgfqpoint{1.772256in}{1.151599in}}%
\pgfpathlineto{\pgfqpoint{1.785173in}{1.150666in}}%
\pgfpathlineto{\pgfqpoint{1.798089in}{1.149793in}}%
\pgfpathlineto{\pgfqpoint{1.798089in}{1.263231in}}%
\pgfpathlineto{\pgfqpoint{1.798089in}{1.263231in}}%
\pgfpathlineto{\pgfqpoint{1.785173in}{1.262853in}}%
\pgfpathlineto{\pgfqpoint{1.772256in}{1.262475in}}%
\pgfpathlineto{\pgfqpoint{1.759339in}{1.262097in}}%
\pgfpathlineto{\pgfqpoint{1.746423in}{1.261719in}}%
\pgfpathlineto{\pgfqpoint{1.733506in}{1.261347in}}%
\pgfpathlineto{\pgfqpoint{1.720589in}{1.261059in}}%
\pgfpathlineto{\pgfqpoint{1.707673in}{1.260736in}}%
\pgfpathlineto{\pgfqpoint{1.694756in}{1.260375in}}%
\pgfpathlineto{\pgfqpoint{1.681839in}{1.260005in}}%
\pgfpathlineto{\pgfqpoint{1.668923in}{1.259635in}}%
\pgfpathlineto{\pgfqpoint{1.656006in}{1.259265in}}%
\pgfpathlineto{\pgfqpoint{1.643089in}{1.258890in}}%
\pgfpathlineto{\pgfqpoint{1.630173in}{1.258514in}}%
\pgfpathlineto{\pgfqpoint{1.617256in}{1.258138in}}%
\pgfpathlineto{\pgfqpoint{1.604339in}{1.257688in}}%
\pgfpathlineto{\pgfqpoint{1.591423in}{1.257190in}}%
\pgfpathlineto{\pgfqpoint{1.578506in}{1.256696in}}%
\pgfpathlineto{\pgfqpoint{1.565589in}{1.256203in}}%
\pgfpathlineto{\pgfqpoint{1.552673in}{1.255710in}}%
\pgfpathlineto{\pgfqpoint{1.539756in}{1.255216in}}%
\pgfpathlineto{\pgfqpoint{1.526839in}{1.254723in}}%
\pgfpathlineto{\pgfqpoint{1.513923in}{1.254230in}}%
\pgfpathlineto{\pgfqpoint{1.501006in}{1.253920in}}%
\pgfpathlineto{\pgfqpoint{1.488089in}{1.253697in}}%
\pgfpathlineto{\pgfqpoint{1.475173in}{1.253475in}}%
\pgfpathlineto{\pgfqpoint{1.462256in}{1.253252in}}%
\pgfpathlineto{\pgfqpoint{1.449339in}{1.253023in}}%
\pgfpathlineto{\pgfqpoint{1.436423in}{1.252646in}}%
\pgfpathlineto{\pgfqpoint{1.423506in}{1.252093in}}%
\pgfpathlineto{\pgfqpoint{1.410589in}{1.251501in}}%
\pgfpathlineto{\pgfqpoint{1.397673in}{1.251262in}}%
\pgfpathlineto{\pgfqpoint{1.384756in}{1.250814in}}%
\pgfpathlineto{\pgfqpoint{1.371839in}{1.250529in}}%
\pgfpathlineto{\pgfqpoint{1.358923in}{1.250245in}}%
\pgfpathlineto{\pgfqpoint{1.346006in}{1.249961in}}%
\pgfpathlineto{\pgfqpoint{1.333089in}{1.249448in}}%
\pgfpathlineto{\pgfqpoint{1.320173in}{1.249193in}}%
\pgfpathlineto{\pgfqpoint{1.307256in}{1.248807in}}%
\pgfpathlineto{\pgfqpoint{1.294339in}{1.248229in}}%
\pgfpathlineto{\pgfqpoint{1.281423in}{1.247639in}}%
\pgfpathlineto{\pgfqpoint{1.268506in}{1.247048in}}%
\pgfpathlineto{\pgfqpoint{1.255589in}{1.246458in}}%
\pgfpathlineto{\pgfqpoint{1.242673in}{1.245941in}}%
\pgfpathlineto{\pgfqpoint{1.229756in}{1.245801in}}%
\pgfpathlineto{\pgfqpoint{1.216839in}{1.245660in}}%
\pgfpathlineto{\pgfqpoint{1.203923in}{1.245760in}}%
\pgfpathlineto{\pgfqpoint{1.191006in}{1.245473in}}%
\pgfpathlineto{\pgfqpoint{1.178089in}{1.245246in}}%
\pgfpathlineto{\pgfqpoint{1.165173in}{1.245085in}}%
\pgfpathlineto{\pgfqpoint{1.152256in}{1.244718in}}%
\pgfpathlineto{\pgfqpoint{1.139339in}{1.244347in}}%
\pgfpathlineto{\pgfqpoint{1.126423in}{1.243969in}}%
\pgfpathlineto{\pgfqpoint{1.113506in}{1.243374in}}%
\pgfpathlineto{\pgfqpoint{1.100589in}{1.243221in}}%
\pgfpathlineto{\pgfqpoint{1.087673in}{1.242851in}}%
\pgfpathlineto{\pgfqpoint{1.074756in}{1.242575in}}%
\pgfpathlineto{\pgfqpoint{1.061839in}{1.242542in}}%
\pgfpathlineto{\pgfqpoint{1.048923in}{1.242394in}}%
\pgfpathlineto{\pgfqpoint{1.036006in}{1.242119in}}%
\pgfpathlineto{\pgfqpoint{1.023089in}{1.241930in}}%
\pgfpathlineto{\pgfqpoint{1.010173in}{1.241926in}}%
\pgfpathlineto{\pgfqpoint{0.997256in}{1.241910in}}%
\pgfpathlineto{\pgfqpoint{0.984339in}{1.241755in}}%
\pgfpathlineto{\pgfqpoint{0.971423in}{1.241561in}}%
\pgfpathlineto{\pgfqpoint{0.958506in}{1.241488in}}%
\pgfpathlineto{\pgfqpoint{0.945589in}{1.241629in}}%
\pgfpathlineto{\pgfqpoint{0.932673in}{1.241725in}}%
\pgfpathlineto{\pgfqpoint{0.919756in}{1.241861in}}%
\pgfpathlineto{\pgfqpoint{0.906839in}{1.242101in}}%
\pgfpathlineto{\pgfqpoint{0.893923in}{1.242522in}}%
\pgfpathlineto{\pgfqpoint{0.881006in}{1.242494in}}%
\pgfpathlineto{\pgfqpoint{0.868089in}{1.242183in}}%
\pgfpathlineto{\pgfqpoint{0.855173in}{1.242265in}}%
\pgfpathlineto{\pgfqpoint{0.842256in}{1.242612in}}%
\pgfpathlineto{\pgfqpoint{0.829339in}{1.242536in}}%
\pgfpathlineto{\pgfqpoint{0.816423in}{1.242699in}}%
\pgfpathlineto{\pgfqpoint{0.803506in}{1.243522in}}%
\pgfpathlineto{\pgfqpoint{0.790589in}{1.244174in}}%
\pgfpathlineto{\pgfqpoint{0.777673in}{1.244265in}}%
\pgfpathlineto{\pgfqpoint{0.764756in}{1.244725in}}%
\pgfpathlineto{\pgfqpoint{0.751839in}{1.245739in}}%
\pgfpathlineto{\pgfqpoint{0.738923in}{1.246088in}}%
\pgfpathlineto{\pgfqpoint{0.726006in}{1.246849in}}%
\pgfpathlineto{\pgfqpoint{0.713089in}{1.247458in}}%
\pgfpathlineto{\pgfqpoint{0.700173in}{1.248317in}}%
\pgfpathlineto{\pgfqpoint{0.687256in}{1.248833in}}%
\pgfpathlineto{\pgfqpoint{0.674339in}{1.249513in}}%
\pgfpathlineto{\pgfqpoint{0.661423in}{1.250094in}}%
\pgfpathlineto{\pgfqpoint{0.648506in}{1.250493in}}%
\pgfpathlineto{\pgfqpoint{0.635589in}{1.251140in}}%
\pgfpathlineto{\pgfqpoint{0.622673in}{1.251883in}}%
\pgfpathlineto{\pgfqpoint{0.609756in}{1.252604in}}%
\pgfpathlineto{\pgfqpoint{0.596839in}{1.253333in}}%
\pgfpathlineto{\pgfqpoint{0.583923in}{1.253906in}}%
\pgfpathlineto{\pgfqpoint{0.571006in}{1.254551in}}%
\pgfpathlineto{\pgfqpoint{0.558089in}{1.255079in}}%
\pgfpathlineto{\pgfqpoint{0.545173in}{1.255516in}}%
\pgfpathlineto{\pgfqpoint{0.532256in}{1.256434in}}%
\pgfpathlineto{\pgfqpoint{0.519339in}{1.257340in}}%
\pgfpathclose%
\pgfusepath{stroke,fill}%
}%
\begin{pgfscope}%
\pgfsys@transformshift{0.000000in}{0.000000in}%
\pgfsys@useobject{currentmarker}{}%
\end{pgfscope}%
\end{pgfscope}%
\begin{pgfscope}%
\pgfpathrectangle{\pgfqpoint{0.519339in}{0.466613in}}{\pgfqpoint{1.278750in}{1.245750in}}%
\pgfusepath{clip}%
\pgfsetroundcap%
\pgfsetroundjoin%
\pgfsetlinewidth{1.505625pt}%
\definecolor{currentstroke}{rgb}{0.298039,0.447059,0.690196}%
\pgfsetstrokecolor{currentstroke}%
\pgfsetdash{}{0pt}%
\pgfpathmoveto{\pgfqpoint{0.519339in}{1.232954in}}%
\pgfpathlineto{\pgfqpoint{0.532256in}{1.232694in}}%
\pgfpathlineto{\pgfqpoint{0.545173in}{1.232433in}}%
\pgfpathlineto{\pgfqpoint{0.558089in}{1.232173in}}%
\pgfpathlineto{\pgfqpoint{0.571006in}{1.231913in}}%
\pgfpathlineto{\pgfqpoint{0.583923in}{1.231653in}}%
\pgfpathlineto{\pgfqpoint{0.596839in}{1.231393in}}%
\pgfpathlineto{\pgfqpoint{0.609756in}{1.231133in}}%
\pgfpathlineto{\pgfqpoint{0.622673in}{1.230873in}}%
\pgfpathlineto{\pgfqpoint{0.635589in}{1.230613in}}%
\pgfpathlineto{\pgfqpoint{0.648506in}{1.230353in}}%
\pgfpathlineto{\pgfqpoint{0.661423in}{1.230093in}}%
\pgfpathlineto{\pgfqpoint{0.674339in}{1.229833in}}%
\pgfpathlineto{\pgfqpoint{0.687256in}{1.229573in}}%
\pgfpathlineto{\pgfqpoint{0.700173in}{1.229313in}}%
\pgfpathlineto{\pgfqpoint{0.713089in}{1.229053in}}%
\pgfpathlineto{\pgfqpoint{0.726006in}{1.228793in}}%
\pgfpathlineto{\pgfqpoint{0.738923in}{1.228533in}}%
\pgfpathlineto{\pgfqpoint{0.751839in}{1.228273in}}%
\pgfpathlineto{\pgfqpoint{0.764756in}{1.228012in}}%
\pgfpathlineto{\pgfqpoint{0.777673in}{1.227752in}}%
\pgfpathlineto{\pgfqpoint{0.790589in}{1.227492in}}%
\pgfpathlineto{\pgfqpoint{0.803506in}{1.227232in}}%
\pgfpathlineto{\pgfqpoint{0.816423in}{1.226972in}}%
\pgfpathlineto{\pgfqpoint{0.829339in}{1.226712in}}%
\pgfpathlineto{\pgfqpoint{0.842256in}{1.226452in}}%
\pgfpathlineto{\pgfqpoint{0.855173in}{1.226192in}}%
\pgfpathlineto{\pgfqpoint{0.868089in}{1.225932in}}%
\pgfpathlineto{\pgfqpoint{0.881006in}{1.225672in}}%
\pgfpathlineto{\pgfqpoint{0.893923in}{1.225412in}}%
\pgfpathlineto{\pgfqpoint{0.906839in}{1.225152in}}%
\pgfpathlineto{\pgfqpoint{0.919756in}{1.224892in}}%
\pgfpathlineto{\pgfqpoint{0.932673in}{1.224632in}}%
\pgfpathlineto{\pgfqpoint{0.945589in}{1.224372in}}%
\pgfpathlineto{\pgfqpoint{0.958506in}{1.224112in}}%
\pgfpathlineto{\pgfqpoint{0.971423in}{1.223852in}}%
\pgfpathlineto{\pgfqpoint{0.984339in}{1.223591in}}%
\pgfpathlineto{\pgfqpoint{0.997256in}{1.223331in}}%
\pgfpathlineto{\pgfqpoint{1.010173in}{1.223071in}}%
\pgfpathlineto{\pgfqpoint{1.023089in}{1.222811in}}%
\pgfpathlineto{\pgfqpoint{1.036006in}{1.222551in}}%
\pgfpathlineto{\pgfqpoint{1.048923in}{1.222291in}}%
\pgfpathlineto{\pgfqpoint{1.061839in}{1.222031in}}%
\pgfpathlineto{\pgfqpoint{1.074756in}{1.221771in}}%
\pgfpathlineto{\pgfqpoint{1.087673in}{1.221511in}}%
\pgfpathlineto{\pgfqpoint{1.100589in}{1.221251in}}%
\pgfpathlineto{\pgfqpoint{1.113506in}{1.220991in}}%
\pgfpathlineto{\pgfqpoint{1.126423in}{1.220731in}}%
\pgfpathlineto{\pgfqpoint{1.139339in}{1.220471in}}%
\pgfpathlineto{\pgfqpoint{1.152256in}{1.220211in}}%
\pgfpathlineto{\pgfqpoint{1.165173in}{1.219951in}}%
\pgfpathlineto{\pgfqpoint{1.178089in}{1.219691in}}%
\pgfpathlineto{\pgfqpoint{1.191006in}{1.219431in}}%
\pgfpathlineto{\pgfqpoint{1.203923in}{1.219170in}}%
\pgfpathlineto{\pgfqpoint{1.216839in}{1.218910in}}%
\pgfpathlineto{\pgfqpoint{1.229756in}{1.218650in}}%
\pgfpathlineto{\pgfqpoint{1.242673in}{1.218390in}}%
\pgfpathlineto{\pgfqpoint{1.255589in}{1.218130in}}%
\pgfpathlineto{\pgfqpoint{1.268506in}{1.217870in}}%
\pgfpathlineto{\pgfqpoint{1.281423in}{1.217610in}}%
\pgfpathlineto{\pgfqpoint{1.294339in}{1.217350in}}%
\pgfpathlineto{\pgfqpoint{1.307256in}{1.217090in}}%
\pgfpathlineto{\pgfqpoint{1.320173in}{1.216830in}}%
\pgfpathlineto{\pgfqpoint{1.333089in}{1.216570in}}%
\pgfpathlineto{\pgfqpoint{1.346006in}{1.216310in}}%
\pgfpathlineto{\pgfqpoint{1.358923in}{1.216050in}}%
\pgfpathlineto{\pgfqpoint{1.371839in}{1.215790in}}%
\pgfpathlineto{\pgfqpoint{1.384756in}{1.215530in}}%
\pgfpathlineto{\pgfqpoint{1.397673in}{1.215270in}}%
\pgfpathlineto{\pgfqpoint{1.410589in}{1.215010in}}%
\pgfpathlineto{\pgfqpoint{1.423506in}{1.214749in}}%
\pgfpathlineto{\pgfqpoint{1.436423in}{1.214489in}}%
\pgfpathlineto{\pgfqpoint{1.449339in}{1.214229in}}%
\pgfpathlineto{\pgfqpoint{1.462256in}{1.213969in}}%
\pgfpathlineto{\pgfqpoint{1.475173in}{1.213709in}}%
\pgfpathlineto{\pgfqpoint{1.488089in}{1.213449in}}%
\pgfpathlineto{\pgfqpoint{1.501006in}{1.213189in}}%
\pgfpathlineto{\pgfqpoint{1.513923in}{1.212929in}}%
\pgfpathlineto{\pgfqpoint{1.526839in}{1.212669in}}%
\pgfpathlineto{\pgfqpoint{1.539756in}{1.212409in}}%
\pgfpathlineto{\pgfqpoint{1.552673in}{1.212149in}}%
\pgfpathlineto{\pgfqpoint{1.565589in}{1.211889in}}%
\pgfpathlineto{\pgfqpoint{1.578506in}{1.211629in}}%
\pgfpathlineto{\pgfqpoint{1.591423in}{1.211369in}}%
\pgfpathlineto{\pgfqpoint{1.604339in}{1.211109in}}%
\pgfpathlineto{\pgfqpoint{1.617256in}{1.210849in}}%
\pgfpathlineto{\pgfqpoint{1.630173in}{1.210589in}}%
\pgfpathlineto{\pgfqpoint{1.643089in}{1.210328in}}%
\pgfpathlineto{\pgfqpoint{1.656006in}{1.210068in}}%
\pgfpathlineto{\pgfqpoint{1.668923in}{1.209808in}}%
\pgfpathlineto{\pgfqpoint{1.681839in}{1.209548in}}%
\pgfpathlineto{\pgfqpoint{1.694756in}{1.209288in}}%
\pgfpathlineto{\pgfqpoint{1.707673in}{1.209028in}}%
\pgfpathlineto{\pgfqpoint{1.720589in}{1.208768in}}%
\pgfpathlineto{\pgfqpoint{1.733506in}{1.208508in}}%
\pgfpathlineto{\pgfqpoint{1.746423in}{1.208248in}}%
\pgfpathlineto{\pgfqpoint{1.759339in}{1.207988in}}%
\pgfpathlineto{\pgfqpoint{1.772256in}{1.207728in}}%
\pgfpathlineto{\pgfqpoint{1.785173in}{1.207468in}}%
\pgfpathlineto{\pgfqpoint{1.798089in}{1.207208in}}%
\pgfusepath{stroke}%
\end{pgfscope}%
\begin{pgfscope}%
\pgfsetrectcap%
\pgfsetmiterjoin%
\pgfsetlinewidth{0.752812pt}%
\definecolor{currentstroke}{rgb}{0.700000,0.700000,0.700000}%
\pgfsetstrokecolor{currentstroke}%
\pgfsetdash{}{0pt}%
\pgfpathmoveto{\pgfqpoint{0.519339in}{0.466613in}}%
\pgfpathlineto{\pgfqpoint{0.519339in}{1.712363in}}%
\pgfusepath{stroke}%
\end{pgfscope}%
\begin{pgfscope}%
\pgfsetrectcap%
\pgfsetmiterjoin%
\pgfsetlinewidth{0.752812pt}%
\definecolor{currentstroke}{rgb}{0.700000,0.700000,0.700000}%
\pgfsetstrokecolor{currentstroke}%
\pgfsetdash{}{0pt}%
\pgfpathmoveto{\pgfqpoint{1.798089in}{0.466613in}}%
\pgfpathlineto{\pgfqpoint{1.798089in}{1.712363in}}%
\pgfusepath{stroke}%
\end{pgfscope}%
\begin{pgfscope}%
\pgfsetrectcap%
\pgfsetmiterjoin%
\pgfsetlinewidth{0.752812pt}%
\definecolor{currentstroke}{rgb}{0.700000,0.700000,0.700000}%
\pgfsetstrokecolor{currentstroke}%
\pgfsetdash{}{0pt}%
\pgfpathmoveto{\pgfqpoint{0.519339in}{0.466613in}}%
\pgfpathlineto{\pgfqpoint{1.798089in}{0.466613in}}%
\pgfusepath{stroke}%
\end{pgfscope}%
\begin{pgfscope}%
\pgfsetrectcap%
\pgfsetmiterjoin%
\pgfsetlinewidth{0.752812pt}%
\definecolor{currentstroke}{rgb}{0.700000,0.700000,0.700000}%
\pgfsetstrokecolor{currentstroke}%
\pgfsetdash{}{0pt}%
\pgfpathmoveto{\pgfqpoint{0.519339in}{1.712363in}}%
\pgfpathlineto{\pgfqpoint{1.798089in}{1.712363in}}%
\pgfusepath{stroke}%
\end{pgfscope}%
\end{pgfpicture}%
\makeatother%
\endgroup%
}} &
      \subfloat[\(\epsilon=0.25\)]{\resizebox{0.5\linewidth}{!}{%% Creator: Matplotlib, PGF backend
%%
%% To include the figure in your LaTeX document, write
%%   \input{<filename>.pgf}
%%
%% Make sure the required packages are loaded in your preamble
%%   \usepackage{pgf}
%%
%% and, on pdftex
%%   \usepackage[utf8]{inputenc}\DeclareUnicodeCharacter{2212}{-}
%%
%% or, on luatex and xetex
%%   \usepackage{unicode-math}
%%
%% Figures using additional raster images can only be included by \input if
%% they are in the same directory as the main LaTeX file. For loading figures
%% from other directories you can use the `import` package
%%   \usepackage{import}
%%
%% and then include the figures with
%%   \import{<path to file>}{<filename>.pgf}
%%
%% Matplotlib used the following preamble
%%   \usepackage[utf8]{inputenc}
%%   \usepackage[T1]{fontenc}
%%   \usepackage{amsmath}
%%   \newcommand*{\mat}[1]{\boldsymbol{#1}}
%%
\begingroup%
\makeatletter%
\begin{pgfpicture}%
\pgfpathrectangle{\pgfpointorigin}{\pgfqpoint{1.898089in}{1.812363in}}%
\pgfusepath{use as bounding box, clip}%
\begin{pgfscope}%
\pgfsetbuttcap%
\pgfsetmiterjoin%
\definecolor{currentfill}{rgb}{1.000000,1.000000,1.000000}%
\pgfsetfillcolor{currentfill}%
\pgfsetlinewidth{0.000000pt}%
\definecolor{currentstroke}{rgb}{1.000000,1.000000,1.000000}%
\pgfsetstrokecolor{currentstroke}%
\pgfsetstrokeopacity{0.000000}%
\pgfsetdash{}{0pt}%
\pgfpathmoveto{\pgfqpoint{0.000000in}{-0.000000in}}%
\pgfpathlineto{\pgfqpoint{1.898089in}{-0.000000in}}%
\pgfpathlineto{\pgfqpoint{1.898089in}{1.812363in}}%
\pgfpathlineto{\pgfqpoint{0.000000in}{1.812363in}}%
\pgfpathclose%
\pgfusepath{fill}%
\end{pgfscope}%
\begin{pgfscope}%
\pgfsetbuttcap%
\pgfsetmiterjoin%
\definecolor{currentfill}{rgb}{1.000000,1.000000,1.000000}%
\pgfsetfillcolor{currentfill}%
\pgfsetlinewidth{0.000000pt}%
\definecolor{currentstroke}{rgb}{0.000000,0.000000,0.000000}%
\pgfsetstrokecolor{currentstroke}%
\pgfsetstrokeopacity{0.000000}%
\pgfsetdash{}{0pt}%
\pgfpathmoveto{\pgfqpoint{0.519339in}{0.466613in}}%
\pgfpathlineto{\pgfqpoint{1.798089in}{0.466613in}}%
\pgfpathlineto{\pgfqpoint{1.798089in}{1.712363in}}%
\pgfpathlineto{\pgfqpoint{0.519339in}{1.712363in}}%
\pgfpathclose%
\pgfusepath{fill}%
\end{pgfscope}%
\begin{pgfscope}%
\pgfpathrectangle{\pgfqpoint{0.519339in}{0.466613in}}{\pgfqpoint{1.278750in}{1.245750in}}%
\pgfusepath{clip}%
\pgfsetroundcap%
\pgfsetroundjoin%
\pgfsetlinewidth{0.501875pt}%
\definecolor{currentstroke}{rgb}{0.800000,0.800000,0.800000}%
\pgfsetstrokecolor{currentstroke}%
\pgfsetdash{}{0pt}%
\pgfpathmoveto{\pgfqpoint{0.768713in}{0.466613in}}%
\pgfpathlineto{\pgfqpoint{0.768713in}{1.712363in}}%
\pgfusepath{stroke}%
\end{pgfscope}%
\begin{pgfscope}%
\definecolor{textcolor}{rgb}{0.150000,0.150000,0.150000}%
\pgfsetstrokecolor{textcolor}%
\pgfsetfillcolor{textcolor}%
\pgftext[x=0.768713in,y=0.376335in,,top]{\color{textcolor}\rmfamily\fontsize{8.000000}{9.600000}\selectfont \(\displaystyle {25000}\)}%
\end{pgfscope}%
\begin{pgfscope}%
\pgfpathrectangle{\pgfqpoint{0.519339in}{0.466613in}}{\pgfqpoint{1.278750in}{1.245750in}}%
\pgfusepath{clip}%
\pgfsetroundcap%
\pgfsetroundjoin%
\pgfsetlinewidth{0.501875pt}%
\definecolor{currentstroke}{rgb}{0.800000,0.800000,0.800000}%
\pgfsetstrokecolor{currentstroke}%
\pgfsetdash{}{0pt}%
\pgfpathmoveto{\pgfqpoint{1.122777in}{0.466613in}}%
\pgfpathlineto{\pgfqpoint{1.122777in}{1.712363in}}%
\pgfusepath{stroke}%
\end{pgfscope}%
\begin{pgfscope}%
\definecolor{textcolor}{rgb}{0.150000,0.150000,0.150000}%
\pgfsetstrokecolor{textcolor}%
\pgfsetfillcolor{textcolor}%
\pgftext[x=1.122777in,y=0.376335in,,top]{\color{textcolor}\rmfamily\fontsize{8.000000}{9.600000}\selectfont \(\displaystyle {50000}\)}%
\end{pgfscope}%
\begin{pgfscope}%
\pgfpathrectangle{\pgfqpoint{0.519339in}{0.466613in}}{\pgfqpoint{1.278750in}{1.245750in}}%
\pgfusepath{clip}%
\pgfsetroundcap%
\pgfsetroundjoin%
\pgfsetlinewidth{0.501875pt}%
\definecolor{currentstroke}{rgb}{0.800000,0.800000,0.800000}%
\pgfsetstrokecolor{currentstroke}%
\pgfsetdash{}{0pt}%
\pgfpathmoveto{\pgfqpoint{1.476841in}{0.466613in}}%
\pgfpathlineto{\pgfqpoint{1.476841in}{1.712363in}}%
\pgfusepath{stroke}%
\end{pgfscope}%
\begin{pgfscope}%
\definecolor{textcolor}{rgb}{0.150000,0.150000,0.150000}%
\pgfsetstrokecolor{textcolor}%
\pgfsetfillcolor{textcolor}%
\pgftext[x=1.476841in,y=0.376335in,,top]{\color{textcolor}\rmfamily\fontsize{8.000000}{9.600000}\selectfont \(\displaystyle {75000}\)}%
\end{pgfscope}%
\begin{pgfscope}%
\definecolor{textcolor}{rgb}{0.150000,0.150000,0.150000}%
\pgfsetstrokecolor{textcolor}%
\pgfsetfillcolor{textcolor}%
\pgftext[x=1.158714in,y=0.222655in,,top]{\color{textcolor}\rmfamily\fontsize{10.000000}{12.000000}\selectfont Number of nodes}%
\end{pgfscope}%
\begin{pgfscope}%
\pgfpathrectangle{\pgfqpoint{0.519339in}{0.466613in}}{\pgfqpoint{1.278750in}{1.245750in}}%
\pgfusepath{clip}%
\pgfsetroundcap%
\pgfsetroundjoin%
\pgfsetlinewidth{0.501875pt}%
\definecolor{currentstroke}{rgb}{0.800000,0.800000,0.800000}%
\pgfsetstrokecolor{currentstroke}%
\pgfsetdash{}{0pt}%
\pgfpathmoveto{\pgfqpoint{0.519339in}{0.672251in}}%
\pgfpathlineto{\pgfqpoint{1.798089in}{0.672251in}}%
\pgfusepath{stroke}%
\end{pgfscope}%
\begin{pgfscope}%
\definecolor{textcolor}{rgb}{0.150000,0.150000,0.150000}%
\pgfsetstrokecolor{textcolor}%
\pgfsetfillcolor{textcolor}%
\pgftext[x=0.278211in, y=0.633988in, left, base]{\color{textcolor}\rmfamily\fontsize{8.000000}{9.600000}\selectfont \(\displaystyle {0.2}\)}%
\end{pgfscope}%
\begin{pgfscope}%
\pgfpathrectangle{\pgfqpoint{0.519339in}{0.466613in}}{\pgfqpoint{1.278750in}{1.245750in}}%
\pgfusepath{clip}%
\pgfsetroundcap%
\pgfsetroundjoin%
\pgfsetlinewidth{0.501875pt}%
\definecolor{currentstroke}{rgb}{0.800000,0.800000,0.800000}%
\pgfsetstrokecolor{currentstroke}%
\pgfsetdash{}{0pt}%
\pgfpathmoveto{\pgfqpoint{0.519339in}{1.019548in}}%
\pgfpathlineto{\pgfqpoint{1.798089in}{1.019548in}}%
\pgfusepath{stroke}%
\end{pgfscope}%
\begin{pgfscope}%
\definecolor{textcolor}{rgb}{0.150000,0.150000,0.150000}%
\pgfsetstrokecolor{textcolor}%
\pgfsetfillcolor{textcolor}%
\pgftext[x=0.278211in, y=0.981286in, left, base]{\color{textcolor}\rmfamily\fontsize{8.000000}{9.600000}\selectfont \(\displaystyle {0.4}\)}%
\end{pgfscope}%
\begin{pgfscope}%
\pgfpathrectangle{\pgfqpoint{0.519339in}{0.466613in}}{\pgfqpoint{1.278750in}{1.245750in}}%
\pgfusepath{clip}%
\pgfsetroundcap%
\pgfsetroundjoin%
\pgfsetlinewidth{0.501875pt}%
\definecolor{currentstroke}{rgb}{0.800000,0.800000,0.800000}%
\pgfsetstrokecolor{currentstroke}%
\pgfsetdash{}{0pt}%
\pgfpathmoveto{\pgfqpoint{0.519339in}{1.366846in}}%
\pgfpathlineto{\pgfqpoint{1.798089in}{1.366846in}}%
\pgfusepath{stroke}%
\end{pgfscope}%
\begin{pgfscope}%
\definecolor{textcolor}{rgb}{0.150000,0.150000,0.150000}%
\pgfsetstrokecolor{textcolor}%
\pgfsetfillcolor{textcolor}%
\pgftext[x=0.278211in, y=1.328584in, left, base]{\color{textcolor}\rmfamily\fontsize{8.000000}{9.600000}\selectfont \(\displaystyle {0.6}\)}%
\end{pgfscope}%
\begin{pgfscope}%
\definecolor{textcolor}{rgb}{0.150000,0.150000,0.150000}%
\pgfsetstrokecolor{textcolor}%
\pgfsetfillcolor{textcolor}%
\pgftext[x=0.222655in,y=1.089488in,,bottom,rotate=90.000000]{\color{textcolor}\rmfamily\fontsize{10.000000}{12.000000}\selectfont Accuracy drop}%
\end{pgfscope}%
\begin{pgfscope}%
\pgfpathrectangle{\pgfqpoint{0.519339in}{0.466613in}}{\pgfqpoint{1.278750in}{1.245750in}}%
\pgfusepath{clip}%
\pgfsetbuttcap%
\pgfsetroundjoin%
\definecolor{currentfill}{rgb}{0.298039,0.447059,0.690196}%
\pgfsetfillcolor{currentfill}%
\pgfsetfillopacity{0.800000}%
\pgfsetlinewidth{1.003750pt}%
\definecolor{currentstroke}{rgb}{0.298039,0.447059,0.690196}%
\pgfsetstrokecolor{currentstroke}%
\pgfsetstrokeopacity{0.800000}%
\pgfsetdash{}{0pt}%
\pgfsys@defobject{currentmarker}{\pgfqpoint{-0.017010in}{-0.017010in}}{\pgfqpoint{0.017010in}{0.017010in}}{%
\pgfpathmoveto{\pgfqpoint{0.000000in}{-0.017010in}}%
\pgfpathcurveto{\pgfqpoint{0.004511in}{-0.017010in}}{\pgfqpoint{0.008838in}{-0.015218in}}{\pgfqpoint{0.012028in}{-0.012028in}}%
\pgfpathcurveto{\pgfqpoint{0.015218in}{-0.008838in}}{\pgfqpoint{0.017010in}{-0.004511in}}{\pgfqpoint{0.017010in}{0.000000in}}%
\pgfpathcurveto{\pgfqpoint{0.017010in}{0.004511in}}{\pgfqpoint{0.015218in}{0.008838in}}{\pgfqpoint{0.012028in}{0.012028in}}%
\pgfpathcurveto{\pgfqpoint{0.008838in}{0.015218in}}{\pgfqpoint{0.004511in}{0.017010in}}{\pgfqpoint{0.000000in}{0.017010in}}%
\pgfpathcurveto{\pgfqpoint{-0.004511in}{0.017010in}}{\pgfqpoint{-0.008838in}{0.015218in}}{\pgfqpoint{-0.012028in}{0.012028in}}%
\pgfpathcurveto{\pgfqpoint{-0.015218in}{0.008838in}}{\pgfqpoint{-0.017010in}{0.004511in}}{\pgfqpoint{-0.017010in}{0.000000in}}%
\pgfpathcurveto{\pgfqpoint{-0.017010in}{-0.004511in}}{\pgfqpoint{-0.015218in}{-0.008838in}}{\pgfqpoint{-0.012028in}{-0.012028in}}%
\pgfpathcurveto{\pgfqpoint{-0.008838in}{-0.015218in}}{\pgfqpoint{-0.004511in}{-0.017010in}}{\pgfqpoint{0.000000in}{-0.017010in}}%
\pgfpathclose%
\pgfusepath{stroke,fill}%
}%
\begin{pgfscope}%
\pgfsys@transformshift{1.014094in}{0.969475in}%
\pgfsys@useobject{currentmarker}{}%
\end{pgfscope}%
\begin{pgfscope}%
\pgfsys@transformshift{0.764422in}{1.196367in}%
\pgfsys@useobject{currentmarker}{}%
\end{pgfscope}%
\begin{pgfscope}%
\pgfsys@transformshift{0.887056in}{1.502885in}%
\pgfsys@useobject{currentmarker}{}%
\end{pgfscope}%
\begin{pgfscope}%
\pgfsys@transformshift{0.701456in}{1.544557in}%
\pgfsys@useobject{currentmarker}{}%
\end{pgfscope}%
\begin{pgfscope}%
\pgfsys@transformshift{0.774506in}{1.448144in}%
\pgfsys@useobject{currentmarker}{}%
\end{pgfscope}%
\begin{pgfscope}%
\pgfsys@transformshift{1.101193in}{1.018377in}%
\pgfsys@useobject{currentmarker}{}%
\end{pgfscope}%
\begin{pgfscope}%
\pgfsys@transformshift{0.628844in}{1.323766in}%
\pgfsys@useobject{currentmarker}{}%
\end{pgfscope}%
\begin{pgfscope}%
\pgfsys@transformshift{0.934939in}{1.048626in}%
\pgfsys@useobject{currentmarker}{}%
\end{pgfscope}%
\begin{pgfscope}%
\pgfsys@transformshift{1.292331in}{0.998633in}%
\pgfsys@useobject{currentmarker}{}%
\end{pgfscope}%
\begin{pgfscope}%
\pgfsys@transformshift{1.147788in}{1.307166in}%
\pgfsys@useobject{currentmarker}{}%
\end{pgfscope}%
\begin{pgfscope}%
\pgfsys@transformshift{1.046441in}{1.497523in}%
\pgfsys@useobject{currentmarker}{}%
\end{pgfscope}%
\begin{pgfscope}%
\pgfsys@transformshift{0.655342in}{1.503570in}%
\pgfsys@useobject{currentmarker}{}%
\end{pgfscope}%
\begin{pgfscope}%
\pgfsys@transformshift{0.791699in}{1.469025in}%
\pgfsys@useobject{currentmarker}{}%
\end{pgfscope}%
\begin{pgfscope}%
\pgfsys@transformshift{0.950532in}{1.270299in}%
\pgfsys@useobject{currentmarker}{}%
\end{pgfscope}%
\begin{pgfscope}%
\pgfsys@transformshift{0.834612in}{1.258866in}%
\pgfsys@useobject{currentmarker}{}%
\end{pgfscope}%
\begin{pgfscope}%
\pgfsys@transformshift{1.334280in}{1.444067in}%
\pgfsys@useobject{currentmarker}{}%
\end{pgfscope}%
\begin{pgfscope}%
\pgfsys@transformshift{0.627740in}{1.333473in}%
\pgfsys@useobject{currentmarker}{}%
\end{pgfscope}%
\begin{pgfscope}%
\pgfsys@transformshift{1.116673in}{1.070209in}%
\pgfsys@useobject{currentmarker}{}%
\end{pgfscope}%
\begin{pgfscope}%
\pgfsys@transformshift{0.838719in}{1.250125in}%
\pgfsys@useobject{currentmarker}{}%
\end{pgfscope}%
\begin{pgfscope}%
\pgfsys@transformshift{1.063011in}{0.978863in}%
\pgfsys@useobject{currentmarker}{}%
\end{pgfscope}%
\begin{pgfscope}%
\pgfsys@transformshift{0.700917in}{1.482372in}%
\pgfsys@useobject{currentmarker}{}%
\end{pgfscope}%
\begin{pgfscope}%
\pgfsys@transformshift{0.955687in}{1.453100in}%
\pgfsys@useobject{currentmarker}{}%
\end{pgfscope}%
\begin{pgfscope}%
\pgfsys@transformshift{0.591696in}{1.505955in}%
\pgfsys@useobject{currentmarker}{}%
\end{pgfscope}%
\begin{pgfscope}%
\pgfsys@transformshift{1.043410in}{1.016851in}%
\pgfsys@useobject{currentmarker}{}%
\end{pgfscope}%
\begin{pgfscope}%
\pgfsys@transformshift{0.590860in}{1.295215in}%
\pgfsys@useobject{currentmarker}{}%
\end{pgfscope}%
\begin{pgfscope}%
\pgfsys@transformshift{0.809686in}{1.508692in}%
\pgfsys@useobject{currentmarker}{}%
\end{pgfscope}%
\begin{pgfscope}%
\pgfsys@transformshift{0.628278in}{1.463160in}%
\pgfsys@useobject{currentmarker}{}%
\end{pgfscope}%
\begin{pgfscope}%
\pgfsys@transformshift{0.998118in}{1.524432in}%
\pgfsys@useobject{currentmarker}{}%
\end{pgfscope}%
\begin{pgfscope}%
\pgfsys@transformshift{0.756604in}{1.438548in}%
\pgfsys@useobject{currentmarker}{}%
\end{pgfscope}%
\begin{pgfscope}%
\pgfsys@transformshift{0.851026in}{1.179902in}%
\pgfsys@useobject{currentmarker}{}%
\end{pgfscope}%
\begin{pgfscope}%
\pgfsys@transformshift{1.033468in}{1.170967in}%
\pgfsys@useobject{currentmarker}{}%
\end{pgfscope}%
\begin{pgfscope}%
\pgfsys@transformshift{0.981336in}{1.076967in}%
\pgfsys@useobject{currentmarker}{}%
\end{pgfscope}%
\begin{pgfscope}%
\pgfsys@transformshift{1.043792in}{1.479785in}%
\pgfsys@useobject{currentmarker}{}%
\end{pgfscope}%
\begin{pgfscope}%
\pgfsys@transformshift{1.062685in}{0.954562in}%
\pgfsys@useobject{currentmarker}{}%
\end{pgfscope}%
\begin{pgfscope}%
\pgfsys@transformshift{0.777282in}{1.504232in}%
\pgfsys@useobject{currentmarker}{}%
\end{pgfscope}%
\begin{pgfscope}%
\pgfsys@transformshift{1.357479in}{1.332614in}%
\pgfsys@useobject{currentmarker}{}%
\end{pgfscope}%
\begin{pgfscope}%
\pgfsys@transformshift{0.664576in}{1.331987in}%
\pgfsys@useobject{currentmarker}{}%
\end{pgfscope}%
\begin{pgfscope}%
\pgfsys@transformshift{1.101547in}{1.059971in}%
\pgfsys@useobject{currentmarker}{}%
\end{pgfscope}%
\begin{pgfscope}%
\pgfsys@transformshift{0.960531in}{1.138548in}%
\pgfsys@useobject{currentmarker}{}%
\end{pgfscope}%
\begin{pgfscope}%
\pgfsys@transformshift{1.258723in}{1.313100in}%
\pgfsys@useobject{currentmarker}{}%
\end{pgfscope}%
\begin{pgfscope}%
\pgfsys@transformshift{1.104082in}{1.042994in}%
\pgfsys@useobject{currentmarker}{}%
\end{pgfscope}%
\begin{pgfscope}%
\pgfsys@transformshift{0.629397in}{1.366268in}%
\pgfsys@useobject{currentmarker}{}%
\end{pgfscope}%
\begin{pgfscope}%
\pgfsys@transformshift{0.934642in}{1.490216in}%
\pgfsys@useobject{currentmarker}{}%
\end{pgfscope}%
\begin{pgfscope}%
\pgfsys@transformshift{0.642993in}{1.382256in}%
\pgfsys@useobject{currentmarker}{}%
\end{pgfscope}%
\begin{pgfscope}%
\pgfsys@transformshift{0.943323in}{1.521172in}%
\pgfsys@useobject{currentmarker}{}%
\end{pgfscope}%
\begin{pgfscope}%
\pgfsys@transformshift{0.681798in}{1.467084in}%
\pgfsys@useobject{currentmarker}{}%
\end{pgfscope}%
\begin{pgfscope}%
\pgfsys@transformshift{0.614044in}{1.226265in}%
\pgfsys@useobject{currentmarker}{}%
\end{pgfscope}%
\begin{pgfscope}%
\pgfsys@transformshift{1.014108in}{1.175338in}%
\pgfsys@useobject{currentmarker}{}%
\end{pgfscope}%
\begin{pgfscope}%
\pgfsys@transformshift{1.147703in}{1.385880in}%
\pgfsys@useobject{currentmarker}{}%
\end{pgfscope}%
\begin{pgfscope}%
\pgfsys@transformshift{0.863121in}{0.989852in}%
\pgfsys@useobject{currentmarker}{}%
\end{pgfscope}%
\begin{pgfscope}%
\pgfsys@transformshift{0.919290in}{1.250879in}%
\pgfsys@useobject{currentmarker}{}%
\end{pgfscope}%
\begin{pgfscope}%
\pgfsys@transformshift{0.609087in}{1.282920in}%
\pgfsys@useobject{currentmarker}{}%
\end{pgfscope}%
\begin{pgfscope}%
\pgfsys@transformshift{0.633801in}{1.394509in}%
\pgfsys@useobject{currentmarker}{}%
\end{pgfscope}%
\begin{pgfscope}%
\pgfsys@transformshift{0.719739in}{1.152854in}%
\pgfsys@useobject{currentmarker}{}%
\end{pgfscope}%
\begin{pgfscope}%
\pgfsys@transformshift{0.673853in}{1.486692in}%
\pgfsys@useobject{currentmarker}{}%
\end{pgfscope}%
\begin{pgfscope}%
\pgfsys@transformshift{0.584246in}{1.164835in}%
\pgfsys@useobject{currentmarker}{}%
\end{pgfscope}%
\begin{pgfscope}%
\pgfsys@transformshift{0.833705in}{1.572414in}%
\pgfsys@useobject{currentmarker}{}%
\end{pgfscope}%
\begin{pgfscope}%
\pgfsys@transformshift{1.097808in}{1.363746in}%
\pgfsys@useobject{currentmarker}{}%
\end{pgfscope}%
\begin{pgfscope}%
\pgfsys@transformshift{0.743646in}{1.543352in}%
\pgfsys@useobject{currentmarker}{}%
\end{pgfscope}%
\begin{pgfscope}%
\pgfsys@transformshift{0.647822in}{1.471453in}%
\pgfsys@useobject{currentmarker}{}%
\end{pgfscope}%
\begin{pgfscope}%
\pgfsys@transformshift{0.688525in}{1.456256in}%
\pgfsys@useobject{currentmarker}{}%
\end{pgfscope}%
\begin{pgfscope}%
\pgfsys@transformshift{0.927674in}{0.739746in}%
\pgfsys@useobject{currentmarker}{}%
\end{pgfscope}%
\begin{pgfscope}%
\pgfsys@transformshift{0.539960in}{1.235747in}%
\pgfsys@useobject{currentmarker}{}%
\end{pgfscope}%
\begin{pgfscope}%
\pgfsys@transformshift{0.959129in}{1.519507in}%
\pgfsys@useobject{currentmarker}{}%
\end{pgfscope}%
\begin{pgfscope}%
\pgfsys@transformshift{0.615843in}{1.553800in}%
\pgfsys@useobject{currentmarker}{}%
\end{pgfscope}%
\begin{pgfscope}%
\pgfsys@transformshift{0.750515in}{1.580555in}%
\pgfsys@useobject{currentmarker}{}%
\end{pgfscope}%
\begin{pgfscope}%
\pgfsys@transformshift{0.904362in}{1.460125in}%
\pgfsys@useobject{currentmarker}{}%
\end{pgfscope}%
\begin{pgfscope}%
\pgfsys@transformshift{0.871151in}{1.185865in}%
\pgfsys@useobject{currentmarker}{}%
\end{pgfscope}%
\begin{pgfscope}%
\pgfsys@transformshift{0.914021in}{1.172944in}%
\pgfsys@useobject{currentmarker}{}%
\end{pgfscope}%
\begin{pgfscope}%
\pgfsys@transformshift{1.300630in}{1.276411in}%
\pgfsys@useobject{currentmarker}{}%
\end{pgfscope}%
\begin{pgfscope}%
\pgfsys@transformshift{1.025013in}{1.035769in}%
\pgfsys@useobject{currentmarker}{}%
\end{pgfscope}%
\begin{pgfscope}%
\pgfsys@transformshift{0.673824in}{1.112902in}%
\pgfsys@useobject{currentmarker}{}%
\end{pgfscope}%
\begin{pgfscope}%
\pgfsys@transformshift{0.931809in}{1.280818in}%
\pgfsys@useobject{currentmarker}{}%
\end{pgfscope}%
\begin{pgfscope}%
\pgfsys@transformshift{0.831397in}{1.213265in}%
\pgfsys@useobject{currentmarker}{}%
\end{pgfscope}%
\begin{pgfscope}%
\pgfsys@transformshift{1.034134in}{0.974767in}%
\pgfsys@useobject{currentmarker}{}%
\end{pgfscope}%
\begin{pgfscope}%
\pgfsys@transformshift{1.007225in}{1.090206in}%
\pgfsys@useobject{currentmarker}{}%
\end{pgfscope}%
\begin{pgfscope}%
\pgfsys@transformshift{0.780879in}{1.539936in}%
\pgfsys@useobject{currentmarker}{}%
\end{pgfscope}%
\begin{pgfscope}%
\pgfsys@transformshift{1.225200in}{1.243664in}%
\pgfsys@useobject{currentmarker}{}%
\end{pgfscope}%
\begin{pgfscope}%
\pgfsys@transformshift{1.224790in}{1.095302in}%
\pgfsys@useobject{currentmarker}{}%
\end{pgfscope}%
\begin{pgfscope}%
\pgfsys@transformshift{1.224861in}{1.382667in}%
\pgfsys@useobject{currentmarker}{}%
\end{pgfscope}%
\begin{pgfscope}%
\pgfsys@transformshift{0.949258in}{0.903487in}%
\pgfsys@useobject{currentmarker}{}%
\end{pgfscope}%
\begin{pgfscope}%
\pgfsys@transformshift{0.737570in}{1.544478in}%
\pgfsys@useobject{currentmarker}{}%
\end{pgfscope}%
\begin{pgfscope}%
\pgfsys@transformshift{0.612260in}{1.434415in}%
\pgfsys@useobject{currentmarker}{}%
\end{pgfscope}%
\begin{pgfscope}%
\pgfsys@transformshift{1.310501in}{1.415411in}%
\pgfsys@useobject{currentmarker}{}%
\end{pgfscope}%
\begin{pgfscope}%
\pgfsys@transformshift{0.729951in}{1.505167in}%
\pgfsys@useobject{currentmarker}{}%
\end{pgfscope}%
\begin{pgfscope}%
\pgfsys@transformshift{0.976605in}{1.468610in}%
\pgfsys@useobject{currentmarker}{}%
\end{pgfscope}%
\begin{pgfscope}%
\pgfsys@transformshift{0.704260in}{0.839230in}%
\pgfsys@useobject{currentmarker}{}%
\end{pgfscope}%
\begin{pgfscope}%
\pgfsys@transformshift{1.293195in}{1.381983in}%
\pgfsys@useobject{currentmarker}{}%
\end{pgfscope}%
\begin{pgfscope}%
\pgfsys@transformshift{1.039558in}{1.372373in}%
\pgfsys@useobject{currentmarker}{}%
\end{pgfscope}%
\begin{pgfscope}%
\pgfsys@transformshift{0.676020in}{1.458601in}%
\pgfsys@useobject{currentmarker}{}%
\end{pgfscope}%
\begin{pgfscope}%
\pgfsys@transformshift{1.177614in}{1.142720in}%
\pgfsys@useobject{currentmarker}{}%
\end{pgfscope}%
\begin{pgfscope}%
\pgfsys@transformshift{1.401552in}{1.084594in}%
\pgfsys@useobject{currentmarker}{}%
\end{pgfscope}%
\begin{pgfscope}%
\pgfsys@transformshift{1.095840in}{1.453704in}%
\pgfsys@useobject{currentmarker}{}%
\end{pgfscope}%
\begin{pgfscope}%
\pgfsys@transformshift{0.849171in}{1.029265in}%
\pgfsys@useobject{currentmarker}{}%
\end{pgfscope}%
\begin{pgfscope}%
\pgfsys@transformshift{0.808340in}{1.128834in}%
\pgfsys@useobject{currentmarker}{}%
\end{pgfscope}%
\begin{pgfscope}%
\pgfsys@transformshift{0.978532in}{0.837779in}%
\pgfsys@useobject{currentmarker}{}%
\end{pgfscope}%
\begin{pgfscope}%
\pgfsys@transformshift{1.004661in}{1.571376in}%
\pgfsys@useobject{currentmarker}{}%
\end{pgfscope}%
\begin{pgfscope}%
\pgfsys@transformshift{1.014958in}{1.178675in}%
\pgfsys@useobject{currentmarker}{}%
\end{pgfscope}%
\begin{pgfscope}%
\pgfsys@transformshift{1.010086in}{0.870646in}%
\pgfsys@useobject{currentmarker}{}%
\end{pgfscope}%
\begin{pgfscope}%
\pgfsys@transformshift{0.942148in}{0.679975in}%
\pgfsys@useobject{currentmarker}{}%
\end{pgfscope}%
\begin{pgfscope}%
\pgfsys@transformshift{0.614752in}{1.387348in}%
\pgfsys@useobject{currentmarker}{}%
\end{pgfscope}%
\begin{pgfscope}%
\pgfsys@transformshift{0.839540in}{1.396994in}%
\pgfsys@useobject{currentmarker}{}%
\end{pgfscope}%
\begin{pgfscope}%
\pgfsys@transformshift{0.792846in}{1.179810in}%
\pgfsys@useobject{currentmarker}{}%
\end{pgfscope}%
\begin{pgfscope}%
\pgfsys@transformshift{0.650952in}{1.439921in}%
\pgfsys@useobject{currentmarker}{}%
\end{pgfscope}%
\begin{pgfscope}%
\pgfsys@transformshift{1.467139in}{1.332252in}%
\pgfsys@useobject{currentmarker}{}%
\end{pgfscope}%
\begin{pgfscope}%
\pgfsys@transformshift{0.519339in}{1.321675in}%
\pgfsys@useobject{currentmarker}{}%
\end{pgfscope}%
\begin{pgfscope}%
\pgfsys@transformshift{0.615503in}{1.494910in}%
\pgfsys@useobject{currentmarker}{}%
\end{pgfscope}%
\begin{pgfscope}%
\pgfsys@transformshift{1.114294in}{1.328565in}%
\pgfsys@useobject{currentmarker}{}%
\end{pgfscope}%
\begin{pgfscope}%
\pgfsys@transformshift{1.368200in}{1.397234in}%
\pgfsys@useobject{currentmarker}{}%
\end{pgfscope}%
\begin{pgfscope}%
\pgfsys@transformshift{0.633107in}{1.462455in}%
\pgfsys@useobject{currentmarker}{}%
\end{pgfscope}%
\begin{pgfscope}%
\pgfsys@transformshift{0.774577in}{1.433662in}%
\pgfsys@useobject{currentmarker}{}%
\end{pgfscope}%
\begin{pgfscope}%
\pgfsys@transformshift{0.783825in}{1.503584in}%
\pgfsys@useobject{currentmarker}{}%
\end{pgfscope}%
\begin{pgfscope}%
\pgfsys@transformshift{0.989989in}{1.527331in}%
\pgfsys@useobject{currentmarker}{}%
\end{pgfscope}%
\begin{pgfscope}%
\pgfsys@transformshift{0.603493in}{1.487396in}%
\pgfsys@useobject{currentmarker}{}%
\end{pgfscope}%
\begin{pgfscope}%
\pgfsys@transformshift{0.782876in}{1.185589in}%
\pgfsys@useobject{currentmarker}{}%
\end{pgfscope}%
\begin{pgfscope}%
\pgfsys@transformshift{0.963335in}{0.921425in}%
\pgfsys@useobject{currentmarker}{}%
\end{pgfscope}%
\begin{pgfscope}%
\pgfsys@transformshift{0.954427in}{1.061095in}%
\pgfsys@useobject{currentmarker}{}%
\end{pgfscope}%
\begin{pgfscope}%
\pgfsys@transformshift{1.301947in}{1.425359in}%
\pgfsys@useobject{currentmarker}{}%
\end{pgfscope}%
\begin{pgfscope}%
\pgfsys@transformshift{1.130680in}{1.049811in}%
\pgfsys@useobject{currentmarker}{}%
\end{pgfscope}%
\begin{pgfscope}%
\pgfsys@transformshift{0.742640in}{1.383831in}%
\pgfsys@useobject{currentmarker}{}%
\end{pgfscope}%
\begin{pgfscope}%
\pgfsys@transformshift{0.611892in}{1.438004in}%
\pgfsys@useobject{currentmarker}{}%
\end{pgfscope}%
\begin{pgfscope}%
\pgfsys@transformshift{0.971691in}{1.115180in}%
\pgfsys@useobject{currentmarker}{}%
\end{pgfscope}%
\begin{pgfscope}%
\pgfsys@transformshift{0.934245in}{1.442034in}%
\pgfsys@useobject{currentmarker}{}%
\end{pgfscope}%
\begin{pgfscope}%
\pgfsys@transformshift{0.831666in}{1.140450in}%
\pgfsys@useobject{currentmarker}{}%
\end{pgfscope}%
\begin{pgfscope}%
\pgfsys@transformshift{0.601964in}{1.369131in}%
\pgfsys@useobject{currentmarker}{}%
\end{pgfscope}%
\begin{pgfscope}%
\pgfsys@transformshift{0.771730in}{1.332557in}%
\pgfsys@useobject{currentmarker}{}%
\end{pgfscope}%
\begin{pgfscope}%
\pgfsys@transformshift{0.764479in}{1.367032in}%
\pgfsys@useobject{currentmarker}{}%
\end{pgfscope}%
\begin{pgfscope}%
\pgfsys@transformshift{0.864721in}{1.217945in}%
\pgfsys@useobject{currentmarker}{}%
\end{pgfscope}%
\begin{pgfscope}%
\pgfsys@transformshift{0.626805in}{1.607272in}%
\pgfsys@useobject{currentmarker}{}%
\end{pgfscope}%
\begin{pgfscope}%
\pgfsys@transformshift{0.621805in}{1.368354in}%
\pgfsys@useobject{currentmarker}{}%
\end{pgfscope}%
\begin{pgfscope}%
\pgfsys@transformshift{0.912860in}{1.467165in}%
\pgfsys@useobject{currentmarker}{}%
\end{pgfscope}%
\begin{pgfscope}%
\pgfsys@transformshift{1.103502in}{1.147158in}%
\pgfsys@useobject{currentmarker}{}%
\end{pgfscope}%
\begin{pgfscope}%
\pgfsys@transformshift{0.971719in}{1.543393in}%
\pgfsys@useobject{currentmarker}{}%
\end{pgfscope}%
\begin{pgfscope}%
\pgfsys@transformshift{1.079312in}{1.251357in}%
\pgfsys@useobject{currentmarker}{}%
\end{pgfscope}%
\begin{pgfscope}%
\pgfsys@transformshift{0.630600in}{1.308946in}%
\pgfsys@useobject{currentmarker}{}%
\end{pgfscope}%
\begin{pgfscope}%
\pgfsys@transformshift{0.820662in}{1.069429in}%
\pgfsys@useobject{currentmarker}{}%
\end{pgfscope}%
\begin{pgfscope}%
\pgfsys@transformshift{1.381810in}{1.271220in}%
\pgfsys@useobject{currentmarker}{}%
\end{pgfscope}%
\begin{pgfscope}%
\pgfsys@transformshift{1.091223in}{1.023675in}%
\pgfsys@useobject{currentmarker}{}%
\end{pgfscope}%
\begin{pgfscope}%
\pgfsys@transformshift{1.078717in}{0.971777in}%
\pgfsys@useobject{currentmarker}{}%
\end{pgfscope}%
\begin{pgfscope}%
\pgfsys@transformshift{1.050265in}{0.989269in}%
\pgfsys@useobject{currentmarker}{}%
\end{pgfscope}%
\begin{pgfscope}%
\pgfsys@transformshift{0.988006in}{1.140247in}%
\pgfsys@useobject{currentmarker}{}%
\end{pgfscope}%
\begin{pgfscope}%
\pgfsys@transformshift{1.003797in}{0.816080in}%
\pgfsys@useobject{currentmarker}{}%
\end{pgfscope}%
\begin{pgfscope}%
\pgfsys@transformshift{0.624865in}{1.443681in}%
\pgfsys@useobject{currentmarker}{}%
\end{pgfscope}%
\begin{pgfscope}%
\pgfsys@transformshift{0.701215in}{1.388599in}%
\pgfsys@useobject{currentmarker}{}%
\end{pgfscope}%
\begin{pgfscope}%
\pgfsys@transformshift{0.860076in}{1.501663in}%
\pgfsys@useobject{currentmarker}{}%
\end{pgfscope}%
\begin{pgfscope}%
\pgfsys@transformshift{0.669590in}{1.545940in}%
\pgfsys@useobject{currentmarker}{}%
\end{pgfscope}%
\begin{pgfscope}%
\pgfsys@transformshift{1.130028in}{1.083238in}%
\pgfsys@useobject{currentmarker}{}%
\end{pgfscope}%
\begin{pgfscope}%
\pgfsys@transformshift{0.778386in}{1.398198in}%
\pgfsys@useobject{currentmarker}{}%
\end{pgfscope}%
\begin{pgfscope}%
\pgfsys@transformshift{0.724569in}{1.413945in}%
\pgfsys@useobject{currentmarker}{}%
\end{pgfscope}%
\begin{pgfscope}%
\pgfsys@transformshift{1.022719in}{1.013699in}%
\pgfsys@useobject{currentmarker}{}%
\end{pgfscope}%
\begin{pgfscope}%
\pgfsys@transformshift{0.640061in}{1.525619in}%
\pgfsys@useobject{currentmarker}{}%
\end{pgfscope}%
\begin{pgfscope}%
\pgfsys@transformshift{0.687477in}{1.541595in}%
\pgfsys@useobject{currentmarker}{}%
\end{pgfscope}%
\begin{pgfscope}%
\pgfsys@transformshift{0.550823in}{1.288111in}%
\pgfsys@useobject{currentmarker}{}%
\end{pgfscope}%
\begin{pgfscope}%
\pgfsys@transformshift{0.866718in}{1.302066in}%
\pgfsys@useobject{currentmarker}{}%
\end{pgfscope}%
\begin{pgfscope}%
\pgfsys@transformshift{0.933197in}{1.544334in}%
\pgfsys@useobject{currentmarker}{}%
\end{pgfscope}%
\begin{pgfscope}%
\pgfsys@transformshift{0.768161in}{1.424741in}%
\pgfsys@useobject{currentmarker}{}%
\end{pgfscope}%
\begin{pgfscope}%
\pgfsys@transformshift{1.310162in}{1.435633in}%
\pgfsys@useobject{currentmarker}{}%
\end{pgfscope}%
\begin{pgfscope}%
\pgfsys@transformshift{0.726396in}{1.212263in}%
\pgfsys@useobject{currentmarker}{}%
\end{pgfscope}%
\begin{pgfscope}%
\pgfsys@transformshift{0.698283in}{1.254796in}%
\pgfsys@useobject{currentmarker}{}%
\end{pgfscope}%
\begin{pgfscope}%
\pgfsys@transformshift{0.889293in}{1.544798in}%
\pgfsys@useobject{currentmarker}{}%
\end{pgfscope}%
\begin{pgfscope}%
\pgfsys@transformshift{0.976775in}{0.698085in}%
\pgfsys@useobject{currentmarker}{}%
\end{pgfscope}%
\begin{pgfscope}%
\pgfsys@transformshift{0.948394in}{1.112286in}%
\pgfsys@useobject{currentmarker}{}%
\end{pgfscope}%
\begin{pgfscope}%
\pgfsys@transformshift{0.941355in}{0.717425in}%
\pgfsys@useobject{currentmarker}{}%
\end{pgfscope}%
\begin{pgfscope}%
\pgfsys@transformshift{1.336915in}{1.067765in}%
\pgfsys@useobject{currentmarker}{}%
\end{pgfscope}%
\begin{pgfscope}%
\pgfsys@transformshift{1.305757in}{1.370205in}%
\pgfsys@useobject{currentmarker}{}%
\end{pgfscope}%
\begin{pgfscope}%
\pgfsys@transformshift{0.687902in}{1.424293in}%
\pgfsys@useobject{currentmarker}{}%
\end{pgfscope}%
\begin{pgfscope}%
\pgfsys@transformshift{0.544549in}{1.452852in}%
\pgfsys@useobject{currentmarker}{}%
\end{pgfscope}%
\begin{pgfscope}%
\pgfsys@transformshift{0.592050in}{1.296776in}%
\pgfsys@useobject{currentmarker}{}%
\end{pgfscope}%
\begin{pgfscope}%
\pgfsys@transformshift{1.087201in}{1.021254in}%
\pgfsys@useobject{currentmarker}{}%
\end{pgfscope}%
\begin{pgfscope}%
\pgfsys@transformshift{0.822772in}{1.551755in}%
\pgfsys@useobject{currentmarker}{}%
\end{pgfscope}%
\begin{pgfscope}%
\pgfsys@transformshift{0.903881in}{1.329876in}%
\pgfsys@useobject{currentmarker}{}%
\end{pgfscope}%
\begin{pgfscope}%
\pgfsys@transformshift{0.794801in}{1.569358in}%
\pgfsys@useobject{currentmarker}{}%
\end{pgfscope}%
\begin{pgfscope}%
\pgfsys@transformshift{0.648488in}{1.251923in}%
\pgfsys@useobject{currentmarker}{}%
\end{pgfscope}%
\begin{pgfscope}%
\pgfsys@transformshift{0.534451in}{1.349882in}%
\pgfsys@useobject{currentmarker}{}%
\end{pgfscope}%
\begin{pgfscope}%
\pgfsys@transformshift{0.745105in}{1.431276in}%
\pgfsys@useobject{currentmarker}{}%
\end{pgfscope}%
\begin{pgfscope}%
\pgfsys@transformshift{1.194553in}{1.156533in}%
\pgfsys@useobject{currentmarker}{}%
\end{pgfscope}%
\begin{pgfscope}%
\pgfsys@transformshift{0.745798in}{1.632211in}%
\pgfsys@useobject{currentmarker}{}%
\end{pgfscope}%
\begin{pgfscope}%
\pgfsys@transformshift{0.619044in}{1.278510in}%
\pgfsys@useobject{currentmarker}{}%
\end{pgfscope}%
\begin{pgfscope}%
\pgfsys@transformshift{0.792903in}{1.466850in}%
\pgfsys@useobject{currentmarker}{}%
\end{pgfscope}%
\begin{pgfscope}%
\pgfsys@transformshift{0.936540in}{1.476726in}%
\pgfsys@useobject{currentmarker}{}%
\end{pgfscope}%
\begin{pgfscope}%
\pgfsys@transformshift{0.741720in}{1.551004in}%
\pgfsys@useobject{currentmarker}{}%
\end{pgfscope}%
\begin{pgfscope}%
\pgfsys@transformshift{0.888444in}{1.354546in}%
\pgfsys@useobject{currentmarker}{}%
\end{pgfscope}%
\begin{pgfscope}%
\pgfsys@transformshift{0.890186in}{1.445839in}%
\pgfsys@useobject{currentmarker}{}%
\end{pgfscope}%
\begin{pgfscope}%
\pgfsys@transformshift{0.846848in}{1.266573in}%
\pgfsys@useobject{currentmarker}{}%
\end{pgfscope}%
\begin{pgfscope}%
\pgfsys@transformshift{0.627499in}{1.452709in}%
\pgfsys@useobject{currentmarker}{}%
\end{pgfscope}%
\begin{pgfscope}%
\pgfsys@transformshift{0.969057in}{1.101489in}%
\pgfsys@useobject{currentmarker}{}%
\end{pgfscope}%
\begin{pgfscope}%
\pgfsys@transformshift{0.787932in}{1.319357in}%
\pgfsys@useobject{currentmarker}{}%
\end{pgfscope}%
\begin{pgfscope}%
\pgfsys@transformshift{1.219210in}{1.111563in}%
\pgfsys@useobject{currentmarker}{}%
\end{pgfscope}%
\begin{pgfscope}%
\pgfsys@transformshift{1.080473in}{1.063267in}%
\pgfsys@useobject{currentmarker}{}%
\end{pgfscope}%
\begin{pgfscope}%
\pgfsys@transformshift{0.889067in}{1.394342in}%
\pgfsys@useobject{currentmarker}{}%
\end{pgfscope}%
\begin{pgfscope}%
\pgfsys@transformshift{0.616849in}{1.463284in}%
\pgfsys@useobject{currentmarker}{}%
\end{pgfscope}%
\begin{pgfscope}%
\pgfsys@transformshift{1.017337in}{1.132805in}%
\pgfsys@useobject{currentmarker}{}%
\end{pgfscope}%
\begin{pgfscope}%
\pgfsys@transformshift{0.983403in}{1.561390in}%
\pgfsys@useobject{currentmarker}{}%
\end{pgfscope}%
\begin{pgfscope}%
\pgfsys@transformshift{0.584076in}{1.312294in}%
\pgfsys@useobject{currentmarker}{}%
\end{pgfscope}%
\begin{pgfscope}%
\pgfsys@transformshift{0.615050in}{1.366124in}%
\pgfsys@useobject{currentmarker}{}%
\end{pgfscope}%
\begin{pgfscope}%
\pgfsys@transformshift{0.913681in}{1.486215in}%
\pgfsys@useobject{currentmarker}{}%
\end{pgfscope}%
\begin{pgfscope}%
\pgfsys@transformshift{1.260918in}{1.435255in}%
\pgfsys@useobject{currentmarker}{}%
\end{pgfscope}%
\begin{pgfscope}%
\pgfsys@transformshift{0.579842in}{1.089687in}%
\pgfsys@useobject{currentmarker}{}%
\end{pgfscope}%
\begin{pgfscope}%
\pgfsys@transformshift{0.908469in}{1.320425in}%
\pgfsys@useobject{currentmarker}{}%
\end{pgfscope}%
\begin{pgfscope}%
\pgfsys@transformshift{0.984097in}{1.527198in}%
\pgfsys@useobject{currentmarker}{}%
\end{pgfscope}%
\begin{pgfscope}%
\pgfsys@transformshift{0.874947in}{1.474790in}%
\pgfsys@useobject{currentmarker}{}%
\end{pgfscope}%
\begin{pgfscope}%
\pgfsys@transformshift{1.138540in}{1.074905in}%
\pgfsys@useobject{currentmarker}{}%
\end{pgfscope}%
\begin{pgfscope}%
\pgfsys@transformshift{1.252874in}{0.905123in}%
\pgfsys@useobject{currentmarker}{}%
\end{pgfscope}%
\begin{pgfscope}%
\pgfsys@transformshift{0.662155in}{1.317678in}%
\pgfsys@useobject{currentmarker}{}%
\end{pgfscope}%
\begin{pgfscope}%
\pgfsys@transformshift{0.914488in}{1.587934in}%
\pgfsys@useobject{currentmarker}{}%
\end{pgfscope}%
\begin{pgfscope}%
\pgfsys@transformshift{1.101264in}{1.063108in}%
\pgfsys@useobject{currentmarker}{}%
\end{pgfscope}%
\begin{pgfscope}%
\pgfsys@transformshift{0.616452in}{1.505311in}%
\pgfsys@useobject{currentmarker}{}%
\end{pgfscope}%
\begin{pgfscope}%
\pgfsys@transformshift{0.718649in}{0.980103in}%
\pgfsys@useobject{currentmarker}{}%
\end{pgfscope}%
\begin{pgfscope}%
\pgfsys@transformshift{0.775327in}{1.338578in}%
\pgfsys@useobject{currentmarker}{}%
\end{pgfscope}%
\begin{pgfscope}%
\pgfsys@transformshift{0.703976in}{1.593579in}%
\pgfsys@useobject{currentmarker}{}%
\end{pgfscope}%
\begin{pgfscope}%
\pgfsys@transformshift{0.802803in}{1.454056in}%
\pgfsys@useobject{currentmarker}{}%
\end{pgfscope}%
\begin{pgfscope}%
\pgfsys@transformshift{0.614795in}{1.510612in}%
\pgfsys@useobject{currentmarker}{}%
\end{pgfscope}%
\begin{pgfscope}%
\pgfsys@transformshift{1.025849in}{1.216874in}%
\pgfsys@useobject{currentmarker}{}%
\end{pgfscope}%
\begin{pgfscope}%
\pgfsys@transformshift{0.633348in}{1.554505in}%
\pgfsys@useobject{currentmarker}{}%
\end{pgfscope}%
\begin{pgfscope}%
\pgfsys@transformshift{0.942020in}{1.538600in}%
\pgfsys@useobject{currentmarker}{}%
\end{pgfscope}%
\begin{pgfscope}%
\pgfsys@transformshift{0.995484in}{1.164660in}%
\pgfsys@useobject{currentmarker}{}%
\end{pgfscope}%
\begin{pgfscope}%
\pgfsys@transformshift{1.024970in}{1.415504in}%
\pgfsys@useobject{currentmarker}{}%
\end{pgfscope}%
\begin{pgfscope}%
\pgfsys@transformshift{0.665114in}{1.288074in}%
\pgfsys@useobject{currentmarker}{}%
\end{pgfscope}%
\begin{pgfscope}%
\pgfsys@transformshift{1.143752in}{0.990882in}%
\pgfsys@useobject{currentmarker}{}%
\end{pgfscope}%
\begin{pgfscope}%
\pgfsys@transformshift{1.266201in}{1.316493in}%
\pgfsys@useobject{currentmarker}{}%
\end{pgfscope}%
\begin{pgfscope}%
\pgfsys@transformshift{0.699473in}{1.477436in}%
\pgfsys@useobject{currentmarker}{}%
\end{pgfscope}%
\begin{pgfscope}%
\pgfsys@transformshift{0.588226in}{1.419524in}%
\pgfsys@useobject{currentmarker}{}%
\end{pgfscope}%
\begin{pgfscope}%
\pgfsys@transformshift{0.536306in}{1.012077in}%
\pgfsys@useobject{currentmarker}{}%
\end{pgfscope}%
\begin{pgfscope}%
\pgfsys@transformshift{0.724725in}{0.982274in}%
\pgfsys@useobject{currentmarker}{}%
\end{pgfscope}%
\begin{pgfscope}%
\pgfsys@transformshift{0.695479in}{1.402595in}%
\pgfsys@useobject{currentmarker}{}%
\end{pgfscope}%
\begin{pgfscope}%
\pgfsys@transformshift{0.704571in}{1.410355in}%
\pgfsys@useobject{currentmarker}{}%
\end{pgfscope}%
\begin{pgfscope}%
\pgfsys@transformshift{0.638730in}{1.395713in}%
\pgfsys@useobject{currentmarker}{}%
\end{pgfscope}%
\begin{pgfscope}%
\pgfsys@transformshift{0.968873in}{1.118921in}%
\pgfsys@useobject{currentmarker}{}%
\end{pgfscope}%
\begin{pgfscope}%
\pgfsys@transformshift{1.205982in}{1.249350in}%
\pgfsys@useobject{currentmarker}{}%
\end{pgfscope}%
\begin{pgfscope}%
\pgfsys@transformshift{0.972881in}{1.187422in}%
\pgfsys@useobject{currentmarker}{}%
\end{pgfscope}%
\begin{pgfscope}%
\pgfsys@transformshift{0.987808in}{1.353209in}%
\pgfsys@useobject{currentmarker}{}%
\end{pgfscope}%
\begin{pgfscope}%
\pgfsys@transformshift{0.997651in}{1.458136in}%
\pgfsys@useobject{currentmarker}{}%
\end{pgfscope}%
\begin{pgfscope}%
\pgfsys@transformshift{0.919573in}{1.333414in}%
\pgfsys@useobject{currentmarker}{}%
\end{pgfscope}%
\begin{pgfscope}%
\pgfsys@transformshift{0.659010in}{1.260236in}%
\pgfsys@useobject{currentmarker}{}%
\end{pgfscope}%
\begin{pgfscope}%
\pgfsys@transformshift{1.035068in}{1.305204in}%
\pgfsys@useobject{currentmarker}{}%
\end{pgfscope}%
\begin{pgfscope}%
\pgfsys@transformshift{0.536873in}{1.404589in}%
\pgfsys@useobject{currentmarker}{}%
\end{pgfscope}%
\begin{pgfscope}%
\pgfsys@transformshift{0.947246in}{1.566510in}%
\pgfsys@useobject{currentmarker}{}%
\end{pgfscope}%
\begin{pgfscope}%
\pgfsys@transformshift{0.719768in}{1.521039in}%
\pgfsys@useobject{currentmarker}{}%
\end{pgfscope}%
\begin{pgfscope}%
\pgfsys@transformshift{1.072316in}{1.589989in}%
\pgfsys@useobject{currentmarker}{}%
\end{pgfscope}%
\begin{pgfscope}%
\pgfsys@transformshift{0.564405in}{1.378541in}%
\pgfsys@useobject{currentmarker}{}%
\end{pgfscope}%
\begin{pgfscope}%
\pgfsys@transformshift{0.666134in}{1.423278in}%
\pgfsys@useobject{currentmarker}{}%
\end{pgfscope}%
\begin{pgfscope}%
\pgfsys@transformshift{0.752639in}{1.514536in}%
\pgfsys@useobject{currentmarker}{}%
\end{pgfscope}%
\begin{pgfscope}%
\pgfsys@transformshift{0.677223in}{1.447508in}%
\pgfsys@useobject{currentmarker}{}%
\end{pgfscope}%
\begin{pgfscope}%
\pgfsys@transformshift{0.836637in}{1.557887in}%
\pgfsys@useobject{currentmarker}{}%
\end{pgfscope}%
\begin{pgfscope}%
\pgfsys@transformshift{0.902606in}{1.505522in}%
\pgfsys@useobject{currentmarker}{}%
\end{pgfscope}%
\begin{pgfscope}%
\pgfsys@transformshift{1.281766in}{0.994146in}%
\pgfsys@useobject{currentmarker}{}%
\end{pgfscope}%
\begin{pgfscope}%
\pgfsys@transformshift{1.338628in}{1.308460in}%
\pgfsys@useobject{currentmarker}{}%
\end{pgfscope}%
\begin{pgfscope}%
\pgfsys@transformshift{0.847471in}{1.260496in}%
\pgfsys@useobject{currentmarker}{}%
\end{pgfscope}%
\begin{pgfscope}%
\pgfsys@transformshift{0.704203in}{1.483524in}%
\pgfsys@useobject{currentmarker}{}%
\end{pgfscope}%
\begin{pgfscope}%
\pgfsys@transformshift{0.629368in}{1.465509in}%
\pgfsys@useobject{currentmarker}{}%
\end{pgfscope}%
\begin{pgfscope}%
\pgfsys@transformshift{0.683639in}{1.511849in}%
\pgfsys@useobject{currentmarker}{}%
\end{pgfscope}%
\begin{pgfscope}%
\pgfsys@transformshift{0.590223in}{1.473678in}%
\pgfsys@useobject{currentmarker}{}%
\end{pgfscope}%
\begin{pgfscope}%
\pgfsys@transformshift{0.727571in}{1.340218in}%
\pgfsys@useobject{currentmarker}{}%
\end{pgfscope}%
\begin{pgfscope}%
\pgfsys@transformshift{0.656957in}{1.525130in}%
\pgfsys@useobject{currentmarker}{}%
\end{pgfscope}%
\begin{pgfscope}%
\pgfsys@transformshift{0.858178in}{1.252056in}%
\pgfsys@useobject{currentmarker}{}%
\end{pgfscope}%
\begin{pgfscope}%
\pgfsys@transformshift{0.939613in}{1.059123in}%
\pgfsys@useobject{currentmarker}{}%
\end{pgfscope}%
\begin{pgfscope}%
\pgfsys@transformshift{0.644324in}{1.194043in}%
\pgfsys@useobject{currentmarker}{}%
\end{pgfscope}%
\begin{pgfscope}%
\pgfsys@transformshift{0.680580in}{1.441816in}%
\pgfsys@useobject{currentmarker}{}%
\end{pgfscope}%
\begin{pgfscope}%
\pgfsys@transformshift{0.596200in}{1.305253in}%
\pgfsys@useobject{currentmarker}{}%
\end{pgfscope}%
\begin{pgfscope}%
\pgfsys@transformshift{0.681954in}{1.471909in}%
\pgfsys@useobject{currentmarker}{}%
\end{pgfscope}%
\begin{pgfscope}%
\pgfsys@transformshift{0.779180in}{1.529934in}%
\pgfsys@useobject{currentmarker}{}%
\end{pgfscope}%
\begin{pgfscope}%
\pgfsys@transformshift{1.013456in}{1.200436in}%
\pgfsys@useobject{currentmarker}{}%
\end{pgfscope}%
\begin{pgfscope}%
\pgfsys@transformshift{0.649507in}{1.390589in}%
\pgfsys@useobject{currentmarker}{}%
\end{pgfscope}%
\begin{pgfscope}%
\pgfsys@transformshift{1.170491in}{1.077939in}%
\pgfsys@useobject{currentmarker}{}%
\end{pgfscope}%
\begin{pgfscope}%
\pgfsys@transformshift{0.929387in}{1.473158in}%
\pgfsys@useobject{currentmarker}{}%
\end{pgfscope}%
\begin{pgfscope}%
\pgfsys@transformshift{1.309581in}{1.314209in}%
\pgfsys@useobject{currentmarker}{}%
\end{pgfscope}%
\begin{pgfscope}%
\pgfsys@transformshift{0.545710in}{1.227700in}%
\pgfsys@useobject{currentmarker}{}%
\end{pgfscope}%
\begin{pgfscope}%
\pgfsys@transformshift{1.308080in}{1.327616in}%
\pgfsys@useobject{currentmarker}{}%
\end{pgfscope}%
\begin{pgfscope}%
\pgfsys@transformshift{1.333246in}{1.322792in}%
\pgfsys@useobject{currentmarker}{}%
\end{pgfscope}%
\begin{pgfscope}%
\pgfsys@transformshift{0.958407in}{0.685400in}%
\pgfsys@useobject{currentmarker}{}%
\end{pgfscope}%
\begin{pgfscope}%
\pgfsys@transformshift{0.581938in}{1.117202in}%
\pgfsys@useobject{currentmarker}{}%
\end{pgfscope}%
\begin{pgfscope}%
\pgfsys@transformshift{1.110271in}{1.567362in}%
\pgfsys@useobject{currentmarker}{}%
\end{pgfscope}%
\begin{pgfscope}%
\pgfsys@transformshift{1.127167in}{1.064172in}%
\pgfsys@useobject{currentmarker}{}%
\end{pgfscope}%
\begin{pgfscope}%
\pgfsys@transformshift{0.846749in}{1.418658in}%
\pgfsys@useobject{currentmarker}{}%
\end{pgfscope}%
\begin{pgfscope}%
\pgfsys@transformshift{0.588240in}{1.400888in}%
\pgfsys@useobject{currentmarker}{}%
\end{pgfscope}%
\begin{pgfscope}%
\pgfsys@transformshift{0.641066in}{1.404268in}%
\pgfsys@useobject{currentmarker}{}%
\end{pgfscope}%
\begin{pgfscope}%
\pgfsys@transformshift{1.128300in}{1.081467in}%
\pgfsys@useobject{currentmarker}{}%
\end{pgfscope}%
\begin{pgfscope}%
\pgfsys@transformshift{0.630770in}{1.367676in}%
\pgfsys@useobject{currentmarker}{}%
\end{pgfscope}%
\begin{pgfscope}%
\pgfsys@transformshift{1.172742in}{1.312476in}%
\pgfsys@useobject{currentmarker}{}%
\end{pgfscope}%
\begin{pgfscope}%
\pgfsys@transformshift{0.640103in}{0.641414in}%
\pgfsys@useobject{currentmarker}{}%
\end{pgfscope}%
\begin{pgfscope}%
\pgfsys@transformshift{1.606272in}{1.287129in}%
\pgfsys@useobject{currentmarker}{}%
\end{pgfscope}%
\begin{pgfscope}%
\pgfsys@transformshift{1.225285in}{1.319196in}%
\pgfsys@useobject{currentmarker}{}%
\end{pgfscope}%
\begin{pgfscope}%
\pgfsys@transformshift{0.702334in}{1.504396in}%
\pgfsys@useobject{currentmarker}{}%
\end{pgfscope}%
\begin{pgfscope}%
\pgfsys@transformshift{0.772707in}{1.358625in}%
\pgfsys@useobject{currentmarker}{}%
\end{pgfscope}%
\begin{pgfscope}%
\pgfsys@transformshift{1.046908in}{1.479260in}%
\pgfsys@useobject{currentmarker}{}%
\end{pgfscope}%
\begin{pgfscope}%
\pgfsys@transformshift{0.999690in}{0.873064in}%
\pgfsys@useobject{currentmarker}{}%
\end{pgfscope}%
\begin{pgfscope}%
\pgfsys@transformshift{1.102213in}{1.086946in}%
\pgfsys@useobject{currentmarker}{}%
\end{pgfscope}%
\begin{pgfscope}%
\pgfsys@transformshift{0.700478in}{1.273476in}%
\pgfsys@useobject{currentmarker}{}%
\end{pgfscope}%
\begin{pgfscope}%
\pgfsys@transformshift{0.583552in}{1.387069in}%
\pgfsys@useobject{currentmarker}{}%
\end{pgfscope}%
\begin{pgfscope}%
\pgfsys@transformshift{0.918624in}{0.796241in}%
\pgfsys@useobject{currentmarker}{}%
\end{pgfscope}%
\begin{pgfscope}%
\pgfsys@transformshift{0.868970in}{1.198328in}%
\pgfsys@useobject{currentmarker}{}%
\end{pgfscope}%
\begin{pgfscope}%
\pgfsys@transformshift{0.968349in}{1.396827in}%
\pgfsys@useobject{currentmarker}{}%
\end{pgfscope}%
\begin{pgfscope}%
\pgfsys@transformshift{0.811640in}{1.105068in}%
\pgfsys@useobject{currentmarker}{}%
\end{pgfscope}%
\begin{pgfscope}%
\pgfsys@transformshift{1.072245in}{0.871908in}%
\pgfsys@useobject{currentmarker}{}%
\end{pgfscope}%
\begin{pgfscope}%
\pgfsys@transformshift{0.983049in}{1.113216in}%
\pgfsys@useobject{currentmarker}{}%
\end{pgfscope}%
\begin{pgfscope}%
\pgfsys@transformshift{0.772297in}{1.362950in}%
\pgfsys@useobject{currentmarker}{}%
\end{pgfscope}%
\begin{pgfscope}%
\pgfsys@transformshift{1.197541in}{1.435199in}%
\pgfsys@useobject{currentmarker}{}%
\end{pgfscope}%
\begin{pgfscope}%
\pgfsys@transformshift{0.578978in}{1.451379in}%
\pgfsys@useobject{currentmarker}{}%
\end{pgfscope}%
\begin{pgfscope}%
\pgfsys@transformshift{0.676076in}{1.550676in}%
\pgfsys@useobject{currentmarker}{}%
\end{pgfscope}%
\begin{pgfscope}%
\pgfsys@transformshift{1.101901in}{1.367947in}%
\pgfsys@useobject{currentmarker}{}%
\end{pgfscope}%
\begin{pgfscope}%
\pgfsys@transformshift{0.718366in}{1.291209in}%
\pgfsys@useobject{currentmarker}{}%
\end{pgfscope}%
\begin{pgfscope}%
\pgfsys@transformshift{0.731268in}{1.398880in}%
\pgfsys@useobject{currentmarker}{}%
\end{pgfscope}%
\begin{pgfscope}%
\pgfsys@transformshift{1.329578in}{1.392156in}%
\pgfsys@useobject{currentmarker}{}%
\end{pgfscope}%
\begin{pgfscope}%
\pgfsys@transformshift{0.742187in}{1.409694in}%
\pgfsys@useobject{currentmarker}{}%
\end{pgfscope}%
\begin{pgfscope}%
\pgfsys@transformshift{0.616296in}{1.501772in}%
\pgfsys@useobject{currentmarker}{}%
\end{pgfscope}%
\begin{pgfscope}%
\pgfsys@transformshift{0.952118in}{1.325540in}%
\pgfsys@useobject{currentmarker}{}%
\end{pgfscope}%
\begin{pgfscope}%
\pgfsys@transformshift{0.921032in}{0.705945in}%
\pgfsys@useobject{currentmarker}{}%
\end{pgfscope}%
\begin{pgfscope}%
\pgfsys@transformshift{0.663160in}{1.321208in}%
\pgfsys@useobject{currentmarker}{}%
\end{pgfscope}%
\begin{pgfscope}%
\pgfsys@transformshift{0.920380in}{1.046003in}%
\pgfsys@useobject{currentmarker}{}%
\end{pgfscope}%
\begin{pgfscope}%
\pgfsys@transformshift{0.557862in}{1.329807in}%
\pgfsys@useobject{currentmarker}{}%
\end{pgfscope}%
\begin{pgfscope}%
\pgfsys@transformshift{0.601426in}{1.512856in}%
\pgfsys@useobject{currentmarker}{}%
\end{pgfscope}%
\begin{pgfscope}%
\pgfsys@transformshift{0.737882in}{1.436134in}%
\pgfsys@useobject{currentmarker}{}%
\end{pgfscope}%
\begin{pgfscope}%
\pgfsys@transformshift{0.903314in}{1.060200in}%
\pgfsys@useobject{currentmarker}{}%
\end{pgfscope}%
\begin{pgfscope}%
\pgfsys@transformshift{1.437412in}{1.313751in}%
\pgfsys@useobject{currentmarker}{}%
\end{pgfscope}%
\begin{pgfscope}%
\pgfsys@transformshift{0.728548in}{1.625403in}%
\pgfsys@useobject{currentmarker}{}%
\end{pgfscope}%
\begin{pgfscope}%
\pgfsys@transformshift{1.170576in}{1.063702in}%
\pgfsys@useobject{currentmarker}{}%
\end{pgfscope}%
\begin{pgfscope}%
\pgfsys@transformshift{0.846537in}{1.359322in}%
\pgfsys@useobject{currentmarker}{}%
\end{pgfscope}%
\begin{pgfscope}%
\pgfsys@transformshift{0.748532in}{1.480613in}%
\pgfsys@useobject{currentmarker}{}%
\end{pgfscope}%
\begin{pgfscope}%
\pgfsys@transformshift{0.808340in}{1.536223in}%
\pgfsys@useobject{currentmarker}{}%
\end{pgfscope}%
\begin{pgfscope}%
\pgfsys@transformshift{0.997693in}{1.502147in}%
\pgfsys@useobject{currentmarker}{}%
\end{pgfscope}%
\begin{pgfscope}%
\pgfsys@transformshift{1.297203in}{1.017870in}%
\pgfsys@useobject{currentmarker}{}%
\end{pgfscope}%
\begin{pgfscope}%
\pgfsys@transformshift{1.280080in}{1.421145in}%
\pgfsys@useobject{currentmarker}{}%
\end{pgfscope}%
\begin{pgfscope}%
\pgfsys@transformshift{0.973985in}{0.670337in}%
\pgfsys@useobject{currentmarker}{}%
\end{pgfscope}%
\begin{pgfscope}%
\pgfsys@transformshift{0.664831in}{1.550632in}%
\pgfsys@useobject{currentmarker}{}%
\end{pgfscope}%
\begin{pgfscope}%
\pgfsys@transformshift{0.921655in}{1.477570in}%
\pgfsys@useobject{currentmarker}{}%
\end{pgfscope}%
\begin{pgfscope}%
\pgfsys@transformshift{0.849157in}{1.282519in}%
\pgfsys@useobject{currentmarker}{}%
\end{pgfscope}%
\begin{pgfscope}%
\pgfsys@transformshift{0.837756in}{1.452344in}%
\pgfsys@useobject{currentmarker}{}%
\end{pgfscope}%
\begin{pgfscope}%
\pgfsys@transformshift{0.592829in}{1.508478in}%
\pgfsys@useobject{currentmarker}{}%
\end{pgfscope}%
\begin{pgfscope}%
\pgfsys@transformshift{0.951708in}{1.091651in}%
\pgfsys@useobject{currentmarker}{}%
\end{pgfscope}%
\begin{pgfscope}%
\pgfsys@transformshift{0.974311in}{1.482814in}%
\pgfsys@useobject{currentmarker}{}%
\end{pgfscope}%
\begin{pgfscope}%
\pgfsys@transformshift{0.795537in}{1.284180in}%
\pgfsys@useobject{currentmarker}{}%
\end{pgfscope}%
\begin{pgfscope}%
\pgfsys@transformshift{0.572888in}{1.047459in}%
\pgfsys@useobject{currentmarker}{}%
\end{pgfscope}%
\begin{pgfscope}%
\pgfsys@transformshift{1.024843in}{1.488582in}%
\pgfsys@useobject{currentmarker}{}%
\end{pgfscope}%
\begin{pgfscope}%
\pgfsys@transformshift{1.125411in}{1.123868in}%
\pgfsys@useobject{currentmarker}{}%
\end{pgfscope}%
\begin{pgfscope}%
\pgfsys@transformshift{1.194170in}{1.089043in}%
\pgfsys@useobject{currentmarker}{}%
\end{pgfscope}%
\begin{pgfscope}%
\pgfsys@transformshift{1.195261in}{1.411466in}%
\pgfsys@useobject{currentmarker}{}%
\end{pgfscope}%
\begin{pgfscope}%
\pgfsys@transformshift{0.810819in}{1.150511in}%
\pgfsys@useobject{currentmarker}{}%
\end{pgfscope}%
\begin{pgfscope}%
\pgfsys@transformshift{0.954795in}{1.292077in}%
\pgfsys@useobject{currentmarker}{}%
\end{pgfscope}%
\begin{pgfscope}%
\pgfsys@transformshift{0.660058in}{1.115169in}%
\pgfsys@useobject{currentmarker}{}%
\end{pgfscope}%
\begin{pgfscope}%
\pgfsys@transformshift{1.010411in}{1.451283in}%
\pgfsys@useobject{currentmarker}{}%
\end{pgfscope}%
\begin{pgfscope}%
\pgfsys@transformshift{0.939995in}{1.083404in}%
\pgfsys@useobject{currentmarker}{}%
\end{pgfscope}%
\begin{pgfscope}%
\pgfsys@transformshift{0.701583in}{1.019182in}%
\pgfsys@useobject{currentmarker}{}%
\end{pgfscope}%
\begin{pgfscope}%
\pgfsys@transformshift{0.855969in}{0.804172in}%
\pgfsys@useobject{currentmarker}{}%
\end{pgfscope}%
\begin{pgfscope}%
\pgfsys@transformshift{0.693397in}{1.586500in}%
\pgfsys@useobject{currentmarker}{}%
\end{pgfscope}%
\begin{pgfscope}%
\pgfsys@transformshift{0.935831in}{1.498126in}%
\pgfsys@useobject{currentmarker}{}%
\end{pgfscope}%
\begin{pgfscope}%
\pgfsys@transformshift{0.931540in}{1.426588in}%
\pgfsys@useobject{currentmarker}{}%
\end{pgfscope}%
\begin{pgfscope}%
\pgfsys@transformshift{0.968320in}{1.536546in}%
\pgfsys@useobject{currentmarker}{}%
\end{pgfscope}%
\begin{pgfscope}%
\pgfsys@transformshift{1.021869in}{1.597856in}%
\pgfsys@useobject{currentmarker}{}%
\end{pgfscope}%
\begin{pgfscope}%
\pgfsys@transformshift{0.760485in}{1.370167in}%
\pgfsys@useobject{currentmarker}{}%
\end{pgfscope}%
\begin{pgfscope}%
\pgfsys@transformshift{1.382844in}{1.265218in}%
\pgfsys@useobject{currentmarker}{}%
\end{pgfscope}%
\begin{pgfscope}%
\pgfsys@transformshift{1.798089in}{1.304264in}%
\pgfsys@useobject{currentmarker}{}%
\end{pgfscope}%
\begin{pgfscope}%
\pgfsys@transformshift{0.667069in}{1.406087in}%
\pgfsys@useobject{currentmarker}{}%
\end{pgfscope}%
\begin{pgfscope}%
\pgfsys@transformshift{1.034799in}{1.462739in}%
\pgfsys@useobject{currentmarker}{}%
\end{pgfscope}%
\begin{pgfscope}%
\pgfsys@transformshift{0.623873in}{1.469540in}%
\pgfsys@useobject{currentmarker}{}%
\end{pgfscope}%
\begin{pgfscope}%
\pgfsys@transformshift{0.821044in}{1.384150in}%
\pgfsys@useobject{currentmarker}{}%
\end{pgfscope}%
\begin{pgfscope}%
\pgfsys@transformshift{0.743688in}{1.655738in}%
\pgfsys@useobject{currentmarker}{}%
\end{pgfscope}%
\begin{pgfscope}%
\pgfsys@transformshift{0.827927in}{1.278973in}%
\pgfsys@useobject{currentmarker}{}%
\end{pgfscope}%
\begin{pgfscope}%
\pgfsys@transformshift{1.024956in}{1.390843in}%
\pgfsys@useobject{currentmarker}{}%
\end{pgfscope}%
\begin{pgfscope}%
\pgfsys@transformshift{0.787691in}{1.211389in}%
\pgfsys@useobject{currentmarker}{}%
\end{pgfscope}%
\begin{pgfscope}%
\pgfsys@transformshift{1.294809in}{0.957329in}%
\pgfsys@useobject{currentmarker}{}%
\end{pgfscope}%
\begin{pgfscope}%
\pgfsys@transformshift{1.000044in}{1.553548in}%
\pgfsys@useobject{currentmarker}{}%
\end{pgfscope}%
\begin{pgfscope}%
\pgfsys@transformshift{0.736026in}{1.124560in}%
\pgfsys@useobject{currentmarker}{}%
\end{pgfscope}%
\begin{pgfscope}%
\pgfsys@transformshift{1.017124in}{1.515305in}%
\pgfsys@useobject{currentmarker}{}%
\end{pgfscope}%
\begin{pgfscope}%
\pgfsys@transformshift{0.648898in}{1.371953in}%
\pgfsys@useobject{currentmarker}{}%
\end{pgfscope}%
\begin{pgfscope}%
\pgfsys@transformshift{0.850488in}{1.516489in}%
\pgfsys@useobject{currentmarker}{}%
\end{pgfscope}%
\begin{pgfscope}%
\pgfsys@transformshift{1.022393in}{1.197964in}%
\pgfsys@useobject{currentmarker}{}%
\end{pgfscope}%
\begin{pgfscope}%
\pgfsys@transformshift{0.886673in}{1.293344in}%
\pgfsys@useobject{currentmarker}{}%
\end{pgfscope}%
\begin{pgfscope}%
\pgfsys@transformshift{0.822616in}{1.494128in}%
\pgfsys@useobject{currentmarker}{}%
\end{pgfscope}%
\begin{pgfscope}%
\pgfsys@transformshift{0.785142in}{1.404153in}%
\pgfsys@useobject{currentmarker}{}%
\end{pgfscope}%
\begin{pgfscope}%
\pgfsys@transformshift{0.778656in}{1.515278in}%
\pgfsys@useobject{currentmarker}{}%
\end{pgfscope}%
\begin{pgfscope}%
\pgfsys@transformshift{1.062459in}{1.003389in}%
\pgfsys@useobject{currentmarker}{}%
\end{pgfscope}%
\begin{pgfscope}%
\pgfsys@transformshift{1.081408in}{1.395958in}%
\pgfsys@useobject{currentmarker}{}%
\end{pgfscope}%
\begin{pgfscope}%
\pgfsys@transformshift{0.948082in}{1.223355in}%
\pgfsys@useobject{currentmarker}{}%
\end{pgfscope}%
\begin{pgfscope}%
\pgfsys@transformshift{0.993671in}{0.790325in}%
\pgfsys@useobject{currentmarker}{}%
\end{pgfscope}%
\begin{pgfscope}%
\pgfsys@transformshift{1.124547in}{1.631049in}%
\pgfsys@useobject{currentmarker}{}%
\end{pgfscope}%
\begin{pgfscope}%
\pgfsys@transformshift{0.578610in}{1.565597in}%
\pgfsys@useobject{currentmarker}{}%
\end{pgfscope}%
\begin{pgfscope}%
\pgfsys@transformshift{0.874210in}{1.228171in}%
\pgfsys@useobject{currentmarker}{}%
\end{pgfscope}%
\begin{pgfscope}%
\pgfsys@transformshift{0.957203in}{1.412273in}%
\pgfsys@useobject{currentmarker}{}%
\end{pgfscope}%
\begin{pgfscope}%
\pgfsys@transformshift{0.919743in}{1.090231in}%
\pgfsys@useobject{currentmarker}{}%
\end{pgfscope}%
\begin{pgfscope}%
\pgfsys@transformshift{0.884535in}{1.409795in}%
\pgfsys@useobject{currentmarker}{}%
\end{pgfscope}%
\begin{pgfscope}%
\pgfsys@transformshift{0.622825in}{1.390259in}%
\pgfsys@useobject{currentmarker}{}%
\end{pgfscope}%
\begin{pgfscope}%
\pgfsys@transformshift{0.861294in}{1.168446in}%
\pgfsys@useobject{currentmarker}{}%
\end{pgfscope}%
\begin{pgfscope}%
\pgfsys@transformshift{0.781176in}{1.473204in}%
\pgfsys@useobject{currentmarker}{}%
\end{pgfscope}%
\begin{pgfscope}%
\pgfsys@transformshift{0.686627in}{1.521981in}%
\pgfsys@useobject{currentmarker}{}%
\end{pgfscope}%
\begin{pgfscope}%
\pgfsys@transformshift{0.627216in}{1.465893in}%
\pgfsys@useobject{currentmarker}{}%
\end{pgfscope}%
\begin{pgfscope}%
\pgfsys@transformshift{0.785935in}{1.232307in}%
\pgfsys@useobject{currentmarker}{}%
\end{pgfscope}%
\begin{pgfscope}%
\pgfsys@transformshift{1.299837in}{1.164631in}%
\pgfsys@useobject{currentmarker}{}%
\end{pgfscope}%
\begin{pgfscope}%
\pgfsys@transformshift{0.887013in}{1.305062in}%
\pgfsys@useobject{currentmarker}{}%
\end{pgfscope}%
\begin{pgfscope}%
\pgfsys@transformshift{0.887679in}{1.017944in}%
\pgfsys@useobject{currentmarker}{}%
\end{pgfscope}%
\begin{pgfscope}%
\pgfsys@transformshift{0.807788in}{0.925541in}%
\pgfsys@useobject{currentmarker}{}%
\end{pgfscope}%
\begin{pgfscope}%
\pgfsys@transformshift{0.632073in}{1.393775in}%
\pgfsys@useobject{currentmarker}{}%
\end{pgfscope}%
\begin{pgfscope}%
\pgfsys@transformshift{0.672975in}{1.511688in}%
\pgfsys@useobject{currentmarker}{}%
\end{pgfscope}%
\begin{pgfscope}%
\pgfsys@transformshift{0.691910in}{1.513413in}%
\pgfsys@useobject{currentmarker}{}%
\end{pgfscope}%
\begin{pgfscope}%
\pgfsys@transformshift{1.230186in}{1.429205in}%
\pgfsys@useobject{currentmarker}{}%
\end{pgfscope}%
\begin{pgfscope}%
\pgfsys@transformshift{0.946977in}{1.114618in}%
\pgfsys@useobject{currentmarker}{}%
\end{pgfscope}%
\begin{pgfscope}%
\pgfsys@transformshift{0.811300in}{1.254642in}%
\pgfsys@useobject{currentmarker}{}%
\end{pgfscope}%
\begin{pgfscope}%
\pgfsys@transformshift{1.092016in}{1.545474in}%
\pgfsys@useobject{currentmarker}{}%
\end{pgfscope}%
\begin{pgfscope}%
\pgfsys@transformshift{1.023243in}{1.217660in}%
\pgfsys@useobject{currentmarker}{}%
\end{pgfscope}%
\begin{pgfscope}%
\pgfsys@transformshift{0.825010in}{1.166174in}%
\pgfsys@useobject{currentmarker}{}%
\end{pgfscope}%
\begin{pgfscope}%
\pgfsys@transformshift{0.877892in}{0.946539in}%
\pgfsys@useobject{currentmarker}{}%
\end{pgfscope}%
\begin{pgfscope}%
\pgfsys@transformshift{1.030621in}{1.535139in}%
\pgfsys@useobject{currentmarker}{}%
\end{pgfscope}%
\begin{pgfscope}%
\pgfsys@transformshift{0.683951in}{1.541990in}%
\pgfsys@useobject{currentmarker}{}%
\end{pgfscope}%
\begin{pgfscope}%
\pgfsys@transformshift{0.677308in}{1.397528in}%
\pgfsys@useobject{currentmarker}{}%
\end{pgfscope}%
\begin{pgfscope}%
\pgfsys@transformshift{1.014207in}{1.599369in}%
\pgfsys@useobject{currentmarker}{}%
\end{pgfscope}%
\begin{pgfscope}%
\pgfsys@transformshift{0.895511in}{1.536679in}%
\pgfsys@useobject{currentmarker}{}%
\end{pgfscope}%
\begin{pgfscope}%
\pgfsys@transformshift{1.250240in}{0.971963in}%
\pgfsys@useobject{currentmarker}{}%
\end{pgfscope}%
\begin{pgfscope}%
\pgfsys@transformshift{0.935506in}{1.098944in}%
\pgfsys@useobject{currentmarker}{}%
\end{pgfscope}%
\begin{pgfscope}%
\pgfsys@transformshift{0.601935in}{1.362281in}%
\pgfsys@useobject{currentmarker}{}%
\end{pgfscope}%
\begin{pgfscope}%
\pgfsys@transformshift{1.117551in}{1.337399in}%
\pgfsys@useobject{currentmarker}{}%
\end{pgfscope}%
\begin{pgfscope}%
\pgfsys@transformshift{0.678597in}{1.631528in}%
\pgfsys@useobject{currentmarker}{}%
\end{pgfscope}%
\begin{pgfscope}%
\pgfsys@transformshift{0.708098in}{1.476561in}%
\pgfsys@useobject{currentmarker}{}%
\end{pgfscope}%
\begin{pgfscope}%
\pgfsys@transformshift{0.602403in}{1.444425in}%
\pgfsys@useobject{currentmarker}{}%
\end{pgfscope}%
\begin{pgfscope}%
\pgfsys@transformshift{0.956962in}{1.280249in}%
\pgfsys@useobject{currentmarker}{}%
\end{pgfscope}%
\begin{pgfscope}%
\pgfsys@transformshift{0.729030in}{1.371006in}%
\pgfsys@useobject{currentmarker}{}%
\end{pgfscope}%
\begin{pgfscope}%
\pgfsys@transformshift{0.641463in}{1.514895in}%
\pgfsys@useobject{currentmarker}{}%
\end{pgfscope}%
\begin{pgfscope}%
\pgfsys@transformshift{0.703821in}{1.364606in}%
\pgfsys@useobject{currentmarker}{}%
\end{pgfscope}%
\begin{pgfscope}%
\pgfsys@transformshift{0.681430in}{1.386457in}%
\pgfsys@useobject{currentmarker}{}%
\end{pgfscope}%
\begin{pgfscope}%
\pgfsys@transformshift{0.955829in}{0.861542in}%
\pgfsys@useobject{currentmarker}{}%
\end{pgfscope}%
\begin{pgfscope}%
\pgfsys@transformshift{1.428291in}{1.336427in}%
\pgfsys@useobject{currentmarker}{}%
\end{pgfscope}%
\begin{pgfscope}%
\pgfsys@transformshift{0.664066in}{1.562303in}%
\pgfsys@useobject{currentmarker}{}%
\end{pgfscope}%
\begin{pgfscope}%
\pgfsys@transformshift{0.726155in}{1.498344in}%
\pgfsys@useobject{currentmarker}{}%
\end{pgfscope}%
\begin{pgfscope}%
\pgfsys@transformshift{0.948351in}{0.766397in}%
\pgfsys@useobject{currentmarker}{}%
\end{pgfscope}%
\begin{pgfscope}%
\pgfsys@transformshift{1.115696in}{1.054869in}%
\pgfsys@useobject{currentmarker}{}%
\end{pgfscope}%
\begin{pgfscope}%
\pgfsys@transformshift{0.784349in}{1.514662in}%
\pgfsys@useobject{currentmarker}{}%
\end{pgfscope}%
\begin{pgfscope}%
\pgfsys@transformshift{0.677266in}{1.454344in}%
\pgfsys@useobject{currentmarker}{}%
\end{pgfscope}%
\begin{pgfscope}%
\pgfsys@transformshift{1.155478in}{1.096777in}%
\pgfsys@useobject{currentmarker}{}%
\end{pgfscope}%
\begin{pgfscope}%
\pgfsys@transformshift{0.882340in}{1.263058in}%
\pgfsys@useobject{currentmarker}{}%
\end{pgfscope}%
\begin{pgfscope}%
\pgfsys@transformshift{0.892055in}{0.938519in}%
\pgfsys@useobject{currentmarker}{}%
\end{pgfscope}%
\begin{pgfscope}%
\pgfsys@transformshift{0.791813in}{1.499408in}%
\pgfsys@useobject{currentmarker}{}%
\end{pgfscope}%
\begin{pgfscope}%
\pgfsys@transformshift{0.725192in}{1.446935in}%
\pgfsys@useobject{currentmarker}{}%
\end{pgfscope}%
\begin{pgfscope}%
\pgfsys@transformshift{0.852995in}{1.157660in}%
\pgfsys@useobject{currentmarker}{}%
\end{pgfscope}%
\begin{pgfscope}%
\pgfsys@transformshift{0.700436in}{1.194459in}%
\pgfsys@useobject{currentmarker}{}%
\end{pgfscope}%
\begin{pgfscope}%
\pgfsys@transformshift{0.983389in}{1.179996in}%
\pgfsys@useobject{currentmarker}{}%
\end{pgfscope}%
\begin{pgfscope}%
\pgfsys@transformshift{0.665199in}{1.514482in}%
\pgfsys@useobject{currentmarker}{}%
\end{pgfscope}%
\begin{pgfscope}%
\pgfsys@transformshift{1.461687in}{1.125354in}%
\pgfsys@useobject{currentmarker}{}%
\end{pgfscope}%
\begin{pgfscope}%
\pgfsys@transformshift{0.670893in}{1.214067in}%
\pgfsys@useobject{currentmarker}{}%
\end{pgfscope}%
\begin{pgfscope}%
\pgfsys@transformshift{0.818891in}{0.826773in}%
\pgfsys@useobject{currentmarker}{}%
\end{pgfscope}%
\begin{pgfscope}%
\pgfsys@transformshift{0.757865in}{1.413630in}%
\pgfsys@useobject{currentmarker}{}%
\end{pgfscope}%
\begin{pgfscope}%
\pgfsys@transformshift{0.966097in}{1.164242in}%
\pgfsys@useobject{currentmarker}{}%
\end{pgfscope}%
\begin{pgfscope}%
\pgfsys@transformshift{0.770625in}{1.489854in}%
\pgfsys@useobject{currentmarker}{}%
\end{pgfscope}%
\begin{pgfscope}%
\pgfsys@transformshift{0.649125in}{1.433044in}%
\pgfsys@useobject{currentmarker}{}%
\end{pgfscope}%
\begin{pgfscope}%
\pgfsys@transformshift{1.157447in}{1.425531in}%
\pgfsys@useobject{currentmarker}{}%
\end{pgfscope}%
\begin{pgfscope}%
\pgfsys@transformshift{0.785624in}{1.539491in}%
\pgfsys@useobject{currentmarker}{}%
\end{pgfscope}%
\begin{pgfscope}%
\pgfsys@transformshift{1.032377in}{1.213767in}%
\pgfsys@useobject{currentmarker}{}%
\end{pgfscope}%
\begin{pgfscope}%
\pgfsys@transformshift{0.874819in}{0.523238in}%
\pgfsys@useobject{currentmarker}{}%
\end{pgfscope}%
\begin{pgfscope}%
\pgfsys@transformshift{0.708820in}{1.417268in}%
\pgfsys@useobject{currentmarker}{}%
\end{pgfscope}%
\begin{pgfscope}%
\pgfsys@transformshift{0.664661in}{1.517898in}%
\pgfsys@useobject{currentmarker}{}%
\end{pgfscope}%
\begin{pgfscope}%
\pgfsys@transformshift{0.580097in}{1.406400in}%
\pgfsys@useobject{currentmarker}{}%
\end{pgfscope}%
\begin{pgfscope}%
\pgfsys@transformshift{0.930223in}{1.405623in}%
\pgfsys@useobject{currentmarker}{}%
\end{pgfscope}%
\begin{pgfscope}%
\pgfsys@transformshift{1.176056in}{0.987863in}%
\pgfsys@useobject{currentmarker}{}%
\end{pgfscope}%
\begin{pgfscope}%
\pgfsys@transformshift{0.790680in}{1.411930in}%
\pgfsys@useobject{currentmarker}{}%
\end{pgfscope}%
\begin{pgfscope}%
\pgfsys@transformshift{0.824995in}{1.279245in}%
\pgfsys@useobject{currentmarker}{}%
\end{pgfscope}%
\begin{pgfscope}%
\pgfsys@transformshift{0.903994in}{1.449769in}%
\pgfsys@useobject{currentmarker}{}%
\end{pgfscope}%
\begin{pgfscope}%
\pgfsys@transformshift{0.956750in}{1.544081in}%
\pgfsys@useobject{currentmarker}{}%
\end{pgfscope}%
\begin{pgfscope}%
\pgfsys@transformshift{0.556205in}{1.372784in}%
\pgfsys@useobject{currentmarker}{}%
\end{pgfscope}%
\begin{pgfscope}%
\pgfsys@transformshift{0.637710in}{1.301917in}%
\pgfsys@useobject{currentmarker}{}%
\end{pgfscope}%
\begin{pgfscope}%
\pgfsys@transformshift{1.014292in}{1.540407in}%
\pgfsys@useobject{currentmarker}{}%
\end{pgfscope}%
\begin{pgfscope}%
\pgfsys@transformshift{0.617330in}{1.446956in}%
\pgfsys@useobject{currentmarker}{}%
\end{pgfscope}%
\begin{pgfscope}%
\pgfsys@transformshift{0.935817in}{1.542779in}%
\pgfsys@useobject{currentmarker}{}%
\end{pgfscope}%
\begin{pgfscope}%
\pgfsys@transformshift{0.873969in}{1.294058in}%
\pgfsys@useobject{currentmarker}{}%
\end{pgfscope}%
\begin{pgfscope}%
\pgfsys@transformshift{1.163622in}{1.398254in}%
\pgfsys@useobject{currentmarker}{}%
\end{pgfscope}%
\begin{pgfscope}%
\pgfsys@transformshift{0.563413in}{1.402708in}%
\pgfsys@useobject{currentmarker}{}%
\end{pgfscope}%
\begin{pgfscope}%
\pgfsys@transformshift{0.702702in}{1.452971in}%
\pgfsys@useobject{currentmarker}{}%
\end{pgfscope}%
\begin{pgfscope}%
\pgfsys@transformshift{0.621154in}{1.441493in}%
\pgfsys@useobject{currentmarker}{}%
\end{pgfscope}%
\begin{pgfscope}%
\pgfsys@transformshift{0.554703in}{1.541656in}%
\pgfsys@useobject{currentmarker}{}%
\end{pgfscope}%
\begin{pgfscope}%
\pgfsys@transformshift{1.266569in}{1.414231in}%
\pgfsys@useobject{currentmarker}{}%
\end{pgfscope}%
\begin{pgfscope}%
\pgfsys@transformshift{1.649000in}{1.297945in}%
\pgfsys@useobject{currentmarker}{}%
\end{pgfscope}%
\begin{pgfscope}%
\pgfsys@transformshift{0.922731in}{1.025972in}%
\pgfsys@useobject{currentmarker}{}%
\end{pgfscope}%
\begin{pgfscope}%
\pgfsys@transformshift{1.194878in}{1.150771in}%
\pgfsys@useobject{currentmarker}{}%
\end{pgfscope}%
\begin{pgfscope}%
\pgfsys@transformshift{0.638361in}{1.145283in}%
\pgfsys@useobject{currentmarker}{}%
\end{pgfscope}%
\begin{pgfscope}%
\pgfsys@transformshift{0.879351in}{1.030929in}%
\pgfsys@useobject{currentmarker}{}%
\end{pgfscope}%
\begin{pgfscope}%
\pgfsys@transformshift{0.681416in}{1.443012in}%
\pgfsys@useobject{currentmarker}{}%
\end{pgfscope}%
\begin{pgfscope}%
\pgfsys@transformshift{0.680070in}{0.932616in}%
\pgfsys@useobject{currentmarker}{}%
\end{pgfscope}%
\begin{pgfscope}%
\pgfsys@transformshift{1.006276in}{1.508049in}%
\pgfsys@useobject{currentmarker}{}%
\end{pgfscope}%
\begin{pgfscope}%
\pgfsys@transformshift{0.981945in}{1.323678in}%
\pgfsys@useobject{currentmarker}{}%
\end{pgfscope}%
\begin{pgfscope}%
\pgfsys@transformshift{0.593325in}{1.444067in}%
\pgfsys@useobject{currentmarker}{}%
\end{pgfscope}%
\begin{pgfscope}%
\pgfsys@transformshift{0.691145in}{1.436165in}%
\pgfsys@useobject{currentmarker}{}%
\end{pgfscope}%
\begin{pgfscope}%
\pgfsys@transformshift{0.609682in}{1.615178in}%
\pgfsys@useobject{currentmarker}{}%
\end{pgfscope}%
\begin{pgfscope}%
\pgfsys@transformshift{0.672210in}{1.327525in}%
\pgfsys@useobject{currentmarker}{}%
\end{pgfscope}%
\begin{pgfscope}%
\pgfsys@transformshift{0.601978in}{1.490375in}%
\pgfsys@useobject{currentmarker}{}%
\end{pgfscope}%
\begin{pgfscope}%
\pgfsys@transformshift{0.921556in}{1.433093in}%
\pgfsys@useobject{currentmarker}{}%
\end{pgfscope}%
\begin{pgfscope}%
\pgfsys@transformshift{0.890795in}{0.928662in}%
\pgfsys@useobject{currentmarker}{}%
\end{pgfscope}%
\begin{pgfscope}%
\pgfsys@transformshift{0.655951in}{1.420923in}%
\pgfsys@useobject{currentmarker}{}%
\end{pgfscope}%
\begin{pgfscope}%
\pgfsys@transformshift{1.242762in}{1.304095in}%
\pgfsys@useobject{currentmarker}{}%
\end{pgfscope}%
\begin{pgfscope}%
\pgfsys@transformshift{0.730276in}{1.434638in}%
\pgfsys@useobject{currentmarker}{}%
\end{pgfscope}%
\begin{pgfscope}%
\pgfsys@transformshift{1.090855in}{1.270413in}%
\pgfsys@useobject{currentmarker}{}%
\end{pgfscope}%
\begin{pgfscope}%
\pgfsys@transformshift{0.750628in}{1.506525in}%
\pgfsys@useobject{currentmarker}{}%
\end{pgfscope}%
\begin{pgfscope}%
\pgfsys@transformshift{0.614625in}{1.273188in}%
\pgfsys@useobject{currentmarker}{}%
\end{pgfscope}%
\begin{pgfscope}%
\pgfsys@transformshift{0.704486in}{1.519183in}%
\pgfsys@useobject{currentmarker}{}%
\end{pgfscope}%
\begin{pgfscope}%
\pgfsys@transformshift{1.084142in}{1.289528in}%
\pgfsys@useobject{currentmarker}{}%
\end{pgfscope}%
\begin{pgfscope}%
\pgfsys@transformshift{0.904079in}{1.445203in}%
\pgfsys@useobject{currentmarker}{}%
\end{pgfscope}%
\begin{pgfscope}%
\pgfsys@transformshift{0.675906in}{1.279693in}%
\pgfsys@useobject{currentmarker}{}%
\end{pgfscope}%
\begin{pgfscope}%
\pgfsys@transformshift{0.701909in}{1.426937in}%
\pgfsys@useobject{currentmarker}{}%
\end{pgfscope}%
\begin{pgfscope}%
\pgfsys@transformshift{0.556757in}{1.007316in}%
\pgfsys@useobject{currentmarker}{}%
\end{pgfscope}%
\begin{pgfscope}%
\pgfsys@transformshift{0.905779in}{1.399719in}%
\pgfsys@useobject{currentmarker}{}%
\end{pgfscope}%
\begin{pgfscope}%
\pgfsys@transformshift{0.693822in}{1.489499in}%
\pgfsys@useobject{currentmarker}{}%
\end{pgfscope}%
\begin{pgfscope}%
\pgfsys@transformshift{1.246813in}{1.034196in}%
\pgfsys@useobject{currentmarker}{}%
\end{pgfscope}%
\begin{pgfscope}%
\pgfsys@transformshift{0.614937in}{1.438179in}%
\pgfsys@useobject{currentmarker}{}%
\end{pgfscope}%
\begin{pgfscope}%
\pgfsys@transformshift{0.903881in}{0.923489in}%
\pgfsys@useobject{currentmarker}{}%
\end{pgfscope}%
\end{pgfscope}%
\begin{pgfscope}%
\pgfpathrectangle{\pgfqpoint{0.519339in}{0.466613in}}{\pgfqpoint{1.278750in}{1.245750in}}%
\pgfusepath{clip}%
\pgfsetbuttcap%
\pgfsetroundjoin%
\definecolor{currentfill}{rgb}{0.298039,0.447059,0.690196}%
\pgfsetfillcolor{currentfill}%
\pgfsetfillopacity{0.150000}%
\pgfsetlinewidth{1.003750pt}%
\definecolor{currentstroke}{rgb}{1.000000,1.000000,1.000000}%
\pgfsetstrokecolor{currentstroke}%
\pgfsetstrokeopacity{0.150000}%
\pgfsetdash{}{0pt}%
\pgfsys@defobject{currentmarker}{\pgfqpoint{0.519339in}{0.988327in}}{\pgfqpoint{1.798089in}{1.430574in}}{%
\pgfpathmoveto{\pgfqpoint{0.519339in}{1.430574in}}%
\pgfpathlineto{\pgfqpoint{0.519339in}{1.371688in}}%
\pgfpathlineto{\pgfqpoint{0.532256in}{1.368827in}}%
\pgfpathlineto{\pgfqpoint{0.545173in}{1.365962in}}%
\pgfpathlineto{\pgfqpoint{0.558089in}{1.363097in}}%
\pgfpathlineto{\pgfqpoint{0.571006in}{1.360802in}}%
\pgfpathlineto{\pgfqpoint{0.583923in}{1.358010in}}%
\pgfpathlineto{\pgfqpoint{0.596839in}{1.355511in}}%
\pgfpathlineto{\pgfqpoint{0.609756in}{1.352514in}}%
\pgfpathlineto{\pgfqpoint{0.622673in}{1.349524in}}%
\pgfpathlineto{\pgfqpoint{0.635589in}{1.346694in}}%
\pgfpathlineto{\pgfqpoint{0.648506in}{1.344080in}}%
\pgfpathlineto{\pgfqpoint{0.661423in}{1.341105in}}%
\pgfpathlineto{\pgfqpoint{0.674339in}{1.338654in}}%
\pgfpathlineto{\pgfqpoint{0.687256in}{1.336219in}}%
\pgfpathlineto{\pgfqpoint{0.700173in}{1.333335in}}%
\pgfpathlineto{\pgfqpoint{0.713089in}{1.329824in}}%
\pgfpathlineto{\pgfqpoint{0.726006in}{1.326313in}}%
\pgfpathlineto{\pgfqpoint{0.738923in}{1.323197in}}%
\pgfpathlineto{\pgfqpoint{0.751839in}{1.320131in}}%
\pgfpathlineto{\pgfqpoint{0.764756in}{1.317080in}}%
\pgfpathlineto{\pgfqpoint{0.777673in}{1.314221in}}%
\pgfpathlineto{\pgfqpoint{0.790589in}{1.311274in}}%
\pgfpathlineto{\pgfqpoint{0.803506in}{1.308399in}}%
\pgfpathlineto{\pgfqpoint{0.816423in}{1.305645in}}%
\pgfpathlineto{\pgfqpoint{0.829339in}{1.302427in}}%
\pgfpathlineto{\pgfqpoint{0.842256in}{1.299090in}}%
\pgfpathlineto{\pgfqpoint{0.855173in}{1.295736in}}%
\pgfpathlineto{\pgfqpoint{0.868089in}{1.292417in}}%
\pgfpathlineto{\pgfqpoint{0.881006in}{1.288862in}}%
\pgfpathlineto{\pgfqpoint{0.893923in}{1.285081in}}%
\pgfpathlineto{\pgfqpoint{0.906839in}{1.281372in}}%
\pgfpathlineto{\pgfqpoint{0.919756in}{1.277583in}}%
\pgfpathlineto{\pgfqpoint{0.932673in}{1.273887in}}%
\pgfpathlineto{\pgfqpoint{0.945589in}{1.269889in}}%
\pgfpathlineto{\pgfqpoint{0.958506in}{1.266355in}}%
\pgfpathlineto{\pgfqpoint{0.971423in}{1.262659in}}%
\pgfpathlineto{\pgfqpoint{0.984339in}{1.258894in}}%
\pgfpathlineto{\pgfqpoint{0.997256in}{1.255156in}}%
\pgfpathlineto{\pgfqpoint{1.010173in}{1.251109in}}%
\pgfpathlineto{\pgfqpoint{1.023089in}{1.247266in}}%
\pgfpathlineto{\pgfqpoint{1.036006in}{1.243475in}}%
\pgfpathlineto{\pgfqpoint{1.048923in}{1.239513in}}%
\pgfpathlineto{\pgfqpoint{1.061839in}{1.235627in}}%
\pgfpathlineto{\pgfqpoint{1.074756in}{1.231249in}}%
\pgfpathlineto{\pgfqpoint{1.087673in}{1.227597in}}%
\pgfpathlineto{\pgfqpoint{1.100589in}{1.223542in}}%
\pgfpathlineto{\pgfqpoint{1.113506in}{1.219744in}}%
\pgfpathlineto{\pgfqpoint{1.126423in}{1.215603in}}%
\pgfpathlineto{\pgfqpoint{1.139339in}{1.211466in}}%
\pgfpathlineto{\pgfqpoint{1.152256in}{1.207133in}}%
\pgfpathlineto{\pgfqpoint{1.165173in}{1.202844in}}%
\pgfpathlineto{\pgfqpoint{1.178089in}{1.198588in}}%
\pgfpathlineto{\pgfqpoint{1.191006in}{1.194406in}}%
\pgfpathlineto{\pgfqpoint{1.203923in}{1.190392in}}%
\pgfpathlineto{\pgfqpoint{1.216839in}{1.185908in}}%
\pgfpathlineto{\pgfqpoint{1.229756in}{1.181851in}}%
\pgfpathlineto{\pgfqpoint{1.242673in}{1.177433in}}%
\pgfpathlineto{\pgfqpoint{1.255589in}{1.173064in}}%
\pgfpathlineto{\pgfqpoint{1.268506in}{1.168805in}}%
\pgfpathlineto{\pgfqpoint{1.281423in}{1.164547in}}%
\pgfpathlineto{\pgfqpoint{1.294339in}{1.160288in}}%
\pgfpathlineto{\pgfqpoint{1.307256in}{1.156030in}}%
\pgfpathlineto{\pgfqpoint{1.320173in}{1.151771in}}%
\pgfpathlineto{\pgfqpoint{1.333089in}{1.147513in}}%
\pgfpathlineto{\pgfqpoint{1.346006in}{1.143099in}}%
\pgfpathlineto{\pgfqpoint{1.358923in}{1.138575in}}%
\pgfpathlineto{\pgfqpoint{1.371839in}{1.133995in}}%
\pgfpathlineto{\pgfqpoint{1.384756in}{1.129751in}}%
\pgfpathlineto{\pgfqpoint{1.397673in}{1.125329in}}%
\pgfpathlineto{\pgfqpoint{1.410589in}{1.120885in}}%
\pgfpathlineto{\pgfqpoint{1.423506in}{1.116442in}}%
\pgfpathlineto{\pgfqpoint{1.436423in}{1.111999in}}%
\pgfpathlineto{\pgfqpoint{1.449339in}{1.107556in}}%
\pgfpathlineto{\pgfqpoint{1.462256in}{1.103114in}}%
\pgfpathlineto{\pgfqpoint{1.475173in}{1.098667in}}%
\pgfpathlineto{\pgfqpoint{1.488089in}{1.094144in}}%
\pgfpathlineto{\pgfqpoint{1.501006in}{1.089627in}}%
\pgfpathlineto{\pgfqpoint{1.513923in}{1.085096in}}%
\pgfpathlineto{\pgfqpoint{1.526839in}{1.080564in}}%
\pgfpathlineto{\pgfqpoint{1.539756in}{1.076034in}}%
\pgfpathlineto{\pgfqpoint{1.552673in}{1.071508in}}%
\pgfpathlineto{\pgfqpoint{1.565589in}{1.066981in}}%
\pgfpathlineto{\pgfqpoint{1.578506in}{1.062462in}}%
\pgfpathlineto{\pgfqpoint{1.591423in}{1.058144in}}%
\pgfpathlineto{\pgfqpoint{1.604339in}{1.053713in}}%
\pgfpathlineto{\pgfqpoint{1.617256in}{1.049223in}}%
\pgfpathlineto{\pgfqpoint{1.630173in}{1.044844in}}%
\pgfpathlineto{\pgfqpoint{1.643089in}{1.040535in}}%
\pgfpathlineto{\pgfqpoint{1.656006in}{1.036225in}}%
\pgfpathlineto{\pgfqpoint{1.668923in}{1.031915in}}%
\pgfpathlineto{\pgfqpoint{1.681839in}{1.027527in}}%
\pgfpathlineto{\pgfqpoint{1.694756in}{1.023084in}}%
\pgfpathlineto{\pgfqpoint{1.707673in}{1.018642in}}%
\pgfpathlineto{\pgfqpoint{1.720589in}{1.014201in}}%
\pgfpathlineto{\pgfqpoint{1.733506in}{1.009886in}}%
\pgfpathlineto{\pgfqpoint{1.746423in}{1.005598in}}%
\pgfpathlineto{\pgfqpoint{1.759339in}{1.001309in}}%
\pgfpathlineto{\pgfqpoint{1.772256in}{0.997021in}}%
\pgfpathlineto{\pgfqpoint{1.785173in}{0.992733in}}%
\pgfpathlineto{\pgfqpoint{1.798089in}{0.988327in}}%
\pgfpathlineto{\pgfqpoint{1.798089in}{1.132979in}}%
\pgfpathlineto{\pgfqpoint{1.798089in}{1.132979in}}%
\pgfpathlineto{\pgfqpoint{1.785173in}{1.135539in}}%
\pgfpathlineto{\pgfqpoint{1.772256in}{1.138098in}}%
\pgfpathlineto{\pgfqpoint{1.759339in}{1.140657in}}%
\pgfpathlineto{\pgfqpoint{1.746423in}{1.143217in}}%
\pgfpathlineto{\pgfqpoint{1.733506in}{1.145715in}}%
\pgfpathlineto{\pgfqpoint{1.720589in}{1.148137in}}%
\pgfpathlineto{\pgfqpoint{1.707673in}{1.150558in}}%
\pgfpathlineto{\pgfqpoint{1.694756in}{1.152980in}}%
\pgfpathlineto{\pgfqpoint{1.681839in}{1.155402in}}%
\pgfpathlineto{\pgfqpoint{1.668923in}{1.157823in}}%
\pgfpathlineto{\pgfqpoint{1.656006in}{1.160460in}}%
\pgfpathlineto{\pgfqpoint{1.643089in}{1.163126in}}%
\pgfpathlineto{\pgfqpoint{1.630173in}{1.165789in}}%
\pgfpathlineto{\pgfqpoint{1.617256in}{1.168451in}}%
\pgfpathlineto{\pgfqpoint{1.604339in}{1.170954in}}%
\pgfpathlineto{\pgfqpoint{1.591423in}{1.173453in}}%
\pgfpathlineto{\pgfqpoint{1.578506in}{1.176059in}}%
\pgfpathlineto{\pgfqpoint{1.565589in}{1.178661in}}%
\pgfpathlineto{\pgfqpoint{1.552673in}{1.181092in}}%
\pgfpathlineto{\pgfqpoint{1.539756in}{1.183478in}}%
\pgfpathlineto{\pgfqpoint{1.526839in}{1.186103in}}%
\pgfpathlineto{\pgfqpoint{1.513923in}{1.188843in}}%
\pgfpathlineto{\pgfqpoint{1.501006in}{1.191582in}}%
\pgfpathlineto{\pgfqpoint{1.488089in}{1.194306in}}%
\pgfpathlineto{\pgfqpoint{1.475173in}{1.196916in}}%
\pgfpathlineto{\pgfqpoint{1.462256in}{1.199492in}}%
\pgfpathlineto{\pgfqpoint{1.449339in}{1.201878in}}%
\pgfpathlineto{\pgfqpoint{1.436423in}{1.204507in}}%
\pgfpathlineto{\pgfqpoint{1.423506in}{1.207270in}}%
\pgfpathlineto{\pgfqpoint{1.410589in}{1.209739in}}%
\pgfpathlineto{\pgfqpoint{1.397673in}{1.212177in}}%
\pgfpathlineto{\pgfqpoint{1.384756in}{1.213948in}}%
\pgfpathlineto{\pgfqpoint{1.371839in}{1.216178in}}%
\pgfpathlineto{\pgfqpoint{1.358923in}{1.218935in}}%
\pgfpathlineto{\pgfqpoint{1.346006in}{1.221694in}}%
\pgfpathlineto{\pgfqpoint{1.333089in}{1.224453in}}%
\pgfpathlineto{\pgfqpoint{1.320173in}{1.227004in}}%
\pgfpathlineto{\pgfqpoint{1.307256in}{1.229895in}}%
\pgfpathlineto{\pgfqpoint{1.294339in}{1.232632in}}%
\pgfpathlineto{\pgfqpoint{1.281423in}{1.235131in}}%
\pgfpathlineto{\pgfqpoint{1.268506in}{1.237974in}}%
\pgfpathlineto{\pgfqpoint{1.255589in}{1.240808in}}%
\pgfpathlineto{\pgfqpoint{1.242673in}{1.243191in}}%
\pgfpathlineto{\pgfqpoint{1.229756in}{1.245754in}}%
\pgfpathlineto{\pgfqpoint{1.216839in}{1.248315in}}%
\pgfpathlineto{\pgfqpoint{1.203923in}{1.250886in}}%
\pgfpathlineto{\pgfqpoint{1.191006in}{1.253610in}}%
\pgfpathlineto{\pgfqpoint{1.178089in}{1.256145in}}%
\pgfpathlineto{\pgfqpoint{1.165173in}{1.258564in}}%
\pgfpathlineto{\pgfqpoint{1.152256in}{1.261425in}}%
\pgfpathlineto{\pgfqpoint{1.139339in}{1.264330in}}%
\pgfpathlineto{\pgfqpoint{1.126423in}{1.267024in}}%
\pgfpathlineto{\pgfqpoint{1.113506in}{1.269712in}}%
\pgfpathlineto{\pgfqpoint{1.100589in}{1.272535in}}%
\pgfpathlineto{\pgfqpoint{1.087673in}{1.275469in}}%
\pgfpathlineto{\pgfqpoint{1.074756in}{1.278438in}}%
\pgfpathlineto{\pgfqpoint{1.061839in}{1.281286in}}%
\pgfpathlineto{\pgfqpoint{1.048923in}{1.283697in}}%
\pgfpathlineto{\pgfqpoint{1.036006in}{1.286400in}}%
\pgfpathlineto{\pgfqpoint{1.023089in}{1.289135in}}%
\pgfpathlineto{\pgfqpoint{1.010173in}{1.292086in}}%
\pgfpathlineto{\pgfqpoint{0.997256in}{1.295076in}}%
\pgfpathlineto{\pgfqpoint{0.984339in}{1.298172in}}%
\pgfpathlineto{\pgfqpoint{0.971423in}{1.301196in}}%
\pgfpathlineto{\pgfqpoint{0.958506in}{1.304077in}}%
\pgfpathlineto{\pgfqpoint{0.945589in}{1.307155in}}%
\pgfpathlineto{\pgfqpoint{0.932673in}{1.310094in}}%
\pgfpathlineto{\pgfqpoint{0.919756in}{1.313188in}}%
\pgfpathlineto{\pgfqpoint{0.906839in}{1.316297in}}%
\pgfpathlineto{\pgfqpoint{0.893923in}{1.319507in}}%
\pgfpathlineto{\pgfqpoint{0.881006in}{1.322980in}}%
\pgfpathlineto{\pgfqpoint{0.868089in}{1.326379in}}%
\pgfpathlineto{\pgfqpoint{0.855173in}{1.329805in}}%
\pgfpathlineto{\pgfqpoint{0.842256in}{1.333337in}}%
\pgfpathlineto{\pgfqpoint{0.829339in}{1.336870in}}%
\pgfpathlineto{\pgfqpoint{0.816423in}{1.340512in}}%
\pgfpathlineto{\pgfqpoint{0.803506in}{1.344212in}}%
\pgfpathlineto{\pgfqpoint{0.790589in}{1.347814in}}%
\pgfpathlineto{\pgfqpoint{0.777673in}{1.351444in}}%
\pgfpathlineto{\pgfqpoint{0.764756in}{1.355470in}}%
\pgfpathlineto{\pgfqpoint{0.751839in}{1.358671in}}%
\pgfpathlineto{\pgfqpoint{0.738923in}{1.362144in}}%
\pgfpathlineto{\pgfqpoint{0.726006in}{1.365696in}}%
\pgfpathlineto{\pgfqpoint{0.713089in}{1.369433in}}%
\pgfpathlineto{\pgfqpoint{0.700173in}{1.373150in}}%
\pgfpathlineto{\pgfqpoint{0.687256in}{1.376748in}}%
\pgfpathlineto{\pgfqpoint{0.674339in}{1.380742in}}%
\pgfpathlineto{\pgfqpoint{0.661423in}{1.384874in}}%
\pgfpathlineto{\pgfqpoint{0.648506in}{1.389067in}}%
\pgfpathlineto{\pgfqpoint{0.635589in}{1.393159in}}%
\pgfpathlineto{\pgfqpoint{0.622673in}{1.397552in}}%
\pgfpathlineto{\pgfqpoint{0.609756in}{1.401484in}}%
\pgfpathlineto{\pgfqpoint{0.596839in}{1.405667in}}%
\pgfpathlineto{\pgfqpoint{0.583923in}{1.409425in}}%
\pgfpathlineto{\pgfqpoint{0.571006in}{1.413746in}}%
\pgfpathlineto{\pgfqpoint{0.558089in}{1.417584in}}%
\pgfpathlineto{\pgfqpoint{0.545173in}{1.421999in}}%
\pgfpathlineto{\pgfqpoint{0.532256in}{1.426218in}}%
\pgfpathlineto{\pgfqpoint{0.519339in}{1.430574in}}%
\pgfpathclose%
\pgfusepath{stroke,fill}%
}%
\begin{pgfscope}%
\pgfsys@transformshift{0.000000in}{0.000000in}%
\pgfsys@useobject{currentmarker}{}%
\end{pgfscope}%
\end{pgfscope}%
\begin{pgfscope}%
\pgfpathrectangle{\pgfqpoint{0.519339in}{0.466613in}}{\pgfqpoint{1.278750in}{1.245750in}}%
\pgfusepath{clip}%
\pgfsetroundcap%
\pgfsetroundjoin%
\pgfsetlinewidth{1.505625pt}%
\definecolor{currentstroke}{rgb}{0.298039,0.447059,0.690196}%
\pgfsetstrokecolor{currentstroke}%
\pgfsetdash{}{0pt}%
\pgfpathmoveto{\pgfqpoint{0.519339in}{1.400800in}}%
\pgfpathlineto{\pgfqpoint{0.532256in}{1.397416in}}%
\pgfpathlineto{\pgfqpoint{0.545173in}{1.394031in}}%
\pgfpathlineto{\pgfqpoint{0.558089in}{1.390647in}}%
\pgfpathlineto{\pgfqpoint{0.571006in}{1.387263in}}%
\pgfpathlineto{\pgfqpoint{0.583923in}{1.383879in}}%
\pgfpathlineto{\pgfqpoint{0.596839in}{1.380495in}}%
\pgfpathlineto{\pgfqpoint{0.609756in}{1.377111in}}%
\pgfpathlineto{\pgfqpoint{0.622673in}{1.373727in}}%
\pgfpathlineto{\pgfqpoint{0.635589in}{1.370343in}}%
\pgfpathlineto{\pgfqpoint{0.648506in}{1.366959in}}%
\pgfpathlineto{\pgfqpoint{0.661423in}{1.363575in}}%
\pgfpathlineto{\pgfqpoint{0.674339in}{1.360191in}}%
\pgfpathlineto{\pgfqpoint{0.687256in}{1.356807in}}%
\pgfpathlineto{\pgfqpoint{0.700173in}{1.353423in}}%
\pgfpathlineto{\pgfqpoint{0.713089in}{1.350039in}}%
\pgfpathlineto{\pgfqpoint{0.726006in}{1.346654in}}%
\pgfpathlineto{\pgfqpoint{0.738923in}{1.343270in}}%
\pgfpathlineto{\pgfqpoint{0.751839in}{1.339886in}}%
\pgfpathlineto{\pgfqpoint{0.764756in}{1.336502in}}%
\pgfpathlineto{\pgfqpoint{0.777673in}{1.333118in}}%
\pgfpathlineto{\pgfqpoint{0.790589in}{1.329734in}}%
\pgfpathlineto{\pgfqpoint{0.803506in}{1.326350in}}%
\pgfpathlineto{\pgfqpoint{0.816423in}{1.322966in}}%
\pgfpathlineto{\pgfqpoint{0.829339in}{1.319582in}}%
\pgfpathlineto{\pgfqpoint{0.842256in}{1.316198in}}%
\pgfpathlineto{\pgfqpoint{0.855173in}{1.312814in}}%
\pgfpathlineto{\pgfqpoint{0.868089in}{1.309430in}}%
\pgfpathlineto{\pgfqpoint{0.881006in}{1.306046in}}%
\pgfpathlineto{\pgfqpoint{0.893923in}{1.302662in}}%
\pgfpathlineto{\pgfqpoint{0.906839in}{1.299278in}}%
\pgfpathlineto{\pgfqpoint{0.919756in}{1.295893in}}%
\pgfpathlineto{\pgfqpoint{0.932673in}{1.292509in}}%
\pgfpathlineto{\pgfqpoint{0.945589in}{1.289125in}}%
\pgfpathlineto{\pgfqpoint{0.958506in}{1.285741in}}%
\pgfpathlineto{\pgfqpoint{0.971423in}{1.282357in}}%
\pgfpathlineto{\pgfqpoint{0.984339in}{1.278973in}}%
\pgfpathlineto{\pgfqpoint{0.997256in}{1.275589in}}%
\pgfpathlineto{\pgfqpoint{1.010173in}{1.272205in}}%
\pgfpathlineto{\pgfqpoint{1.023089in}{1.268821in}}%
\pgfpathlineto{\pgfqpoint{1.036006in}{1.265437in}}%
\pgfpathlineto{\pgfqpoint{1.048923in}{1.262053in}}%
\pgfpathlineto{\pgfqpoint{1.061839in}{1.258669in}}%
\pgfpathlineto{\pgfqpoint{1.074756in}{1.255285in}}%
\pgfpathlineto{\pgfqpoint{1.087673in}{1.251901in}}%
\pgfpathlineto{\pgfqpoint{1.100589in}{1.248517in}}%
\pgfpathlineto{\pgfqpoint{1.113506in}{1.245132in}}%
\pgfpathlineto{\pgfqpoint{1.126423in}{1.241748in}}%
\pgfpathlineto{\pgfqpoint{1.139339in}{1.238364in}}%
\pgfpathlineto{\pgfqpoint{1.152256in}{1.234980in}}%
\pgfpathlineto{\pgfqpoint{1.165173in}{1.231596in}}%
\pgfpathlineto{\pgfqpoint{1.178089in}{1.228212in}}%
\pgfpathlineto{\pgfqpoint{1.191006in}{1.224828in}}%
\pgfpathlineto{\pgfqpoint{1.203923in}{1.221444in}}%
\pgfpathlineto{\pgfqpoint{1.216839in}{1.218060in}}%
\pgfpathlineto{\pgfqpoint{1.229756in}{1.214676in}}%
\pgfpathlineto{\pgfqpoint{1.242673in}{1.211292in}}%
\pgfpathlineto{\pgfqpoint{1.255589in}{1.207908in}}%
\pgfpathlineto{\pgfqpoint{1.268506in}{1.204524in}}%
\pgfpathlineto{\pgfqpoint{1.281423in}{1.201140in}}%
\pgfpathlineto{\pgfqpoint{1.294339in}{1.197756in}}%
\pgfpathlineto{\pgfqpoint{1.307256in}{1.194371in}}%
\pgfpathlineto{\pgfqpoint{1.320173in}{1.190987in}}%
\pgfpathlineto{\pgfqpoint{1.333089in}{1.187603in}}%
\pgfpathlineto{\pgfqpoint{1.346006in}{1.184219in}}%
\pgfpathlineto{\pgfqpoint{1.358923in}{1.180835in}}%
\pgfpathlineto{\pgfqpoint{1.371839in}{1.177451in}}%
\pgfpathlineto{\pgfqpoint{1.384756in}{1.174067in}}%
\pgfpathlineto{\pgfqpoint{1.397673in}{1.170683in}}%
\pgfpathlineto{\pgfqpoint{1.410589in}{1.167299in}}%
\pgfpathlineto{\pgfqpoint{1.423506in}{1.163915in}}%
\pgfpathlineto{\pgfqpoint{1.436423in}{1.160531in}}%
\pgfpathlineto{\pgfqpoint{1.449339in}{1.157147in}}%
\pgfpathlineto{\pgfqpoint{1.462256in}{1.153763in}}%
\pgfpathlineto{\pgfqpoint{1.475173in}{1.150379in}}%
\pgfpathlineto{\pgfqpoint{1.488089in}{1.146995in}}%
\pgfpathlineto{\pgfqpoint{1.501006in}{1.143610in}}%
\pgfpathlineto{\pgfqpoint{1.513923in}{1.140226in}}%
\pgfpathlineto{\pgfqpoint{1.526839in}{1.136842in}}%
\pgfpathlineto{\pgfqpoint{1.539756in}{1.133458in}}%
\pgfpathlineto{\pgfqpoint{1.552673in}{1.130074in}}%
\pgfpathlineto{\pgfqpoint{1.565589in}{1.126690in}}%
\pgfpathlineto{\pgfqpoint{1.578506in}{1.123306in}}%
\pgfpathlineto{\pgfqpoint{1.591423in}{1.119922in}}%
\pgfpathlineto{\pgfqpoint{1.604339in}{1.116538in}}%
\pgfpathlineto{\pgfqpoint{1.617256in}{1.113154in}}%
\pgfpathlineto{\pgfqpoint{1.630173in}{1.109770in}}%
\pgfpathlineto{\pgfqpoint{1.643089in}{1.106386in}}%
\pgfpathlineto{\pgfqpoint{1.656006in}{1.103002in}}%
\pgfpathlineto{\pgfqpoint{1.668923in}{1.099618in}}%
\pgfpathlineto{\pgfqpoint{1.681839in}{1.096233in}}%
\pgfpathlineto{\pgfqpoint{1.694756in}{1.092849in}}%
\pgfpathlineto{\pgfqpoint{1.707673in}{1.089465in}}%
\pgfpathlineto{\pgfqpoint{1.720589in}{1.086081in}}%
\pgfpathlineto{\pgfqpoint{1.733506in}{1.082697in}}%
\pgfpathlineto{\pgfqpoint{1.746423in}{1.079313in}}%
\pgfpathlineto{\pgfqpoint{1.759339in}{1.075929in}}%
\pgfpathlineto{\pgfqpoint{1.772256in}{1.072545in}}%
\pgfpathlineto{\pgfqpoint{1.785173in}{1.069161in}}%
\pgfpathlineto{\pgfqpoint{1.798089in}{1.065777in}}%
\pgfusepath{stroke}%
\end{pgfscope}%
\begin{pgfscope}%
\pgfsetrectcap%
\pgfsetmiterjoin%
\pgfsetlinewidth{0.752812pt}%
\definecolor{currentstroke}{rgb}{0.700000,0.700000,0.700000}%
\pgfsetstrokecolor{currentstroke}%
\pgfsetdash{}{0pt}%
\pgfpathmoveto{\pgfqpoint{0.519339in}{0.466613in}}%
\pgfpathlineto{\pgfqpoint{0.519339in}{1.712363in}}%
\pgfusepath{stroke}%
\end{pgfscope}%
\begin{pgfscope}%
\pgfsetrectcap%
\pgfsetmiterjoin%
\pgfsetlinewidth{0.752812pt}%
\definecolor{currentstroke}{rgb}{0.700000,0.700000,0.700000}%
\pgfsetstrokecolor{currentstroke}%
\pgfsetdash{}{0pt}%
\pgfpathmoveto{\pgfqpoint{1.798089in}{0.466613in}}%
\pgfpathlineto{\pgfqpoint{1.798089in}{1.712363in}}%
\pgfusepath{stroke}%
\end{pgfscope}%
\begin{pgfscope}%
\pgfsetrectcap%
\pgfsetmiterjoin%
\pgfsetlinewidth{0.752812pt}%
\definecolor{currentstroke}{rgb}{0.700000,0.700000,0.700000}%
\pgfsetstrokecolor{currentstroke}%
\pgfsetdash{}{0pt}%
\pgfpathmoveto{\pgfqpoint{0.519339in}{0.466613in}}%
\pgfpathlineto{\pgfqpoint{1.798089in}{0.466613in}}%
\pgfusepath{stroke}%
\end{pgfscope}%
\begin{pgfscope}%
\pgfsetrectcap%
\pgfsetmiterjoin%
\pgfsetlinewidth{0.752812pt}%
\definecolor{currentstroke}{rgb}{0.700000,0.700000,0.700000}%
\pgfsetstrokecolor{currentstroke}%
\pgfsetdash{}{0pt}%
\pgfpathmoveto{\pgfqpoint{0.519339in}{1.712363in}}%
\pgfpathlineto{\pgfqpoint{1.798089in}{1.712363in}}%
\pgfusepath{stroke}%
\end{pgfscope}%
\end{pgfpicture}%
\makeatother%
\endgroup%
}}  \\
    \end{array}\)
  }
  \caption{For each of the 500 experiments, we sample 8 classes out of the 40 classes of the arXiv dataset and then take the largest connected component. We argue that the difficulty of a classification task is strongly influenced by the number of classes which is apparent by the lower random chance for more classes. In (a) we see, that there is some variance in the results, but the slope is not significant. For low perturbation budgets (b), we see a clear trend that the large graph is less robust (i.e.\ the  drop in accuracy is stronger for large graphs). Around a budget of \(\epsilon\) the slope becomes insignificant and (d) with a large budget the smaller graphs are less robust. In (b) we observe a correlation of \(\rho_b=0.36\) with a significance of \(\alpha_b=2e-16\) and in (d) \(\rho_b=-0.28\) with a significance of \(\alpha_b=2e-10\) \label{fig:graphsizevsrobustness}}
\end{figure}
\fi

\begin{figure*}[t]
  \centering
  \makebox[\linewidth][c]{
    \(\begin{array}{cccc}
      \subfloat[Clean graph]{\resizebox{0.3\linewidth}{!}{%% Creator: Matplotlib, PGF backend
%%
%% To include the figure in your LaTeX document, write
%%   \input{<filename>.pgf}
%%
%% Make sure the required packages are loaded in your preamble
%%   \usepackage{pgf}
%%
%% and, on pdftex
%%   \usepackage[utf8]{inputenc}\DeclareUnicodeCharacter{2212}{-}
%%
%% or, on luatex and xetex
%%   \usepackage{unicode-math}
%%
%% Figures using additional raster images can only be included by \input if
%% they are in the same directory as the main LaTeX file. For loading figures
%% from other directories you can use the `import` package
%%   \usepackage{import}
%%
%% and then include the figures with
%%   \import{<path to file>}{<filename>.pgf}
%%
%% Matplotlib used the following preamble
%%   \usepackage[utf8]{inputenc}
%%   \usepackage[T1]{fontenc}
%%   \usepackage{amsmath}
%%   \newcommand*{\mat}[1]{\boldsymbol{#1}}
%%
\begingroup%
\makeatletter%
\begin{pgfpicture}%
\pgfpathrectangle{\pgfpointorigin}{\pgfqpoint{2.278211in}{2.278211in}}%
\pgfusepath{use as bounding box, clip}%
\begin{pgfscope}%
\pgfsetbuttcap%
\pgfsetmiterjoin%
\pgfsetlinewidth{0.000000pt}%
\definecolor{currentstroke}{rgb}{1.000000,1.000000,1.000000}%
\pgfsetstrokecolor{currentstroke}%
\pgfsetstrokeopacity{0.000000}%
\pgfsetdash{}{0pt}%
\pgfpathmoveto{\pgfqpoint{0.000000in}{0.000000in}}%
\pgfpathlineto{\pgfqpoint{2.278211in}{0.000000in}}%
\pgfpathlineto{\pgfqpoint{2.278211in}{2.278211in}}%
\pgfpathlineto{\pgfqpoint{0.000000in}{2.278211in}}%
\pgfpathclose%
\pgfusepath{}%
\end{pgfscope}%
\begin{pgfscope}%
\pgfsetbuttcap%
\pgfsetmiterjoin%
\definecolor{currentfill}{rgb}{1.000000,1.000000,1.000000}%
\pgfsetfillcolor{currentfill}%
\pgfsetlinewidth{0.000000pt}%
\definecolor{currentstroke}{rgb}{0.000000,0.000000,0.000000}%
\pgfsetstrokecolor{currentstroke}%
\pgfsetstrokeopacity{0.000000}%
\pgfsetdash{}{0pt}%
\pgfpathmoveto{\pgfqpoint{0.526284in}{0.473557in}}%
\pgfpathlineto{\pgfqpoint{2.178211in}{0.473557in}}%
\pgfpathlineto{\pgfqpoint{2.178211in}{2.178211in}}%
\pgfpathlineto{\pgfqpoint{0.526284in}{2.178211in}}%
\pgfpathclose%
\pgfusepath{fill}%
\end{pgfscope}%
\begin{pgfscope}%
\pgfpathrectangle{\pgfqpoint{0.526284in}{0.473557in}}{\pgfqpoint{1.651927in}{1.704653in}}%
\pgfusepath{clip}%
\pgfsetbuttcap%
\pgfsetroundjoin%
\definecolor{currentfill}{rgb}{0.121569,0.466667,0.705882}%
\pgfsetfillcolor{currentfill}%
\pgfsetfillopacity{0.250000}%
\pgfsetlinewidth{1.003750pt}%
\definecolor{currentstroke}{rgb}{0.121569,0.466667,0.705882}%
\pgfsetstrokecolor{currentstroke}%
\pgfsetstrokeopacity{0.250000}%
\pgfsetdash{}{0pt}%
\pgfsys@defobject{currentmarker}{\pgfqpoint{-0.017010in}{-0.017010in}}{\pgfqpoint{0.017010in}{0.017010in}}{%
\pgfpathmoveto{\pgfqpoint{0.000000in}{-0.017010in}}%
\pgfpathcurveto{\pgfqpoint{0.004511in}{-0.017010in}}{\pgfqpoint{0.008838in}{-0.015218in}}{\pgfqpoint{0.012028in}{-0.012028in}}%
\pgfpathcurveto{\pgfqpoint{0.015218in}{-0.008838in}}{\pgfqpoint{0.017010in}{-0.004511in}}{\pgfqpoint{0.017010in}{0.000000in}}%
\pgfpathcurveto{\pgfqpoint{0.017010in}{0.004511in}}{\pgfqpoint{0.015218in}{0.008838in}}{\pgfqpoint{0.012028in}{0.012028in}}%
\pgfpathcurveto{\pgfqpoint{0.008838in}{0.015218in}}{\pgfqpoint{0.004511in}{0.017010in}}{\pgfqpoint{0.000000in}{0.017010in}}%
\pgfpathcurveto{\pgfqpoint{-0.004511in}{0.017010in}}{\pgfqpoint{-0.008838in}{0.015218in}}{\pgfqpoint{-0.012028in}{0.012028in}}%
\pgfpathcurveto{\pgfqpoint{-0.015218in}{0.008838in}}{\pgfqpoint{-0.017010in}{0.004511in}}{\pgfqpoint{-0.017010in}{0.000000in}}%
\pgfpathcurveto{\pgfqpoint{-0.017010in}{-0.004511in}}{\pgfqpoint{-0.015218in}{-0.008838in}}{\pgfqpoint{-0.012028in}{-0.012028in}}%
\pgfpathcurveto{\pgfqpoint{-0.008838in}{-0.015218in}}{\pgfqpoint{-0.004511in}{-0.017010in}}{\pgfqpoint{0.000000in}{-0.017010in}}%
\pgfpathclose%
\pgfusepath{stroke,fill}%
}%
\begin{pgfscope}%
\pgfsys@transformshift{1.139286in}{2.036453in}%
\pgfsys@useobject{currentmarker}{}%
\end{pgfscope}%
\begin{pgfscope}%
\pgfsys@transformshift{0.826629in}{2.042833in}%
\pgfsys@useobject{currentmarker}{}%
\end{pgfscope}%
\begin{pgfscope}%
\pgfsys@transformshift{0.969498in}{2.011675in}%
\pgfsys@useobject{currentmarker}{}%
\end{pgfscope}%
\begin{pgfscope}%
\pgfsys@transformshift{0.732099in}{1.967422in}%
\pgfsys@useobject{currentmarker}{}%
\end{pgfscope}%
\begin{pgfscope}%
\pgfsys@transformshift{0.815558in}{1.843174in}%
\pgfsys@useobject{currentmarker}{}%
\end{pgfscope}%
\begin{pgfscope}%
\pgfsys@transformshift{1.268362in}{2.032697in}%
\pgfsys@useobject{currentmarker}{}%
\end{pgfscope}%
\begin{pgfscope}%
\pgfsys@transformshift{0.640142in}{1.790729in}%
\pgfsys@useobject{currentmarker}{}%
\end{pgfscope}%
\begin{pgfscope}%
\pgfsys@transformshift{1.031370in}{2.014466in}%
\pgfsys@useobject{currentmarker}{}%
\end{pgfscope}%
\begin{pgfscope}%
\pgfsys@transformshift{1.501355in}{2.044049in}%
\pgfsys@useobject{currentmarker}{}%
\end{pgfscope}%
\begin{pgfscope}%
\pgfsys@transformshift{1.307778in}{1.953296in}%
\pgfsys@useobject{currentmarker}{}%
\end{pgfscope}%
\begin{pgfscope}%
\pgfsys@transformshift{1.194068in}{2.005706in}%
\pgfsys@useobject{currentmarker}{}%
\end{pgfscope}%
\begin{pgfscope}%
\pgfsys@transformshift{0.667950in}{1.968786in}%
\pgfsys@useobject{currentmarker}{}%
\end{pgfscope}%
\begin{pgfscope}%
\pgfsys@transformshift{0.827370in}{1.822301in}%
\pgfsys@useobject{currentmarker}{}%
\end{pgfscope}%
\begin{pgfscope}%
\pgfsys@transformshift{1.071193in}{1.986628in}%
\pgfsys@useobject{currentmarker}{}%
\end{pgfscope}%
\begin{pgfscope}%
\pgfsys@transformshift{0.921937in}{2.081652in}%
\pgfsys@useobject{currentmarker}{}%
\end{pgfscope}%
\begin{pgfscope}%
\pgfsys@transformshift{1.556989in}{1.968778in}%
\pgfsys@useobject{currentmarker}{}%
\end{pgfscope}%
\begin{pgfscope}%
\pgfsys@transformshift{0.645159in}{1.968561in}%
\pgfsys@useobject{currentmarker}{}%
\end{pgfscope}%
\begin{pgfscope}%
\pgfsys@transformshift{1.282858in}{2.048901in}%
\pgfsys@useobject{currentmarker}{}%
\end{pgfscope}%
\begin{pgfscope}%
\pgfsys@transformshift{0.923659in}{1.996943in}%
\pgfsys@useobject{currentmarker}{}%
\end{pgfscope}%
\begin{pgfscope}%
\pgfsys@transformshift{1.196863in}{2.001685in}%
\pgfsys@useobject{currentmarker}{}%
\end{pgfscope}%
\begin{pgfscope}%
\pgfsys@transformshift{0.734895in}{1.917976in}%
\pgfsys@useobject{currentmarker}{}%
\end{pgfscope}%
\begin{pgfscope}%
\pgfsys@transformshift{1.049754in}{1.945383in}%
\pgfsys@useobject{currentmarker}{}%
\end{pgfscope}%
\begin{pgfscope}%
\pgfsys@transformshift{0.612705in}{1.985498in}%
\pgfsys@useobject{currentmarker}{}%
\end{pgfscope}%
\begin{pgfscope}%
\pgfsys@transformshift{1.196567in}{2.078340in}%
\pgfsys@useobject{currentmarker}{}%
\end{pgfscope}%
\begin{pgfscope}%
\pgfsys@transformshift{0.616815in}{1.931825in}%
\pgfsys@useobject{currentmarker}{}%
\end{pgfscope}%
\begin{pgfscope}%
\pgfsys@transformshift{0.881170in}{1.785679in}%
\pgfsys@useobject{currentmarker}{}%
\end{pgfscope}%
\begin{pgfscope}%
\pgfsys@transformshift{0.644863in}{1.890679in}%
\pgfsys@useobject{currentmarker}{}%
\end{pgfscope}%
\begin{pgfscope}%
\pgfsys@transformshift{1.119421in}{1.988404in}%
\pgfsys@useobject{currentmarker}{}%
\end{pgfscope}%
\begin{pgfscope}%
\pgfsys@transformshift{0.796952in}{1.941590in}%
\pgfsys@useobject{currentmarker}{}%
\end{pgfscope}%
\begin{pgfscope}%
\pgfsys@transformshift{0.928991in}{1.974211in}%
\pgfsys@useobject{currentmarker}{}%
\end{pgfscope}%
\begin{pgfscope}%
\pgfsys@transformshift{1.146006in}{1.996771in}%
\pgfsys@useobject{currentmarker}{}%
\end{pgfscope}%
\begin{pgfscope}%
\pgfsys@transformshift{1.088355in}{1.972168in}%
\pgfsys@useobject{currentmarker}{}%
\end{pgfscope}%
\begin{pgfscope}%
\pgfsys@transformshift{1.172758in}{1.955094in}%
\pgfsys@useobject{currentmarker}{}%
\end{pgfscope}%
\begin{pgfscope}%
\pgfsys@transformshift{1.211137in}{2.015651in}%
\pgfsys@useobject{currentmarker}{}%
\end{pgfscope}%
\begin{pgfscope}%
\pgfsys@transformshift{0.829388in}{1.953542in}%
\pgfsys@useobject{currentmarker}{}%
\end{pgfscope}%
\begin{pgfscope}%
\pgfsys@transformshift{1.584944in}{1.872382in}%
\pgfsys@useobject{currentmarker}{}%
\end{pgfscope}%
\begin{pgfscope}%
\pgfsys@transformshift{0.693739in}{2.011712in}%
\pgfsys@useobject{currentmarker}{}%
\end{pgfscope}%
\begin{pgfscope}%
\pgfsys@transformshift{1.271769in}{2.072842in}%
\pgfsys@useobject{currentmarker}{}%
\end{pgfscope}%
\begin{pgfscope}%
\pgfsys@transformshift{1.064010in}{1.970209in}%
\pgfsys@useobject{currentmarker}{}%
\end{pgfscope}%
\begin{pgfscope}%
\pgfsys@transformshift{1.461162in}{1.856487in}%
\pgfsys@useobject{currentmarker}{}%
\end{pgfscope}%
\begin{pgfscope}%
\pgfsys@transformshift{1.254162in}{2.000830in}%
\pgfsys@useobject{currentmarker}{}%
\end{pgfscope}%
\begin{pgfscope}%
\pgfsys@transformshift{0.641290in}{2.009119in}%
\pgfsys@useobject{currentmarker}{}%
\end{pgfscope}%
\begin{pgfscope}%
\pgfsys@transformshift{1.047681in}{2.028877in}%
\pgfsys@useobject{currentmarker}{}%
\end{pgfscope}%
\begin{pgfscope}%
\pgfsys@transformshift{0.651435in}{1.983595in}%
\pgfsys@useobject{currentmarker}{}%
\end{pgfscope}%
\begin{pgfscope}%
\pgfsys@transformshift{1.047699in}{1.985842in}%
\pgfsys@useobject{currentmarker}{}%
\end{pgfscope}%
\begin{pgfscope}%
\pgfsys@transformshift{0.706939in}{1.856552in}%
\pgfsys@useobject{currentmarker}{}%
\end{pgfscope}%
\begin{pgfscope}%
\pgfsys@transformshift{0.621869in}{2.039657in}%
\pgfsys@useobject{currentmarker}{}%
\end{pgfscope}%
\begin{pgfscope}%
\pgfsys@transformshift{1.117533in}{1.957334in}%
\pgfsys@useobject{currentmarker}{}%
\end{pgfscope}%
\begin{pgfscope}%
\pgfsys@transformshift{1.307648in}{1.825486in}%
\pgfsys@useobject{currentmarker}{}%
\end{pgfscope}%
\begin{pgfscope}%
\pgfsys@transformshift{0.948856in}{2.034542in}%
\pgfsys@useobject{currentmarker}{}%
\end{pgfscope}%
\begin{pgfscope}%
\pgfsys@transformshift{1.014153in}{1.982693in}%
\pgfsys@useobject{currentmarker}{}%
\end{pgfscope}%
\begin{pgfscope}%
\pgfsys@transformshift{0.631737in}{2.022024in}%
\pgfsys@useobject{currentmarker}{}%
\end{pgfscope}%
\begin{pgfscope}%
\pgfsys@transformshift{0.643493in}{2.043724in}%
\pgfsys@useobject{currentmarker}{}%
\end{pgfscope}%
\begin{pgfscope}%
\pgfsys@transformshift{0.772755in}{1.959350in}%
\pgfsys@useobject{currentmarker}{}%
\end{pgfscope}%
\begin{pgfscope}%
\pgfsys@transformshift{0.692091in}{1.866055in}%
\pgfsys@useobject{currentmarker}{}%
\end{pgfscope}%
\begin{pgfscope}%
\pgfsys@transformshift{0.595358in}{2.037027in}%
\pgfsys@useobject{currentmarker}{}%
\end{pgfscope}%
\begin{pgfscope}%
\pgfsys@transformshift{0.903997in}{1.913935in}%
\pgfsys@useobject{currentmarker}{}%
\end{pgfscope}%
\begin{pgfscope}%
\pgfsys@transformshift{1.258735in}{2.064720in}%
\pgfsys@useobject{currentmarker}{}%
\end{pgfscope}%
\begin{pgfscope}%
\pgfsys@transformshift{0.771996in}{1.918802in}%
\pgfsys@useobject{currentmarker}{}%
\end{pgfscope}%
\begin{pgfscope}%
\pgfsys@transformshift{0.665932in}{1.897643in}%
\pgfsys@useobject{currentmarker}{}%
\end{pgfscope}%
\begin{pgfscope}%
\pgfsys@transformshift{0.723120in}{1.893339in}%
\pgfsys@useobject{currentmarker}{}%
\end{pgfscope}%
\begin{pgfscope}%
\pgfsys@transformshift{1.030500in}{2.044978in}%
\pgfsys@useobject{currentmarker}{}%
\end{pgfscope}%
\begin{pgfscope}%
\pgfsys@transformshift{0.538429in}{1.933971in}%
\pgfsys@useobject{currentmarker}{}%
\end{pgfscope}%
\begin{pgfscope}%
\pgfsys@transformshift{1.081468in}{1.972456in}%
\pgfsys@useobject{currentmarker}{}%
\end{pgfscope}%
\begin{pgfscope}%
\pgfsys@transformshift{0.629293in}{1.974840in}%
\pgfsys@useobject{currentmarker}{}%
\end{pgfscope}%
\begin{pgfscope}%
\pgfsys@transformshift{0.792175in}{1.901483in}%
\pgfsys@useobject{currentmarker}{}%
\end{pgfscope}%
\begin{pgfscope}%
\pgfsys@transformshift{1.017763in}{1.986499in}%
\pgfsys@useobject{currentmarker}{}%
\end{pgfscope}%
\begin{pgfscope}%
\pgfsys@transformshift{0.953410in}{1.959547in}%
\pgfsys@useobject{currentmarker}{}%
\end{pgfscope}%
\begin{pgfscope}%
\pgfsys@transformshift{1.026687in}{1.987147in}%
\pgfsys@useobject{currentmarker}{}%
\end{pgfscope}%
\begin{pgfscope}%
\pgfsys@transformshift{1.527422in}{1.808673in}%
\pgfsys@useobject{currentmarker}{}%
\end{pgfscope}%
\begin{pgfscope}%
\pgfsys@transformshift{1.160817in}{1.997767in}%
\pgfsys@useobject{currentmarker}{}%
\end{pgfscope}%
\begin{pgfscope}%
\pgfsys@transformshift{0.698349in}{1.999049in}%
\pgfsys@useobject{currentmarker}{}%
\end{pgfscope}%
\begin{pgfscope}%
\pgfsys@transformshift{1.035203in}{1.974214in}%
\pgfsys@useobject{currentmarker}{}%
\end{pgfscope}%
\begin{pgfscope}%
\pgfsys@transformshift{0.922307in}{2.055659in}%
\pgfsys@useobject{currentmarker}{}%
\end{pgfscope}%
\begin{pgfscope}%
\pgfsys@transformshift{1.166982in}{2.017533in}%
\pgfsys@useobject{currentmarker}{}%
\end{pgfscope}%
\begin{pgfscope}%
\pgfsys@transformshift{1.119569in}{1.958589in}%
\pgfsys@useobject{currentmarker}{}%
\end{pgfscope}%
\begin{pgfscope}%
\pgfsys@transformshift{0.825518in}{1.893288in}%
\pgfsys@useobject{currentmarker}{}%
\end{pgfscope}%
\begin{pgfscope}%
\pgfsys@transformshift{1.426209in}{1.817966in}%
\pgfsys@useobject{currentmarker}{}%
\end{pgfscope}%
\begin{pgfscope}%
\pgfsys@transformshift{1.406677in}{2.025080in}%
\pgfsys@useobject{currentmarker}{}%
\end{pgfscope}%
\begin{pgfscope}%
\pgfsys@transformshift{1.415453in}{2.023957in}%
\pgfsys@useobject{currentmarker}{}%
\end{pgfscope}%
\begin{pgfscope}%
\pgfsys@transformshift{1.076914in}{2.017140in}%
\pgfsys@useobject{currentmarker}{}%
\end{pgfscope}%
\begin{pgfscope}%
\pgfsys@transformshift{0.779031in}{1.869181in}%
\pgfsys@useobject{currentmarker}{}%
\end{pgfscope}%
\begin{pgfscope}%
\pgfsys@transformshift{0.629497in}{1.971164in}%
\pgfsys@useobject{currentmarker}{}%
\end{pgfscope}%
\begin{pgfscope}%
\pgfsys@transformshift{1.519184in}{1.960315in}%
\pgfsys@useobject{currentmarker}{}%
\end{pgfscope}%
\begin{pgfscope}%
\pgfsys@transformshift{0.769237in}{1.936422in}%
\pgfsys@useobject{currentmarker}{}%
\end{pgfscope}%
\begin{pgfscope}%
\pgfsys@transformshift{1.087429in}{2.042777in}%
\pgfsys@useobject{currentmarker}{}%
\end{pgfscope}%
\begin{pgfscope}%
\pgfsys@transformshift{0.750705in}{2.062567in}%
\pgfsys@useobject{currentmarker}{}%
\end{pgfscope}%
\begin{pgfscope}%
\pgfsys@transformshift{1.495746in}{1.893740in}%
\pgfsys@useobject{currentmarker}{}%
\end{pgfscope}%
\begin{pgfscope}%
\pgfsys@transformshift{1.182959in}{1.950023in}%
\pgfsys@useobject{currentmarker}{}%
\end{pgfscope}%
\begin{pgfscope}%
\pgfsys@transformshift{0.713789in}{2.012763in}%
\pgfsys@useobject{currentmarker}{}%
\end{pgfscope}%
\begin{pgfscope}%
\pgfsys@transformshift{1.355894in}{1.972687in}%
\pgfsys@useobject{currentmarker}{}%
\end{pgfscope}%
\begin{pgfscope}%
\pgfsys@transformshift{1.635208in}{2.011589in}%
\pgfsys@useobject{currentmarker}{}%
\end{pgfscope}%
\begin{pgfscope}%
\pgfsys@transformshift{1.253848in}{1.963214in}%
\pgfsys@useobject{currentmarker}{}%
\end{pgfscope}%
\begin{pgfscope}%
\pgfsys@transformshift{0.935989in}{2.052703in}%
\pgfsys@useobject{currentmarker}{}%
\end{pgfscope}%
\begin{pgfscope}%
\pgfsys@transformshift{0.875320in}{2.028628in}%
\pgfsys@useobject{currentmarker}{}%
\end{pgfscope}%
\begin{pgfscope}%
\pgfsys@transformshift{1.109831in}{2.072629in}%
\pgfsys@useobject{currentmarker}{}%
\end{pgfscope}%
\begin{pgfscope}%
\pgfsys@transformshift{1.118292in}{1.954275in}%
\pgfsys@useobject{currentmarker}{}%
\end{pgfscope}%
\begin{pgfscope}%
\pgfsys@transformshift{1.130029in}{2.013655in}%
\pgfsys@useobject{currentmarker}{}%
\end{pgfscope}%
\begin{pgfscope}%
\pgfsys@transformshift{1.140971in}{2.052603in}%
\pgfsys@useobject{currentmarker}{}%
\end{pgfscope}%
\begin{pgfscope}%
\pgfsys@transformshift{1.052346in}{2.084532in}%
\pgfsys@useobject{currentmarker}{}%
\end{pgfscope}%
\begin{pgfscope}%
\pgfsys@transformshift{0.621629in}{1.943166in}%
\pgfsys@useobject{currentmarker}{}%
\end{pgfscope}%
\begin{pgfscope}%
\pgfsys@transformshift{0.917753in}{2.044353in}%
\pgfsys@useobject{currentmarker}{}%
\end{pgfscope}%
\begin{pgfscope}%
\pgfsys@transformshift{0.863027in}{1.972479in}%
\pgfsys@useobject{currentmarker}{}%
\end{pgfscope}%
\begin{pgfscope}%
\pgfsys@transformshift{0.672485in}{1.859331in}%
\pgfsys@useobject{currentmarker}{}%
\end{pgfscope}%
\begin{pgfscope}%
\pgfsys@transformshift{1.747437in}{1.824949in}%
\pgfsys@useobject{currentmarker}{}%
\end{pgfscope}%
\begin{pgfscope}%
\pgfsys@transformshift{0.526284in}{2.009926in}%
\pgfsys@useobject{currentmarker}{}%
\end{pgfscope}%
\begin{pgfscope}%
\pgfsys@transformshift{0.627405in}{1.939655in}%
\pgfsys@useobject{currentmarker}{}%
\end{pgfscope}%
\begin{pgfscope}%
\pgfsys@transformshift{1.270658in}{1.802738in}%
\pgfsys@useobject{currentmarker}{}%
\end{pgfscope}%
\begin{pgfscope}%
\pgfsys@transformshift{1.594053in}{1.837511in}%
\pgfsys@useobject{currentmarker}{}%
\end{pgfscope}%
\begin{pgfscope}%
\pgfsys@transformshift{0.652472in}{1.872934in}%
\pgfsys@useobject{currentmarker}{}%
\end{pgfscope}%
\begin{pgfscope}%
\pgfsys@transformshift{0.819168in}{1.947889in}%
\pgfsys@useobject{currentmarker}{}%
\end{pgfscope}%
\begin{pgfscope}%
\pgfsys@transformshift{0.847883in}{1.845631in}%
\pgfsys@useobject{currentmarker}{}%
\end{pgfscope}%
\begin{pgfscope}%
\pgfsys@transformshift{1.118495in}{1.902737in}%
\pgfsys@useobject{currentmarker}{}%
\end{pgfscope}%
\begin{pgfscope}%
\pgfsys@transformshift{0.618167in}{1.962234in}%
\pgfsys@useobject{currentmarker}{}%
\end{pgfscope}%
\begin{pgfscope}%
\pgfsys@transformshift{0.857547in}{2.030841in}%
\pgfsys@useobject{currentmarker}{}%
\end{pgfscope}%
\begin{pgfscope}%
\pgfsys@transformshift{1.090392in}{2.073540in}%
\pgfsys@useobject{currentmarker}{}%
\end{pgfscope}%
\begin{pgfscope}%
\pgfsys@transformshift{1.063566in}{2.022606in}%
\pgfsys@useobject{currentmarker}{}%
\end{pgfscope}%
\begin{pgfscope}%
\pgfsys@transformshift{1.510205in}{1.862037in}%
\pgfsys@useobject{currentmarker}{}%
\end{pgfscope}%
\begin{pgfscope}%
\pgfsys@transformshift{1.287209in}{2.035366in}%
\pgfsys@useobject{currentmarker}{}%
\end{pgfscope}%
\begin{pgfscope}%
\pgfsys@transformshift{0.778753in}{1.833180in}%
\pgfsys@useobject{currentmarker}{}%
\end{pgfscope}%
\begin{pgfscope}%
\pgfsys@transformshift{0.627757in}{2.032888in}%
\pgfsys@useobject{currentmarker}{}%
\end{pgfscope}%
\begin{pgfscope}%
\pgfsys@transformshift{1.071860in}{1.982310in}%
\pgfsys@useobject{currentmarker}{}%
\end{pgfscope}%
\begin{pgfscope}%
\pgfsys@transformshift{1.053938in}{2.019243in}%
\pgfsys@useobject{currentmarker}{}%
\end{pgfscope}%
\begin{pgfscope}%
\pgfsys@transformshift{0.910458in}{2.045783in}%
\pgfsys@useobject{currentmarker}{}%
\end{pgfscope}%
\begin{pgfscope}%
\pgfsys@transformshift{0.621962in}{2.010052in}%
\pgfsys@useobject{currentmarker}{}%
\end{pgfscope}%
\begin{pgfscope}%
\pgfsys@transformshift{0.836867in}{2.006336in}%
\pgfsys@useobject{currentmarker}{}%
\end{pgfscope}%
\begin{pgfscope}%
\pgfsys@transformshift{0.800229in}{2.031747in}%
\pgfsys@useobject{currentmarker}{}%
\end{pgfscope}%
\begin{pgfscope}%
\pgfsys@transformshift{0.958260in}{2.012553in}%
\pgfsys@useobject{currentmarker}{}%
\end{pgfscope}%
\begin{pgfscope}%
\pgfsys@transformshift{0.656045in}{2.020219in}%
\pgfsys@useobject{currentmarker}{}%
\end{pgfscope}%
\begin{pgfscope}%
\pgfsys@transformshift{0.629960in}{1.906290in}%
\pgfsys@useobject{currentmarker}{}%
\end{pgfscope}%
\begin{pgfscope}%
\pgfsys@transformshift{1.008358in}{1.879843in}%
\pgfsys@useobject{currentmarker}{}%
\end{pgfscope}%
\begin{pgfscope}%
\pgfsys@transformshift{1.258698in}{2.010058in}%
\pgfsys@useobject{currentmarker}{}%
\end{pgfscope}%
\begin{pgfscope}%
\pgfsys@transformshift{1.100889in}{2.001603in}%
\pgfsys@useobject{currentmarker}{}%
\end{pgfscope}%
\begin{pgfscope}%
\pgfsys@transformshift{1.230410in}{2.019979in}%
\pgfsys@useobject{currentmarker}{}%
\end{pgfscope}%
\begin{pgfscope}%
\pgfsys@transformshift{0.647325in}{2.003202in}%
\pgfsys@useobject{currentmarker}{}%
\end{pgfscope}%
\begin{pgfscope}%
\pgfsys@transformshift{0.904423in}{2.009498in}%
\pgfsys@useobject{currentmarker}{}%
\end{pgfscope}%
\begin{pgfscope}%
\pgfsys@transformshift{1.617602in}{1.984318in}%
\pgfsys@useobject{currentmarker}{}%
\end{pgfscope}%
\begin{pgfscope}%
\pgfsys@transformshift{1.245406in}{2.057681in}%
\pgfsys@useobject{currentmarker}{}%
\end{pgfscope}%
\begin{pgfscope}%
\pgfsys@transformshift{1.238907in}{2.019247in}%
\pgfsys@useobject{currentmarker}{}%
\end{pgfscope}%
\begin{pgfscope}%
\pgfsys@transformshift{1.187366in}{2.024473in}%
\pgfsys@useobject{currentmarker}{}%
\end{pgfscope}%
\begin{pgfscope}%
\pgfsys@transformshift{1.095724in}{2.029796in}%
\pgfsys@useobject{currentmarker}{}%
\end{pgfscope}%
\begin{pgfscope}%
\pgfsys@transformshift{1.131084in}{2.053636in}%
\pgfsys@useobject{currentmarker}{}%
\end{pgfscope}%
\begin{pgfscope}%
\pgfsys@transformshift{0.643549in}{1.960232in}%
\pgfsys@useobject{currentmarker}{}%
\end{pgfscope}%
\begin{pgfscope}%
\pgfsys@transformshift{0.726878in}{1.988582in}%
\pgfsys@useobject{currentmarker}{}%
\end{pgfscope}%
\begin{pgfscope}%
\pgfsys@transformshift{0.940728in}{1.977401in}%
\pgfsys@useobject{currentmarker}{}%
\end{pgfscope}%
\begin{pgfscope}%
\pgfsys@transformshift{0.686537in}{1.977462in}%
\pgfsys@useobject{currentmarker}{}%
\end{pgfscope}%
\begin{pgfscope}%
\pgfsys@transformshift{1.304686in}{2.021196in}%
\pgfsys@useobject{currentmarker}{}%
\end{pgfscope}%
\begin{pgfscope}%
\pgfsys@transformshift{0.822223in}{1.884899in}%
\pgfsys@useobject{currentmarker}{}%
\end{pgfscope}%
\begin{pgfscope}%
\pgfsys@transformshift{0.753593in}{1.919780in}%
\pgfsys@useobject{currentmarker}{}%
\end{pgfscope}%
\begin{pgfscope}%
\pgfsys@transformshift{1.155670in}{1.994964in}%
\pgfsys@useobject{currentmarker}{}%
\end{pgfscope}%
\begin{pgfscope}%
\pgfsys@transformshift{0.665617in}{1.937997in}%
\pgfsys@useobject{currentmarker}{}%
\end{pgfscope}%
\begin{pgfscope}%
\pgfsys@transformshift{0.722713in}{1.868843in}%
\pgfsys@useobject{currentmarker}{}%
\end{pgfscope}%
\begin{pgfscope}%
\pgfsys@transformshift{0.537651in}{1.840263in}%
\pgfsys@useobject{currentmarker}{}%
\end{pgfscope}%
\begin{pgfscope}%
\pgfsys@transformshift{0.956631in}{2.011633in}%
\pgfsys@useobject{currentmarker}{}%
\end{pgfscope}%
\begin{pgfscope}%
\pgfsys@transformshift{1.046922in}{2.054824in}%
\pgfsys@useobject{currentmarker}{}%
\end{pgfscope}%
\begin{pgfscope}%
\pgfsys@transformshift{0.812559in}{2.027411in}%
\pgfsys@useobject{currentmarker}{}%
\end{pgfscope}%
\begin{pgfscope}%
\pgfsys@transformshift{1.535254in}{1.965416in}%
\pgfsys@useobject{currentmarker}{}%
\end{pgfscope}%
\begin{pgfscope}%
\pgfsys@transformshift{0.778087in}{1.981701in}%
\pgfsys@useobject{currentmarker}{}%
\end{pgfscope}%
\begin{pgfscope}%
\pgfsys@transformshift{0.723860in}{2.036060in}%
\pgfsys@useobject{currentmarker}{}%
\end{pgfscope}%
\begin{pgfscope}%
\pgfsys@transformshift{0.970961in}{1.854203in}%
\pgfsys@useobject{currentmarker}{}%
\end{pgfscope}%
\begin{pgfscope}%
\pgfsys@transformshift{1.099149in}{2.080888in}%
\pgfsys@useobject{currentmarker}{}%
\end{pgfscope}%
\begin{pgfscope}%
\pgfsys@transformshift{1.042182in}{1.976741in}%
\pgfsys@useobject{currentmarker}{}%
\end{pgfscope}%
\begin{pgfscope}%
\pgfsys@transformshift{1.047181in}{2.035677in}%
\pgfsys@useobject{currentmarker}{}%
\end{pgfscope}%
\begin{pgfscope}%
\pgfsys@transformshift{1.569152in}{2.032685in}%
\pgfsys@useobject{currentmarker}{}%
\end{pgfscope}%
\begin{pgfscope}%
\pgfsys@transformshift{1.518018in}{1.918853in}%
\pgfsys@useobject{currentmarker}{}%
\end{pgfscope}%
\begin{pgfscope}%
\pgfsys@transformshift{0.733265in}{1.957331in}%
\pgfsys@useobject{currentmarker}{}%
\end{pgfscope}%
\begin{pgfscope}%
\pgfsys@transformshift{0.558701in}{1.985806in}%
\pgfsys@useobject{currentmarker}{}%
\end{pgfscope}%
\begin{pgfscope}%
\pgfsys@transformshift{0.600190in}{2.040528in}%
\pgfsys@useobject{currentmarker}{}%
\end{pgfscope}%
\begin{pgfscope}%
\pgfsys@transformshift{1.222060in}{2.019108in}%
\pgfsys@useobject{currentmarker}{}%
\end{pgfscope}%
\begin{pgfscope}%
\pgfsys@transformshift{0.884484in}{1.989697in}%
\pgfsys@useobject{currentmarker}{}%
\end{pgfscope}%
\begin{pgfscope}%
\pgfsys@transformshift{1.004359in}{1.979692in}%
\pgfsys@useobject{currentmarker}{}%
\end{pgfscope}%
\begin{pgfscope}%
\pgfsys@transformshift{0.853585in}{1.921380in}%
\pgfsys@useobject{currentmarker}{}%
\end{pgfscope}%
\begin{pgfscope}%
\pgfsys@transformshift{0.668357in}{2.013353in}%
\pgfsys@useobject{currentmarker}{}%
\end{pgfscope}%
\begin{pgfscope}%
\pgfsys@transformshift{0.536244in}{1.930329in}%
\pgfsys@useobject{currentmarker}{}%
\end{pgfscope}%
\begin{pgfscope}%
\pgfsys@transformshift{0.785103in}{1.979576in}%
\pgfsys@useobject{currentmarker}{}%
\end{pgfscope}%
\begin{pgfscope}%
\pgfsys@transformshift{1.369965in}{1.967270in}%
\pgfsys@useobject{currentmarker}{}%
\end{pgfscope}%
\begin{pgfscope}%
\pgfsys@transformshift{0.794379in}{1.891496in}%
\pgfsys@useobject{currentmarker}{}%
\end{pgfscope}%
\begin{pgfscope}%
\pgfsys@transformshift{0.633237in}{1.965715in}%
\pgfsys@useobject{currentmarker}{}%
\end{pgfscope}%
\begin{pgfscope}%
\pgfsys@transformshift{0.859287in}{1.923059in}%
\pgfsys@useobject{currentmarker}{}%
\end{pgfscope}%
\begin{pgfscope}%
\pgfsys@transformshift{1.032685in}{1.986230in}%
\pgfsys@useobject{currentmarker}{}%
\end{pgfscope}%
\begin{pgfscope}%
\pgfsys@transformshift{0.784141in}{1.945833in}%
\pgfsys@useobject{currentmarker}{}%
\end{pgfscope}%
\begin{pgfscope}%
\pgfsys@transformshift{0.996010in}{2.011496in}%
\pgfsys@useobject{currentmarker}{}%
\end{pgfscope}%
\begin{pgfscope}%
\pgfsys@transformshift{1.002563in}{1.935202in}%
\pgfsys@useobject{currentmarker}{}%
\end{pgfscope}%
\begin{pgfscope}%
\pgfsys@transformshift{0.930287in}{2.054083in}%
\pgfsys@useobject{currentmarker}{}%
\end{pgfscope}%
\begin{pgfscope}%
\pgfsys@transformshift{0.659693in}{2.001123in}%
\pgfsys@useobject{currentmarker}{}%
\end{pgfscope}%
\begin{pgfscope}%
\pgfsys@transformshift{1.068472in}{2.018643in}%
\pgfsys@useobject{currentmarker}{}%
\end{pgfscope}%
\begin{pgfscope}%
\pgfsys@transformshift{0.856640in}{2.048927in}%
\pgfsys@useobject{currentmarker}{}%
\end{pgfscope}%
\begin{pgfscope}%
\pgfsys@transformshift{1.405029in}{1.969646in}%
\pgfsys@useobject{currentmarker}{}%
\end{pgfscope}%
\begin{pgfscope}%
\pgfsys@transformshift{1.220135in}{1.991811in}%
\pgfsys@useobject{currentmarker}{}%
\end{pgfscope}%
\begin{pgfscope}%
\pgfsys@transformshift{0.999324in}{2.032218in}%
\pgfsys@useobject{currentmarker}{}%
\end{pgfscope}%
\begin{pgfscope}%
\pgfsys@transformshift{0.633699in}{1.964139in}%
\pgfsys@useobject{currentmarker}{}%
\end{pgfscope}%
\begin{pgfscope}%
\pgfsys@transformshift{1.135768in}{1.974898in}%
\pgfsys@useobject{currentmarker}{}%
\end{pgfscope}%
\begin{pgfscope}%
\pgfsys@transformshift{1.116959in}{1.890463in}%
\pgfsys@useobject{currentmarker}{}%
\end{pgfscope}%
\begin{pgfscope}%
\pgfsys@transformshift{0.598172in}{2.003551in}%
\pgfsys@useobject{currentmarker}{}%
\end{pgfscope}%
\begin{pgfscope}%
\pgfsys@transformshift{0.619907in}{1.977058in}%
\pgfsys@useobject{currentmarker}{}%
\end{pgfscope}%
\begin{pgfscope}%
\pgfsys@transformshift{1.023539in}{1.999271in}%
\pgfsys@useobject{currentmarker}{}%
\end{pgfscope}%
\begin{pgfscope}%
\pgfsys@transformshift{1.456182in}{1.977732in}%
\pgfsys@useobject{currentmarker}{}%
\end{pgfscope}%
\begin{pgfscope}%
\pgfsys@transformshift{0.579547in}{1.966609in}%
\pgfsys@useobject{currentmarker}{}%
\end{pgfscope}%
\begin{pgfscope}%
\pgfsys@transformshift{1.012949in}{2.014307in}%
\pgfsys@useobject{currentmarker}{}%
\end{pgfscope}%
\begin{pgfscope}%
\pgfsys@transformshift{1.111293in}{2.005554in}%
\pgfsys@useobject{currentmarker}{}%
\end{pgfscope}%
\begin{pgfscope}%
\pgfsys@transformshift{0.977218in}{2.034045in}%
\pgfsys@useobject{currentmarker}{}%
\end{pgfscope}%
\begin{pgfscope}%
\pgfsys@transformshift{1.304371in}{2.014467in}%
\pgfsys@useobject{currentmarker}{}%
\end{pgfscope}%
\begin{pgfscope}%
\pgfsys@transformshift{1.456756in}{2.026150in}%
\pgfsys@useobject{currentmarker}{}%
\end{pgfscope}%
\begin{pgfscope}%
\pgfsys@transformshift{0.684908in}{1.865296in}%
\pgfsys@useobject{currentmarker}{}%
\end{pgfscope}%
\begin{pgfscope}%
\pgfsys@transformshift{0.997176in}{1.828914in}%
\pgfsys@useobject{currentmarker}{}%
\end{pgfscope}%
\begin{pgfscope}%
\pgfsys@transformshift{1.265659in}{2.052202in}%
\pgfsys@useobject{currentmarker}{}%
\end{pgfscope}%
\begin{pgfscope}%
\pgfsys@transformshift{0.624998in}{1.961976in}%
\pgfsys@useobject{currentmarker}{}%
\end{pgfscope}%
\begin{pgfscope}%
\pgfsys@transformshift{0.771255in}{2.052271in}%
\pgfsys@useobject{currentmarker}{}%
\end{pgfscope}%
\begin{pgfscope}%
\pgfsys@transformshift{0.828869in}{1.957371in}%
\pgfsys@useobject{currentmarker}{}%
\end{pgfscope}%
\begin{pgfscope}%
\pgfsys@transformshift{0.739412in}{1.921537in}%
\pgfsys@useobject{currentmarker}{}%
\end{pgfscope}%
\begin{pgfscope}%
\pgfsys@transformshift{0.873931in}{1.806483in}%
\pgfsys@useobject{currentmarker}{}%
\end{pgfscope}%
\begin{pgfscope}%
\pgfsys@transformshift{0.630219in}{1.973521in}%
\pgfsys@useobject{currentmarker}{}%
\end{pgfscope}%
\begin{pgfscope}%
\pgfsys@transformshift{1.172814in}{2.084833in}%
\pgfsys@useobject{currentmarker}{}%
\end{pgfscope}%
\begin{pgfscope}%
\pgfsys@transformshift{0.647992in}{1.940997in}%
\pgfsys@useobject{currentmarker}{}%
\end{pgfscope}%
\begin{pgfscope}%
\pgfsys@transformshift{1.062251in}{1.924775in}%
\pgfsys@useobject{currentmarker}{}%
\end{pgfscope}%
\begin{pgfscope}%
\pgfsys@transformshift{1.096816in}{1.947464in}%
\pgfsys@useobject{currentmarker}{}%
\end{pgfscope}%
\begin{pgfscope}%
\pgfsys@transformshift{1.149450in}{1.938505in}%
\pgfsys@useobject{currentmarker}{}%
\end{pgfscope}%
\begin{pgfscope}%
\pgfsys@transformshift{0.677632in}{1.920641in}%
\pgfsys@useobject{currentmarker}{}%
\end{pgfscope}%
\begin{pgfscope}%
\pgfsys@transformshift{1.294966in}{2.021261in}%
\pgfsys@useobject{currentmarker}{}%
\end{pgfscope}%
\begin{pgfscope}%
\pgfsys@transformshift{1.468198in}{1.877781in}%
\pgfsys@useobject{currentmarker}{}%
\end{pgfscope}%
\begin{pgfscope}%
\pgfsys@transformshift{0.730525in}{1.983404in}%
\pgfsys@useobject{currentmarker}{}%
\end{pgfscope}%
\begin{pgfscope}%
\pgfsys@transformshift{0.597968in}{1.986427in}%
\pgfsys@useobject{currentmarker}{}%
\end{pgfscope}%
\begin{pgfscope}%
\pgfsys@transformshift{0.539928in}{1.930375in}%
\pgfsys@useobject{currentmarker}{}%
\end{pgfscope}%
\begin{pgfscope}%
\pgfsys@transformshift{0.782197in}{2.080832in}%
\pgfsys@useobject{currentmarker}{}%
\end{pgfscope}%
\begin{pgfscope}%
\pgfsys@transformshift{0.726193in}{1.978098in}%
\pgfsys@useobject{currentmarker}{}%
\end{pgfscope}%
\begin{pgfscope}%
\pgfsys@transformshift{0.756148in}{2.027244in}%
\pgfsys@useobject{currentmarker}{}%
\end{pgfscope}%
\begin{pgfscope}%
\pgfsys@transformshift{0.663543in}{2.003713in}%
\pgfsys@useobject{currentmarker}{}%
\end{pgfscope}%
\begin{pgfscope}%
\pgfsys@transformshift{1.069286in}{1.934814in}%
\pgfsys@useobject{currentmarker}{}%
\end{pgfscope}%
\begin{pgfscope}%
\pgfsys@transformshift{1.411009in}{1.844797in}%
\pgfsys@useobject{currentmarker}{}%
\end{pgfscope}%
\begin{pgfscope}%
\pgfsys@transformshift{1.079839in}{1.954203in}%
\pgfsys@useobject{currentmarker}{}%
\end{pgfscope}%
\begin{pgfscope}%
\pgfsys@transformshift{1.104758in}{1.836109in}%
\pgfsys@useobject{currentmarker}{}%
\end{pgfscope}%
\begin{pgfscope}%
\pgfsys@transformshift{1.128918in}{1.977378in}%
\pgfsys@useobject{currentmarker}{}%
\end{pgfscope}%
\begin{pgfscope}%
\pgfsys@transformshift{1.010728in}{2.040149in}%
\pgfsys@useobject{currentmarker}{}%
\end{pgfscope}%
\begin{pgfscope}%
\pgfsys@transformshift{0.687426in}{1.962023in}%
\pgfsys@useobject{currentmarker}{}%
\end{pgfscope}%
\begin{pgfscope}%
\pgfsys@transformshift{1.166556in}{1.981168in}%
\pgfsys@useobject{currentmarker}{}%
\end{pgfscope}%
\begin{pgfscope}%
\pgfsys@transformshift{0.539336in}{2.074437in}%
\pgfsys@useobject{currentmarker}{}%
\end{pgfscope}%
\begin{pgfscope}%
\pgfsys@transformshift{1.057604in}{2.012281in}%
\pgfsys@useobject{currentmarker}{}%
\end{pgfscope}%
\begin{pgfscope}%
\pgfsys@transformshift{0.759925in}{1.862000in}%
\pgfsys@useobject{currentmarker}{}%
\end{pgfscope}%
\begin{pgfscope}%
\pgfsys@transformshift{1.233927in}{1.884191in}%
\pgfsys@useobject{currentmarker}{}%
\end{pgfscope}%
\begin{pgfscope}%
\pgfsys@transformshift{0.557127in}{1.859765in}%
\pgfsys@useobject{currentmarker}{}%
\end{pgfscope}%
\begin{pgfscope}%
\pgfsys@transformshift{0.685815in}{1.969596in}%
\pgfsys@useobject{currentmarker}{}%
\end{pgfscope}%
\begin{pgfscope}%
\pgfsys@transformshift{0.794138in}{1.760658in}%
\pgfsys@useobject{currentmarker}{}%
\end{pgfscope}%
\begin{pgfscope}%
\pgfsys@transformshift{0.700941in}{1.895638in}%
\pgfsys@useobject{currentmarker}{}%
\end{pgfscope}%
\begin{pgfscope}%
\pgfsys@transformshift{0.901516in}{1.891628in}%
\pgfsys@useobject{currentmarker}{}%
\end{pgfscope}%
\begin{pgfscope}%
\pgfsys@transformshift{0.999546in}{1.931375in}%
\pgfsys@useobject{currentmarker}{}%
\end{pgfscope}%
\begin{pgfscope}%
\pgfsys@transformshift{1.495820in}{2.025829in}%
\pgfsys@useobject{currentmarker}{}%
\end{pgfscope}%
\begin{pgfscope}%
\pgfsys@transformshift{1.570837in}{1.789707in}%
\pgfsys@useobject{currentmarker}{}%
\end{pgfscope}%
\begin{pgfscope}%
\pgfsys@transformshift{0.930379in}{1.988507in}%
\pgfsys@useobject{currentmarker}{}%
\end{pgfscope}%
\begin{pgfscope}%
\pgfsys@transformshift{0.725045in}{1.960391in}%
\pgfsys@useobject{currentmarker}{}%
\end{pgfscope}%
\begin{pgfscope}%
\pgfsys@transformshift{0.663488in}{1.909486in}%
\pgfsys@useobject{currentmarker}{}%
\end{pgfscope}%
\begin{pgfscope}%
\pgfsys@transformshift{0.729081in}{2.026886in}%
\pgfsys@useobject{currentmarker}{}%
\end{pgfscope}%
\begin{pgfscope}%
\pgfsys@transformshift{0.600893in}{2.016721in}%
\pgfsys@useobject{currentmarker}{}%
\end{pgfscope}%
\begin{pgfscope}%
\pgfsys@transformshift{0.769145in}{1.973047in}%
\pgfsys@useobject{currentmarker}{}%
\end{pgfscope}%
\begin{pgfscope}%
\pgfsys@transformshift{0.677817in}{1.963075in}%
\pgfsys@useobject{currentmarker}{}%
\end{pgfscope}%
\begin{pgfscope}%
\pgfsys@transformshift{0.942968in}{1.999518in}%
\pgfsys@useobject{currentmarker}{}%
\end{pgfscope}%
\begin{pgfscope}%
\pgfsys@transformshift{1.032870in}{2.005164in}%
\pgfsys@useobject{currentmarker}{}%
\end{pgfscope}%
\begin{pgfscope}%
\pgfsys@transformshift{0.670412in}{2.012104in}%
\pgfsys@useobject{currentmarker}{}%
\end{pgfscope}%
\begin{pgfscope}%
\pgfsys@transformshift{0.718584in}{1.968782in}%
\pgfsys@useobject{currentmarker}{}%
\end{pgfscope}%
\begin{pgfscope}%
\pgfsys@transformshift{0.609262in}{2.031654in}%
\pgfsys@useobject{currentmarker}{}%
\end{pgfscope}%
\begin{pgfscope}%
\pgfsys@transformshift{0.724620in}{1.991648in}%
\pgfsys@useobject{currentmarker}{}%
\end{pgfscope}%
\begin{pgfscope}%
\pgfsys@transformshift{0.839274in}{1.856160in}%
\pgfsys@useobject{currentmarker}{}%
\end{pgfscope}%
\begin{pgfscope}%
\pgfsys@transformshift{1.123142in}{2.012595in}%
\pgfsys@useobject{currentmarker}{}%
\end{pgfscope}%
\begin{pgfscope}%
\pgfsys@transformshift{0.660581in}{1.923522in}%
\pgfsys@useobject{currentmarker}{}%
\end{pgfscope}%
\begin{pgfscope}%
\pgfsys@transformshift{1.346619in}{2.036789in}%
\pgfsys@useobject{currentmarker}{}%
\end{pgfscope}%
\begin{pgfscope}%
\pgfsys@transformshift{1.032759in}{1.950955in}%
\pgfsys@useobject{currentmarker}{}%
\end{pgfscope}%
\begin{pgfscope}%
\pgfsys@transformshift{1.531144in}{1.856690in}%
\pgfsys@useobject{currentmarker}{}%
\end{pgfscope}%
\begin{pgfscope}%
\pgfsys@transformshift{0.545427in}{2.026499in}%
\pgfsys@useobject{currentmarker}{}%
\end{pgfscope}%
\begin{pgfscope}%
\pgfsys@transformshift{1.530607in}{1.859832in}%
\pgfsys@useobject{currentmarker}{}%
\end{pgfscope}%
\begin{pgfscope}%
\pgfsys@transformshift{1.562080in}{1.869870in}%
\pgfsys@useobject{currentmarker}{}%
\end{pgfscope}%
\begin{pgfscope}%
\pgfsys@transformshift{1.076266in}{2.034090in}%
\pgfsys@useobject{currentmarker}{}%
\end{pgfscope}%
\begin{pgfscope}%
\pgfsys@transformshift{0.596024in}{2.028774in}%
\pgfsys@useobject{currentmarker}{}%
\end{pgfscope}%
\begin{pgfscope}%
\pgfsys@transformshift{1.283932in}{1.867886in}%
\pgfsys@useobject{currentmarker}{}%
\end{pgfscope}%
\begin{pgfscope}%
\pgfsys@transformshift{1.277638in}{1.976762in}%
\pgfsys@useobject{currentmarker}{}%
\end{pgfscope}%
\begin{pgfscope}%
\pgfsys@transformshift{0.915846in}{1.875080in}%
\pgfsys@useobject{currentmarker}{}%
\end{pgfscope}%
\begin{pgfscope}%
\pgfsys@transformshift{0.580084in}{1.966785in}%
\pgfsys@useobject{currentmarker}{}%
\end{pgfscope}%
\begin{pgfscope}%
\pgfsys@transformshift{0.665210in}{1.973612in}%
\pgfsys@useobject{currentmarker}{}%
\end{pgfscope}%
\begin{pgfscope}%
\pgfsys@transformshift{1.282470in}{2.014388in}%
\pgfsys@useobject{currentmarker}{}%
\end{pgfscope}%
\begin{pgfscope}%
\pgfsys@transformshift{0.656693in}{2.043196in}%
\pgfsys@useobject{currentmarker}{}%
\end{pgfscope}%
\begin{pgfscope}%
\pgfsys@transformshift{1.353432in}{1.991566in}%
\pgfsys@useobject{currentmarker}{}%
\end{pgfscope}%
\begin{pgfscope}%
\pgfsys@transformshift{0.684445in}{2.088056in}%
\pgfsys@useobject{currentmarker}{}%
\end{pgfscope}%
\begin{pgfscope}%
\pgfsys@transformshift{1.918484in}{1.870224in}%
\pgfsys@useobject{currentmarker}{}%
\end{pgfscope}%
\begin{pgfscope}%
\pgfsys@transformshift{1.424728in}{1.887072in}%
\pgfsys@useobject{currentmarker}{}%
\end{pgfscope}%
\begin{pgfscope}%
\pgfsys@transformshift{0.746503in}{1.971120in}%
\pgfsys@useobject{currentmarker}{}%
\end{pgfscope}%
\begin{pgfscope}%
\pgfsys@transformshift{0.827814in}{2.040072in}%
\pgfsys@useobject{currentmarker}{}%
\end{pgfscope}%
\begin{pgfscope}%
\pgfsys@transformshift{1.171537in}{1.944879in}%
\pgfsys@useobject{currentmarker}{}%
\end{pgfscope}%
\begin{pgfscope}%
\pgfsys@transformshift{1.125382in}{2.026552in}%
\pgfsys@useobject{currentmarker}{}%
\end{pgfscope}%
\begin{pgfscope}%
\pgfsys@transformshift{1.248979in}{1.978258in}%
\pgfsys@useobject{currentmarker}{}%
\end{pgfscope}%
\begin{pgfscope}%
\pgfsys@transformshift{0.729433in}{2.020271in}%
\pgfsys@useobject{currentmarker}{}%
\end{pgfscope}%
\begin{pgfscope}%
\pgfsys@transformshift{0.585879in}{2.023403in}%
\pgfsys@useobject{currentmarker}{}%
\end{pgfscope}%
\begin{pgfscope}%
\pgfsys@transformshift{1.040794in}{2.016765in}%
\pgfsys@useobject{currentmarker}{}%
\end{pgfscope}%
\begin{pgfscope}%
\pgfsys@transformshift{0.953003in}{2.028228in}%
\pgfsys@useobject{currentmarker}{}%
\end{pgfscope}%
\begin{pgfscope}%
\pgfsys@transformshift{1.076747in}{1.980385in}%
\pgfsys@useobject{currentmarker}{}%
\end{pgfscope}%
\begin{pgfscope}%
\pgfsys@transformshift{0.880318in}{2.076556in}%
\pgfsys@useobject{currentmarker}{}%
\end{pgfscope}%
\begin{pgfscope}%
\pgfsys@transformshift{1.223782in}{2.037559in}%
\pgfsys@useobject{currentmarker}{}%
\end{pgfscope}%
\begin{pgfscope}%
\pgfsys@transformshift{1.089466in}{2.014859in}%
\pgfsys@useobject{currentmarker}{}%
\end{pgfscope}%
\begin{pgfscope}%
\pgfsys@transformshift{0.826666in}{1.983505in}%
\pgfsys@useobject{currentmarker}{}%
\end{pgfscope}%
\begin{pgfscope}%
\pgfsys@transformshift{1.377851in}{1.986353in}%
\pgfsys@useobject{currentmarker}{}%
\end{pgfscope}%
\begin{pgfscope}%
\pgfsys@transformshift{0.598024in}{2.041453in}%
\pgfsys@useobject{currentmarker}{}%
\end{pgfscope}%
\begin{pgfscope}%
\pgfsys@transformshift{0.697849in}{1.893444in}%
\pgfsys@useobject{currentmarker}{}%
\end{pgfscope}%
\begin{pgfscope}%
\pgfsys@transformshift{1.266270in}{1.972278in}%
\pgfsys@useobject{currentmarker}{}%
\end{pgfscope}%
\begin{pgfscope}%
\pgfsys@transformshift{0.769293in}{1.890142in}%
\pgfsys@useobject{currentmarker}{}%
\end{pgfscope}%
\begin{pgfscope}%
\pgfsys@transformshift{0.763591in}{1.827403in}%
\pgfsys@useobject{currentmarker}{}%
\end{pgfscope}%
\begin{pgfscope}%
\pgfsys@transformshift{1.545270in}{1.864081in}%
\pgfsys@useobject{currentmarker}{}%
\end{pgfscope}%
\begin{pgfscope}%
\pgfsys@transformshift{0.781974in}{1.818782in}%
\pgfsys@useobject{currentmarker}{}%
\end{pgfscope}%
\begin{pgfscope}%
\pgfsys@transformshift{0.637143in}{1.992324in}%
\pgfsys@useobject{currentmarker}{}%
\end{pgfscope}%
\begin{pgfscope}%
\pgfsys@transformshift{1.059196in}{1.888948in}%
\pgfsys@useobject{currentmarker}{}%
\end{pgfscope}%
\begin{pgfscope}%
\pgfsys@transformshift{1.032796in}{2.044775in}%
\pgfsys@useobject{currentmarker}{}%
\end{pgfscope}%
\begin{pgfscope}%
\pgfsys@transformshift{0.683742in}{2.015235in}%
\pgfsys@useobject{currentmarker}{}%
\end{pgfscope}%
\begin{pgfscope}%
\pgfsys@transformshift{1.021114in}{1.977867in}%
\pgfsys@useobject{currentmarker}{}%
\end{pgfscope}%
\begin{pgfscope}%
\pgfsys@transformshift{0.563237in}{2.013715in}%
\pgfsys@useobject{currentmarker}{}%
\end{pgfscope}%
\begin{pgfscope}%
\pgfsys@transformshift{0.614223in}{2.005069in}%
\pgfsys@useobject{currentmarker}{}%
\end{pgfscope}%
\begin{pgfscope}%
\pgfsys@transformshift{0.779401in}{1.995653in}%
\pgfsys@useobject{currentmarker}{}%
\end{pgfscope}%
\begin{pgfscope}%
\pgfsys@transformshift{0.992603in}{1.995789in}%
\pgfsys@useobject{currentmarker}{}%
\end{pgfscope}%
\begin{pgfscope}%
\pgfsys@transformshift{1.694285in}{1.788652in}%
\pgfsys@useobject{currentmarker}{}%
\end{pgfscope}%
\begin{pgfscope}%
\pgfsys@transformshift{0.778624in}{1.887671in}%
\pgfsys@useobject{currentmarker}{}%
\end{pgfscope}%
\begin{pgfscope}%
\pgfsys@transformshift{1.340343in}{1.994457in}%
\pgfsys@useobject{currentmarker}{}%
\end{pgfscope}%
\begin{pgfscope}%
\pgfsys@transformshift{0.923566in}{1.983184in}%
\pgfsys@useobject{currentmarker}{}%
\end{pgfscope}%
\begin{pgfscope}%
\pgfsys@transformshift{0.789621in}{1.685869in}%
\pgfsys@useobject{currentmarker}{}%
\end{pgfscope}%
\begin{pgfscope}%
\pgfsys@transformshift{0.869895in}{1.792739in}%
\pgfsys@useobject{currentmarker}{}%
\end{pgfscope}%
\begin{pgfscope}%
\pgfsys@transformshift{1.119513in}{1.917093in}%
\pgfsys@useobject{currentmarker}{}%
\end{pgfscope}%
\begin{pgfscope}%
\pgfsys@transformshift{1.521480in}{2.036628in}%
\pgfsys@useobject{currentmarker}{}%
\end{pgfscope}%
\begin{pgfscope}%
\pgfsys@transformshift{1.492025in}{1.980598in}%
\pgfsys@useobject{currentmarker}{}%
\end{pgfscope}%
\begin{pgfscope}%
\pgfsys@transformshift{1.103758in}{2.100726in}%
\pgfsys@useobject{currentmarker}{}%
\end{pgfscope}%
\begin{pgfscope}%
\pgfsys@transformshift{0.698127in}{1.987984in}%
\pgfsys@useobject{currentmarker}{}%
\end{pgfscope}%
\begin{pgfscope}%
\pgfsys@transformshift{1.029871in}{2.043772in}%
\pgfsys@useobject{currentmarker}{}%
\end{pgfscope}%
\begin{pgfscope}%
\pgfsys@transformshift{0.931582in}{2.054460in}%
\pgfsys@useobject{currentmarker}{}%
\end{pgfscope}%
\begin{pgfscope}%
\pgfsys@transformshift{0.910792in}{1.912305in}%
\pgfsys@useobject{currentmarker}{}%
\end{pgfscope}%
\begin{pgfscope}%
\pgfsys@transformshift{0.605355in}{1.960905in}%
\pgfsys@useobject{currentmarker}{}%
\end{pgfscope}%
\begin{pgfscope}%
\pgfsys@transformshift{1.054013in}{2.007874in}%
\pgfsys@useobject{currentmarker}{}%
\end{pgfscope}%
\begin{pgfscope}%
\pgfsys@transformshift{1.098612in}{1.908473in}%
\pgfsys@useobject{currentmarker}{}%
\end{pgfscope}%
\begin{pgfscope}%
\pgfsys@transformshift{0.861509in}{2.012303in}%
\pgfsys@useobject{currentmarker}{}%
\end{pgfscope}%
\begin{pgfscope}%
\pgfsys@transformshift{0.582121in}{2.053269in}%
\pgfsys@useobject{currentmarker}{}%
\end{pgfscope}%
\begin{pgfscope}%
\pgfsys@transformshift{1.167241in}{1.903157in}%
\pgfsys@useobject{currentmarker}{}%
\end{pgfscope}%
\begin{pgfscope}%
\pgfsys@transformshift{1.281729in}{2.017226in}%
\pgfsys@useobject{currentmarker}{}%
\end{pgfscope}%
\begin{pgfscope}%
\pgfsys@transformshift{1.375426in}{2.049855in}%
\pgfsys@useobject{currentmarker}{}%
\end{pgfscope}%
\begin{pgfscope}%
\pgfsys@transformshift{1.374241in}{2.010275in}%
\pgfsys@useobject{currentmarker}{}%
\end{pgfscope}%
\begin{pgfscope}%
\pgfsys@transformshift{0.897036in}{2.053543in}%
\pgfsys@useobject{currentmarker}{}%
\end{pgfscope}%
\begin{pgfscope}%
\pgfsys@transformshift{1.073507in}{2.004103in}%
\pgfsys@useobject{currentmarker}{}%
\end{pgfscope}%
\begin{pgfscope}%
\pgfsys@transformshift{0.689259in}{1.994698in}%
\pgfsys@useobject{currentmarker}{}%
\end{pgfscope}%
\begin{pgfscope}%
\pgfsys@transformshift{1.117181in}{1.871260in}%
\pgfsys@useobject{currentmarker}{}%
\end{pgfscope}%
\begin{pgfscope}%
\pgfsys@transformshift{1.040257in}{2.010641in}%
\pgfsys@useobject{currentmarker}{}%
\end{pgfscope}%
\begin{pgfscope}%
\pgfsys@transformshift{0.757259in}{2.075771in}%
\pgfsys@useobject{currentmarker}{}%
\end{pgfscope}%
\begin{pgfscope}%
\pgfsys@transformshift{0.929602in}{2.040481in}%
\pgfsys@useobject{currentmarker}{}%
\end{pgfscope}%
\begin{pgfscope}%
\pgfsys@transformshift{0.739375in}{1.894178in}%
\pgfsys@useobject{currentmarker}{}%
\end{pgfscope}%
\begin{pgfscope}%
\pgfsys@transformshift{1.051791in}{1.884953in}%
\pgfsys@useobject{currentmarker}{}%
\end{pgfscope}%
\begin{pgfscope}%
\pgfsys@transformshift{1.033888in}{2.008531in}%
\pgfsys@useobject{currentmarker}{}%
\end{pgfscope}%
\begin{pgfscope}%
\pgfsys@transformshift{1.087559in}{1.927554in}%
\pgfsys@useobject{currentmarker}{}%
\end{pgfscope}%
\begin{pgfscope}%
\pgfsys@transformshift{1.159095in}{1.921443in}%
\pgfsys@useobject{currentmarker}{}%
\end{pgfscope}%
\begin{pgfscope}%
\pgfsys@transformshift{0.813373in}{2.010583in}%
\pgfsys@useobject{currentmarker}{}%
\end{pgfscope}%
\begin{pgfscope}%
\pgfsys@transformshift{1.640577in}{1.847951in}%
\pgfsys@useobject{currentmarker}{}%
\end{pgfscope}%
\begin{pgfscope}%
\pgfsys@transformshift{2.178211in}{1.843277in}%
\pgfsys@useobject{currentmarker}{}%
\end{pgfscope}%
\begin{pgfscope}%
\pgfsys@transformshift{0.693850in}{1.990432in}%
\pgfsys@useobject{currentmarker}{}%
\end{pgfscope}%
\begin{pgfscope}%
\pgfsys@transformshift{1.166519in}{1.909206in}%
\pgfsys@useobject{currentmarker}{}%
\end{pgfscope}%
\begin{pgfscope}%
\pgfsys@transformshift{0.643734in}{1.955973in}%
\pgfsys@useobject{currentmarker}{}%
\end{pgfscope}%
\begin{pgfscope}%
\pgfsys@transformshift{0.891278in}{1.962821in}%
\pgfsys@useobject{currentmarker}{}%
\end{pgfscope}%
\begin{pgfscope}%
\pgfsys@transformshift{0.789639in}{1.897406in}%
\pgfsys@useobject{currentmarker}{}%
\end{pgfscope}%
\begin{pgfscope}%
\pgfsys@transformshift{0.905830in}{1.924636in}%
\pgfsys@useobject{currentmarker}{}%
\end{pgfscope}%
\begin{pgfscope}%
\pgfsys@transformshift{1.170666in}{2.077307in}%
\pgfsys@useobject{currentmarker}{}%
\end{pgfscope}%
\begin{pgfscope}%
\pgfsys@transformshift{0.851123in}{2.032509in}%
\pgfsys@useobject{currentmarker}{}%
\end{pgfscope}%
\begin{pgfscope}%
\pgfsys@transformshift{1.511890in}{2.069276in}%
\pgfsys@useobject{currentmarker}{}%
\end{pgfscope}%
\begin{pgfscope}%
\pgfsys@transformshift{1.126178in}{1.966352in}%
\pgfsys@useobject{currentmarker}{}%
\end{pgfscope}%
\begin{pgfscope}%
\pgfsys@transformshift{0.794434in}{2.026609in}%
\pgfsys@useobject{currentmarker}{}%
\end{pgfscope}%
\begin{pgfscope}%
\pgfsys@transformshift{1.144285in}{1.890494in}%
\pgfsys@useobject{currentmarker}{}%
\end{pgfscope}%
\begin{pgfscope}%
\pgfsys@transformshift{0.668153in}{1.973253in}%
\pgfsys@useobject{currentmarker}{}%
\end{pgfscope}%
\begin{pgfscope}%
\pgfsys@transformshift{0.919012in}{1.923283in}%
\pgfsys@useobject{currentmarker}{}%
\end{pgfscope}%
\begin{pgfscope}%
\pgfsys@transformshift{1.167260in}{2.062689in}%
\pgfsys@useobject{currentmarker}{}%
\end{pgfscope}%
\begin{pgfscope}%
\pgfsys@transformshift{0.990696in}{2.074525in}%
\pgfsys@useobject{currentmarker}{}%
\end{pgfscope}%
\begin{pgfscope}%
\pgfsys@transformshift{0.884725in}{1.864126in}%
\pgfsys@useobject{currentmarker}{}%
\end{pgfscope}%
\begin{pgfscope}%
\pgfsys@transformshift{0.825592in}{1.907053in}%
\pgfsys@useobject{currentmarker}{}%
\end{pgfscope}%
\begin{pgfscope}%
\pgfsys@transformshift{0.824667in}{1.894515in}%
\pgfsys@useobject{currentmarker}{}%
\end{pgfscope}%
\begin{pgfscope}%
\pgfsys@transformshift{1.211526in}{2.037414in}%
\pgfsys@useobject{currentmarker}{}%
\end{pgfscope}%
\begin{pgfscope}%
\pgfsys@transformshift{1.235205in}{2.051345in}%
\pgfsys@useobject{currentmarker}{}%
\end{pgfscope}%
\begin{pgfscope}%
\pgfsys@transformshift{1.055049in}{1.971629in}%
\pgfsys@useobject{currentmarker}{}%
\end{pgfscope}%
\begin{pgfscope}%
\pgfsys@transformshift{1.121531in}{2.006535in}%
\pgfsys@useobject{currentmarker}{}%
\end{pgfscope}%
\begin{pgfscope}%
\pgfsys@transformshift{1.293522in}{1.892655in}%
\pgfsys@useobject{currentmarker}{}%
\end{pgfscope}%
\begin{pgfscope}%
\pgfsys@transformshift{0.598931in}{1.936714in}%
\pgfsys@useobject{currentmarker}{}%
\end{pgfscope}%
\begin{pgfscope}%
\pgfsys@transformshift{0.969776in}{2.036617in}%
\pgfsys@useobject{currentmarker}{}%
\end{pgfscope}%
\begin{pgfscope}%
\pgfsys@transformshift{1.073637in}{1.997028in}%
\pgfsys@useobject{currentmarker}{}%
\end{pgfscope}%
\begin{pgfscope}%
\pgfsys@transformshift{1.012061in}{2.015656in}%
\pgfsys@useobject{currentmarker}{}%
\end{pgfscope}%
\begin{pgfscope}%
\pgfsys@transformshift{0.981662in}{2.040140in}%
\pgfsys@useobject{currentmarker}{}%
\end{pgfscope}%
\begin{pgfscope}%
\pgfsys@transformshift{0.639994in}{2.002733in}%
\pgfsys@useobject{currentmarker}{}%
\end{pgfscope}%
\begin{pgfscope}%
\pgfsys@transformshift{0.958131in}{1.987800in}%
\pgfsys@useobject{currentmarker}{}%
\end{pgfscope}%
\begin{pgfscope}%
\pgfsys@transformshift{0.822908in}{1.829647in}%
\pgfsys@useobject{currentmarker}{}%
\end{pgfscope}%
\begin{pgfscope}%
\pgfsys@transformshift{0.720176in}{1.888058in}%
\pgfsys@useobject{currentmarker}{}%
\end{pgfscope}%
\begin{pgfscope}%
\pgfsys@transformshift{0.636847in}{1.896038in}%
\pgfsys@useobject{currentmarker}{}%
\end{pgfscope}%
\begin{pgfscope}%
\pgfsys@transformshift{0.850863in}{1.941612in}%
\pgfsys@useobject{currentmarker}{}%
\end{pgfscope}%
\begin{pgfscope}%
\pgfsys@transformshift{1.514574in}{2.011407in}%
\pgfsys@useobject{currentmarker}{}%
\end{pgfscope}%
\begin{pgfscope}%
\pgfsys@transformshift{0.970535in}{2.011680in}%
\pgfsys@useobject{currentmarker}{}%
\end{pgfscope}%
\begin{pgfscope}%
\pgfsys@transformshift{0.976256in}{2.008401in}%
\pgfsys@useobject{currentmarker}{}%
\end{pgfscope}%
\begin{pgfscope}%
\pgfsys@transformshift{0.879633in}{2.001812in}%
\pgfsys@useobject{currentmarker}{}%
\end{pgfscope}%
\begin{pgfscope}%
\pgfsys@transformshift{0.655508in}{1.798558in}%
\pgfsys@useobject{currentmarker}{}%
\end{pgfscope}%
\begin{pgfscope}%
\pgfsys@transformshift{0.702514in}{1.957564in}%
\pgfsys@useobject{currentmarker}{}%
\end{pgfscope}%
\begin{pgfscope}%
\pgfsys@transformshift{0.714474in}{1.850830in}%
\pgfsys@useobject{currentmarker}{}%
\end{pgfscope}%
\begin{pgfscope}%
\pgfsys@transformshift{1.412398in}{1.872459in}%
\pgfsys@useobject{currentmarker}{}%
\end{pgfscope}%
\begin{pgfscope}%
\pgfsys@transformshift{1.041442in}{2.002337in}%
\pgfsys@useobject{currentmarker}{}%
\end{pgfscope}%
\begin{pgfscope}%
\pgfsys@transformshift{0.890575in}{1.959506in}%
\pgfsys@useobject{currentmarker}{}%
\end{pgfscope}%
\begin{pgfscope}%
\pgfsys@transformshift{1.229854in}{1.941812in}%
\pgfsys@useobject{currentmarker}{}%
\end{pgfscope}%
\begin{pgfscope}%
\pgfsys@transformshift{1.163502in}{2.048637in}%
\pgfsys@useobject{currentmarker}{}%
\end{pgfscope}%
\begin{pgfscope}%
\pgfsys@transformshift{0.904941in}{2.063051in}%
\pgfsys@useobject{currentmarker}{}%
\end{pgfscope}%
\begin{pgfscope}%
\pgfsys@transformshift{0.963778in}{2.050187in}%
\pgfsys@useobject{currentmarker}{}%
\end{pgfscope}%
\begin{pgfscope}%
\pgfsys@transformshift{1.162317in}{1.962689in}%
\pgfsys@useobject{currentmarker}{}%
\end{pgfscope}%
\begin{pgfscope}%
\pgfsys@transformshift{0.708402in}{1.912127in}%
\pgfsys@useobject{currentmarker}{}%
\end{pgfscope}%
\begin{pgfscope}%
\pgfsys@transformshift{0.704162in}{2.002371in}%
\pgfsys@useobject{currentmarker}{}%
\end{pgfscope}%
\begin{pgfscope}%
\pgfsys@transformshift{1.154782in}{1.880313in}%
\pgfsys@useobject{currentmarker}{}%
\end{pgfscope}%
\begin{pgfscope}%
\pgfsys@transformshift{0.978699in}{1.871984in}%
\pgfsys@useobject{currentmarker}{}%
\end{pgfscope}%
\begin{pgfscope}%
\pgfsys@transformshift{1.451036in}{2.050749in}%
\pgfsys@useobject{currentmarker}{}%
\end{pgfscope}%
\begin{pgfscope}%
\pgfsys@transformshift{1.034425in}{2.017903in}%
\pgfsys@useobject{currentmarker}{}%
\end{pgfscope}%
\begin{pgfscope}%
\pgfsys@transformshift{0.615926in}{1.966087in}%
\pgfsys@useobject{currentmarker}{}%
\end{pgfscope}%
\begin{pgfscope}%
\pgfsys@transformshift{1.270158in}{1.922948in}%
\pgfsys@useobject{currentmarker}{}%
\end{pgfscope}%
\begin{pgfscope}%
\pgfsys@transformshift{0.708883in}{1.933846in}%
\pgfsys@useobject{currentmarker}{}%
\end{pgfscope}%
\begin{pgfscope}%
\pgfsys@transformshift{0.740763in}{1.832401in}%
\pgfsys@useobject{currentmarker}{}%
\end{pgfscope}%
\begin{pgfscope}%
\pgfsys@transformshift{0.614020in}{1.913367in}%
\pgfsys@useobject{currentmarker}{}%
\end{pgfscope}%
\begin{pgfscope}%
\pgfsys@transformshift{1.067953in}{1.994083in}%
\pgfsys@useobject{currentmarker}{}%
\end{pgfscope}%
\begin{pgfscope}%
\pgfsys@transformshift{0.777420in}{1.964697in}%
\pgfsys@useobject{currentmarker}{}%
\end{pgfscope}%
\begin{pgfscope}%
\pgfsys@transformshift{0.653398in}{1.949569in}%
\pgfsys@useobject{currentmarker}{}%
\end{pgfscope}%
\begin{pgfscope}%
\pgfsys@transformshift{0.733321in}{1.891370in}%
\pgfsys@useobject{currentmarker}{}%
\end{pgfscope}%
\begin{pgfscope}%
\pgfsys@transformshift{0.711753in}{2.011826in}%
\pgfsys@useobject{currentmarker}{}%
\end{pgfscope}%
\begin{pgfscope}%
\pgfsys@transformshift{1.082709in}{2.064932in}%
\pgfsys@useobject{currentmarker}{}%
\end{pgfscope}%
\begin{pgfscope}%
\pgfsys@transformshift{1.677845in}{1.820001in}%
\pgfsys@useobject{currentmarker}{}%
\end{pgfscope}%
\begin{pgfscope}%
\pgfsys@transformshift{0.696497in}{1.980161in}%
\pgfsys@useobject{currentmarker}{}%
\end{pgfscope}%
\begin{pgfscope}%
\pgfsys@transformshift{0.757129in}{1.937091in}%
\pgfsys@useobject{currentmarker}{}%
\end{pgfscope}%
\begin{pgfscope}%
\pgfsys@transformshift{1.072767in}{2.083918in}%
\pgfsys@useobject{currentmarker}{}%
\end{pgfscope}%
\begin{pgfscope}%
\pgfsys@transformshift{1.280452in}{2.075040in}%
\pgfsys@useobject{currentmarker}{}%
\end{pgfscope}%
\begin{pgfscope}%
\pgfsys@transformshift{0.830258in}{1.937079in}%
\pgfsys@useobject{currentmarker}{}%
\end{pgfscope}%
\begin{pgfscope}%
\pgfsys@transformshift{0.709420in}{1.843691in}%
\pgfsys@useobject{currentmarker}{}%
\end{pgfscope}%
\begin{pgfscope}%
\pgfsys@transformshift{1.313295in}{1.990028in}%
\pgfsys@useobject{currentmarker}{}%
\end{pgfscope}%
\begin{pgfscope}%
\pgfsys@transformshift{0.966851in}{2.029552in}%
\pgfsys@useobject{currentmarker}{}%
\end{pgfscope}%
\begin{pgfscope}%
\pgfsys@transformshift{0.979088in}{2.032419in}%
\pgfsys@useobject{currentmarker}{}%
\end{pgfscope}%
\begin{pgfscope}%
\pgfsys@transformshift{0.852955in}{1.844996in}%
\pgfsys@useobject{currentmarker}{}%
\end{pgfscope}%
\begin{pgfscope}%
\pgfsys@transformshift{0.752519in}{1.901183in}%
\pgfsys@useobject{currentmarker}{}%
\end{pgfscope}%
\begin{pgfscope}%
\pgfsys@transformshift{0.942839in}{2.015536in}%
\pgfsys@useobject{currentmarker}{}%
\end{pgfscope}%
\begin{pgfscope}%
\pgfsys@transformshift{0.747724in}{1.997038in}%
\pgfsys@useobject{currentmarker}{}%
\end{pgfscope}%
\begin{pgfscope}%
\pgfsys@transformshift{1.116200in}{2.062279in}%
\pgfsys@useobject{currentmarker}{}%
\end{pgfscope}%
\begin{pgfscope}%
\pgfsys@transformshift{0.695424in}{1.903673in}%
\pgfsys@useobject{currentmarker}{}%
\end{pgfscope}%
\begin{pgfscope}%
\pgfsys@transformshift{1.740661in}{1.997382in}%
\pgfsys@useobject{currentmarker}{}%
\end{pgfscope}%
\begin{pgfscope}%
\pgfsys@transformshift{0.688629in}{1.984815in}%
\pgfsys@useobject{currentmarker}{}%
\end{pgfscope}%
\begin{pgfscope}%
\pgfsys@transformshift{0.892574in}{2.043607in}%
\pgfsys@useobject{currentmarker}{}%
\end{pgfscope}%
\begin{pgfscope}%
\pgfsys@transformshift{0.811004in}{1.955391in}%
\pgfsys@useobject{currentmarker}{}%
\end{pgfscope}%
\begin{pgfscope}%
\pgfsys@transformshift{1.098889in}{2.075656in}%
\pgfsys@useobject{currentmarker}{}%
\end{pgfscope}%
\begin{pgfscope}%
\pgfsys@transformshift{0.812874in}{1.829306in}%
\pgfsys@useobject{currentmarker}{}%
\end{pgfscope}%
\begin{pgfscope}%
\pgfsys@transformshift{0.678317in}{2.009037in}%
\pgfsys@useobject{currentmarker}{}%
\end{pgfscope}%
\begin{pgfscope}%
\pgfsys@transformshift{1.330735in}{1.823368in}%
\pgfsys@useobject{currentmarker}{}%
\end{pgfscope}%
\begin{pgfscope}%
\pgfsys@transformshift{0.845531in}{1.835160in}%
\pgfsys@useobject{currentmarker}{}%
\end{pgfscope}%
\begin{pgfscope}%
\pgfsys@transformshift{1.154911in}{1.987812in}%
\pgfsys@useobject{currentmarker}{}%
\end{pgfscope}%
\begin{pgfscope}%
\pgfsys@transformshift{0.984050in}{2.097649in}%
\pgfsys@useobject{currentmarker}{}%
\end{pgfscope}%
\begin{pgfscope}%
\pgfsys@transformshift{0.736913in}{1.967089in}%
\pgfsys@useobject{currentmarker}{}%
\end{pgfscope}%
\begin{pgfscope}%
\pgfsys@transformshift{0.696775in}{1.865922in}%
\pgfsys@useobject{currentmarker}{}%
\end{pgfscope}%
\begin{pgfscope}%
\pgfsys@transformshift{0.594914in}{2.024362in}%
\pgfsys@useobject{currentmarker}{}%
\end{pgfscope}%
\begin{pgfscope}%
\pgfsys@transformshift{1.029056in}{1.989443in}%
\pgfsys@useobject{currentmarker}{}%
\end{pgfscope}%
\begin{pgfscope}%
\pgfsys@transformshift{1.353377in}{2.053639in}%
\pgfsys@useobject{currentmarker}{}%
\end{pgfscope}%
\begin{pgfscope}%
\pgfsys@transformshift{0.844328in}{2.015870in}%
\pgfsys@useobject{currentmarker}{}%
\end{pgfscope}%
\begin{pgfscope}%
\pgfsys@transformshift{0.897277in}{2.014388in}%
\pgfsys@useobject{currentmarker}{}%
\end{pgfscope}%
\begin{pgfscope}%
\pgfsys@transformshift{0.995325in}{1.894373in}%
\pgfsys@useobject{currentmarker}{}%
\end{pgfscope}%
\begin{pgfscope}%
\pgfsys@transformshift{1.056327in}{1.931272in}%
\pgfsys@useobject{currentmarker}{}%
\end{pgfscope}%
\begin{pgfscope}%
\pgfsys@transformshift{0.564107in}{1.951761in}%
\pgfsys@useobject{currentmarker}{}%
\end{pgfscope}%
\begin{pgfscope}%
\pgfsys@transformshift{0.656175in}{1.970073in}%
\pgfsys@useobject{currentmarker}{}%
\end{pgfscope}%
\begin{pgfscope}%
\pgfsys@transformshift{1.144433in}{1.934768in}%
\pgfsys@useobject{currentmarker}{}%
\end{pgfscope}%
\begin{pgfscope}%
\pgfsys@transformshift{0.628238in}{1.980185in}%
\pgfsys@useobject{currentmarker}{}%
\end{pgfscope}%
\begin{pgfscope}%
\pgfsys@transformshift{1.040257in}{1.962259in}%
\pgfsys@useobject{currentmarker}{}%
\end{pgfscope}%
\begin{pgfscope}%
\pgfsys@transformshift{0.967406in}{1.969977in}%
\pgfsys@useobject{currentmarker}{}%
\end{pgfscope}%
\begin{pgfscope}%
\pgfsys@transformshift{1.329216in}{2.018483in}%
\pgfsys@useobject{currentmarker}{}%
\end{pgfscope}%
\begin{pgfscope}%
\pgfsys@transformshift{0.564607in}{2.004427in}%
\pgfsys@useobject{currentmarker}{}%
\end{pgfscope}%
\begin{pgfscope}%
\pgfsys@transformshift{0.742263in}{2.040926in}%
\pgfsys@useobject{currentmarker}{}%
\end{pgfscope}%
\begin{pgfscope}%
\pgfsys@transformshift{0.629367in}{1.964603in}%
\pgfsys@useobject{currentmarker}{}%
\end{pgfscope}%
\begin{pgfscope}%
\pgfsys@transformshift{0.569846in}{1.915175in}%
\pgfsys@useobject{currentmarker}{}%
\end{pgfscope}%
\begin{pgfscope}%
\pgfsys@transformshift{1.470660in}{1.939663in}%
\pgfsys@useobject{currentmarker}{}%
\end{pgfscope}%
\begin{pgfscope}%
\pgfsys@transformshift{1.976543in}{1.812085in}%
\pgfsys@useobject{currentmarker}{}%
\end{pgfscope}%
\begin{pgfscope}%
\pgfsys@transformshift{1.014486in}{2.006046in}%
\pgfsys@useobject{currentmarker}{}%
\end{pgfscope}%
\begin{pgfscope}%
\pgfsys@transformshift{1.379055in}{1.971658in}%
\pgfsys@useobject{currentmarker}{}%
\end{pgfscope}%
\begin{pgfscope}%
\pgfsys@transformshift{0.658063in}{2.028671in}%
\pgfsys@useobject{currentmarker}{}%
\end{pgfscope}%
\begin{pgfscope}%
\pgfsys@transformshift{0.965129in}{2.025534in}%
\pgfsys@useobject{currentmarker}{}%
\end{pgfscope}%
\begin{pgfscope}%
\pgfsys@transformshift{0.714733in}{1.960440in}%
\pgfsys@useobject{currentmarker}{}%
\end{pgfscope}%
\begin{pgfscope}%
\pgfsys@transformshift{0.734580in}{2.074540in}%
\pgfsys@useobject{currentmarker}{}%
\end{pgfscope}%
\begin{pgfscope}%
\pgfsys@transformshift{1.146876in}{1.970869in}%
\pgfsys@useobject{currentmarker}{}%
\end{pgfscope}%
\begin{pgfscope}%
\pgfsys@transformshift{1.112552in}{1.993008in}%
\pgfsys@useobject{currentmarker}{}%
\end{pgfscope}%
\begin{pgfscope}%
\pgfsys@transformshift{0.609724in}{1.989967in}%
\pgfsys@useobject{currentmarker}{}%
\end{pgfscope}%
\begin{pgfscope}%
\pgfsys@transformshift{0.713974in}{1.933826in}%
\pgfsys@useobject{currentmarker}{}%
\end{pgfscope}%
\begin{pgfscope}%
\pgfsys@transformshift{0.625054in}{1.957694in}%
\pgfsys@useobject{currentmarker}{}%
\end{pgfscope}%
\begin{pgfscope}%
\pgfsys@transformshift{0.695701in}{1.973624in}%
\pgfsys@useobject{currentmarker}{}%
\end{pgfscope}%
\begin{pgfscope}%
\pgfsys@transformshift{0.630052in}{1.713314in}%
\pgfsys@useobject{currentmarker}{}%
\end{pgfscope}%
\begin{pgfscope}%
\pgfsys@transformshift{1.032148in}{1.937363in}%
\pgfsys@useobject{currentmarker}{}%
\end{pgfscope}%
\begin{pgfscope}%
\pgfsys@transformshift{0.974293in}{2.050689in}%
\pgfsys@useobject{currentmarker}{}%
\end{pgfscope}%
\begin{pgfscope}%
\pgfsys@transformshift{0.679965in}{1.945508in}%
\pgfsys@useobject{currentmarker}{}%
\end{pgfscope}%
\begin{pgfscope}%
\pgfsys@transformshift{1.437521in}{1.948701in}%
\pgfsys@useobject{currentmarker}{}%
\end{pgfscope}%
\begin{pgfscope}%
\pgfsys@transformshift{0.779957in}{1.936503in}%
\pgfsys@useobject{currentmarker}{}%
\end{pgfscope}%
\begin{pgfscope}%
\pgfsys@transformshift{1.260216in}{1.980669in}%
\pgfsys@useobject{currentmarker}{}%
\end{pgfscope}%
\begin{pgfscope}%
\pgfsys@transformshift{0.783159in}{1.989054in}%
\pgfsys@useobject{currentmarker}{}%
\end{pgfscope}%
\begin{pgfscope}%
\pgfsys@transformshift{0.627534in}{2.009778in}%
\pgfsys@useobject{currentmarker}{}%
\end{pgfscope}%
\begin{pgfscope}%
\pgfsys@transformshift{0.732803in}{1.853653in}%
\pgfsys@useobject{currentmarker}{}%
\end{pgfscope}%
\begin{pgfscope}%
\pgfsys@transformshift{1.221153in}{1.952819in}%
\pgfsys@useobject{currentmarker}{}%
\end{pgfscope}%
\begin{pgfscope}%
\pgfsys@transformshift{0.997343in}{1.919404in}%
\pgfsys@useobject{currentmarker}{}%
\end{pgfscope}%
\begin{pgfscope}%
\pgfsys@transformshift{0.699904in}{1.967514in}%
\pgfsys@useobject{currentmarker}{}%
\end{pgfscope}%
\begin{pgfscope}%
\pgfsys@transformshift{0.739856in}{1.922309in}%
\pgfsys@useobject{currentmarker}{}%
\end{pgfscope}%
\begin{pgfscope}%
\pgfsys@transformshift{0.561941in}{2.055095in}%
\pgfsys@useobject{currentmarker}{}%
\end{pgfscope}%
\begin{pgfscope}%
\pgfsys@transformshift{1.009173in}{2.009975in}%
\pgfsys@useobject{currentmarker}{}%
\end{pgfscope}%
\begin{pgfscope}%
\pgfsys@transformshift{0.728952in}{1.903830in}%
\pgfsys@useobject{currentmarker}{}%
\end{pgfscope}%
\begin{pgfscope}%
\pgfsys@transformshift{1.463440in}{2.056669in}%
\pgfsys@useobject{currentmarker}{}%
\end{pgfscope}%
\begin{pgfscope}%
\pgfsys@transformshift{0.638106in}{1.816489in}%
\pgfsys@useobject{currentmarker}{}%
\end{pgfscope}%
\begin{pgfscope}%
\pgfsys@transformshift{0.986864in}{2.043593in}%
\pgfsys@useobject{currentmarker}{}%
\end{pgfscope}%
\end{pgfscope}%
\begin{pgfscope}%
\pgfpathrectangle{\pgfqpoint{0.526284in}{0.473557in}}{\pgfqpoint{1.651927in}{1.704653in}}%
\pgfusepath{clip}%
\pgfsetbuttcap%
\pgfsetroundjoin%
\definecolor{currentfill}{rgb}{0.121569,0.466667,0.705882}%
\pgfsetfillcolor{currentfill}%
\pgfsetfillopacity{0.150000}%
\pgfsetlinewidth{0.000000pt}%
\definecolor{currentstroke}{rgb}{0.000000,0.000000,0.000000}%
\pgfsetstrokecolor{currentstroke}%
\pgfsetdash{}{0pt}%
\pgfpathmoveto{\pgfqpoint{0.526284in}{1.981159in}}%
\pgfpathlineto{\pgfqpoint{0.526284in}{1.956932in}}%
\pgfpathlineto{\pgfqpoint{0.542970in}{1.957305in}}%
\pgfpathlineto{\pgfqpoint{0.559656in}{1.957537in}}%
\pgfpathlineto{\pgfqpoint{0.576342in}{1.957881in}}%
\pgfpathlineto{\pgfqpoint{0.593028in}{1.958421in}}%
\pgfpathlineto{\pgfqpoint{0.609715in}{1.958748in}}%
\pgfpathlineto{\pgfqpoint{0.626401in}{1.959082in}}%
\pgfpathlineto{\pgfqpoint{0.643087in}{1.959133in}}%
\pgfpathlineto{\pgfqpoint{0.659773in}{1.959408in}}%
\pgfpathlineto{\pgfqpoint{0.676459in}{1.959771in}}%
\pgfpathlineto{\pgfqpoint{0.693145in}{1.959952in}}%
\pgfpathlineto{\pgfqpoint{0.709831in}{1.960198in}}%
\pgfpathlineto{\pgfqpoint{0.726517in}{1.960226in}}%
\pgfpathlineto{\pgfqpoint{0.743204in}{1.960377in}}%
\pgfpathlineto{\pgfqpoint{0.759890in}{1.960465in}}%
\pgfpathlineto{\pgfqpoint{0.776576in}{1.960809in}}%
\pgfpathlineto{\pgfqpoint{0.793262in}{1.960972in}}%
\pgfpathlineto{\pgfqpoint{0.809948in}{1.961118in}}%
\pgfpathlineto{\pgfqpoint{0.826634in}{1.961255in}}%
\pgfpathlineto{\pgfqpoint{0.843320in}{1.961367in}}%
\pgfpathlineto{\pgfqpoint{0.860006in}{1.961639in}}%
\pgfpathlineto{\pgfqpoint{0.876693in}{1.961607in}}%
\pgfpathlineto{\pgfqpoint{0.893379in}{1.961571in}}%
\pgfpathlineto{\pgfqpoint{0.910065in}{1.961586in}}%
\pgfpathlineto{\pgfqpoint{0.926751in}{1.961474in}}%
\pgfpathlineto{\pgfqpoint{0.943437in}{1.961407in}}%
\pgfpathlineto{\pgfqpoint{0.960123in}{1.961394in}}%
\pgfpathlineto{\pgfqpoint{0.976809in}{1.961332in}}%
\pgfpathlineto{\pgfqpoint{0.993495in}{1.961048in}}%
\pgfpathlineto{\pgfqpoint{1.010182in}{1.960858in}}%
\pgfpathlineto{\pgfqpoint{1.026868in}{1.960567in}}%
\pgfpathlineto{\pgfqpoint{1.043554in}{1.960337in}}%
\pgfpathlineto{\pgfqpoint{1.060240in}{1.960167in}}%
\pgfpathlineto{\pgfqpoint{1.076926in}{1.959927in}}%
\pgfpathlineto{\pgfqpoint{1.093612in}{1.959649in}}%
\pgfpathlineto{\pgfqpoint{1.110298in}{1.959190in}}%
\pgfpathlineto{\pgfqpoint{1.126985in}{1.958883in}}%
\pgfpathlineto{\pgfqpoint{1.143671in}{1.958820in}}%
\pgfpathlineto{\pgfqpoint{1.160357in}{1.958307in}}%
\pgfpathlineto{\pgfqpoint{1.177043in}{1.957884in}}%
\pgfpathlineto{\pgfqpoint{1.193729in}{1.957468in}}%
\pgfpathlineto{\pgfqpoint{1.210415in}{1.957234in}}%
\pgfpathlineto{\pgfqpoint{1.227101in}{1.956870in}}%
\pgfpathlineto{\pgfqpoint{1.243787in}{1.956434in}}%
\pgfpathlineto{\pgfqpoint{1.260474in}{1.956105in}}%
\pgfpathlineto{\pgfqpoint{1.277160in}{1.955779in}}%
\pgfpathlineto{\pgfqpoint{1.293846in}{1.955453in}}%
\pgfpathlineto{\pgfqpoint{1.310532in}{1.955029in}}%
\pgfpathlineto{\pgfqpoint{1.327218in}{1.954544in}}%
\pgfpathlineto{\pgfqpoint{1.343904in}{1.954078in}}%
\pgfpathlineto{\pgfqpoint{1.360590in}{1.953787in}}%
\pgfpathlineto{\pgfqpoint{1.377276in}{1.953470in}}%
\pgfpathlineto{\pgfqpoint{1.393963in}{1.953156in}}%
\pgfpathlineto{\pgfqpoint{1.410649in}{1.952683in}}%
\pgfpathlineto{\pgfqpoint{1.427335in}{1.952293in}}%
\pgfpathlineto{\pgfqpoint{1.444021in}{1.951881in}}%
\pgfpathlineto{\pgfqpoint{1.460707in}{1.951400in}}%
\pgfpathlineto{\pgfqpoint{1.477393in}{1.950903in}}%
\pgfpathlineto{\pgfqpoint{1.494079in}{1.950449in}}%
\pgfpathlineto{\pgfqpoint{1.510765in}{1.949969in}}%
\pgfpathlineto{\pgfqpoint{1.527452in}{1.949486in}}%
\pgfpathlineto{\pgfqpoint{1.544138in}{1.949012in}}%
\pgfpathlineto{\pgfqpoint{1.560824in}{1.948595in}}%
\pgfpathlineto{\pgfqpoint{1.577510in}{1.948161in}}%
\pgfpathlineto{\pgfqpoint{1.594196in}{1.947691in}}%
\pgfpathlineto{\pgfqpoint{1.610882in}{1.947203in}}%
\pgfpathlineto{\pgfqpoint{1.627568in}{1.946714in}}%
\pgfpathlineto{\pgfqpoint{1.644255in}{1.946225in}}%
\pgfpathlineto{\pgfqpoint{1.660941in}{1.945737in}}%
\pgfpathlineto{\pgfqpoint{1.677627in}{1.945248in}}%
\pgfpathlineto{\pgfqpoint{1.694313in}{1.944760in}}%
\pgfpathlineto{\pgfqpoint{1.710999in}{1.944271in}}%
\pgfpathlineto{\pgfqpoint{1.727685in}{1.943783in}}%
\pgfpathlineto{\pgfqpoint{1.744371in}{1.943294in}}%
\pgfpathlineto{\pgfqpoint{1.761057in}{1.942856in}}%
\pgfpathlineto{\pgfqpoint{1.777744in}{1.942496in}}%
\pgfpathlineto{\pgfqpoint{1.794430in}{1.942136in}}%
\pgfpathlineto{\pgfqpoint{1.811116in}{1.941778in}}%
\pgfpathlineto{\pgfqpoint{1.827802in}{1.941421in}}%
\pgfpathlineto{\pgfqpoint{1.844488in}{1.941039in}}%
\pgfpathlineto{\pgfqpoint{1.861174in}{1.940587in}}%
\pgfpathlineto{\pgfqpoint{1.877860in}{1.940135in}}%
\pgfpathlineto{\pgfqpoint{1.894546in}{1.939682in}}%
\pgfpathlineto{\pgfqpoint{1.911233in}{1.939298in}}%
\pgfpathlineto{\pgfqpoint{1.927919in}{1.938941in}}%
\pgfpathlineto{\pgfqpoint{1.944605in}{1.938584in}}%
\pgfpathlineto{\pgfqpoint{1.961291in}{1.938204in}}%
\pgfpathlineto{\pgfqpoint{1.977977in}{1.937771in}}%
\pgfpathlineto{\pgfqpoint{1.994663in}{1.937337in}}%
\pgfpathlineto{\pgfqpoint{2.011349in}{1.936904in}}%
\pgfpathlineto{\pgfqpoint{2.028035in}{1.936471in}}%
\pgfpathlineto{\pgfqpoint{2.044722in}{1.936026in}}%
\pgfpathlineto{\pgfqpoint{2.061408in}{1.935518in}}%
\pgfpathlineto{\pgfqpoint{2.078094in}{1.935011in}}%
\pgfpathlineto{\pgfqpoint{2.094780in}{1.934504in}}%
\pgfpathlineto{\pgfqpoint{2.111466in}{1.933997in}}%
\pgfpathlineto{\pgfqpoint{2.128152in}{1.933490in}}%
\pgfpathlineto{\pgfqpoint{2.144838in}{1.932982in}}%
\pgfpathlineto{\pgfqpoint{2.161525in}{1.932475in}}%
\pgfpathlineto{\pgfqpoint{2.178211in}{1.931968in}}%
\pgfpathlineto{\pgfqpoint{2.178211in}{1.995724in}}%
\pgfpathlineto{\pgfqpoint{2.178211in}{1.995724in}}%
\pgfpathlineto{\pgfqpoint{2.161525in}{1.995350in}}%
\pgfpathlineto{\pgfqpoint{2.144838in}{1.994971in}}%
\pgfpathlineto{\pgfqpoint{2.128152in}{1.994591in}}%
\pgfpathlineto{\pgfqpoint{2.111466in}{1.994236in}}%
\pgfpathlineto{\pgfqpoint{2.094780in}{1.993946in}}%
\pgfpathlineto{\pgfqpoint{2.078094in}{1.993657in}}%
\pgfpathlineto{\pgfqpoint{2.061408in}{1.993367in}}%
\pgfpathlineto{\pgfqpoint{2.044722in}{1.993078in}}%
\pgfpathlineto{\pgfqpoint{2.028035in}{1.992785in}}%
\pgfpathlineto{\pgfqpoint{2.011349in}{1.992392in}}%
\pgfpathlineto{\pgfqpoint{1.994663in}{1.991984in}}%
\pgfpathlineto{\pgfqpoint{1.977977in}{1.991575in}}%
\pgfpathlineto{\pgfqpoint{1.961291in}{1.991165in}}%
\pgfpathlineto{\pgfqpoint{1.944605in}{1.990755in}}%
\pgfpathlineto{\pgfqpoint{1.927919in}{1.990345in}}%
\pgfpathlineto{\pgfqpoint{1.911233in}{1.989934in}}%
\pgfpathlineto{\pgfqpoint{1.894546in}{1.989586in}}%
\pgfpathlineto{\pgfqpoint{1.877860in}{1.989312in}}%
\pgfpathlineto{\pgfqpoint{1.861174in}{1.989038in}}%
\pgfpathlineto{\pgfqpoint{1.844488in}{1.988763in}}%
\pgfpathlineto{\pgfqpoint{1.827802in}{1.988489in}}%
\pgfpathlineto{\pgfqpoint{1.811116in}{1.988175in}}%
\pgfpathlineto{\pgfqpoint{1.794430in}{1.987792in}}%
\pgfpathlineto{\pgfqpoint{1.777744in}{1.987409in}}%
\pgfpathlineto{\pgfqpoint{1.761057in}{1.987026in}}%
\pgfpathlineto{\pgfqpoint{1.744371in}{1.986643in}}%
\pgfpathlineto{\pgfqpoint{1.727685in}{1.986309in}}%
\pgfpathlineto{\pgfqpoint{1.710999in}{1.986018in}}%
\pgfpathlineto{\pgfqpoint{1.694313in}{1.985726in}}%
\pgfpathlineto{\pgfqpoint{1.677627in}{1.985431in}}%
\pgfpathlineto{\pgfqpoint{1.660941in}{1.985137in}}%
\pgfpathlineto{\pgfqpoint{1.644255in}{1.984784in}}%
\pgfpathlineto{\pgfqpoint{1.627568in}{1.984341in}}%
\pgfpathlineto{\pgfqpoint{1.610882in}{1.983898in}}%
\pgfpathlineto{\pgfqpoint{1.594196in}{1.983456in}}%
\pgfpathlineto{\pgfqpoint{1.577510in}{1.983100in}}%
\pgfpathlineto{\pgfqpoint{1.560824in}{1.982787in}}%
\pgfpathlineto{\pgfqpoint{1.544138in}{1.982471in}}%
\pgfpathlineto{\pgfqpoint{1.527452in}{1.982154in}}%
\pgfpathlineto{\pgfqpoint{1.510765in}{1.981837in}}%
\pgfpathlineto{\pgfqpoint{1.494079in}{1.981520in}}%
\pgfpathlineto{\pgfqpoint{1.477393in}{1.981185in}}%
\pgfpathlineto{\pgfqpoint{1.460707in}{1.980743in}}%
\pgfpathlineto{\pgfqpoint{1.444021in}{1.980431in}}%
\pgfpathlineto{\pgfqpoint{1.427335in}{1.979997in}}%
\pgfpathlineto{\pgfqpoint{1.410649in}{1.979706in}}%
\pgfpathlineto{\pgfqpoint{1.393963in}{1.979389in}}%
\pgfpathlineto{\pgfqpoint{1.377276in}{1.979035in}}%
\pgfpathlineto{\pgfqpoint{1.360590in}{1.978756in}}%
\pgfpathlineto{\pgfqpoint{1.343904in}{1.978444in}}%
\pgfpathlineto{\pgfqpoint{1.327218in}{1.978127in}}%
\pgfpathlineto{\pgfqpoint{1.310532in}{1.977830in}}%
\pgfpathlineto{\pgfqpoint{1.293846in}{1.977538in}}%
\pgfpathlineto{\pgfqpoint{1.277160in}{1.977247in}}%
\pgfpathlineto{\pgfqpoint{1.260474in}{1.976954in}}%
\pgfpathlineto{\pgfqpoint{1.243787in}{1.976661in}}%
\pgfpathlineto{\pgfqpoint{1.227101in}{1.976298in}}%
\pgfpathlineto{\pgfqpoint{1.210415in}{1.975911in}}%
\pgfpathlineto{\pgfqpoint{1.193729in}{1.975665in}}%
\pgfpathlineto{\pgfqpoint{1.177043in}{1.975425in}}%
\pgfpathlineto{\pgfqpoint{1.160357in}{1.975195in}}%
\pgfpathlineto{\pgfqpoint{1.143671in}{1.974956in}}%
\pgfpathlineto{\pgfqpoint{1.126985in}{1.974778in}}%
\pgfpathlineto{\pgfqpoint{1.110298in}{1.974637in}}%
\pgfpathlineto{\pgfqpoint{1.093612in}{1.974687in}}%
\pgfpathlineto{\pgfqpoint{1.076926in}{1.974741in}}%
\pgfpathlineto{\pgfqpoint{1.060240in}{1.974491in}}%
\pgfpathlineto{\pgfqpoint{1.043554in}{1.974400in}}%
\pgfpathlineto{\pgfqpoint{1.026868in}{1.974222in}}%
\pgfpathlineto{\pgfqpoint{1.010182in}{1.974149in}}%
\pgfpathlineto{\pgfqpoint{0.993495in}{1.974059in}}%
\pgfpathlineto{\pgfqpoint{0.976809in}{1.974184in}}%
\pgfpathlineto{\pgfqpoint{0.960123in}{1.974280in}}%
\pgfpathlineto{\pgfqpoint{0.943437in}{1.974215in}}%
\pgfpathlineto{\pgfqpoint{0.926751in}{1.974203in}}%
\pgfpathlineto{\pgfqpoint{0.910065in}{1.974443in}}%
\pgfpathlineto{\pgfqpoint{0.893379in}{1.974463in}}%
\pgfpathlineto{\pgfqpoint{0.876693in}{1.974566in}}%
\pgfpathlineto{\pgfqpoint{0.860006in}{1.974829in}}%
\pgfpathlineto{\pgfqpoint{0.843320in}{1.975222in}}%
\pgfpathlineto{\pgfqpoint{0.826634in}{1.975275in}}%
\pgfpathlineto{\pgfqpoint{0.809948in}{1.975559in}}%
\pgfpathlineto{\pgfqpoint{0.793262in}{1.975656in}}%
\pgfpathlineto{\pgfqpoint{0.776576in}{1.975929in}}%
\pgfpathlineto{\pgfqpoint{0.759890in}{1.976183in}}%
\pgfpathlineto{\pgfqpoint{0.743204in}{1.976442in}}%
\pgfpathlineto{\pgfqpoint{0.726517in}{1.976756in}}%
\pgfpathlineto{\pgfqpoint{0.709831in}{1.977083in}}%
\pgfpathlineto{\pgfqpoint{0.693145in}{1.977470in}}%
\pgfpathlineto{\pgfqpoint{0.676459in}{1.977887in}}%
\pgfpathlineto{\pgfqpoint{0.659773in}{1.978159in}}%
\pgfpathlineto{\pgfqpoint{0.643087in}{1.978450in}}%
\pgfpathlineto{\pgfqpoint{0.626401in}{1.978859in}}%
\pgfpathlineto{\pgfqpoint{0.609715in}{1.979206in}}%
\pgfpathlineto{\pgfqpoint{0.593028in}{1.979545in}}%
\pgfpathlineto{\pgfqpoint{0.576342in}{1.979956in}}%
\pgfpathlineto{\pgfqpoint{0.559656in}{1.980488in}}%
\pgfpathlineto{\pgfqpoint{0.542970in}{1.980826in}}%
\pgfpathlineto{\pgfqpoint{0.526284in}{1.981159in}}%
\pgfpathclose%
\pgfusepath{fill}%
\end{pgfscope}%
\begin{pgfscope}%
\pgfpathrectangle{\pgfqpoint{0.526284in}{0.473557in}}{\pgfqpoint{1.651927in}{1.704653in}}%
\pgfusepath{clip}%
\pgfsetbuttcap%
\pgfsetroundjoin%
\definecolor{currentfill}{rgb}{1.000000,0.498039,0.054902}%
\pgfsetfillcolor{currentfill}%
\pgfsetfillopacity{0.250000}%
\pgfsetlinewidth{1.003750pt}%
\definecolor{currentstroke}{rgb}{1.000000,0.498039,0.054902}%
\pgfsetstrokecolor{currentstroke}%
\pgfsetstrokeopacity{0.250000}%
\pgfsetdash{}{0pt}%
\pgfsys@defobject{currentmarker}{\pgfqpoint{-0.017010in}{-0.017010in}}{\pgfqpoint{0.017010in}{0.017010in}}{%
\pgfpathmoveto{\pgfqpoint{0.000000in}{-0.017010in}}%
\pgfpathcurveto{\pgfqpoint{0.004511in}{-0.017010in}}{\pgfqpoint{0.008838in}{-0.015218in}}{\pgfqpoint{0.012028in}{-0.012028in}}%
\pgfpathcurveto{\pgfqpoint{0.015218in}{-0.008838in}}{\pgfqpoint{0.017010in}{-0.004511in}}{\pgfqpoint{0.017010in}{0.000000in}}%
\pgfpathcurveto{\pgfqpoint{0.017010in}{0.004511in}}{\pgfqpoint{0.015218in}{0.008838in}}{\pgfqpoint{0.012028in}{0.012028in}}%
\pgfpathcurveto{\pgfqpoint{0.008838in}{0.015218in}}{\pgfqpoint{0.004511in}{0.017010in}}{\pgfqpoint{0.000000in}{0.017010in}}%
\pgfpathcurveto{\pgfqpoint{-0.004511in}{0.017010in}}{\pgfqpoint{-0.008838in}{0.015218in}}{\pgfqpoint{-0.012028in}{0.012028in}}%
\pgfpathcurveto{\pgfqpoint{-0.015218in}{0.008838in}}{\pgfqpoint{-0.017010in}{0.004511in}}{\pgfqpoint{-0.017010in}{0.000000in}}%
\pgfpathcurveto{\pgfqpoint{-0.017010in}{-0.004511in}}{\pgfqpoint{-0.015218in}{-0.008838in}}{\pgfqpoint{-0.012028in}{-0.012028in}}%
\pgfpathcurveto{\pgfqpoint{-0.008838in}{-0.015218in}}{\pgfqpoint{-0.004511in}{-0.017010in}}{\pgfqpoint{0.000000in}{-0.017010in}}%
\pgfpathclose%
\pgfusepath{stroke,fill}%
}%
\begin{pgfscope}%
\pgfsys@transformshift{1.139286in}{1.855194in}%
\pgfsys@useobject{currentmarker}{}%
\end{pgfscope}%
\begin{pgfscope}%
\pgfsys@transformshift{0.826629in}{1.872123in}%
\pgfsys@useobject{currentmarker}{}%
\end{pgfscope}%
\begin{pgfscope}%
\pgfsys@transformshift{0.969498in}{1.825698in}%
\pgfsys@useobject{currentmarker}{}%
\end{pgfscope}%
\begin{pgfscope}%
\pgfsys@transformshift{0.732099in}{1.775464in}%
\pgfsys@useobject{currentmarker}{}%
\end{pgfscope}%
\begin{pgfscope}%
\pgfsys@transformshift{0.815558in}{1.673537in}%
\pgfsys@useobject{currentmarker}{}%
\end{pgfscope}%
\begin{pgfscope}%
\pgfsys@transformshift{1.268362in}{1.851595in}%
\pgfsys@useobject{currentmarker}{}%
\end{pgfscope}%
\begin{pgfscope}%
\pgfsys@transformshift{0.640142in}{1.599493in}%
\pgfsys@useobject{currentmarker}{}%
\end{pgfscope}%
\begin{pgfscope}%
\pgfsys@transformshift{1.031370in}{1.885418in}%
\pgfsys@useobject{currentmarker}{}%
\end{pgfscope}%
\begin{pgfscope}%
\pgfsys@transformshift{1.501355in}{1.852936in}%
\pgfsys@useobject{currentmarker}{}%
\end{pgfscope}%
\begin{pgfscope}%
\pgfsys@transformshift{1.307778in}{1.778666in}%
\pgfsys@useobject{currentmarker}{}%
\end{pgfscope}%
\begin{pgfscope}%
\pgfsys@transformshift{1.194068in}{1.788957in}%
\pgfsys@useobject{currentmarker}{}%
\end{pgfscope}%
\begin{pgfscope}%
\pgfsys@transformshift{0.667950in}{1.798738in}%
\pgfsys@useobject{currentmarker}{}%
\end{pgfscope}%
\begin{pgfscope}%
\pgfsys@transformshift{0.827370in}{1.646264in}%
\pgfsys@useobject{currentmarker}{}%
\end{pgfscope}%
\begin{pgfscope}%
\pgfsys@transformshift{1.071193in}{1.798673in}%
\pgfsys@useobject{currentmarker}{}%
\end{pgfscope}%
\begin{pgfscope}%
\pgfsys@transformshift{0.921937in}{1.819346in}%
\pgfsys@useobject{currentmarker}{}%
\end{pgfscope}%
\begin{pgfscope}%
\pgfsys@transformshift{1.556989in}{1.769465in}%
\pgfsys@useobject{currentmarker}{}%
\end{pgfscope}%
\begin{pgfscope}%
\pgfsys@transformshift{0.645159in}{1.814494in}%
\pgfsys@useobject{currentmarker}{}%
\end{pgfscope}%
\begin{pgfscope}%
\pgfsys@transformshift{1.282858in}{1.870480in}%
\pgfsys@useobject{currentmarker}{}%
\end{pgfscope}%
\begin{pgfscope}%
\pgfsys@transformshift{0.923659in}{1.824011in}%
\pgfsys@useobject{currentmarker}{}%
\end{pgfscope}%
\begin{pgfscope}%
\pgfsys@transformshift{1.196863in}{1.802162in}%
\pgfsys@useobject{currentmarker}{}%
\end{pgfscope}%
\begin{pgfscope}%
\pgfsys@transformshift{0.734895in}{1.755258in}%
\pgfsys@useobject{currentmarker}{}%
\end{pgfscope}%
\begin{pgfscope}%
\pgfsys@transformshift{1.049754in}{1.756214in}%
\pgfsys@useobject{currentmarker}{}%
\end{pgfscope}%
\begin{pgfscope}%
\pgfsys@transformshift{0.612705in}{1.801506in}%
\pgfsys@useobject{currentmarker}{}%
\end{pgfscope}%
\begin{pgfscope}%
\pgfsys@transformshift{1.196567in}{1.915187in}%
\pgfsys@useobject{currentmarker}{}%
\end{pgfscope}%
\begin{pgfscope}%
\pgfsys@transformshift{0.616815in}{1.717709in}%
\pgfsys@useobject{currentmarker}{}%
\end{pgfscope}%
\begin{pgfscope}%
\pgfsys@transformshift{0.881170in}{1.544866in}%
\pgfsys@useobject{currentmarker}{}%
\end{pgfscope}%
\begin{pgfscope}%
\pgfsys@transformshift{0.644863in}{1.704861in}%
\pgfsys@useobject{currentmarker}{}%
\end{pgfscope}%
\begin{pgfscope}%
\pgfsys@transformshift{1.119421in}{1.738578in}%
\pgfsys@useobject{currentmarker}{}%
\end{pgfscope}%
\begin{pgfscope}%
\pgfsys@transformshift{0.796952in}{1.794097in}%
\pgfsys@useobject{currentmarker}{}%
\end{pgfscope}%
\begin{pgfscope}%
\pgfsys@transformshift{0.928991in}{1.782448in}%
\pgfsys@useobject{currentmarker}{}%
\end{pgfscope}%
\begin{pgfscope}%
\pgfsys@transformshift{1.146006in}{1.850129in}%
\pgfsys@useobject{currentmarker}{}%
\end{pgfscope}%
\begin{pgfscope}%
\pgfsys@transformshift{1.088355in}{1.829200in}%
\pgfsys@useobject{currentmarker}{}%
\end{pgfscope}%
\begin{pgfscope}%
\pgfsys@transformshift{1.172758in}{1.657997in}%
\pgfsys@useobject{currentmarker}{}%
\end{pgfscope}%
\begin{pgfscope}%
\pgfsys@transformshift{1.211137in}{1.830813in}%
\pgfsys@useobject{currentmarker}{}%
\end{pgfscope}%
\begin{pgfscope}%
\pgfsys@transformshift{0.829388in}{1.784986in}%
\pgfsys@useobject{currentmarker}{}%
\end{pgfscope}%
\begin{pgfscope}%
\pgfsys@transformshift{1.584944in}{1.647546in}%
\pgfsys@useobject{currentmarker}{}%
\end{pgfscope}%
\begin{pgfscope}%
\pgfsys@transformshift{0.693739in}{1.871409in}%
\pgfsys@useobject{currentmarker}{}%
\end{pgfscope}%
\begin{pgfscope}%
\pgfsys@transformshift{1.271769in}{1.905651in}%
\pgfsys@useobject{currentmarker}{}%
\end{pgfscope}%
\begin{pgfscope}%
\pgfsys@transformshift{1.064010in}{1.822140in}%
\pgfsys@useobject{currentmarker}{}%
\end{pgfscope}%
\begin{pgfscope}%
\pgfsys@transformshift{1.461162in}{1.634356in}%
\pgfsys@useobject{currentmarker}{}%
\end{pgfscope}%
\begin{pgfscope}%
\pgfsys@transformshift{1.254162in}{1.819026in}%
\pgfsys@useobject{currentmarker}{}%
\end{pgfscope}%
\begin{pgfscope}%
\pgfsys@transformshift{0.641290in}{1.857264in}%
\pgfsys@useobject{currentmarker}{}%
\end{pgfscope}%
\begin{pgfscope}%
\pgfsys@transformshift{1.047681in}{1.749359in}%
\pgfsys@useobject{currentmarker}{}%
\end{pgfscope}%
\begin{pgfscope}%
\pgfsys@transformshift{0.651435in}{1.830370in}%
\pgfsys@useobject{currentmarker}{}%
\end{pgfscope}%
\begin{pgfscope}%
\pgfsys@transformshift{1.047699in}{1.743526in}%
\pgfsys@useobject{currentmarker}{}%
\end{pgfscope}%
\begin{pgfscope}%
\pgfsys@transformshift{0.706939in}{1.644369in}%
\pgfsys@useobject{currentmarker}{}%
\end{pgfscope}%
\begin{pgfscope}%
\pgfsys@transformshift{0.621869in}{1.904024in}%
\pgfsys@useobject{currentmarker}{}%
\end{pgfscope}%
\begin{pgfscope}%
\pgfsys@transformshift{1.117533in}{1.809950in}%
\pgfsys@useobject{currentmarker}{}%
\end{pgfscope}%
\begin{pgfscope}%
\pgfsys@transformshift{1.307648in}{1.638896in}%
\pgfsys@useobject{currentmarker}{}%
\end{pgfscope}%
\begin{pgfscope}%
\pgfsys@transformshift{0.948856in}{1.912397in}%
\pgfsys@useobject{currentmarker}{}%
\end{pgfscope}%
\begin{pgfscope}%
\pgfsys@transformshift{1.014153in}{1.791539in}%
\pgfsys@useobject{currentmarker}{}%
\end{pgfscope}%
\begin{pgfscope}%
\pgfsys@transformshift{0.631737in}{1.887561in}%
\pgfsys@useobject{currentmarker}{}%
\end{pgfscope}%
\begin{pgfscope}%
\pgfsys@transformshift{0.643493in}{1.898487in}%
\pgfsys@useobject{currentmarker}{}%
\end{pgfscope}%
\begin{pgfscope}%
\pgfsys@transformshift{0.772755in}{1.773527in}%
\pgfsys@useobject{currentmarker}{}%
\end{pgfscope}%
\begin{pgfscope}%
\pgfsys@transformshift{0.692091in}{1.707892in}%
\pgfsys@useobject{currentmarker}{}%
\end{pgfscope}%
\begin{pgfscope}%
\pgfsys@transformshift{0.595358in}{1.887891in}%
\pgfsys@useobject{currentmarker}{}%
\end{pgfscope}%
\begin{pgfscope}%
\pgfsys@transformshift{0.903997in}{1.723288in}%
\pgfsys@useobject{currentmarker}{}%
\end{pgfscope}%
\begin{pgfscope}%
\pgfsys@transformshift{1.258735in}{1.897058in}%
\pgfsys@useobject{currentmarker}{}%
\end{pgfscope}%
\begin{pgfscope}%
\pgfsys@transformshift{0.771996in}{1.731792in}%
\pgfsys@useobject{currentmarker}{}%
\end{pgfscope}%
\begin{pgfscope}%
\pgfsys@transformshift{0.665932in}{1.732190in}%
\pgfsys@useobject{currentmarker}{}%
\end{pgfscope}%
\begin{pgfscope}%
\pgfsys@transformshift{0.723120in}{1.727563in}%
\pgfsys@useobject{currentmarker}{}%
\end{pgfscope}%
\begin{pgfscope}%
\pgfsys@transformshift{1.030500in}{1.879071in}%
\pgfsys@useobject{currentmarker}{}%
\end{pgfscope}%
\begin{pgfscope}%
\pgfsys@transformshift{0.538429in}{1.800718in}%
\pgfsys@useobject{currentmarker}{}%
\end{pgfscope}%
\begin{pgfscope}%
\pgfsys@transformshift{1.081468in}{1.721218in}%
\pgfsys@useobject{currentmarker}{}%
\end{pgfscope}%
\begin{pgfscope}%
\pgfsys@transformshift{0.629293in}{1.770118in}%
\pgfsys@useobject{currentmarker}{}%
\end{pgfscope}%
\begin{pgfscope}%
\pgfsys@transformshift{0.792175in}{1.707311in}%
\pgfsys@useobject{currentmarker}{}%
\end{pgfscope}%
\begin{pgfscope}%
\pgfsys@transformshift{1.017763in}{1.754209in}%
\pgfsys@useobject{currentmarker}{}%
\end{pgfscope}%
\begin{pgfscope}%
\pgfsys@transformshift{0.953410in}{1.772146in}%
\pgfsys@useobject{currentmarker}{}%
\end{pgfscope}%
\begin{pgfscope}%
\pgfsys@transformshift{1.026687in}{1.800059in}%
\pgfsys@useobject{currentmarker}{}%
\end{pgfscope}%
\begin{pgfscope}%
\pgfsys@transformshift{1.527422in}{1.567703in}%
\pgfsys@useobject{currentmarker}{}%
\end{pgfscope}%
\begin{pgfscope}%
\pgfsys@transformshift{1.160817in}{1.815748in}%
\pgfsys@useobject{currentmarker}{}%
\end{pgfscope}%
\begin{pgfscope}%
\pgfsys@transformshift{0.698349in}{1.879184in}%
\pgfsys@useobject{currentmarker}{}%
\end{pgfscope}%
\begin{pgfscope}%
\pgfsys@transformshift{1.035203in}{1.774086in}%
\pgfsys@useobject{currentmarker}{}%
\end{pgfscope}%
\begin{pgfscope}%
\pgfsys@transformshift{0.922307in}{1.799077in}%
\pgfsys@useobject{currentmarker}{}%
\end{pgfscope}%
\begin{pgfscope}%
\pgfsys@transformshift{1.166982in}{1.834206in}%
\pgfsys@useobject{currentmarker}{}%
\end{pgfscope}%
\begin{pgfscope}%
\pgfsys@transformshift{1.119569in}{1.806368in}%
\pgfsys@useobject{currentmarker}{}%
\end{pgfscope}%
\begin{pgfscope}%
\pgfsys@transformshift{0.825518in}{1.700734in}%
\pgfsys@useobject{currentmarker}{}%
\end{pgfscope}%
\begin{pgfscope}%
\pgfsys@transformshift{1.426209in}{1.592054in}%
\pgfsys@useobject{currentmarker}{}%
\end{pgfscope}%
\begin{pgfscope}%
\pgfsys@transformshift{1.406677in}{1.843619in}%
\pgfsys@useobject{currentmarker}{}%
\end{pgfscope}%
\begin{pgfscope}%
\pgfsys@transformshift{1.415453in}{1.843294in}%
\pgfsys@useobject{currentmarker}{}%
\end{pgfscope}%
\begin{pgfscope}%
\pgfsys@transformshift{1.076914in}{1.841802in}%
\pgfsys@useobject{currentmarker}{}%
\end{pgfscope}%
\begin{pgfscope}%
\pgfsys@transformshift{0.779031in}{1.666365in}%
\pgfsys@useobject{currentmarker}{}%
\end{pgfscope}%
\begin{pgfscope}%
\pgfsys@transformshift{0.629497in}{1.790395in}%
\pgfsys@useobject{currentmarker}{}%
\end{pgfscope}%
\begin{pgfscope}%
\pgfsys@transformshift{1.519184in}{1.753608in}%
\pgfsys@useobject{currentmarker}{}%
\end{pgfscope}%
\begin{pgfscope}%
\pgfsys@transformshift{0.769237in}{1.757466in}%
\pgfsys@useobject{currentmarker}{}%
\end{pgfscope}%
\begin{pgfscope}%
\pgfsys@transformshift{1.087429in}{1.808259in}%
\pgfsys@useobject{currentmarker}{}%
\end{pgfscope}%
\begin{pgfscope}%
\pgfsys@transformshift{0.750705in}{1.904546in}%
\pgfsys@useobject{currentmarker}{}%
\end{pgfscope}%
\begin{pgfscope}%
\pgfsys@transformshift{1.495746in}{1.669646in}%
\pgfsys@useobject{currentmarker}{}%
\end{pgfscope}%
\begin{pgfscope}%
\pgfsys@transformshift{1.182959in}{1.750399in}%
\pgfsys@useobject{currentmarker}{}%
\end{pgfscope}%
\begin{pgfscope}%
\pgfsys@transformshift{0.713789in}{1.803456in}%
\pgfsys@useobject{currentmarker}{}%
\end{pgfscope}%
\begin{pgfscope}%
\pgfsys@transformshift{1.355894in}{1.774109in}%
\pgfsys@useobject{currentmarker}{}%
\end{pgfscope}%
\begin{pgfscope}%
\pgfsys@transformshift{1.635208in}{1.786845in}%
\pgfsys@useobject{currentmarker}{}%
\end{pgfscope}%
\begin{pgfscope}%
\pgfsys@transformshift{1.253848in}{1.735090in}%
\pgfsys@useobject{currentmarker}{}%
\end{pgfscope}%
\begin{pgfscope}%
\pgfsys@transformshift{0.935989in}{1.935861in}%
\pgfsys@useobject{currentmarker}{}%
\end{pgfscope}%
\begin{pgfscope}%
\pgfsys@transformshift{0.875320in}{1.858039in}%
\pgfsys@useobject{currentmarker}{}%
\end{pgfscope}%
\begin{pgfscope}%
\pgfsys@transformshift{1.109831in}{1.912997in}%
\pgfsys@useobject{currentmarker}{}%
\end{pgfscope}%
\begin{pgfscope}%
\pgfsys@transformshift{1.118292in}{1.735075in}%
\pgfsys@useobject{currentmarker}{}%
\end{pgfscope}%
\begin{pgfscope}%
\pgfsys@transformshift{1.130029in}{1.881621in}%
\pgfsys@useobject{currentmarker}{}%
\end{pgfscope}%
\begin{pgfscope}%
\pgfsys@transformshift{1.140971in}{1.874638in}%
\pgfsys@useobject{currentmarker}{}%
\end{pgfscope}%
\begin{pgfscope}%
\pgfsys@transformshift{1.052346in}{1.924267in}%
\pgfsys@useobject{currentmarker}{}%
\end{pgfscope}%
\begin{pgfscope}%
\pgfsys@transformshift{0.621629in}{1.811388in}%
\pgfsys@useobject{currentmarker}{}%
\end{pgfscope}%
\begin{pgfscope}%
\pgfsys@transformshift{0.917753in}{1.867120in}%
\pgfsys@useobject{currentmarker}{}%
\end{pgfscope}%
\begin{pgfscope}%
\pgfsys@transformshift{0.863027in}{1.792899in}%
\pgfsys@useobject{currentmarker}{}%
\end{pgfscope}%
\begin{pgfscope}%
\pgfsys@transformshift{0.672485in}{1.684021in}%
\pgfsys@useobject{currentmarker}{}%
\end{pgfscope}%
\begin{pgfscope}%
\pgfsys@transformshift{1.747437in}{1.588062in}%
\pgfsys@useobject{currentmarker}{}%
\end{pgfscope}%
\begin{pgfscope}%
\pgfsys@transformshift{0.526284in}{1.829925in}%
\pgfsys@useobject{currentmarker}{}%
\end{pgfscope}%
\begin{pgfscope}%
\pgfsys@transformshift{0.627405in}{1.753679in}%
\pgfsys@useobject{currentmarker}{}%
\end{pgfscope}%
\begin{pgfscope}%
\pgfsys@transformshift{1.270658in}{1.620601in}%
\pgfsys@useobject{currentmarker}{}%
\end{pgfscope}%
\begin{pgfscope}%
\pgfsys@transformshift{1.594053in}{1.605446in}%
\pgfsys@useobject{currentmarker}{}%
\end{pgfscope}%
\begin{pgfscope}%
\pgfsys@transformshift{0.652472in}{1.712917in}%
\pgfsys@useobject{currentmarker}{}%
\end{pgfscope}%
\begin{pgfscope}%
\pgfsys@transformshift{0.819168in}{1.793774in}%
\pgfsys@useobject{currentmarker}{}%
\end{pgfscope}%
\begin{pgfscope}%
\pgfsys@transformshift{0.847883in}{1.640007in}%
\pgfsys@useobject{currentmarker}{}%
\end{pgfscope}%
\begin{pgfscope}%
\pgfsys@transformshift{1.118495in}{1.597179in}%
\pgfsys@useobject{currentmarker}{}%
\end{pgfscope}%
\begin{pgfscope}%
\pgfsys@transformshift{0.618167in}{1.764939in}%
\pgfsys@useobject{currentmarker}{}%
\end{pgfscope}%
\begin{pgfscope}%
\pgfsys@transformshift{0.857547in}{1.845355in}%
\pgfsys@useobject{currentmarker}{}%
\end{pgfscope}%
\begin{pgfscope}%
\pgfsys@transformshift{1.090392in}{1.907999in}%
\pgfsys@useobject{currentmarker}{}%
\end{pgfscope}%
\begin{pgfscope}%
\pgfsys@transformshift{1.063566in}{1.898625in}%
\pgfsys@useobject{currentmarker}{}%
\end{pgfscope}%
\begin{pgfscope}%
\pgfsys@transformshift{1.510205in}{1.637905in}%
\pgfsys@useobject{currentmarker}{}%
\end{pgfscope}%
\begin{pgfscope}%
\pgfsys@transformshift{1.287209in}{1.859796in}%
\pgfsys@useobject{currentmarker}{}%
\end{pgfscope}%
\begin{pgfscope}%
\pgfsys@transformshift{0.778753in}{1.663953in}%
\pgfsys@useobject{currentmarker}{}%
\end{pgfscope}%
\begin{pgfscope}%
\pgfsys@transformshift{0.627757in}{1.873937in}%
\pgfsys@useobject{currentmarker}{}%
\end{pgfscope}%
\begin{pgfscope}%
\pgfsys@transformshift{1.071860in}{1.839355in}%
\pgfsys@useobject{currentmarker}{}%
\end{pgfscope}%
\begin{pgfscope}%
\pgfsys@transformshift{1.053938in}{1.803459in}%
\pgfsys@useobject{currentmarker}{}%
\end{pgfscope}%
\begin{pgfscope}%
\pgfsys@transformshift{0.910458in}{1.764236in}%
\pgfsys@useobject{currentmarker}{}%
\end{pgfscope}%
\begin{pgfscope}%
\pgfsys@transformshift{0.621962in}{1.837290in}%
\pgfsys@useobject{currentmarker}{}%
\end{pgfscope}%
\begin{pgfscope}%
\pgfsys@transformshift{0.836867in}{1.841786in}%
\pgfsys@useobject{currentmarker}{}%
\end{pgfscope}%
\begin{pgfscope}%
\pgfsys@transformshift{0.800229in}{1.885649in}%
\pgfsys@useobject{currentmarker}{}%
\end{pgfscope}%
\begin{pgfscope}%
\pgfsys@transformshift{0.958260in}{1.827813in}%
\pgfsys@useobject{currentmarker}{}%
\end{pgfscope}%
\begin{pgfscope}%
\pgfsys@transformshift{0.656045in}{1.822641in}%
\pgfsys@useobject{currentmarker}{}%
\end{pgfscope}%
\begin{pgfscope}%
\pgfsys@transformshift{0.629960in}{1.743379in}%
\pgfsys@useobject{currentmarker}{}%
\end{pgfscope}%
\begin{pgfscope}%
\pgfsys@transformshift{1.008358in}{1.672643in}%
\pgfsys@useobject{currentmarker}{}%
\end{pgfscope}%
\begin{pgfscope}%
\pgfsys@transformshift{1.258698in}{1.814808in}%
\pgfsys@useobject{currentmarker}{}%
\end{pgfscope}%
\begin{pgfscope}%
\pgfsys@transformshift{1.100889in}{1.744143in}%
\pgfsys@useobject{currentmarker}{}%
\end{pgfscope}%
\begin{pgfscope}%
\pgfsys@transformshift{1.230410in}{1.821788in}%
\pgfsys@useobject{currentmarker}{}%
\end{pgfscope}%
\begin{pgfscope}%
\pgfsys@transformshift{0.647325in}{1.864139in}%
\pgfsys@useobject{currentmarker}{}%
\end{pgfscope}%
\begin{pgfscope}%
\pgfsys@transformshift{0.904423in}{1.838590in}%
\pgfsys@useobject{currentmarker}{}%
\end{pgfscope}%
\begin{pgfscope}%
\pgfsys@transformshift{1.617602in}{1.768027in}%
\pgfsys@useobject{currentmarker}{}%
\end{pgfscope}%
\begin{pgfscope}%
\pgfsys@transformshift{1.245406in}{1.867284in}%
\pgfsys@useobject{currentmarker}{}%
\end{pgfscope}%
\begin{pgfscope}%
\pgfsys@transformshift{1.238907in}{1.838948in}%
\pgfsys@useobject{currentmarker}{}%
\end{pgfscope}%
\begin{pgfscope}%
\pgfsys@transformshift{1.187366in}{1.827873in}%
\pgfsys@useobject{currentmarker}{}%
\end{pgfscope}%
\begin{pgfscope}%
\pgfsys@transformshift{1.095724in}{1.900478in}%
\pgfsys@useobject{currentmarker}{}%
\end{pgfscope}%
\begin{pgfscope}%
\pgfsys@transformshift{1.131084in}{1.881616in}%
\pgfsys@useobject{currentmarker}{}%
\end{pgfscope}%
\begin{pgfscope}%
\pgfsys@transformshift{0.643549in}{1.774855in}%
\pgfsys@useobject{currentmarker}{}%
\end{pgfscope}%
\begin{pgfscope}%
\pgfsys@transformshift{0.726878in}{1.809529in}%
\pgfsys@useobject{currentmarker}{}%
\end{pgfscope}%
\begin{pgfscope}%
\pgfsys@transformshift{0.940728in}{1.784634in}%
\pgfsys@useobject{currentmarker}{}%
\end{pgfscope}%
\begin{pgfscope}%
\pgfsys@transformshift{0.686537in}{1.780097in}%
\pgfsys@useobject{currentmarker}{}%
\end{pgfscope}%
\begin{pgfscope}%
\pgfsys@transformshift{1.304686in}{1.842874in}%
\pgfsys@useobject{currentmarker}{}%
\end{pgfscope}%
\begin{pgfscope}%
\pgfsys@transformshift{0.822223in}{1.740313in}%
\pgfsys@useobject{currentmarker}{}%
\end{pgfscope}%
\begin{pgfscope}%
\pgfsys@transformshift{0.753593in}{1.748128in}%
\pgfsys@useobject{currentmarker}{}%
\end{pgfscope}%
\begin{pgfscope}%
\pgfsys@transformshift{1.155670in}{1.808075in}%
\pgfsys@useobject{currentmarker}{}%
\end{pgfscope}%
\begin{pgfscope}%
\pgfsys@transformshift{0.665617in}{1.695685in}%
\pgfsys@useobject{currentmarker}{}%
\end{pgfscope}%
\begin{pgfscope}%
\pgfsys@transformshift{0.722713in}{1.689686in}%
\pgfsys@useobject{currentmarker}{}%
\end{pgfscope}%
\begin{pgfscope}%
\pgfsys@transformshift{0.537651in}{1.703636in}%
\pgfsys@useobject{currentmarker}{}%
\end{pgfscope}%
\begin{pgfscope}%
\pgfsys@transformshift{0.956631in}{1.828168in}%
\pgfsys@useobject{currentmarker}{}%
\end{pgfscope}%
\begin{pgfscope}%
\pgfsys@transformshift{1.046922in}{1.843678in}%
\pgfsys@useobject{currentmarker}{}%
\end{pgfscope}%
\begin{pgfscope}%
\pgfsys@transformshift{0.812559in}{1.862806in}%
\pgfsys@useobject{currentmarker}{}%
\end{pgfscope}%
\begin{pgfscope}%
\pgfsys@transformshift{1.535254in}{1.768479in}%
\pgfsys@useobject{currentmarker}{}%
\end{pgfscope}%
\begin{pgfscope}%
\pgfsys@transformshift{0.778087in}{1.793979in}%
\pgfsys@useobject{currentmarker}{}%
\end{pgfscope}%
\begin{pgfscope}%
\pgfsys@transformshift{0.723860in}{1.903183in}%
\pgfsys@useobject{currentmarker}{}%
\end{pgfscope}%
\begin{pgfscope}%
\pgfsys@transformshift{0.970961in}{1.663435in}%
\pgfsys@useobject{currentmarker}{}%
\end{pgfscope}%
\begin{pgfscope}%
\pgfsys@transformshift{1.099149in}{1.929258in}%
\pgfsys@useobject{currentmarker}{}%
\end{pgfscope}%
\begin{pgfscope}%
\pgfsys@transformshift{1.042182in}{1.834815in}%
\pgfsys@useobject{currentmarker}{}%
\end{pgfscope}%
\begin{pgfscope}%
\pgfsys@transformshift{1.047181in}{1.864465in}%
\pgfsys@useobject{currentmarker}{}%
\end{pgfscope}%
\begin{pgfscope}%
\pgfsys@transformshift{1.569152in}{1.830162in}%
\pgfsys@useobject{currentmarker}{}%
\end{pgfscope}%
\begin{pgfscope}%
\pgfsys@transformshift{1.518018in}{1.700081in}%
\pgfsys@useobject{currentmarker}{}%
\end{pgfscope}%
\begin{pgfscope}%
\pgfsys@transformshift{0.733265in}{1.801026in}%
\pgfsys@useobject{currentmarker}{}%
\end{pgfscope}%
\begin{pgfscope}%
\pgfsys@transformshift{0.558701in}{1.765170in}%
\pgfsys@useobject{currentmarker}{}%
\end{pgfscope}%
\begin{pgfscope}%
\pgfsys@transformshift{0.600190in}{1.903485in}%
\pgfsys@useobject{currentmarker}{}%
\end{pgfscope}%
\begin{pgfscope}%
\pgfsys@transformshift{1.222060in}{1.815395in}%
\pgfsys@useobject{currentmarker}{}%
\end{pgfscope}%
\begin{pgfscope}%
\pgfsys@transformshift{0.884484in}{1.790735in}%
\pgfsys@useobject{currentmarker}{}%
\end{pgfscope}%
\begin{pgfscope}%
\pgfsys@transformshift{1.004359in}{1.790686in}%
\pgfsys@useobject{currentmarker}{}%
\end{pgfscope}%
\begin{pgfscope}%
\pgfsys@transformshift{0.853585in}{1.689840in}%
\pgfsys@useobject{currentmarker}{}%
\end{pgfscope}%
\begin{pgfscope}%
\pgfsys@transformshift{0.668357in}{1.896243in}%
\pgfsys@useobject{currentmarker}{}%
\end{pgfscope}%
\begin{pgfscope}%
\pgfsys@transformshift{0.536244in}{1.790351in}%
\pgfsys@useobject{currentmarker}{}%
\end{pgfscope}%
\begin{pgfscope}%
\pgfsys@transformshift{0.785103in}{1.823140in}%
\pgfsys@useobject{currentmarker}{}%
\end{pgfscope}%
\begin{pgfscope}%
\pgfsys@transformshift{1.369965in}{1.762467in}%
\pgfsys@useobject{currentmarker}{}%
\end{pgfscope}%
\begin{pgfscope}%
\pgfsys@transformshift{0.794379in}{1.673745in}%
\pgfsys@useobject{currentmarker}{}%
\end{pgfscope}%
\begin{pgfscope}%
\pgfsys@transformshift{0.633237in}{1.821142in}%
\pgfsys@useobject{currentmarker}{}%
\end{pgfscope}%
\begin{pgfscope}%
\pgfsys@transformshift{0.859287in}{1.722084in}%
\pgfsys@useobject{currentmarker}{}%
\end{pgfscope}%
\begin{pgfscope}%
\pgfsys@transformshift{1.032685in}{1.728594in}%
\pgfsys@useobject{currentmarker}{}%
\end{pgfscope}%
\begin{pgfscope}%
\pgfsys@transformshift{0.784141in}{1.779456in}%
\pgfsys@useobject{currentmarker}{}%
\end{pgfscope}%
\begin{pgfscope}%
\pgfsys@transformshift{0.996010in}{1.741184in}%
\pgfsys@useobject{currentmarker}{}%
\end{pgfscope}%
\begin{pgfscope}%
\pgfsys@transformshift{1.002563in}{1.628603in}%
\pgfsys@useobject{currentmarker}{}%
\end{pgfscope}%
\begin{pgfscope}%
\pgfsys@transformshift{0.930287in}{1.900571in}%
\pgfsys@useobject{currentmarker}{}%
\end{pgfscope}%
\begin{pgfscope}%
\pgfsys@transformshift{0.659693in}{1.782993in}%
\pgfsys@useobject{currentmarker}{}%
\end{pgfscope}%
\begin{pgfscope}%
\pgfsys@transformshift{1.068472in}{1.884667in}%
\pgfsys@useobject{currentmarker}{}%
\end{pgfscope}%
\begin{pgfscope}%
\pgfsys@transformshift{0.856640in}{1.868819in}%
\pgfsys@useobject{currentmarker}{}%
\end{pgfscope}%
\begin{pgfscope}%
\pgfsys@transformshift{1.405029in}{1.769510in}%
\pgfsys@useobject{currentmarker}{}%
\end{pgfscope}%
\begin{pgfscope}%
\pgfsys@transformshift{1.220135in}{1.793033in}%
\pgfsys@useobject{currentmarker}{}%
\end{pgfscope}%
\begin{pgfscope}%
\pgfsys@transformshift{0.999324in}{1.765998in}%
\pgfsys@useobject{currentmarker}{}%
\end{pgfscope}%
\begin{pgfscope}%
\pgfsys@transformshift{0.633699in}{1.793321in}%
\pgfsys@useobject{currentmarker}{}%
\end{pgfscope}%
\begin{pgfscope}%
\pgfsys@transformshift{1.135768in}{1.821054in}%
\pgfsys@useobject{currentmarker}{}%
\end{pgfscope}%
\begin{pgfscope}%
\pgfsys@transformshift{1.116959in}{1.570935in}%
\pgfsys@useobject{currentmarker}{}%
\end{pgfscope}%
\begin{pgfscope}%
\pgfsys@transformshift{0.598172in}{1.882542in}%
\pgfsys@useobject{currentmarker}{}%
\end{pgfscope}%
\begin{pgfscope}%
\pgfsys@transformshift{0.619907in}{1.813012in}%
\pgfsys@useobject{currentmarker}{}%
\end{pgfscope}%
\begin{pgfscope}%
\pgfsys@transformshift{1.023539in}{1.739543in}%
\pgfsys@useobject{currentmarker}{}%
\end{pgfscope}%
\begin{pgfscope}%
\pgfsys@transformshift{1.456182in}{1.777817in}%
\pgfsys@useobject{currentmarker}{}%
\end{pgfscope}%
\begin{pgfscope}%
\pgfsys@transformshift{0.579547in}{1.853300in}%
\pgfsys@useobject{currentmarker}{}%
\end{pgfscope}%
\begin{pgfscope}%
\pgfsys@transformshift{1.012949in}{1.761399in}%
\pgfsys@useobject{currentmarker}{}%
\end{pgfscope}%
\begin{pgfscope}%
\pgfsys@transformshift{1.111293in}{1.754193in}%
\pgfsys@useobject{currentmarker}{}%
\end{pgfscope}%
\begin{pgfscope}%
\pgfsys@transformshift{0.977218in}{1.791184in}%
\pgfsys@useobject{currentmarker}{}%
\end{pgfscope}%
\begin{pgfscope}%
\pgfsys@transformshift{1.304371in}{1.831733in}%
\pgfsys@useobject{currentmarker}{}%
\end{pgfscope}%
\begin{pgfscope}%
\pgfsys@transformshift{1.456756in}{1.843413in}%
\pgfsys@useobject{currentmarker}{}%
\end{pgfscope}%
\begin{pgfscope}%
\pgfsys@transformshift{0.684908in}{1.649945in}%
\pgfsys@useobject{currentmarker}{}%
\end{pgfscope}%
\begin{pgfscope}%
\pgfsys@transformshift{0.997176in}{1.607042in}%
\pgfsys@useobject{currentmarker}{}%
\end{pgfscope}%
\begin{pgfscope}%
\pgfsys@transformshift{1.265659in}{1.876496in}%
\pgfsys@useobject{currentmarker}{}%
\end{pgfscope}%
\begin{pgfscope}%
\pgfsys@transformshift{0.624998in}{1.784967in}%
\pgfsys@useobject{currentmarker}{}%
\end{pgfscope}%
\begin{pgfscope}%
\pgfsys@transformshift{0.771255in}{1.888653in}%
\pgfsys@useobject{currentmarker}{}%
\end{pgfscope}%
\begin{pgfscope}%
\pgfsys@transformshift{0.828869in}{1.795396in}%
\pgfsys@useobject{currentmarker}{}%
\end{pgfscope}%
\begin{pgfscope}%
\pgfsys@transformshift{0.739412in}{1.712611in}%
\pgfsys@useobject{currentmarker}{}%
\end{pgfscope}%
\begin{pgfscope}%
\pgfsys@transformshift{0.873931in}{1.633577in}%
\pgfsys@useobject{currentmarker}{}%
\end{pgfscope}%
\begin{pgfscope}%
\pgfsys@transformshift{0.630219in}{1.793951in}%
\pgfsys@useobject{currentmarker}{}%
\end{pgfscope}%
\begin{pgfscope}%
\pgfsys@transformshift{1.172814in}{1.896393in}%
\pgfsys@useobject{currentmarker}{}%
\end{pgfscope}%
\begin{pgfscope}%
\pgfsys@transformshift{0.647992in}{1.764203in}%
\pgfsys@useobject{currentmarker}{}%
\end{pgfscope}%
\begin{pgfscope}%
\pgfsys@transformshift{1.062251in}{1.608702in}%
\pgfsys@useobject{currentmarker}{}%
\end{pgfscope}%
\begin{pgfscope}%
\pgfsys@transformshift{1.096816in}{1.804287in}%
\pgfsys@useobject{currentmarker}{}%
\end{pgfscope}%
\begin{pgfscope}%
\pgfsys@transformshift{1.149450in}{1.675868in}%
\pgfsys@useobject{currentmarker}{}%
\end{pgfscope}%
\begin{pgfscope}%
\pgfsys@transformshift{0.677632in}{1.782849in}%
\pgfsys@useobject{currentmarker}{}%
\end{pgfscope}%
\begin{pgfscope}%
\pgfsys@transformshift{1.294966in}{1.820439in}%
\pgfsys@useobject{currentmarker}{}%
\end{pgfscope}%
\begin{pgfscope}%
\pgfsys@transformshift{1.468198in}{1.659095in}%
\pgfsys@useobject{currentmarker}{}%
\end{pgfscope}%
\begin{pgfscope}%
\pgfsys@transformshift{0.730525in}{1.805852in}%
\pgfsys@useobject{currentmarker}{}%
\end{pgfscope}%
\begin{pgfscope}%
\pgfsys@transformshift{0.597968in}{1.834649in}%
\pgfsys@useobject{currentmarker}{}%
\end{pgfscope}%
\begin{pgfscope}%
\pgfsys@transformshift{0.539928in}{1.804765in}%
\pgfsys@useobject{currentmarker}{}%
\end{pgfscope}%
\begin{pgfscope}%
\pgfsys@transformshift{0.782197in}{1.924610in}%
\pgfsys@useobject{currentmarker}{}%
\end{pgfscope}%
\begin{pgfscope}%
\pgfsys@transformshift{0.726193in}{1.810914in}%
\pgfsys@useobject{currentmarker}{}%
\end{pgfscope}%
\begin{pgfscope}%
\pgfsys@transformshift{0.756148in}{1.861262in}%
\pgfsys@useobject{currentmarker}{}%
\end{pgfscope}%
\begin{pgfscope}%
\pgfsys@transformshift{0.663543in}{1.839500in}%
\pgfsys@useobject{currentmarker}{}%
\end{pgfscope}%
\begin{pgfscope}%
\pgfsys@transformshift{1.069286in}{1.779624in}%
\pgfsys@useobject{currentmarker}{}%
\end{pgfscope}%
\begin{pgfscope}%
\pgfsys@transformshift{1.411009in}{1.618752in}%
\pgfsys@useobject{currentmarker}{}%
\end{pgfscope}%
\begin{pgfscope}%
\pgfsys@transformshift{1.079839in}{1.808613in}%
\pgfsys@useobject{currentmarker}{}%
\end{pgfscope}%
\begin{pgfscope}%
\pgfsys@transformshift{1.104758in}{1.653182in}%
\pgfsys@useobject{currentmarker}{}%
\end{pgfscope}%
\begin{pgfscope}%
\pgfsys@transformshift{1.128918in}{1.708336in}%
\pgfsys@useobject{currentmarker}{}%
\end{pgfscope}%
\begin{pgfscope}%
\pgfsys@transformshift{1.010728in}{1.780932in}%
\pgfsys@useobject{currentmarker}{}%
\end{pgfscope}%
\begin{pgfscope}%
\pgfsys@transformshift{0.687426in}{1.809041in}%
\pgfsys@useobject{currentmarker}{}%
\end{pgfscope}%
\begin{pgfscope}%
\pgfsys@transformshift{1.166556in}{1.825467in}%
\pgfsys@useobject{currentmarker}{}%
\end{pgfscope}%
\begin{pgfscope}%
\pgfsys@transformshift{0.539336in}{1.923599in}%
\pgfsys@useobject{currentmarker}{}%
\end{pgfscope}%
\begin{pgfscope}%
\pgfsys@transformshift{1.057604in}{1.763187in}%
\pgfsys@useobject{currentmarker}{}%
\end{pgfscope}%
\begin{pgfscope}%
\pgfsys@transformshift{0.759925in}{1.656626in}%
\pgfsys@useobject{currentmarker}{}%
\end{pgfscope}%
\begin{pgfscope}%
\pgfsys@transformshift{1.233927in}{1.596803in}%
\pgfsys@useobject{currentmarker}{}%
\end{pgfscope}%
\begin{pgfscope}%
\pgfsys@transformshift{0.557127in}{1.695707in}%
\pgfsys@useobject{currentmarker}{}%
\end{pgfscope}%
\begin{pgfscope}%
\pgfsys@transformshift{0.685815in}{1.835620in}%
\pgfsys@useobject{currentmarker}{}%
\end{pgfscope}%
\begin{pgfscope}%
\pgfsys@transformshift{0.794138in}{1.529583in}%
\pgfsys@useobject{currentmarker}{}%
\end{pgfscope}%
\begin{pgfscope}%
\pgfsys@transformshift{0.700941in}{1.741647in}%
\pgfsys@useobject{currentmarker}{}%
\end{pgfscope}%
\begin{pgfscope}%
\pgfsys@transformshift{0.901516in}{1.690204in}%
\pgfsys@useobject{currentmarker}{}%
\end{pgfscope}%
\begin{pgfscope}%
\pgfsys@transformshift{0.999546in}{1.718948in}%
\pgfsys@useobject{currentmarker}{}%
\end{pgfscope}%
\begin{pgfscope}%
\pgfsys@transformshift{1.495820in}{1.836220in}%
\pgfsys@useobject{currentmarker}{}%
\end{pgfscope}%
\begin{pgfscope}%
\pgfsys@transformshift{1.570837in}{1.541757in}%
\pgfsys@useobject{currentmarker}{}%
\end{pgfscope}%
\begin{pgfscope}%
\pgfsys@transformshift{0.930379in}{1.821806in}%
\pgfsys@useobject{currentmarker}{}%
\end{pgfscope}%
\begin{pgfscope}%
\pgfsys@transformshift{0.725045in}{1.797481in}%
\pgfsys@useobject{currentmarker}{}%
\end{pgfscope}%
\begin{pgfscope}%
\pgfsys@transformshift{0.663488in}{1.669076in}%
\pgfsys@useobject{currentmarker}{}%
\end{pgfscope}%
\begin{pgfscope}%
\pgfsys@transformshift{0.729081in}{1.827912in}%
\pgfsys@useobject{currentmarker}{}%
\end{pgfscope}%
\begin{pgfscope}%
\pgfsys@transformshift{0.600893in}{1.831538in}%
\pgfsys@useobject{currentmarker}{}%
\end{pgfscope}%
\begin{pgfscope}%
\pgfsys@transformshift{0.769145in}{1.838203in}%
\pgfsys@useobject{currentmarker}{}%
\end{pgfscope}%
\begin{pgfscope}%
\pgfsys@transformshift{0.677817in}{1.769498in}%
\pgfsys@useobject{currentmarker}{}%
\end{pgfscope}%
\begin{pgfscope}%
\pgfsys@transformshift{0.942968in}{1.830107in}%
\pgfsys@useobject{currentmarker}{}%
\end{pgfscope}%
\begin{pgfscope}%
\pgfsys@transformshift{1.032870in}{1.869718in}%
\pgfsys@useobject{currentmarker}{}%
\end{pgfscope}%
\begin{pgfscope}%
\pgfsys@transformshift{0.670412in}{1.889178in}%
\pgfsys@useobject{currentmarker}{}%
\end{pgfscope}%
\begin{pgfscope}%
\pgfsys@transformshift{0.718584in}{1.795246in}%
\pgfsys@useobject{currentmarker}{}%
\end{pgfscope}%
\begin{pgfscope}%
\pgfsys@transformshift{0.609262in}{1.872214in}%
\pgfsys@useobject{currentmarker}{}%
\end{pgfscope}%
\begin{pgfscope}%
\pgfsys@transformshift{0.724620in}{1.823494in}%
\pgfsys@useobject{currentmarker}{}%
\end{pgfscope}%
\begin{pgfscope}%
\pgfsys@transformshift{0.839274in}{1.675420in}%
\pgfsys@useobject{currentmarker}{}%
\end{pgfscope}%
\begin{pgfscope}%
\pgfsys@transformshift{1.123142in}{1.877108in}%
\pgfsys@useobject{currentmarker}{}%
\end{pgfscope}%
\begin{pgfscope}%
\pgfsys@transformshift{0.660581in}{1.723480in}%
\pgfsys@useobject{currentmarker}{}%
\end{pgfscope}%
\begin{pgfscope}%
\pgfsys@transformshift{1.346619in}{1.848465in}%
\pgfsys@useobject{currentmarker}{}%
\end{pgfscope}%
\begin{pgfscope}%
\pgfsys@transformshift{1.032759in}{1.684588in}%
\pgfsys@useobject{currentmarker}{}%
\end{pgfscope}%
\begin{pgfscope}%
\pgfsys@transformshift{1.531144in}{1.630062in}%
\pgfsys@useobject{currentmarker}{}%
\end{pgfscope}%
\begin{pgfscope}%
\pgfsys@transformshift{0.545427in}{1.915109in}%
\pgfsys@useobject{currentmarker}{}%
\end{pgfscope}%
\begin{pgfscope}%
\pgfsys@transformshift{1.530607in}{1.634596in}%
\pgfsys@useobject{currentmarker}{}%
\end{pgfscope}%
\begin{pgfscope}%
\pgfsys@transformshift{1.562080in}{1.645870in}%
\pgfsys@useobject{currentmarker}{}%
\end{pgfscope}%
\begin{pgfscope}%
\pgfsys@transformshift{1.076266in}{1.892916in}%
\pgfsys@useobject{currentmarker}{}%
\end{pgfscope}%
\begin{pgfscope}%
\pgfsys@transformshift{0.596024in}{1.919979in}%
\pgfsys@useobject{currentmarker}{}%
\end{pgfscope}%
\begin{pgfscope}%
\pgfsys@transformshift{1.283932in}{1.587818in}%
\pgfsys@useobject{currentmarker}{}%
\end{pgfscope}%
\begin{pgfscope}%
\pgfsys@transformshift{1.277638in}{1.756359in}%
\pgfsys@useobject{currentmarker}{}%
\end{pgfscope}%
\begin{pgfscope}%
\pgfsys@transformshift{0.915846in}{1.673810in}%
\pgfsys@useobject{currentmarker}{}%
\end{pgfscope}%
\begin{pgfscope}%
\pgfsys@transformshift{0.580084in}{1.802242in}%
\pgfsys@useobject{currentmarker}{}%
\end{pgfscope}%
\begin{pgfscope}%
\pgfsys@transformshift{0.665210in}{1.787633in}%
\pgfsys@useobject{currentmarker}{}%
\end{pgfscope}%
\begin{pgfscope}%
\pgfsys@transformshift{1.282470in}{1.808720in}%
\pgfsys@useobject{currentmarker}{}%
\end{pgfscope}%
\begin{pgfscope}%
\pgfsys@transformshift{0.656693in}{1.908224in}%
\pgfsys@useobject{currentmarker}{}%
\end{pgfscope}%
\begin{pgfscope}%
\pgfsys@transformshift{1.353432in}{1.774690in}%
\pgfsys@useobject{currentmarker}{}%
\end{pgfscope}%
\begin{pgfscope}%
\pgfsys@transformshift{0.684445in}{1.961169in}%
\pgfsys@useobject{currentmarker}{}%
\end{pgfscope}%
\begin{pgfscope}%
\pgfsys@transformshift{1.918484in}{1.632956in}%
\pgfsys@useobject{currentmarker}{}%
\end{pgfscope}%
\begin{pgfscope}%
\pgfsys@transformshift{1.424728in}{1.669831in}%
\pgfsys@useobject{currentmarker}{}%
\end{pgfscope}%
\begin{pgfscope}%
\pgfsys@transformshift{0.746503in}{1.809594in}%
\pgfsys@useobject{currentmarker}{}%
\end{pgfscope}%
\begin{pgfscope}%
\pgfsys@transformshift{0.827814in}{1.899655in}%
\pgfsys@useobject{currentmarker}{}%
\end{pgfscope}%
\begin{pgfscope}%
\pgfsys@transformshift{1.171537in}{1.665366in}%
\pgfsys@useobject{currentmarker}{}%
\end{pgfscope}%
\begin{pgfscope}%
\pgfsys@transformshift{1.125382in}{1.855033in}%
\pgfsys@useobject{currentmarker}{}%
\end{pgfscope}%
\begin{pgfscope}%
\pgfsys@transformshift{1.248979in}{1.783108in}%
\pgfsys@useobject{currentmarker}{}%
\end{pgfscope}%
\begin{pgfscope}%
\pgfsys@transformshift{0.729433in}{1.888250in}%
\pgfsys@useobject{currentmarker}{}%
\end{pgfscope}%
\begin{pgfscope}%
\pgfsys@transformshift{0.585879in}{1.854179in}%
\pgfsys@useobject{currentmarker}{}%
\end{pgfscope}%
\begin{pgfscope}%
\pgfsys@transformshift{1.040794in}{1.853118in}%
\pgfsys@useobject{currentmarker}{}%
\end{pgfscope}%
\begin{pgfscope}%
\pgfsys@transformshift{0.953003in}{1.871396in}%
\pgfsys@useobject{currentmarker}{}%
\end{pgfscope}%
\begin{pgfscope}%
\pgfsys@transformshift{1.076747in}{1.681381in}%
\pgfsys@useobject{currentmarker}{}%
\end{pgfscope}%
\begin{pgfscope}%
\pgfsys@transformshift{0.880318in}{1.914595in}%
\pgfsys@useobject{currentmarker}{}%
\end{pgfscope}%
\begin{pgfscope}%
\pgfsys@transformshift{1.223782in}{1.865594in}%
\pgfsys@useobject{currentmarker}{}%
\end{pgfscope}%
\begin{pgfscope}%
\pgfsys@transformshift{1.089466in}{1.877823in}%
\pgfsys@useobject{currentmarker}{}%
\end{pgfscope}%
\begin{pgfscope}%
\pgfsys@transformshift{0.826666in}{1.827440in}%
\pgfsys@useobject{currentmarker}{}%
\end{pgfscope}%
\begin{pgfscope}%
\pgfsys@transformshift{1.377851in}{1.801564in}%
\pgfsys@useobject{currentmarker}{}%
\end{pgfscope}%
\begin{pgfscope}%
\pgfsys@transformshift{0.598024in}{1.876435in}%
\pgfsys@useobject{currentmarker}{}%
\end{pgfscope}%
\begin{pgfscope}%
\pgfsys@transformshift{0.697849in}{1.694618in}%
\pgfsys@useobject{currentmarker}{}%
\end{pgfscope}%
\begin{pgfscope}%
\pgfsys@transformshift{1.266270in}{1.793770in}%
\pgfsys@useobject{currentmarker}{}%
\end{pgfscope}%
\begin{pgfscope}%
\pgfsys@transformshift{0.769293in}{1.703485in}%
\pgfsys@useobject{currentmarker}{}%
\end{pgfscope}%
\begin{pgfscope}%
\pgfsys@transformshift{0.763591in}{1.630991in}%
\pgfsys@useobject{currentmarker}{}%
\end{pgfscope}%
\begin{pgfscope}%
\pgfsys@transformshift{1.545270in}{1.643280in}%
\pgfsys@useobject{currentmarker}{}%
\end{pgfscope}%
\begin{pgfscope}%
\pgfsys@transformshift{0.781974in}{1.634657in}%
\pgfsys@useobject{currentmarker}{}%
\end{pgfscope}%
\begin{pgfscope}%
\pgfsys@transformshift{0.637143in}{1.827450in}%
\pgfsys@useobject{currentmarker}{}%
\end{pgfscope}%
\begin{pgfscope}%
\pgfsys@transformshift{1.059196in}{1.709311in}%
\pgfsys@useobject{currentmarker}{}%
\end{pgfscope}%
\begin{pgfscope}%
\pgfsys@transformshift{1.032796in}{1.872293in}%
\pgfsys@useobject{currentmarker}{}%
\end{pgfscope}%
\begin{pgfscope}%
\pgfsys@transformshift{0.683742in}{1.868126in}%
\pgfsys@useobject{currentmarker}{}%
\end{pgfscope}%
\begin{pgfscope}%
\pgfsys@transformshift{1.021114in}{1.842396in}%
\pgfsys@useobject{currentmarker}{}%
\end{pgfscope}%
\begin{pgfscope}%
\pgfsys@transformshift{0.563237in}{1.844207in}%
\pgfsys@useobject{currentmarker}{}%
\end{pgfscope}%
\begin{pgfscope}%
\pgfsys@transformshift{0.614223in}{1.840902in}%
\pgfsys@useobject{currentmarker}{}%
\end{pgfscope}%
\begin{pgfscope}%
\pgfsys@transformshift{0.779401in}{1.815626in}%
\pgfsys@useobject{currentmarker}{}%
\end{pgfscope}%
\begin{pgfscope}%
\pgfsys@transformshift{0.992603in}{1.864425in}%
\pgfsys@useobject{currentmarker}{}%
\end{pgfscope}%
\begin{pgfscope}%
\pgfsys@transformshift{1.694285in}{1.543559in}%
\pgfsys@useobject{currentmarker}{}%
\end{pgfscope}%
\begin{pgfscope}%
\pgfsys@transformshift{0.778624in}{1.665112in}%
\pgfsys@useobject{currentmarker}{}%
\end{pgfscope}%
\begin{pgfscope}%
\pgfsys@transformshift{1.340343in}{1.797838in}%
\pgfsys@useobject{currentmarker}{}%
\end{pgfscope}%
\begin{pgfscope}%
\pgfsys@transformshift{0.923566in}{1.822925in}%
\pgfsys@useobject{currentmarker}{}%
\end{pgfscope}%
\begin{pgfscope}%
\pgfsys@transformshift{0.789621in}{1.467989in}%
\pgfsys@useobject{currentmarker}{}%
\end{pgfscope}%
\begin{pgfscope}%
\pgfsys@transformshift{0.869895in}{1.549493in}%
\pgfsys@useobject{currentmarker}{}%
\end{pgfscope}%
\begin{pgfscope}%
\pgfsys@transformshift{1.119513in}{1.628763in}%
\pgfsys@useobject{currentmarker}{}%
\end{pgfscope}%
\begin{pgfscope}%
\pgfsys@transformshift{1.521480in}{1.838866in}%
\pgfsys@useobject{currentmarker}{}%
\end{pgfscope}%
\begin{pgfscope}%
\pgfsys@transformshift{1.492025in}{1.786012in}%
\pgfsys@useobject{currentmarker}{}%
\end{pgfscope}%
\begin{pgfscope}%
\pgfsys@transformshift{1.103758in}{1.965091in}%
\pgfsys@useobject{currentmarker}{}%
\end{pgfscope}%
\begin{pgfscope}%
\pgfsys@transformshift{0.698127in}{1.795410in}%
\pgfsys@useobject{currentmarker}{}%
\end{pgfscope}%
\begin{pgfscope}%
\pgfsys@transformshift{1.029871in}{1.807465in}%
\pgfsys@useobject{currentmarker}{}%
\end{pgfscope}%
\begin{pgfscope}%
\pgfsys@transformshift{0.931582in}{1.887874in}%
\pgfsys@useobject{currentmarker}{}%
\end{pgfscope}%
\begin{pgfscope}%
\pgfsys@transformshift{0.910792in}{1.728962in}%
\pgfsys@useobject{currentmarker}{}%
\end{pgfscope}%
\begin{pgfscope}%
\pgfsys@transformshift{0.605355in}{1.800271in}%
\pgfsys@useobject{currentmarker}{}%
\end{pgfscope}%
\begin{pgfscope}%
\pgfsys@transformshift{1.054013in}{1.875777in}%
\pgfsys@useobject{currentmarker}{}%
\end{pgfscope}%
\begin{pgfscope}%
\pgfsys@transformshift{1.098612in}{1.593766in}%
\pgfsys@useobject{currentmarker}{}%
\end{pgfscope}%
\begin{pgfscope}%
\pgfsys@transformshift{0.861509in}{1.834953in}%
\pgfsys@useobject{currentmarker}{}%
\end{pgfscope}%
\begin{pgfscope}%
\pgfsys@transformshift{0.582121in}{1.937046in}%
\pgfsys@useobject{currentmarker}{}%
\end{pgfscope}%
\begin{pgfscope}%
\pgfsys@transformshift{1.167241in}{1.628420in}%
\pgfsys@useobject{currentmarker}{}%
\end{pgfscope}%
\begin{pgfscope}%
\pgfsys@transformshift{1.281729in}{1.806456in}%
\pgfsys@useobject{currentmarker}{}%
\end{pgfscope}%
\begin{pgfscope}%
\pgfsys@transformshift{1.375426in}{1.861385in}%
\pgfsys@useobject{currentmarker}{}%
\end{pgfscope}%
\begin{pgfscope}%
\pgfsys@transformshift{1.374241in}{1.832208in}%
\pgfsys@useobject{currentmarker}{}%
\end{pgfscope}%
\begin{pgfscope}%
\pgfsys@transformshift{0.897036in}{1.784758in}%
\pgfsys@useobject{currentmarker}{}%
\end{pgfscope}%
\begin{pgfscope}%
\pgfsys@transformshift{1.073507in}{1.855132in}%
\pgfsys@useobject{currentmarker}{}%
\end{pgfscope}%
\begin{pgfscope}%
\pgfsys@transformshift{0.689259in}{1.873160in}%
\pgfsys@useobject{currentmarker}{}%
\end{pgfscope}%
\begin{pgfscope}%
\pgfsys@transformshift{1.117181in}{1.630222in}%
\pgfsys@useobject{currentmarker}{}%
\end{pgfscope}%
\begin{pgfscope}%
\pgfsys@transformshift{1.040257in}{1.878936in}%
\pgfsys@useobject{currentmarker}{}%
\end{pgfscope}%
\begin{pgfscope}%
\pgfsys@transformshift{0.757259in}{1.939685in}%
\pgfsys@useobject{currentmarker}{}%
\end{pgfscope}%
\begin{pgfscope}%
\pgfsys@transformshift{0.929602in}{1.928785in}%
\pgfsys@useobject{currentmarker}{}%
\end{pgfscope}%
\begin{pgfscope}%
\pgfsys@transformshift{0.739375in}{1.644954in}%
\pgfsys@useobject{currentmarker}{}%
\end{pgfscope}%
\begin{pgfscope}%
\pgfsys@transformshift{1.051791in}{1.591792in}%
\pgfsys@useobject{currentmarker}{}%
\end{pgfscope}%
\begin{pgfscope}%
\pgfsys@transformshift{1.033888in}{1.782919in}%
\pgfsys@useobject{currentmarker}{}%
\end{pgfscope}%
\begin{pgfscope}%
\pgfsys@transformshift{1.087559in}{1.611531in}%
\pgfsys@useobject{currentmarker}{}%
\end{pgfscope}%
\begin{pgfscope}%
\pgfsys@transformshift{1.159095in}{1.674818in}%
\pgfsys@useobject{currentmarker}{}%
\end{pgfscope}%
\begin{pgfscope}%
\pgfsys@transformshift{0.813373in}{1.867754in}%
\pgfsys@useobject{currentmarker}{}%
\end{pgfscope}%
\begin{pgfscope}%
\pgfsys@transformshift{1.640577in}{1.620096in}%
\pgfsys@useobject{currentmarker}{}%
\end{pgfscope}%
\begin{pgfscope}%
\pgfsys@transformshift{2.178211in}{1.605394in}%
\pgfsys@useobject{currentmarker}{}%
\end{pgfscope}%
\begin{pgfscope}%
\pgfsys@transformshift{0.693850in}{1.840471in}%
\pgfsys@useobject{currentmarker}{}%
\end{pgfscope}%
\begin{pgfscope}%
\pgfsys@transformshift{1.166519in}{1.621985in}%
\pgfsys@useobject{currentmarker}{}%
\end{pgfscope}%
\begin{pgfscope}%
\pgfsys@transformshift{0.643734in}{1.780881in}%
\pgfsys@useobject{currentmarker}{}%
\end{pgfscope}%
\begin{pgfscope}%
\pgfsys@transformshift{0.891278in}{1.777823in}%
\pgfsys@useobject{currentmarker}{}%
\end{pgfscope}%
\begin{pgfscope}%
\pgfsys@transformshift{0.789639in}{1.675559in}%
\pgfsys@useobject{currentmarker}{}%
\end{pgfscope}%
\begin{pgfscope}%
\pgfsys@transformshift{0.905830in}{1.742443in}%
\pgfsys@useobject{currentmarker}{}%
\end{pgfscope}%
\begin{pgfscope}%
\pgfsys@transformshift{1.170666in}{1.898037in}%
\pgfsys@useobject{currentmarker}{}%
\end{pgfscope}%
\begin{pgfscope}%
\pgfsys@transformshift{0.851123in}{1.864286in}%
\pgfsys@useobject{currentmarker}{}%
\end{pgfscope}%
\begin{pgfscope}%
\pgfsys@transformshift{1.511890in}{1.874294in}%
\pgfsys@useobject{currentmarker}{}%
\end{pgfscope}%
\begin{pgfscope}%
\pgfsys@transformshift{1.126178in}{1.700618in}%
\pgfsys@useobject{currentmarker}{}%
\end{pgfscope}%
\begin{pgfscope}%
\pgfsys@transformshift{0.794434in}{1.858765in}%
\pgfsys@useobject{currentmarker}{}%
\end{pgfscope}%
\begin{pgfscope}%
\pgfsys@transformshift{1.144285in}{1.620962in}%
\pgfsys@useobject{currentmarker}{}%
\end{pgfscope}%
\begin{pgfscope}%
\pgfsys@transformshift{0.668153in}{1.824243in}%
\pgfsys@useobject{currentmarker}{}%
\end{pgfscope}%
\begin{pgfscope}%
\pgfsys@transformshift{0.919012in}{1.763856in}%
\pgfsys@useobject{currentmarker}{}%
\end{pgfscope}%
\begin{pgfscope}%
\pgfsys@transformshift{1.167260in}{1.856000in}%
\pgfsys@useobject{currentmarker}{}%
\end{pgfscope}%
\begin{pgfscope}%
\pgfsys@transformshift{0.990696in}{1.811800in}%
\pgfsys@useobject{currentmarker}{}%
\end{pgfscope}%
\begin{pgfscope}%
\pgfsys@transformshift{0.884725in}{1.653594in}%
\pgfsys@useobject{currentmarker}{}%
\end{pgfscope}%
\begin{pgfscope}%
\pgfsys@transformshift{0.825592in}{1.746739in}%
\pgfsys@useobject{currentmarker}{}%
\end{pgfscope}%
\begin{pgfscope}%
\pgfsys@transformshift{0.824667in}{1.712923in}%
\pgfsys@useobject{currentmarker}{}%
\end{pgfscope}%
\begin{pgfscope}%
\pgfsys@transformshift{1.211526in}{1.854338in}%
\pgfsys@useobject{currentmarker}{}%
\end{pgfscope}%
\begin{pgfscope}%
\pgfsys@transformshift{1.235205in}{1.856644in}%
\pgfsys@useobject{currentmarker}{}%
\end{pgfscope}%
\begin{pgfscope}%
\pgfsys@transformshift{1.055049in}{1.788785in}%
\pgfsys@useobject{currentmarker}{}%
\end{pgfscope}%
\begin{pgfscope}%
\pgfsys@transformshift{1.121531in}{1.838366in}%
\pgfsys@useobject{currentmarker}{}%
\end{pgfscope}%
\begin{pgfscope}%
\pgfsys@transformshift{1.293522in}{1.600464in}%
\pgfsys@useobject{currentmarker}{}%
\end{pgfscope}%
\begin{pgfscope}%
\pgfsys@transformshift{0.598931in}{1.743980in}%
\pgfsys@useobject{currentmarker}{}%
\end{pgfscope}%
\begin{pgfscope}%
\pgfsys@transformshift{0.969776in}{1.758676in}%
\pgfsys@useobject{currentmarker}{}%
\end{pgfscope}%
\begin{pgfscope}%
\pgfsys@transformshift{1.073637in}{1.778614in}%
\pgfsys@useobject{currentmarker}{}%
\end{pgfscope}%
\begin{pgfscope}%
\pgfsys@transformshift{1.012061in}{1.885254in}%
\pgfsys@useobject{currentmarker}{}%
\end{pgfscope}%
\begin{pgfscope}%
\pgfsys@transformshift{0.981662in}{1.757521in}%
\pgfsys@useobject{currentmarker}{}%
\end{pgfscope}%
\begin{pgfscope}%
\pgfsys@transformshift{0.639994in}{1.817526in}%
\pgfsys@useobject{currentmarker}{}%
\end{pgfscope}%
\begin{pgfscope}%
\pgfsys@transformshift{0.958131in}{1.792722in}%
\pgfsys@useobject{currentmarker}{}%
\end{pgfscope}%
\begin{pgfscope}%
\pgfsys@transformshift{0.822908in}{1.642397in}%
\pgfsys@useobject{currentmarker}{}%
\end{pgfscope}%
\begin{pgfscope}%
\pgfsys@transformshift{0.720176in}{1.656103in}%
\pgfsys@useobject{currentmarker}{}%
\end{pgfscope}%
\begin{pgfscope}%
\pgfsys@transformshift{0.636847in}{1.736430in}%
\pgfsys@useobject{currentmarker}{}%
\end{pgfscope}%
\begin{pgfscope}%
\pgfsys@transformshift{0.850863in}{1.761569in}%
\pgfsys@useobject{currentmarker}{}%
\end{pgfscope}%
\begin{pgfscope}%
\pgfsys@transformshift{1.514574in}{1.810843in}%
\pgfsys@useobject{currentmarker}{}%
\end{pgfscope}%
\begin{pgfscope}%
\pgfsys@transformshift{0.970535in}{1.826298in}%
\pgfsys@useobject{currentmarker}{}%
\end{pgfscope}%
\begin{pgfscope}%
\pgfsys@transformshift{0.976256in}{1.872356in}%
\pgfsys@useobject{currentmarker}{}%
\end{pgfscope}%
\begin{pgfscope}%
\pgfsys@transformshift{0.879633in}{1.875131in}%
\pgfsys@useobject{currentmarker}{}%
\end{pgfscope}%
\begin{pgfscope}%
\pgfsys@transformshift{0.655508in}{1.627494in}%
\pgfsys@useobject{currentmarker}{}%
\end{pgfscope}%
\begin{pgfscope}%
\pgfsys@transformshift{0.702514in}{1.788896in}%
\pgfsys@useobject{currentmarker}{}%
\end{pgfscope}%
\begin{pgfscope}%
\pgfsys@transformshift{0.714474in}{1.659653in}%
\pgfsys@useobject{currentmarker}{}%
\end{pgfscope}%
\begin{pgfscope}%
\pgfsys@transformshift{1.412398in}{1.673498in}%
\pgfsys@useobject{currentmarker}{}%
\end{pgfscope}%
\begin{pgfscope}%
\pgfsys@transformshift{1.041442in}{1.865828in}%
\pgfsys@useobject{currentmarker}{}%
\end{pgfscope}%
\begin{pgfscope}%
\pgfsys@transformshift{0.890575in}{1.775652in}%
\pgfsys@useobject{currentmarker}{}%
\end{pgfscope}%
\begin{pgfscope}%
\pgfsys@transformshift{1.229854in}{1.677172in}%
\pgfsys@useobject{currentmarker}{}%
\end{pgfscope}%
\begin{pgfscope}%
\pgfsys@transformshift{1.163502in}{1.847641in}%
\pgfsys@useobject{currentmarker}{}%
\end{pgfscope}%
\begin{pgfscope}%
\pgfsys@transformshift{0.904941in}{1.910894in}%
\pgfsys@useobject{currentmarker}{}%
\end{pgfscope}%
\begin{pgfscope}%
\pgfsys@transformshift{0.963778in}{1.931088in}%
\pgfsys@useobject{currentmarker}{}%
\end{pgfscope}%
\begin{pgfscope}%
\pgfsys@transformshift{1.162317in}{1.696720in}%
\pgfsys@useobject{currentmarker}{}%
\end{pgfscope}%
\begin{pgfscope}%
\pgfsys@transformshift{0.708402in}{1.686769in}%
\pgfsys@useobject{currentmarker}{}%
\end{pgfscope}%
\begin{pgfscope}%
\pgfsys@transformshift{0.704162in}{1.862790in}%
\pgfsys@useobject{currentmarker}{}%
\end{pgfscope}%
\begin{pgfscope}%
\pgfsys@transformshift{1.154782in}{1.588008in}%
\pgfsys@useobject{currentmarker}{}%
\end{pgfscope}%
\begin{pgfscope}%
\pgfsys@transformshift{0.978699in}{1.658147in}%
\pgfsys@useobject{currentmarker}{}%
\end{pgfscope}%
\begin{pgfscope}%
\pgfsys@transformshift{1.451036in}{1.868532in}%
\pgfsys@useobject{currentmarker}{}%
\end{pgfscope}%
\begin{pgfscope}%
\pgfsys@transformshift{1.034425in}{1.877738in}%
\pgfsys@useobject{currentmarker}{}%
\end{pgfscope}%
\begin{pgfscope}%
\pgfsys@transformshift{0.615926in}{1.823296in}%
\pgfsys@useobject{currentmarker}{}%
\end{pgfscope}%
\begin{pgfscope}%
\pgfsys@transformshift{1.270158in}{1.750729in}%
\pgfsys@useobject{currentmarker}{}%
\end{pgfscope}%
\begin{pgfscope}%
\pgfsys@transformshift{0.708883in}{1.711251in}%
\pgfsys@useobject{currentmarker}{}%
\end{pgfscope}%
\begin{pgfscope}%
\pgfsys@transformshift{0.740763in}{1.590289in}%
\pgfsys@useobject{currentmarker}{}%
\end{pgfscope}%
\begin{pgfscope}%
\pgfsys@transformshift{0.614020in}{1.739033in}%
\pgfsys@useobject{currentmarker}{}%
\end{pgfscope}%
\begin{pgfscope}%
\pgfsys@transformshift{1.067953in}{1.813492in}%
\pgfsys@useobject{currentmarker}{}%
\end{pgfscope}%
\begin{pgfscope}%
\pgfsys@transformshift{0.777420in}{1.783219in}%
\pgfsys@useobject{currentmarker}{}%
\end{pgfscope}%
\begin{pgfscope}%
\pgfsys@transformshift{0.653398in}{1.768011in}%
\pgfsys@useobject{currentmarker}{}%
\end{pgfscope}%
\begin{pgfscope}%
\pgfsys@transformshift{0.733321in}{1.744320in}%
\pgfsys@useobject{currentmarker}{}%
\end{pgfscope}%
\begin{pgfscope}%
\pgfsys@transformshift{0.711753in}{1.871267in}%
\pgfsys@useobject{currentmarker}{}%
\end{pgfscope}%
\begin{pgfscope}%
\pgfsys@transformshift{1.082709in}{1.915411in}%
\pgfsys@useobject{currentmarker}{}%
\end{pgfscope}%
\begin{pgfscope}%
\pgfsys@transformshift{1.677845in}{1.579415in}%
\pgfsys@useobject{currentmarker}{}%
\end{pgfscope}%
\begin{pgfscope}%
\pgfsys@transformshift{0.696497in}{1.785302in}%
\pgfsys@useobject{currentmarker}{}%
\end{pgfscope}%
\begin{pgfscope}%
\pgfsys@transformshift{0.757129in}{1.755305in}%
\pgfsys@useobject{currentmarker}{}%
\end{pgfscope}%
\begin{pgfscope}%
\pgfsys@transformshift{1.072767in}{1.936922in}%
\pgfsys@useobject{currentmarker}{}%
\end{pgfscope}%
\begin{pgfscope}%
\pgfsys@transformshift{1.280452in}{1.913006in}%
\pgfsys@useobject{currentmarker}{}%
\end{pgfscope}%
\begin{pgfscope}%
\pgfsys@transformshift{0.830258in}{1.739904in}%
\pgfsys@useobject{currentmarker}{}%
\end{pgfscope}%
\begin{pgfscope}%
\pgfsys@transformshift{0.709420in}{1.612620in}%
\pgfsys@useobject{currentmarker}{}%
\end{pgfscope}%
\begin{pgfscope}%
\pgfsys@transformshift{1.313295in}{1.791582in}%
\pgfsys@useobject{currentmarker}{}%
\end{pgfscope}%
\begin{pgfscope}%
\pgfsys@transformshift{0.966851in}{1.846614in}%
\pgfsys@useobject{currentmarker}{}%
\end{pgfscope}%
\begin{pgfscope}%
\pgfsys@transformshift{0.979088in}{1.907271in}%
\pgfsys@useobject{currentmarker}{}%
\end{pgfscope}%
\begin{pgfscope}%
\pgfsys@transformshift{0.852955in}{1.633751in}%
\pgfsys@useobject{currentmarker}{}%
\end{pgfscope}%
\begin{pgfscope}%
\pgfsys@transformshift{0.752519in}{1.705180in}%
\pgfsys@useobject{currentmarker}{}%
\end{pgfscope}%
\begin{pgfscope}%
\pgfsys@transformshift{0.942839in}{1.844467in}%
\pgfsys@useobject{currentmarker}{}%
\end{pgfscope}%
\begin{pgfscope}%
\pgfsys@transformshift{0.747724in}{1.828915in}%
\pgfsys@useobject{currentmarker}{}%
\end{pgfscope}%
\begin{pgfscope}%
\pgfsys@transformshift{1.116200in}{1.878553in}%
\pgfsys@useobject{currentmarker}{}%
\end{pgfscope}%
\begin{pgfscope}%
\pgfsys@transformshift{0.695424in}{1.667130in}%
\pgfsys@useobject{currentmarker}{}%
\end{pgfscope}%
\begin{pgfscope}%
\pgfsys@transformshift{1.740661in}{1.793193in}%
\pgfsys@useobject{currentmarker}{}%
\end{pgfscope}%
\begin{pgfscope}%
\pgfsys@transformshift{0.688629in}{1.857871in}%
\pgfsys@useobject{currentmarker}{}%
\end{pgfscope}%
\begin{pgfscope}%
\pgfsys@transformshift{0.892574in}{1.929878in}%
\pgfsys@useobject{currentmarker}{}%
\end{pgfscope}%
\begin{pgfscope}%
\pgfsys@transformshift{0.811004in}{1.765420in}%
\pgfsys@useobject{currentmarker}{}%
\end{pgfscope}%
\begin{pgfscope}%
\pgfsys@transformshift{1.098889in}{1.892486in}%
\pgfsys@useobject{currentmarker}{}%
\end{pgfscope}%
\begin{pgfscope}%
\pgfsys@transformshift{0.812874in}{1.623338in}%
\pgfsys@useobject{currentmarker}{}%
\end{pgfscope}%
\begin{pgfscope}%
\pgfsys@transformshift{0.678317in}{1.856209in}%
\pgfsys@useobject{currentmarker}{}%
\end{pgfscope}%
\begin{pgfscope}%
\pgfsys@transformshift{1.330735in}{1.620592in}%
\pgfsys@useobject{currentmarker}{}%
\end{pgfscope}%
\begin{pgfscope}%
\pgfsys@transformshift{0.845531in}{1.598996in}%
\pgfsys@useobject{currentmarker}{}%
\end{pgfscope}%
\begin{pgfscope}%
\pgfsys@transformshift{1.154911in}{1.840140in}%
\pgfsys@useobject{currentmarker}{}%
\end{pgfscope}%
\begin{pgfscope}%
\pgfsys@transformshift{0.984050in}{2.007105in}%
\pgfsys@useobject{currentmarker}{}%
\end{pgfscope}%
\begin{pgfscope}%
\pgfsys@transformshift{0.736913in}{1.802349in}%
\pgfsys@useobject{currentmarker}{}%
\end{pgfscope}%
\begin{pgfscope}%
\pgfsys@transformshift{0.696775in}{1.661353in}%
\pgfsys@useobject{currentmarker}{}%
\end{pgfscope}%
\begin{pgfscope}%
\pgfsys@transformshift{0.594914in}{1.838686in}%
\pgfsys@useobject{currentmarker}{}%
\end{pgfscope}%
\begin{pgfscope}%
\pgfsys@transformshift{1.029056in}{1.759462in}%
\pgfsys@useobject{currentmarker}{}%
\end{pgfscope}%
\begin{pgfscope}%
\pgfsys@transformshift{1.353377in}{1.845944in}%
\pgfsys@useobject{currentmarker}{}%
\end{pgfscope}%
\begin{pgfscope}%
\pgfsys@transformshift{0.844328in}{1.863547in}%
\pgfsys@useobject{currentmarker}{}%
\end{pgfscope}%
\begin{pgfscope}%
\pgfsys@transformshift{0.897277in}{1.850610in}%
\pgfsys@useobject{currentmarker}{}%
\end{pgfscope}%
\begin{pgfscope}%
\pgfsys@transformshift{0.995325in}{1.693386in}%
\pgfsys@useobject{currentmarker}{}%
\end{pgfscope}%
\begin{pgfscope}%
\pgfsys@transformshift{1.056327in}{1.727723in}%
\pgfsys@useobject{currentmarker}{}%
\end{pgfscope}%
\begin{pgfscope}%
\pgfsys@transformshift{0.564107in}{1.791870in}%
\pgfsys@useobject{currentmarker}{}%
\end{pgfscope}%
\begin{pgfscope}%
\pgfsys@transformshift{0.656175in}{1.806705in}%
\pgfsys@useobject{currentmarker}{}%
\end{pgfscope}%
\begin{pgfscope}%
\pgfsys@transformshift{1.144433in}{1.704642in}%
\pgfsys@useobject{currentmarker}{}%
\end{pgfscope}%
\begin{pgfscope}%
\pgfsys@transformshift{0.628238in}{1.803839in}%
\pgfsys@useobject{currentmarker}{}%
\end{pgfscope}%
\begin{pgfscope}%
\pgfsys@transformshift{1.040257in}{1.750829in}%
\pgfsys@useobject{currentmarker}{}%
\end{pgfscope}%
\begin{pgfscope}%
\pgfsys@transformshift{0.967406in}{1.786932in}%
\pgfsys@useobject{currentmarker}{}%
\end{pgfscope}%
\begin{pgfscope}%
\pgfsys@transformshift{1.329216in}{1.843517in}%
\pgfsys@useobject{currentmarker}{}%
\end{pgfscope}%
\begin{pgfscope}%
\pgfsys@transformshift{0.564607in}{1.835762in}%
\pgfsys@useobject{currentmarker}{}%
\end{pgfscope}%
\begin{pgfscope}%
\pgfsys@transformshift{0.742263in}{1.884015in}%
\pgfsys@useobject{currentmarker}{}%
\end{pgfscope}%
\begin{pgfscope}%
\pgfsys@transformshift{0.629367in}{1.824833in}%
\pgfsys@useobject{currentmarker}{}%
\end{pgfscope}%
\begin{pgfscope}%
\pgfsys@transformshift{0.569846in}{1.679928in}%
\pgfsys@useobject{currentmarker}{}%
\end{pgfscope}%
\begin{pgfscope}%
\pgfsys@transformshift{1.470660in}{1.734043in}%
\pgfsys@useobject{currentmarker}{}%
\end{pgfscope}%
\begin{pgfscope}%
\pgfsys@transformshift{1.976543in}{1.558779in}%
\pgfsys@useobject{currentmarker}{}%
\end{pgfscope}%
\begin{pgfscope}%
\pgfsys@transformshift{1.014486in}{1.869416in}%
\pgfsys@useobject{currentmarker}{}%
\end{pgfscope}%
\begin{pgfscope}%
\pgfsys@transformshift{1.379055in}{1.782340in}%
\pgfsys@useobject{currentmarker}{}%
\end{pgfscope}%
\begin{pgfscope}%
\pgfsys@transformshift{0.658063in}{1.917360in}%
\pgfsys@useobject{currentmarker}{}%
\end{pgfscope}%
\begin{pgfscope}%
\pgfsys@transformshift{0.965129in}{1.904860in}%
\pgfsys@useobject{currentmarker}{}%
\end{pgfscope}%
\begin{pgfscope}%
\pgfsys@transformshift{0.714733in}{1.797450in}%
\pgfsys@useobject{currentmarker}{}%
\end{pgfscope}%
\begin{pgfscope}%
\pgfsys@transformshift{0.734580in}{1.916747in}%
\pgfsys@useobject{currentmarker}{}%
\end{pgfscope}%
\begin{pgfscope}%
\pgfsys@transformshift{1.146876in}{1.696587in}%
\pgfsys@useobject{currentmarker}{}%
\end{pgfscope}%
\begin{pgfscope}%
\pgfsys@transformshift{1.112552in}{1.839599in}%
\pgfsys@useobject{currentmarker}{}%
\end{pgfscope}%
\begin{pgfscope}%
\pgfsys@transformshift{0.609724in}{1.818679in}%
\pgfsys@useobject{currentmarker}{}%
\end{pgfscope}%
\begin{pgfscope}%
\pgfsys@transformshift{0.713974in}{1.766018in}%
\pgfsys@useobject{currentmarker}{}%
\end{pgfscope}%
\begin{pgfscope}%
\pgfsys@transformshift{0.625054in}{1.764230in}%
\pgfsys@useobject{currentmarker}{}%
\end{pgfscope}%
\begin{pgfscope}%
\pgfsys@transformshift{0.695701in}{1.822753in}%
\pgfsys@useobject{currentmarker}{}%
\end{pgfscope}%
\begin{pgfscope}%
\pgfsys@transformshift{0.630052in}{1.476381in}%
\pgfsys@useobject{currentmarker}{}%
\end{pgfscope}%
\begin{pgfscope}%
\pgfsys@transformshift{1.032148in}{1.689393in}%
\pgfsys@useobject{currentmarker}{}%
\end{pgfscope}%
\begin{pgfscope}%
\pgfsys@transformshift{0.974293in}{1.929446in}%
\pgfsys@useobject{currentmarker}{}%
\end{pgfscope}%
\begin{pgfscope}%
\pgfsys@transformshift{0.679965in}{1.725110in}%
\pgfsys@useobject{currentmarker}{}%
\end{pgfscope}%
\begin{pgfscope}%
\pgfsys@transformshift{1.437521in}{1.745444in}%
\pgfsys@useobject{currentmarker}{}%
\end{pgfscope}%
\begin{pgfscope}%
\pgfsys@transformshift{0.779957in}{1.727436in}%
\pgfsys@useobject{currentmarker}{}%
\end{pgfscope}%
\begin{pgfscope}%
\pgfsys@transformshift{1.260216in}{1.766004in}%
\pgfsys@useobject{currentmarker}{}%
\end{pgfscope}%
\begin{pgfscope}%
\pgfsys@transformshift{0.783159in}{1.796884in}%
\pgfsys@useobject{currentmarker}{}%
\end{pgfscope}%
\begin{pgfscope}%
\pgfsys@transformshift{0.627534in}{1.881811in}%
\pgfsys@useobject{currentmarker}{}%
\end{pgfscope}%
\begin{pgfscope}%
\pgfsys@transformshift{0.732803in}{1.652619in}%
\pgfsys@useobject{currentmarker}{}%
\end{pgfscope}%
\begin{pgfscope}%
\pgfsys@transformshift{1.221153in}{1.800866in}%
\pgfsys@useobject{currentmarker}{}%
\end{pgfscope}%
\begin{pgfscope}%
\pgfsys@transformshift{0.997343in}{1.722368in}%
\pgfsys@useobject{currentmarker}{}%
\end{pgfscope}%
\begin{pgfscope}%
\pgfsys@transformshift{0.699904in}{1.808259in}%
\pgfsys@useobject{currentmarker}{}%
\end{pgfscope}%
\begin{pgfscope}%
\pgfsys@transformshift{0.739856in}{1.742498in}%
\pgfsys@useobject{currentmarker}{}%
\end{pgfscope}%
\begin{pgfscope}%
\pgfsys@transformshift{0.561941in}{1.953820in}%
\pgfsys@useobject{currentmarker}{}%
\end{pgfscope}%
\begin{pgfscope}%
\pgfsys@transformshift{1.009173in}{1.751604in}%
\pgfsys@useobject{currentmarker}{}%
\end{pgfscope}%
\begin{pgfscope}%
\pgfsys@transformshift{0.728952in}{1.724207in}%
\pgfsys@useobject{currentmarker}{}%
\end{pgfscope}%
\begin{pgfscope}%
\pgfsys@transformshift{1.463440in}{1.871359in}%
\pgfsys@useobject{currentmarker}{}%
\end{pgfscope}%
\begin{pgfscope}%
\pgfsys@transformshift{0.638106in}{1.605994in}%
\pgfsys@useobject{currentmarker}{}%
\end{pgfscope}%
\begin{pgfscope}%
\pgfsys@transformshift{0.986864in}{1.919433in}%
\pgfsys@useobject{currentmarker}{}%
\end{pgfscope}%
\end{pgfscope}%
\begin{pgfscope}%
\pgfpathrectangle{\pgfqpoint{0.526284in}{0.473557in}}{\pgfqpoint{1.651927in}{1.704653in}}%
\pgfusepath{clip}%
\pgfsetbuttcap%
\pgfsetroundjoin%
\definecolor{currentfill}{rgb}{1.000000,0.498039,0.054902}%
\pgfsetfillcolor{currentfill}%
\pgfsetfillopacity{0.150000}%
\pgfsetlinewidth{0.000000pt}%
\definecolor{currentstroke}{rgb}{0.000000,0.000000,0.000000}%
\pgfsetstrokecolor{currentstroke}%
\pgfsetdash{}{0pt}%
\pgfpathmoveto{\pgfqpoint{0.526284in}{1.818805in}}%
\pgfpathlineto{\pgfqpoint{0.526284in}{1.790189in}}%
\pgfpathlineto{\pgfqpoint{0.542970in}{1.789551in}}%
\pgfpathlineto{\pgfqpoint{0.559656in}{1.789046in}}%
\pgfpathlineto{\pgfqpoint{0.576342in}{1.788497in}}%
\pgfpathlineto{\pgfqpoint{0.593028in}{1.787913in}}%
\pgfpathlineto{\pgfqpoint{0.609715in}{1.787222in}}%
\pgfpathlineto{\pgfqpoint{0.626401in}{1.786661in}}%
\pgfpathlineto{\pgfqpoint{0.643087in}{1.786312in}}%
\pgfpathlineto{\pgfqpoint{0.659773in}{1.785628in}}%
\pgfpathlineto{\pgfqpoint{0.676459in}{1.785019in}}%
\pgfpathlineto{\pgfqpoint{0.693145in}{1.784562in}}%
\pgfpathlineto{\pgfqpoint{0.709831in}{1.784078in}}%
\pgfpathlineto{\pgfqpoint{0.726517in}{1.783561in}}%
\pgfpathlineto{\pgfqpoint{0.743204in}{1.782978in}}%
\pgfpathlineto{\pgfqpoint{0.759890in}{1.782224in}}%
\pgfpathlineto{\pgfqpoint{0.776576in}{1.781677in}}%
\pgfpathlineto{\pgfqpoint{0.793262in}{1.780731in}}%
\pgfpathlineto{\pgfqpoint{0.809948in}{1.779928in}}%
\pgfpathlineto{\pgfqpoint{0.826634in}{1.779155in}}%
\pgfpathlineto{\pgfqpoint{0.843320in}{1.778229in}}%
\pgfpathlineto{\pgfqpoint{0.860006in}{1.777419in}}%
\pgfpathlineto{\pgfqpoint{0.876693in}{1.776706in}}%
\pgfpathlineto{\pgfqpoint{0.893379in}{1.775959in}}%
\pgfpathlineto{\pgfqpoint{0.910065in}{1.774887in}}%
\pgfpathlineto{\pgfqpoint{0.926751in}{1.773978in}}%
\pgfpathlineto{\pgfqpoint{0.943437in}{1.772956in}}%
\pgfpathlineto{\pgfqpoint{0.960123in}{1.772034in}}%
\pgfpathlineto{\pgfqpoint{0.976809in}{1.771170in}}%
\pgfpathlineto{\pgfqpoint{0.993495in}{1.770084in}}%
\pgfpathlineto{\pgfqpoint{1.010182in}{1.768958in}}%
\pgfpathlineto{\pgfqpoint{1.026868in}{1.767796in}}%
\pgfpathlineto{\pgfqpoint{1.043554in}{1.766573in}}%
\pgfpathlineto{\pgfqpoint{1.060240in}{1.765519in}}%
\pgfpathlineto{\pgfqpoint{1.076926in}{1.764283in}}%
\pgfpathlineto{\pgfqpoint{1.093612in}{1.763178in}}%
\pgfpathlineto{\pgfqpoint{1.110298in}{1.761902in}}%
\pgfpathlineto{\pgfqpoint{1.126985in}{1.760560in}}%
\pgfpathlineto{\pgfqpoint{1.143671in}{1.759384in}}%
\pgfpathlineto{\pgfqpoint{1.160357in}{1.758260in}}%
\pgfpathlineto{\pgfqpoint{1.177043in}{1.757022in}}%
\pgfpathlineto{\pgfqpoint{1.193729in}{1.755673in}}%
\pgfpathlineto{\pgfqpoint{1.210415in}{1.754378in}}%
\pgfpathlineto{\pgfqpoint{1.227101in}{1.753008in}}%
\pgfpathlineto{\pgfqpoint{1.243787in}{1.751718in}}%
\pgfpathlineto{\pgfqpoint{1.260474in}{1.750213in}}%
\pgfpathlineto{\pgfqpoint{1.277160in}{1.748791in}}%
\pgfpathlineto{\pgfqpoint{1.293846in}{1.747455in}}%
\pgfpathlineto{\pgfqpoint{1.310532in}{1.746342in}}%
\pgfpathlineto{\pgfqpoint{1.327218in}{1.745230in}}%
\pgfpathlineto{\pgfqpoint{1.343904in}{1.744012in}}%
\pgfpathlineto{\pgfqpoint{1.360590in}{1.742720in}}%
\pgfpathlineto{\pgfqpoint{1.377276in}{1.741389in}}%
\pgfpathlineto{\pgfqpoint{1.393963in}{1.739876in}}%
\pgfpathlineto{\pgfqpoint{1.410649in}{1.738630in}}%
\pgfpathlineto{\pgfqpoint{1.427335in}{1.737223in}}%
\pgfpathlineto{\pgfqpoint{1.444021in}{1.735862in}}%
\pgfpathlineto{\pgfqpoint{1.460707in}{1.734449in}}%
\pgfpathlineto{\pgfqpoint{1.477393in}{1.732941in}}%
\pgfpathlineto{\pgfqpoint{1.494079in}{1.731623in}}%
\pgfpathlineto{\pgfqpoint{1.510765in}{1.730242in}}%
\pgfpathlineto{\pgfqpoint{1.527452in}{1.728835in}}%
\pgfpathlineto{\pgfqpoint{1.544138in}{1.727428in}}%
\pgfpathlineto{\pgfqpoint{1.560824in}{1.726092in}}%
\pgfpathlineto{\pgfqpoint{1.577510in}{1.724900in}}%
\pgfpathlineto{\pgfqpoint{1.594196in}{1.723609in}}%
\pgfpathlineto{\pgfqpoint{1.610882in}{1.722127in}}%
\pgfpathlineto{\pgfqpoint{1.627568in}{1.720728in}}%
\pgfpathlineto{\pgfqpoint{1.644255in}{1.719431in}}%
\pgfpathlineto{\pgfqpoint{1.660941in}{1.718134in}}%
\pgfpathlineto{\pgfqpoint{1.677627in}{1.716856in}}%
\pgfpathlineto{\pgfqpoint{1.694313in}{1.715512in}}%
\pgfpathlineto{\pgfqpoint{1.710999in}{1.714093in}}%
\pgfpathlineto{\pgfqpoint{1.727685in}{1.712673in}}%
\pgfpathlineto{\pgfqpoint{1.744371in}{1.711254in}}%
\pgfpathlineto{\pgfqpoint{1.761057in}{1.709834in}}%
\pgfpathlineto{\pgfqpoint{1.777744in}{1.708411in}}%
\pgfpathlineto{\pgfqpoint{1.794430in}{1.706908in}}%
\pgfpathlineto{\pgfqpoint{1.811116in}{1.705404in}}%
\pgfpathlineto{\pgfqpoint{1.827802in}{1.703900in}}%
\pgfpathlineto{\pgfqpoint{1.844488in}{1.702397in}}%
\pgfpathlineto{\pgfqpoint{1.861174in}{1.700893in}}%
\pgfpathlineto{\pgfqpoint{1.877860in}{1.699392in}}%
\pgfpathlineto{\pgfqpoint{1.894546in}{1.698012in}}%
\pgfpathlineto{\pgfqpoint{1.911233in}{1.696639in}}%
\pgfpathlineto{\pgfqpoint{1.927919in}{1.695266in}}%
\pgfpathlineto{\pgfqpoint{1.944605in}{1.693892in}}%
\pgfpathlineto{\pgfqpoint{1.961291in}{1.692520in}}%
\pgfpathlineto{\pgfqpoint{1.977977in}{1.691073in}}%
\pgfpathlineto{\pgfqpoint{1.994663in}{1.689581in}}%
\pgfpathlineto{\pgfqpoint{2.011349in}{1.688204in}}%
\pgfpathlineto{\pgfqpoint{2.028035in}{1.686838in}}%
\pgfpathlineto{\pgfqpoint{2.044722in}{1.685474in}}%
\pgfpathlineto{\pgfqpoint{2.061408in}{1.684110in}}%
\pgfpathlineto{\pgfqpoint{2.078094in}{1.682745in}}%
\pgfpathlineto{\pgfqpoint{2.094780in}{1.681381in}}%
\pgfpathlineto{\pgfqpoint{2.111466in}{1.679970in}}%
\pgfpathlineto{\pgfqpoint{2.128152in}{1.678461in}}%
\pgfpathlineto{\pgfqpoint{2.144838in}{1.676952in}}%
\pgfpathlineto{\pgfqpoint{2.161525in}{1.675443in}}%
\pgfpathlineto{\pgfqpoint{2.178211in}{1.673980in}}%
\pgfpathlineto{\pgfqpoint{2.178211in}{1.749582in}}%
\pgfpathlineto{\pgfqpoint{2.178211in}{1.749582in}}%
\pgfpathlineto{\pgfqpoint{2.161525in}{1.750086in}}%
\pgfpathlineto{\pgfqpoint{2.144838in}{1.750578in}}%
\pgfpathlineto{\pgfqpoint{2.128152in}{1.751069in}}%
\pgfpathlineto{\pgfqpoint{2.111466in}{1.751560in}}%
\pgfpathlineto{\pgfqpoint{2.094780in}{1.752052in}}%
\pgfpathlineto{\pgfqpoint{2.078094in}{1.752539in}}%
\pgfpathlineto{\pgfqpoint{2.061408in}{1.753018in}}%
\pgfpathlineto{\pgfqpoint{2.044722in}{1.753498in}}%
\pgfpathlineto{\pgfqpoint{2.028035in}{1.753977in}}%
\pgfpathlineto{\pgfqpoint{2.011349in}{1.754456in}}%
\pgfpathlineto{\pgfqpoint{1.994663in}{1.754936in}}%
\pgfpathlineto{\pgfqpoint{1.977977in}{1.755415in}}%
\pgfpathlineto{\pgfqpoint{1.961291in}{1.755894in}}%
\pgfpathlineto{\pgfqpoint{1.944605in}{1.756374in}}%
\pgfpathlineto{\pgfqpoint{1.927919in}{1.756853in}}%
\pgfpathlineto{\pgfqpoint{1.911233in}{1.757329in}}%
\pgfpathlineto{\pgfqpoint{1.894546in}{1.757722in}}%
\pgfpathlineto{\pgfqpoint{1.877860in}{1.758268in}}%
\pgfpathlineto{\pgfqpoint{1.861174in}{1.758768in}}%
\pgfpathlineto{\pgfqpoint{1.844488in}{1.759248in}}%
\pgfpathlineto{\pgfqpoint{1.827802in}{1.759694in}}%
\pgfpathlineto{\pgfqpoint{1.811116in}{1.760057in}}%
\pgfpathlineto{\pgfqpoint{1.794430in}{1.760421in}}%
\pgfpathlineto{\pgfqpoint{1.777744in}{1.760784in}}%
\pgfpathlineto{\pgfqpoint{1.761057in}{1.761181in}}%
\pgfpathlineto{\pgfqpoint{1.744371in}{1.761687in}}%
\pgfpathlineto{\pgfqpoint{1.727685in}{1.762194in}}%
\pgfpathlineto{\pgfqpoint{1.710999in}{1.762700in}}%
\pgfpathlineto{\pgfqpoint{1.694313in}{1.763206in}}%
\pgfpathlineto{\pgfqpoint{1.677627in}{1.763713in}}%
\pgfpathlineto{\pgfqpoint{1.660941in}{1.764219in}}%
\pgfpathlineto{\pgfqpoint{1.644255in}{1.764723in}}%
\pgfpathlineto{\pgfqpoint{1.627568in}{1.765228in}}%
\pgfpathlineto{\pgfqpoint{1.610882in}{1.765731in}}%
\pgfpathlineto{\pgfqpoint{1.594196in}{1.766159in}}%
\pgfpathlineto{\pgfqpoint{1.577510in}{1.766572in}}%
\pgfpathlineto{\pgfqpoint{1.560824in}{1.766986in}}%
\pgfpathlineto{\pgfqpoint{1.544138in}{1.767399in}}%
\pgfpathlineto{\pgfqpoint{1.527452in}{1.767813in}}%
\pgfpathlineto{\pgfqpoint{1.510765in}{1.768226in}}%
\pgfpathlineto{\pgfqpoint{1.494079in}{1.768640in}}%
\pgfpathlineto{\pgfqpoint{1.477393in}{1.769053in}}%
\pgfpathlineto{\pgfqpoint{1.460707in}{1.769620in}}%
\pgfpathlineto{\pgfqpoint{1.444021in}{1.770249in}}%
\pgfpathlineto{\pgfqpoint{1.427335in}{1.770878in}}%
\pgfpathlineto{\pgfqpoint{1.410649in}{1.771507in}}%
\pgfpathlineto{\pgfqpoint{1.393963in}{1.772136in}}%
\pgfpathlineto{\pgfqpoint{1.377276in}{1.772662in}}%
\pgfpathlineto{\pgfqpoint{1.360590in}{1.773143in}}%
\pgfpathlineto{\pgfqpoint{1.343904in}{1.773622in}}%
\pgfpathlineto{\pgfqpoint{1.327218in}{1.774099in}}%
\pgfpathlineto{\pgfqpoint{1.310532in}{1.774559in}}%
\pgfpathlineto{\pgfqpoint{1.293846in}{1.774823in}}%
\pgfpathlineto{\pgfqpoint{1.277160in}{1.775343in}}%
\pgfpathlineto{\pgfqpoint{1.260474in}{1.775740in}}%
\pgfpathlineto{\pgfqpoint{1.243787in}{1.776353in}}%
\pgfpathlineto{\pgfqpoint{1.227101in}{1.776907in}}%
\pgfpathlineto{\pgfqpoint{1.210415in}{1.777451in}}%
\pgfpathlineto{\pgfqpoint{1.193729in}{1.777766in}}%
\pgfpathlineto{\pgfqpoint{1.177043in}{1.778163in}}%
\pgfpathlineto{\pgfqpoint{1.160357in}{1.778723in}}%
\pgfpathlineto{\pgfqpoint{1.143671in}{1.779470in}}%
\pgfpathlineto{\pgfqpoint{1.126985in}{1.780045in}}%
\pgfpathlineto{\pgfqpoint{1.110298in}{1.780742in}}%
\pgfpathlineto{\pgfqpoint{1.093612in}{1.781345in}}%
\pgfpathlineto{\pgfqpoint{1.076926in}{1.782000in}}%
\pgfpathlineto{\pgfqpoint{1.060240in}{1.782659in}}%
\pgfpathlineto{\pgfqpoint{1.043554in}{1.783516in}}%
\pgfpathlineto{\pgfqpoint{1.026868in}{1.784349in}}%
\pgfpathlineto{\pgfqpoint{1.010182in}{1.785248in}}%
\pgfpathlineto{\pgfqpoint{0.993495in}{1.786133in}}%
\pgfpathlineto{\pgfqpoint{0.976809in}{1.787045in}}%
\pgfpathlineto{\pgfqpoint{0.960123in}{1.788158in}}%
\pgfpathlineto{\pgfqpoint{0.943437in}{1.789020in}}%
\pgfpathlineto{\pgfqpoint{0.926751in}{1.789923in}}%
\pgfpathlineto{\pgfqpoint{0.910065in}{1.790952in}}%
\pgfpathlineto{\pgfqpoint{0.893379in}{1.792015in}}%
\pgfpathlineto{\pgfqpoint{0.876693in}{1.793085in}}%
\pgfpathlineto{\pgfqpoint{0.860006in}{1.794207in}}%
\pgfpathlineto{\pgfqpoint{0.843320in}{1.795281in}}%
\pgfpathlineto{\pgfqpoint{0.826634in}{1.796363in}}%
\pgfpathlineto{\pgfqpoint{0.809948in}{1.797614in}}%
\pgfpathlineto{\pgfqpoint{0.793262in}{1.798834in}}%
\pgfpathlineto{\pgfqpoint{0.776576in}{1.799911in}}%
\pgfpathlineto{\pgfqpoint{0.759890in}{1.801227in}}%
\pgfpathlineto{\pgfqpoint{0.743204in}{1.802295in}}%
\pgfpathlineto{\pgfqpoint{0.726517in}{1.803291in}}%
\pgfpathlineto{\pgfqpoint{0.709831in}{1.804548in}}%
\pgfpathlineto{\pgfqpoint{0.693145in}{1.805906in}}%
\pgfpathlineto{\pgfqpoint{0.676459in}{1.807074in}}%
\pgfpathlineto{\pgfqpoint{0.659773in}{1.808404in}}%
\pgfpathlineto{\pgfqpoint{0.643087in}{1.809913in}}%
\pgfpathlineto{\pgfqpoint{0.626401in}{1.811129in}}%
\pgfpathlineto{\pgfqpoint{0.609715in}{1.812443in}}%
\pgfpathlineto{\pgfqpoint{0.593028in}{1.813892in}}%
\pgfpathlineto{\pgfqpoint{0.576342in}{1.815060in}}%
\pgfpathlineto{\pgfqpoint{0.559656in}{1.816176in}}%
\pgfpathlineto{\pgfqpoint{0.542970in}{1.817361in}}%
\pgfpathlineto{\pgfqpoint{0.526284in}{1.818805in}}%
\pgfpathclose%
\pgfusepath{fill}%
\end{pgfscope}%
\begin{pgfscope}%
\pgfpathrectangle{\pgfqpoint{0.526284in}{0.473557in}}{\pgfqpoint{1.651927in}{1.704653in}}%
\pgfusepath{clip}%
\pgfsetbuttcap%
\pgfsetroundjoin%
\definecolor{currentfill}{rgb}{0.172549,0.627451,0.172549}%
\pgfsetfillcolor{currentfill}%
\pgfsetfillopacity{0.250000}%
\pgfsetlinewidth{1.003750pt}%
\definecolor{currentstroke}{rgb}{0.172549,0.627451,0.172549}%
\pgfsetstrokecolor{currentstroke}%
\pgfsetstrokeopacity{0.250000}%
\pgfsetdash{}{0pt}%
\pgfsys@defobject{currentmarker}{\pgfqpoint{-0.017010in}{-0.017010in}}{\pgfqpoint{0.017010in}{0.017010in}}{%
\pgfpathmoveto{\pgfqpoint{0.000000in}{-0.017010in}}%
\pgfpathcurveto{\pgfqpoint{0.004511in}{-0.017010in}}{\pgfqpoint{0.008838in}{-0.015218in}}{\pgfqpoint{0.012028in}{-0.012028in}}%
\pgfpathcurveto{\pgfqpoint{0.015218in}{-0.008838in}}{\pgfqpoint{0.017010in}{-0.004511in}}{\pgfqpoint{0.017010in}{0.000000in}}%
\pgfpathcurveto{\pgfqpoint{0.017010in}{0.004511in}}{\pgfqpoint{0.015218in}{0.008838in}}{\pgfqpoint{0.012028in}{0.012028in}}%
\pgfpathcurveto{\pgfqpoint{0.008838in}{0.015218in}}{\pgfqpoint{0.004511in}{0.017010in}}{\pgfqpoint{0.000000in}{0.017010in}}%
\pgfpathcurveto{\pgfqpoint{-0.004511in}{0.017010in}}{\pgfqpoint{-0.008838in}{0.015218in}}{\pgfqpoint{-0.012028in}{0.012028in}}%
\pgfpathcurveto{\pgfqpoint{-0.015218in}{0.008838in}}{\pgfqpoint{-0.017010in}{0.004511in}}{\pgfqpoint{-0.017010in}{0.000000in}}%
\pgfpathcurveto{\pgfqpoint{-0.017010in}{-0.004511in}}{\pgfqpoint{-0.015218in}{-0.008838in}}{\pgfqpoint{-0.012028in}{-0.012028in}}%
\pgfpathcurveto{\pgfqpoint{-0.008838in}{-0.015218in}}{\pgfqpoint{-0.004511in}{-0.017010in}}{\pgfqpoint{0.000000in}{-0.017010in}}%
\pgfpathclose%
\pgfusepath{stroke,fill}%
}%
\begin{pgfscope}%
\pgfsys@transformshift{1.139286in}{1.606714in}%
\pgfsys@useobject{currentmarker}{}%
\end{pgfscope}%
\begin{pgfscope}%
\pgfsys@transformshift{0.826629in}{1.617853in}%
\pgfsys@useobject{currentmarker}{}%
\end{pgfscope}%
\begin{pgfscope}%
\pgfsys@transformshift{0.969498in}{1.515884in}%
\pgfsys@useobject{currentmarker}{}%
\end{pgfscope}%
\begin{pgfscope}%
\pgfsys@transformshift{0.732099in}{1.419968in}%
\pgfsys@useobject{currentmarker}{}%
\end{pgfscope}%
\begin{pgfscope}%
\pgfsys@transformshift{0.815558in}{1.351896in}%
\pgfsys@useobject{currentmarker}{}%
\end{pgfscope}%
\begin{pgfscope}%
\pgfsys@transformshift{1.268362in}{1.613791in}%
\pgfsys@useobject{currentmarker}{}%
\end{pgfscope}%
\begin{pgfscope}%
\pgfsys@transformshift{0.640142in}{1.290908in}%
\pgfsys@useobject{currentmarker}{}%
\end{pgfscope}%
\begin{pgfscope}%
\pgfsys@transformshift{1.031370in}{1.673935in}%
\pgfsys@useobject{currentmarker}{}%
\end{pgfscope}%
\begin{pgfscope}%
\pgfsys@transformshift{1.501355in}{1.606296in}%
\pgfsys@useobject{currentmarker}{}%
\end{pgfscope}%
\begin{pgfscope}%
\pgfsys@transformshift{1.307778in}{1.521602in}%
\pgfsys@useobject{currentmarker}{}%
\end{pgfscope}%
\begin{pgfscope}%
\pgfsys@transformshift{1.194068in}{1.457893in}%
\pgfsys@useobject{currentmarker}{}%
\end{pgfscope}%
\begin{pgfscope}%
\pgfsys@transformshift{0.667950in}{1.485851in}%
\pgfsys@useobject{currentmarker}{}%
\end{pgfscope}%
\begin{pgfscope}%
\pgfsys@transformshift{0.827370in}{1.318673in}%
\pgfsys@useobject{currentmarker}{}%
\end{pgfscope}%
\begin{pgfscope}%
\pgfsys@transformshift{1.071193in}{1.521593in}%
\pgfsys@useobject{currentmarker}{}%
\end{pgfscope}%
\begin{pgfscope}%
\pgfsys@transformshift{0.921937in}{1.440372in}%
\pgfsys@useobject{currentmarker}{}%
\end{pgfscope}%
\begin{pgfscope}%
\pgfsys@transformshift{1.556989in}{1.450585in}%
\pgfsys@useobject{currentmarker}{}%
\end{pgfscope}%
\begin{pgfscope}%
\pgfsys@transformshift{0.645159in}{1.528555in}%
\pgfsys@useobject{currentmarker}{}%
\end{pgfscope}%
\begin{pgfscope}%
\pgfsys@transformshift{1.282858in}{1.621237in}%
\pgfsys@useobject{currentmarker}{}%
\end{pgfscope}%
\begin{pgfscope}%
\pgfsys@transformshift{0.923659in}{1.542495in}%
\pgfsys@useobject{currentmarker}{}%
\end{pgfscope}%
\begin{pgfscope}%
\pgfsys@transformshift{1.196863in}{1.550727in}%
\pgfsys@useobject{currentmarker}{}%
\end{pgfscope}%
\begin{pgfscope}%
\pgfsys@transformshift{0.734895in}{1.443588in}%
\pgfsys@useobject{currentmarker}{}%
\end{pgfscope}%
\begin{pgfscope}%
\pgfsys@transformshift{1.049754in}{1.443659in}%
\pgfsys@useobject{currentmarker}{}%
\end{pgfscope}%
\begin{pgfscope}%
\pgfsys@transformshift{0.612705in}{1.457681in}%
\pgfsys@useobject{currentmarker}{}%
\end{pgfscope}%
\begin{pgfscope}%
\pgfsys@transformshift{1.196567in}{1.683004in}%
\pgfsys@useobject{currentmarker}{}%
\end{pgfscope}%
\begin{pgfscope}%
\pgfsys@transformshift{0.616815in}{1.435661in}%
\pgfsys@useobject{currentmarker}{}%
\end{pgfscope}%
\begin{pgfscope}%
\pgfsys@transformshift{0.881170in}{1.167210in}%
\pgfsys@useobject{currentmarker}{}%
\end{pgfscope}%
\begin{pgfscope}%
\pgfsys@transformshift{0.644863in}{1.394979in}%
\pgfsys@useobject{currentmarker}{}%
\end{pgfscope}%
\begin{pgfscope}%
\pgfsys@transformshift{1.119421in}{1.352817in}%
\pgfsys@useobject{currentmarker}{}%
\end{pgfscope}%
\begin{pgfscope}%
\pgfsys@transformshift{0.796952in}{1.496107in}%
\pgfsys@useobject{currentmarker}{}%
\end{pgfscope}%
\begin{pgfscope}%
\pgfsys@transformshift{0.928991in}{1.504413in}%
\pgfsys@useobject{currentmarker}{}%
\end{pgfscope}%
\begin{pgfscope}%
\pgfsys@transformshift{1.146006in}{1.603342in}%
\pgfsys@useobject{currentmarker}{}%
\end{pgfscope}%
\begin{pgfscope}%
\pgfsys@transformshift{1.088355in}{1.614999in}%
\pgfsys@useobject{currentmarker}{}%
\end{pgfscope}%
\begin{pgfscope}%
\pgfsys@transformshift{1.172758in}{1.280074in}%
\pgfsys@useobject{currentmarker}{}%
\end{pgfscope}%
\begin{pgfscope}%
\pgfsys@transformshift{1.211137in}{1.595723in}%
\pgfsys@useobject{currentmarker}{}%
\end{pgfscope}%
\begin{pgfscope}%
\pgfsys@transformshift{0.829388in}{1.474572in}%
\pgfsys@useobject{currentmarker}{}%
\end{pgfscope}%
\begin{pgfscope}%
\pgfsys@transformshift{1.584944in}{1.326983in}%
\pgfsys@useobject{currentmarker}{}%
\end{pgfscope}%
\begin{pgfscope}%
\pgfsys@transformshift{0.693739in}{1.626725in}%
\pgfsys@useobject{currentmarker}{}%
\end{pgfscope}%
\begin{pgfscope}%
\pgfsys@transformshift{1.271769in}{1.651243in}%
\pgfsys@useobject{currentmarker}{}%
\end{pgfscope}%
\begin{pgfscope}%
\pgfsys@transformshift{1.064010in}{1.588037in}%
\pgfsys@useobject{currentmarker}{}%
\end{pgfscope}%
\begin{pgfscope}%
\pgfsys@transformshift{1.461162in}{1.320850in}%
\pgfsys@useobject{currentmarker}{}%
\end{pgfscope}%
\begin{pgfscope}%
\pgfsys@transformshift{1.254162in}{1.565318in}%
\pgfsys@useobject{currentmarker}{}%
\end{pgfscope}%
\begin{pgfscope}%
\pgfsys@transformshift{0.641290in}{1.535102in}%
\pgfsys@useobject{currentmarker}{}%
\end{pgfscope}%
\begin{pgfscope}%
\pgfsys@transformshift{1.047681in}{1.344910in}%
\pgfsys@useobject{currentmarker}{}%
\end{pgfscope}%
\begin{pgfscope}%
\pgfsys@transformshift{0.651435in}{1.508661in}%
\pgfsys@useobject{currentmarker}{}%
\end{pgfscope}%
\begin{pgfscope}%
\pgfsys@transformshift{1.047699in}{1.352487in}%
\pgfsys@useobject{currentmarker}{}%
\end{pgfscope}%
\begin{pgfscope}%
\pgfsys@transformshift{0.706939in}{1.280625in}%
\pgfsys@useobject{currentmarker}{}%
\end{pgfscope}%
\begin{pgfscope}%
\pgfsys@transformshift{0.621869in}{1.666666in}%
\pgfsys@useobject{currentmarker}{}%
\end{pgfscope}%
\begin{pgfscope}%
\pgfsys@transformshift{1.117533in}{1.557343in}%
\pgfsys@useobject{currentmarker}{}%
\end{pgfscope}%
\begin{pgfscope}%
\pgfsys@transformshift{1.307648in}{1.310242in}%
\pgfsys@useobject{currentmarker}{}%
\end{pgfscope}%
\begin{pgfscope}%
\pgfsys@transformshift{0.948856in}{1.714855in}%
\pgfsys@useobject{currentmarker}{}%
\end{pgfscope}%
\begin{pgfscope}%
\pgfsys@transformshift{1.014153in}{1.517137in}%
\pgfsys@useobject{currentmarker}{}%
\end{pgfscope}%
\begin{pgfscope}%
\pgfsys@transformshift{0.631737in}{1.633275in}%
\pgfsys@useobject{currentmarker}{}%
\end{pgfscope}%
\begin{pgfscope}%
\pgfsys@transformshift{0.643493in}{1.616083in}%
\pgfsys@useobject{currentmarker}{}%
\end{pgfscope}%
\begin{pgfscope}%
\pgfsys@transformshift{0.772755in}{1.492469in}%
\pgfsys@useobject{currentmarker}{}%
\end{pgfscope}%
\begin{pgfscope}%
\pgfsys@transformshift{0.692091in}{1.371854in}%
\pgfsys@useobject{currentmarker}{}%
\end{pgfscope}%
\begin{pgfscope}%
\pgfsys@transformshift{0.595358in}{1.636336in}%
\pgfsys@useobject{currentmarker}{}%
\end{pgfscope}%
\begin{pgfscope}%
\pgfsys@transformshift{0.903997in}{1.362697in}%
\pgfsys@useobject{currentmarker}{}%
\end{pgfscope}%
\begin{pgfscope}%
\pgfsys@transformshift{1.258735in}{1.625356in}%
\pgfsys@useobject{currentmarker}{}%
\end{pgfscope}%
\begin{pgfscope}%
\pgfsys@transformshift{0.771996in}{1.387100in}%
\pgfsys@useobject{currentmarker}{}%
\end{pgfscope}%
\begin{pgfscope}%
\pgfsys@transformshift{0.665932in}{1.403441in}%
\pgfsys@useobject{currentmarker}{}%
\end{pgfscope}%
\begin{pgfscope}%
\pgfsys@transformshift{0.723120in}{1.423291in}%
\pgfsys@useobject{currentmarker}{}%
\end{pgfscope}%
\begin{pgfscope}%
\pgfsys@transformshift{1.030500in}{1.722866in}%
\pgfsys@useobject{currentmarker}{}%
\end{pgfscope}%
\begin{pgfscope}%
\pgfsys@transformshift{0.538429in}{1.510497in}%
\pgfsys@useobject{currentmarker}{}%
\end{pgfscope}%
\begin{pgfscope}%
\pgfsys@transformshift{1.081468in}{1.314181in}%
\pgfsys@useobject{currentmarker}{}%
\end{pgfscope}%
\begin{pgfscope}%
\pgfsys@transformshift{0.629293in}{1.378150in}%
\pgfsys@useobject{currentmarker}{}%
\end{pgfscope}%
\begin{pgfscope}%
\pgfsys@transformshift{0.792175in}{1.331969in}%
\pgfsys@useobject{currentmarker}{}%
\end{pgfscope}%
\begin{pgfscope}%
\pgfsys@transformshift{1.017763in}{1.379676in}%
\pgfsys@useobject{currentmarker}{}%
\end{pgfscope}%
\begin{pgfscope}%
\pgfsys@transformshift{0.953410in}{1.508156in}%
\pgfsys@useobject{currentmarker}{}%
\end{pgfscope}%
\begin{pgfscope}%
\pgfsys@transformshift{1.026687in}{1.538415in}%
\pgfsys@useobject{currentmarker}{}%
\end{pgfscope}%
\begin{pgfscope}%
\pgfsys@transformshift{1.527422in}{1.232134in}%
\pgfsys@useobject{currentmarker}{}%
\end{pgfscope}%
\begin{pgfscope}%
\pgfsys@transformshift{1.160817in}{1.566959in}%
\pgfsys@useobject{currentmarker}{}%
\end{pgfscope}%
\begin{pgfscope}%
\pgfsys@transformshift{0.698349in}{1.687640in}%
\pgfsys@useobject{currentmarker}{}%
\end{pgfscope}%
\begin{pgfscope}%
\pgfsys@transformshift{1.035203in}{1.489062in}%
\pgfsys@useobject{currentmarker}{}%
\end{pgfscope}%
\begin{pgfscope}%
\pgfsys@transformshift{0.922307in}{1.475735in}%
\pgfsys@useobject{currentmarker}{}%
\end{pgfscope}%
\begin{pgfscope}%
\pgfsys@transformshift{1.166982in}{1.580237in}%
\pgfsys@useobject{currentmarker}{}%
\end{pgfscope}%
\begin{pgfscope}%
\pgfsys@transformshift{1.119569in}{1.574437in}%
\pgfsys@useobject{currentmarker}{}%
\end{pgfscope}%
\begin{pgfscope}%
\pgfsys@transformshift{0.825518in}{1.340338in}%
\pgfsys@useobject{currentmarker}{}%
\end{pgfscope}%
\begin{pgfscope}%
\pgfsys@transformshift{1.426209in}{1.296664in}%
\pgfsys@useobject{currentmarker}{}%
\end{pgfscope}%
\begin{pgfscope}%
\pgfsys@transformshift{1.406677in}{1.584296in}%
\pgfsys@useobject{currentmarker}{}%
\end{pgfscope}%
\begin{pgfscope}%
\pgfsys@transformshift{1.415453in}{1.549304in}%
\pgfsys@useobject{currentmarker}{}%
\end{pgfscope}%
\begin{pgfscope}%
\pgfsys@transformshift{1.076914in}{1.610177in}%
\pgfsys@useobject{currentmarker}{}%
\end{pgfscope}%
\begin{pgfscope}%
\pgfsys@transformshift{0.779031in}{1.312747in}%
\pgfsys@useobject{currentmarker}{}%
\end{pgfscope}%
\begin{pgfscope}%
\pgfsys@transformshift{0.629497in}{1.473876in}%
\pgfsys@useobject{currentmarker}{}%
\end{pgfscope}%
\begin{pgfscope}%
\pgfsys@transformshift{1.519184in}{1.451883in}%
\pgfsys@useobject{currentmarker}{}%
\end{pgfscope}%
\begin{pgfscope}%
\pgfsys@transformshift{0.769237in}{1.453769in}%
\pgfsys@useobject{currentmarker}{}%
\end{pgfscope}%
\begin{pgfscope}%
\pgfsys@transformshift{1.087429in}{1.430540in}%
\pgfsys@useobject{currentmarker}{}%
\end{pgfscope}%
\begin{pgfscope}%
\pgfsys@transformshift{0.750705in}{1.725496in}%
\pgfsys@useobject{currentmarker}{}%
\end{pgfscope}%
\begin{pgfscope}%
\pgfsys@transformshift{1.495746in}{1.333971in}%
\pgfsys@useobject{currentmarker}{}%
\end{pgfscope}%
\begin{pgfscope}%
\pgfsys@transformshift{1.182959in}{1.468160in}%
\pgfsys@useobject{currentmarker}{}%
\end{pgfscope}%
\begin{pgfscope}%
\pgfsys@transformshift{0.713789in}{1.500802in}%
\pgfsys@useobject{currentmarker}{}%
\end{pgfscope}%
\begin{pgfscope}%
\pgfsys@transformshift{1.355894in}{1.482957in}%
\pgfsys@useobject{currentmarker}{}%
\end{pgfscope}%
\begin{pgfscope}%
\pgfsys@transformshift{1.635208in}{1.508340in}%
\pgfsys@useobject{currentmarker}{}%
\end{pgfscope}%
\begin{pgfscope}%
\pgfsys@transformshift{1.253848in}{1.423422in}%
\pgfsys@useobject{currentmarker}{}%
\end{pgfscope}%
\begin{pgfscope}%
\pgfsys@transformshift{0.935989in}{1.743416in}%
\pgfsys@useobject{currentmarker}{}%
\end{pgfscope}%
\begin{pgfscope}%
\pgfsys@transformshift{0.875320in}{1.591522in}%
\pgfsys@useobject{currentmarker}{}%
\end{pgfscope}%
\begin{pgfscope}%
\pgfsys@transformshift{1.109831in}{1.727708in}%
\pgfsys@useobject{currentmarker}{}%
\end{pgfscope}%
\begin{pgfscope}%
\pgfsys@transformshift{1.118292in}{1.388279in}%
\pgfsys@useobject{currentmarker}{}%
\end{pgfscope}%
\begin{pgfscope}%
\pgfsys@transformshift{1.130029in}{1.642666in}%
\pgfsys@useobject{currentmarker}{}%
\end{pgfscope}%
\begin{pgfscope}%
\pgfsys@transformshift{1.140971in}{1.647472in}%
\pgfsys@useobject{currentmarker}{}%
\end{pgfscope}%
\begin{pgfscope}%
\pgfsys@transformshift{1.052346in}{1.771868in}%
\pgfsys@useobject{currentmarker}{}%
\end{pgfscope}%
\begin{pgfscope}%
\pgfsys@transformshift{0.621629in}{1.531970in}%
\pgfsys@useobject{currentmarker}{}%
\end{pgfscope}%
\begin{pgfscope}%
\pgfsys@transformshift{0.917753in}{1.566500in}%
\pgfsys@useobject{currentmarker}{}%
\end{pgfscope}%
\begin{pgfscope}%
\pgfsys@transformshift{0.863027in}{1.522708in}%
\pgfsys@useobject{currentmarker}{}%
\end{pgfscope}%
\begin{pgfscope}%
\pgfsys@transformshift{0.672485in}{1.399824in}%
\pgfsys@useobject{currentmarker}{}%
\end{pgfscope}%
\begin{pgfscope}%
\pgfsys@transformshift{1.747437in}{1.253740in}%
\pgfsys@useobject{currentmarker}{}%
\end{pgfscope}%
\begin{pgfscope}%
\pgfsys@transformshift{0.526284in}{1.537424in}%
\pgfsys@useobject{currentmarker}{}%
\end{pgfscope}%
\begin{pgfscope}%
\pgfsys@transformshift{0.627405in}{1.407489in}%
\pgfsys@useobject{currentmarker}{}%
\end{pgfscope}%
\begin{pgfscope}%
\pgfsys@transformshift{1.270658in}{1.317278in}%
\pgfsys@useobject{currentmarker}{}%
\end{pgfscope}%
\begin{pgfscope}%
\pgfsys@transformshift{1.594053in}{1.259682in}%
\pgfsys@useobject{currentmarker}{}%
\end{pgfscope}%
\begin{pgfscope}%
\pgfsys@transformshift{0.652472in}{1.382881in}%
\pgfsys@useobject{currentmarker}{}%
\end{pgfscope}%
\begin{pgfscope}%
\pgfsys@transformshift{0.819168in}{1.493855in}%
\pgfsys@useobject{currentmarker}{}%
\end{pgfscope}%
\begin{pgfscope}%
\pgfsys@transformshift{0.847883in}{1.286424in}%
\pgfsys@useobject{currentmarker}{}%
\end{pgfscope}%
\begin{pgfscope}%
\pgfsys@transformshift{1.118495in}{1.188210in}%
\pgfsys@useobject{currentmarker}{}%
\end{pgfscope}%
\begin{pgfscope}%
\pgfsys@transformshift{0.618167in}{1.419675in}%
\pgfsys@useobject{currentmarker}{}%
\end{pgfscope}%
\begin{pgfscope}%
\pgfsys@transformshift{0.857547in}{1.556691in}%
\pgfsys@useobject{currentmarker}{}%
\end{pgfscope}%
\begin{pgfscope}%
\pgfsys@transformshift{1.090392in}{1.685948in}%
\pgfsys@useobject{currentmarker}{}%
\end{pgfscope}%
\begin{pgfscope}%
\pgfsys@transformshift{1.063566in}{1.689752in}%
\pgfsys@useobject{currentmarker}{}%
\end{pgfscope}%
\begin{pgfscope}%
\pgfsys@transformshift{1.510205in}{1.283252in}%
\pgfsys@useobject{currentmarker}{}%
\end{pgfscope}%
\begin{pgfscope}%
\pgfsys@transformshift{1.287209in}{1.613141in}%
\pgfsys@useobject{currentmarker}{}%
\end{pgfscope}%
\begin{pgfscope}%
\pgfsys@transformshift{0.778753in}{1.349922in}%
\pgfsys@useobject{currentmarker}{}%
\end{pgfscope}%
\begin{pgfscope}%
\pgfsys@transformshift{0.627757in}{1.589084in}%
\pgfsys@useobject{currentmarker}{}%
\end{pgfscope}%
\begin{pgfscope}%
\pgfsys@transformshift{1.071860in}{1.610238in}%
\pgfsys@useobject{currentmarker}{}%
\end{pgfscope}%
\begin{pgfscope}%
\pgfsys@transformshift{1.053938in}{1.472296in}%
\pgfsys@useobject{currentmarker}{}%
\end{pgfscope}%
\begin{pgfscope}%
\pgfsys@transformshift{0.910458in}{1.451741in}%
\pgfsys@useobject{currentmarker}{}%
\end{pgfscope}%
\begin{pgfscope}%
\pgfsys@transformshift{0.621962in}{1.553092in}%
\pgfsys@useobject{currentmarker}{}%
\end{pgfscope}%
\begin{pgfscope}%
\pgfsys@transformshift{0.836867in}{1.562860in}%
\pgfsys@useobject{currentmarker}{}%
\end{pgfscope}%
\begin{pgfscope}%
\pgfsys@transformshift{0.800229in}{1.609248in}%
\pgfsys@useobject{currentmarker}{}%
\end{pgfscope}%
\begin{pgfscope}%
\pgfsys@transformshift{0.958260in}{1.544516in}%
\pgfsys@useobject{currentmarker}{}%
\end{pgfscope}%
\begin{pgfscope}%
\pgfsys@transformshift{0.656045in}{1.466857in}%
\pgfsys@useobject{currentmarker}{}%
\end{pgfscope}%
\begin{pgfscope}%
\pgfsys@transformshift{0.629960in}{1.446503in}%
\pgfsys@useobject{currentmarker}{}%
\end{pgfscope}%
\begin{pgfscope}%
\pgfsys@transformshift{1.008358in}{1.341783in}%
\pgfsys@useobject{currentmarker}{}%
\end{pgfscope}%
\begin{pgfscope}%
\pgfsys@transformshift{1.258698in}{1.524954in}%
\pgfsys@useobject{currentmarker}{}%
\end{pgfscope}%
\begin{pgfscope}%
\pgfsys@transformshift{1.100889in}{1.325998in}%
\pgfsys@useobject{currentmarker}{}%
\end{pgfscope}%
\begin{pgfscope}%
\pgfsys@transformshift{1.230410in}{1.527562in}%
\pgfsys@useobject{currentmarker}{}%
\end{pgfscope}%
\begin{pgfscope}%
\pgfsys@transformshift{0.647325in}{1.601921in}%
\pgfsys@useobject{currentmarker}{}%
\end{pgfscope}%
\begin{pgfscope}%
\pgfsys@transformshift{0.904423in}{1.635150in}%
\pgfsys@useobject{currentmarker}{}%
\end{pgfscope}%
\begin{pgfscope}%
\pgfsys@transformshift{1.617602in}{1.468649in}%
\pgfsys@useobject{currentmarker}{}%
\end{pgfscope}%
\begin{pgfscope}%
\pgfsys@transformshift{1.245406in}{1.603989in}%
\pgfsys@useobject{currentmarker}{}%
\end{pgfscope}%
\begin{pgfscope}%
\pgfsys@transformshift{1.238907in}{1.592156in}%
\pgfsys@useobject{currentmarker}{}%
\end{pgfscope}%
\begin{pgfscope}%
\pgfsys@transformshift{1.187366in}{1.577575in}%
\pgfsys@useobject{currentmarker}{}%
\end{pgfscope}%
\begin{pgfscope}%
\pgfsys@transformshift{1.095724in}{1.675993in}%
\pgfsys@useobject{currentmarker}{}%
\end{pgfscope}%
\begin{pgfscope}%
\pgfsys@transformshift{1.131084in}{1.677015in}%
\pgfsys@useobject{currentmarker}{}%
\end{pgfscope}%
\begin{pgfscope}%
\pgfsys@transformshift{0.643549in}{1.476231in}%
\pgfsys@useobject{currentmarker}{}%
\end{pgfscope}%
\begin{pgfscope}%
\pgfsys@transformshift{0.726878in}{1.511632in}%
\pgfsys@useobject{currentmarker}{}%
\end{pgfscope}%
\begin{pgfscope}%
\pgfsys@transformshift{0.940728in}{1.466533in}%
\pgfsys@useobject{currentmarker}{}%
\end{pgfscope}%
\begin{pgfscope}%
\pgfsys@transformshift{0.686537in}{1.421201in}%
\pgfsys@useobject{currentmarker}{}%
\end{pgfscope}%
\begin{pgfscope}%
\pgfsys@transformshift{1.304686in}{1.588429in}%
\pgfsys@useobject{currentmarker}{}%
\end{pgfscope}%
\begin{pgfscope}%
\pgfsys@transformshift{0.822223in}{1.438161in}%
\pgfsys@useobject{currentmarker}{}%
\end{pgfscope}%
\begin{pgfscope}%
\pgfsys@transformshift{0.753593in}{1.465259in}%
\pgfsys@useobject{currentmarker}{}%
\end{pgfscope}%
\begin{pgfscope}%
\pgfsys@transformshift{1.155670in}{1.556271in}%
\pgfsys@useobject{currentmarker}{}%
\end{pgfscope}%
\begin{pgfscope}%
\pgfsys@transformshift{0.665617in}{1.330832in}%
\pgfsys@useobject{currentmarker}{}%
\end{pgfscope}%
\begin{pgfscope}%
\pgfsys@transformshift{0.722713in}{1.349042in}%
\pgfsys@useobject{currentmarker}{}%
\end{pgfscope}%
\begin{pgfscope}%
\pgfsys@transformshift{0.537651in}{1.410260in}%
\pgfsys@useobject{currentmarker}{}%
\end{pgfscope}%
\begin{pgfscope}%
\pgfsys@transformshift{0.956631in}{1.536515in}%
\pgfsys@useobject{currentmarker}{}%
\end{pgfscope}%
\begin{pgfscope}%
\pgfsys@transformshift{1.046922in}{1.469182in}%
\pgfsys@useobject{currentmarker}{}%
\end{pgfscope}%
\begin{pgfscope}%
\pgfsys@transformshift{0.812559in}{1.573709in}%
\pgfsys@useobject{currentmarker}{}%
\end{pgfscope}%
\begin{pgfscope}%
\pgfsys@transformshift{1.535254in}{1.473296in}%
\pgfsys@useobject{currentmarker}{}%
\end{pgfscope}%
\begin{pgfscope}%
\pgfsys@transformshift{0.778087in}{1.519480in}%
\pgfsys@useobject{currentmarker}{}%
\end{pgfscope}%
\begin{pgfscope}%
\pgfsys@transformshift{0.723860in}{1.647743in}%
\pgfsys@useobject{currentmarker}{}%
\end{pgfscope}%
\begin{pgfscope}%
\pgfsys@transformshift{0.970961in}{1.310723in}%
\pgfsys@useobject{currentmarker}{}%
\end{pgfscope}%
\begin{pgfscope}%
\pgfsys@transformshift{1.099149in}{1.758240in}%
\pgfsys@useobject{currentmarker}{}%
\end{pgfscope}%
\begin{pgfscope}%
\pgfsys@transformshift{1.042182in}{1.601772in}%
\pgfsys@useobject{currentmarker}{}%
\end{pgfscope}%
\begin{pgfscope}%
\pgfsys@transformshift{1.047181in}{1.705492in}%
\pgfsys@useobject{currentmarker}{}%
\end{pgfscope}%
\begin{pgfscope}%
\pgfsys@transformshift{1.569152in}{1.569986in}%
\pgfsys@useobject{currentmarker}{}%
\end{pgfscope}%
\begin{pgfscope}%
\pgfsys@transformshift{1.518018in}{1.376026in}%
\pgfsys@useobject{currentmarker}{}%
\end{pgfscope}%
\begin{pgfscope}%
\pgfsys@transformshift{0.733265in}{1.506571in}%
\pgfsys@useobject{currentmarker}{}%
\end{pgfscope}%
\begin{pgfscope}%
\pgfsys@transformshift{0.558701in}{1.420333in}%
\pgfsys@useobject{currentmarker}{}%
\end{pgfscope}%
\begin{pgfscope}%
\pgfsys@transformshift{0.600190in}{1.628685in}%
\pgfsys@useobject{currentmarker}{}%
\end{pgfscope}%
\begin{pgfscope}%
\pgfsys@transformshift{1.222060in}{1.559443in}%
\pgfsys@useobject{currentmarker}{}%
\end{pgfscope}%
\begin{pgfscope}%
\pgfsys@transformshift{0.884484in}{1.419362in}%
\pgfsys@useobject{currentmarker}{}%
\end{pgfscope}%
\begin{pgfscope}%
\pgfsys@transformshift{1.004359in}{1.498605in}%
\pgfsys@useobject{currentmarker}{}%
\end{pgfscope}%
\begin{pgfscope}%
\pgfsys@transformshift{0.853585in}{1.294906in}%
\pgfsys@useobject{currentmarker}{}%
\end{pgfscope}%
\begin{pgfscope}%
\pgfsys@transformshift{0.668357in}{1.669565in}%
\pgfsys@useobject{currentmarker}{}%
\end{pgfscope}%
\begin{pgfscope}%
\pgfsys@transformshift{0.536244in}{1.490728in}%
\pgfsys@useobject{currentmarker}{}%
\end{pgfscope}%
\begin{pgfscope}%
\pgfsys@transformshift{0.785103in}{1.522062in}%
\pgfsys@useobject{currentmarker}{}%
\end{pgfscope}%
\begin{pgfscope}%
\pgfsys@transformshift{1.369965in}{1.469308in}%
\pgfsys@useobject{currentmarker}{}%
\end{pgfscope}%
\begin{pgfscope}%
\pgfsys@transformshift{0.794379in}{1.306690in}%
\pgfsys@useobject{currentmarker}{}%
\end{pgfscope}%
\begin{pgfscope}%
\pgfsys@transformshift{0.633237in}{1.540400in}%
\pgfsys@useobject{currentmarker}{}%
\end{pgfscope}%
\begin{pgfscope}%
\pgfsys@transformshift{0.859287in}{1.401335in}%
\pgfsys@useobject{currentmarker}{}%
\end{pgfscope}%
\begin{pgfscope}%
\pgfsys@transformshift{1.032685in}{1.361684in}%
\pgfsys@useobject{currentmarker}{}%
\end{pgfscope}%
\begin{pgfscope}%
\pgfsys@transformshift{0.784141in}{1.442491in}%
\pgfsys@useobject{currentmarker}{}%
\end{pgfscope}%
\begin{pgfscope}%
\pgfsys@transformshift{0.996010in}{1.367549in}%
\pgfsys@useobject{currentmarker}{}%
\end{pgfscope}%
\begin{pgfscope}%
\pgfsys@transformshift{1.002563in}{1.218355in}%
\pgfsys@useobject{currentmarker}{}%
\end{pgfscope}%
\begin{pgfscope}%
\pgfsys@transformshift{0.930287in}{1.638321in}%
\pgfsys@useobject{currentmarker}{}%
\end{pgfscope}%
\begin{pgfscope}%
\pgfsys@transformshift{0.659693in}{1.451604in}%
\pgfsys@useobject{currentmarker}{}%
\end{pgfscope}%
\begin{pgfscope}%
\pgfsys@transformshift{1.068472in}{1.657118in}%
\pgfsys@useobject{currentmarker}{}%
\end{pgfscope}%
\begin{pgfscope}%
\pgfsys@transformshift{0.856640in}{1.571884in}%
\pgfsys@useobject{currentmarker}{}%
\end{pgfscope}%
\begin{pgfscope}%
\pgfsys@transformshift{1.405029in}{1.484051in}%
\pgfsys@useobject{currentmarker}{}%
\end{pgfscope}%
\begin{pgfscope}%
\pgfsys@transformshift{1.220135in}{1.523787in}%
\pgfsys@useobject{currentmarker}{}%
\end{pgfscope}%
\begin{pgfscope}%
\pgfsys@transformshift{0.999324in}{1.376800in}%
\pgfsys@useobject{currentmarker}{}%
\end{pgfscope}%
\begin{pgfscope}%
\pgfsys@transformshift{0.633699in}{1.484649in}%
\pgfsys@useobject{currentmarker}{}%
\end{pgfscope}%
\begin{pgfscope}%
\pgfsys@transformshift{1.135768in}{1.580402in}%
\pgfsys@useobject{currentmarker}{}%
\end{pgfscope}%
\begin{pgfscope}%
\pgfsys@transformshift{1.116959in}{1.127445in}%
\pgfsys@useobject{currentmarker}{}%
\end{pgfscope}%
\begin{pgfscope}%
\pgfsys@transformshift{0.598172in}{1.627244in}%
\pgfsys@useobject{currentmarker}{}%
\end{pgfscope}%
\begin{pgfscope}%
\pgfsys@transformshift{0.619907in}{1.492015in}%
\pgfsys@useobject{currentmarker}{}%
\end{pgfscope}%
\begin{pgfscope}%
\pgfsys@transformshift{1.023539in}{1.331230in}%
\pgfsys@useobject{currentmarker}{}%
\end{pgfscope}%
\begin{pgfscope}%
\pgfsys@transformshift{1.456182in}{1.472571in}%
\pgfsys@useobject{currentmarker}{}%
\end{pgfscope}%
\begin{pgfscope}%
\pgfsys@transformshift{0.579547in}{1.670007in}%
\pgfsys@useobject{currentmarker}{}%
\end{pgfscope}%
\begin{pgfscope}%
\pgfsys@transformshift{1.012949in}{1.404830in}%
\pgfsys@useobject{currentmarker}{}%
\end{pgfscope}%
\begin{pgfscope}%
\pgfsys@transformshift{1.111293in}{1.351913in}%
\pgfsys@useobject{currentmarker}{}%
\end{pgfscope}%
\begin{pgfscope}%
\pgfsys@transformshift{0.977218in}{1.387930in}%
\pgfsys@useobject{currentmarker}{}%
\end{pgfscope}%
\begin{pgfscope}%
\pgfsys@transformshift{1.304371in}{1.561041in}%
\pgfsys@useobject{currentmarker}{}%
\end{pgfscope}%
\begin{pgfscope}%
\pgfsys@transformshift{1.456756in}{1.620091in}%
\pgfsys@useobject{currentmarker}{}%
\end{pgfscope}%
\begin{pgfscope}%
\pgfsys@transformshift{0.684908in}{1.330366in}%
\pgfsys@useobject{currentmarker}{}%
\end{pgfscope}%
\begin{pgfscope}%
\pgfsys@transformshift{0.997176in}{1.233379in}%
\pgfsys@useobject{currentmarker}{}%
\end{pgfscope}%
\begin{pgfscope}%
\pgfsys@transformshift{1.265659in}{1.626388in}%
\pgfsys@useobject{currentmarker}{}%
\end{pgfscope}%
\begin{pgfscope}%
\pgfsys@transformshift{0.624998in}{1.463073in}%
\pgfsys@useobject{currentmarker}{}%
\end{pgfscope}%
\begin{pgfscope}%
\pgfsys@transformshift{0.771255in}{1.689975in}%
\pgfsys@useobject{currentmarker}{}%
\end{pgfscope}%
\begin{pgfscope}%
\pgfsys@transformshift{0.828869in}{1.517180in}%
\pgfsys@useobject{currentmarker}{}%
\end{pgfscope}%
\begin{pgfscope}%
\pgfsys@transformshift{0.739412in}{1.314509in}%
\pgfsys@useobject{currentmarker}{}%
\end{pgfscope}%
\begin{pgfscope}%
\pgfsys@transformshift{0.873931in}{1.316293in}%
\pgfsys@useobject{currentmarker}{}%
\end{pgfscope}%
\begin{pgfscope}%
\pgfsys@transformshift{0.630219in}{1.463279in}%
\pgfsys@useobject{currentmarker}{}%
\end{pgfscope}%
\begin{pgfscope}%
\pgfsys@transformshift{1.172814in}{1.601195in}%
\pgfsys@useobject{currentmarker}{}%
\end{pgfscope}%
\begin{pgfscope}%
\pgfsys@transformshift{0.647992in}{1.423287in}%
\pgfsys@useobject{currentmarker}{}%
\end{pgfscope}%
\begin{pgfscope}%
\pgfsys@transformshift{1.062251in}{1.169089in}%
\pgfsys@useobject{currentmarker}{}%
\end{pgfscope}%
\begin{pgfscope}%
\pgfsys@transformshift{1.096816in}{1.568874in}%
\pgfsys@useobject{currentmarker}{}%
\end{pgfscope}%
\begin{pgfscope}%
\pgfsys@transformshift{1.149450in}{1.272392in}%
\pgfsys@useobject{currentmarker}{}%
\end{pgfscope}%
\begin{pgfscope}%
\pgfsys@transformshift{0.677632in}{1.518224in}%
\pgfsys@useobject{currentmarker}{}%
\end{pgfscope}%
\begin{pgfscope}%
\pgfsys@transformshift{1.294966in}{1.565715in}%
\pgfsys@useobject{currentmarker}{}%
\end{pgfscope}%
\begin{pgfscope}%
\pgfsys@transformshift{1.468198in}{1.348959in}%
\pgfsys@useobject{currentmarker}{}%
\end{pgfscope}%
\begin{pgfscope}%
\pgfsys@transformshift{0.730525in}{1.486161in}%
\pgfsys@useobject{currentmarker}{}%
\end{pgfscope}%
\begin{pgfscope}%
\pgfsys@transformshift{0.597968in}{1.509409in}%
\pgfsys@useobject{currentmarker}{}%
\end{pgfscope}%
\begin{pgfscope}%
\pgfsys@transformshift{0.539928in}{1.598009in}%
\pgfsys@useobject{currentmarker}{}%
\end{pgfscope}%
\begin{pgfscope}%
\pgfsys@transformshift{0.782197in}{1.710581in}%
\pgfsys@useobject{currentmarker}{}%
\end{pgfscope}%
\begin{pgfscope}%
\pgfsys@transformshift{0.726193in}{1.493681in}%
\pgfsys@useobject{currentmarker}{}%
\end{pgfscope}%
\begin{pgfscope}%
\pgfsys@transformshift{0.756148in}{1.573914in}%
\pgfsys@useobject{currentmarker}{}%
\end{pgfscope}%
\begin{pgfscope}%
\pgfsys@transformshift{0.663543in}{1.546666in}%
\pgfsys@useobject{currentmarker}{}%
\end{pgfscope}%
\begin{pgfscope}%
\pgfsys@transformshift{1.069286in}{1.547220in}%
\pgfsys@useobject{currentmarker}{}%
\end{pgfscope}%
\begin{pgfscope}%
\pgfsys@transformshift{1.411009in}{1.322353in}%
\pgfsys@useobject{currentmarker}{}%
\end{pgfscope}%
\begin{pgfscope}%
\pgfsys@transformshift{1.079839in}{1.579118in}%
\pgfsys@useobject{currentmarker}{}%
\end{pgfscope}%
\begin{pgfscope}%
\pgfsys@transformshift{1.104758in}{1.332416in}%
\pgfsys@useobject{currentmarker}{}%
\end{pgfscope}%
\begin{pgfscope}%
\pgfsys@transformshift{1.128918in}{1.354008in}%
\pgfsys@useobject{currentmarker}{}%
\end{pgfscope}%
\begin{pgfscope}%
\pgfsys@transformshift{1.010728in}{1.396862in}%
\pgfsys@useobject{currentmarker}{}%
\end{pgfscope}%
\begin{pgfscope}%
\pgfsys@transformshift{0.687426in}{1.548919in}%
\pgfsys@useobject{currentmarker}{}%
\end{pgfscope}%
\begin{pgfscope}%
\pgfsys@transformshift{1.166556in}{1.585661in}%
\pgfsys@useobject{currentmarker}{}%
\end{pgfscope}%
\begin{pgfscope}%
\pgfsys@transformshift{0.539336in}{1.735411in}%
\pgfsys@useobject{currentmarker}{}%
\end{pgfscope}%
\begin{pgfscope}%
\pgfsys@transformshift{1.057604in}{1.351352in}%
\pgfsys@useobject{currentmarker}{}%
\end{pgfscope}%
\begin{pgfscope}%
\pgfsys@transformshift{0.759925in}{1.329101in}%
\pgfsys@useobject{currentmarker}{}%
\end{pgfscope}%
\begin{pgfscope}%
\pgfsys@transformshift{1.233927in}{1.177577in}%
\pgfsys@useobject{currentmarker}{}%
\end{pgfscope}%
\begin{pgfscope}%
\pgfsys@transformshift{0.557127in}{1.382970in}%
\pgfsys@useobject{currentmarker}{}%
\end{pgfscope}%
\begin{pgfscope}%
\pgfsys@transformshift{0.685815in}{1.542699in}%
\pgfsys@useobject{currentmarker}{}%
\end{pgfscope}%
\begin{pgfscope}%
\pgfsys@transformshift{0.794138in}{1.127862in}%
\pgfsys@useobject{currentmarker}{}%
\end{pgfscope}%
\begin{pgfscope}%
\pgfsys@transformshift{0.700941in}{1.431860in}%
\pgfsys@useobject{currentmarker}{}%
\end{pgfscope}%
\begin{pgfscope}%
\pgfsys@transformshift{0.901516in}{1.335884in}%
\pgfsys@useobject{currentmarker}{}%
\end{pgfscope}%
\begin{pgfscope}%
\pgfsys@transformshift{0.999546in}{1.386074in}%
\pgfsys@useobject{currentmarker}{}%
\end{pgfscope}%
\begin{pgfscope}%
\pgfsys@transformshift{1.495820in}{1.578396in}%
\pgfsys@useobject{currentmarker}{}%
\end{pgfscope}%
\begin{pgfscope}%
\pgfsys@transformshift{1.570837in}{1.186686in}%
\pgfsys@useobject{currentmarker}{}%
\end{pgfscope}%
\begin{pgfscope}%
\pgfsys@transformshift{0.930379in}{1.580675in}%
\pgfsys@useobject{currentmarker}{}%
\end{pgfscope}%
\begin{pgfscope}%
\pgfsys@transformshift{0.725045in}{1.490011in}%
\pgfsys@useobject{currentmarker}{}%
\end{pgfscope}%
\begin{pgfscope}%
\pgfsys@transformshift{0.663488in}{1.325016in}%
\pgfsys@useobject{currentmarker}{}%
\end{pgfscope}%
\begin{pgfscope}%
\pgfsys@transformshift{0.729081in}{1.476844in}%
\pgfsys@useobject{currentmarker}{}%
\end{pgfscope}%
\begin{pgfscope}%
\pgfsys@transformshift{0.600893in}{1.481853in}%
\pgfsys@useobject{currentmarker}{}%
\end{pgfscope}%
\begin{pgfscope}%
\pgfsys@transformshift{0.769145in}{1.578776in}%
\pgfsys@useobject{currentmarker}{}%
\end{pgfscope}%
\begin{pgfscope}%
\pgfsys@transformshift{0.677817in}{1.433585in}%
\pgfsys@useobject{currentmarker}{}%
\end{pgfscope}%
\begin{pgfscope}%
\pgfsys@transformshift{0.942968in}{1.559671in}%
\pgfsys@useobject{currentmarker}{}%
\end{pgfscope}%
\begin{pgfscope}%
\pgfsys@transformshift{1.032870in}{1.672841in}%
\pgfsys@useobject{currentmarker}{}%
\end{pgfscope}%
\begin{pgfscope}%
\pgfsys@transformshift{0.670412in}{1.669074in}%
\pgfsys@useobject{currentmarker}{}%
\end{pgfscope}%
\begin{pgfscope}%
\pgfsys@transformshift{0.718584in}{1.495060in}%
\pgfsys@useobject{currentmarker}{}%
\end{pgfscope}%
\begin{pgfscope}%
\pgfsys@transformshift{0.609262in}{1.574293in}%
\pgfsys@useobject{currentmarker}{}%
\end{pgfscope}%
\begin{pgfscope}%
\pgfsys@transformshift{0.724620in}{1.508643in}%
\pgfsys@useobject{currentmarker}{}%
\end{pgfscope}%
\begin{pgfscope}%
\pgfsys@transformshift{0.839274in}{1.320706in}%
\pgfsys@useobject{currentmarker}{}%
\end{pgfscope}%
\begin{pgfscope}%
\pgfsys@transformshift{1.123142in}{1.628372in}%
\pgfsys@useobject{currentmarker}{}%
\end{pgfscope}%
\begin{pgfscope}%
\pgfsys@transformshift{0.660581in}{1.350298in}%
\pgfsys@useobject{currentmarker}{}%
\end{pgfscope}%
\begin{pgfscope}%
\pgfsys@transformshift{1.346619in}{1.595214in}%
\pgfsys@useobject{currentmarker}{}%
\end{pgfscope}%
\begin{pgfscope}%
\pgfsys@transformshift{1.032759in}{1.280627in}%
\pgfsys@useobject{currentmarker}{}%
\end{pgfscope}%
\begin{pgfscope}%
\pgfsys@transformshift{1.531144in}{1.311222in}%
\pgfsys@useobject{currentmarker}{}%
\end{pgfscope}%
\begin{pgfscope}%
\pgfsys@transformshift{0.545427in}{1.670053in}%
\pgfsys@useobject{currentmarker}{}%
\end{pgfscope}%
\begin{pgfscope}%
\pgfsys@transformshift{1.530607in}{1.312799in}%
\pgfsys@useobject{currentmarker}{}%
\end{pgfscope}%
\begin{pgfscope}%
\pgfsys@transformshift{1.562080in}{1.328342in}%
\pgfsys@useobject{currentmarker}{}%
\end{pgfscope}%
\begin{pgfscope}%
\pgfsys@transformshift{1.076266in}{1.753638in}%
\pgfsys@useobject{currentmarker}{}%
\end{pgfscope}%
\begin{pgfscope}%
\pgfsys@transformshift{0.596024in}{1.695829in}%
\pgfsys@useobject{currentmarker}{}%
\end{pgfscope}%
\begin{pgfscope}%
\pgfsys@transformshift{1.283932in}{1.177191in}%
\pgfsys@useobject{currentmarker}{}%
\end{pgfscope}%
\begin{pgfscope}%
\pgfsys@transformshift{1.277638in}{1.486463in}%
\pgfsys@useobject{currentmarker}{}%
\end{pgfscope}%
\begin{pgfscope}%
\pgfsys@transformshift{0.915846in}{1.361677in}%
\pgfsys@useobject{currentmarker}{}%
\end{pgfscope}%
\begin{pgfscope}%
\pgfsys@transformshift{0.580084in}{1.486146in}%
\pgfsys@useobject{currentmarker}{}%
\end{pgfscope}%
\begin{pgfscope}%
\pgfsys@transformshift{0.665210in}{1.453766in}%
\pgfsys@useobject{currentmarker}{}%
\end{pgfscope}%
\begin{pgfscope}%
\pgfsys@transformshift{1.282470in}{1.539708in}%
\pgfsys@useobject{currentmarker}{}%
\end{pgfscope}%
\begin{pgfscope}%
\pgfsys@transformshift{0.656693in}{1.618499in}%
\pgfsys@useobject{currentmarker}{}%
\end{pgfscope}%
\begin{pgfscope}%
\pgfsys@transformshift{1.353432in}{1.479516in}%
\pgfsys@useobject{currentmarker}{}%
\end{pgfscope}%
\begin{pgfscope}%
\pgfsys@transformshift{0.684445in}{1.833620in}%
\pgfsys@useobject{currentmarker}{}%
\end{pgfscope}%
\begin{pgfscope}%
\pgfsys@transformshift{1.918484in}{1.321036in}%
\pgfsys@useobject{currentmarker}{}%
\end{pgfscope}%
\begin{pgfscope}%
\pgfsys@transformshift{1.424728in}{1.355280in}%
\pgfsys@useobject{currentmarker}{}%
\end{pgfscope}%
\begin{pgfscope}%
\pgfsys@transformshift{0.746503in}{1.497457in}%
\pgfsys@useobject{currentmarker}{}%
\end{pgfscope}%
\begin{pgfscope}%
\pgfsys@transformshift{0.827814in}{1.622232in}%
\pgfsys@useobject{currentmarker}{}%
\end{pgfscope}%
\begin{pgfscope}%
\pgfsys@transformshift{1.171537in}{1.281287in}%
\pgfsys@useobject{currentmarker}{}%
\end{pgfscope}%
\begin{pgfscope}%
\pgfsys@transformshift{1.125382in}{1.651553in}%
\pgfsys@useobject{currentmarker}{}%
\end{pgfscope}%
\begin{pgfscope}%
\pgfsys@transformshift{1.248979in}{1.514073in}%
\pgfsys@useobject{currentmarker}{}%
\end{pgfscope}%
\begin{pgfscope}%
\pgfsys@transformshift{0.729433in}{1.638069in}%
\pgfsys@useobject{currentmarker}{}%
\end{pgfscope}%
\begin{pgfscope}%
\pgfsys@transformshift{0.585879in}{1.558038in}%
\pgfsys@useobject{currentmarker}{}%
\end{pgfscope}%
\begin{pgfscope}%
\pgfsys@transformshift{1.040794in}{1.681330in}%
\pgfsys@useobject{currentmarker}{}%
\end{pgfscope}%
\begin{pgfscope}%
\pgfsys@transformshift{0.953003in}{1.614894in}%
\pgfsys@useobject{currentmarker}{}%
\end{pgfscope}%
\begin{pgfscope}%
\pgfsys@transformshift{1.076747in}{1.315101in}%
\pgfsys@useobject{currentmarker}{}%
\end{pgfscope}%
\begin{pgfscope}%
\pgfsys@transformshift{0.880318in}{1.660049in}%
\pgfsys@useobject{currentmarker}{}%
\end{pgfscope}%
\begin{pgfscope}%
\pgfsys@transformshift{1.223782in}{1.679288in}%
\pgfsys@useobject{currentmarker}{}%
\end{pgfscope}%
\begin{pgfscope}%
\pgfsys@transformshift{1.089466in}{1.647246in}%
\pgfsys@useobject{currentmarker}{}%
\end{pgfscope}%
\begin{pgfscope}%
\pgfsys@transformshift{0.826666in}{1.540509in}%
\pgfsys@useobject{currentmarker}{}%
\end{pgfscope}%
\begin{pgfscope}%
\pgfsys@transformshift{1.377851in}{1.491491in}%
\pgfsys@useobject{currentmarker}{}%
\end{pgfscope}%
\begin{pgfscope}%
\pgfsys@transformshift{0.598024in}{1.540398in}%
\pgfsys@useobject{currentmarker}{}%
\end{pgfscope}%
\begin{pgfscope}%
\pgfsys@transformshift{0.697849in}{1.351446in}%
\pgfsys@useobject{currentmarker}{}%
\end{pgfscope}%
\begin{pgfscope}%
\pgfsys@transformshift{1.266270in}{1.523656in}%
\pgfsys@useobject{currentmarker}{}%
\end{pgfscope}%
\begin{pgfscope}%
\pgfsys@transformshift{0.769293in}{1.428945in}%
\pgfsys@useobject{currentmarker}{}%
\end{pgfscope}%
\begin{pgfscope}%
\pgfsys@transformshift{0.763591in}{1.325630in}%
\pgfsys@useobject{currentmarker}{}%
\end{pgfscope}%
\begin{pgfscope}%
\pgfsys@transformshift{1.545270in}{1.301283in}%
\pgfsys@useobject{currentmarker}{}%
\end{pgfscope}%
\begin{pgfscope}%
\pgfsys@transformshift{0.781974in}{1.323716in}%
\pgfsys@useobject{currentmarker}{}%
\end{pgfscope}%
\begin{pgfscope}%
\pgfsys@transformshift{0.637143in}{1.491760in}%
\pgfsys@useobject{currentmarker}{}%
\end{pgfscope}%
\begin{pgfscope}%
\pgfsys@transformshift{1.059196in}{1.398695in}%
\pgfsys@useobject{currentmarker}{}%
\end{pgfscope}%
\begin{pgfscope}%
\pgfsys@transformshift{1.032796in}{1.730356in}%
\pgfsys@useobject{currentmarker}{}%
\end{pgfscope}%
\begin{pgfscope}%
\pgfsys@transformshift{0.683742in}{1.610344in}%
\pgfsys@useobject{currentmarker}{}%
\end{pgfscope}%
\begin{pgfscope}%
\pgfsys@transformshift{1.021114in}{1.628661in}%
\pgfsys@useobject{currentmarker}{}%
\end{pgfscope}%
\begin{pgfscope}%
\pgfsys@transformshift{0.563237in}{1.594996in}%
\pgfsys@useobject{currentmarker}{}%
\end{pgfscope}%
\begin{pgfscope}%
\pgfsys@transformshift{0.614223in}{1.538143in}%
\pgfsys@useobject{currentmarker}{}%
\end{pgfscope}%
\begin{pgfscope}%
\pgfsys@transformshift{0.779401in}{1.500790in}%
\pgfsys@useobject{currentmarker}{}%
\end{pgfscope}%
\begin{pgfscope}%
\pgfsys@transformshift{0.992603in}{1.651086in}%
\pgfsys@useobject{currentmarker}{}%
\end{pgfscope}%
\begin{pgfscope}%
\pgfsys@transformshift{1.694285in}{1.199216in}%
\pgfsys@useobject{currentmarker}{}%
\end{pgfscope}%
\begin{pgfscope}%
\pgfsys@transformshift{0.778624in}{1.268459in}%
\pgfsys@useobject{currentmarker}{}%
\end{pgfscope}%
\begin{pgfscope}%
\pgfsys@transformshift{1.340343in}{1.518494in}%
\pgfsys@useobject{currentmarker}{}%
\end{pgfscope}%
\begin{pgfscope}%
\pgfsys@transformshift{0.923566in}{1.562317in}%
\pgfsys@useobject{currentmarker}{}%
\end{pgfscope}%
\begin{pgfscope}%
\pgfsys@transformshift{0.789621in}{1.075973in}%
\pgfsys@useobject{currentmarker}{}%
\end{pgfscope}%
\begin{pgfscope}%
\pgfsys@transformshift{0.869895in}{1.172639in}%
\pgfsys@useobject{currentmarker}{}%
\end{pgfscope}%
\begin{pgfscope}%
\pgfsys@transformshift{1.119513in}{1.238914in}%
\pgfsys@useobject{currentmarker}{}%
\end{pgfscope}%
\begin{pgfscope}%
\pgfsys@transformshift{1.521480in}{1.594238in}%
\pgfsys@useobject{currentmarker}{}%
\end{pgfscope}%
\begin{pgfscope}%
\pgfsys@transformshift{1.492025in}{1.500156in}%
\pgfsys@useobject{currentmarker}{}%
\end{pgfscope}%
\begin{pgfscope}%
\pgfsys@transformshift{1.103758in}{1.821461in}%
\pgfsys@useobject{currentmarker}{}%
\end{pgfscope}%
\begin{pgfscope}%
\pgfsys@transformshift{0.698127in}{1.441492in}%
\pgfsys@useobject{currentmarker}{}%
\end{pgfscope}%
\begin{pgfscope}%
\pgfsys@transformshift{1.029871in}{1.431536in}%
\pgfsys@useobject{currentmarker}{}%
\end{pgfscope}%
\begin{pgfscope}%
\pgfsys@transformshift{0.931582in}{1.622530in}%
\pgfsys@useobject{currentmarker}{}%
\end{pgfscope}%
\begin{pgfscope}%
\pgfsys@transformshift{0.910792in}{1.417631in}%
\pgfsys@useobject{currentmarker}{}%
\end{pgfscope}%
\begin{pgfscope}%
\pgfsys@transformshift{0.605355in}{1.481064in}%
\pgfsys@useobject{currentmarker}{}%
\end{pgfscope}%
\begin{pgfscope}%
\pgfsys@transformshift{1.054013in}{1.663624in}%
\pgfsys@useobject{currentmarker}{}%
\end{pgfscope}%
\begin{pgfscope}%
\pgfsys@transformshift{1.098612in}{1.206548in}%
\pgfsys@useobject{currentmarker}{}%
\end{pgfscope}%
\begin{pgfscope}%
\pgfsys@transformshift{0.861509in}{1.555466in}%
\pgfsys@useobject{currentmarker}{}%
\end{pgfscope}%
\begin{pgfscope}%
\pgfsys@transformshift{0.582121in}{1.759242in}%
\pgfsys@useobject{currentmarker}{}%
\end{pgfscope}%
\begin{pgfscope}%
\pgfsys@transformshift{1.167241in}{1.218578in}%
\pgfsys@useobject{currentmarker}{}%
\end{pgfscope}%
\begin{pgfscope}%
\pgfsys@transformshift{1.281729in}{1.521877in}%
\pgfsys@useobject{currentmarker}{}%
\end{pgfscope}%
\begin{pgfscope}%
\pgfsys@transformshift{1.375426in}{1.599626in}%
\pgfsys@useobject{currentmarker}{}%
\end{pgfscope}%
\begin{pgfscope}%
\pgfsys@transformshift{1.374241in}{1.535851in}%
\pgfsys@useobject{currentmarker}{}%
\end{pgfscope}%
\begin{pgfscope}%
\pgfsys@transformshift{0.897036in}{1.455475in}%
\pgfsys@useobject{currentmarker}{}%
\end{pgfscope}%
\begin{pgfscope}%
\pgfsys@transformshift{1.073507in}{1.623030in}%
\pgfsys@useobject{currentmarker}{}%
\end{pgfscope}%
\begin{pgfscope}%
\pgfsys@transformshift{0.689259in}{1.672359in}%
\pgfsys@useobject{currentmarker}{}%
\end{pgfscope}%
\begin{pgfscope}%
\pgfsys@transformshift{1.117181in}{1.267683in}%
\pgfsys@useobject{currentmarker}{}%
\end{pgfscope}%
\begin{pgfscope}%
\pgfsys@transformshift{1.040257in}{1.655098in}%
\pgfsys@useobject{currentmarker}{}%
\end{pgfscope}%
\begin{pgfscope}%
\pgfsys@transformshift{0.757259in}{1.731190in}%
\pgfsys@useobject{currentmarker}{}%
\end{pgfscope}%
\begin{pgfscope}%
\pgfsys@transformshift{0.929602in}{1.779014in}%
\pgfsys@useobject{currentmarker}{}%
\end{pgfscope}%
\begin{pgfscope}%
\pgfsys@transformshift{0.739375in}{1.257335in}%
\pgfsys@useobject{currentmarker}{}%
\end{pgfscope}%
\begin{pgfscope}%
\pgfsys@transformshift{1.051791in}{1.159574in}%
\pgfsys@useobject{currentmarker}{}%
\end{pgfscope}%
\begin{pgfscope}%
\pgfsys@transformshift{1.033888in}{1.429292in}%
\pgfsys@useobject{currentmarker}{}%
\end{pgfscope}%
\begin{pgfscope}%
\pgfsys@transformshift{1.087559in}{1.175687in}%
\pgfsys@useobject{currentmarker}{}%
\end{pgfscope}%
\begin{pgfscope}%
\pgfsys@transformshift{1.159095in}{1.277350in}%
\pgfsys@useobject{currentmarker}{}%
\end{pgfscope}%
\begin{pgfscope}%
\pgfsys@transformshift{0.813373in}{1.592666in}%
\pgfsys@useobject{currentmarker}{}%
\end{pgfscope}%
\begin{pgfscope}%
\pgfsys@transformshift{1.640577in}{1.328602in}%
\pgfsys@useobject{currentmarker}{}%
\end{pgfscope}%
\begin{pgfscope}%
\pgfsys@transformshift{2.178211in}{1.301110in}%
\pgfsys@useobject{currentmarker}{}%
\end{pgfscope}%
\begin{pgfscope}%
\pgfsys@transformshift{0.693850in}{1.552467in}%
\pgfsys@useobject{currentmarker}{}%
\end{pgfscope}%
\begin{pgfscope}%
\pgfsys@transformshift{1.166519in}{1.213004in}%
\pgfsys@useobject{currentmarker}{}%
\end{pgfscope}%
\begin{pgfscope}%
\pgfsys@transformshift{0.643734in}{1.461384in}%
\pgfsys@useobject{currentmarker}{}%
\end{pgfscope}%
\begin{pgfscope}%
\pgfsys@transformshift{0.891278in}{1.482282in}%
\pgfsys@useobject{currentmarker}{}%
\end{pgfscope}%
\begin{pgfscope}%
\pgfsys@transformshift{0.789639in}{1.266427in}%
\pgfsys@useobject{currentmarker}{}%
\end{pgfscope}%
\begin{pgfscope}%
\pgfsys@transformshift{0.905830in}{1.466930in}%
\pgfsys@useobject{currentmarker}{}%
\end{pgfscope}%
\begin{pgfscope}%
\pgfsys@transformshift{1.170666in}{1.600607in}%
\pgfsys@useobject{currentmarker}{}%
\end{pgfscope}%
\begin{pgfscope}%
\pgfsys@transformshift{0.851123in}{1.594797in}%
\pgfsys@useobject{currentmarker}{}%
\end{pgfscope}%
\begin{pgfscope}%
\pgfsys@transformshift{1.511890in}{1.628562in}%
\pgfsys@useobject{currentmarker}{}%
\end{pgfscope}%
\begin{pgfscope}%
\pgfsys@transformshift{1.126178in}{1.300758in}%
\pgfsys@useobject{currentmarker}{}%
\end{pgfscope}%
\begin{pgfscope}%
\pgfsys@transformshift{0.794434in}{1.598346in}%
\pgfsys@useobject{currentmarker}{}%
\end{pgfscope}%
\begin{pgfscope}%
\pgfsys@transformshift{1.144285in}{1.218628in}%
\pgfsys@useobject{currentmarker}{}%
\end{pgfscope}%
\begin{pgfscope}%
\pgfsys@transformshift{0.668153in}{1.537795in}%
\pgfsys@useobject{currentmarker}{}%
\end{pgfscope}%
\begin{pgfscope}%
\pgfsys@transformshift{0.919012in}{1.447529in}%
\pgfsys@useobject{currentmarker}{}%
\end{pgfscope}%
\begin{pgfscope}%
\pgfsys@transformshift{1.167260in}{1.580789in}%
\pgfsys@useobject{currentmarker}{}%
\end{pgfscope}%
\begin{pgfscope}%
\pgfsys@transformshift{0.990696in}{1.459936in}%
\pgfsys@useobject{currentmarker}{}%
\end{pgfscope}%
\begin{pgfscope}%
\pgfsys@transformshift{0.884725in}{1.289918in}%
\pgfsys@useobject{currentmarker}{}%
\end{pgfscope}%
\begin{pgfscope}%
\pgfsys@transformshift{0.825592in}{1.437057in}%
\pgfsys@useobject{currentmarker}{}%
\end{pgfscope}%
\begin{pgfscope}%
\pgfsys@transformshift{0.824667in}{1.365444in}%
\pgfsys@useobject{currentmarker}{}%
\end{pgfscope}%
\begin{pgfscope}%
\pgfsys@transformshift{1.211526in}{1.602725in}%
\pgfsys@useobject{currentmarker}{}%
\end{pgfscope}%
\begin{pgfscope}%
\pgfsys@transformshift{1.235205in}{1.536158in}%
\pgfsys@useobject{currentmarker}{}%
\end{pgfscope}%
\begin{pgfscope}%
\pgfsys@transformshift{1.055049in}{1.520210in}%
\pgfsys@useobject{currentmarker}{}%
\end{pgfscope}%
\begin{pgfscope}%
\pgfsys@transformshift{1.121531in}{1.669672in}%
\pgfsys@useobject{currentmarker}{}%
\end{pgfscope}%
\begin{pgfscope}%
\pgfsys@transformshift{1.293522in}{1.165388in}%
\pgfsys@useobject{currentmarker}{}%
\end{pgfscope}%
\begin{pgfscope}%
\pgfsys@transformshift{0.598931in}{1.399089in}%
\pgfsys@useobject{currentmarker}{}%
\end{pgfscope}%
\begin{pgfscope}%
\pgfsys@transformshift{0.969776in}{1.417639in}%
\pgfsys@useobject{currentmarker}{}%
\end{pgfscope}%
\begin{pgfscope}%
\pgfsys@transformshift{1.073637in}{1.474004in}%
\pgfsys@useobject{currentmarker}{}%
\end{pgfscope}%
\begin{pgfscope}%
\pgfsys@transformshift{1.012061in}{1.658132in}%
\pgfsys@useobject{currentmarker}{}%
\end{pgfscope}%
\begin{pgfscope}%
\pgfsys@transformshift{0.981662in}{1.379117in}%
\pgfsys@useobject{currentmarker}{}%
\end{pgfscope}%
\begin{pgfscope}%
\pgfsys@transformshift{0.639994in}{1.485430in}%
\pgfsys@useobject{currentmarker}{}%
\end{pgfscope}%
\begin{pgfscope}%
\pgfsys@transformshift{0.958131in}{1.551781in}%
\pgfsys@useobject{currentmarker}{}%
\end{pgfscope}%
\begin{pgfscope}%
\pgfsys@transformshift{0.822908in}{1.309472in}%
\pgfsys@useobject{currentmarker}{}%
\end{pgfscope}%
\begin{pgfscope}%
\pgfsys@transformshift{0.720176in}{1.258017in}%
\pgfsys@useobject{currentmarker}{}%
\end{pgfscope}%
\begin{pgfscope}%
\pgfsys@transformshift{0.636847in}{1.415344in}%
\pgfsys@useobject{currentmarker}{}%
\end{pgfscope}%
\begin{pgfscope}%
\pgfsys@transformshift{0.850863in}{1.487305in}%
\pgfsys@useobject{currentmarker}{}%
\end{pgfscope}%
\begin{pgfscope}%
\pgfsys@transformshift{1.514574in}{1.546546in}%
\pgfsys@useobject{currentmarker}{}%
\end{pgfscope}%
\begin{pgfscope}%
\pgfsys@transformshift{0.970535in}{1.558277in}%
\pgfsys@useobject{currentmarker}{}%
\end{pgfscope}%
\begin{pgfscope}%
\pgfsys@transformshift{0.976256in}{1.672225in}%
\pgfsys@useobject{currentmarker}{}%
\end{pgfscope}%
\begin{pgfscope}%
\pgfsys@transformshift{0.879633in}{1.707234in}%
\pgfsys@useobject{currentmarker}{}%
\end{pgfscope}%
\begin{pgfscope}%
\pgfsys@transformshift{0.655508in}{1.342615in}%
\pgfsys@useobject{currentmarker}{}%
\end{pgfscope}%
\begin{pgfscope}%
\pgfsys@transformshift{0.702514in}{1.463721in}%
\pgfsys@useobject{currentmarker}{}%
\end{pgfscope}%
\begin{pgfscope}%
\pgfsys@transformshift{0.714474in}{1.312110in}%
\pgfsys@useobject{currentmarker}{}%
\end{pgfscope}%
\begin{pgfscope}%
\pgfsys@transformshift{1.412398in}{1.349740in}%
\pgfsys@useobject{currentmarker}{}%
\end{pgfscope}%
\begin{pgfscope}%
\pgfsys@transformshift{1.041442in}{1.634732in}%
\pgfsys@useobject{currentmarker}{}%
\end{pgfscope}%
\begin{pgfscope}%
\pgfsys@transformshift{0.890575in}{1.506102in}%
\pgfsys@useobject{currentmarker}{}%
\end{pgfscope}%
\begin{pgfscope}%
\pgfsys@transformshift{1.229854in}{1.286426in}%
\pgfsys@useobject{currentmarker}{}%
\end{pgfscope}%
\begin{pgfscope}%
\pgfsys@transformshift{1.163502in}{1.568385in}%
\pgfsys@useobject{currentmarker}{}%
\end{pgfscope}%
\begin{pgfscope}%
\pgfsys@transformshift{0.904941in}{1.673563in}%
\pgfsys@useobject{currentmarker}{}%
\end{pgfscope}%
\begin{pgfscope}%
\pgfsys@transformshift{0.963778in}{1.752002in}%
\pgfsys@useobject{currentmarker}{}%
\end{pgfscope}%
\begin{pgfscope}%
\pgfsys@transformshift{1.162317in}{1.309283in}%
\pgfsys@useobject{currentmarker}{}%
\end{pgfscope}%
\begin{pgfscope}%
\pgfsys@transformshift{0.708402in}{1.299256in}%
\pgfsys@useobject{currentmarker}{}%
\end{pgfscope}%
\begin{pgfscope}%
\pgfsys@transformshift{0.704162in}{1.577671in}%
\pgfsys@useobject{currentmarker}{}%
\end{pgfscope}%
\begin{pgfscope}%
\pgfsys@transformshift{1.154782in}{1.133107in}%
\pgfsys@useobject{currentmarker}{}%
\end{pgfscope}%
\begin{pgfscope}%
\pgfsys@transformshift{0.978699in}{1.287790in}%
\pgfsys@useobject{currentmarker}{}%
\end{pgfscope}%
\begin{pgfscope}%
\pgfsys@transformshift{1.451036in}{1.631414in}%
\pgfsys@useobject{currentmarker}{}%
\end{pgfscope}%
\begin{pgfscope}%
\pgfsys@transformshift{1.034425in}{1.650487in}%
\pgfsys@useobject{currentmarker}{}%
\end{pgfscope}%
\begin{pgfscope}%
\pgfsys@transformshift{0.615926in}{1.529413in}%
\pgfsys@useobject{currentmarker}{}%
\end{pgfscope}%
\begin{pgfscope}%
\pgfsys@transformshift{1.270158in}{1.451365in}%
\pgfsys@useobject{currentmarker}{}%
\end{pgfscope}%
\begin{pgfscope}%
\pgfsys@transformshift{0.708883in}{1.318019in}%
\pgfsys@useobject{currentmarker}{}%
\end{pgfscope}%
\begin{pgfscope}%
\pgfsys@transformshift{0.740763in}{1.248162in}%
\pgfsys@useobject{currentmarker}{}%
\end{pgfscope}%
\begin{pgfscope}%
\pgfsys@transformshift{0.614020in}{1.419310in}%
\pgfsys@useobject{currentmarker}{}%
\end{pgfscope}%
\begin{pgfscope}%
\pgfsys@transformshift{1.067953in}{1.547865in}%
\pgfsys@useobject{currentmarker}{}%
\end{pgfscope}%
\begin{pgfscope}%
\pgfsys@transformshift{0.777420in}{1.486648in}%
\pgfsys@useobject{currentmarker}{}%
\end{pgfscope}%
\begin{pgfscope}%
\pgfsys@transformshift{0.653398in}{1.446792in}%
\pgfsys@useobject{currentmarker}{}%
\end{pgfscope}%
\begin{pgfscope}%
\pgfsys@transformshift{0.733321in}{1.468190in}%
\pgfsys@useobject{currentmarker}{}%
\end{pgfscope}%
\begin{pgfscope}%
\pgfsys@transformshift{0.711753in}{1.591010in}%
\pgfsys@useobject{currentmarker}{}%
\end{pgfscope}%
\begin{pgfscope}%
\pgfsys@transformshift{1.082709in}{1.714735in}%
\pgfsys@useobject{currentmarker}{}%
\end{pgfscope}%
\begin{pgfscope}%
\pgfsys@transformshift{1.677845in}{1.231356in}%
\pgfsys@useobject{currentmarker}{}%
\end{pgfscope}%
\begin{pgfscope}%
\pgfsys@transformshift{0.696497in}{1.437481in}%
\pgfsys@useobject{currentmarker}{}%
\end{pgfscope}%
\begin{pgfscope}%
\pgfsys@transformshift{0.757129in}{1.425695in}%
\pgfsys@useobject{currentmarker}{}%
\end{pgfscope}%
\begin{pgfscope}%
\pgfsys@transformshift{1.072767in}{1.747827in}%
\pgfsys@useobject{currentmarker}{}%
\end{pgfscope}%
\begin{pgfscope}%
\pgfsys@transformshift{1.280452in}{1.670157in}%
\pgfsys@useobject{currentmarker}{}%
\end{pgfscope}%
\begin{pgfscope}%
\pgfsys@transformshift{0.830258in}{1.389160in}%
\pgfsys@useobject{currentmarker}{}%
\end{pgfscope}%
\begin{pgfscope}%
\pgfsys@transformshift{0.709420in}{1.248089in}%
\pgfsys@useobject{currentmarker}{}%
\end{pgfscope}%
\begin{pgfscope}%
\pgfsys@transformshift{1.313295in}{1.514234in}%
\pgfsys@useobject{currentmarker}{}%
\end{pgfscope}%
\begin{pgfscope}%
\pgfsys@transformshift{0.966851in}{1.571578in}%
\pgfsys@useobject{currentmarker}{}%
\end{pgfscope}%
\begin{pgfscope}%
\pgfsys@transformshift{0.979088in}{1.720034in}%
\pgfsys@useobject{currentmarker}{}%
\end{pgfscope}%
\begin{pgfscope}%
\pgfsys@transformshift{0.852955in}{1.275966in}%
\pgfsys@useobject{currentmarker}{}%
\end{pgfscope}%
\begin{pgfscope}%
\pgfsys@transformshift{0.752519in}{1.402857in}%
\pgfsys@useobject{currentmarker}{}%
\end{pgfscope}%
\begin{pgfscope}%
\pgfsys@transformshift{0.942839in}{1.596340in}%
\pgfsys@useobject{currentmarker}{}%
\end{pgfscope}%
\begin{pgfscope}%
\pgfsys@transformshift{0.747724in}{1.578590in}%
\pgfsys@useobject{currentmarker}{}%
\end{pgfscope}%
\begin{pgfscope}%
\pgfsys@transformshift{1.116200in}{1.624398in}%
\pgfsys@useobject{currentmarker}{}%
\end{pgfscope}%
\begin{pgfscope}%
\pgfsys@transformshift{0.695424in}{1.307437in}%
\pgfsys@useobject{currentmarker}{}%
\end{pgfscope}%
\begin{pgfscope}%
\pgfsys@transformshift{1.740661in}{1.510839in}%
\pgfsys@useobject{currentmarker}{}%
\end{pgfscope}%
\begin{pgfscope}%
\pgfsys@transformshift{0.688629in}{1.615126in}%
\pgfsys@useobject{currentmarker}{}%
\end{pgfscope}%
\begin{pgfscope}%
\pgfsys@transformshift{0.892574in}{1.775766in}%
\pgfsys@useobject{currentmarker}{}%
\end{pgfscope}%
\begin{pgfscope}%
\pgfsys@transformshift{0.811004in}{1.427280in}%
\pgfsys@useobject{currentmarker}{}%
\end{pgfscope}%
\begin{pgfscope}%
\pgfsys@transformshift{1.098889in}{1.634936in}%
\pgfsys@useobject{currentmarker}{}%
\end{pgfscope}%
\begin{pgfscope}%
\pgfsys@transformshift{0.812874in}{1.262689in}%
\pgfsys@useobject{currentmarker}{}%
\end{pgfscope}%
\begin{pgfscope}%
\pgfsys@transformshift{0.678317in}{1.559973in}%
\pgfsys@useobject{currentmarker}{}%
\end{pgfscope}%
\begin{pgfscope}%
\pgfsys@transformshift{1.330735in}{1.288677in}%
\pgfsys@useobject{currentmarker}{}%
\end{pgfscope}%
\begin{pgfscope}%
\pgfsys@transformshift{0.845531in}{1.220898in}%
\pgfsys@useobject{currentmarker}{}%
\end{pgfscope}%
\begin{pgfscope}%
\pgfsys@transformshift{1.154911in}{1.593640in}%
\pgfsys@useobject{currentmarker}{}%
\end{pgfscope}%
\begin{pgfscope}%
\pgfsys@transformshift{0.984050in}{1.934359in}%
\pgfsys@useobject{currentmarker}{}%
\end{pgfscope}%
\begin{pgfscope}%
\pgfsys@transformshift{0.736913in}{1.496536in}%
\pgfsys@useobject{currentmarker}{}%
\end{pgfscope}%
\begin{pgfscope}%
\pgfsys@transformshift{0.696775in}{1.329512in}%
\pgfsys@useobject{currentmarker}{}%
\end{pgfscope}%
\begin{pgfscope}%
\pgfsys@transformshift{0.594914in}{1.524174in}%
\pgfsys@useobject{currentmarker}{}%
\end{pgfscope}%
\begin{pgfscope}%
\pgfsys@transformshift{1.029056in}{1.426343in}%
\pgfsys@useobject{currentmarker}{}%
\end{pgfscope}%
\begin{pgfscope}%
\pgfsys@transformshift{1.353377in}{1.593265in}%
\pgfsys@useobject{currentmarker}{}%
\end{pgfscope}%
\begin{pgfscope}%
\pgfsys@transformshift{0.844328in}{1.585731in}%
\pgfsys@useobject{currentmarker}{}%
\end{pgfscope}%
\begin{pgfscope}%
\pgfsys@transformshift{0.897277in}{1.579642in}%
\pgfsys@useobject{currentmarker}{}%
\end{pgfscope}%
\begin{pgfscope}%
\pgfsys@transformshift{0.995325in}{1.378541in}%
\pgfsys@useobject{currentmarker}{}%
\end{pgfscope}%
\begin{pgfscope}%
\pgfsys@transformshift{1.056327in}{1.396243in}%
\pgfsys@useobject{currentmarker}{}%
\end{pgfscope}%
\begin{pgfscope}%
\pgfsys@transformshift{0.564107in}{1.487138in}%
\pgfsys@useobject{currentmarker}{}%
\end{pgfscope}%
\begin{pgfscope}%
\pgfsys@transformshift{0.656175in}{1.526149in}%
\pgfsys@useobject{currentmarker}{}%
\end{pgfscope}%
\begin{pgfscope}%
\pgfsys@transformshift{1.144433in}{1.348205in}%
\pgfsys@useobject{currentmarker}{}%
\end{pgfscope}%
\begin{pgfscope}%
\pgfsys@transformshift{0.628238in}{1.452682in}%
\pgfsys@useobject{currentmarker}{}%
\end{pgfscope}%
\begin{pgfscope}%
\pgfsys@transformshift{1.040257in}{1.399229in}%
\pgfsys@useobject{currentmarker}{}%
\end{pgfscope}%
\begin{pgfscope}%
\pgfsys@transformshift{0.967406in}{1.499205in}%
\pgfsys@useobject{currentmarker}{}%
\end{pgfscope}%
\begin{pgfscope}%
\pgfsys@transformshift{1.329216in}{1.558350in}%
\pgfsys@useobject{currentmarker}{}%
\end{pgfscope}%
\begin{pgfscope}%
\pgfsys@transformshift{0.564607in}{1.532721in}%
\pgfsys@useobject{currentmarker}{}%
\end{pgfscope}%
\begin{pgfscope}%
\pgfsys@transformshift{0.742263in}{1.589253in}%
\pgfsys@useobject{currentmarker}{}%
\end{pgfscope}%
\begin{pgfscope}%
\pgfsys@transformshift{0.629367in}{1.540946in}%
\pgfsys@useobject{currentmarker}{}%
\end{pgfscope}%
\begin{pgfscope}%
\pgfsys@transformshift{0.569846in}{1.336325in}%
\pgfsys@useobject{currentmarker}{}%
\end{pgfscope}%
\begin{pgfscope}%
\pgfsys@transformshift{1.470660in}{1.424778in}%
\pgfsys@useobject{currentmarker}{}%
\end{pgfscope}%
\begin{pgfscope}%
\pgfsys@transformshift{1.976543in}{1.216384in}%
\pgfsys@useobject{currentmarker}{}%
\end{pgfscope}%
\begin{pgfscope}%
\pgfsys@transformshift{1.014486in}{1.648903in}%
\pgfsys@useobject{currentmarker}{}%
\end{pgfscope}%
\begin{pgfscope}%
\pgfsys@transformshift{1.379055in}{1.517635in}%
\pgfsys@useobject{currentmarker}{}%
\end{pgfscope}%
\begin{pgfscope}%
\pgfsys@transformshift{0.658063in}{1.704990in}%
\pgfsys@useobject{currentmarker}{}%
\end{pgfscope}%
\begin{pgfscope}%
\pgfsys@transformshift{0.965129in}{1.696397in}%
\pgfsys@useobject{currentmarker}{}%
\end{pgfscope}%
\begin{pgfscope}%
\pgfsys@transformshift{0.714733in}{1.481400in}%
\pgfsys@useobject{currentmarker}{}%
\end{pgfscope}%
\begin{pgfscope}%
\pgfsys@transformshift{0.734580in}{1.693057in}%
\pgfsys@useobject{currentmarker}{}%
\end{pgfscope}%
\begin{pgfscope}%
\pgfsys@transformshift{1.146876in}{1.313335in}%
\pgfsys@useobject{currentmarker}{}%
\end{pgfscope}%
\begin{pgfscope}%
\pgfsys@transformshift{1.112552in}{1.590133in}%
\pgfsys@useobject{currentmarker}{}%
\end{pgfscope}%
\begin{pgfscope}%
\pgfsys@transformshift{0.609724in}{1.486687in}%
\pgfsys@useobject{currentmarker}{}%
\end{pgfscope}%
\begin{pgfscope}%
\pgfsys@transformshift{0.713974in}{1.471279in}%
\pgfsys@useobject{currentmarker}{}%
\end{pgfscope}%
\begin{pgfscope}%
\pgfsys@transformshift{0.625054in}{1.358405in}%
\pgfsys@useobject{currentmarker}{}%
\end{pgfscope}%
\begin{pgfscope}%
\pgfsys@transformshift{0.695701in}{1.551270in}%
\pgfsys@useobject{currentmarker}{}%
\end{pgfscope}%
\begin{pgfscope}%
\pgfsys@transformshift{0.630052in}{1.065636in}%
\pgfsys@useobject{currentmarker}{}%
\end{pgfscope}%
\begin{pgfscope}%
\pgfsys@transformshift{1.032148in}{1.341972in}%
\pgfsys@useobject{currentmarker}{}%
\end{pgfscope}%
\begin{pgfscope}%
\pgfsys@transformshift{0.974293in}{1.746428in}%
\pgfsys@useobject{currentmarker}{}%
\end{pgfscope}%
\begin{pgfscope}%
\pgfsys@transformshift{0.679965in}{1.386135in}%
\pgfsys@useobject{currentmarker}{}%
\end{pgfscope}%
\begin{pgfscope}%
\pgfsys@transformshift{1.437521in}{1.468851in}%
\pgfsys@useobject{currentmarker}{}%
\end{pgfscope}%
\begin{pgfscope}%
\pgfsys@transformshift{0.779957in}{1.405308in}%
\pgfsys@useobject{currentmarker}{}%
\end{pgfscope}%
\begin{pgfscope}%
\pgfsys@transformshift{1.260216in}{1.479728in}%
\pgfsys@useobject{currentmarker}{}%
\end{pgfscope}%
\begin{pgfscope}%
\pgfsys@transformshift{0.783159in}{1.455536in}%
\pgfsys@useobject{currentmarker}{}%
\end{pgfscope}%
\begin{pgfscope}%
\pgfsys@transformshift{0.627534in}{1.635761in}%
\pgfsys@useobject{currentmarker}{}%
\end{pgfscope}%
\begin{pgfscope}%
\pgfsys@transformshift{0.732803in}{1.296942in}%
\pgfsys@useobject{currentmarker}{}%
\end{pgfscope}%
\begin{pgfscope}%
\pgfsys@transformshift{1.221153in}{1.523689in}%
\pgfsys@useobject{currentmarker}{}%
\end{pgfscope}%
\begin{pgfscope}%
\pgfsys@transformshift{0.997343in}{1.409197in}%
\pgfsys@useobject{currentmarker}{}%
\end{pgfscope}%
\begin{pgfscope}%
\pgfsys@transformshift{0.699904in}{1.539602in}%
\pgfsys@useobject{currentmarker}{}%
\end{pgfscope}%
\begin{pgfscope}%
\pgfsys@transformshift{0.739856in}{1.432591in}%
\pgfsys@useobject{currentmarker}{}%
\end{pgfscope}%
\begin{pgfscope}%
\pgfsys@transformshift{0.561941in}{1.742438in}%
\pgfsys@useobject{currentmarker}{}%
\end{pgfscope}%
\begin{pgfscope}%
\pgfsys@transformshift{1.009173in}{1.355671in}%
\pgfsys@useobject{currentmarker}{}%
\end{pgfscope}%
\begin{pgfscope}%
\pgfsys@transformshift{0.728952in}{1.402567in}%
\pgfsys@useobject{currentmarker}{}%
\end{pgfscope}%
\begin{pgfscope}%
\pgfsys@transformshift{1.463440in}{1.599019in}%
\pgfsys@useobject{currentmarker}{}%
\end{pgfscope}%
\begin{pgfscope}%
\pgfsys@transformshift{0.638106in}{1.266372in}%
\pgfsys@useobject{currentmarker}{}%
\end{pgfscope}%
\begin{pgfscope}%
\pgfsys@transformshift{0.986864in}{1.741033in}%
\pgfsys@useobject{currentmarker}{}%
\end{pgfscope}%
\end{pgfscope}%
\begin{pgfscope}%
\pgfpathrectangle{\pgfqpoint{0.526284in}{0.473557in}}{\pgfqpoint{1.651927in}{1.704653in}}%
\pgfusepath{clip}%
\pgfsetbuttcap%
\pgfsetroundjoin%
\definecolor{currentfill}{rgb}{0.172549,0.627451,0.172549}%
\pgfsetfillcolor{currentfill}%
\pgfsetfillopacity{0.150000}%
\pgfsetlinewidth{0.000000pt}%
\definecolor{currentstroke}{rgb}{0.000000,0.000000,0.000000}%
\pgfsetstrokecolor{currentstroke}%
\pgfsetdash{}{0pt}%
\pgfpathmoveto{\pgfqpoint{0.526284in}{1.523939in}}%
\pgfpathlineto{\pgfqpoint{0.526284in}{1.480941in}}%
\pgfpathlineto{\pgfqpoint{0.542970in}{1.480627in}}%
\pgfpathlineto{\pgfqpoint{0.559656in}{1.480673in}}%
\pgfpathlineto{\pgfqpoint{0.576342in}{1.480560in}}%
\pgfpathlineto{\pgfqpoint{0.593028in}{1.480259in}}%
\pgfpathlineto{\pgfqpoint{0.609715in}{1.479923in}}%
\pgfpathlineto{\pgfqpoint{0.626401in}{1.479548in}}%
\pgfpathlineto{\pgfqpoint{0.643087in}{1.479110in}}%
\pgfpathlineto{\pgfqpoint{0.659773in}{1.478977in}}%
\pgfpathlineto{\pgfqpoint{0.676459in}{1.478896in}}%
\pgfpathlineto{\pgfqpoint{0.693145in}{1.478771in}}%
\pgfpathlineto{\pgfqpoint{0.709831in}{1.478426in}}%
\pgfpathlineto{\pgfqpoint{0.726517in}{1.478097in}}%
\pgfpathlineto{\pgfqpoint{0.743204in}{1.477724in}}%
\pgfpathlineto{\pgfqpoint{0.759890in}{1.477381in}}%
\pgfpathlineto{\pgfqpoint{0.776576in}{1.476987in}}%
\pgfpathlineto{\pgfqpoint{0.793262in}{1.476530in}}%
\pgfpathlineto{\pgfqpoint{0.809948in}{1.476184in}}%
\pgfpathlineto{\pgfqpoint{0.826634in}{1.475604in}}%
\pgfpathlineto{\pgfqpoint{0.843320in}{1.474767in}}%
\pgfpathlineto{\pgfqpoint{0.860006in}{1.474362in}}%
\pgfpathlineto{\pgfqpoint{0.876693in}{1.473737in}}%
\pgfpathlineto{\pgfqpoint{0.893379in}{1.473089in}}%
\pgfpathlineto{\pgfqpoint{0.910065in}{1.472654in}}%
\pgfpathlineto{\pgfqpoint{0.926751in}{1.471967in}}%
\pgfpathlineto{\pgfqpoint{0.943437in}{1.471333in}}%
\pgfpathlineto{\pgfqpoint{0.960123in}{1.470593in}}%
\pgfpathlineto{\pgfqpoint{0.976809in}{1.469586in}}%
\pgfpathlineto{\pgfqpoint{0.993495in}{1.468816in}}%
\pgfpathlineto{\pgfqpoint{1.010182in}{1.468044in}}%
\pgfpathlineto{\pgfqpoint{1.026868in}{1.467217in}}%
\pgfpathlineto{\pgfqpoint{1.043554in}{1.465931in}}%
\pgfpathlineto{\pgfqpoint{1.060240in}{1.464716in}}%
\pgfpathlineto{\pgfqpoint{1.076926in}{1.463522in}}%
\pgfpathlineto{\pgfqpoint{1.093612in}{1.462442in}}%
\pgfpathlineto{\pgfqpoint{1.110298in}{1.461297in}}%
\pgfpathlineto{\pgfqpoint{1.126985in}{1.460300in}}%
\pgfpathlineto{\pgfqpoint{1.143671in}{1.459067in}}%
\pgfpathlineto{\pgfqpoint{1.160357in}{1.457775in}}%
\pgfpathlineto{\pgfqpoint{1.177043in}{1.456460in}}%
\pgfpathlineto{\pgfqpoint{1.193729in}{1.455127in}}%
\pgfpathlineto{\pgfqpoint{1.210415in}{1.454291in}}%
\pgfpathlineto{\pgfqpoint{1.227101in}{1.452836in}}%
\pgfpathlineto{\pgfqpoint{1.243787in}{1.451740in}}%
\pgfpathlineto{\pgfqpoint{1.260474in}{1.450451in}}%
\pgfpathlineto{\pgfqpoint{1.277160in}{1.449193in}}%
\pgfpathlineto{\pgfqpoint{1.293846in}{1.447869in}}%
\pgfpathlineto{\pgfqpoint{1.310532in}{1.446573in}}%
\pgfpathlineto{\pgfqpoint{1.327218in}{1.445285in}}%
\pgfpathlineto{\pgfqpoint{1.343904in}{1.443990in}}%
\pgfpathlineto{\pgfqpoint{1.360590in}{1.442444in}}%
\pgfpathlineto{\pgfqpoint{1.377276in}{1.440791in}}%
\pgfpathlineto{\pgfqpoint{1.393963in}{1.439412in}}%
\pgfpathlineto{\pgfqpoint{1.410649in}{1.438141in}}%
\pgfpathlineto{\pgfqpoint{1.427335in}{1.436664in}}%
\pgfpathlineto{\pgfqpoint{1.444021in}{1.435188in}}%
\pgfpathlineto{\pgfqpoint{1.460707in}{1.433712in}}%
\pgfpathlineto{\pgfqpoint{1.477393in}{1.432306in}}%
\pgfpathlineto{\pgfqpoint{1.494079in}{1.430963in}}%
\pgfpathlineto{\pgfqpoint{1.510765in}{1.429710in}}%
\pgfpathlineto{\pgfqpoint{1.527452in}{1.428552in}}%
\pgfpathlineto{\pgfqpoint{1.544138in}{1.427395in}}%
\pgfpathlineto{\pgfqpoint{1.560824in}{1.426238in}}%
\pgfpathlineto{\pgfqpoint{1.577510in}{1.425012in}}%
\pgfpathlineto{\pgfqpoint{1.594196in}{1.423648in}}%
\pgfpathlineto{\pgfqpoint{1.610882in}{1.422134in}}%
\pgfpathlineto{\pgfqpoint{1.627568in}{1.420739in}}%
\pgfpathlineto{\pgfqpoint{1.644255in}{1.419223in}}%
\pgfpathlineto{\pgfqpoint{1.660941in}{1.417854in}}%
\pgfpathlineto{\pgfqpoint{1.677627in}{1.416549in}}%
\pgfpathlineto{\pgfqpoint{1.694313in}{1.415249in}}%
\pgfpathlineto{\pgfqpoint{1.710999in}{1.413949in}}%
\pgfpathlineto{\pgfqpoint{1.727685in}{1.412649in}}%
\pgfpathlineto{\pgfqpoint{1.744371in}{1.411148in}}%
\pgfpathlineto{\pgfqpoint{1.761057in}{1.409697in}}%
\pgfpathlineto{\pgfqpoint{1.777744in}{1.408433in}}%
\pgfpathlineto{\pgfqpoint{1.794430in}{1.407171in}}%
\pgfpathlineto{\pgfqpoint{1.811116in}{1.405884in}}%
\pgfpathlineto{\pgfqpoint{1.827802in}{1.404466in}}%
\pgfpathlineto{\pgfqpoint{1.844488in}{1.403048in}}%
\pgfpathlineto{\pgfqpoint{1.861174in}{1.401561in}}%
\pgfpathlineto{\pgfqpoint{1.877860in}{1.400087in}}%
\pgfpathlineto{\pgfqpoint{1.894546in}{1.398642in}}%
\pgfpathlineto{\pgfqpoint{1.911233in}{1.397164in}}%
\pgfpathlineto{\pgfqpoint{1.927919in}{1.395685in}}%
\pgfpathlineto{\pgfqpoint{1.944605in}{1.394211in}}%
\pgfpathlineto{\pgfqpoint{1.961291in}{1.392738in}}%
\pgfpathlineto{\pgfqpoint{1.977977in}{1.391318in}}%
\pgfpathlineto{\pgfqpoint{1.994663in}{1.389979in}}%
\pgfpathlineto{\pgfqpoint{2.011349in}{1.388639in}}%
\pgfpathlineto{\pgfqpoint{2.028035in}{1.387300in}}%
\pgfpathlineto{\pgfqpoint{2.044722in}{1.385960in}}%
\pgfpathlineto{\pgfqpoint{2.061408in}{1.384579in}}%
\pgfpathlineto{\pgfqpoint{2.078094in}{1.383157in}}%
\pgfpathlineto{\pgfqpoint{2.094780in}{1.381736in}}%
\pgfpathlineto{\pgfqpoint{2.111466in}{1.380315in}}%
\pgfpathlineto{\pgfqpoint{2.128152in}{1.378877in}}%
\pgfpathlineto{\pgfqpoint{2.144838in}{1.377269in}}%
\pgfpathlineto{\pgfqpoint{2.161525in}{1.375660in}}%
\pgfpathlineto{\pgfqpoint{2.178211in}{1.374051in}}%
\pgfpathlineto{\pgfqpoint{2.178211in}{1.483284in}}%
\pgfpathlineto{\pgfqpoint{2.178211in}{1.483284in}}%
\pgfpathlineto{\pgfqpoint{2.161525in}{1.483337in}}%
\pgfpathlineto{\pgfqpoint{2.144838in}{1.483390in}}%
\pgfpathlineto{\pgfqpoint{2.128152in}{1.483443in}}%
\pgfpathlineto{\pgfqpoint{2.111466in}{1.483497in}}%
\pgfpathlineto{\pgfqpoint{2.094780in}{1.483550in}}%
\pgfpathlineto{\pgfqpoint{2.078094in}{1.483603in}}%
\pgfpathlineto{\pgfqpoint{2.061408in}{1.483656in}}%
\pgfpathlineto{\pgfqpoint{2.044722in}{1.483709in}}%
\pgfpathlineto{\pgfqpoint{2.028035in}{1.483763in}}%
\pgfpathlineto{\pgfqpoint{2.011349in}{1.483816in}}%
\pgfpathlineto{\pgfqpoint{1.994663in}{1.483869in}}%
\pgfpathlineto{\pgfqpoint{1.977977in}{1.483922in}}%
\pgfpathlineto{\pgfqpoint{1.961291in}{1.483976in}}%
\pgfpathlineto{\pgfqpoint{1.944605in}{1.484029in}}%
\pgfpathlineto{\pgfqpoint{1.927919in}{1.484117in}}%
\pgfpathlineto{\pgfqpoint{1.911233in}{1.484233in}}%
\pgfpathlineto{\pgfqpoint{1.894546in}{1.484347in}}%
\pgfpathlineto{\pgfqpoint{1.877860in}{1.484460in}}%
\pgfpathlineto{\pgfqpoint{1.861174in}{1.484574in}}%
\pgfpathlineto{\pgfqpoint{1.844488in}{1.484647in}}%
\pgfpathlineto{\pgfqpoint{1.827802in}{1.484546in}}%
\pgfpathlineto{\pgfqpoint{1.811116in}{1.484445in}}%
\pgfpathlineto{\pgfqpoint{1.794430in}{1.484480in}}%
\pgfpathlineto{\pgfqpoint{1.777744in}{1.484529in}}%
\pgfpathlineto{\pgfqpoint{1.761057in}{1.484579in}}%
\pgfpathlineto{\pgfqpoint{1.744371in}{1.484628in}}%
\pgfpathlineto{\pgfqpoint{1.727685in}{1.484610in}}%
\pgfpathlineto{\pgfqpoint{1.710999in}{1.484556in}}%
\pgfpathlineto{\pgfqpoint{1.694313in}{1.484503in}}%
\pgfpathlineto{\pgfqpoint{1.677627in}{1.484450in}}%
\pgfpathlineto{\pgfqpoint{1.660941in}{1.484707in}}%
\pgfpathlineto{\pgfqpoint{1.644255in}{1.484940in}}%
\pgfpathlineto{\pgfqpoint{1.627568in}{1.484998in}}%
\pgfpathlineto{\pgfqpoint{1.610882in}{1.485056in}}%
\pgfpathlineto{\pgfqpoint{1.594196in}{1.485114in}}%
\pgfpathlineto{\pgfqpoint{1.577510in}{1.485172in}}%
\pgfpathlineto{\pgfqpoint{1.560824in}{1.485230in}}%
\pgfpathlineto{\pgfqpoint{1.544138in}{1.485284in}}%
\pgfpathlineto{\pgfqpoint{1.527452in}{1.485450in}}%
\pgfpathlineto{\pgfqpoint{1.510765in}{1.485754in}}%
\pgfpathlineto{\pgfqpoint{1.494079in}{1.486156in}}%
\pgfpathlineto{\pgfqpoint{1.477393in}{1.485982in}}%
\pgfpathlineto{\pgfqpoint{1.460707in}{1.485897in}}%
\pgfpathlineto{\pgfqpoint{1.444021in}{1.485973in}}%
\pgfpathlineto{\pgfqpoint{1.427335in}{1.486172in}}%
\pgfpathlineto{\pgfqpoint{1.410649in}{1.486313in}}%
\pgfpathlineto{\pgfqpoint{1.393963in}{1.486560in}}%
\pgfpathlineto{\pgfqpoint{1.377276in}{1.486927in}}%
\pgfpathlineto{\pgfqpoint{1.360590in}{1.487385in}}%
\pgfpathlineto{\pgfqpoint{1.343904in}{1.487686in}}%
\pgfpathlineto{\pgfqpoint{1.327218in}{1.487748in}}%
\pgfpathlineto{\pgfqpoint{1.310532in}{1.487807in}}%
\pgfpathlineto{\pgfqpoint{1.293846in}{1.487981in}}%
\pgfpathlineto{\pgfqpoint{1.277160in}{1.488465in}}%
\pgfpathlineto{\pgfqpoint{1.260474in}{1.488865in}}%
\pgfpathlineto{\pgfqpoint{1.243787in}{1.488991in}}%
\pgfpathlineto{\pgfqpoint{1.227101in}{1.489106in}}%
\pgfpathlineto{\pgfqpoint{1.210415in}{1.489573in}}%
\pgfpathlineto{\pgfqpoint{1.193729in}{1.489752in}}%
\pgfpathlineto{\pgfqpoint{1.177043in}{1.490238in}}%
\pgfpathlineto{\pgfqpoint{1.160357in}{1.490609in}}%
\pgfpathlineto{\pgfqpoint{1.143671in}{1.490698in}}%
\pgfpathlineto{\pgfqpoint{1.126985in}{1.491154in}}%
\pgfpathlineto{\pgfqpoint{1.110298in}{1.491497in}}%
\pgfpathlineto{\pgfqpoint{1.093612in}{1.492071in}}%
\pgfpathlineto{\pgfqpoint{1.076926in}{1.492358in}}%
\pgfpathlineto{\pgfqpoint{1.060240in}{1.492805in}}%
\pgfpathlineto{\pgfqpoint{1.043554in}{1.493353in}}%
\pgfpathlineto{\pgfqpoint{1.026868in}{1.494065in}}%
\pgfpathlineto{\pgfqpoint{1.010182in}{1.494480in}}%
\pgfpathlineto{\pgfqpoint{0.993495in}{1.494982in}}%
\pgfpathlineto{\pgfqpoint{0.976809in}{1.495533in}}%
\pgfpathlineto{\pgfqpoint{0.960123in}{1.496371in}}%
\pgfpathlineto{\pgfqpoint{0.943437in}{1.496895in}}%
\pgfpathlineto{\pgfqpoint{0.926751in}{1.497575in}}%
\pgfpathlineto{\pgfqpoint{0.910065in}{1.498455in}}%
\pgfpathlineto{\pgfqpoint{0.893379in}{1.499431in}}%
\pgfpathlineto{\pgfqpoint{0.876693in}{1.500386in}}%
\pgfpathlineto{\pgfqpoint{0.860006in}{1.501197in}}%
\pgfpathlineto{\pgfqpoint{0.843320in}{1.502348in}}%
\pgfpathlineto{\pgfqpoint{0.826634in}{1.503239in}}%
\pgfpathlineto{\pgfqpoint{0.809948in}{1.504043in}}%
\pgfpathlineto{\pgfqpoint{0.793262in}{1.505177in}}%
\pgfpathlineto{\pgfqpoint{0.776576in}{1.506305in}}%
\pgfpathlineto{\pgfqpoint{0.759890in}{1.507440in}}%
\pgfpathlineto{\pgfqpoint{0.743204in}{1.508534in}}%
\pgfpathlineto{\pgfqpoint{0.726517in}{1.509588in}}%
\pgfpathlineto{\pgfqpoint{0.709831in}{1.510465in}}%
\pgfpathlineto{\pgfqpoint{0.693145in}{1.511609in}}%
\pgfpathlineto{\pgfqpoint{0.676459in}{1.512526in}}%
\pgfpathlineto{\pgfqpoint{0.659773in}{1.513685in}}%
\pgfpathlineto{\pgfqpoint{0.643087in}{1.514803in}}%
\pgfpathlineto{\pgfqpoint{0.626401in}{1.515902in}}%
\pgfpathlineto{\pgfqpoint{0.609715in}{1.517242in}}%
\pgfpathlineto{\pgfqpoint{0.593028in}{1.518767in}}%
\pgfpathlineto{\pgfqpoint{0.576342in}{1.520149in}}%
\pgfpathlineto{\pgfqpoint{0.559656in}{1.521396in}}%
\pgfpathlineto{\pgfqpoint{0.542970in}{1.522618in}}%
\pgfpathlineto{\pgfqpoint{0.526284in}{1.523939in}}%
\pgfpathclose%
\pgfusepath{fill}%
\end{pgfscope}%
\begin{pgfscope}%
\pgfpathrectangle{\pgfqpoint{0.526284in}{0.473557in}}{\pgfqpoint{1.651927in}{1.704653in}}%
\pgfusepath{clip}%
\pgfsetbuttcap%
\pgfsetroundjoin%
\definecolor{currentfill}{rgb}{0.839216,0.152941,0.156863}%
\pgfsetfillcolor{currentfill}%
\pgfsetfillopacity{0.250000}%
\pgfsetlinewidth{1.003750pt}%
\definecolor{currentstroke}{rgb}{0.839216,0.152941,0.156863}%
\pgfsetstrokecolor{currentstroke}%
\pgfsetstrokeopacity{0.250000}%
\pgfsetdash{}{0pt}%
\pgfsys@defobject{currentmarker}{\pgfqpoint{-0.017010in}{-0.017010in}}{\pgfqpoint{0.017010in}{0.017010in}}{%
\pgfpathmoveto{\pgfqpoint{0.000000in}{-0.017010in}}%
\pgfpathcurveto{\pgfqpoint{0.004511in}{-0.017010in}}{\pgfqpoint{0.008838in}{-0.015218in}}{\pgfqpoint{0.012028in}{-0.012028in}}%
\pgfpathcurveto{\pgfqpoint{0.015218in}{-0.008838in}}{\pgfqpoint{0.017010in}{-0.004511in}}{\pgfqpoint{0.017010in}{0.000000in}}%
\pgfpathcurveto{\pgfqpoint{0.017010in}{0.004511in}}{\pgfqpoint{0.015218in}{0.008838in}}{\pgfqpoint{0.012028in}{0.012028in}}%
\pgfpathcurveto{\pgfqpoint{0.008838in}{0.015218in}}{\pgfqpoint{0.004511in}{0.017010in}}{\pgfqpoint{0.000000in}{0.017010in}}%
\pgfpathcurveto{\pgfqpoint{-0.004511in}{0.017010in}}{\pgfqpoint{-0.008838in}{0.015218in}}{\pgfqpoint{-0.012028in}{0.012028in}}%
\pgfpathcurveto{\pgfqpoint{-0.015218in}{0.008838in}}{\pgfqpoint{-0.017010in}{0.004511in}}{\pgfqpoint{-0.017010in}{0.000000in}}%
\pgfpathcurveto{\pgfqpoint{-0.017010in}{-0.004511in}}{\pgfqpoint{-0.015218in}{-0.008838in}}{\pgfqpoint{-0.012028in}{-0.012028in}}%
\pgfpathcurveto{\pgfqpoint{-0.008838in}{-0.015218in}}{\pgfqpoint{-0.004511in}{-0.017010in}}{\pgfqpoint{0.000000in}{-0.017010in}}%
\pgfpathclose%
\pgfusepath{stroke,fill}%
}%
\begin{pgfscope}%
\pgfsys@transformshift{1.139286in}{1.490423in}%
\pgfsys@useobject{currentmarker}{}%
\end{pgfscope}%
\begin{pgfscope}%
\pgfsys@transformshift{0.826629in}{1.430383in}%
\pgfsys@useobject{currentmarker}{}%
\end{pgfscope}%
\begin{pgfscope}%
\pgfsys@transformshift{0.969498in}{1.269306in}%
\pgfsys@useobject{currentmarker}{}%
\end{pgfscope}%
\begin{pgfscope}%
\pgfsys@transformshift{0.732099in}{1.156213in}%
\pgfsys@useobject{currentmarker}{}%
\end{pgfscope}%
\begin{pgfscope}%
\pgfsys@transformshift{0.815558in}{1.122977in}%
\pgfsys@useobject{currentmarker}{}%
\end{pgfscope}%
\begin{pgfscope}%
\pgfsys@transformshift{1.268362in}{1.461746in}%
\pgfsys@useobject{currentmarker}{}%
\end{pgfscope}%
\begin{pgfscope}%
\pgfsys@transformshift{0.640142in}{1.095326in}%
\pgfsys@useobject{currentmarker}{}%
\end{pgfscope}%
\begin{pgfscope}%
\pgfsys@transformshift{1.031370in}{1.524645in}%
\pgfsys@useobject{currentmarker}{}%
\end{pgfscope}%
\begin{pgfscope}%
\pgfsys@transformshift{1.501355in}{1.479742in}%
\pgfsys@useobject{currentmarker}{}%
\end{pgfscope}%
\begin{pgfscope}%
\pgfsys@transformshift{1.307778in}{1.330353in}%
\pgfsys@useobject{currentmarker}{}%
\end{pgfscope}%
\begin{pgfscope}%
\pgfsys@transformshift{1.194068in}{1.219362in}%
\pgfsys@useobject{currentmarker}{}%
\end{pgfscope}%
\begin{pgfscope}%
\pgfsys@transformshift{0.667950in}{1.255109in}%
\pgfsys@useobject{currentmarker}{}%
\end{pgfscope}%
\begin{pgfscope}%
\pgfsys@transformshift{0.827370in}{1.076962in}%
\pgfsys@useobject{currentmarker}{}%
\end{pgfscope}%
\begin{pgfscope}%
\pgfsys@transformshift{1.071193in}{1.335756in}%
\pgfsys@useobject{currentmarker}{}%
\end{pgfscope}%
\begin{pgfscope}%
\pgfsys@transformshift{0.921937in}{1.288386in}%
\pgfsys@useobject{currentmarker}{}%
\end{pgfscope}%
\begin{pgfscope}%
\pgfsys@transformshift{1.556989in}{1.231981in}%
\pgfsys@useobject{currentmarker}{}%
\end{pgfscope}%
\begin{pgfscope}%
\pgfsys@transformshift{0.645159in}{1.312469in}%
\pgfsys@useobject{currentmarker}{}%
\end{pgfscope}%
\begin{pgfscope}%
\pgfsys@transformshift{1.282858in}{1.459330in}%
\pgfsys@useobject{currentmarker}{}%
\end{pgfscope}%
\begin{pgfscope}%
\pgfsys@transformshift{0.923659in}{1.349455in}%
\pgfsys@useobject{currentmarker}{}%
\end{pgfscope}%
\begin{pgfscope}%
\pgfsys@transformshift{1.196863in}{1.436075in}%
\pgfsys@useobject{currentmarker}{}%
\end{pgfscope}%
\begin{pgfscope}%
\pgfsys@transformshift{0.734895in}{1.218394in}%
\pgfsys@useobject{currentmarker}{}%
\end{pgfscope}%
\begin{pgfscope}%
\pgfsys@transformshift{1.049754in}{1.210398in}%
\pgfsys@useobject{currentmarker}{}%
\end{pgfscope}%
\begin{pgfscope}%
\pgfsys@transformshift{0.612705in}{1.204923in}%
\pgfsys@useobject{currentmarker}{}%
\end{pgfscope}%
\begin{pgfscope}%
\pgfsys@transformshift{1.196567in}{1.527498in}%
\pgfsys@useobject{currentmarker}{}%
\end{pgfscope}%
\begin{pgfscope}%
\pgfsys@transformshift{0.616815in}{1.238744in}%
\pgfsys@useobject{currentmarker}{}%
\end{pgfscope}%
\begin{pgfscope}%
\pgfsys@transformshift{0.881170in}{0.929455in}%
\pgfsys@useobject{currentmarker}{}%
\end{pgfscope}%
\begin{pgfscope}%
\pgfsys@transformshift{0.644863in}{1.152415in}%
\pgfsys@useobject{currentmarker}{}%
\end{pgfscope}%
\begin{pgfscope}%
\pgfsys@transformshift{1.119421in}{1.088051in}%
\pgfsys@useobject{currentmarker}{}%
\end{pgfscope}%
\begin{pgfscope}%
\pgfsys@transformshift{0.796952in}{1.267525in}%
\pgfsys@useobject{currentmarker}{}%
\end{pgfscope}%
\begin{pgfscope}%
\pgfsys@transformshift{0.928991in}{1.346073in}%
\pgfsys@useobject{currentmarker}{}%
\end{pgfscope}%
\begin{pgfscope}%
\pgfsys@transformshift{1.146006in}{1.425498in}%
\pgfsys@useobject{currentmarker}{}%
\end{pgfscope}%
\begin{pgfscope}%
\pgfsys@transformshift{1.088355in}{1.471529in}%
\pgfsys@useobject{currentmarker}{}%
\end{pgfscope}%
\begin{pgfscope}%
\pgfsys@transformshift{1.172758in}{1.073845in}%
\pgfsys@useobject{currentmarker}{}%
\end{pgfscope}%
\begin{pgfscope}%
\pgfsys@transformshift{1.211137in}{1.484169in}%
\pgfsys@useobject{currentmarker}{}%
\end{pgfscope}%
\begin{pgfscope}%
\pgfsys@transformshift{0.829388in}{1.216591in}%
\pgfsys@useobject{currentmarker}{}%
\end{pgfscope}%
\begin{pgfscope}%
\pgfsys@transformshift{1.584944in}{1.131457in}%
\pgfsys@useobject{currentmarker}{}%
\end{pgfscope}%
\begin{pgfscope}%
\pgfsys@transformshift{0.693739in}{1.409086in}%
\pgfsys@useobject{currentmarker}{}%
\end{pgfscope}%
\begin{pgfscope}%
\pgfsys@transformshift{1.271769in}{1.484471in}%
\pgfsys@useobject{currentmarker}{}%
\end{pgfscope}%
\begin{pgfscope}%
\pgfsys@transformshift{1.064010in}{1.424461in}%
\pgfsys@useobject{currentmarker}{}%
\end{pgfscope}%
\begin{pgfscope}%
\pgfsys@transformshift{1.461162in}{1.130313in}%
\pgfsys@useobject{currentmarker}{}%
\end{pgfscope}%
\begin{pgfscope}%
\pgfsys@transformshift{1.254162in}{1.425025in}%
\pgfsys@useobject{currentmarker}{}%
\end{pgfscope}%
\begin{pgfscope}%
\pgfsys@transformshift{0.641290in}{1.327187in}%
\pgfsys@useobject{currentmarker}{}%
\end{pgfscope}%
\begin{pgfscope}%
\pgfsys@transformshift{1.047681in}{1.114292in}%
\pgfsys@useobject{currentmarker}{}%
\end{pgfscope}%
\begin{pgfscope}%
\pgfsys@transformshift{0.651435in}{1.296307in}%
\pgfsys@useobject{currentmarker}{}%
\end{pgfscope}%
\begin{pgfscope}%
\pgfsys@transformshift{1.047699in}{1.085812in}%
\pgfsys@useobject{currentmarker}{}%
\end{pgfscope}%
\begin{pgfscope}%
\pgfsys@transformshift{0.706939in}{1.065135in}%
\pgfsys@useobject{currentmarker}{}%
\end{pgfscope}%
\begin{pgfscope}%
\pgfsys@transformshift{0.621869in}{1.491377in}%
\pgfsys@useobject{currentmarker}{}%
\end{pgfscope}%
\begin{pgfscope}%
\pgfsys@transformshift{1.117533in}{1.381380in}%
\pgfsys@useobject{currentmarker}{}%
\end{pgfscope}%
\begin{pgfscope}%
\pgfsys@transformshift{1.307648in}{1.108461in}%
\pgfsys@useobject{currentmarker}{}%
\end{pgfscope}%
\begin{pgfscope}%
\pgfsys@transformshift{0.948856in}{1.578743in}%
\pgfsys@useobject{currentmarker}{}%
\end{pgfscope}%
\begin{pgfscope}%
\pgfsys@transformshift{1.014153in}{1.333399in}%
\pgfsys@useobject{currentmarker}{}%
\end{pgfscope}%
\begin{pgfscope}%
\pgfsys@transformshift{0.631737in}{1.444050in}%
\pgfsys@useobject{currentmarker}{}%
\end{pgfscope}%
\begin{pgfscope}%
\pgfsys@transformshift{0.643493in}{1.392848in}%
\pgfsys@useobject{currentmarker}{}%
\end{pgfscope}%
\begin{pgfscope}%
\pgfsys@transformshift{0.772755in}{1.344327in}%
\pgfsys@useobject{currentmarker}{}%
\end{pgfscope}%
\begin{pgfscope}%
\pgfsys@transformshift{0.692091in}{1.140137in}%
\pgfsys@useobject{currentmarker}{}%
\end{pgfscope}%
\begin{pgfscope}%
\pgfsys@transformshift{0.595358in}{1.428803in}%
\pgfsys@useobject{currentmarker}{}%
\end{pgfscope}%
\begin{pgfscope}%
\pgfsys@transformshift{0.903997in}{1.098437in}%
\pgfsys@useobject{currentmarker}{}%
\end{pgfscope}%
\begin{pgfscope}%
\pgfsys@transformshift{1.258735in}{1.409042in}%
\pgfsys@useobject{currentmarker}{}%
\end{pgfscope}%
\begin{pgfscope}%
\pgfsys@transformshift{0.771996in}{1.160139in}%
\pgfsys@useobject{currentmarker}{}%
\end{pgfscope}%
\begin{pgfscope}%
\pgfsys@transformshift{0.665932in}{1.171700in}%
\pgfsys@useobject{currentmarker}{}%
\end{pgfscope}%
\begin{pgfscope}%
\pgfsys@transformshift{0.723120in}{1.192463in}%
\pgfsys@useobject{currentmarker}{}%
\end{pgfscope}%
\begin{pgfscope}%
\pgfsys@transformshift{1.030500in}{1.663955in}%
\pgfsys@useobject{currentmarker}{}%
\end{pgfscope}%
\begin{pgfscope}%
\pgfsys@transformshift{0.538429in}{1.303842in}%
\pgfsys@useobject{currentmarker}{}%
\end{pgfscope}%
\begin{pgfscope}%
\pgfsys@transformshift{1.081468in}{1.050184in}%
\pgfsys@useobject{currentmarker}{}%
\end{pgfscope}%
\begin{pgfscope}%
\pgfsys@transformshift{0.629293in}{1.107684in}%
\pgfsys@useobject{currentmarker}{}%
\end{pgfscope}%
\begin{pgfscope}%
\pgfsys@transformshift{0.792175in}{1.058955in}%
\pgfsys@useobject{currentmarker}{}%
\end{pgfscope}%
\begin{pgfscope}%
\pgfsys@transformshift{1.017763in}{1.122435in}%
\pgfsys@useobject{currentmarker}{}%
\end{pgfscope}%
\begin{pgfscope}%
\pgfsys@transformshift{0.953410in}{1.328267in}%
\pgfsys@useobject{currentmarker}{}%
\end{pgfscope}%
\begin{pgfscope}%
\pgfsys@transformshift{1.026687in}{1.380708in}%
\pgfsys@useobject{currentmarker}{}%
\end{pgfscope}%
\begin{pgfscope}%
\pgfsys@transformshift{1.527422in}{1.050978in}%
\pgfsys@useobject{currentmarker}{}%
\end{pgfscope}%
\begin{pgfscope}%
\pgfsys@transformshift{1.160817in}{1.430294in}%
\pgfsys@useobject{currentmarker}{}%
\end{pgfscope}%
\begin{pgfscope}%
\pgfsys@transformshift{0.698349in}{1.537510in}%
\pgfsys@useobject{currentmarker}{}%
\end{pgfscope}%
\begin{pgfscope}%
\pgfsys@transformshift{1.035203in}{1.295040in}%
\pgfsys@useobject{currentmarker}{}%
\end{pgfscope}%
\begin{pgfscope}%
\pgfsys@transformshift{0.922307in}{1.319293in}%
\pgfsys@useobject{currentmarker}{}%
\end{pgfscope}%
\begin{pgfscope}%
\pgfsys@transformshift{1.166982in}{1.458263in}%
\pgfsys@useobject{currentmarker}{}%
\end{pgfscope}%
\begin{pgfscope}%
\pgfsys@transformshift{1.119569in}{1.430273in}%
\pgfsys@useobject{currentmarker}{}%
\end{pgfscope}%
\begin{pgfscope}%
\pgfsys@transformshift{0.825518in}{1.088404in}%
\pgfsys@useobject{currentmarker}{}%
\end{pgfscope}%
\begin{pgfscope}%
\pgfsys@transformshift{1.426209in}{1.119521in}%
\pgfsys@useobject{currentmarker}{}%
\end{pgfscope}%
\begin{pgfscope}%
\pgfsys@transformshift{1.406677in}{1.421714in}%
\pgfsys@useobject{currentmarker}{}%
\end{pgfscope}%
\begin{pgfscope}%
\pgfsys@transformshift{1.415453in}{1.336623in}%
\pgfsys@useobject{currentmarker}{}%
\end{pgfscope}%
\begin{pgfscope}%
\pgfsys@transformshift{1.076914in}{1.515534in}%
\pgfsys@useobject{currentmarker}{}%
\end{pgfscope}%
\begin{pgfscope}%
\pgfsys@transformshift{0.779031in}{1.053482in}%
\pgfsys@useobject{currentmarker}{}%
\end{pgfscope}%
\begin{pgfscope}%
\pgfsys@transformshift{0.629497in}{1.238391in}%
\pgfsys@useobject{currentmarker}{}%
\end{pgfscope}%
\begin{pgfscope}%
\pgfsys@transformshift{1.519184in}{1.240656in}%
\pgfsys@useobject{currentmarker}{}%
\end{pgfscope}%
\begin{pgfscope}%
\pgfsys@transformshift{0.769237in}{1.206327in}%
\pgfsys@useobject{currentmarker}{}%
\end{pgfscope}%
\begin{pgfscope}%
\pgfsys@transformshift{1.087429in}{1.198616in}%
\pgfsys@useobject{currentmarker}{}%
\end{pgfscope}%
\begin{pgfscope}%
\pgfsys@transformshift{0.750705in}{1.637773in}%
\pgfsys@useobject{currentmarker}{}%
\end{pgfscope}%
\begin{pgfscope}%
\pgfsys@transformshift{1.495746in}{1.140515in}%
\pgfsys@useobject{currentmarker}{}%
\end{pgfscope}%
\begin{pgfscope}%
\pgfsys@transformshift{1.182959in}{1.253327in}%
\pgfsys@useobject{currentmarker}{}%
\end{pgfscope}%
\begin{pgfscope}%
\pgfsys@transformshift{0.713789in}{1.267723in}%
\pgfsys@useobject{currentmarker}{}%
\end{pgfscope}%
\begin{pgfscope}%
\pgfsys@transformshift{1.355894in}{1.322209in}%
\pgfsys@useobject{currentmarker}{}%
\end{pgfscope}%
\begin{pgfscope}%
\pgfsys@transformshift{1.635208in}{1.369139in}%
\pgfsys@useobject{currentmarker}{}%
\end{pgfscope}%
\begin{pgfscope}%
\pgfsys@transformshift{1.253848in}{1.204526in}%
\pgfsys@useobject{currentmarker}{}%
\end{pgfscope}%
\begin{pgfscope}%
\pgfsys@transformshift{0.935989in}{1.589784in}%
\pgfsys@useobject{currentmarker}{}%
\end{pgfscope}%
\begin{pgfscope}%
\pgfsys@transformshift{0.875320in}{1.421406in}%
\pgfsys@useobject{currentmarker}{}%
\end{pgfscope}%
\begin{pgfscope}%
\pgfsys@transformshift{1.109831in}{1.637560in}%
\pgfsys@useobject{currentmarker}{}%
\end{pgfscope}%
\begin{pgfscope}%
\pgfsys@transformshift{1.118292in}{1.135966in}%
\pgfsys@useobject{currentmarker}{}%
\end{pgfscope}%
\begin{pgfscope}%
\pgfsys@transformshift{1.130029in}{1.473818in}%
\pgfsys@useobject{currentmarker}{}%
\end{pgfscope}%
\begin{pgfscope}%
\pgfsys@transformshift{1.140971in}{1.559406in}%
\pgfsys@useobject{currentmarker}{}%
\end{pgfscope}%
\begin{pgfscope}%
\pgfsys@transformshift{1.052346in}{1.755855in}%
\pgfsys@useobject{currentmarker}{}%
\end{pgfscope}%
\begin{pgfscope}%
\pgfsys@transformshift{0.621629in}{1.319660in}%
\pgfsys@useobject{currentmarker}{}%
\end{pgfscope}%
\begin{pgfscope}%
\pgfsys@transformshift{0.917753in}{1.336947in}%
\pgfsys@useobject{currentmarker}{}%
\end{pgfscope}%
\begin{pgfscope}%
\pgfsys@transformshift{0.863027in}{1.360499in}%
\pgfsys@useobject{currentmarker}{}%
\end{pgfscope}%
\begin{pgfscope}%
\pgfsys@transformshift{0.672485in}{1.171159in}%
\pgfsys@useobject{currentmarker}{}%
\end{pgfscope}%
\begin{pgfscope}%
\pgfsys@transformshift{1.747437in}{1.065784in}%
\pgfsys@useobject{currentmarker}{}%
\end{pgfscope}%
\begin{pgfscope}%
\pgfsys@transformshift{0.526284in}{1.299566in}%
\pgfsys@useobject{currentmarker}{}%
\end{pgfscope}%
\begin{pgfscope}%
\pgfsys@transformshift{0.627405in}{1.145029in}%
\pgfsys@useobject{currentmarker}{}%
\end{pgfscope}%
\begin{pgfscope}%
\pgfsys@transformshift{1.270658in}{1.118091in}%
\pgfsys@useobject{currentmarker}{}%
\end{pgfscope}%
\begin{pgfscope}%
\pgfsys@transformshift{1.594053in}{1.069049in}%
\pgfsys@useobject{currentmarker}{}%
\end{pgfscope}%
\begin{pgfscope}%
\pgfsys@transformshift{0.652472in}{1.144730in}%
\pgfsys@useobject{currentmarker}{}%
\end{pgfscope}%
\begin{pgfscope}%
\pgfsys@transformshift{0.819168in}{1.272594in}%
\pgfsys@useobject{currentmarker}{}%
\end{pgfscope}%
\begin{pgfscope}%
\pgfsys@transformshift{0.847883in}{1.046583in}%
\pgfsys@useobject{currentmarker}{}%
\end{pgfscope}%
\begin{pgfscope}%
\pgfsys@transformshift{1.118495in}{0.965986in}%
\pgfsys@useobject{currentmarker}{}%
\end{pgfscope}%
\begin{pgfscope}%
\pgfsys@transformshift{0.618167in}{1.178342in}%
\pgfsys@useobject{currentmarker}{}%
\end{pgfscope}%
\begin{pgfscope}%
\pgfsys@transformshift{0.857547in}{1.392078in}%
\pgfsys@useobject{currentmarker}{}%
\end{pgfscope}%
\begin{pgfscope}%
\pgfsys@transformshift{1.090392in}{1.583928in}%
\pgfsys@useobject{currentmarker}{}%
\end{pgfscope}%
\begin{pgfscope}%
\pgfsys@transformshift{1.063566in}{1.533037in}%
\pgfsys@useobject{currentmarker}{}%
\end{pgfscope}%
\begin{pgfscope}%
\pgfsys@transformshift{1.510205in}{1.081062in}%
\pgfsys@useobject{currentmarker}{}%
\end{pgfscope}%
\begin{pgfscope}%
\pgfsys@transformshift{1.287209in}{1.464198in}%
\pgfsys@useobject{currentmarker}{}%
\end{pgfscope}%
\begin{pgfscope}%
\pgfsys@transformshift{0.778753in}{1.142662in}%
\pgfsys@useobject{currentmarker}{}%
\end{pgfscope}%
\begin{pgfscope}%
\pgfsys@transformshift{0.627757in}{1.381345in}%
\pgfsys@useobject{currentmarker}{}%
\end{pgfscope}%
\begin{pgfscope}%
\pgfsys@transformshift{1.071860in}{1.445841in}%
\pgfsys@useobject{currentmarker}{}%
\end{pgfscope}%
\begin{pgfscope}%
\pgfsys@transformshift{1.053938in}{1.213800in}%
\pgfsys@useobject{currentmarker}{}%
\end{pgfscope}%
\begin{pgfscope}%
\pgfsys@transformshift{0.910458in}{1.341271in}%
\pgfsys@useobject{currentmarker}{}%
\end{pgfscope}%
\begin{pgfscope}%
\pgfsys@transformshift{0.621962in}{1.360885in}%
\pgfsys@useobject{currentmarker}{}%
\end{pgfscope}%
\begin{pgfscope}%
\pgfsys@transformshift{0.836867in}{1.351941in}%
\pgfsys@useobject{currentmarker}{}%
\end{pgfscope}%
\begin{pgfscope}%
\pgfsys@transformshift{0.800229in}{1.399061in}%
\pgfsys@useobject{currentmarker}{}%
\end{pgfscope}%
\begin{pgfscope}%
\pgfsys@transformshift{0.958260in}{1.356401in}%
\pgfsys@useobject{currentmarker}{}%
\end{pgfscope}%
\begin{pgfscope}%
\pgfsys@transformshift{0.656045in}{1.181944in}%
\pgfsys@useobject{currentmarker}{}%
\end{pgfscope}%
\begin{pgfscope}%
\pgfsys@transformshift{0.629960in}{1.226371in}%
\pgfsys@useobject{currentmarker}{}%
\end{pgfscope}%
\begin{pgfscope}%
\pgfsys@transformshift{1.008358in}{1.112179in}%
\pgfsys@useobject{currentmarker}{}%
\end{pgfscope}%
\begin{pgfscope}%
\pgfsys@transformshift{1.258698in}{1.356667in}%
\pgfsys@useobject{currentmarker}{}%
\end{pgfscope}%
\begin{pgfscope}%
\pgfsys@transformshift{1.100889in}{1.069832in}%
\pgfsys@useobject{currentmarker}{}%
\end{pgfscope}%
\begin{pgfscope}%
\pgfsys@transformshift{1.230410in}{1.343480in}%
\pgfsys@useobject{currentmarker}{}%
\end{pgfscope}%
\begin{pgfscope}%
\pgfsys@transformshift{0.647325in}{1.396150in}%
\pgfsys@useobject{currentmarker}{}%
\end{pgfscope}%
\begin{pgfscope}%
\pgfsys@transformshift{0.904423in}{1.498377in}%
\pgfsys@useobject{currentmarker}{}%
\end{pgfscope}%
\begin{pgfscope}%
\pgfsys@transformshift{1.617602in}{1.274943in}%
\pgfsys@useobject{currentmarker}{}%
\end{pgfscope}%
\begin{pgfscope}%
\pgfsys@transformshift{1.245406in}{1.469139in}%
\pgfsys@useobject{currentmarker}{}%
\end{pgfscope}%
\begin{pgfscope}%
\pgfsys@transformshift{1.238907in}{1.479411in}%
\pgfsys@useobject{currentmarker}{}%
\end{pgfscope}%
\begin{pgfscope}%
\pgfsys@transformshift{1.187366in}{1.456206in}%
\pgfsys@useobject{currentmarker}{}%
\end{pgfscope}%
\begin{pgfscope}%
\pgfsys@transformshift{1.095724in}{1.502546in}%
\pgfsys@useobject{currentmarker}{}%
\end{pgfscope}%
\begin{pgfscope}%
\pgfsys@transformshift{1.131084in}{1.603407in}%
\pgfsys@useobject{currentmarker}{}%
\end{pgfscope}%
\begin{pgfscope}%
\pgfsys@transformshift{0.643549in}{1.235312in}%
\pgfsys@useobject{currentmarker}{}%
\end{pgfscope}%
\begin{pgfscope}%
\pgfsys@transformshift{0.726878in}{1.312162in}%
\pgfsys@useobject{currentmarker}{}%
\end{pgfscope}%
\begin{pgfscope}%
\pgfsys@transformshift{0.940728in}{1.215869in}%
\pgfsys@useobject{currentmarker}{}%
\end{pgfscope}%
\begin{pgfscope}%
\pgfsys@transformshift{0.686537in}{1.162408in}%
\pgfsys@useobject{currentmarker}{}%
\end{pgfscope}%
\begin{pgfscope}%
\pgfsys@transformshift{1.304686in}{1.421473in}%
\pgfsys@useobject{currentmarker}{}%
\end{pgfscope}%
\begin{pgfscope}%
\pgfsys@transformshift{0.822223in}{1.227900in}%
\pgfsys@useobject{currentmarker}{}%
\end{pgfscope}%
\begin{pgfscope}%
\pgfsys@transformshift{0.753593in}{1.251639in}%
\pgfsys@useobject{currentmarker}{}%
\end{pgfscope}%
\begin{pgfscope}%
\pgfsys@transformshift{1.155670in}{1.428115in}%
\pgfsys@useobject{currentmarker}{}%
\end{pgfscope}%
\begin{pgfscope}%
\pgfsys@transformshift{0.665617in}{1.075366in}%
\pgfsys@useobject{currentmarker}{}%
\end{pgfscope}%
\begin{pgfscope}%
\pgfsys@transformshift{0.722713in}{1.107547in}%
\pgfsys@useobject{currentmarker}{}%
\end{pgfscope}%
\begin{pgfscope}%
\pgfsys@transformshift{0.537651in}{1.225972in}%
\pgfsys@useobject{currentmarker}{}%
\end{pgfscope}%
\begin{pgfscope}%
\pgfsys@transformshift{0.956631in}{1.335116in}%
\pgfsys@useobject{currentmarker}{}%
\end{pgfscope}%
\begin{pgfscope}%
\pgfsys@transformshift{1.046922in}{1.203094in}%
\pgfsys@useobject{currentmarker}{}%
\end{pgfscope}%
\begin{pgfscope}%
\pgfsys@transformshift{0.812559in}{1.353429in}%
\pgfsys@useobject{currentmarker}{}%
\end{pgfscope}%
\begin{pgfscope}%
\pgfsys@transformshift{1.535254in}{1.257794in}%
\pgfsys@useobject{currentmarker}{}%
\end{pgfscope}%
\begin{pgfscope}%
\pgfsys@transformshift{0.778087in}{1.359587in}%
\pgfsys@useobject{currentmarker}{}%
\end{pgfscope}%
\begin{pgfscope}%
\pgfsys@transformshift{0.723860in}{1.452989in}%
\pgfsys@useobject{currentmarker}{}%
\end{pgfscope}%
\begin{pgfscope}%
\pgfsys@transformshift{0.970961in}{1.050829in}%
\pgfsys@useobject{currentmarker}{}%
\end{pgfscope}%
\begin{pgfscope}%
\pgfsys@transformshift{1.099149in}{1.729438in}%
\pgfsys@useobject{currentmarker}{}%
\end{pgfscope}%
\begin{pgfscope}%
\pgfsys@transformshift{1.042182in}{1.444292in}%
\pgfsys@useobject{currentmarker}{}%
\end{pgfscope}%
\begin{pgfscope}%
\pgfsys@transformshift{1.047181in}{1.664966in}%
\pgfsys@useobject{currentmarker}{}%
\end{pgfscope}%
\begin{pgfscope}%
\pgfsys@transformshift{1.569152in}{1.419625in}%
\pgfsys@useobject{currentmarker}{}%
\end{pgfscope}%
\begin{pgfscope}%
\pgfsys@transformshift{1.518018in}{1.175161in}%
\pgfsys@useobject{currentmarker}{}%
\end{pgfscope}%
\begin{pgfscope}%
\pgfsys@transformshift{0.733265in}{1.281633in}%
\pgfsys@useobject{currentmarker}{}%
\end{pgfscope}%
\begin{pgfscope}%
\pgfsys@transformshift{0.558701in}{1.171935in}%
\pgfsys@useobject{currentmarker}{}%
\end{pgfscope}%
\begin{pgfscope}%
\pgfsys@transformshift{0.600190in}{1.423121in}%
\pgfsys@useobject{currentmarker}{}%
\end{pgfscope}%
\begin{pgfscope}%
\pgfsys@transformshift{1.222060in}{1.437590in}%
\pgfsys@useobject{currentmarker}{}%
\end{pgfscope}%
\begin{pgfscope}%
\pgfsys@transformshift{0.884484in}{1.169840in}%
\pgfsys@useobject{currentmarker}{}%
\end{pgfscope}%
\begin{pgfscope}%
\pgfsys@transformshift{1.004359in}{1.292060in}%
\pgfsys@useobject{currentmarker}{}%
\end{pgfscope}%
\begin{pgfscope}%
\pgfsys@transformshift{0.853585in}{1.046330in}%
\pgfsys@useobject{currentmarker}{}%
\end{pgfscope}%
\begin{pgfscope}%
\pgfsys@transformshift{0.668357in}{1.481924in}%
\pgfsys@useobject{currentmarker}{}%
\end{pgfscope}%
\begin{pgfscope}%
\pgfsys@transformshift{0.536244in}{1.254733in}%
\pgfsys@useobject{currentmarker}{}%
\end{pgfscope}%
\begin{pgfscope}%
\pgfsys@transformshift{0.785103in}{1.307353in}%
\pgfsys@useobject{currentmarker}{}%
\end{pgfscope}%
\begin{pgfscope}%
\pgfsys@transformshift{1.369965in}{1.309156in}%
\pgfsys@useobject{currentmarker}{}%
\end{pgfscope}%
\begin{pgfscope}%
\pgfsys@transformshift{0.794379in}{1.036101in}%
\pgfsys@useobject{currentmarker}{}%
\end{pgfscope}%
\begin{pgfscope}%
\pgfsys@transformshift{0.633237in}{1.343152in}%
\pgfsys@useobject{currentmarker}{}%
\end{pgfscope}%
\begin{pgfscope}%
\pgfsys@transformshift{0.859287in}{1.173215in}%
\pgfsys@useobject{currentmarker}{}%
\end{pgfscope}%
\begin{pgfscope}%
\pgfsys@transformshift{1.032685in}{1.139831in}%
\pgfsys@useobject{currentmarker}{}%
\end{pgfscope}%
\begin{pgfscope}%
\pgfsys@transformshift{0.784141in}{1.191171in}%
\pgfsys@useobject{currentmarker}{}%
\end{pgfscope}%
\begin{pgfscope}%
\pgfsys@transformshift{0.996010in}{1.173637in}%
\pgfsys@useobject{currentmarker}{}%
\end{pgfscope}%
\begin{pgfscope}%
\pgfsys@transformshift{1.002563in}{1.032456in}%
\pgfsys@useobject{currentmarker}{}%
\end{pgfscope}%
\begin{pgfscope}%
\pgfsys@transformshift{0.930287in}{1.443798in}%
\pgfsys@useobject{currentmarker}{}%
\end{pgfscope}%
\begin{pgfscope}%
\pgfsys@transformshift{0.659693in}{1.215996in}%
\pgfsys@useobject{currentmarker}{}%
\end{pgfscope}%
\begin{pgfscope}%
\pgfsys@transformshift{1.068472in}{1.490822in}%
\pgfsys@useobject{currentmarker}{}%
\end{pgfscope}%
\begin{pgfscope}%
\pgfsys@transformshift{0.856640in}{1.353564in}%
\pgfsys@useobject{currentmarker}{}%
\end{pgfscope}%
\begin{pgfscope}%
\pgfsys@transformshift{1.405029in}{1.336898in}%
\pgfsys@useobject{currentmarker}{}%
\end{pgfscope}%
\begin{pgfscope}%
\pgfsys@transformshift{1.220135in}{1.381287in}%
\pgfsys@useobject{currentmarker}{}%
\end{pgfscope}%
\begin{pgfscope}%
\pgfsys@transformshift{0.999324in}{1.174129in}%
\pgfsys@useobject{currentmarker}{}%
\end{pgfscope}%
\begin{pgfscope}%
\pgfsys@transformshift{0.633699in}{1.250898in}%
\pgfsys@useobject{currentmarker}{}%
\end{pgfscope}%
\begin{pgfscope}%
\pgfsys@transformshift{1.135768in}{1.420050in}%
\pgfsys@useobject{currentmarker}{}%
\end{pgfscope}%
\begin{pgfscope}%
\pgfsys@transformshift{1.116959in}{0.903012in}%
\pgfsys@useobject{currentmarker}{}%
\end{pgfscope}%
\begin{pgfscope}%
\pgfsys@transformshift{0.598172in}{1.439091in}%
\pgfsys@useobject{currentmarker}{}%
\end{pgfscope}%
\begin{pgfscope}%
\pgfsys@transformshift{0.619907in}{1.272106in}%
\pgfsys@useobject{currentmarker}{}%
\end{pgfscope}%
\begin{pgfscope}%
\pgfsys@transformshift{1.023539in}{1.089485in}%
\pgfsys@useobject{currentmarker}{}%
\end{pgfscope}%
\begin{pgfscope}%
\pgfsys@transformshift{1.456182in}{1.254707in}%
\pgfsys@useobject{currentmarker}{}%
\end{pgfscope}%
\begin{pgfscope}%
\pgfsys@transformshift{0.579547in}{1.475049in}%
\pgfsys@useobject{currentmarker}{}%
\end{pgfscope}%
\begin{pgfscope}%
\pgfsys@transformshift{1.012949in}{1.215402in}%
\pgfsys@useobject{currentmarker}{}%
\end{pgfscope}%
\begin{pgfscope}%
\pgfsys@transformshift{1.111293in}{1.085872in}%
\pgfsys@useobject{currentmarker}{}%
\end{pgfscope}%
\begin{pgfscope}%
\pgfsys@transformshift{0.977218in}{1.130818in}%
\pgfsys@useobject{currentmarker}{}%
\end{pgfscope}%
\begin{pgfscope}%
\pgfsys@transformshift{1.304371in}{1.406674in}%
\pgfsys@useobject{currentmarker}{}%
\end{pgfscope}%
\begin{pgfscope}%
\pgfsys@transformshift{1.456756in}{1.553292in}%
\pgfsys@useobject{currentmarker}{}%
\end{pgfscope}%
\begin{pgfscope}%
\pgfsys@transformshift{0.684908in}{1.121146in}%
\pgfsys@useobject{currentmarker}{}%
\end{pgfscope}%
\begin{pgfscope}%
\pgfsys@transformshift{0.997176in}{0.968179in}%
\pgfsys@useobject{currentmarker}{}%
\end{pgfscope}%
\begin{pgfscope}%
\pgfsys@transformshift{1.265659in}{1.461645in}%
\pgfsys@useobject{currentmarker}{}%
\end{pgfscope}%
\begin{pgfscope}%
\pgfsys@transformshift{0.624998in}{1.233617in}%
\pgfsys@useobject{currentmarker}{}%
\end{pgfscope}%
\begin{pgfscope}%
\pgfsys@transformshift{0.771255in}{1.577279in}%
\pgfsys@useobject{currentmarker}{}%
\end{pgfscope}%
\begin{pgfscope}%
\pgfsys@transformshift{0.828869in}{1.324080in}%
\pgfsys@useobject{currentmarker}{}%
\end{pgfscope}%
\begin{pgfscope}%
\pgfsys@transformshift{0.739412in}{1.046856in}%
\pgfsys@useobject{currentmarker}{}%
\end{pgfscope}%
\begin{pgfscope}%
\pgfsys@transformshift{0.873931in}{1.068578in}%
\pgfsys@useobject{currentmarker}{}%
\end{pgfscope}%
\begin{pgfscope}%
\pgfsys@transformshift{0.630219in}{1.188084in}%
\pgfsys@useobject{currentmarker}{}%
\end{pgfscope}%
\begin{pgfscope}%
\pgfsys@transformshift{1.172814in}{1.434002in}%
\pgfsys@useobject{currentmarker}{}%
\end{pgfscope}%
\begin{pgfscope}%
\pgfsys@transformshift{0.647992in}{1.169862in}%
\pgfsys@useobject{currentmarker}{}%
\end{pgfscope}%
\begin{pgfscope}%
\pgfsys@transformshift{1.062251in}{0.958267in}%
\pgfsys@useobject{currentmarker}{}%
\end{pgfscope}%
\begin{pgfscope}%
\pgfsys@transformshift{1.096816in}{1.405163in}%
\pgfsys@useobject{currentmarker}{}%
\end{pgfscope}%
\begin{pgfscope}%
\pgfsys@transformshift{1.149450in}{1.050181in}%
\pgfsys@useobject{currentmarker}{}%
\end{pgfscope}%
\begin{pgfscope}%
\pgfsys@transformshift{0.677632in}{1.335414in}%
\pgfsys@useobject{currentmarker}{}%
\end{pgfscope}%
\begin{pgfscope}%
\pgfsys@transformshift{1.294966in}{1.444174in}%
\pgfsys@useobject{currentmarker}{}%
\end{pgfscope}%
\begin{pgfscope}%
\pgfsys@transformshift{1.468198in}{1.158216in}%
\pgfsys@useobject{currentmarker}{}%
\end{pgfscope}%
\begin{pgfscope}%
\pgfsys@transformshift{0.730525in}{1.277624in}%
\pgfsys@useobject{currentmarker}{}%
\end{pgfscope}%
\begin{pgfscope}%
\pgfsys@transformshift{0.597968in}{1.304198in}%
\pgfsys@useobject{currentmarker}{}%
\end{pgfscope}%
\begin{pgfscope}%
\pgfsys@transformshift{0.539928in}{1.499077in}%
\pgfsys@useobject{currentmarker}{}%
\end{pgfscope}%
\begin{pgfscope}%
\pgfsys@transformshift{0.782197in}{1.583686in}%
\pgfsys@useobject{currentmarker}{}%
\end{pgfscope}%
\begin{pgfscope}%
\pgfsys@transformshift{0.726193in}{1.274165in}%
\pgfsys@useobject{currentmarker}{}%
\end{pgfscope}%
\begin{pgfscope}%
\pgfsys@transformshift{0.756148in}{1.347438in}%
\pgfsys@useobject{currentmarker}{}%
\end{pgfscope}%
\begin{pgfscope}%
\pgfsys@transformshift{0.663543in}{1.326883in}%
\pgfsys@useobject{currentmarker}{}%
\end{pgfscope}%
\begin{pgfscope}%
\pgfsys@transformshift{1.069286in}{1.393816in}%
\pgfsys@useobject{currentmarker}{}%
\end{pgfscope}%
\begin{pgfscope}%
\pgfsys@transformshift{1.411009in}{1.145347in}%
\pgfsys@useobject{currentmarker}{}%
\end{pgfscope}%
\begin{pgfscope}%
\pgfsys@transformshift{1.079839in}{1.413507in}%
\pgfsys@useobject{currentmarker}{}%
\end{pgfscope}%
\begin{pgfscope}%
\pgfsys@transformshift{1.104758in}{1.133518in}%
\pgfsys@useobject{currentmarker}{}%
\end{pgfscope}%
\begin{pgfscope}%
\pgfsys@transformshift{1.128918in}{1.139582in}%
\pgfsys@useobject{currentmarker}{}%
\end{pgfscope}%
\begin{pgfscope}%
\pgfsys@transformshift{1.010728in}{1.218453in}%
\pgfsys@useobject{currentmarker}{}%
\end{pgfscope}%
\begin{pgfscope}%
\pgfsys@transformshift{0.687426in}{1.355426in}%
\pgfsys@useobject{currentmarker}{}%
\end{pgfscope}%
\begin{pgfscope}%
\pgfsys@transformshift{1.166556in}{1.394366in}%
\pgfsys@useobject{currentmarker}{}%
\end{pgfscope}%
\begin{pgfscope}%
\pgfsys@transformshift{0.539336in}{1.578827in}%
\pgfsys@useobject{currentmarker}{}%
\end{pgfscope}%
\begin{pgfscope}%
\pgfsys@transformshift{1.057604in}{1.076242in}%
\pgfsys@useobject{currentmarker}{}%
\end{pgfscope}%
\begin{pgfscope}%
\pgfsys@transformshift{0.759925in}{1.081668in}%
\pgfsys@useobject{currentmarker}{}%
\end{pgfscope}%
\begin{pgfscope}%
\pgfsys@transformshift{1.233927in}{0.924345in}%
\pgfsys@useobject{currentmarker}{}%
\end{pgfscope}%
\begin{pgfscope}%
\pgfsys@transformshift{0.557127in}{1.159440in}%
\pgfsys@useobject{currentmarker}{}%
\end{pgfscope}%
\begin{pgfscope}%
\pgfsys@transformshift{0.685815in}{1.306441in}%
\pgfsys@useobject{currentmarker}{}%
\end{pgfscope}%
\begin{pgfscope}%
\pgfsys@transformshift{0.794138in}{0.898065in}%
\pgfsys@useobject{currentmarker}{}%
\end{pgfscope}%
\begin{pgfscope}%
\pgfsys@transformshift{0.700941in}{1.204713in}%
\pgfsys@useobject{currentmarker}{}%
\end{pgfscope}%
\begin{pgfscope}%
\pgfsys@transformshift{0.901516in}{1.086748in}%
\pgfsys@useobject{currentmarker}{}%
\end{pgfscope}%
\begin{pgfscope}%
\pgfsys@transformshift{0.999546in}{1.139006in}%
\pgfsys@useobject{currentmarker}{}%
\end{pgfscope}%
\begin{pgfscope}%
\pgfsys@transformshift{1.495820in}{1.460764in}%
\pgfsys@useobject{currentmarker}{}%
\end{pgfscope}%
\begin{pgfscope}%
\pgfsys@transformshift{1.570837in}{1.013770in}%
\pgfsys@useobject{currentmarker}{}%
\end{pgfscope}%
\begin{pgfscope}%
\pgfsys@transformshift{0.930379in}{1.389227in}%
\pgfsys@useobject{currentmarker}{}%
\end{pgfscope}%
\begin{pgfscope}%
\pgfsys@transformshift{0.725045in}{1.251016in}%
\pgfsys@useobject{currentmarker}{}%
\end{pgfscope}%
\begin{pgfscope}%
\pgfsys@transformshift{0.663488in}{1.088204in}%
\pgfsys@useobject{currentmarker}{}%
\end{pgfscope}%
\begin{pgfscope}%
\pgfsys@transformshift{0.729081in}{1.223901in}%
\pgfsys@useobject{currentmarker}{}%
\end{pgfscope}%
\begin{pgfscope}%
\pgfsys@transformshift{0.600893in}{1.237449in}%
\pgfsys@useobject{currentmarker}{}%
\end{pgfscope}%
\begin{pgfscope}%
\pgfsys@transformshift{0.769145in}{1.360095in}%
\pgfsys@useobject{currentmarker}{}%
\end{pgfscope}%
\begin{pgfscope}%
\pgfsys@transformshift{0.677817in}{1.197307in}%
\pgfsys@useobject{currentmarker}{}%
\end{pgfscope}%
\begin{pgfscope}%
\pgfsys@transformshift{0.942968in}{1.372775in}%
\pgfsys@useobject{currentmarker}{}%
\end{pgfscope}%
\begin{pgfscope}%
\pgfsys@transformshift{1.032870in}{1.532118in}%
\pgfsys@useobject{currentmarker}{}%
\end{pgfscope}%
\begin{pgfscope}%
\pgfsys@transformshift{0.670412in}{1.487176in}%
\pgfsys@useobject{currentmarker}{}%
\end{pgfscope}%
\begin{pgfscope}%
\pgfsys@transformshift{0.718584in}{1.299427in}%
\pgfsys@useobject{currentmarker}{}%
\end{pgfscope}%
\begin{pgfscope}%
\pgfsys@transformshift{0.609262in}{1.382666in}%
\pgfsys@useobject{currentmarker}{}%
\end{pgfscope}%
\begin{pgfscope}%
\pgfsys@transformshift{0.724620in}{1.265608in}%
\pgfsys@useobject{currentmarker}{}%
\end{pgfscope}%
\begin{pgfscope}%
\pgfsys@transformshift{0.839274in}{1.065191in}%
\pgfsys@useobject{currentmarker}{}%
\end{pgfscope}%
\begin{pgfscope}%
\pgfsys@transformshift{1.123142in}{1.440514in}%
\pgfsys@useobject{currentmarker}{}%
\end{pgfscope}%
\begin{pgfscope}%
\pgfsys@transformshift{0.660581in}{1.131632in}%
\pgfsys@useobject{currentmarker}{}%
\end{pgfscope}%
\begin{pgfscope}%
\pgfsys@transformshift{1.346619in}{1.433486in}%
\pgfsys@useobject{currentmarker}{}%
\end{pgfscope}%
\begin{pgfscope}%
\pgfsys@transformshift{1.032759in}{1.041283in}%
\pgfsys@useobject{currentmarker}{}%
\end{pgfscope}%
\begin{pgfscope}%
\pgfsys@transformshift{1.531144in}{1.116981in}%
\pgfsys@useobject{currentmarker}{}%
\end{pgfscope}%
\begin{pgfscope}%
\pgfsys@transformshift{0.545427in}{1.487189in}%
\pgfsys@useobject{currentmarker}{}%
\end{pgfscope}%
\begin{pgfscope}%
\pgfsys@transformshift{1.530607in}{1.114842in}%
\pgfsys@useobject{currentmarker}{}%
\end{pgfscope}%
\begin{pgfscope}%
\pgfsys@transformshift{1.562080in}{1.134293in}%
\pgfsys@useobject{currentmarker}{}%
\end{pgfscope}%
\begin{pgfscope}%
\pgfsys@transformshift{1.076266in}{1.707617in}%
\pgfsys@useobject{currentmarker}{}%
\end{pgfscope}%
\begin{pgfscope}%
\pgfsys@transformshift{0.596024in}{1.544937in}%
\pgfsys@useobject{currentmarker}{}%
\end{pgfscope}%
\begin{pgfscope}%
\pgfsys@transformshift{1.283932in}{0.935854in}%
\pgfsys@useobject{currentmarker}{}%
\end{pgfscope}%
\begin{pgfscope}%
\pgfsys@transformshift{1.277638in}{1.354838in}%
\pgfsys@useobject{currentmarker}{}%
\end{pgfscope}%
\begin{pgfscope}%
\pgfsys@transformshift{0.915846in}{1.166676in}%
\pgfsys@useobject{currentmarker}{}%
\end{pgfscope}%
\begin{pgfscope}%
\pgfsys@transformshift{0.580084in}{1.259250in}%
\pgfsys@useobject{currentmarker}{}%
\end{pgfscope}%
\begin{pgfscope}%
\pgfsys@transformshift{0.665210in}{1.229694in}%
\pgfsys@useobject{currentmarker}{}%
\end{pgfscope}%
\begin{pgfscope}%
\pgfsys@transformshift{1.282470in}{1.396569in}%
\pgfsys@useobject{currentmarker}{}%
\end{pgfscope}%
\begin{pgfscope}%
\pgfsys@transformshift{0.656693in}{1.399170in}%
\pgfsys@useobject{currentmarker}{}%
\end{pgfscope}%
\begin{pgfscope}%
\pgfsys@transformshift{1.353432in}{1.291823in}%
\pgfsys@useobject{currentmarker}{}%
\end{pgfscope}%
\begin{pgfscope}%
\pgfsys@transformshift{0.684445in}{1.802147in}%
\pgfsys@useobject{currentmarker}{}%
\end{pgfscope}%
\begin{pgfscope}%
\pgfsys@transformshift{1.918484in}{1.135580in}%
\pgfsys@useobject{currentmarker}{}%
\end{pgfscope}%
\begin{pgfscope}%
\pgfsys@transformshift{1.424728in}{1.161044in}%
\pgfsys@useobject{currentmarker}{}%
\end{pgfscope}%
\begin{pgfscope}%
\pgfsys@transformshift{0.746503in}{1.244255in}%
\pgfsys@useobject{currentmarker}{}%
\end{pgfscope}%
\begin{pgfscope}%
\pgfsys@transformshift{0.827814in}{1.415586in}%
\pgfsys@useobject{currentmarker}{}%
\end{pgfscope}%
\begin{pgfscope}%
\pgfsys@transformshift{1.171537in}{1.052917in}%
\pgfsys@useobject{currentmarker}{}%
\end{pgfscope}%
\begin{pgfscope}%
\pgfsys@transformshift{1.125382in}{1.552377in}%
\pgfsys@useobject{currentmarker}{}%
\end{pgfscope}%
\begin{pgfscope}%
\pgfsys@transformshift{1.248979in}{1.362734in}%
\pgfsys@useobject{currentmarker}{}%
\end{pgfscope}%
\begin{pgfscope}%
\pgfsys@transformshift{0.729433in}{1.444369in}%
\pgfsys@useobject{currentmarker}{}%
\end{pgfscope}%
\begin{pgfscope}%
\pgfsys@transformshift{0.585879in}{1.331831in}%
\pgfsys@useobject{currentmarker}{}%
\end{pgfscope}%
\begin{pgfscope}%
\pgfsys@transformshift{1.040794in}{1.606543in}%
\pgfsys@useobject{currentmarker}{}%
\end{pgfscope}%
\begin{pgfscope}%
\pgfsys@transformshift{0.953003in}{1.428198in}%
\pgfsys@useobject{currentmarker}{}%
\end{pgfscope}%
\begin{pgfscope}%
\pgfsys@transformshift{1.076747in}{1.124199in}%
\pgfsys@useobject{currentmarker}{}%
\end{pgfscope}%
\begin{pgfscope}%
\pgfsys@transformshift{0.880318in}{1.500022in}%
\pgfsys@useobject{currentmarker}{}%
\end{pgfscope}%
\begin{pgfscope}%
\pgfsys@transformshift{1.223782in}{1.580773in}%
\pgfsys@useobject{currentmarker}{}%
\end{pgfscope}%
\begin{pgfscope}%
\pgfsys@transformshift{1.089466in}{1.479829in}%
\pgfsys@useobject{currentmarker}{}%
\end{pgfscope}%
\begin{pgfscope}%
\pgfsys@transformshift{0.826666in}{1.329183in}%
\pgfsys@useobject{currentmarker}{}%
\end{pgfscope}%
\begin{pgfscope}%
\pgfsys@transformshift{1.377851in}{1.267754in}%
\pgfsys@useobject{currentmarker}{}%
\end{pgfscope}%
\begin{pgfscope}%
\pgfsys@transformshift{0.598024in}{1.307373in}%
\pgfsys@useobject{currentmarker}{}%
\end{pgfscope}%
\begin{pgfscope}%
\pgfsys@transformshift{0.697849in}{1.100386in}%
\pgfsys@useobject{currentmarker}{}%
\end{pgfscope}%
\begin{pgfscope}%
\pgfsys@transformshift{1.266270in}{1.327427in}%
\pgfsys@useobject{currentmarker}{}%
\end{pgfscope}%
\begin{pgfscope}%
\pgfsys@transformshift{0.769293in}{1.243455in}%
\pgfsys@useobject{currentmarker}{}%
\end{pgfscope}%
\begin{pgfscope}%
\pgfsys@transformshift{0.763591in}{1.111571in}%
\pgfsys@useobject{currentmarker}{}%
\end{pgfscope}%
\begin{pgfscope}%
\pgfsys@transformshift{1.545270in}{1.104664in}%
\pgfsys@useobject{currentmarker}{}%
\end{pgfscope}%
\begin{pgfscope}%
\pgfsys@transformshift{0.781974in}{1.107055in}%
\pgfsys@useobject{currentmarker}{}%
\end{pgfscope}%
\begin{pgfscope}%
\pgfsys@transformshift{0.637143in}{1.255589in}%
\pgfsys@useobject{currentmarker}{}%
\end{pgfscope}%
\begin{pgfscope}%
\pgfsys@transformshift{1.059196in}{1.211835in}%
\pgfsys@useobject{currentmarker}{}%
\end{pgfscope}%
\begin{pgfscope}%
\pgfsys@transformshift{1.032796in}{1.690414in}%
\pgfsys@useobject{currentmarker}{}%
\end{pgfscope}%
\begin{pgfscope}%
\pgfsys@transformshift{0.683742in}{1.407216in}%
\pgfsys@useobject{currentmarker}{}%
\end{pgfscope}%
\begin{pgfscope}%
\pgfsys@transformshift{1.021114in}{1.487228in}%
\pgfsys@useobject{currentmarker}{}%
\end{pgfscope}%
\begin{pgfscope}%
\pgfsys@transformshift{0.563237in}{1.336804in}%
\pgfsys@useobject{currentmarker}{}%
\end{pgfscope}%
\begin{pgfscope}%
\pgfsys@transformshift{0.614223in}{1.319530in}%
\pgfsys@useobject{currentmarker}{}%
\end{pgfscope}%
\begin{pgfscope}%
\pgfsys@transformshift{0.779401in}{1.265825in}%
\pgfsys@useobject{currentmarker}{}%
\end{pgfscope}%
\begin{pgfscope}%
\pgfsys@transformshift{0.992603in}{1.503047in}%
\pgfsys@useobject{currentmarker}{}%
\end{pgfscope}%
\begin{pgfscope}%
\pgfsys@transformshift{1.694285in}{1.018046in}%
\pgfsys@useobject{currentmarker}{}%
\end{pgfscope}%
\begin{pgfscope}%
\pgfsys@transformshift{0.778624in}{0.980968in}%
\pgfsys@useobject{currentmarker}{}%
\end{pgfscope}%
\begin{pgfscope}%
\pgfsys@transformshift{1.340343in}{1.372566in}%
\pgfsys@useobject{currentmarker}{}%
\end{pgfscope}%
\begin{pgfscope}%
\pgfsys@transformshift{0.923566in}{1.342397in}%
\pgfsys@useobject{currentmarker}{}%
\end{pgfscope}%
\begin{pgfscope}%
\pgfsys@transformshift{0.789621in}{0.854307in}%
\pgfsys@useobject{currentmarker}{}%
\end{pgfscope}%
\begin{pgfscope}%
\pgfsys@transformshift{0.869895in}{0.936092in}%
\pgfsys@useobject{currentmarker}{}%
\end{pgfscope}%
\begin{pgfscope}%
\pgfsys@transformshift{1.119513in}{1.015511in}%
\pgfsys@useobject{currentmarker}{}%
\end{pgfscope}%
\begin{pgfscope}%
\pgfsys@transformshift{1.521480in}{1.459924in}%
\pgfsys@useobject{currentmarker}{}%
\end{pgfscope}%
\begin{pgfscope}%
\pgfsys@transformshift{1.492025in}{1.286860in}%
\pgfsys@useobject{currentmarker}{}%
\end{pgfscope}%
\begin{pgfscope}%
\pgfsys@transformshift{1.103758in}{1.798598in}%
\pgfsys@useobject{currentmarker}{}%
\end{pgfscope}%
\begin{pgfscope}%
\pgfsys@transformshift{0.698127in}{1.196871in}%
\pgfsys@useobject{currentmarker}{}%
\end{pgfscope}%
\begin{pgfscope}%
\pgfsys@transformshift{1.029871in}{1.176055in}%
\pgfsys@useobject{currentmarker}{}%
\end{pgfscope}%
\begin{pgfscope}%
\pgfsys@transformshift{0.931582in}{1.410652in}%
\pgfsys@useobject{currentmarker}{}%
\end{pgfscope}%
\begin{pgfscope}%
\pgfsys@transformshift{0.910792in}{1.193712in}%
\pgfsys@useobject{currentmarker}{}%
\end{pgfscope}%
\begin{pgfscope}%
\pgfsys@transformshift{0.605355in}{1.240114in}%
\pgfsys@useobject{currentmarker}{}%
\end{pgfscope}%
\begin{pgfscope}%
\pgfsys@transformshift{1.054013in}{1.510325in}%
\pgfsys@useobject{currentmarker}{}%
\end{pgfscope}%
\begin{pgfscope}%
\pgfsys@transformshift{1.098612in}{0.998229in}%
\pgfsys@useobject{currentmarker}{}%
\end{pgfscope}%
\begin{pgfscope}%
\pgfsys@transformshift{0.861509in}{1.356086in}%
\pgfsys@useobject{currentmarker}{}%
\end{pgfscope}%
\begin{pgfscope}%
\pgfsys@transformshift{0.582121in}{1.643886in}%
\pgfsys@useobject{currentmarker}{}%
\end{pgfscope}%
\begin{pgfscope}%
\pgfsys@transformshift{1.167241in}{0.981147in}%
\pgfsys@useobject{currentmarker}{}%
\end{pgfscope}%
\begin{pgfscope}%
\pgfsys@transformshift{1.281729in}{1.372194in}%
\pgfsys@useobject{currentmarker}{}%
\end{pgfscope}%
\begin{pgfscope}%
\pgfsys@transformshift{1.375426in}{1.434751in}%
\pgfsys@useobject{currentmarker}{}%
\end{pgfscope}%
\begin{pgfscope}%
\pgfsys@transformshift{1.374241in}{1.324574in}%
\pgfsys@useobject{currentmarker}{}%
\end{pgfscope}%
\begin{pgfscope}%
\pgfsys@transformshift{0.897036in}{1.340960in}%
\pgfsys@useobject{currentmarker}{}%
\end{pgfscope}%
\begin{pgfscope}%
\pgfsys@transformshift{1.073507in}{1.437814in}%
\pgfsys@useobject{currentmarker}{}%
\end{pgfscope}%
\begin{pgfscope}%
\pgfsys@transformshift{0.689259in}{1.523587in}%
\pgfsys@useobject{currentmarker}{}%
\end{pgfscope}%
\begin{pgfscope}%
\pgfsys@transformshift{1.117181in}{1.020754in}%
\pgfsys@useobject{currentmarker}{}%
\end{pgfscope}%
\begin{pgfscope}%
\pgfsys@transformshift{1.040257in}{1.503545in}%
\pgfsys@useobject{currentmarker}{}%
\end{pgfscope}%
\begin{pgfscope}%
\pgfsys@transformshift{0.757259in}{1.584834in}%
\pgfsys@useobject{currentmarker}{}%
\end{pgfscope}%
\begin{pgfscope}%
\pgfsys@transformshift{0.929602in}{1.696023in}%
\pgfsys@useobject{currentmarker}{}%
\end{pgfscope}%
\begin{pgfscope}%
\pgfsys@transformshift{0.739375in}{1.009799in}%
\pgfsys@useobject{currentmarker}{}%
\end{pgfscope}%
\begin{pgfscope}%
\pgfsys@transformshift{1.051791in}{0.951407in}%
\pgfsys@useobject{currentmarker}{}%
\end{pgfscope}%
\begin{pgfscope}%
\pgfsys@transformshift{1.033888in}{1.192570in}%
\pgfsys@useobject{currentmarker}{}%
\end{pgfscope}%
\begin{pgfscope}%
\pgfsys@transformshift{1.087559in}{0.959076in}%
\pgfsys@useobject{currentmarker}{}%
\end{pgfscope}%
\begin{pgfscope}%
\pgfsys@transformshift{1.159095in}{1.018882in}%
\pgfsys@useobject{currentmarker}{}%
\end{pgfscope}%
\begin{pgfscope}%
\pgfsys@transformshift{0.813373in}{1.366194in}%
\pgfsys@useobject{currentmarker}{}%
\end{pgfscope}%
\begin{pgfscope}%
\pgfsys@transformshift{1.640577in}{1.145850in}%
\pgfsys@useobject{currentmarker}{}%
\end{pgfscope}%
\begin{pgfscope}%
\pgfsys@transformshift{2.178211in}{1.113858in}%
\pgfsys@useobject{currentmarker}{}%
\end{pgfscope}%
\begin{pgfscope}%
\pgfsys@transformshift{0.693850in}{1.325539in}%
\pgfsys@useobject{currentmarker}{}%
\end{pgfscope}%
\begin{pgfscope}%
\pgfsys@transformshift{1.166519in}{0.986526in}%
\pgfsys@useobject{currentmarker}{}%
\end{pgfscope}%
\begin{pgfscope}%
\pgfsys@transformshift{0.643734in}{1.211684in}%
\pgfsys@useobject{currentmarker}{}%
\end{pgfscope}%
\begin{pgfscope}%
\pgfsys@transformshift{0.891278in}{1.265536in}%
\pgfsys@useobject{currentmarker}{}%
\end{pgfscope}%
\begin{pgfscope}%
\pgfsys@transformshift{0.789639in}{0.971254in}%
\pgfsys@useobject{currentmarker}{}%
\end{pgfscope}%
\begin{pgfscope}%
\pgfsys@transformshift{0.905830in}{1.274840in}%
\pgfsys@useobject{currentmarker}{}%
\end{pgfscope}%
\begin{pgfscope}%
\pgfsys@transformshift{1.170666in}{1.364627in}%
\pgfsys@useobject{currentmarker}{}%
\end{pgfscope}%
\begin{pgfscope}%
\pgfsys@transformshift{0.851123in}{1.413560in}%
\pgfsys@useobject{currentmarker}{}%
\end{pgfscope}%
\begin{pgfscope}%
\pgfsys@transformshift{1.511890in}{1.538266in}%
\pgfsys@useobject{currentmarker}{}%
\end{pgfscope}%
\begin{pgfscope}%
\pgfsys@transformshift{1.126178in}{1.071874in}%
\pgfsys@useobject{currentmarker}{}%
\end{pgfscope}%
\begin{pgfscope}%
\pgfsys@transformshift{0.794434in}{1.428666in}%
\pgfsys@useobject{currentmarker}{}%
\end{pgfscope}%
\begin{pgfscope}%
\pgfsys@transformshift{1.144285in}{0.966649in}%
\pgfsys@useobject{currentmarker}{}%
\end{pgfscope}%
\begin{pgfscope}%
\pgfsys@transformshift{0.668153in}{1.327058in}%
\pgfsys@useobject{currentmarker}{}%
\end{pgfscope}%
\begin{pgfscope}%
\pgfsys@transformshift{0.919012in}{1.207504in}%
\pgfsys@useobject{currentmarker}{}%
\end{pgfscope}%
\begin{pgfscope}%
\pgfsys@transformshift{1.167260in}{1.417347in}%
\pgfsys@useobject{currentmarker}{}%
\end{pgfscope}%
\begin{pgfscope}%
\pgfsys@transformshift{0.990696in}{1.295012in}%
\pgfsys@useobject{currentmarker}{}%
\end{pgfscope}%
\begin{pgfscope}%
\pgfsys@transformshift{0.884725in}{1.054337in}%
\pgfsys@useobject{currentmarker}{}%
\end{pgfscope}%
\begin{pgfscope}%
\pgfsys@transformshift{0.825592in}{1.226846in}%
\pgfsys@useobject{currentmarker}{}%
\end{pgfscope}%
\begin{pgfscope}%
\pgfsys@transformshift{0.824667in}{1.110169in}%
\pgfsys@useobject{currentmarker}{}%
\end{pgfscope}%
\begin{pgfscope}%
\pgfsys@transformshift{1.211526in}{1.468903in}%
\pgfsys@useobject{currentmarker}{}%
\end{pgfscope}%
\begin{pgfscope}%
\pgfsys@transformshift{1.235205in}{1.311479in}%
\pgfsys@useobject{currentmarker}{}%
\end{pgfscope}%
\begin{pgfscope}%
\pgfsys@transformshift{1.055049in}{1.343590in}%
\pgfsys@useobject{currentmarker}{}%
\end{pgfscope}%
\begin{pgfscope}%
\pgfsys@transformshift{1.121531in}{1.590511in}%
\pgfsys@useobject{currentmarker}{}%
\end{pgfscope}%
\begin{pgfscope}%
\pgfsys@transformshift{1.293522in}{0.908316in}%
\pgfsys@useobject{currentmarker}{}%
\end{pgfscope}%
\begin{pgfscope}%
\pgfsys@transformshift{0.598931in}{1.172880in}%
\pgfsys@useobject{currentmarker}{}%
\end{pgfscope}%
\begin{pgfscope}%
\pgfsys@transformshift{0.969776in}{1.267487in}%
\pgfsys@useobject{currentmarker}{}%
\end{pgfscope}%
\begin{pgfscope}%
\pgfsys@transformshift{1.073637in}{1.256320in}%
\pgfsys@useobject{currentmarker}{}%
\end{pgfscope}%
\begin{pgfscope}%
\pgfsys@transformshift{1.012061in}{1.493655in}%
\pgfsys@useobject{currentmarker}{}%
\end{pgfscope}%
\begin{pgfscope}%
\pgfsys@transformshift{0.981662in}{1.175893in}%
\pgfsys@useobject{currentmarker}{}%
\end{pgfscope}%
\begin{pgfscope}%
\pgfsys@transformshift{0.639994in}{1.277231in}%
\pgfsys@useobject{currentmarker}{}%
\end{pgfscope}%
\begin{pgfscope}%
\pgfsys@transformshift{0.958131in}{1.388406in}%
\pgfsys@useobject{currentmarker}{}%
\end{pgfscope}%
\begin{pgfscope}%
\pgfsys@transformshift{0.822908in}{1.064426in}%
\pgfsys@useobject{currentmarker}{}%
\end{pgfscope}%
\begin{pgfscope}%
\pgfsys@transformshift{0.720176in}{1.036510in}%
\pgfsys@useobject{currentmarker}{}%
\end{pgfscope}%
\begin{pgfscope}%
\pgfsys@transformshift{0.636847in}{1.184661in}%
\pgfsys@useobject{currentmarker}{}%
\end{pgfscope}%
\begin{pgfscope}%
\pgfsys@transformshift{0.850863in}{1.298045in}%
\pgfsys@useobject{currentmarker}{}%
\end{pgfscope}%
\begin{pgfscope}%
\pgfsys@transformshift{1.514574in}{1.379886in}%
\pgfsys@useobject{currentmarker}{}%
\end{pgfscope}%
\begin{pgfscope}%
\pgfsys@transformshift{0.970535in}{1.364977in}%
\pgfsys@useobject{currentmarker}{}%
\end{pgfscope}%
\begin{pgfscope}%
\pgfsys@transformshift{0.976256in}{1.534550in}%
\pgfsys@useobject{currentmarker}{}%
\end{pgfscope}%
\begin{pgfscope}%
\pgfsys@transformshift{0.879633in}{1.589040in}%
\pgfsys@useobject{currentmarker}{}%
\end{pgfscope}%
\begin{pgfscope}%
\pgfsys@transformshift{0.655508in}{1.134748in}%
\pgfsys@useobject{currentmarker}{}%
\end{pgfscope}%
\begin{pgfscope}%
\pgfsys@transformshift{0.702514in}{1.220501in}%
\pgfsys@useobject{currentmarker}{}%
\end{pgfscope}%
\begin{pgfscope}%
\pgfsys@transformshift{0.714474in}{1.061012in}%
\pgfsys@useobject{currentmarker}{}%
\end{pgfscope}%
\begin{pgfscope}%
\pgfsys@transformshift{1.412398in}{1.142199in}%
\pgfsys@useobject{currentmarker}{}%
\end{pgfscope}%
\begin{pgfscope}%
\pgfsys@transformshift{1.041442in}{1.473594in}%
\pgfsys@useobject{currentmarker}{}%
\end{pgfscope}%
\begin{pgfscope}%
\pgfsys@transformshift{0.890575in}{1.322248in}%
\pgfsys@useobject{currentmarker}{}%
\end{pgfscope}%
\begin{pgfscope}%
\pgfsys@transformshift{1.229854in}{1.051123in}%
\pgfsys@useobject{currentmarker}{}%
\end{pgfscope}%
\begin{pgfscope}%
\pgfsys@transformshift{1.163502in}{1.394977in}%
\pgfsys@useobject{currentmarker}{}%
\end{pgfscope}%
\begin{pgfscope}%
\pgfsys@transformshift{0.904941in}{1.500006in}%
\pgfsys@useobject{currentmarker}{}%
\end{pgfscope}%
\begin{pgfscope}%
\pgfsys@transformshift{0.963778in}{1.620497in}%
\pgfsys@useobject{currentmarker}{}%
\end{pgfscope}%
\begin{pgfscope}%
\pgfsys@transformshift{1.162317in}{1.077287in}%
\pgfsys@useobject{currentmarker}{}%
\end{pgfscope}%
\begin{pgfscope}%
\pgfsys@transformshift{0.708402in}{1.043893in}%
\pgfsys@useobject{currentmarker}{}%
\end{pgfscope}%
\begin{pgfscope}%
\pgfsys@transformshift{0.704162in}{1.362767in}%
\pgfsys@useobject{currentmarker}{}%
\end{pgfscope}%
\begin{pgfscope}%
\pgfsys@transformshift{1.154782in}{0.891759in}%
\pgfsys@useobject{currentmarker}{}%
\end{pgfscope}%
\begin{pgfscope}%
\pgfsys@transformshift{0.978699in}{1.030414in}%
\pgfsys@useobject{currentmarker}{}%
\end{pgfscope}%
\begin{pgfscope}%
\pgfsys@transformshift{1.451036in}{1.515774in}%
\pgfsys@useobject{currentmarker}{}%
\end{pgfscope}%
\begin{pgfscope}%
\pgfsys@transformshift{1.034425in}{1.488317in}%
\pgfsys@useobject{currentmarker}{}%
\end{pgfscope}%
\begin{pgfscope}%
\pgfsys@transformshift{0.615926in}{1.320209in}%
\pgfsys@useobject{currentmarker}{}%
\end{pgfscope}%
\begin{pgfscope}%
\pgfsys@transformshift{1.270158in}{1.242469in}%
\pgfsys@useobject{currentmarker}{}%
\end{pgfscope}%
\begin{pgfscope}%
\pgfsys@transformshift{0.708883in}{1.036971in}%
\pgfsys@useobject{currentmarker}{}%
\end{pgfscope}%
\begin{pgfscope}%
\pgfsys@transformshift{0.740763in}{1.021894in}%
\pgfsys@useobject{currentmarker}{}%
\end{pgfscope}%
\begin{pgfscope}%
\pgfsys@transformshift{0.614020in}{1.190373in}%
\pgfsys@useobject{currentmarker}{}%
\end{pgfscope}%
\begin{pgfscope}%
\pgfsys@transformshift{1.067953in}{1.356221in}%
\pgfsys@useobject{currentmarker}{}%
\end{pgfscope}%
\begin{pgfscope}%
\pgfsys@transformshift{0.777420in}{1.294564in}%
\pgfsys@useobject{currentmarker}{}%
\end{pgfscope}%
\begin{pgfscope}%
\pgfsys@transformshift{0.653398in}{1.208039in}%
\pgfsys@useobject{currentmarker}{}%
\end{pgfscope}%
\begin{pgfscope}%
\pgfsys@transformshift{0.733321in}{1.258801in}%
\pgfsys@useobject{currentmarker}{}%
\end{pgfscope}%
\begin{pgfscope}%
\pgfsys@transformshift{0.711753in}{1.387694in}%
\pgfsys@useobject{currentmarker}{}%
\end{pgfscope}%
\begin{pgfscope}%
\pgfsys@transformshift{1.082709in}{1.621161in}%
\pgfsys@useobject{currentmarker}{}%
\end{pgfscope}%
\begin{pgfscope}%
\pgfsys@transformshift{1.677845in}{1.038873in}%
\pgfsys@useobject{currentmarker}{}%
\end{pgfscope}%
\begin{pgfscope}%
\pgfsys@transformshift{0.696497in}{1.173457in}%
\pgfsys@useobject{currentmarker}{}%
\end{pgfscope}%
\begin{pgfscope}%
\pgfsys@transformshift{0.757129in}{1.192770in}%
\pgfsys@useobject{currentmarker}{}%
\end{pgfscope}%
\begin{pgfscope}%
\pgfsys@transformshift{1.072767in}{1.680303in}%
\pgfsys@useobject{currentmarker}{}%
\end{pgfscope}%
\begin{pgfscope}%
\pgfsys@transformshift{1.280452in}{1.510441in}%
\pgfsys@useobject{currentmarker}{}%
\end{pgfscope}%
\begin{pgfscope}%
\pgfsys@transformshift{0.830258in}{1.140416in}%
\pgfsys@useobject{currentmarker}{}%
\end{pgfscope}%
\begin{pgfscope}%
\pgfsys@transformshift{0.709420in}{1.035217in}%
\pgfsys@useobject{currentmarker}{}%
\end{pgfscope}%
\begin{pgfscope}%
\pgfsys@transformshift{1.313295in}{1.366080in}%
\pgfsys@useobject{currentmarker}{}%
\end{pgfscope}%
\begin{pgfscope}%
\pgfsys@transformshift{0.966851in}{1.384024in}%
\pgfsys@useobject{currentmarker}{}%
\end{pgfscope}%
\begin{pgfscope}%
\pgfsys@transformshift{0.979088in}{1.593780in}%
\pgfsys@useobject{currentmarker}{}%
\end{pgfscope}%
\begin{pgfscope}%
\pgfsys@transformshift{0.852955in}{1.045690in}%
\pgfsys@useobject{currentmarker}{}%
\end{pgfscope}%
\begin{pgfscope}%
\pgfsys@transformshift{0.752519in}{1.179665in}%
\pgfsys@useobject{currentmarker}{}%
\end{pgfscope}%
\begin{pgfscope}%
\pgfsys@transformshift{0.942839in}{1.440022in}%
\pgfsys@useobject{currentmarker}{}%
\end{pgfscope}%
\begin{pgfscope}%
\pgfsys@transformshift{0.747724in}{1.409228in}%
\pgfsys@useobject{currentmarker}{}%
\end{pgfscope}%
\begin{pgfscope}%
\pgfsys@transformshift{1.116200in}{1.452307in}%
\pgfsys@useobject{currentmarker}{}%
\end{pgfscope}%
\begin{pgfscope}%
\pgfsys@transformshift{0.695424in}{1.090403in}%
\pgfsys@useobject{currentmarker}{}%
\end{pgfscope}%
\begin{pgfscope}%
\pgfsys@transformshift{1.740661in}{1.343367in}%
\pgfsys@useobject{currentmarker}{}%
\end{pgfscope}%
\begin{pgfscope}%
\pgfsys@transformshift{0.688629in}{1.451174in}%
\pgfsys@useobject{currentmarker}{}%
\end{pgfscope}%
\begin{pgfscope}%
\pgfsys@transformshift{0.892574in}{1.675103in}%
\pgfsys@useobject{currentmarker}{}%
\end{pgfscope}%
\begin{pgfscope}%
\pgfsys@transformshift{0.811004in}{1.224893in}%
\pgfsys@useobject{currentmarker}{}%
\end{pgfscope}%
\begin{pgfscope}%
\pgfsys@transformshift{1.098889in}{1.484733in}%
\pgfsys@useobject{currentmarker}{}%
\end{pgfscope}%
\begin{pgfscope}%
\pgfsys@transformshift{0.812874in}{1.015281in}%
\pgfsys@useobject{currentmarker}{}%
\end{pgfscope}%
\begin{pgfscope}%
\pgfsys@transformshift{0.678317in}{1.332301in}%
\pgfsys@useobject{currentmarker}{}%
\end{pgfscope}%
\begin{pgfscope}%
\pgfsys@transformshift{1.330735in}{1.083886in}%
\pgfsys@useobject{currentmarker}{}%
\end{pgfscope}%
\begin{pgfscope}%
\pgfsys@transformshift{0.845531in}{0.965811in}%
\pgfsys@useobject{currentmarker}{}%
\end{pgfscope}%
\begin{pgfscope}%
\pgfsys@transformshift{1.154911in}{1.424856in}%
\pgfsys@useobject{currentmarker}{}%
\end{pgfscope}%
\begin{pgfscope}%
\pgfsys@transformshift{0.984050in}{1.919500in}%
\pgfsys@useobject{currentmarker}{}%
\end{pgfscope}%
\begin{pgfscope}%
\pgfsys@transformshift{0.736913in}{1.278097in}%
\pgfsys@useobject{currentmarker}{}%
\end{pgfscope}%
\begin{pgfscope}%
\pgfsys@transformshift{0.696775in}{1.104952in}%
\pgfsys@useobject{currentmarker}{}%
\end{pgfscope}%
\begin{pgfscope}%
\pgfsys@transformshift{0.594914in}{1.301553in}%
\pgfsys@useobject{currentmarker}{}%
\end{pgfscope}%
\begin{pgfscope}%
\pgfsys@transformshift{1.029056in}{1.189274in}%
\pgfsys@useobject{currentmarker}{}%
\end{pgfscope}%
\begin{pgfscope}%
\pgfsys@transformshift{1.353377in}{1.474064in}%
\pgfsys@useobject{currentmarker}{}%
\end{pgfscope}%
\begin{pgfscope}%
\pgfsys@transformshift{0.844328in}{1.356547in}%
\pgfsys@useobject{currentmarker}{}%
\end{pgfscope}%
\begin{pgfscope}%
\pgfsys@transformshift{0.897277in}{1.383804in}%
\pgfsys@useobject{currentmarker}{}%
\end{pgfscope}%
\begin{pgfscope}%
\pgfsys@transformshift{0.995325in}{1.152074in}%
\pgfsys@useobject{currentmarker}{}%
\end{pgfscope}%
\begin{pgfscope}%
\pgfsys@transformshift{1.056327in}{1.136860in}%
\pgfsys@useobject{currentmarker}{}%
\end{pgfscope}%
\begin{pgfscope}%
\pgfsys@transformshift{0.564107in}{1.287745in}%
\pgfsys@useobject{currentmarker}{}%
\end{pgfscope}%
\begin{pgfscope}%
\pgfsys@transformshift{0.656175in}{1.317594in}%
\pgfsys@useobject{currentmarker}{}%
\end{pgfscope}%
\begin{pgfscope}%
\pgfsys@transformshift{1.144433in}{1.111158in}%
\pgfsys@useobject{currentmarker}{}%
\end{pgfscope}%
\begin{pgfscope}%
\pgfsys@transformshift{0.628238in}{1.211932in}%
\pgfsys@useobject{currentmarker}{}%
\end{pgfscope}%
\begin{pgfscope}%
\pgfsys@transformshift{1.040257in}{1.150722in}%
\pgfsys@useobject{currentmarker}{}%
\end{pgfscope}%
\begin{pgfscope}%
\pgfsys@transformshift{0.967406in}{1.307453in}%
\pgfsys@useobject{currentmarker}{}%
\end{pgfscope}%
\begin{pgfscope}%
\pgfsys@transformshift{1.329216in}{1.344238in}%
\pgfsys@useobject{currentmarker}{}%
\end{pgfscope}%
\begin{pgfscope}%
\pgfsys@transformshift{0.564607in}{1.314939in}%
\pgfsys@useobject{currentmarker}{}%
\end{pgfscope}%
\begin{pgfscope}%
\pgfsys@transformshift{0.742263in}{1.360092in}%
\pgfsys@useobject{currentmarker}{}%
\end{pgfscope}%
\begin{pgfscope}%
\pgfsys@transformshift{0.629367in}{1.319179in}%
\pgfsys@useobject{currentmarker}{}%
\end{pgfscope}%
\begin{pgfscope}%
\pgfsys@transformshift{0.569846in}{1.108206in}%
\pgfsys@useobject{currentmarker}{}%
\end{pgfscope}%
\begin{pgfscope}%
\pgfsys@transformshift{1.470660in}{1.225838in}%
\pgfsys@useobject{currentmarker}{}%
\end{pgfscope}%
\begin{pgfscope}%
\pgfsys@transformshift{1.976543in}{1.043586in}%
\pgfsys@useobject{currentmarker}{}%
\end{pgfscope}%
\begin{pgfscope}%
\pgfsys@transformshift{1.014486in}{1.509053in}%
\pgfsys@useobject{currentmarker}{}%
\end{pgfscope}%
\begin{pgfscope}%
\pgfsys@transformshift{1.379055in}{1.357980in}%
\pgfsys@useobject{currentmarker}{}%
\end{pgfscope}%
\begin{pgfscope}%
\pgfsys@transformshift{0.658063in}{1.543881in}%
\pgfsys@useobject{currentmarker}{}%
\end{pgfscope}%
\begin{pgfscope}%
\pgfsys@transformshift{0.965129in}{1.551337in}%
\pgfsys@useobject{currentmarker}{}%
\end{pgfscope}%
\begin{pgfscope}%
\pgfsys@transformshift{0.714733in}{1.232778in}%
\pgfsys@useobject{currentmarker}{}%
\end{pgfscope}%
\begin{pgfscope}%
\pgfsys@transformshift{0.734580in}{1.583328in}%
\pgfsys@useobject{currentmarker}{}%
\end{pgfscope}%
\begin{pgfscope}%
\pgfsys@transformshift{1.146876in}{1.082811in}%
\pgfsys@useobject{currentmarker}{}%
\end{pgfscope}%
\begin{pgfscope}%
\pgfsys@transformshift{1.112552in}{1.399638in}%
\pgfsys@useobject{currentmarker}{}%
\end{pgfscope}%
\begin{pgfscope}%
\pgfsys@transformshift{0.609724in}{1.266322in}%
\pgfsys@useobject{currentmarker}{}%
\end{pgfscope}%
\begin{pgfscope}%
\pgfsys@transformshift{0.713974in}{1.248826in}%
\pgfsys@useobject{currentmarker}{}%
\end{pgfscope}%
\begin{pgfscope}%
\pgfsys@transformshift{0.625054in}{1.100152in}%
\pgfsys@useobject{currentmarker}{}%
\end{pgfscope}%
\begin{pgfscope}%
\pgfsys@transformshift{0.695701in}{1.342011in}%
\pgfsys@useobject{currentmarker}{}%
\end{pgfscope}%
\begin{pgfscope}%
\pgfsys@transformshift{0.630052in}{0.847914in}%
\pgfsys@useobject{currentmarker}{}%
\end{pgfscope}%
\begin{pgfscope}%
\pgfsys@transformshift{1.032148in}{1.123020in}%
\pgfsys@useobject{currentmarker}{}%
\end{pgfscope}%
\begin{pgfscope}%
\pgfsys@transformshift{0.974293in}{1.630512in}%
\pgfsys@useobject{currentmarker}{}%
\end{pgfscope}%
\begin{pgfscope}%
\pgfsys@transformshift{0.679965in}{1.154137in}%
\pgfsys@useobject{currentmarker}{}%
\end{pgfscope}%
\begin{pgfscope}%
\pgfsys@transformshift{1.437521in}{1.285719in}%
\pgfsys@useobject{currentmarker}{}%
\end{pgfscope}%
\begin{pgfscope}%
\pgfsys@transformshift{0.779957in}{1.184057in}%
\pgfsys@useobject{currentmarker}{}%
\end{pgfscope}%
\begin{pgfscope}%
\pgfsys@transformshift{1.260216in}{1.304829in}%
\pgfsys@useobject{currentmarker}{}%
\end{pgfscope}%
\begin{pgfscope}%
\pgfsys@transformshift{0.783159in}{1.196029in}%
\pgfsys@useobject{currentmarker}{}%
\end{pgfscope}%
\begin{pgfscope}%
\pgfsys@transformshift{0.627534in}{1.445371in}%
\pgfsys@useobject{currentmarker}{}%
\end{pgfscope}%
\begin{pgfscope}%
\pgfsys@transformshift{0.732803in}{1.055701in}%
\pgfsys@useobject{currentmarker}{}%
\end{pgfscope}%
\begin{pgfscope}%
\pgfsys@transformshift{1.221153in}{1.321396in}%
\pgfsys@useobject{currentmarker}{}%
\end{pgfscope}%
\begin{pgfscope}%
\pgfsys@transformshift{0.997343in}{1.184475in}%
\pgfsys@useobject{currentmarker}{}%
\end{pgfscope}%
\begin{pgfscope}%
\pgfsys@transformshift{0.699904in}{1.354958in}%
\pgfsys@useobject{currentmarker}{}%
\end{pgfscope}%
\begin{pgfscope}%
\pgfsys@transformshift{0.739856in}{1.217003in}%
\pgfsys@useobject{currentmarker}{}%
\end{pgfscope}%
\begin{pgfscope}%
\pgfsys@transformshift{0.561941in}{1.617022in}%
\pgfsys@useobject{currentmarker}{}%
\end{pgfscope}%
\begin{pgfscope}%
\pgfsys@transformshift{1.009173in}{1.140469in}%
\pgfsys@useobject{currentmarker}{}%
\end{pgfscope}%
\begin{pgfscope}%
\pgfsys@transformshift{0.728952in}{1.182862in}%
\pgfsys@useobject{currentmarker}{}%
\end{pgfscope}%
\begin{pgfscope}%
\pgfsys@transformshift{1.463440in}{1.464669in}%
\pgfsys@useobject{currentmarker}{}%
\end{pgfscope}%
\begin{pgfscope}%
\pgfsys@transformshift{0.638106in}{1.025806in}%
\pgfsys@useobject{currentmarker}{}%
\end{pgfscope}%
\begin{pgfscope}%
\pgfsys@transformshift{0.986864in}{1.618426in}%
\pgfsys@useobject{currentmarker}{}%
\end{pgfscope}%
\end{pgfscope}%
\begin{pgfscope}%
\pgfpathrectangle{\pgfqpoint{0.526284in}{0.473557in}}{\pgfqpoint{1.651927in}{1.704653in}}%
\pgfusepath{clip}%
\pgfsetbuttcap%
\pgfsetroundjoin%
\definecolor{currentfill}{rgb}{0.839216,0.152941,0.156863}%
\pgfsetfillcolor{currentfill}%
\pgfsetfillopacity{0.150000}%
\pgfsetlinewidth{0.000000pt}%
\definecolor{currentstroke}{rgb}{0.000000,0.000000,0.000000}%
\pgfsetstrokecolor{currentstroke}%
\pgfsetdash{}{0pt}%
\pgfpathmoveto{\pgfqpoint{0.526284in}{1.308098in}}%
\pgfpathlineto{\pgfqpoint{0.526284in}{1.258259in}}%
\pgfpathlineto{\pgfqpoint{0.542970in}{1.258964in}}%
\pgfpathlineto{\pgfqpoint{0.559656in}{1.259536in}}%
\pgfpathlineto{\pgfqpoint{0.576342in}{1.260140in}}%
\pgfpathlineto{\pgfqpoint{0.593028in}{1.261133in}}%
\pgfpathlineto{\pgfqpoint{0.609715in}{1.262153in}}%
\pgfpathlineto{\pgfqpoint{0.626401in}{1.262935in}}%
\pgfpathlineto{\pgfqpoint{0.643087in}{1.263531in}}%
\pgfpathlineto{\pgfqpoint{0.659773in}{1.264335in}}%
\pgfpathlineto{\pgfqpoint{0.676459in}{1.265227in}}%
\pgfpathlineto{\pgfqpoint{0.693145in}{1.265907in}}%
\pgfpathlineto{\pgfqpoint{0.709831in}{1.266345in}}%
\pgfpathlineto{\pgfqpoint{0.726517in}{1.267101in}}%
\pgfpathlineto{\pgfqpoint{0.743204in}{1.267721in}}%
\pgfpathlineto{\pgfqpoint{0.759890in}{1.268282in}}%
\pgfpathlineto{\pgfqpoint{0.776576in}{1.268703in}}%
\pgfpathlineto{\pgfqpoint{0.793262in}{1.269405in}}%
\pgfpathlineto{\pgfqpoint{0.809948in}{1.269959in}}%
\pgfpathlineto{\pgfqpoint{0.826634in}{1.270505in}}%
\pgfpathlineto{\pgfqpoint{0.843320in}{1.270379in}}%
\pgfpathlineto{\pgfqpoint{0.860006in}{1.270223in}}%
\pgfpathlineto{\pgfqpoint{0.876693in}{1.271074in}}%
\pgfpathlineto{\pgfqpoint{0.893379in}{1.271507in}}%
\pgfpathlineto{\pgfqpoint{0.910065in}{1.271548in}}%
\pgfpathlineto{\pgfqpoint{0.926751in}{1.271628in}}%
\pgfpathlineto{\pgfqpoint{0.943437in}{1.271675in}}%
\pgfpathlineto{\pgfqpoint{0.960123in}{1.271642in}}%
\pgfpathlineto{\pgfqpoint{0.976809in}{1.271513in}}%
\pgfpathlineto{\pgfqpoint{0.993495in}{1.271343in}}%
\pgfpathlineto{\pgfqpoint{1.010182in}{1.271192in}}%
\pgfpathlineto{\pgfqpoint{1.026868in}{1.271046in}}%
\pgfpathlineto{\pgfqpoint{1.043554in}{1.271113in}}%
\pgfpathlineto{\pgfqpoint{1.060240in}{1.271100in}}%
\pgfpathlineto{\pgfqpoint{1.076926in}{1.270881in}}%
\pgfpathlineto{\pgfqpoint{1.093612in}{1.270684in}}%
\pgfpathlineto{\pgfqpoint{1.110298in}{1.270479in}}%
\pgfpathlineto{\pgfqpoint{1.126985in}{1.270150in}}%
\pgfpathlineto{\pgfqpoint{1.143671in}{1.270009in}}%
\pgfpathlineto{\pgfqpoint{1.160357in}{1.269850in}}%
\pgfpathlineto{\pgfqpoint{1.177043in}{1.269495in}}%
\pgfpathlineto{\pgfqpoint{1.193729in}{1.269205in}}%
\pgfpathlineto{\pgfqpoint{1.210415in}{1.268691in}}%
\pgfpathlineto{\pgfqpoint{1.227101in}{1.268287in}}%
\pgfpathlineto{\pgfqpoint{1.243787in}{1.267886in}}%
\pgfpathlineto{\pgfqpoint{1.260474in}{1.267487in}}%
\pgfpathlineto{\pgfqpoint{1.277160in}{1.267077in}}%
\pgfpathlineto{\pgfqpoint{1.293846in}{1.266623in}}%
\pgfpathlineto{\pgfqpoint{1.310532in}{1.266275in}}%
\pgfpathlineto{\pgfqpoint{1.327218in}{1.265820in}}%
\pgfpathlineto{\pgfqpoint{1.343904in}{1.265245in}}%
\pgfpathlineto{\pgfqpoint{1.360590in}{1.264830in}}%
\pgfpathlineto{\pgfqpoint{1.377276in}{1.264107in}}%
\pgfpathlineto{\pgfqpoint{1.393963in}{1.263317in}}%
\pgfpathlineto{\pgfqpoint{1.410649in}{1.262797in}}%
\pgfpathlineto{\pgfqpoint{1.427335in}{1.262573in}}%
\pgfpathlineto{\pgfqpoint{1.444021in}{1.262353in}}%
\pgfpathlineto{\pgfqpoint{1.460707in}{1.261631in}}%
\pgfpathlineto{\pgfqpoint{1.477393in}{1.261237in}}%
\pgfpathlineto{\pgfqpoint{1.494079in}{1.260888in}}%
\pgfpathlineto{\pgfqpoint{1.510765in}{1.260539in}}%
\pgfpathlineto{\pgfqpoint{1.527452in}{1.260198in}}%
\pgfpathlineto{\pgfqpoint{1.544138in}{1.259670in}}%
\pgfpathlineto{\pgfqpoint{1.560824in}{1.259381in}}%
\pgfpathlineto{\pgfqpoint{1.577510in}{1.258878in}}%
\pgfpathlineto{\pgfqpoint{1.594196in}{1.258305in}}%
\pgfpathlineto{\pgfqpoint{1.610882in}{1.257626in}}%
\pgfpathlineto{\pgfqpoint{1.627568in}{1.256805in}}%
\pgfpathlineto{\pgfqpoint{1.644255in}{1.255989in}}%
\pgfpathlineto{\pgfqpoint{1.660941in}{1.255166in}}%
\pgfpathlineto{\pgfqpoint{1.677627in}{1.254407in}}%
\pgfpathlineto{\pgfqpoint{1.694313in}{1.253745in}}%
\pgfpathlineto{\pgfqpoint{1.710999in}{1.253086in}}%
\pgfpathlineto{\pgfqpoint{1.727685in}{1.252432in}}%
\pgfpathlineto{\pgfqpoint{1.744371in}{1.251834in}}%
\pgfpathlineto{\pgfqpoint{1.761057in}{1.251339in}}%
\pgfpathlineto{\pgfqpoint{1.777744in}{1.250582in}}%
\pgfpathlineto{\pgfqpoint{1.794430in}{1.249799in}}%
\pgfpathlineto{\pgfqpoint{1.811116in}{1.249366in}}%
\pgfpathlineto{\pgfqpoint{1.827802in}{1.248637in}}%
\pgfpathlineto{\pgfqpoint{1.844488in}{1.247849in}}%
\pgfpathlineto{\pgfqpoint{1.861174in}{1.247166in}}%
\pgfpathlineto{\pgfqpoint{1.877860in}{1.246505in}}%
\pgfpathlineto{\pgfqpoint{1.894546in}{1.245843in}}%
\pgfpathlineto{\pgfqpoint{1.911233in}{1.245182in}}%
\pgfpathlineto{\pgfqpoint{1.927919in}{1.244520in}}%
\pgfpathlineto{\pgfqpoint{1.944605in}{1.243859in}}%
\pgfpathlineto{\pgfqpoint{1.961291in}{1.243197in}}%
\pgfpathlineto{\pgfqpoint{1.977977in}{1.242538in}}%
\pgfpathlineto{\pgfqpoint{1.994663in}{1.241887in}}%
\pgfpathlineto{\pgfqpoint{2.011349in}{1.241170in}}%
\pgfpathlineto{\pgfqpoint{2.028035in}{1.240484in}}%
\pgfpathlineto{\pgfqpoint{2.044722in}{1.239923in}}%
\pgfpathlineto{\pgfqpoint{2.061408in}{1.239262in}}%
\pgfpathlineto{\pgfqpoint{2.078094in}{1.238602in}}%
\pgfpathlineto{\pgfqpoint{2.094780in}{1.237941in}}%
\pgfpathlineto{\pgfqpoint{2.111466in}{1.237281in}}%
\pgfpathlineto{\pgfqpoint{2.128152in}{1.236451in}}%
\pgfpathlineto{\pgfqpoint{2.144838in}{1.235615in}}%
\pgfpathlineto{\pgfqpoint{2.161525in}{1.234779in}}%
\pgfpathlineto{\pgfqpoint{2.178211in}{1.233943in}}%
\pgfpathlineto{\pgfqpoint{2.178211in}{1.369921in}}%
\pgfpathlineto{\pgfqpoint{2.178211in}{1.369921in}}%
\pgfpathlineto{\pgfqpoint{2.161525in}{1.368833in}}%
\pgfpathlineto{\pgfqpoint{2.144838in}{1.367847in}}%
\pgfpathlineto{\pgfqpoint{2.128152in}{1.366809in}}%
\pgfpathlineto{\pgfqpoint{2.111466in}{1.365769in}}%
\pgfpathlineto{\pgfqpoint{2.094780in}{1.364727in}}%
\pgfpathlineto{\pgfqpoint{2.078094in}{1.363677in}}%
\pgfpathlineto{\pgfqpoint{2.061408in}{1.362565in}}%
\pgfpathlineto{\pgfqpoint{2.044722in}{1.361454in}}%
\pgfpathlineto{\pgfqpoint{2.028035in}{1.360564in}}%
\pgfpathlineto{\pgfqpoint{2.011349in}{1.359528in}}%
\pgfpathlineto{\pgfqpoint{1.994663in}{1.358490in}}%
\pgfpathlineto{\pgfqpoint{1.977977in}{1.357448in}}%
\pgfpathlineto{\pgfqpoint{1.961291in}{1.356406in}}%
\pgfpathlineto{\pgfqpoint{1.944605in}{1.355363in}}%
\pgfpathlineto{\pgfqpoint{1.927919in}{1.354293in}}%
\pgfpathlineto{\pgfqpoint{1.911233in}{1.353173in}}%
\pgfpathlineto{\pgfqpoint{1.894546in}{1.352054in}}%
\pgfpathlineto{\pgfqpoint{1.877860in}{1.350934in}}%
\pgfpathlineto{\pgfqpoint{1.861174in}{1.349814in}}%
\pgfpathlineto{\pgfqpoint{1.844488in}{1.348694in}}%
\pgfpathlineto{\pgfqpoint{1.827802in}{1.347575in}}%
\pgfpathlineto{\pgfqpoint{1.811116in}{1.346455in}}%
\pgfpathlineto{\pgfqpoint{1.794430in}{1.345335in}}%
\pgfpathlineto{\pgfqpoint{1.777744in}{1.344216in}}%
\pgfpathlineto{\pgfqpoint{1.761057in}{1.343096in}}%
\pgfpathlineto{\pgfqpoint{1.744371in}{1.341976in}}%
\pgfpathlineto{\pgfqpoint{1.727685in}{1.340856in}}%
\pgfpathlineto{\pgfqpoint{1.710999in}{1.339950in}}%
\pgfpathlineto{\pgfqpoint{1.694313in}{1.339251in}}%
\pgfpathlineto{\pgfqpoint{1.677627in}{1.338480in}}%
\pgfpathlineto{\pgfqpoint{1.660941in}{1.337374in}}%
\pgfpathlineto{\pgfqpoint{1.644255in}{1.336263in}}%
\pgfpathlineto{\pgfqpoint{1.627568in}{1.335153in}}%
\pgfpathlineto{\pgfqpoint{1.610882in}{1.334042in}}%
\pgfpathlineto{\pgfqpoint{1.594196in}{1.333122in}}%
\pgfpathlineto{\pgfqpoint{1.577510in}{1.332184in}}%
\pgfpathlineto{\pgfqpoint{1.560824in}{1.331146in}}%
\pgfpathlineto{\pgfqpoint{1.544138in}{1.330303in}}%
\pgfpathlineto{\pgfqpoint{1.527452in}{1.329356in}}%
\pgfpathlineto{\pgfqpoint{1.510765in}{1.328321in}}%
\pgfpathlineto{\pgfqpoint{1.494079in}{1.327282in}}%
\pgfpathlineto{\pgfqpoint{1.477393in}{1.326239in}}%
\pgfpathlineto{\pgfqpoint{1.460707in}{1.325196in}}%
\pgfpathlineto{\pgfqpoint{1.444021in}{1.324153in}}%
\pgfpathlineto{\pgfqpoint{1.427335in}{1.323071in}}%
\pgfpathlineto{\pgfqpoint{1.410649in}{1.321929in}}%
\pgfpathlineto{\pgfqpoint{1.393963in}{1.321032in}}%
\pgfpathlineto{\pgfqpoint{1.377276in}{1.319936in}}%
\pgfpathlineto{\pgfqpoint{1.360590in}{1.318795in}}%
\pgfpathlineto{\pgfqpoint{1.343904in}{1.317694in}}%
\pgfpathlineto{\pgfqpoint{1.327218in}{1.316646in}}%
\pgfpathlineto{\pgfqpoint{1.310532in}{1.315759in}}%
\pgfpathlineto{\pgfqpoint{1.293846in}{1.314830in}}%
\pgfpathlineto{\pgfqpoint{1.277160in}{1.314128in}}%
\pgfpathlineto{\pgfqpoint{1.260474in}{1.313422in}}%
\pgfpathlineto{\pgfqpoint{1.243787in}{1.312622in}}%
\pgfpathlineto{\pgfqpoint{1.227101in}{1.311741in}}%
\pgfpathlineto{\pgfqpoint{1.210415in}{1.310740in}}%
\pgfpathlineto{\pgfqpoint{1.193729in}{1.310076in}}%
\pgfpathlineto{\pgfqpoint{1.177043in}{1.309084in}}%
\pgfpathlineto{\pgfqpoint{1.160357in}{1.308756in}}%
\pgfpathlineto{\pgfqpoint{1.143671in}{1.308211in}}%
\pgfpathlineto{\pgfqpoint{1.126985in}{1.307478in}}%
\pgfpathlineto{\pgfqpoint{1.110298in}{1.306774in}}%
\pgfpathlineto{\pgfqpoint{1.093612in}{1.306106in}}%
\pgfpathlineto{\pgfqpoint{1.076926in}{1.305449in}}%
\pgfpathlineto{\pgfqpoint{1.060240in}{1.304896in}}%
\pgfpathlineto{\pgfqpoint{1.043554in}{1.304453in}}%
\pgfpathlineto{\pgfqpoint{1.026868in}{1.304035in}}%
\pgfpathlineto{\pgfqpoint{1.010182in}{1.303492in}}%
\pgfpathlineto{\pgfqpoint{0.993495in}{1.302872in}}%
\pgfpathlineto{\pgfqpoint{0.976809in}{1.302539in}}%
\pgfpathlineto{\pgfqpoint{0.960123in}{1.302448in}}%
\pgfpathlineto{\pgfqpoint{0.943437in}{1.302274in}}%
\pgfpathlineto{\pgfqpoint{0.926751in}{1.301673in}}%
\pgfpathlineto{\pgfqpoint{0.910065in}{1.301547in}}%
\pgfpathlineto{\pgfqpoint{0.893379in}{1.301324in}}%
\pgfpathlineto{\pgfqpoint{0.876693in}{1.301438in}}%
\pgfpathlineto{\pgfqpoint{0.860006in}{1.301692in}}%
\pgfpathlineto{\pgfqpoint{0.843320in}{1.301875in}}%
\pgfpathlineto{\pgfqpoint{0.826634in}{1.302126in}}%
\pgfpathlineto{\pgfqpoint{0.809948in}{1.302379in}}%
\pgfpathlineto{\pgfqpoint{0.793262in}{1.302183in}}%
\pgfpathlineto{\pgfqpoint{0.776576in}{1.302336in}}%
\pgfpathlineto{\pgfqpoint{0.759890in}{1.302750in}}%
\pgfpathlineto{\pgfqpoint{0.743204in}{1.303123in}}%
\pgfpathlineto{\pgfqpoint{0.726517in}{1.303581in}}%
\pgfpathlineto{\pgfqpoint{0.709831in}{1.304059in}}%
\pgfpathlineto{\pgfqpoint{0.693145in}{1.304338in}}%
\pgfpathlineto{\pgfqpoint{0.676459in}{1.304617in}}%
\pgfpathlineto{\pgfqpoint{0.659773in}{1.304962in}}%
\pgfpathlineto{\pgfqpoint{0.643087in}{1.305311in}}%
\pgfpathlineto{\pgfqpoint{0.626401in}{1.305679in}}%
\pgfpathlineto{\pgfqpoint{0.609715in}{1.306052in}}%
\pgfpathlineto{\pgfqpoint{0.593028in}{1.306567in}}%
\pgfpathlineto{\pgfqpoint{0.576342in}{1.307085in}}%
\pgfpathlineto{\pgfqpoint{0.559656in}{1.307409in}}%
\pgfpathlineto{\pgfqpoint{0.542970in}{1.307730in}}%
\pgfpathlineto{\pgfqpoint{0.526284in}{1.308098in}}%
\pgfpathclose%
\pgfusepath{fill}%
\end{pgfscope}%
\begin{pgfscope}%
\pgfpathrectangle{\pgfqpoint{0.526284in}{0.473557in}}{\pgfqpoint{1.651927in}{1.704653in}}%
\pgfusepath{clip}%
\pgfsetbuttcap%
\pgfsetroundjoin%
\definecolor{currentfill}{rgb}{0.580392,0.403922,0.741176}%
\pgfsetfillcolor{currentfill}%
\pgfsetfillopacity{0.250000}%
\pgfsetlinewidth{1.003750pt}%
\definecolor{currentstroke}{rgb}{0.580392,0.403922,0.741176}%
\pgfsetstrokecolor{currentstroke}%
\pgfsetstrokeopacity{0.250000}%
\pgfsetdash{}{0pt}%
\pgfsys@defobject{currentmarker}{\pgfqpoint{-0.017010in}{-0.017010in}}{\pgfqpoint{0.017010in}{0.017010in}}{%
\pgfpathmoveto{\pgfqpoint{0.000000in}{-0.017010in}}%
\pgfpathcurveto{\pgfqpoint{0.004511in}{-0.017010in}}{\pgfqpoint{0.008838in}{-0.015218in}}{\pgfqpoint{0.012028in}{-0.012028in}}%
\pgfpathcurveto{\pgfqpoint{0.015218in}{-0.008838in}}{\pgfqpoint{0.017010in}{-0.004511in}}{\pgfqpoint{0.017010in}{0.000000in}}%
\pgfpathcurveto{\pgfqpoint{0.017010in}{0.004511in}}{\pgfqpoint{0.015218in}{0.008838in}}{\pgfqpoint{0.012028in}{0.012028in}}%
\pgfpathcurveto{\pgfqpoint{0.008838in}{0.015218in}}{\pgfqpoint{0.004511in}{0.017010in}}{\pgfqpoint{0.000000in}{0.017010in}}%
\pgfpathcurveto{\pgfqpoint{-0.004511in}{0.017010in}}{\pgfqpoint{-0.008838in}{0.015218in}}{\pgfqpoint{-0.012028in}{0.012028in}}%
\pgfpathcurveto{\pgfqpoint{-0.015218in}{0.008838in}}{\pgfqpoint{-0.017010in}{0.004511in}}{\pgfqpoint{-0.017010in}{0.000000in}}%
\pgfpathcurveto{\pgfqpoint{-0.017010in}{-0.004511in}}{\pgfqpoint{-0.015218in}{-0.008838in}}{\pgfqpoint{-0.012028in}{-0.012028in}}%
\pgfpathcurveto{\pgfqpoint{-0.008838in}{-0.015218in}}{\pgfqpoint{-0.004511in}{-0.017010in}}{\pgfqpoint{0.000000in}{-0.017010in}}%
\pgfpathclose%
\pgfusepath{stroke,fill}%
}%
\begin{pgfscope}%
\pgfsys@transformshift{1.139286in}{1.403549in}%
\pgfsys@useobject{currentmarker}{}%
\end{pgfscope}%
\begin{pgfscope}%
\pgfsys@transformshift{0.826629in}{1.187127in}%
\pgfsys@useobject{currentmarker}{}%
\end{pgfscope}%
\begin{pgfscope}%
\pgfsys@transformshift{0.969498in}{0.854976in}%
\pgfsys@useobject{currentmarker}{}%
\end{pgfscope}%
\begin{pgfscope}%
\pgfsys@transformshift{0.732099in}{0.769804in}%
\pgfsys@useobject{currentmarker}{}%
\end{pgfscope}%
\begin{pgfscope}%
\pgfsys@transformshift{0.815558in}{0.740230in}%
\pgfsys@useobject{currentmarker}{}%
\end{pgfscope}%
\begin{pgfscope}%
\pgfsys@transformshift{1.268362in}{1.351773in}%
\pgfsys@useobject{currentmarker}{}%
\end{pgfscope}%
\begin{pgfscope}%
\pgfsys@transformshift{0.640142in}{0.809921in}%
\pgfsys@useobject{currentmarker}{}%
\end{pgfscope}%
\begin{pgfscope}%
\pgfsys@transformshift{1.031370in}{1.303839in}%
\pgfsys@useobject{currentmarker}{}%
\end{pgfscope}%
\begin{pgfscope}%
\pgfsys@transformshift{1.501355in}{1.382513in}%
\pgfsys@useobject{currentmarker}{}%
\end{pgfscope}%
\begin{pgfscope}%
\pgfsys@transformshift{1.307778in}{0.988790in}%
\pgfsys@useobject{currentmarker}{}%
\end{pgfscope}%
\begin{pgfscope}%
\pgfsys@transformshift{1.194068in}{0.854273in}%
\pgfsys@useobject{currentmarker}{}%
\end{pgfscope}%
\begin{pgfscope}%
\pgfsys@transformshift{0.667950in}{0.811416in}%
\pgfsys@useobject{currentmarker}{}%
\end{pgfscope}%
\begin{pgfscope}%
\pgfsys@transformshift{0.827370in}{0.698852in}%
\pgfsys@useobject{currentmarker}{}%
\end{pgfscope}%
\begin{pgfscope}%
\pgfsys@transformshift{1.071193in}{1.058324in}%
\pgfsys@useobject{currentmarker}{}%
\end{pgfscope}%
\begin{pgfscope}%
\pgfsys@transformshift{0.921937in}{1.164574in}%
\pgfsys@useobject{currentmarker}{}%
\end{pgfscope}%
\begin{pgfscope}%
\pgfsys@transformshift{1.556989in}{0.869838in}%
\pgfsys@useobject{currentmarker}{}%
\end{pgfscope}%
\begin{pgfscope}%
\pgfsys@transformshift{0.645159in}{0.978222in}%
\pgfsys@useobject{currentmarker}{}%
\end{pgfscope}%
\begin{pgfscope}%
\pgfsys@transformshift{1.282858in}{1.317080in}%
\pgfsys@useobject{currentmarker}{}%
\end{pgfscope}%
\begin{pgfscope}%
\pgfsys@transformshift{0.923659in}{1.088449in}%
\pgfsys@useobject{currentmarker}{}%
\end{pgfscope}%
\begin{pgfscope}%
\pgfsys@transformshift{1.196863in}{1.359562in}%
\pgfsys@useobject{currentmarker}{}%
\end{pgfscope}%
\begin{pgfscope}%
\pgfsys@transformshift{0.734895in}{0.781421in}%
\pgfsys@useobject{currentmarker}{}%
\end{pgfscope}%
\begin{pgfscope}%
\pgfsys@transformshift{1.049754in}{0.837572in}%
\pgfsys@useobject{currentmarker}{}%
\end{pgfscope}%
\begin{pgfscope}%
\pgfsys@transformshift{0.612705in}{0.825786in}%
\pgfsys@useobject{currentmarker}{}%
\end{pgfscope}%
\begin{pgfscope}%
\pgfsys@transformshift{1.196567in}{1.398915in}%
\pgfsys@useobject{currentmarker}{}%
\end{pgfscope}%
\begin{pgfscope}%
\pgfsys@transformshift{0.616815in}{0.979054in}%
\pgfsys@useobject{currentmarker}{}%
\end{pgfscope}%
\begin{pgfscope}%
\pgfsys@transformshift{0.881170in}{0.623278in}%
\pgfsys@useobject{currentmarker}{}%
\end{pgfscope}%
\begin{pgfscope}%
\pgfsys@transformshift{0.644863in}{0.772990in}%
\pgfsys@useobject{currentmarker}{}%
\end{pgfscope}%
\begin{pgfscope}%
\pgfsys@transformshift{1.119421in}{0.810548in}%
\pgfsys@useobject{currentmarker}{}%
\end{pgfscope}%
\begin{pgfscope}%
\pgfsys@transformshift{0.796952in}{0.848069in}%
\pgfsys@useobject{currentmarker}{}%
\end{pgfscope}%
\begin{pgfscope}%
\pgfsys@transformshift{0.928991in}{1.134674in}%
\pgfsys@useobject{currentmarker}{}%
\end{pgfscope}%
\begin{pgfscope}%
\pgfsys@transformshift{1.146006in}{1.166008in}%
\pgfsys@useobject{currentmarker}{}%
\end{pgfscope}%
\begin{pgfscope}%
\pgfsys@transformshift{1.088355in}{1.233710in}%
\pgfsys@useobject{currentmarker}{}%
\end{pgfscope}%
\begin{pgfscope}%
\pgfsys@transformshift{1.172758in}{0.821080in}%
\pgfsys@useobject{currentmarker}{}%
\end{pgfscope}%
\begin{pgfscope}%
\pgfsys@transformshift{1.211137in}{1.397392in}%
\pgfsys@useobject{currentmarker}{}%
\end{pgfscope}%
\begin{pgfscope}%
\pgfsys@transformshift{0.829388in}{0.795522in}%
\pgfsys@useobject{currentmarker}{}%
\end{pgfscope}%
\begin{pgfscope}%
\pgfsys@transformshift{1.584944in}{0.882886in}%
\pgfsys@useobject{currentmarker}{}%
\end{pgfscope}%
\begin{pgfscope}%
\pgfsys@transformshift{0.693739in}{1.022831in}%
\pgfsys@useobject{currentmarker}{}%
\end{pgfscope}%
\begin{pgfscope}%
\pgfsys@transformshift{1.271769in}{1.351075in}%
\pgfsys@useobject{currentmarker}{}%
\end{pgfscope}%
\begin{pgfscope}%
\pgfsys@transformshift{1.064010in}{1.171280in}%
\pgfsys@useobject{currentmarker}{}%
\end{pgfscope}%
\begin{pgfscope}%
\pgfsys@transformshift{1.461162in}{0.886154in}%
\pgfsys@useobject{currentmarker}{}%
\end{pgfscope}%
\begin{pgfscope}%
\pgfsys@transformshift{1.254162in}{1.295732in}%
\pgfsys@useobject{currentmarker}{}%
\end{pgfscope}%
\begin{pgfscope}%
\pgfsys@transformshift{0.641290in}{0.986576in}%
\pgfsys@useobject{currentmarker}{}%
\end{pgfscope}%
\begin{pgfscope}%
\pgfsys@transformshift{1.047681in}{0.884620in}%
\pgfsys@useobject{currentmarker}{}%
\end{pgfscope}%
\begin{pgfscope}%
\pgfsys@transformshift{0.651435in}{0.945352in}%
\pgfsys@useobject{currentmarker}{}%
\end{pgfscope}%
\begin{pgfscope}%
\pgfsys@transformshift{1.047699in}{0.811187in}%
\pgfsys@useobject{currentmarker}{}%
\end{pgfscope}%
\begin{pgfscope}%
\pgfsys@transformshift{0.706939in}{0.735010in}%
\pgfsys@useobject{currentmarker}{}%
\end{pgfscope}%
\begin{pgfscope}%
\pgfsys@transformshift{0.621869in}{1.154593in}%
\pgfsys@useobject{currentmarker}{}%
\end{pgfscope}%
\begin{pgfscope}%
\pgfsys@transformshift{1.117533in}{1.122278in}%
\pgfsys@useobject{currentmarker}{}%
\end{pgfscope}%
\begin{pgfscope}%
\pgfsys@transformshift{1.307648in}{0.783684in}%
\pgfsys@useobject{currentmarker}{}%
\end{pgfscope}%
\begin{pgfscope}%
\pgfsys@transformshift{0.948856in}{1.381629in}%
\pgfsys@useobject{currentmarker}{}%
\end{pgfscope}%
\begin{pgfscope}%
\pgfsys@transformshift{1.014153in}{1.073459in}%
\pgfsys@useobject{currentmarker}{}%
\end{pgfscope}%
\begin{pgfscope}%
\pgfsys@transformshift{0.631737in}{1.081326in}%
\pgfsys@useobject{currentmarker}{}%
\end{pgfscope}%
\begin{pgfscope}%
\pgfsys@transformshift{0.643493in}{0.993448in}%
\pgfsys@useobject{currentmarker}{}%
\end{pgfscope}%
\begin{pgfscope}%
\pgfsys@transformshift{0.772755in}{1.146374in}%
\pgfsys@useobject{currentmarker}{}%
\end{pgfscope}%
\begin{pgfscope}%
\pgfsys@transformshift{0.692091in}{0.725259in}%
\pgfsys@useobject{currentmarker}{}%
\end{pgfscope}%
\begin{pgfscope}%
\pgfsys@transformshift{0.595358in}{1.212285in}%
\pgfsys@useobject{currentmarker}{}%
\end{pgfscope}%
\begin{pgfscope}%
\pgfsys@transformshift{0.903997in}{0.688962in}%
\pgfsys@useobject{currentmarker}{}%
\end{pgfscope}%
\begin{pgfscope}%
\pgfsys@transformshift{1.258735in}{1.044652in}%
\pgfsys@useobject{currentmarker}{}%
\end{pgfscope}%
\begin{pgfscope}%
\pgfsys@transformshift{0.771996in}{0.722367in}%
\pgfsys@useobject{currentmarker}{}%
\end{pgfscope}%
\begin{pgfscope}%
\pgfsys@transformshift{0.665932in}{0.771811in}%
\pgfsys@useobject{currentmarker}{}%
\end{pgfscope}%
\begin{pgfscope}%
\pgfsys@transformshift{0.723120in}{0.782430in}%
\pgfsys@useobject{currentmarker}{}%
\end{pgfscope}%
\begin{pgfscope}%
\pgfsys@transformshift{1.030500in}{1.637663in}%
\pgfsys@useobject{currentmarker}{}%
\end{pgfscope}%
\begin{pgfscope}%
\pgfsys@transformshift{0.538429in}{1.039595in}%
\pgfsys@useobject{currentmarker}{}%
\end{pgfscope}%
\begin{pgfscope}%
\pgfsys@transformshift{1.081468in}{0.799436in}%
\pgfsys@useobject{currentmarker}{}%
\end{pgfscope}%
\begin{pgfscope}%
\pgfsys@transformshift{0.629293in}{0.768145in}%
\pgfsys@useobject{currentmarker}{}%
\end{pgfscope}%
\begin{pgfscope}%
\pgfsys@transformshift{0.792175in}{0.668515in}%
\pgfsys@useobject{currentmarker}{}%
\end{pgfscope}%
\begin{pgfscope}%
\pgfsys@transformshift{1.017763in}{0.871791in}%
\pgfsys@useobject{currentmarker}{}%
\end{pgfscope}%
\begin{pgfscope}%
\pgfsys@transformshift{0.953410in}{1.114154in}%
\pgfsys@useobject{currentmarker}{}%
\end{pgfscope}%
\begin{pgfscope}%
\pgfsys@transformshift{1.026687in}{1.154443in}%
\pgfsys@useobject{currentmarker}{}%
\end{pgfscope}%
\begin{pgfscope}%
\pgfsys@transformshift{1.527422in}{0.874366in}%
\pgfsys@useobject{currentmarker}{}%
\end{pgfscope}%
\begin{pgfscope}%
\pgfsys@transformshift{1.160817in}{1.299765in}%
\pgfsys@useobject{currentmarker}{}%
\end{pgfscope}%
\begin{pgfscope}%
\pgfsys@transformshift{0.698349in}{1.225304in}%
\pgfsys@useobject{currentmarker}{}%
\end{pgfscope}%
\begin{pgfscope}%
\pgfsys@transformshift{1.035203in}{1.035580in}%
\pgfsys@useobject{currentmarker}{}%
\end{pgfscope}%
\begin{pgfscope}%
\pgfsys@transformshift{0.922307in}{1.183360in}%
\pgfsys@useobject{currentmarker}{}%
\end{pgfscope}%
\begin{pgfscope}%
\pgfsys@transformshift{1.166982in}{1.379433in}%
\pgfsys@useobject{currentmarker}{}%
\end{pgfscope}%
\begin{pgfscope}%
\pgfsys@transformshift{1.119569in}{1.207130in}%
\pgfsys@useobject{currentmarker}{}%
\end{pgfscope}%
\begin{pgfscope}%
\pgfsys@transformshift{0.825518in}{0.700207in}%
\pgfsys@useobject{currentmarker}{}%
\end{pgfscope}%
\begin{pgfscope}%
\pgfsys@transformshift{1.426209in}{0.915817in}%
\pgfsys@useobject{currentmarker}{}%
\end{pgfscope}%
\begin{pgfscope}%
\pgfsys@transformshift{1.406677in}{1.268618in}%
\pgfsys@useobject{currentmarker}{}%
\end{pgfscope}%
\begin{pgfscope}%
\pgfsys@transformshift{1.415453in}{0.985310in}%
\pgfsys@useobject{currentmarker}{}%
\end{pgfscope}%
\begin{pgfscope}%
\pgfsys@transformshift{1.076914in}{1.449036in}%
\pgfsys@useobject{currentmarker}{}%
\end{pgfscope}%
\begin{pgfscope}%
\pgfsys@transformshift{0.779031in}{0.671639in}%
\pgfsys@useobject{currentmarker}{}%
\end{pgfscope}%
\begin{pgfscope}%
\pgfsys@transformshift{0.629497in}{0.881701in}%
\pgfsys@useobject{currentmarker}{}%
\end{pgfscope}%
\begin{pgfscope}%
\pgfsys@transformshift{1.519184in}{0.889515in}%
\pgfsys@useobject{currentmarker}{}%
\end{pgfscope}%
\begin{pgfscope}%
\pgfsys@transformshift{0.769237in}{0.777483in}%
\pgfsys@useobject{currentmarker}{}%
\end{pgfscope}%
\begin{pgfscope}%
\pgfsys@transformshift{1.087429in}{0.919736in}%
\pgfsys@useobject{currentmarker}{}%
\end{pgfscope}%
\begin{pgfscope}%
\pgfsys@transformshift{0.750705in}{1.557561in}%
\pgfsys@useobject{currentmarker}{}%
\end{pgfscope}%
\begin{pgfscope}%
\pgfsys@transformshift{1.495746in}{0.855765in}%
\pgfsys@useobject{currentmarker}{}%
\end{pgfscope}%
\begin{pgfscope}%
\pgfsys@transformshift{1.182959in}{0.921485in}%
\pgfsys@useobject{currentmarker}{}%
\end{pgfscope}%
\begin{pgfscope}%
\pgfsys@transformshift{0.713789in}{0.899551in}%
\pgfsys@useobject{currentmarker}{}%
\end{pgfscope}%
\begin{pgfscope}%
\pgfsys@transformshift{1.355894in}{1.169661in}%
\pgfsys@useobject{currentmarker}{}%
\end{pgfscope}%
\begin{pgfscope}%
\pgfsys@transformshift{1.635208in}{1.265642in}%
\pgfsys@useobject{currentmarker}{}%
\end{pgfscope}%
\begin{pgfscope}%
\pgfsys@transformshift{1.253848in}{0.854811in}%
\pgfsys@useobject{currentmarker}{}%
\end{pgfscope}%
\begin{pgfscope}%
\pgfsys@transformshift{0.935989in}{1.361087in}%
\pgfsys@useobject{currentmarker}{}%
\end{pgfscope}%
\begin{pgfscope}%
\pgfsys@transformshift{0.875320in}{1.239239in}%
\pgfsys@useobject{currentmarker}{}%
\end{pgfscope}%
\begin{pgfscope}%
\pgfsys@transformshift{1.109831in}{1.569048in}%
\pgfsys@useobject{currentmarker}{}%
\end{pgfscope}%
\begin{pgfscope}%
\pgfsys@transformshift{1.118292in}{0.730320in}%
\pgfsys@useobject{currentmarker}{}%
\end{pgfscope}%
\begin{pgfscope}%
\pgfsys@transformshift{1.130029in}{1.175322in}%
\pgfsys@useobject{currentmarker}{}%
\end{pgfscope}%
\begin{pgfscope}%
\pgfsys@transformshift{1.140971in}{1.516747in}%
\pgfsys@useobject{currentmarker}{}%
\end{pgfscope}%
\begin{pgfscope}%
\pgfsys@transformshift{1.052346in}{1.735910in}%
\pgfsys@useobject{currentmarker}{}%
\end{pgfscope}%
\begin{pgfscope}%
\pgfsys@transformshift{0.621629in}{0.899922in}%
\pgfsys@useobject{currentmarker}{}%
\end{pgfscope}%
\begin{pgfscope}%
\pgfsys@transformshift{0.917753in}{0.991637in}%
\pgfsys@useobject{currentmarker}{}%
\end{pgfscope}%
\begin{pgfscope}%
\pgfsys@transformshift{0.863027in}{1.133032in}%
\pgfsys@useobject{currentmarker}{}%
\end{pgfscope}%
\begin{pgfscope}%
\pgfsys@transformshift{0.672485in}{0.764463in}%
\pgfsys@useobject{currentmarker}{}%
\end{pgfscope}%
\begin{pgfscope}%
\pgfsys@transformshift{1.747437in}{0.835808in}%
\pgfsys@useobject{currentmarker}{}%
\end{pgfscope}%
\begin{pgfscope}%
\pgfsys@transformshift{0.526284in}{1.031172in}%
\pgfsys@useobject{currentmarker}{}%
\end{pgfscope}%
\begin{pgfscope}%
\pgfsys@transformshift{0.627405in}{0.790789in}%
\pgfsys@useobject{currentmarker}{}%
\end{pgfscope}%
\begin{pgfscope}%
\pgfsys@transformshift{1.270658in}{0.817218in}%
\pgfsys@useobject{currentmarker}{}%
\end{pgfscope}%
\begin{pgfscope}%
\pgfsys@transformshift{1.594053in}{0.784560in}%
\pgfsys@useobject{currentmarker}{}%
\end{pgfscope}%
\begin{pgfscope}%
\pgfsys@transformshift{0.652472in}{0.755938in}%
\pgfsys@useobject{currentmarker}{}%
\end{pgfscope}%
\begin{pgfscope}%
\pgfsys@transformshift{0.819168in}{0.859167in}%
\pgfsys@useobject{currentmarker}{}%
\end{pgfscope}%
\begin{pgfscope}%
\pgfsys@transformshift{0.847883in}{0.688247in}%
\pgfsys@useobject{currentmarker}{}%
\end{pgfscope}%
\begin{pgfscope}%
\pgfsys@transformshift{1.118495in}{0.722034in}%
\pgfsys@useobject{currentmarker}{}%
\end{pgfscope}%
\begin{pgfscope}%
\pgfsys@transformshift{0.618167in}{0.820746in}%
\pgfsys@useobject{currentmarker}{}%
\end{pgfscope}%
\begin{pgfscope}%
\pgfsys@transformshift{0.857547in}{1.185720in}%
\pgfsys@useobject{currentmarker}{}%
\end{pgfscope}%
\begin{pgfscope}%
\pgfsys@transformshift{1.090392in}{1.487821in}%
\pgfsys@useobject{currentmarker}{}%
\end{pgfscope}%
\begin{pgfscope}%
\pgfsys@transformshift{1.063566in}{1.299734in}%
\pgfsys@useobject{currentmarker}{}%
\end{pgfscope}%
\begin{pgfscope}%
\pgfsys@transformshift{1.510205in}{0.781468in}%
\pgfsys@useobject{currentmarker}{}%
\end{pgfscope}%
\begin{pgfscope}%
\pgfsys@transformshift{1.287209in}{1.323575in}%
\pgfsys@useobject{currentmarker}{}%
\end{pgfscope}%
\begin{pgfscope}%
\pgfsys@transformshift{0.778753in}{0.793390in}%
\pgfsys@useobject{currentmarker}{}%
\end{pgfscope}%
\begin{pgfscope}%
\pgfsys@transformshift{0.627757in}{0.939901in}%
\pgfsys@useobject{currentmarker}{}%
\end{pgfscope}%
\begin{pgfscope}%
\pgfsys@transformshift{1.071860in}{1.206327in}%
\pgfsys@useobject{currentmarker}{}%
\end{pgfscope}%
\begin{pgfscope}%
\pgfsys@transformshift{1.053938in}{0.922299in}%
\pgfsys@useobject{currentmarker}{}%
\end{pgfscope}%
\begin{pgfscope}%
\pgfsys@transformshift{0.910458in}{1.244986in}%
\pgfsys@useobject{currentmarker}{}%
\end{pgfscope}%
\begin{pgfscope}%
\pgfsys@transformshift{0.621962in}{0.984697in}%
\pgfsys@useobject{currentmarker}{}%
\end{pgfscope}%
\begin{pgfscope}%
\pgfsys@transformshift{0.836867in}{1.016896in}%
\pgfsys@useobject{currentmarker}{}%
\end{pgfscope}%
\begin{pgfscope}%
\pgfsys@transformshift{0.800229in}{1.008454in}%
\pgfsys@useobject{currentmarker}{}%
\end{pgfscope}%
\begin{pgfscope}%
\pgfsys@transformshift{0.958260in}{1.135658in}%
\pgfsys@useobject{currentmarker}{}%
\end{pgfscope}%
\begin{pgfscope}%
\pgfsys@transformshift{0.656045in}{0.761016in}%
\pgfsys@useobject{currentmarker}{}%
\end{pgfscope}%
\begin{pgfscope}%
\pgfsys@transformshift{0.629960in}{0.881698in}%
\pgfsys@useobject{currentmarker}{}%
\end{pgfscope}%
\begin{pgfscope}%
\pgfsys@transformshift{1.008358in}{0.758220in}%
\pgfsys@useobject{currentmarker}{}%
\end{pgfscope}%
\begin{pgfscope}%
\pgfsys@transformshift{1.258698in}{1.202675in}%
\pgfsys@useobject{currentmarker}{}%
\end{pgfscope}%
\begin{pgfscope}%
\pgfsys@transformshift{1.100889in}{0.805127in}%
\pgfsys@useobject{currentmarker}{}%
\end{pgfscope}%
\begin{pgfscope}%
\pgfsys@transformshift{1.230410in}{1.110274in}%
\pgfsys@useobject{currentmarker}{}%
\end{pgfscope}%
\begin{pgfscope}%
\pgfsys@transformshift{0.647325in}{1.036947in}%
\pgfsys@useobject{currentmarker}{}%
\end{pgfscope}%
\begin{pgfscope}%
\pgfsys@transformshift{0.904423in}{1.278442in}%
\pgfsys@useobject{currentmarker}{}%
\end{pgfscope}%
\begin{pgfscope}%
\pgfsys@transformshift{1.617602in}{1.055108in}%
\pgfsys@useobject{currentmarker}{}%
\end{pgfscope}%
\begin{pgfscope}%
\pgfsys@transformshift{1.245406in}{1.371554in}%
\pgfsys@useobject{currentmarker}{}%
\end{pgfscope}%
\begin{pgfscope}%
\pgfsys@transformshift{1.238907in}{1.384083in}%
\pgfsys@useobject{currentmarker}{}%
\end{pgfscope}%
\begin{pgfscope}%
\pgfsys@transformshift{1.187366in}{1.372132in}%
\pgfsys@useobject{currentmarker}{}%
\end{pgfscope}%
\begin{pgfscope}%
\pgfsys@transformshift{1.095724in}{1.229199in}%
\pgfsys@useobject{currentmarker}{}%
\end{pgfscope}%
\begin{pgfscope}%
\pgfsys@transformshift{1.131084in}{1.571363in}%
\pgfsys@useobject{currentmarker}{}%
\end{pgfscope}%
\begin{pgfscope}%
\pgfsys@transformshift{0.643549in}{0.861671in}%
\pgfsys@useobject{currentmarker}{}%
\end{pgfscope}%
\begin{pgfscope}%
\pgfsys@transformshift{0.726878in}{0.944110in}%
\pgfsys@useobject{currentmarker}{}%
\end{pgfscope}%
\begin{pgfscope}%
\pgfsys@transformshift{0.940728in}{0.821904in}%
\pgfsys@useobject{currentmarker}{}%
\end{pgfscope}%
\begin{pgfscope}%
\pgfsys@transformshift{0.686537in}{0.778486in}%
\pgfsys@useobject{currentmarker}{}%
\end{pgfscope}%
\begin{pgfscope}%
\pgfsys@transformshift{1.304686in}{1.276580in}%
\pgfsys@useobject{currentmarker}{}%
\end{pgfscope}%
\begin{pgfscope}%
\pgfsys@transformshift{0.822223in}{0.831001in}%
\pgfsys@useobject{currentmarker}{}%
\end{pgfscope}%
\begin{pgfscope}%
\pgfsys@transformshift{0.753593in}{0.850419in}%
\pgfsys@useobject{currentmarker}{}%
\end{pgfscope}%
\begin{pgfscope}%
\pgfsys@transformshift{1.155670in}{1.318635in}%
\pgfsys@useobject{currentmarker}{}%
\end{pgfscope}%
\begin{pgfscope}%
\pgfsys@transformshift{0.665617in}{0.758975in}%
\pgfsys@useobject{currentmarker}{}%
\end{pgfscope}%
\begin{pgfscope}%
\pgfsys@transformshift{0.722713in}{0.674134in}%
\pgfsys@useobject{currentmarker}{}%
\end{pgfscope}%
\begin{pgfscope}%
\pgfsys@transformshift{0.537651in}{0.894467in}%
\pgfsys@useobject{currentmarker}{}%
\end{pgfscope}%
\begin{pgfscope}%
\pgfsys@transformshift{0.956631in}{1.052134in}%
\pgfsys@useobject{currentmarker}{}%
\end{pgfscope}%
\begin{pgfscope}%
\pgfsys@transformshift{1.046922in}{0.857425in}%
\pgfsys@useobject{currentmarker}{}%
\end{pgfscope}%
\begin{pgfscope}%
\pgfsys@transformshift{0.812559in}{0.947449in}%
\pgfsys@useobject{currentmarker}{}%
\end{pgfscope}%
\begin{pgfscope}%
\pgfsys@transformshift{1.535254in}{0.874758in}%
\pgfsys@useobject{currentmarker}{}%
\end{pgfscope}%
\begin{pgfscope}%
\pgfsys@transformshift{0.778087in}{1.110387in}%
\pgfsys@useobject{currentmarker}{}%
\end{pgfscope}%
\begin{pgfscope}%
\pgfsys@transformshift{0.723860in}{1.122979in}%
\pgfsys@useobject{currentmarker}{}%
\end{pgfscope}%
\begin{pgfscope}%
\pgfsys@transformshift{0.970961in}{0.656348in}%
\pgfsys@useobject{currentmarker}{}%
\end{pgfscope}%
\begin{pgfscope}%
\pgfsys@transformshift{1.099149in}{1.714482in}%
\pgfsys@useobject{currentmarker}{}%
\end{pgfscope}%
\begin{pgfscope}%
\pgfsys@transformshift{1.042182in}{1.203601in}%
\pgfsys@useobject{currentmarker}{}%
\end{pgfscope}%
\begin{pgfscope}%
\pgfsys@transformshift{1.047181in}{1.650280in}%
\pgfsys@useobject{currentmarker}{}%
\end{pgfscope}%
\begin{pgfscope}%
\pgfsys@transformshift{1.569152in}{1.303264in}%
\pgfsys@useobject{currentmarker}{}%
\end{pgfscope}%
\begin{pgfscope}%
\pgfsys@transformshift{1.518018in}{0.892444in}%
\pgfsys@useobject{currentmarker}{}%
\end{pgfscope}%
\begin{pgfscope}%
\pgfsys@transformshift{0.733265in}{0.877808in}%
\pgfsys@useobject{currentmarker}{}%
\end{pgfscope}%
\begin{pgfscope}%
\pgfsys@transformshift{0.558701in}{0.878239in}%
\pgfsys@useobject{currentmarker}{}%
\end{pgfscope}%
\begin{pgfscope}%
\pgfsys@transformshift{0.600190in}{1.086223in}%
\pgfsys@useobject{currentmarker}{}%
\end{pgfscope}%
\begin{pgfscope}%
\pgfsys@transformshift{1.222060in}{1.335359in}%
\pgfsys@useobject{currentmarker}{}%
\end{pgfscope}%
\begin{pgfscope}%
\pgfsys@transformshift{0.884484in}{0.785010in}%
\pgfsys@useobject{currentmarker}{}%
\end{pgfscope}%
\begin{pgfscope}%
\pgfsys@transformshift{1.004359in}{0.992883in}%
\pgfsys@useobject{currentmarker}{}%
\end{pgfscope}%
\begin{pgfscope}%
\pgfsys@transformshift{0.853585in}{0.699407in}%
\pgfsys@useobject{currentmarker}{}%
\end{pgfscope}%
\begin{pgfscope}%
\pgfsys@transformshift{0.668357in}{1.103092in}%
\pgfsys@useobject{currentmarker}{}%
\end{pgfscope}%
\begin{pgfscope}%
\pgfsys@transformshift{0.536244in}{0.923876in}%
\pgfsys@useobject{currentmarker}{}%
\end{pgfscope}%
\begin{pgfscope}%
\pgfsys@transformshift{0.785103in}{0.893197in}%
\pgfsys@useobject{currentmarker}{}%
\end{pgfscope}%
\begin{pgfscope}%
\pgfsys@transformshift{1.369965in}{1.150680in}%
\pgfsys@useobject{currentmarker}{}%
\end{pgfscope}%
\begin{pgfscope}%
\pgfsys@transformshift{0.794379in}{0.607804in}%
\pgfsys@useobject{currentmarker}{}%
\end{pgfscope}%
\begin{pgfscope}%
\pgfsys@transformshift{0.633237in}{1.029348in}%
\pgfsys@useobject{currentmarker}{}%
\end{pgfscope}%
\begin{pgfscope}%
\pgfsys@transformshift{0.859287in}{0.801746in}%
\pgfsys@useobject{currentmarker}{}%
\end{pgfscope}%
\begin{pgfscope}%
\pgfsys@transformshift{1.032685in}{0.855220in}%
\pgfsys@useobject{currentmarker}{}%
\end{pgfscope}%
\begin{pgfscope}%
\pgfsys@transformshift{0.784141in}{0.741884in}%
\pgfsys@useobject{currentmarker}{}%
\end{pgfscope}%
\begin{pgfscope}%
\pgfsys@transformshift{0.996010in}{1.000463in}%
\pgfsys@useobject{currentmarker}{}%
\end{pgfscope}%
\begin{pgfscope}%
\pgfsys@transformshift{1.002563in}{0.834521in}%
\pgfsys@useobject{currentmarker}{}%
\end{pgfscope}%
\begin{pgfscope}%
\pgfsys@transformshift{0.930287in}{1.129437in}%
\pgfsys@useobject{currentmarker}{}%
\end{pgfscope}%
\begin{pgfscope}%
\pgfsys@transformshift{0.659693in}{0.893696in}%
\pgfsys@useobject{currentmarker}{}%
\end{pgfscope}%
\begin{pgfscope}%
\pgfsys@transformshift{1.068472in}{1.256105in}%
\pgfsys@useobject{currentmarker}{}%
\end{pgfscope}%
\begin{pgfscope}%
\pgfsys@transformshift{0.856640in}{1.072449in}%
\pgfsys@useobject{currentmarker}{}%
\end{pgfscope}%
\begin{pgfscope}%
\pgfsys@transformshift{1.405029in}{1.197215in}%
\pgfsys@useobject{currentmarker}{}%
\end{pgfscope}%
\begin{pgfscope}%
\pgfsys@transformshift{1.220135in}{1.266806in}%
\pgfsys@useobject{currentmarker}{}%
\end{pgfscope}%
\begin{pgfscope}%
\pgfsys@transformshift{0.999324in}{0.982107in}%
\pgfsys@useobject{currentmarker}{}%
\end{pgfscope}%
\begin{pgfscope}%
\pgfsys@transformshift{0.633699in}{0.846329in}%
\pgfsys@useobject{currentmarker}{}%
\end{pgfscope}%
\begin{pgfscope}%
\pgfsys@transformshift{1.135768in}{1.181609in}%
\pgfsys@useobject{currentmarker}{}%
\end{pgfscope}%
\begin{pgfscope}%
\pgfsys@transformshift{1.116959in}{0.676314in}%
\pgfsys@useobject{currentmarker}{}%
\end{pgfscope}%
\begin{pgfscope}%
\pgfsys@transformshift{0.598172in}{1.034008in}%
\pgfsys@useobject{currentmarker}{}%
\end{pgfscope}%
\begin{pgfscope}%
\pgfsys@transformshift{0.619907in}{0.954656in}%
\pgfsys@useobject{currentmarker}{}%
\end{pgfscope}%
\begin{pgfscope}%
\pgfsys@transformshift{1.023539in}{0.858943in}%
\pgfsys@useobject{currentmarker}{}%
\end{pgfscope}%
\begin{pgfscope}%
\pgfsys@transformshift{1.456182in}{0.887445in}%
\pgfsys@useobject{currentmarker}{}%
\end{pgfscope}%
\begin{pgfscope}%
\pgfsys@transformshift{0.579547in}{1.215661in}%
\pgfsys@useobject{currentmarker}{}%
\end{pgfscope}%
\begin{pgfscope}%
\pgfsys@transformshift{1.012949in}{1.036780in}%
\pgfsys@useobject{currentmarker}{}%
\end{pgfscope}%
\begin{pgfscope}%
\pgfsys@transformshift{1.111293in}{0.824982in}%
\pgfsys@useobject{currentmarker}{}%
\end{pgfscope}%
\begin{pgfscope}%
\pgfsys@transformshift{0.977218in}{0.904936in}%
\pgfsys@useobject{currentmarker}{}%
\end{pgfscope}%
\begin{pgfscope}%
\pgfsys@transformshift{1.304371in}{1.278034in}%
\pgfsys@useobject{currentmarker}{}%
\end{pgfscope}%
\begin{pgfscope}%
\pgfsys@transformshift{1.456756in}{1.456439in}%
\pgfsys@useobject{currentmarker}{}%
\end{pgfscope}%
\begin{pgfscope}%
\pgfsys@transformshift{0.684908in}{0.890467in}%
\pgfsys@useobject{currentmarker}{}%
\end{pgfscope}%
\begin{pgfscope}%
\pgfsys@transformshift{0.997176in}{0.588700in}%
\pgfsys@useobject{currentmarker}{}%
\end{pgfscope}%
\begin{pgfscope}%
\pgfsys@transformshift{1.265659in}{1.327353in}%
\pgfsys@useobject{currentmarker}{}%
\end{pgfscope}%
\begin{pgfscope}%
\pgfsys@transformshift{0.624998in}{0.802896in}%
\pgfsys@useobject{currentmarker}{}%
\end{pgfscope}%
\begin{pgfscope}%
\pgfsys@transformshift{0.771255in}{1.408931in}%
\pgfsys@useobject{currentmarker}{}%
\end{pgfscope}%
\begin{pgfscope}%
\pgfsys@transformshift{0.828869in}{0.962018in}%
\pgfsys@useobject{currentmarker}{}%
\end{pgfscope}%
\begin{pgfscope}%
\pgfsys@transformshift{0.739412in}{0.675779in}%
\pgfsys@useobject{currentmarker}{}%
\end{pgfscope}%
\begin{pgfscope}%
\pgfsys@transformshift{0.873931in}{0.697734in}%
\pgfsys@useobject{currentmarker}{}%
\end{pgfscope}%
\begin{pgfscope}%
\pgfsys@transformshift{0.630219in}{0.809235in}%
\pgfsys@useobject{currentmarker}{}%
\end{pgfscope}%
\begin{pgfscope}%
\pgfsys@transformshift{1.172814in}{1.208990in}%
\pgfsys@useobject{currentmarker}{}%
\end{pgfscope}%
\begin{pgfscope}%
\pgfsys@transformshift{0.647992in}{0.733610in}%
\pgfsys@useobject{currentmarker}{}%
\end{pgfscope}%
\begin{pgfscope}%
\pgfsys@transformshift{1.062251in}{0.733006in}%
\pgfsys@useobject{currentmarker}{}%
\end{pgfscope}%
\begin{pgfscope}%
\pgfsys@transformshift{1.096816in}{1.122894in}%
\pgfsys@useobject{currentmarker}{}%
\end{pgfscope}%
\begin{pgfscope}%
\pgfsys@transformshift{1.149450in}{0.867613in}%
\pgfsys@useobject{currentmarker}{}%
\end{pgfscope}%
\begin{pgfscope}%
\pgfsys@transformshift{0.677632in}{0.974882in}%
\pgfsys@useobject{currentmarker}{}%
\end{pgfscope}%
\begin{pgfscope}%
\pgfsys@transformshift{1.294966in}{1.367337in}%
\pgfsys@useobject{currentmarker}{}%
\end{pgfscope}%
\begin{pgfscope}%
\pgfsys@transformshift{1.468198in}{0.904115in}%
\pgfsys@useobject{currentmarker}{}%
\end{pgfscope}%
\begin{pgfscope}%
\pgfsys@transformshift{0.730525in}{0.851696in}%
\pgfsys@useobject{currentmarker}{}%
\end{pgfscope}%
\begin{pgfscope}%
\pgfsys@transformshift{0.597968in}{0.911588in}%
\pgfsys@useobject{currentmarker}{}%
\end{pgfscope}%
\begin{pgfscope}%
\pgfsys@transformshift{0.539928in}{1.255638in}%
\pgfsys@useobject{currentmarker}{}%
\end{pgfscope}%
\begin{pgfscope}%
\pgfsys@transformshift{0.782197in}{1.435360in}%
\pgfsys@useobject{currentmarker}{}%
\end{pgfscope}%
\begin{pgfscope}%
\pgfsys@transformshift{0.726193in}{0.919882in}%
\pgfsys@useobject{currentmarker}{}%
\end{pgfscope}%
\begin{pgfscope}%
\pgfsys@transformshift{0.756148in}{0.961408in}%
\pgfsys@useobject{currentmarker}{}%
\end{pgfscope}%
\begin{pgfscope}%
\pgfsys@transformshift{0.663543in}{0.952255in}%
\pgfsys@useobject{currentmarker}{}%
\end{pgfscope}%
\begin{pgfscope}%
\pgfsys@transformshift{1.069286in}{1.155159in}%
\pgfsys@useobject{currentmarker}{}%
\end{pgfscope}%
\begin{pgfscope}%
\pgfsys@transformshift{1.411009in}{0.937064in}%
\pgfsys@useobject{currentmarker}{}%
\end{pgfscope}%
\begin{pgfscope}%
\pgfsys@transformshift{1.079839in}{1.107281in}%
\pgfsys@useobject{currentmarker}{}%
\end{pgfscope}%
\begin{pgfscope}%
\pgfsys@transformshift{1.104758in}{0.826388in}%
\pgfsys@useobject{currentmarker}{}%
\end{pgfscope}%
\begin{pgfscope}%
\pgfsys@transformshift{1.128918in}{0.864622in}%
\pgfsys@useobject{currentmarker}{}%
\end{pgfscope}%
\begin{pgfscope}%
\pgfsys@transformshift{1.010728in}{1.049867in}%
\pgfsys@useobject{currentmarker}{}%
\end{pgfscope}%
\begin{pgfscope}%
\pgfsys@transformshift{0.687426in}{1.043600in}%
\pgfsys@useobject{currentmarker}{}%
\end{pgfscope}%
\begin{pgfscope}%
\pgfsys@transformshift{1.166556in}{1.018588in}%
\pgfsys@useobject{currentmarker}{}%
\end{pgfscope}%
\begin{pgfscope}%
\pgfsys@transformshift{0.539336in}{1.014262in}%
\pgfsys@useobject{currentmarker}{}%
\end{pgfscope}%
\begin{pgfscope}%
\pgfsys@transformshift{1.057604in}{0.793105in}%
\pgfsys@useobject{currentmarker}{}%
\end{pgfscope}%
\begin{pgfscope}%
\pgfsys@transformshift{0.759925in}{0.687475in}%
\pgfsys@useobject{currentmarker}{}%
\end{pgfscope}%
\begin{pgfscope}%
\pgfsys@transformshift{1.233927in}{0.641960in}%
\pgfsys@useobject{currentmarker}{}%
\end{pgfscope}%
\begin{pgfscope}%
\pgfsys@transformshift{0.557127in}{0.825170in}%
\pgfsys@useobject{currentmarker}{}%
\end{pgfscope}%
\begin{pgfscope}%
\pgfsys@transformshift{0.685815in}{0.891069in}%
\pgfsys@useobject{currentmarker}{}%
\end{pgfscope}%
\begin{pgfscope}%
\pgfsys@transformshift{0.794138in}{0.592518in}%
\pgfsys@useobject{currentmarker}{}%
\end{pgfscope}%
\begin{pgfscope}%
\pgfsys@transformshift{0.700941in}{0.793319in}%
\pgfsys@useobject{currentmarker}{}%
\end{pgfscope}%
\begin{pgfscope}%
\pgfsys@transformshift{0.901516in}{0.680919in}%
\pgfsys@useobject{currentmarker}{}%
\end{pgfscope}%
\begin{pgfscope}%
\pgfsys@transformshift{0.999546in}{0.772087in}%
\pgfsys@useobject{currentmarker}{}%
\end{pgfscope}%
\begin{pgfscope}%
\pgfsys@transformshift{1.495820in}{1.368699in}%
\pgfsys@useobject{currentmarker}{}%
\end{pgfscope}%
\begin{pgfscope}%
\pgfsys@transformshift{1.570837in}{0.823929in}%
\pgfsys@useobject{currentmarker}{}%
\end{pgfscope}%
\begin{pgfscope}%
\pgfsys@transformshift{0.930379in}{1.069829in}%
\pgfsys@useobject{currentmarker}{}%
\end{pgfscope}%
\begin{pgfscope}%
\pgfsys@transformshift{0.725045in}{0.822706in}%
\pgfsys@useobject{currentmarker}{}%
\end{pgfscope}%
\begin{pgfscope}%
\pgfsys@transformshift{0.663488in}{0.789491in}%
\pgfsys@useobject{currentmarker}{}%
\end{pgfscope}%
\begin{pgfscope}%
\pgfsys@transformshift{0.729081in}{0.861385in}%
\pgfsys@useobject{currentmarker}{}%
\end{pgfscope}%
\begin{pgfscope}%
\pgfsys@transformshift{0.600893in}{0.888704in}%
\pgfsys@useobject{currentmarker}{}%
\end{pgfscope}%
\begin{pgfscope}%
\pgfsys@transformshift{0.769145in}{0.976084in}%
\pgfsys@useobject{currentmarker}{}%
\end{pgfscope}%
\begin{pgfscope}%
\pgfsys@transformshift{0.677817in}{0.784533in}%
\pgfsys@useobject{currentmarker}{}%
\end{pgfscope}%
\begin{pgfscope}%
\pgfsys@transformshift{0.942968in}{1.089128in}%
\pgfsys@useobject{currentmarker}{}%
\end{pgfscope}%
\begin{pgfscope}%
\pgfsys@transformshift{1.032870in}{1.284229in}%
\pgfsys@useobject{currentmarker}{}%
\end{pgfscope}%
\begin{pgfscope}%
\pgfsys@transformshift{0.670412in}{1.158681in}%
\pgfsys@useobject{currentmarker}{}%
\end{pgfscope}%
\begin{pgfscope}%
\pgfsys@transformshift{0.718584in}{0.872053in}%
\pgfsys@useobject{currentmarker}{}%
\end{pgfscope}%
\begin{pgfscope}%
\pgfsys@transformshift{0.609262in}{1.069025in}%
\pgfsys@useobject{currentmarker}{}%
\end{pgfscope}%
\begin{pgfscope}%
\pgfsys@transformshift{0.724620in}{0.865367in}%
\pgfsys@useobject{currentmarker}{}%
\end{pgfscope}%
\begin{pgfscope}%
\pgfsys@transformshift{0.839274in}{0.672901in}%
\pgfsys@useobject{currentmarker}{}%
\end{pgfscope}%
\begin{pgfscope}%
\pgfsys@transformshift{1.123142in}{1.152894in}%
\pgfsys@useobject{currentmarker}{}%
\end{pgfscope}%
\begin{pgfscope}%
\pgfsys@transformshift{0.660581in}{0.877096in}%
\pgfsys@useobject{currentmarker}{}%
\end{pgfscope}%
\begin{pgfscope}%
\pgfsys@transformshift{1.346619in}{1.297377in}%
\pgfsys@useobject{currentmarker}{}%
\end{pgfscope}%
\begin{pgfscope}%
\pgfsys@transformshift{1.032759in}{0.823447in}%
\pgfsys@useobject{currentmarker}{}%
\end{pgfscope}%
\begin{pgfscope}%
\pgfsys@transformshift{1.531144in}{0.885266in}%
\pgfsys@useobject{currentmarker}{}%
\end{pgfscope}%
\begin{pgfscope}%
\pgfsys@transformshift{0.545427in}{1.140025in}%
\pgfsys@useobject{currentmarker}{}%
\end{pgfscope}%
\begin{pgfscope}%
\pgfsys@transformshift{1.530607in}{0.875244in}%
\pgfsys@useobject{currentmarker}{}%
\end{pgfscope}%
\begin{pgfscope}%
\pgfsys@transformshift{1.562080in}{0.890019in}%
\pgfsys@useobject{currentmarker}{}%
\end{pgfscope}%
\begin{pgfscope}%
\pgfsys@transformshift{1.076266in}{1.680141in}%
\pgfsys@useobject{currentmarker}{}%
\end{pgfscope}%
\begin{pgfscope}%
\pgfsys@transformshift{0.596024in}{1.250807in}%
\pgfsys@useobject{currentmarker}{}%
\end{pgfscope}%
\begin{pgfscope}%
\pgfsys@transformshift{1.283932in}{0.647873in}%
\pgfsys@useobject{currentmarker}{}%
\end{pgfscope}%
\begin{pgfscope}%
\pgfsys@transformshift{1.277638in}{1.250868in}%
\pgfsys@useobject{currentmarker}{}%
\end{pgfscope}%
\begin{pgfscope}%
\pgfsys@transformshift{0.915846in}{0.801091in}%
\pgfsys@useobject{currentmarker}{}%
\end{pgfscope}%
\begin{pgfscope}%
\pgfsys@transformshift{0.580084in}{0.910246in}%
\pgfsys@useobject{currentmarker}{}%
\end{pgfscope}%
\begin{pgfscope}%
\pgfsys@transformshift{0.665210in}{0.913753in}%
\pgfsys@useobject{currentmarker}{}%
\end{pgfscope}%
\begin{pgfscope}%
\pgfsys@transformshift{1.282470in}{1.271512in}%
\pgfsys@useobject{currentmarker}{}%
\end{pgfscope}%
\begin{pgfscope}%
\pgfsys@transformshift{0.656693in}{1.019270in}%
\pgfsys@useobject{currentmarker}{}%
\end{pgfscope}%
\begin{pgfscope}%
\pgfsys@transformshift{1.353432in}{1.021844in}%
\pgfsys@useobject{currentmarker}{}%
\end{pgfscope}%
\begin{pgfscope}%
\pgfsys@transformshift{0.684445in}{1.777300in}%
\pgfsys@useobject{currentmarker}{}%
\end{pgfscope}%
\begin{pgfscope}%
\pgfsys@transformshift{1.918484in}{0.925393in}%
\pgfsys@useobject{currentmarker}{}%
\end{pgfscope}%
\begin{pgfscope}%
\pgfsys@transformshift{1.424728in}{0.910753in}%
\pgfsys@useobject{currentmarker}{}%
\end{pgfscope}%
\begin{pgfscope}%
\pgfsys@transformshift{0.746503in}{0.812938in}%
\pgfsys@useobject{currentmarker}{}%
\end{pgfscope}%
\begin{pgfscope}%
\pgfsys@transformshift{0.827814in}{1.025034in}%
\pgfsys@useobject{currentmarker}{}%
\end{pgfscope}%
\begin{pgfscope}%
\pgfsys@transformshift{1.171537in}{0.811380in}%
\pgfsys@useobject{currentmarker}{}%
\end{pgfscope}%
\begin{pgfscope}%
\pgfsys@transformshift{1.125382in}{1.488321in}%
\pgfsys@useobject{currentmarker}{}%
\end{pgfscope}%
\begin{pgfscope}%
\pgfsys@transformshift{1.248979in}{1.230001in}%
\pgfsys@useobject{currentmarker}{}%
\end{pgfscope}%
\begin{pgfscope}%
\pgfsys@transformshift{0.729433in}{1.088847in}%
\pgfsys@useobject{currentmarker}{}%
\end{pgfscope}%
\begin{pgfscope}%
\pgfsys@transformshift{0.585879in}{0.980433in}%
\pgfsys@useobject{currentmarker}{}%
\end{pgfscope}%
\begin{pgfscope}%
\pgfsys@transformshift{1.040794in}{1.553972in}%
\pgfsys@useobject{currentmarker}{}%
\end{pgfscope}%
\begin{pgfscope}%
\pgfsys@transformshift{0.953003in}{1.170597in}%
\pgfsys@useobject{currentmarker}{}%
\end{pgfscope}%
\begin{pgfscope}%
\pgfsys@transformshift{1.076747in}{0.927833in}%
\pgfsys@useobject{currentmarker}{}%
\end{pgfscope}%
\begin{pgfscope}%
\pgfsys@transformshift{0.880318in}{1.310503in}%
\pgfsys@useobject{currentmarker}{}%
\end{pgfscope}%
\begin{pgfscope}%
\pgfsys@transformshift{1.223782in}{1.500464in}%
\pgfsys@useobject{currentmarker}{}%
\end{pgfscope}%
\begin{pgfscope}%
\pgfsys@transformshift{1.089466in}{1.240806in}%
\pgfsys@useobject{currentmarker}{}%
\end{pgfscope}%
\begin{pgfscope}%
\pgfsys@transformshift{0.826666in}{0.964220in}%
\pgfsys@useobject{currentmarker}{}%
\end{pgfscope}%
\begin{pgfscope}%
\pgfsys@transformshift{1.377851in}{0.896120in}%
\pgfsys@useobject{currentmarker}{}%
\end{pgfscope}%
\begin{pgfscope}%
\pgfsys@transformshift{0.598024in}{0.935332in}%
\pgfsys@useobject{currentmarker}{}%
\end{pgfscope}%
\begin{pgfscope}%
\pgfsys@transformshift{0.697849in}{0.689816in}%
\pgfsys@useobject{currentmarker}{}%
\end{pgfscope}%
\begin{pgfscope}%
\pgfsys@transformshift{1.266270in}{0.948085in}%
\pgfsys@useobject{currentmarker}{}%
\end{pgfscope}%
\begin{pgfscope}%
\pgfsys@transformshift{0.769293in}{0.941304in}%
\pgfsys@useobject{currentmarker}{}%
\end{pgfscope}%
\begin{pgfscope}%
\pgfsys@transformshift{0.763591in}{0.772835in}%
\pgfsys@useobject{currentmarker}{}%
\end{pgfscope}%
\begin{pgfscope}%
\pgfsys@transformshift{1.545270in}{0.816116in}%
\pgfsys@useobject{currentmarker}{}%
\end{pgfscope}%
\begin{pgfscope}%
\pgfsys@transformshift{0.781974in}{0.753595in}%
\pgfsys@useobject{currentmarker}{}%
\end{pgfscope}%
\begin{pgfscope}%
\pgfsys@transformshift{0.637143in}{0.836720in}%
\pgfsys@useobject{currentmarker}{}%
\end{pgfscope}%
\begin{pgfscope}%
\pgfsys@transformshift{1.059196in}{0.906398in}%
\pgfsys@useobject{currentmarker}{}%
\end{pgfscope}%
\begin{pgfscope}%
\pgfsys@transformshift{1.032796in}{1.670651in}%
\pgfsys@useobject{currentmarker}{}%
\end{pgfscope}%
\begin{pgfscope}%
\pgfsys@transformshift{0.683742in}{1.036939in}%
\pgfsys@useobject{currentmarker}{}%
\end{pgfscope}%
\begin{pgfscope}%
\pgfsys@transformshift{1.021114in}{1.269815in}%
\pgfsys@useobject{currentmarker}{}%
\end{pgfscope}%
\begin{pgfscope}%
\pgfsys@transformshift{0.563237in}{1.026975in}%
\pgfsys@useobject{currentmarker}{}%
\end{pgfscope}%
\begin{pgfscope}%
\pgfsys@transformshift{0.614223in}{0.838580in}%
\pgfsys@useobject{currentmarker}{}%
\end{pgfscope}%
\begin{pgfscope}%
\pgfsys@transformshift{0.779401in}{0.904504in}%
\pgfsys@useobject{currentmarker}{}%
\end{pgfscope}%
\begin{pgfscope}%
\pgfsys@transformshift{0.992603in}{1.273796in}%
\pgfsys@useobject{currentmarker}{}%
\end{pgfscope}%
\begin{pgfscope}%
\pgfsys@transformshift{1.694285in}{0.817679in}%
\pgfsys@useobject{currentmarker}{}%
\end{pgfscope}%
\begin{pgfscope}%
\pgfsys@transformshift{0.778624in}{0.610664in}%
\pgfsys@useobject{currentmarker}{}%
\end{pgfscope}%
\begin{pgfscope}%
\pgfsys@transformshift{1.340343in}{1.269025in}%
\pgfsys@useobject{currentmarker}{}%
\end{pgfscope}%
\begin{pgfscope}%
\pgfsys@transformshift{0.923566in}{0.967461in}%
\pgfsys@useobject{currentmarker}{}%
\end{pgfscope}%
\begin{pgfscope}%
\pgfsys@transformshift{0.789621in}{0.551041in}%
\pgfsys@useobject{currentmarker}{}%
\end{pgfscope}%
\begin{pgfscope}%
\pgfsys@transformshift{0.869895in}{0.603304in}%
\pgfsys@useobject{currentmarker}{}%
\end{pgfscope}%
\begin{pgfscope}%
\pgfsys@transformshift{1.119513in}{0.761120in}%
\pgfsys@useobject{currentmarker}{}%
\end{pgfscope}%
\begin{pgfscope}%
\pgfsys@transformshift{1.521480in}{1.356202in}%
\pgfsys@useobject{currentmarker}{}%
\end{pgfscope}%
\begin{pgfscope}%
\pgfsys@transformshift{1.492025in}{0.904167in}%
\pgfsys@useobject{currentmarker}{}%
\end{pgfscope}%
\begin{pgfscope}%
\pgfsys@transformshift{1.103758in}{1.761569in}%
\pgfsys@useobject{currentmarker}{}%
\end{pgfscope}%
\begin{pgfscope}%
\pgfsys@transformshift{0.698127in}{0.784400in}%
\pgfsys@useobject{currentmarker}{}%
\end{pgfscope}%
\begin{pgfscope}%
\pgfsys@transformshift{1.029871in}{0.911932in}%
\pgfsys@useobject{currentmarker}{}%
\end{pgfscope}%
\begin{pgfscope}%
\pgfsys@transformshift{0.931582in}{1.114156in}%
\pgfsys@useobject{currentmarker}{}%
\end{pgfscope}%
\begin{pgfscope}%
\pgfsys@transformshift{0.910792in}{0.805236in}%
\pgfsys@useobject{currentmarker}{}%
\end{pgfscope}%
\begin{pgfscope}%
\pgfsys@transformshift{0.605355in}{0.798715in}%
\pgfsys@useobject{currentmarker}{}%
\end{pgfscope}%
\begin{pgfscope}%
\pgfsys@transformshift{1.054013in}{1.254997in}%
\pgfsys@useobject{currentmarker}{}%
\end{pgfscope}%
\begin{pgfscope}%
\pgfsys@transformshift{1.098612in}{0.771485in}%
\pgfsys@useobject{currentmarker}{}%
\end{pgfscope}%
\begin{pgfscope}%
\pgfsys@transformshift{0.861509in}{1.070368in}%
\pgfsys@useobject{currentmarker}{}%
\end{pgfscope}%
\begin{pgfscope}%
\pgfsys@transformshift{0.582121in}{1.343787in}%
\pgfsys@useobject{currentmarker}{}%
\end{pgfscope}%
\begin{pgfscope}%
\pgfsys@transformshift{1.167241in}{0.760505in}%
\pgfsys@useobject{currentmarker}{}%
\end{pgfscope}%
\begin{pgfscope}%
\pgfsys@transformshift{1.281729in}{1.232713in}%
\pgfsys@useobject{currentmarker}{}%
\end{pgfscope}%
\begin{pgfscope}%
\pgfsys@transformshift{1.375426in}{1.299538in}%
\pgfsys@useobject{currentmarker}{}%
\end{pgfscope}%
\begin{pgfscope}%
\pgfsys@transformshift{1.374241in}{0.943348in}%
\pgfsys@useobject{currentmarker}{}%
\end{pgfscope}%
\begin{pgfscope}%
\pgfsys@transformshift{0.897036in}{1.242867in}%
\pgfsys@useobject{currentmarker}{}%
\end{pgfscope}%
\begin{pgfscope}%
\pgfsys@transformshift{1.073507in}{1.054413in}%
\pgfsys@useobject{currentmarker}{}%
\end{pgfscope}%
\begin{pgfscope}%
\pgfsys@transformshift{0.689259in}{1.218727in}%
\pgfsys@useobject{currentmarker}{}%
\end{pgfscope}%
\begin{pgfscope}%
\pgfsys@transformshift{1.117181in}{0.765234in}%
\pgfsys@useobject{currentmarker}{}%
\end{pgfscope}%
\begin{pgfscope}%
\pgfsys@transformshift{1.040257in}{1.265863in}%
\pgfsys@useobject{currentmarker}{}%
\end{pgfscope}%
\begin{pgfscope}%
\pgfsys@transformshift{0.757259in}{1.394056in}%
\pgfsys@useobject{currentmarker}{}%
\end{pgfscope}%
\begin{pgfscope}%
\pgfsys@transformshift{0.929602in}{1.569900in}%
\pgfsys@useobject{currentmarker}{}%
\end{pgfscope}%
\begin{pgfscope}%
\pgfsys@transformshift{0.739375in}{0.655373in}%
\pgfsys@useobject{currentmarker}{}%
\end{pgfscope}%
\begin{pgfscope}%
\pgfsys@transformshift{1.051791in}{0.732929in}%
\pgfsys@useobject{currentmarker}{}%
\end{pgfscope}%
\begin{pgfscope}%
\pgfsys@transformshift{1.033888in}{0.926755in}%
\pgfsys@useobject{currentmarker}{}%
\end{pgfscope}%
\begin{pgfscope}%
\pgfsys@transformshift{1.087559in}{0.737802in}%
\pgfsys@useobject{currentmarker}{}%
\end{pgfscope}%
\begin{pgfscope}%
\pgfsys@transformshift{1.159095in}{0.671487in}%
\pgfsys@useobject{currentmarker}{}%
\end{pgfscope}%
\begin{pgfscope}%
\pgfsys@transformshift{0.813373in}{0.984210in}%
\pgfsys@useobject{currentmarker}{}%
\end{pgfscope}%
\begin{pgfscope}%
\pgfsys@transformshift{1.640577in}{0.924636in}%
\pgfsys@useobject{currentmarker}{}%
\end{pgfscope}%
\begin{pgfscope}%
\pgfsys@transformshift{2.178211in}{0.881620in}%
\pgfsys@useobject{currentmarker}{}%
\end{pgfscope}%
\begin{pgfscope}%
\pgfsys@transformshift{0.693850in}{0.928788in}%
\pgfsys@useobject{currentmarker}{}%
\end{pgfscope}%
\begin{pgfscope}%
\pgfsys@transformshift{1.166519in}{0.791930in}%
\pgfsys@useobject{currentmarker}{}%
\end{pgfscope}%
\begin{pgfscope}%
\pgfsys@transformshift{0.643734in}{0.832018in}%
\pgfsys@useobject{currentmarker}{}%
\end{pgfscope}%
\begin{pgfscope}%
\pgfsys@transformshift{0.891278in}{0.922718in}%
\pgfsys@useobject{currentmarker}{}%
\end{pgfscope}%
\begin{pgfscope}%
\pgfsys@transformshift{0.789639in}{0.590611in}%
\pgfsys@useobject{currentmarker}{}%
\end{pgfscope}%
\begin{pgfscope}%
\pgfsys@transformshift{0.905830in}{0.987814in}%
\pgfsys@useobject{currentmarker}{}%
\end{pgfscope}%
\begin{pgfscope}%
\pgfsys@transformshift{1.170666in}{1.030632in}%
\pgfsys@useobject{currentmarker}{}%
\end{pgfscope}%
\begin{pgfscope}%
\pgfsys@transformshift{0.851123in}{1.162053in}%
\pgfsys@useobject{currentmarker}{}%
\end{pgfscope}%
\begin{pgfscope}%
\pgfsys@transformshift{1.511890in}{1.448300in}%
\pgfsys@useobject{currentmarker}{}%
\end{pgfscope}%
\begin{pgfscope}%
\pgfsys@transformshift{1.126178in}{0.759905in}%
\pgfsys@useobject{currentmarker}{}%
\end{pgfscope}%
\begin{pgfscope}%
\pgfsys@transformshift{0.794434in}{1.241416in}%
\pgfsys@useobject{currentmarker}{}%
\end{pgfscope}%
\begin{pgfscope}%
\pgfsys@transformshift{1.144285in}{0.721600in}%
\pgfsys@useobject{currentmarker}{}%
\end{pgfscope}%
\begin{pgfscope}%
\pgfsys@transformshift{0.668153in}{0.945127in}%
\pgfsys@useobject{currentmarker}{}%
\end{pgfscope}%
\begin{pgfscope}%
\pgfsys@transformshift{0.919012in}{0.753226in}%
\pgfsys@useobject{currentmarker}{}%
\end{pgfscope}%
\begin{pgfscope}%
\pgfsys@transformshift{1.167260in}{1.205416in}%
\pgfsys@useobject{currentmarker}{}%
\end{pgfscope}%
\begin{pgfscope}%
\pgfsys@transformshift{0.990696in}{1.123591in}%
\pgfsys@useobject{currentmarker}{}%
\end{pgfscope}%
\begin{pgfscope}%
\pgfsys@transformshift{0.884725in}{0.716027in}%
\pgfsys@useobject{currentmarker}{}%
\end{pgfscope}%
\begin{pgfscope}%
\pgfsys@transformshift{0.825592in}{0.847308in}%
\pgfsys@useobject{currentmarker}{}%
\end{pgfscope}%
\begin{pgfscope}%
\pgfsys@transformshift{0.824667in}{0.725647in}%
\pgfsys@useobject{currentmarker}{}%
\end{pgfscope}%
\begin{pgfscope}%
\pgfsys@transformshift{1.211526in}{1.371208in}%
\pgfsys@useobject{currentmarker}{}%
\end{pgfscope}%
\begin{pgfscope}%
\pgfsys@transformshift{1.235205in}{0.999647in}%
\pgfsys@useobject{currentmarker}{}%
\end{pgfscope}%
\begin{pgfscope}%
\pgfsys@transformshift{1.055049in}{1.089422in}%
\pgfsys@useobject{currentmarker}{}%
\end{pgfscope}%
\begin{pgfscope}%
\pgfsys@transformshift{1.121531in}{1.549552in}%
\pgfsys@useobject{currentmarker}{}%
\end{pgfscope}%
\begin{pgfscope}%
\pgfsys@transformshift{1.293522in}{0.610104in}%
\pgfsys@useobject{currentmarker}{}%
\end{pgfscope}%
\begin{pgfscope}%
\pgfsys@transformshift{0.598931in}{0.718435in}%
\pgfsys@useobject{currentmarker}{}%
\end{pgfscope}%
\begin{pgfscope}%
\pgfsys@transformshift{0.969776in}{1.149681in}%
\pgfsys@useobject{currentmarker}{}%
\end{pgfscope}%
\begin{pgfscope}%
\pgfsys@transformshift{1.073637in}{0.929308in}%
\pgfsys@useobject{currentmarker}{}%
\end{pgfscope}%
\begin{pgfscope}%
\pgfsys@transformshift{1.012061in}{1.264174in}%
\pgfsys@useobject{currentmarker}{}%
\end{pgfscope}%
\begin{pgfscope}%
\pgfsys@transformshift{0.981662in}{0.974855in}%
\pgfsys@useobject{currentmarker}{}%
\end{pgfscope}%
\begin{pgfscope}%
\pgfsys@transformshift{0.639994in}{0.956630in}%
\pgfsys@useobject{currentmarker}{}%
\end{pgfscope}%
\begin{pgfscope}%
\pgfsys@transformshift{0.958131in}{1.159513in}%
\pgfsys@useobject{currentmarker}{}%
\end{pgfscope}%
\begin{pgfscope}%
\pgfsys@transformshift{0.822908in}{0.702095in}%
\pgfsys@useobject{currentmarker}{}%
\end{pgfscope}%
\begin{pgfscope}%
\pgfsys@transformshift{0.720176in}{0.712609in}%
\pgfsys@useobject{currentmarker}{}%
\end{pgfscope}%
\begin{pgfscope}%
\pgfsys@transformshift{0.636847in}{0.775666in}%
\pgfsys@useobject{currentmarker}{}%
\end{pgfscope}%
\begin{pgfscope}%
\pgfsys@transformshift{0.850863in}{1.050614in}%
\pgfsys@useobject{currentmarker}{}%
\end{pgfscope}%
\begin{pgfscope}%
\pgfsys@transformshift{1.514574in}{1.186866in}%
\pgfsys@useobject{currentmarker}{}%
\end{pgfscope}%
\begin{pgfscope}%
\pgfsys@transformshift{0.970535in}{1.049239in}%
\pgfsys@useobject{currentmarker}{}%
\end{pgfscope}%
\begin{pgfscope}%
\pgfsys@transformshift{0.976256in}{1.327902in}%
\pgfsys@useobject{currentmarker}{}%
\end{pgfscope}%
\begin{pgfscope}%
\pgfsys@transformshift{0.879633in}{1.412051in}%
\pgfsys@useobject{currentmarker}{}%
\end{pgfscope}%
\begin{pgfscope}%
\pgfsys@transformshift{0.655508in}{0.749003in}%
\pgfsys@useobject{currentmarker}{}%
\end{pgfscope}%
\begin{pgfscope}%
\pgfsys@transformshift{0.702514in}{0.792222in}%
\pgfsys@useobject{currentmarker}{}%
\end{pgfscope}%
\begin{pgfscope}%
\pgfsys@transformshift{0.714474in}{0.683794in}%
\pgfsys@useobject{currentmarker}{}%
\end{pgfscope}%
\begin{pgfscope}%
\pgfsys@transformshift{1.412398in}{0.788113in}%
\pgfsys@useobject{currentmarker}{}%
\end{pgfscope}%
\begin{pgfscope}%
\pgfsys@transformshift{1.041442in}{1.226908in}%
\pgfsys@useobject{currentmarker}{}%
\end{pgfscope}%
\begin{pgfscope}%
\pgfsys@transformshift{0.890575in}{1.046576in}%
\pgfsys@useobject{currentmarker}{}%
\end{pgfscope}%
\begin{pgfscope}%
\pgfsys@transformshift{1.229854in}{0.743293in}%
\pgfsys@useobject{currentmarker}{}%
\end{pgfscope}%
\begin{pgfscope}%
\pgfsys@transformshift{1.163502in}{1.172023in}%
\pgfsys@useobject{currentmarker}{}%
\end{pgfscope}%
\begin{pgfscope}%
\pgfsys@transformshift{0.904941in}{1.236995in}%
\pgfsys@useobject{currentmarker}{}%
\end{pgfscope}%
\begin{pgfscope}%
\pgfsys@transformshift{0.963778in}{1.439806in}%
\pgfsys@useobject{currentmarker}{}%
\end{pgfscope}%
\begin{pgfscope}%
\pgfsys@transformshift{1.162317in}{0.774318in}%
\pgfsys@useobject{currentmarker}{}%
\end{pgfscope}%
\begin{pgfscope}%
\pgfsys@transformshift{0.708402in}{0.717029in}%
\pgfsys@useobject{currentmarker}{}%
\end{pgfscope}%
\begin{pgfscope}%
\pgfsys@transformshift{0.704162in}{0.949131in}%
\pgfsys@useobject{currentmarker}{}%
\end{pgfscope}%
\begin{pgfscope}%
\pgfsys@transformshift{1.154782in}{0.628871in}%
\pgfsys@useobject{currentmarker}{}%
\end{pgfscope}%
\begin{pgfscope}%
\pgfsys@transformshift{0.978699in}{0.682102in}%
\pgfsys@useobject{currentmarker}{}%
\end{pgfscope}%
\begin{pgfscope}%
\pgfsys@transformshift{1.451036in}{1.415403in}%
\pgfsys@useobject{currentmarker}{}%
\end{pgfscope}%
\begin{pgfscope}%
\pgfsys@transformshift{1.034425in}{1.257865in}%
\pgfsys@useobject{currentmarker}{}%
\end{pgfscope}%
\begin{pgfscope}%
\pgfsys@transformshift{0.615926in}{0.947459in}%
\pgfsys@useobject{currentmarker}{}%
\end{pgfscope}%
\begin{pgfscope}%
\pgfsys@transformshift{1.270158in}{0.928753in}%
\pgfsys@useobject{currentmarker}{}%
\end{pgfscope}%
\begin{pgfscope}%
\pgfsys@transformshift{0.708883in}{0.650824in}%
\pgfsys@useobject{currentmarker}{}%
\end{pgfscope}%
\begin{pgfscope}%
\pgfsys@transformshift{0.740763in}{0.701552in}%
\pgfsys@useobject{currentmarker}{}%
\end{pgfscope}%
\begin{pgfscope}%
\pgfsys@transformshift{0.614020in}{0.814075in}%
\pgfsys@useobject{currentmarker}{}%
\end{pgfscope}%
\begin{pgfscope}%
\pgfsys@transformshift{1.067953in}{1.056008in}%
\pgfsys@useobject{currentmarker}{}%
\end{pgfscope}%
\begin{pgfscope}%
\pgfsys@transformshift{0.777420in}{0.937500in}%
\pgfsys@useobject{currentmarker}{}%
\end{pgfscope}%
\begin{pgfscope}%
\pgfsys@transformshift{0.653398in}{0.781078in}%
\pgfsys@useobject{currentmarker}{}%
\end{pgfscope}%
\begin{pgfscope}%
\pgfsys@transformshift{0.733321in}{0.870458in}%
\pgfsys@useobject{currentmarker}{}%
\end{pgfscope}%
\begin{pgfscope}%
\pgfsys@transformshift{0.711753in}{0.969457in}%
\pgfsys@useobject{currentmarker}{}%
\end{pgfscope}%
\begin{pgfscope}%
\pgfsys@transformshift{1.082709in}{1.538016in}%
\pgfsys@useobject{currentmarker}{}%
\end{pgfscope}%
\begin{pgfscope}%
\pgfsys@transformshift{1.677845in}{0.826761in}%
\pgfsys@useobject{currentmarker}{}%
\end{pgfscope}%
\begin{pgfscope}%
\pgfsys@transformshift{0.696497in}{0.765117in}%
\pgfsys@useobject{currentmarker}{}%
\end{pgfscope}%
\begin{pgfscope}%
\pgfsys@transformshift{0.757129in}{0.784852in}%
\pgfsys@useobject{currentmarker}{}%
\end{pgfscope}%
\begin{pgfscope}%
\pgfsys@transformshift{1.072767in}{1.650432in}%
\pgfsys@useobject{currentmarker}{}%
\end{pgfscope}%
\begin{pgfscope}%
\pgfsys@transformshift{1.280452in}{1.358282in}%
\pgfsys@useobject{currentmarker}{}%
\end{pgfscope}%
\begin{pgfscope}%
\pgfsys@transformshift{0.830258in}{0.768816in}%
\pgfsys@useobject{currentmarker}{}%
\end{pgfscope}%
\begin{pgfscope}%
\pgfsys@transformshift{0.709420in}{0.734659in}%
\pgfsys@useobject{currentmarker}{}%
\end{pgfscope}%
\begin{pgfscope}%
\pgfsys@transformshift{1.313295in}{1.232118in}%
\pgfsys@useobject{currentmarker}{}%
\end{pgfscope}%
\begin{pgfscope}%
\pgfsys@transformshift{0.966851in}{1.108358in}%
\pgfsys@useobject{currentmarker}{}%
\end{pgfscope}%
\begin{pgfscope}%
\pgfsys@transformshift{0.979088in}{1.429913in}%
\pgfsys@useobject{currentmarker}{}%
\end{pgfscope}%
\begin{pgfscope}%
\pgfsys@transformshift{0.852955in}{0.691712in}%
\pgfsys@useobject{currentmarker}{}%
\end{pgfscope}%
\begin{pgfscope}%
\pgfsys@transformshift{0.752519in}{0.799427in}%
\pgfsys@useobject{currentmarker}{}%
\end{pgfscope}%
\begin{pgfscope}%
\pgfsys@transformshift{0.942839in}{1.197839in}%
\pgfsys@useobject{currentmarker}{}%
\end{pgfscope}%
\begin{pgfscope}%
\pgfsys@transformshift{0.747724in}{1.143206in}%
\pgfsys@useobject{currentmarker}{}%
\end{pgfscope}%
\begin{pgfscope}%
\pgfsys@transformshift{1.116200in}{1.222649in}%
\pgfsys@useobject{currentmarker}{}%
\end{pgfscope}%
\begin{pgfscope}%
\pgfsys@transformshift{0.695424in}{0.735588in}%
\pgfsys@useobject{currentmarker}{}%
\end{pgfscope}%
\begin{pgfscope}%
\pgfsys@transformshift{1.740661in}{1.211410in}%
\pgfsys@useobject{currentmarker}{}%
\end{pgfscope}%
\begin{pgfscope}%
\pgfsys@transformshift{0.688629in}{1.111728in}%
\pgfsys@useobject{currentmarker}{}%
\end{pgfscope}%
\begin{pgfscope}%
\pgfsys@transformshift{0.892574in}{1.550833in}%
\pgfsys@useobject{currentmarker}{}%
\end{pgfscope}%
\begin{pgfscope}%
\pgfsys@transformshift{0.811004in}{0.886339in}%
\pgfsys@useobject{currentmarker}{}%
\end{pgfscope}%
\begin{pgfscope}%
\pgfsys@transformshift{1.098889in}{1.251496in}%
\pgfsys@useobject{currentmarker}{}%
\end{pgfscope}%
\begin{pgfscope}%
\pgfsys@transformshift{0.812874in}{0.685404in}%
\pgfsys@useobject{currentmarker}{}%
\end{pgfscope}%
\begin{pgfscope}%
\pgfsys@transformshift{0.678317in}{0.920921in}%
\pgfsys@useobject{currentmarker}{}%
\end{pgfscope}%
\begin{pgfscope}%
\pgfsys@transformshift{1.330735in}{0.742629in}%
\pgfsys@useobject{currentmarker}{}%
\end{pgfscope}%
\begin{pgfscope}%
\pgfsys@transformshift{0.845531in}{0.642515in}%
\pgfsys@useobject{currentmarker}{}%
\end{pgfscope}%
\begin{pgfscope}%
\pgfsys@transformshift{1.154911in}{1.115020in}%
\pgfsys@useobject{currentmarker}{}%
\end{pgfscope}%
\begin{pgfscope}%
\pgfsys@transformshift{0.984050in}{1.902939in}%
\pgfsys@useobject{currentmarker}{}%
\end{pgfscope}%
\begin{pgfscope}%
\pgfsys@transformshift{0.736913in}{0.894464in}%
\pgfsys@useobject{currentmarker}{}%
\end{pgfscope}%
\begin{pgfscope}%
\pgfsys@transformshift{0.696775in}{0.694481in}%
\pgfsys@useobject{currentmarker}{}%
\end{pgfscope}%
\begin{pgfscope}%
\pgfsys@transformshift{0.594914in}{0.962410in}%
\pgfsys@useobject{currentmarker}{}%
\end{pgfscope}%
\begin{pgfscope}%
\pgfsys@transformshift{1.029056in}{0.928254in}%
\pgfsys@useobject{currentmarker}{}%
\end{pgfscope}%
\begin{pgfscope}%
\pgfsys@transformshift{1.353377in}{1.402679in}%
\pgfsys@useobject{currentmarker}{}%
\end{pgfscope}%
\begin{pgfscope}%
\pgfsys@transformshift{0.844328in}{0.948487in}%
\pgfsys@useobject{currentmarker}{}%
\end{pgfscope}%
\begin{pgfscope}%
\pgfsys@transformshift{0.897277in}{1.077299in}%
\pgfsys@useobject{currentmarker}{}%
\end{pgfscope}%
\begin{pgfscope}%
\pgfsys@transformshift{0.995325in}{0.789834in}%
\pgfsys@useobject{currentmarker}{}%
\end{pgfscope}%
\begin{pgfscope}%
\pgfsys@transformshift{1.056327in}{0.734121in}%
\pgfsys@useobject{currentmarker}{}%
\end{pgfscope}%
\begin{pgfscope}%
\pgfsys@transformshift{0.564107in}{0.922818in}%
\pgfsys@useobject{currentmarker}{}%
\end{pgfscope}%
\begin{pgfscope}%
\pgfsys@transformshift{0.656175in}{1.010720in}%
\pgfsys@useobject{currentmarker}{}%
\end{pgfscope}%
\begin{pgfscope}%
\pgfsys@transformshift{1.144433in}{0.741225in}%
\pgfsys@useobject{currentmarker}{}%
\end{pgfscope}%
\begin{pgfscope}%
\pgfsys@transformshift{0.628238in}{0.878408in}%
\pgfsys@useobject{currentmarker}{}%
\end{pgfscope}%
\begin{pgfscope}%
\pgfsys@transformshift{1.040257in}{0.766386in}%
\pgfsys@useobject{currentmarker}{}%
\end{pgfscope}%
\begin{pgfscope}%
\pgfsys@transformshift{0.967406in}{1.018342in}%
\pgfsys@useobject{currentmarker}{}%
\end{pgfscope}%
\begin{pgfscope}%
\pgfsys@transformshift{1.329216in}{0.964531in}%
\pgfsys@useobject{currentmarker}{}%
\end{pgfscope}%
\begin{pgfscope}%
\pgfsys@transformshift{0.564607in}{0.946100in}%
\pgfsys@useobject{currentmarker}{}%
\end{pgfscope}%
\begin{pgfscope}%
\pgfsys@transformshift{0.742263in}{0.933242in}%
\pgfsys@useobject{currentmarker}{}%
\end{pgfscope}%
\begin{pgfscope}%
\pgfsys@transformshift{0.629367in}{0.868190in}%
\pgfsys@useobject{currentmarker}{}%
\end{pgfscope}%
\begin{pgfscope}%
\pgfsys@transformshift{0.569846in}{0.720405in}%
\pgfsys@useobject{currentmarker}{}%
\end{pgfscope}%
\begin{pgfscope}%
\pgfsys@transformshift{1.470660in}{0.870021in}%
\pgfsys@useobject{currentmarker}{}%
\end{pgfscope}%
\begin{pgfscope}%
\pgfsys@transformshift{1.976543in}{0.856633in}%
\pgfsys@useobject{currentmarker}{}%
\end{pgfscope}%
\begin{pgfscope}%
\pgfsys@transformshift{1.014486in}{1.317664in}%
\pgfsys@useobject{currentmarker}{}%
\end{pgfscope}%
\begin{pgfscope}%
\pgfsys@transformshift{1.379055in}{1.160727in}%
\pgfsys@useobject{currentmarker}{}%
\end{pgfscope}%
\begin{pgfscope}%
\pgfsys@transformshift{0.658063in}{1.223129in}%
\pgfsys@useobject{currentmarker}{}%
\end{pgfscope}%
\begin{pgfscope}%
\pgfsys@transformshift{0.965129in}{1.332284in}%
\pgfsys@useobject{currentmarker}{}%
\end{pgfscope}%
\begin{pgfscope}%
\pgfsys@transformshift{0.714733in}{0.862536in}%
\pgfsys@useobject{currentmarker}{}%
\end{pgfscope}%
\begin{pgfscope}%
\pgfsys@transformshift{0.734580in}{1.477830in}%
\pgfsys@useobject{currentmarker}{}%
\end{pgfscope}%
\begin{pgfscope}%
\pgfsys@transformshift{1.146876in}{0.809101in}%
\pgfsys@useobject{currentmarker}{}%
\end{pgfscope}%
\begin{pgfscope}%
\pgfsys@transformshift{1.112552in}{1.012287in}%
\pgfsys@useobject{currentmarker}{}%
\end{pgfscope}%
\begin{pgfscope}%
\pgfsys@transformshift{0.609724in}{0.891027in}%
\pgfsys@useobject{currentmarker}{}%
\end{pgfscope}%
\begin{pgfscope}%
\pgfsys@transformshift{0.713974in}{0.842645in}%
\pgfsys@useobject{currentmarker}{}%
\end{pgfscope}%
\begin{pgfscope}%
\pgfsys@transformshift{0.625054in}{0.690728in}%
\pgfsys@useobject{currentmarker}{}%
\end{pgfscope}%
\begin{pgfscope}%
\pgfsys@transformshift{0.695701in}{0.989126in}%
\pgfsys@useobject{currentmarker}{}%
\end{pgfscope}%
\begin{pgfscope}%
\pgfsys@transformshift{0.630052in}{0.568900in}%
\pgfsys@useobject{currentmarker}{}%
\end{pgfscope}%
\begin{pgfscope}%
\pgfsys@transformshift{1.032148in}{0.849199in}%
\pgfsys@useobject{currentmarker}{}%
\end{pgfscope}%
\begin{pgfscope}%
\pgfsys@transformshift{0.974293in}{1.457863in}%
\pgfsys@useobject{currentmarker}{}%
\end{pgfscope}%
\begin{pgfscope}%
\pgfsys@transformshift{0.679965in}{0.869295in}%
\pgfsys@useobject{currentmarker}{}%
\end{pgfscope}%
\begin{pgfscope}%
\pgfsys@transformshift{1.437521in}{0.987209in}%
\pgfsys@useobject{currentmarker}{}%
\end{pgfscope}%
\begin{pgfscope}%
\pgfsys@transformshift{0.779957in}{0.846822in}%
\pgfsys@useobject{currentmarker}{}%
\end{pgfscope}%
\begin{pgfscope}%
\pgfsys@transformshift{1.260216in}{1.052252in}%
\pgfsys@useobject{currentmarker}{}%
\end{pgfscope}%
\begin{pgfscope}%
\pgfsys@transformshift{0.783159in}{0.828782in}%
\pgfsys@useobject{currentmarker}{}%
\end{pgfscope}%
\begin{pgfscope}%
\pgfsys@transformshift{0.627534in}{1.078636in}%
\pgfsys@useobject{currentmarker}{}%
\end{pgfscope}%
\begin{pgfscope}%
\pgfsys@transformshift{0.732803in}{0.680951in}%
\pgfsys@useobject{currentmarker}{}%
\end{pgfscope}%
\begin{pgfscope}%
\pgfsys@transformshift{1.221153in}{1.005633in}%
\pgfsys@useobject{currentmarker}{}%
\end{pgfscope}%
\begin{pgfscope}%
\pgfsys@transformshift{0.997343in}{0.819348in}%
\pgfsys@useobject{currentmarker}{}%
\end{pgfscope}%
\begin{pgfscope}%
\pgfsys@transformshift{0.699904in}{1.029985in}%
\pgfsys@useobject{currentmarker}{}%
\end{pgfscope}%
\begin{pgfscope}%
\pgfsys@transformshift{0.739856in}{0.840190in}%
\pgfsys@useobject{currentmarker}{}%
\end{pgfscope}%
\begin{pgfscope}%
\pgfsys@transformshift{0.561941in}{1.385032in}%
\pgfsys@useobject{currentmarker}{}%
\end{pgfscope}%
\begin{pgfscope}%
\pgfsys@transformshift{1.009173in}{0.954583in}%
\pgfsys@useobject{currentmarker}{}%
\end{pgfscope}%
\begin{pgfscope}%
\pgfsys@transformshift{0.728952in}{0.760277in}%
\pgfsys@useobject{currentmarker}{}%
\end{pgfscope}%
\begin{pgfscope}%
\pgfsys@transformshift{1.463440in}{1.360211in}%
\pgfsys@useobject{currentmarker}{}%
\end{pgfscope}%
\begin{pgfscope}%
\pgfsys@transformshift{0.638106in}{0.723330in}%
\pgfsys@useobject{currentmarker}{}%
\end{pgfscope}%
\begin{pgfscope}%
\pgfsys@transformshift{0.986864in}{1.455846in}%
\pgfsys@useobject{currentmarker}{}%
\end{pgfscope}%
\end{pgfscope}%
\begin{pgfscope}%
\pgfpathrectangle{\pgfqpoint{0.526284in}{0.473557in}}{\pgfqpoint{1.651927in}{1.704653in}}%
\pgfusepath{clip}%
\pgfsetbuttcap%
\pgfsetroundjoin%
\definecolor{currentfill}{rgb}{0.580392,0.403922,0.741176}%
\pgfsetfillcolor{currentfill}%
\pgfsetfillopacity{0.150000}%
\pgfsetlinewidth{0.000000pt}%
\definecolor{currentstroke}{rgb}{0.000000,0.000000,0.000000}%
\pgfsetstrokecolor{currentstroke}%
\pgfsetdash{}{0pt}%
\pgfpathmoveto{\pgfqpoint{0.526284in}{0.948954in}}%
\pgfpathlineto{\pgfqpoint{0.526284in}{0.879703in}}%
\pgfpathlineto{\pgfqpoint{0.542970in}{0.884155in}}%
\pgfpathlineto{\pgfqpoint{0.559656in}{0.888800in}}%
\pgfpathlineto{\pgfqpoint{0.576342in}{0.893513in}}%
\pgfpathlineto{\pgfqpoint{0.593028in}{0.898272in}}%
\pgfpathlineto{\pgfqpoint{0.609715in}{0.902522in}}%
\pgfpathlineto{\pgfqpoint{0.626401in}{0.906726in}}%
\pgfpathlineto{\pgfqpoint{0.643087in}{0.910963in}}%
\pgfpathlineto{\pgfqpoint{0.659773in}{0.915201in}}%
\pgfpathlineto{\pgfqpoint{0.676459in}{0.919447in}}%
\pgfpathlineto{\pgfqpoint{0.693145in}{0.923878in}}%
\pgfpathlineto{\pgfqpoint{0.709831in}{0.928317in}}%
\pgfpathlineto{\pgfqpoint{0.726517in}{0.932557in}}%
\pgfpathlineto{\pgfqpoint{0.743204in}{0.936708in}}%
\pgfpathlineto{\pgfqpoint{0.759890in}{0.940794in}}%
\pgfpathlineto{\pgfqpoint{0.776576in}{0.944984in}}%
\pgfpathlineto{\pgfqpoint{0.793262in}{0.948534in}}%
\pgfpathlineto{\pgfqpoint{0.809948in}{0.952653in}}%
\pgfpathlineto{\pgfqpoint{0.826634in}{0.956252in}}%
\pgfpathlineto{\pgfqpoint{0.843320in}{0.959873in}}%
\pgfpathlineto{\pgfqpoint{0.860006in}{0.963370in}}%
\pgfpathlineto{\pgfqpoint{0.876693in}{0.966836in}}%
\pgfpathlineto{\pgfqpoint{0.893379in}{0.970063in}}%
\pgfpathlineto{\pgfqpoint{0.910065in}{0.973043in}}%
\pgfpathlineto{\pgfqpoint{0.926751in}{0.975632in}}%
\pgfpathlineto{\pgfqpoint{0.943437in}{0.978685in}}%
\pgfpathlineto{\pgfqpoint{0.960123in}{0.981941in}}%
\pgfpathlineto{\pgfqpoint{0.976809in}{0.985152in}}%
\pgfpathlineto{\pgfqpoint{0.993495in}{0.988067in}}%
\pgfpathlineto{\pgfqpoint{1.010182in}{0.991160in}}%
\pgfpathlineto{\pgfqpoint{1.026868in}{0.994050in}}%
\pgfpathlineto{\pgfqpoint{1.043554in}{0.997077in}}%
\pgfpathlineto{\pgfqpoint{1.060240in}{1.000382in}}%
\pgfpathlineto{\pgfqpoint{1.076926in}{1.003008in}}%
\pgfpathlineto{\pgfqpoint{1.093612in}{1.005606in}}%
\pgfpathlineto{\pgfqpoint{1.110298in}{1.008178in}}%
\pgfpathlineto{\pgfqpoint{1.126985in}{1.010779in}}%
\pgfpathlineto{\pgfqpoint{1.143671in}{1.013041in}}%
\pgfpathlineto{\pgfqpoint{1.160357in}{1.015706in}}%
\pgfpathlineto{\pgfqpoint{1.177043in}{1.018148in}}%
\pgfpathlineto{\pgfqpoint{1.193729in}{1.020933in}}%
\pgfpathlineto{\pgfqpoint{1.210415in}{1.023407in}}%
\pgfpathlineto{\pgfqpoint{1.227101in}{1.026073in}}%
\pgfpathlineto{\pgfqpoint{1.243787in}{1.028687in}}%
\pgfpathlineto{\pgfqpoint{1.260474in}{1.030946in}}%
\pgfpathlineto{\pgfqpoint{1.277160in}{1.033328in}}%
\pgfpathlineto{\pgfqpoint{1.293846in}{1.035654in}}%
\pgfpathlineto{\pgfqpoint{1.310532in}{1.037965in}}%
\pgfpathlineto{\pgfqpoint{1.327218in}{1.040002in}}%
\pgfpathlineto{\pgfqpoint{1.343904in}{1.042184in}}%
\pgfpathlineto{\pgfqpoint{1.360590in}{1.044434in}}%
\pgfpathlineto{\pgfqpoint{1.377276in}{1.046532in}}%
\pgfpathlineto{\pgfqpoint{1.393963in}{1.048908in}}%
\pgfpathlineto{\pgfqpoint{1.410649in}{1.051507in}}%
\pgfpathlineto{\pgfqpoint{1.427335in}{1.053945in}}%
\pgfpathlineto{\pgfqpoint{1.444021in}{1.056067in}}%
\pgfpathlineto{\pgfqpoint{1.460707in}{1.058319in}}%
\pgfpathlineto{\pgfqpoint{1.477393in}{1.060470in}}%
\pgfpathlineto{\pgfqpoint{1.494079in}{1.062641in}}%
\pgfpathlineto{\pgfqpoint{1.510765in}{1.064921in}}%
\pgfpathlineto{\pgfqpoint{1.527452in}{1.067221in}}%
\pgfpathlineto{\pgfqpoint{1.544138in}{1.069522in}}%
\pgfpathlineto{\pgfqpoint{1.560824in}{1.071822in}}%
\pgfpathlineto{\pgfqpoint{1.577510in}{1.074123in}}%
\pgfpathlineto{\pgfqpoint{1.594196in}{1.076423in}}%
\pgfpathlineto{\pgfqpoint{1.610882in}{1.078695in}}%
\pgfpathlineto{\pgfqpoint{1.627568in}{1.080828in}}%
\pgfpathlineto{\pgfqpoint{1.644255in}{1.083092in}}%
\pgfpathlineto{\pgfqpoint{1.660941in}{1.085361in}}%
\pgfpathlineto{\pgfqpoint{1.677627in}{1.087630in}}%
\pgfpathlineto{\pgfqpoint{1.694313in}{1.089903in}}%
\pgfpathlineto{\pgfqpoint{1.710999in}{1.092182in}}%
\pgfpathlineto{\pgfqpoint{1.727685in}{1.094601in}}%
\pgfpathlineto{\pgfqpoint{1.744371in}{1.097128in}}%
\pgfpathlineto{\pgfqpoint{1.761057in}{1.099428in}}%
\pgfpathlineto{\pgfqpoint{1.777744in}{1.101674in}}%
\pgfpathlineto{\pgfqpoint{1.794430in}{1.103864in}}%
\pgfpathlineto{\pgfqpoint{1.811116in}{1.106055in}}%
\pgfpathlineto{\pgfqpoint{1.827802in}{1.108246in}}%
\pgfpathlineto{\pgfqpoint{1.844488in}{1.110436in}}%
\pgfpathlineto{\pgfqpoint{1.861174in}{1.112639in}}%
\pgfpathlineto{\pgfqpoint{1.877860in}{1.114909in}}%
\pgfpathlineto{\pgfqpoint{1.894546in}{1.117180in}}%
\pgfpathlineto{\pgfqpoint{1.911233in}{1.119453in}}%
\pgfpathlineto{\pgfqpoint{1.927919in}{1.121723in}}%
\pgfpathlineto{\pgfqpoint{1.944605in}{1.123991in}}%
\pgfpathlineto{\pgfqpoint{1.961291in}{1.126262in}}%
\pgfpathlineto{\pgfqpoint{1.977977in}{1.128532in}}%
\pgfpathlineto{\pgfqpoint{1.994663in}{1.130803in}}%
\pgfpathlineto{\pgfqpoint{2.011349in}{1.133073in}}%
\pgfpathlineto{\pgfqpoint{2.028035in}{1.135344in}}%
\pgfpathlineto{\pgfqpoint{2.044722in}{1.137615in}}%
\pgfpathlineto{\pgfqpoint{2.061408in}{1.139885in}}%
\pgfpathlineto{\pgfqpoint{2.078094in}{1.142156in}}%
\pgfpathlineto{\pgfqpoint{2.094780in}{1.144426in}}%
\pgfpathlineto{\pgfqpoint{2.111466in}{1.146702in}}%
\pgfpathlineto{\pgfqpoint{2.128152in}{1.148779in}}%
\pgfpathlineto{\pgfqpoint{2.144838in}{1.150841in}}%
\pgfpathlineto{\pgfqpoint{2.161525in}{1.152903in}}%
\pgfpathlineto{\pgfqpoint{2.178211in}{1.155255in}}%
\pgfpathlineto{\pgfqpoint{2.178211in}{1.349158in}}%
\pgfpathlineto{\pgfqpoint{2.178211in}{1.349158in}}%
\pgfpathlineto{\pgfqpoint{2.161525in}{1.344505in}}%
\pgfpathlineto{\pgfqpoint{2.144838in}{1.339853in}}%
\pgfpathlineto{\pgfqpoint{2.128152in}{1.335200in}}%
\pgfpathlineto{\pgfqpoint{2.111466in}{1.330544in}}%
\pgfpathlineto{\pgfqpoint{2.094780in}{1.325884in}}%
\pgfpathlineto{\pgfqpoint{2.078094in}{1.321225in}}%
\pgfpathlineto{\pgfqpoint{2.061408in}{1.316565in}}%
\pgfpathlineto{\pgfqpoint{2.044722in}{1.311905in}}%
\pgfpathlineto{\pgfqpoint{2.028035in}{1.307246in}}%
\pgfpathlineto{\pgfqpoint{2.011349in}{1.302586in}}%
\pgfpathlineto{\pgfqpoint{1.994663in}{1.297927in}}%
\pgfpathlineto{\pgfqpoint{1.977977in}{1.293267in}}%
\pgfpathlineto{\pgfqpoint{1.961291in}{1.288613in}}%
\pgfpathlineto{\pgfqpoint{1.944605in}{1.283962in}}%
\pgfpathlineto{\pgfqpoint{1.927919in}{1.279312in}}%
\pgfpathlineto{\pgfqpoint{1.911233in}{1.274655in}}%
\pgfpathlineto{\pgfqpoint{1.894546in}{1.269996in}}%
\pgfpathlineto{\pgfqpoint{1.877860in}{1.265337in}}%
\pgfpathlineto{\pgfqpoint{1.861174in}{1.260678in}}%
\pgfpathlineto{\pgfqpoint{1.844488in}{1.256019in}}%
\pgfpathlineto{\pgfqpoint{1.827802in}{1.251360in}}%
\pgfpathlineto{\pgfqpoint{1.811116in}{1.246833in}}%
\pgfpathlineto{\pgfqpoint{1.794430in}{1.242154in}}%
\pgfpathlineto{\pgfqpoint{1.777744in}{1.237427in}}%
\pgfpathlineto{\pgfqpoint{1.761057in}{1.232743in}}%
\pgfpathlineto{\pgfqpoint{1.744371in}{1.228090in}}%
\pgfpathlineto{\pgfqpoint{1.727685in}{1.223438in}}%
\pgfpathlineto{\pgfqpoint{1.710999in}{1.218782in}}%
\pgfpathlineto{\pgfqpoint{1.694313in}{1.214280in}}%
\pgfpathlineto{\pgfqpoint{1.677627in}{1.209782in}}%
\pgfpathlineto{\pgfqpoint{1.660941in}{1.205285in}}%
\pgfpathlineto{\pgfqpoint{1.644255in}{1.200786in}}%
\pgfpathlineto{\pgfqpoint{1.627568in}{1.196285in}}%
\pgfpathlineto{\pgfqpoint{1.610882in}{1.191603in}}%
\pgfpathlineto{\pgfqpoint{1.594196in}{1.186854in}}%
\pgfpathlineto{\pgfqpoint{1.577510in}{1.182615in}}%
\pgfpathlineto{\pgfqpoint{1.560824in}{1.177958in}}%
\pgfpathlineto{\pgfqpoint{1.544138in}{1.173346in}}%
\pgfpathlineto{\pgfqpoint{1.527452in}{1.168649in}}%
\pgfpathlineto{\pgfqpoint{1.510765in}{1.163997in}}%
\pgfpathlineto{\pgfqpoint{1.494079in}{1.159346in}}%
\pgfpathlineto{\pgfqpoint{1.477393in}{1.154687in}}%
\pgfpathlineto{\pgfqpoint{1.460707in}{1.150024in}}%
\pgfpathlineto{\pgfqpoint{1.444021in}{1.145492in}}%
\pgfpathlineto{\pgfqpoint{1.427335in}{1.140725in}}%
\pgfpathlineto{\pgfqpoint{1.410649in}{1.136082in}}%
\pgfpathlineto{\pgfqpoint{1.393963in}{1.131424in}}%
\pgfpathlineto{\pgfqpoint{1.377276in}{1.127327in}}%
\pgfpathlineto{\pgfqpoint{1.360590in}{1.123351in}}%
\pgfpathlineto{\pgfqpoint{1.343904in}{1.118744in}}%
\pgfpathlineto{\pgfqpoint{1.327218in}{1.114046in}}%
\pgfpathlineto{\pgfqpoint{1.310532in}{1.109355in}}%
\pgfpathlineto{\pgfqpoint{1.293846in}{1.104835in}}%
\pgfpathlineto{\pgfqpoint{1.277160in}{1.100640in}}%
\pgfpathlineto{\pgfqpoint{1.260474in}{1.096427in}}%
\pgfpathlineto{\pgfqpoint{1.243787in}{1.091984in}}%
\pgfpathlineto{\pgfqpoint{1.227101in}{1.087553in}}%
\pgfpathlineto{\pgfqpoint{1.210415in}{1.083823in}}%
\pgfpathlineto{\pgfqpoint{1.193729in}{1.079450in}}%
\pgfpathlineto{\pgfqpoint{1.177043in}{1.075353in}}%
\pgfpathlineto{\pgfqpoint{1.160357in}{1.071195in}}%
\pgfpathlineto{\pgfqpoint{1.143671in}{1.067040in}}%
\pgfpathlineto{\pgfqpoint{1.126985in}{1.062640in}}%
\pgfpathlineto{\pgfqpoint{1.110298in}{1.058439in}}%
\pgfpathlineto{\pgfqpoint{1.093612in}{1.054634in}}%
\pgfpathlineto{\pgfqpoint{1.076926in}{1.050398in}}%
\pgfpathlineto{\pgfqpoint{1.060240in}{1.046060in}}%
\pgfpathlineto{\pgfqpoint{1.043554in}{1.042173in}}%
\pgfpathlineto{\pgfqpoint{1.026868in}{1.038919in}}%
\pgfpathlineto{\pgfqpoint{1.010182in}{1.034350in}}%
\pgfpathlineto{\pgfqpoint{0.993495in}{1.030890in}}%
\pgfpathlineto{\pgfqpoint{0.976809in}{1.027225in}}%
\pgfpathlineto{\pgfqpoint{0.960123in}{1.023563in}}%
\pgfpathlineto{\pgfqpoint{0.943437in}{1.020414in}}%
\pgfpathlineto{\pgfqpoint{0.926751in}{1.016713in}}%
\pgfpathlineto{\pgfqpoint{0.910065in}{1.013389in}}%
\pgfpathlineto{\pgfqpoint{0.893379in}{1.010096in}}%
\pgfpathlineto{\pgfqpoint{0.876693in}{1.006936in}}%
\pgfpathlineto{\pgfqpoint{0.860006in}{1.003851in}}%
\pgfpathlineto{\pgfqpoint{0.843320in}{1.000695in}}%
\pgfpathlineto{\pgfqpoint{0.826634in}{0.997919in}}%
\pgfpathlineto{\pgfqpoint{0.809948in}{0.994985in}}%
\pgfpathlineto{\pgfqpoint{0.793262in}{0.991969in}}%
\pgfpathlineto{\pgfqpoint{0.776576in}{0.988747in}}%
\pgfpathlineto{\pgfqpoint{0.759890in}{0.985765in}}%
\pgfpathlineto{\pgfqpoint{0.743204in}{0.982691in}}%
\pgfpathlineto{\pgfqpoint{0.726517in}{0.979622in}}%
\pgfpathlineto{\pgfqpoint{0.709831in}{0.976909in}}%
\pgfpathlineto{\pgfqpoint{0.693145in}{0.974466in}}%
\pgfpathlineto{\pgfqpoint{0.676459in}{0.971525in}}%
\pgfpathlineto{\pgfqpoint{0.659773in}{0.969295in}}%
\pgfpathlineto{\pgfqpoint{0.643087in}{0.966362in}}%
\pgfpathlineto{\pgfqpoint{0.626401in}{0.964120in}}%
\pgfpathlineto{\pgfqpoint{0.609715in}{0.961517in}}%
\pgfpathlineto{\pgfqpoint{0.593028in}{0.958905in}}%
\pgfpathlineto{\pgfqpoint{0.576342in}{0.956291in}}%
\pgfpathlineto{\pgfqpoint{0.559656in}{0.953702in}}%
\pgfpathlineto{\pgfqpoint{0.542970in}{0.951321in}}%
\pgfpathlineto{\pgfqpoint{0.526284in}{0.948954in}}%
\pgfpathclose%
\pgfusepath{fill}%
\end{pgfscope}%
\begin{pgfscope}%
\pgfsetbuttcap%
\pgfsetroundjoin%
\definecolor{currentfill}{rgb}{0.000000,0.000000,0.000000}%
\pgfsetfillcolor{currentfill}%
\pgfsetlinewidth{0.803000pt}%
\definecolor{currentstroke}{rgb}{0.000000,0.000000,0.000000}%
\pgfsetstrokecolor{currentstroke}%
\pgfsetdash{}{0pt}%
\pgfsys@defobject{currentmarker}{\pgfqpoint{0.000000in}{-0.048611in}}{\pgfqpoint{0.000000in}{0.000000in}}{%
\pgfpathmoveto{\pgfqpoint{0.000000in}{0.000000in}}%
\pgfpathlineto{\pgfqpoint{0.000000in}{-0.048611in}}%
\pgfusepath{stroke,fill}%
}%
\begin{pgfscope}%
\pgfsys@transformshift{0.880281in}{0.473557in}%
\pgfsys@useobject{currentmarker}{}%
\end{pgfscope}%
\end{pgfscope}%
\begin{pgfscope}%
\definecolor{textcolor}{rgb}{0.000000,0.000000,0.000000}%
\pgfsetstrokecolor{textcolor}%
\pgfsetfillcolor{textcolor}%
\pgftext[x=0.880281in,y=0.376335in,,top]{\color{textcolor}\rmfamily\fontsize{8.000000}{9.600000}\selectfont \(\displaystyle {25000}\)}%
\end{pgfscope}%
\begin{pgfscope}%
\pgfsetbuttcap%
\pgfsetroundjoin%
\definecolor{currentfill}{rgb}{0.000000,0.000000,0.000000}%
\pgfsetfillcolor{currentfill}%
\pgfsetlinewidth{0.803000pt}%
\definecolor{currentstroke}{rgb}{0.000000,0.000000,0.000000}%
\pgfsetstrokecolor{currentstroke}%
\pgfsetdash{}{0pt}%
\pgfsys@defobject{currentmarker}{\pgfqpoint{0.000000in}{-0.048611in}}{\pgfqpoint{0.000000in}{0.000000in}}{%
\pgfpathmoveto{\pgfqpoint{0.000000in}{0.000000in}}%
\pgfpathlineto{\pgfqpoint{0.000000in}{-0.048611in}}%
\pgfusepath{stroke,fill}%
}%
\begin{pgfscope}%
\pgfsys@transformshift{1.343120in}{0.473557in}%
\pgfsys@useobject{currentmarker}{}%
\end{pgfscope}%
\end{pgfscope}%
\begin{pgfscope}%
\definecolor{textcolor}{rgb}{0.000000,0.000000,0.000000}%
\pgfsetstrokecolor{textcolor}%
\pgfsetfillcolor{textcolor}%
\pgftext[x=1.343120in,y=0.376335in,,top]{\color{textcolor}\rmfamily\fontsize{8.000000}{9.600000}\selectfont \(\displaystyle {50000}\)}%
\end{pgfscope}%
\begin{pgfscope}%
\pgfsetbuttcap%
\pgfsetroundjoin%
\definecolor{currentfill}{rgb}{0.000000,0.000000,0.000000}%
\pgfsetfillcolor{currentfill}%
\pgfsetlinewidth{0.803000pt}%
\definecolor{currentstroke}{rgb}{0.000000,0.000000,0.000000}%
\pgfsetstrokecolor{currentstroke}%
\pgfsetdash{}{0pt}%
\pgfsys@defobject{currentmarker}{\pgfqpoint{0.000000in}{-0.048611in}}{\pgfqpoint{0.000000in}{0.000000in}}{%
\pgfpathmoveto{\pgfqpoint{0.000000in}{0.000000in}}%
\pgfpathlineto{\pgfqpoint{0.000000in}{-0.048611in}}%
\pgfusepath{stroke,fill}%
}%
\begin{pgfscope}%
\pgfsys@transformshift{1.805959in}{0.473557in}%
\pgfsys@useobject{currentmarker}{}%
\end{pgfscope}%
\end{pgfscope}%
\begin{pgfscope}%
\definecolor{textcolor}{rgb}{0.000000,0.000000,0.000000}%
\pgfsetstrokecolor{textcolor}%
\pgfsetfillcolor{textcolor}%
\pgftext[x=1.805959in,y=0.376335in,,top]{\color{textcolor}\rmfamily\fontsize{8.000000}{9.600000}\selectfont \(\displaystyle {75000}\)}%
\end{pgfscope}%
\begin{pgfscope}%
\definecolor{textcolor}{rgb}{0.000000,0.000000,0.000000}%
\pgfsetstrokecolor{textcolor}%
\pgfsetfillcolor{textcolor}%
\pgftext[x=1.352247in,y=0.222655in,,top]{\color{textcolor}\rmfamily\fontsize{10.000000}{12.000000}\selectfont Number of nodes}%
\end{pgfscope}%
\begin{pgfscope}%
\pgfsetbuttcap%
\pgfsetroundjoin%
\definecolor{currentfill}{rgb}{0.000000,0.000000,0.000000}%
\pgfsetfillcolor{currentfill}%
\pgfsetlinewidth{0.803000pt}%
\definecolor{currentstroke}{rgb}{0.000000,0.000000,0.000000}%
\pgfsetstrokecolor{currentstroke}%
\pgfsetdash{}{0pt}%
\pgfsys@defobject{currentmarker}{\pgfqpoint{-0.048611in}{0.000000in}}{\pgfqpoint{-0.000000in}{0.000000in}}{%
\pgfpathmoveto{\pgfqpoint{-0.000000in}{0.000000in}}%
\pgfpathlineto{\pgfqpoint{-0.048611in}{0.000000in}}%
\pgfusepath{stroke,fill}%
}%
\begin{pgfscope}%
\pgfsys@transformshift{0.526284in}{0.766314in}%
\pgfsys@useobject{currentmarker}{}%
\end{pgfscope}%
\end{pgfscope}%
\begin{pgfscope}%
\definecolor{textcolor}{rgb}{0.000000,0.000000,0.000000}%
\pgfsetstrokecolor{textcolor}%
\pgfsetfillcolor{textcolor}%
\pgftext[x=0.278211in, y=0.728051in, left, base]{\color{textcolor}\rmfamily\fontsize{8.000000}{9.600000}\selectfont \(\displaystyle {0.2}\)}%
\end{pgfscope}%
\begin{pgfscope}%
\pgfsetbuttcap%
\pgfsetroundjoin%
\definecolor{currentfill}{rgb}{0.000000,0.000000,0.000000}%
\pgfsetfillcolor{currentfill}%
\pgfsetlinewidth{0.803000pt}%
\definecolor{currentstroke}{rgb}{0.000000,0.000000,0.000000}%
\pgfsetstrokecolor{currentstroke}%
\pgfsetdash{}{0pt}%
\pgfsys@defobject{currentmarker}{\pgfqpoint{-0.048611in}{0.000000in}}{\pgfqpoint{-0.000000in}{0.000000in}}{%
\pgfpathmoveto{\pgfqpoint{-0.000000in}{0.000000in}}%
\pgfpathlineto{\pgfqpoint{-0.048611in}{0.000000in}}%
\pgfusepath{stroke,fill}%
}%
\begin{pgfscope}%
\pgfsys@transformshift{0.526284in}{1.107351in}%
\pgfsys@useobject{currentmarker}{}%
\end{pgfscope}%
\end{pgfscope}%
\begin{pgfscope}%
\definecolor{textcolor}{rgb}{0.000000,0.000000,0.000000}%
\pgfsetstrokecolor{textcolor}%
\pgfsetfillcolor{textcolor}%
\pgftext[x=0.278211in, y=1.069089in, left, base]{\color{textcolor}\rmfamily\fontsize{8.000000}{9.600000}\selectfont \(\displaystyle {0.4}\)}%
\end{pgfscope}%
\begin{pgfscope}%
\pgfsetbuttcap%
\pgfsetroundjoin%
\definecolor{currentfill}{rgb}{0.000000,0.000000,0.000000}%
\pgfsetfillcolor{currentfill}%
\pgfsetlinewidth{0.803000pt}%
\definecolor{currentstroke}{rgb}{0.000000,0.000000,0.000000}%
\pgfsetstrokecolor{currentstroke}%
\pgfsetdash{}{0pt}%
\pgfsys@defobject{currentmarker}{\pgfqpoint{-0.048611in}{0.000000in}}{\pgfqpoint{-0.000000in}{0.000000in}}{%
\pgfpathmoveto{\pgfqpoint{-0.000000in}{0.000000in}}%
\pgfpathlineto{\pgfqpoint{-0.048611in}{0.000000in}}%
\pgfusepath{stroke,fill}%
}%
\begin{pgfscope}%
\pgfsys@transformshift{0.526284in}{1.448388in}%
\pgfsys@useobject{currentmarker}{}%
\end{pgfscope}%
\end{pgfscope}%
\begin{pgfscope}%
\definecolor{textcolor}{rgb}{0.000000,0.000000,0.000000}%
\pgfsetstrokecolor{textcolor}%
\pgfsetfillcolor{textcolor}%
\pgftext[x=0.278211in, y=1.410126in, left, base]{\color{textcolor}\rmfamily\fontsize{8.000000}{9.600000}\selectfont \(\displaystyle {0.6}\)}%
\end{pgfscope}%
\begin{pgfscope}%
\pgfsetbuttcap%
\pgfsetroundjoin%
\definecolor{currentfill}{rgb}{0.000000,0.000000,0.000000}%
\pgfsetfillcolor{currentfill}%
\pgfsetlinewidth{0.803000pt}%
\definecolor{currentstroke}{rgb}{0.000000,0.000000,0.000000}%
\pgfsetstrokecolor{currentstroke}%
\pgfsetdash{}{0pt}%
\pgfsys@defobject{currentmarker}{\pgfqpoint{-0.048611in}{0.000000in}}{\pgfqpoint{-0.000000in}{0.000000in}}{%
\pgfpathmoveto{\pgfqpoint{-0.000000in}{0.000000in}}%
\pgfpathlineto{\pgfqpoint{-0.048611in}{0.000000in}}%
\pgfusepath{stroke,fill}%
}%
\begin{pgfscope}%
\pgfsys@transformshift{0.526284in}{1.789425in}%
\pgfsys@useobject{currentmarker}{}%
\end{pgfscope}%
\end{pgfscope}%
\begin{pgfscope}%
\definecolor{textcolor}{rgb}{0.000000,0.000000,0.000000}%
\pgfsetstrokecolor{textcolor}%
\pgfsetfillcolor{textcolor}%
\pgftext[x=0.278211in, y=1.751163in, left, base]{\color{textcolor}\rmfamily\fontsize{8.000000}{9.600000}\selectfont \(\displaystyle {0.8}\)}%
\end{pgfscope}%
\begin{pgfscope}%
\pgfsetbuttcap%
\pgfsetroundjoin%
\definecolor{currentfill}{rgb}{0.000000,0.000000,0.000000}%
\pgfsetfillcolor{currentfill}%
\pgfsetlinewidth{0.803000pt}%
\definecolor{currentstroke}{rgb}{0.000000,0.000000,0.000000}%
\pgfsetstrokecolor{currentstroke}%
\pgfsetdash{}{0pt}%
\pgfsys@defobject{currentmarker}{\pgfqpoint{-0.048611in}{0.000000in}}{\pgfqpoint{-0.000000in}{0.000000in}}{%
\pgfpathmoveto{\pgfqpoint{-0.000000in}{0.000000in}}%
\pgfpathlineto{\pgfqpoint{-0.048611in}{0.000000in}}%
\pgfusepath{stroke,fill}%
}%
\begin{pgfscope}%
\pgfsys@transformshift{0.526284in}{2.130462in}%
\pgfsys@useobject{currentmarker}{}%
\end{pgfscope}%
\end{pgfscope}%
\begin{pgfscope}%
\definecolor{textcolor}{rgb}{0.000000,0.000000,0.000000}%
\pgfsetstrokecolor{textcolor}%
\pgfsetfillcolor{textcolor}%
\pgftext[x=0.278211in, y=2.092200in, left, base]{\color{textcolor}\rmfamily\fontsize{8.000000}{9.600000}\selectfont \(\displaystyle {1.0}\)}%
\end{pgfscope}%
\begin{pgfscope}%
\definecolor{textcolor}{rgb}{0.000000,0.000000,0.000000}%
\pgfsetstrokecolor{textcolor}%
\pgfsetfillcolor{textcolor}%
\pgftext[x=0.222655in,y=1.325884in,,bottom,rotate=90.000000]{\color{textcolor}\rmfamily\fontsize{10.000000}{12.000000}\selectfont Accuracy}%
\end{pgfscope}%
\begin{pgfscope}%
\pgfpathrectangle{\pgfqpoint{0.526284in}{0.473557in}}{\pgfqpoint{1.651927in}{1.704653in}}%
\pgfusepath{clip}%
\pgfsetrectcap%
\pgfsetroundjoin%
\pgfsetlinewidth{2.258437pt}%
\definecolor{currentstroke}{rgb}{0.121569,0.466667,0.705882}%
\pgfsetstrokecolor{currentstroke}%
\pgfsetdash{}{0pt}%
\pgfpathmoveto{\pgfqpoint{0.526284in}{1.969458in}}%
\pgfpathlineto{\pgfqpoint{0.542970in}{1.969395in}}%
\pgfpathlineto{\pgfqpoint{0.559656in}{1.969333in}}%
\pgfpathlineto{\pgfqpoint{0.576342in}{1.969271in}}%
\pgfpathlineto{\pgfqpoint{0.593028in}{1.969208in}}%
\pgfpathlineto{\pgfqpoint{0.609715in}{1.969146in}}%
\pgfpathlineto{\pgfqpoint{0.626401in}{1.969084in}}%
\pgfpathlineto{\pgfqpoint{0.643087in}{1.969021in}}%
\pgfpathlineto{\pgfqpoint{0.659773in}{1.968959in}}%
\pgfpathlineto{\pgfqpoint{0.676459in}{1.968897in}}%
\pgfpathlineto{\pgfqpoint{0.693145in}{1.968835in}}%
\pgfpathlineto{\pgfqpoint{0.709831in}{1.968772in}}%
\pgfpathlineto{\pgfqpoint{0.726517in}{1.968710in}}%
\pgfpathlineto{\pgfqpoint{0.743204in}{1.968648in}}%
\pgfpathlineto{\pgfqpoint{0.759890in}{1.968585in}}%
\pgfpathlineto{\pgfqpoint{0.776576in}{1.968523in}}%
\pgfpathlineto{\pgfqpoint{0.793262in}{1.968461in}}%
\pgfpathlineto{\pgfqpoint{0.809948in}{1.968398in}}%
\pgfpathlineto{\pgfqpoint{0.826634in}{1.968336in}}%
\pgfpathlineto{\pgfqpoint{0.843320in}{1.968274in}}%
\pgfpathlineto{\pgfqpoint{0.860006in}{1.968211in}}%
\pgfpathlineto{\pgfqpoint{0.876693in}{1.968149in}}%
\pgfpathlineto{\pgfqpoint{0.893379in}{1.968087in}}%
\pgfpathlineto{\pgfqpoint{0.910065in}{1.968025in}}%
\pgfpathlineto{\pgfqpoint{0.926751in}{1.967962in}}%
\pgfpathlineto{\pgfqpoint{0.943437in}{1.967900in}}%
\pgfpathlineto{\pgfqpoint{0.960123in}{1.967838in}}%
\pgfpathlineto{\pgfqpoint{0.976809in}{1.967775in}}%
\pgfpathlineto{\pgfqpoint{0.993495in}{1.967713in}}%
\pgfpathlineto{\pgfqpoint{1.010182in}{1.967651in}}%
\pgfpathlineto{\pgfqpoint{1.026868in}{1.967588in}}%
\pgfpathlineto{\pgfqpoint{1.043554in}{1.967526in}}%
\pgfpathlineto{\pgfqpoint{1.060240in}{1.967464in}}%
\pgfpathlineto{\pgfqpoint{1.076926in}{1.967401in}}%
\pgfpathlineto{\pgfqpoint{1.093612in}{1.967339in}}%
\pgfpathlineto{\pgfqpoint{1.110298in}{1.967277in}}%
\pgfpathlineto{\pgfqpoint{1.126985in}{1.967215in}}%
\pgfpathlineto{\pgfqpoint{1.143671in}{1.967152in}}%
\pgfpathlineto{\pgfqpoint{1.160357in}{1.967090in}}%
\pgfpathlineto{\pgfqpoint{1.177043in}{1.967028in}}%
\pgfpathlineto{\pgfqpoint{1.193729in}{1.966965in}}%
\pgfpathlineto{\pgfqpoint{1.210415in}{1.966903in}}%
\pgfpathlineto{\pgfqpoint{1.227101in}{1.966841in}}%
\pgfpathlineto{\pgfqpoint{1.243787in}{1.966778in}}%
\pgfpathlineto{\pgfqpoint{1.260474in}{1.966716in}}%
\pgfpathlineto{\pgfqpoint{1.277160in}{1.966654in}}%
\pgfpathlineto{\pgfqpoint{1.293846in}{1.966591in}}%
\pgfpathlineto{\pgfqpoint{1.310532in}{1.966529in}}%
\pgfpathlineto{\pgfqpoint{1.327218in}{1.966467in}}%
\pgfpathlineto{\pgfqpoint{1.343904in}{1.966405in}}%
\pgfpathlineto{\pgfqpoint{1.360590in}{1.966342in}}%
\pgfpathlineto{\pgfqpoint{1.377276in}{1.966280in}}%
\pgfpathlineto{\pgfqpoint{1.393963in}{1.966218in}}%
\pgfpathlineto{\pgfqpoint{1.410649in}{1.966155in}}%
\pgfpathlineto{\pgfqpoint{1.427335in}{1.966093in}}%
\pgfpathlineto{\pgfqpoint{1.444021in}{1.966031in}}%
\pgfpathlineto{\pgfqpoint{1.460707in}{1.965968in}}%
\pgfpathlineto{\pgfqpoint{1.477393in}{1.965906in}}%
\pgfpathlineto{\pgfqpoint{1.494079in}{1.965844in}}%
\pgfpathlineto{\pgfqpoint{1.510765in}{1.965781in}}%
\pgfpathlineto{\pgfqpoint{1.527452in}{1.965719in}}%
\pgfpathlineto{\pgfqpoint{1.544138in}{1.965657in}}%
\pgfpathlineto{\pgfqpoint{1.560824in}{1.965595in}}%
\pgfpathlineto{\pgfqpoint{1.577510in}{1.965532in}}%
\pgfpathlineto{\pgfqpoint{1.594196in}{1.965470in}}%
\pgfpathlineto{\pgfqpoint{1.610882in}{1.965408in}}%
\pgfpathlineto{\pgfqpoint{1.627568in}{1.965345in}}%
\pgfpathlineto{\pgfqpoint{1.644255in}{1.965283in}}%
\pgfpathlineto{\pgfqpoint{1.660941in}{1.965221in}}%
\pgfpathlineto{\pgfqpoint{1.677627in}{1.965158in}}%
\pgfpathlineto{\pgfqpoint{1.694313in}{1.965096in}}%
\pgfpathlineto{\pgfqpoint{1.710999in}{1.965034in}}%
\pgfpathlineto{\pgfqpoint{1.727685in}{1.964971in}}%
\pgfpathlineto{\pgfqpoint{1.744371in}{1.964909in}}%
\pgfpathlineto{\pgfqpoint{1.761057in}{1.964847in}}%
\pgfpathlineto{\pgfqpoint{1.777744in}{1.964785in}}%
\pgfpathlineto{\pgfqpoint{1.794430in}{1.964722in}}%
\pgfpathlineto{\pgfqpoint{1.811116in}{1.964660in}}%
\pgfpathlineto{\pgfqpoint{1.827802in}{1.964598in}}%
\pgfpathlineto{\pgfqpoint{1.844488in}{1.964535in}}%
\pgfpathlineto{\pgfqpoint{1.861174in}{1.964473in}}%
\pgfpathlineto{\pgfqpoint{1.877860in}{1.964411in}}%
\pgfpathlineto{\pgfqpoint{1.894546in}{1.964348in}}%
\pgfpathlineto{\pgfqpoint{1.911233in}{1.964286in}}%
\pgfpathlineto{\pgfqpoint{1.927919in}{1.964224in}}%
\pgfpathlineto{\pgfqpoint{1.944605in}{1.964161in}}%
\pgfpathlineto{\pgfqpoint{1.961291in}{1.964099in}}%
\pgfpathlineto{\pgfqpoint{1.977977in}{1.964037in}}%
\pgfpathlineto{\pgfqpoint{1.994663in}{1.963975in}}%
\pgfpathlineto{\pgfqpoint{2.011349in}{1.963912in}}%
\pgfpathlineto{\pgfqpoint{2.028035in}{1.963850in}}%
\pgfpathlineto{\pgfqpoint{2.044722in}{1.963788in}}%
\pgfpathlineto{\pgfqpoint{2.061408in}{1.963725in}}%
\pgfpathlineto{\pgfqpoint{2.078094in}{1.963663in}}%
\pgfpathlineto{\pgfqpoint{2.094780in}{1.963601in}}%
\pgfpathlineto{\pgfqpoint{2.111466in}{1.963538in}}%
\pgfpathlineto{\pgfqpoint{2.128152in}{1.963476in}}%
\pgfpathlineto{\pgfqpoint{2.144838in}{1.963414in}}%
\pgfpathlineto{\pgfqpoint{2.161525in}{1.963351in}}%
\pgfpathlineto{\pgfqpoint{2.178211in}{1.963289in}}%
\pgfusepath{stroke}%
\end{pgfscope}%
\begin{pgfscope}%
\pgfpathrectangle{\pgfqpoint{0.526284in}{0.473557in}}{\pgfqpoint{1.651927in}{1.704653in}}%
\pgfusepath{clip}%
\pgfsetrectcap%
\pgfsetroundjoin%
\pgfsetlinewidth{2.258437pt}%
\definecolor{currentstroke}{rgb}{1.000000,0.498039,0.054902}%
\pgfsetstrokecolor{currentstroke}%
\pgfsetdash{}{0pt}%
\pgfpathmoveto{\pgfqpoint{0.526284in}{1.804974in}}%
\pgfpathlineto{\pgfqpoint{0.542970in}{1.804031in}}%
\pgfpathlineto{\pgfqpoint{0.559656in}{1.803087in}}%
\pgfpathlineto{\pgfqpoint{0.576342in}{1.802143in}}%
\pgfpathlineto{\pgfqpoint{0.593028in}{1.801199in}}%
\pgfpathlineto{\pgfqpoint{0.609715in}{1.800255in}}%
\pgfpathlineto{\pgfqpoint{0.626401in}{1.799311in}}%
\pgfpathlineto{\pgfqpoint{0.643087in}{1.798367in}}%
\pgfpathlineto{\pgfqpoint{0.659773in}{1.797424in}}%
\pgfpathlineto{\pgfqpoint{0.676459in}{1.796480in}}%
\pgfpathlineto{\pgfqpoint{0.693145in}{1.795536in}}%
\pgfpathlineto{\pgfqpoint{0.709831in}{1.794592in}}%
\pgfpathlineto{\pgfqpoint{0.726517in}{1.793648in}}%
\pgfpathlineto{\pgfqpoint{0.743204in}{1.792704in}}%
\pgfpathlineto{\pgfqpoint{0.759890in}{1.791760in}}%
\pgfpathlineto{\pgfqpoint{0.776576in}{1.790817in}}%
\pgfpathlineto{\pgfqpoint{0.793262in}{1.789873in}}%
\pgfpathlineto{\pgfqpoint{0.809948in}{1.788929in}}%
\pgfpathlineto{\pgfqpoint{0.826634in}{1.787985in}}%
\pgfpathlineto{\pgfqpoint{0.843320in}{1.787041in}}%
\pgfpathlineto{\pgfqpoint{0.860006in}{1.786097in}}%
\pgfpathlineto{\pgfqpoint{0.876693in}{1.785153in}}%
\pgfpathlineto{\pgfqpoint{0.893379in}{1.784210in}}%
\pgfpathlineto{\pgfqpoint{0.910065in}{1.783266in}}%
\pgfpathlineto{\pgfqpoint{0.926751in}{1.782322in}}%
\pgfpathlineto{\pgfqpoint{0.943437in}{1.781378in}}%
\pgfpathlineto{\pgfqpoint{0.960123in}{1.780434in}}%
\pgfpathlineto{\pgfqpoint{0.976809in}{1.779490in}}%
\pgfpathlineto{\pgfqpoint{0.993495in}{1.778546in}}%
\pgfpathlineto{\pgfqpoint{1.010182in}{1.777603in}}%
\pgfpathlineto{\pgfqpoint{1.026868in}{1.776659in}}%
\pgfpathlineto{\pgfqpoint{1.043554in}{1.775715in}}%
\pgfpathlineto{\pgfqpoint{1.060240in}{1.774771in}}%
\pgfpathlineto{\pgfqpoint{1.076926in}{1.773827in}}%
\pgfpathlineto{\pgfqpoint{1.093612in}{1.772883in}}%
\pgfpathlineto{\pgfqpoint{1.110298in}{1.771939in}}%
\pgfpathlineto{\pgfqpoint{1.126985in}{1.770996in}}%
\pgfpathlineto{\pgfqpoint{1.143671in}{1.770052in}}%
\pgfpathlineto{\pgfqpoint{1.160357in}{1.769108in}}%
\pgfpathlineto{\pgfqpoint{1.177043in}{1.768164in}}%
\pgfpathlineto{\pgfqpoint{1.193729in}{1.767220in}}%
\pgfpathlineto{\pgfqpoint{1.210415in}{1.766276in}}%
\pgfpathlineto{\pgfqpoint{1.227101in}{1.765332in}}%
\pgfpathlineto{\pgfqpoint{1.243787in}{1.764389in}}%
\pgfpathlineto{\pgfqpoint{1.260474in}{1.763445in}}%
\pgfpathlineto{\pgfqpoint{1.277160in}{1.762501in}}%
\pgfpathlineto{\pgfqpoint{1.293846in}{1.761557in}}%
\pgfpathlineto{\pgfqpoint{1.310532in}{1.760613in}}%
\pgfpathlineto{\pgfqpoint{1.327218in}{1.759669in}}%
\pgfpathlineto{\pgfqpoint{1.343904in}{1.758725in}}%
\pgfpathlineto{\pgfqpoint{1.360590in}{1.757782in}}%
\pgfpathlineto{\pgfqpoint{1.377276in}{1.756838in}}%
\pgfpathlineto{\pgfqpoint{1.393963in}{1.755894in}}%
\pgfpathlineto{\pgfqpoint{1.410649in}{1.754950in}}%
\pgfpathlineto{\pgfqpoint{1.427335in}{1.754006in}}%
\pgfpathlineto{\pgfqpoint{1.444021in}{1.753062in}}%
\pgfpathlineto{\pgfqpoint{1.460707in}{1.752118in}}%
\pgfpathlineto{\pgfqpoint{1.477393in}{1.751175in}}%
\pgfpathlineto{\pgfqpoint{1.494079in}{1.750231in}}%
\pgfpathlineto{\pgfqpoint{1.510765in}{1.749287in}}%
\pgfpathlineto{\pgfqpoint{1.527452in}{1.748343in}}%
\pgfpathlineto{\pgfqpoint{1.544138in}{1.747399in}}%
\pgfpathlineto{\pgfqpoint{1.560824in}{1.746455in}}%
\pgfpathlineto{\pgfqpoint{1.577510in}{1.745511in}}%
\pgfpathlineto{\pgfqpoint{1.594196in}{1.744568in}}%
\pgfpathlineto{\pgfqpoint{1.610882in}{1.743624in}}%
\pgfpathlineto{\pgfqpoint{1.627568in}{1.742680in}}%
\pgfpathlineto{\pgfqpoint{1.644255in}{1.741736in}}%
\pgfpathlineto{\pgfqpoint{1.660941in}{1.740792in}}%
\pgfpathlineto{\pgfqpoint{1.677627in}{1.739848in}}%
\pgfpathlineto{\pgfqpoint{1.694313in}{1.738904in}}%
\pgfpathlineto{\pgfqpoint{1.710999in}{1.737961in}}%
\pgfpathlineto{\pgfqpoint{1.727685in}{1.737017in}}%
\pgfpathlineto{\pgfqpoint{1.744371in}{1.736073in}}%
\pgfpathlineto{\pgfqpoint{1.761057in}{1.735129in}}%
\pgfpathlineto{\pgfqpoint{1.777744in}{1.734185in}}%
\pgfpathlineto{\pgfqpoint{1.794430in}{1.733241in}}%
\pgfpathlineto{\pgfqpoint{1.811116in}{1.732297in}}%
\pgfpathlineto{\pgfqpoint{1.827802in}{1.731354in}}%
\pgfpathlineto{\pgfqpoint{1.844488in}{1.730410in}}%
\pgfpathlineto{\pgfqpoint{1.861174in}{1.729466in}}%
\pgfpathlineto{\pgfqpoint{1.877860in}{1.728522in}}%
\pgfpathlineto{\pgfqpoint{1.894546in}{1.727578in}}%
\pgfpathlineto{\pgfqpoint{1.911233in}{1.726634in}}%
\pgfpathlineto{\pgfqpoint{1.927919in}{1.725690in}}%
\pgfpathlineto{\pgfqpoint{1.944605in}{1.724747in}}%
\pgfpathlineto{\pgfqpoint{1.961291in}{1.723803in}}%
\pgfpathlineto{\pgfqpoint{1.977977in}{1.722859in}}%
\pgfpathlineto{\pgfqpoint{1.994663in}{1.721915in}}%
\pgfpathlineto{\pgfqpoint{2.011349in}{1.720971in}}%
\pgfpathlineto{\pgfqpoint{2.028035in}{1.720027in}}%
\pgfpathlineto{\pgfqpoint{2.044722in}{1.719083in}}%
\pgfpathlineto{\pgfqpoint{2.061408in}{1.718140in}}%
\pgfpathlineto{\pgfqpoint{2.078094in}{1.717196in}}%
\pgfpathlineto{\pgfqpoint{2.094780in}{1.716252in}}%
\pgfpathlineto{\pgfqpoint{2.111466in}{1.715308in}}%
\pgfpathlineto{\pgfqpoint{2.128152in}{1.714364in}}%
\pgfpathlineto{\pgfqpoint{2.144838in}{1.713420in}}%
\pgfpathlineto{\pgfqpoint{2.161525in}{1.712476in}}%
\pgfpathlineto{\pgfqpoint{2.178211in}{1.711533in}}%
\pgfusepath{stroke}%
\end{pgfscope}%
\begin{pgfscope}%
\pgfpathrectangle{\pgfqpoint{0.526284in}{0.473557in}}{\pgfqpoint{1.651927in}{1.704653in}}%
\pgfusepath{clip}%
\pgfsetrectcap%
\pgfsetroundjoin%
\pgfsetlinewidth{2.258437pt}%
\definecolor{currentstroke}{rgb}{0.172549,0.627451,0.172549}%
\pgfsetstrokecolor{currentstroke}%
\pgfsetdash{}{0pt}%
\pgfpathmoveto{\pgfqpoint{0.526284in}{1.503515in}}%
\pgfpathlineto{\pgfqpoint{0.542970in}{1.502724in}}%
\pgfpathlineto{\pgfqpoint{0.559656in}{1.501933in}}%
\pgfpathlineto{\pgfqpoint{0.576342in}{1.501142in}}%
\pgfpathlineto{\pgfqpoint{0.593028in}{1.500352in}}%
\pgfpathlineto{\pgfqpoint{0.609715in}{1.499561in}}%
\pgfpathlineto{\pgfqpoint{0.626401in}{1.498770in}}%
\pgfpathlineto{\pgfqpoint{0.643087in}{1.497979in}}%
\pgfpathlineto{\pgfqpoint{0.659773in}{1.497189in}}%
\pgfpathlineto{\pgfqpoint{0.676459in}{1.496398in}}%
\pgfpathlineto{\pgfqpoint{0.693145in}{1.495607in}}%
\pgfpathlineto{\pgfqpoint{0.709831in}{1.494816in}}%
\pgfpathlineto{\pgfqpoint{0.726517in}{1.494026in}}%
\pgfpathlineto{\pgfqpoint{0.743204in}{1.493235in}}%
\pgfpathlineto{\pgfqpoint{0.759890in}{1.492444in}}%
\pgfpathlineto{\pgfqpoint{0.776576in}{1.491653in}}%
\pgfpathlineto{\pgfqpoint{0.793262in}{1.490863in}}%
\pgfpathlineto{\pgfqpoint{0.809948in}{1.490072in}}%
\pgfpathlineto{\pgfqpoint{0.826634in}{1.489281in}}%
\pgfpathlineto{\pgfqpoint{0.843320in}{1.488490in}}%
\pgfpathlineto{\pgfqpoint{0.860006in}{1.487700in}}%
\pgfpathlineto{\pgfqpoint{0.876693in}{1.486909in}}%
\pgfpathlineto{\pgfqpoint{0.893379in}{1.486118in}}%
\pgfpathlineto{\pgfqpoint{0.910065in}{1.485327in}}%
\pgfpathlineto{\pgfqpoint{0.926751in}{1.484537in}}%
\pgfpathlineto{\pgfqpoint{0.943437in}{1.483746in}}%
\pgfpathlineto{\pgfqpoint{0.960123in}{1.482955in}}%
\pgfpathlineto{\pgfqpoint{0.976809in}{1.482164in}}%
\pgfpathlineto{\pgfqpoint{0.993495in}{1.481374in}}%
\pgfpathlineto{\pgfqpoint{1.010182in}{1.480583in}}%
\pgfpathlineto{\pgfqpoint{1.026868in}{1.479792in}}%
\pgfpathlineto{\pgfqpoint{1.043554in}{1.479001in}}%
\pgfpathlineto{\pgfqpoint{1.060240in}{1.478210in}}%
\pgfpathlineto{\pgfqpoint{1.076926in}{1.477420in}}%
\pgfpathlineto{\pgfqpoint{1.093612in}{1.476629in}}%
\pgfpathlineto{\pgfqpoint{1.110298in}{1.475838in}}%
\pgfpathlineto{\pgfqpoint{1.126985in}{1.475047in}}%
\pgfpathlineto{\pgfqpoint{1.143671in}{1.474257in}}%
\pgfpathlineto{\pgfqpoint{1.160357in}{1.473466in}}%
\pgfpathlineto{\pgfqpoint{1.177043in}{1.472675in}}%
\pgfpathlineto{\pgfqpoint{1.193729in}{1.471884in}}%
\pgfpathlineto{\pgfqpoint{1.210415in}{1.471094in}}%
\pgfpathlineto{\pgfqpoint{1.227101in}{1.470303in}}%
\pgfpathlineto{\pgfqpoint{1.243787in}{1.469512in}}%
\pgfpathlineto{\pgfqpoint{1.260474in}{1.468721in}}%
\pgfpathlineto{\pgfqpoint{1.277160in}{1.467931in}}%
\pgfpathlineto{\pgfqpoint{1.293846in}{1.467140in}}%
\pgfpathlineto{\pgfqpoint{1.310532in}{1.466349in}}%
\pgfpathlineto{\pgfqpoint{1.327218in}{1.465558in}}%
\pgfpathlineto{\pgfqpoint{1.343904in}{1.464768in}}%
\pgfpathlineto{\pgfqpoint{1.360590in}{1.463977in}}%
\pgfpathlineto{\pgfqpoint{1.377276in}{1.463186in}}%
\pgfpathlineto{\pgfqpoint{1.393963in}{1.462395in}}%
\pgfpathlineto{\pgfqpoint{1.410649in}{1.461605in}}%
\pgfpathlineto{\pgfqpoint{1.427335in}{1.460814in}}%
\pgfpathlineto{\pgfqpoint{1.444021in}{1.460023in}}%
\pgfpathlineto{\pgfqpoint{1.460707in}{1.459232in}}%
\pgfpathlineto{\pgfqpoint{1.477393in}{1.458442in}}%
\pgfpathlineto{\pgfqpoint{1.494079in}{1.457651in}}%
\pgfpathlineto{\pgfqpoint{1.510765in}{1.456860in}}%
\pgfpathlineto{\pgfqpoint{1.527452in}{1.456069in}}%
\pgfpathlineto{\pgfqpoint{1.544138in}{1.455279in}}%
\pgfpathlineto{\pgfqpoint{1.560824in}{1.454488in}}%
\pgfpathlineto{\pgfqpoint{1.577510in}{1.453697in}}%
\pgfpathlineto{\pgfqpoint{1.594196in}{1.452906in}}%
\pgfpathlineto{\pgfqpoint{1.610882in}{1.452116in}}%
\pgfpathlineto{\pgfqpoint{1.627568in}{1.451325in}}%
\pgfpathlineto{\pgfqpoint{1.644255in}{1.450534in}}%
\pgfpathlineto{\pgfqpoint{1.660941in}{1.449743in}}%
\pgfpathlineto{\pgfqpoint{1.677627in}{1.448953in}}%
\pgfpathlineto{\pgfqpoint{1.694313in}{1.448162in}}%
\pgfpathlineto{\pgfqpoint{1.710999in}{1.447371in}}%
\pgfpathlineto{\pgfqpoint{1.727685in}{1.446580in}}%
\pgfpathlineto{\pgfqpoint{1.744371in}{1.445789in}}%
\pgfpathlineto{\pgfqpoint{1.761057in}{1.444999in}}%
\pgfpathlineto{\pgfqpoint{1.777744in}{1.444208in}}%
\pgfpathlineto{\pgfqpoint{1.794430in}{1.443417in}}%
\pgfpathlineto{\pgfqpoint{1.811116in}{1.442626in}}%
\pgfpathlineto{\pgfqpoint{1.827802in}{1.441836in}}%
\pgfpathlineto{\pgfqpoint{1.844488in}{1.441045in}}%
\pgfpathlineto{\pgfqpoint{1.861174in}{1.440254in}}%
\pgfpathlineto{\pgfqpoint{1.877860in}{1.439463in}}%
\pgfpathlineto{\pgfqpoint{1.894546in}{1.438673in}}%
\pgfpathlineto{\pgfqpoint{1.911233in}{1.437882in}}%
\pgfpathlineto{\pgfqpoint{1.927919in}{1.437091in}}%
\pgfpathlineto{\pgfqpoint{1.944605in}{1.436300in}}%
\pgfpathlineto{\pgfqpoint{1.961291in}{1.435510in}}%
\pgfpathlineto{\pgfqpoint{1.977977in}{1.434719in}}%
\pgfpathlineto{\pgfqpoint{1.994663in}{1.433928in}}%
\pgfpathlineto{\pgfqpoint{2.011349in}{1.433137in}}%
\pgfpathlineto{\pgfqpoint{2.028035in}{1.432347in}}%
\pgfpathlineto{\pgfqpoint{2.044722in}{1.431556in}}%
\pgfpathlineto{\pgfqpoint{2.061408in}{1.430765in}}%
\pgfpathlineto{\pgfqpoint{2.078094in}{1.429974in}}%
\pgfpathlineto{\pgfqpoint{2.094780in}{1.429184in}}%
\pgfpathlineto{\pgfqpoint{2.111466in}{1.428393in}}%
\pgfpathlineto{\pgfqpoint{2.128152in}{1.427602in}}%
\pgfpathlineto{\pgfqpoint{2.144838in}{1.426811in}}%
\pgfpathlineto{\pgfqpoint{2.161525in}{1.426021in}}%
\pgfpathlineto{\pgfqpoint{2.178211in}{1.425230in}}%
\pgfusepath{stroke}%
\end{pgfscope}%
\begin{pgfscope}%
\pgfpathrectangle{\pgfqpoint{0.526284in}{0.473557in}}{\pgfqpoint{1.651927in}{1.704653in}}%
\pgfusepath{clip}%
\pgfsetrectcap%
\pgfsetroundjoin%
\pgfsetlinewidth{2.258437pt}%
\definecolor{currentstroke}{rgb}{0.839216,0.152941,0.156863}%
\pgfsetstrokecolor{currentstroke}%
\pgfsetdash{}{0pt}%
\pgfpathmoveto{\pgfqpoint{0.526284in}{1.283652in}}%
\pgfpathlineto{\pgfqpoint{0.542970in}{1.283777in}}%
\pgfpathlineto{\pgfqpoint{0.559656in}{1.283902in}}%
\pgfpathlineto{\pgfqpoint{0.576342in}{1.284028in}}%
\pgfpathlineto{\pgfqpoint{0.593028in}{1.284153in}}%
\pgfpathlineto{\pgfqpoint{0.609715in}{1.284278in}}%
\pgfpathlineto{\pgfqpoint{0.626401in}{1.284403in}}%
\pgfpathlineto{\pgfqpoint{0.643087in}{1.284528in}}%
\pgfpathlineto{\pgfqpoint{0.659773in}{1.284654in}}%
\pgfpathlineto{\pgfqpoint{0.676459in}{1.284779in}}%
\pgfpathlineto{\pgfqpoint{0.693145in}{1.284904in}}%
\pgfpathlineto{\pgfqpoint{0.709831in}{1.285029in}}%
\pgfpathlineto{\pgfqpoint{0.726517in}{1.285154in}}%
\pgfpathlineto{\pgfqpoint{0.743204in}{1.285280in}}%
\pgfpathlineto{\pgfqpoint{0.759890in}{1.285405in}}%
\pgfpathlineto{\pgfqpoint{0.776576in}{1.285530in}}%
\pgfpathlineto{\pgfqpoint{0.793262in}{1.285655in}}%
\pgfpathlineto{\pgfqpoint{0.809948in}{1.285780in}}%
\pgfpathlineto{\pgfqpoint{0.826634in}{1.285906in}}%
\pgfpathlineto{\pgfqpoint{0.843320in}{1.286031in}}%
\pgfpathlineto{\pgfqpoint{0.860006in}{1.286156in}}%
\pgfpathlineto{\pgfqpoint{0.876693in}{1.286281in}}%
\pgfpathlineto{\pgfqpoint{0.893379in}{1.286407in}}%
\pgfpathlineto{\pgfqpoint{0.910065in}{1.286532in}}%
\pgfpathlineto{\pgfqpoint{0.926751in}{1.286657in}}%
\pgfpathlineto{\pgfqpoint{0.943437in}{1.286782in}}%
\pgfpathlineto{\pgfqpoint{0.960123in}{1.286907in}}%
\pgfpathlineto{\pgfqpoint{0.976809in}{1.287033in}}%
\pgfpathlineto{\pgfqpoint{0.993495in}{1.287158in}}%
\pgfpathlineto{\pgfqpoint{1.010182in}{1.287283in}}%
\pgfpathlineto{\pgfqpoint{1.026868in}{1.287408in}}%
\pgfpathlineto{\pgfqpoint{1.043554in}{1.287533in}}%
\pgfpathlineto{\pgfqpoint{1.060240in}{1.287659in}}%
\pgfpathlineto{\pgfqpoint{1.076926in}{1.287784in}}%
\pgfpathlineto{\pgfqpoint{1.093612in}{1.287909in}}%
\pgfpathlineto{\pgfqpoint{1.110298in}{1.288034in}}%
\pgfpathlineto{\pgfqpoint{1.126985in}{1.288159in}}%
\pgfpathlineto{\pgfqpoint{1.143671in}{1.288285in}}%
\pgfpathlineto{\pgfqpoint{1.160357in}{1.288410in}}%
\pgfpathlineto{\pgfqpoint{1.177043in}{1.288535in}}%
\pgfpathlineto{\pgfqpoint{1.193729in}{1.288660in}}%
\pgfpathlineto{\pgfqpoint{1.210415in}{1.288785in}}%
\pgfpathlineto{\pgfqpoint{1.227101in}{1.288911in}}%
\pgfpathlineto{\pgfqpoint{1.243787in}{1.289036in}}%
\pgfpathlineto{\pgfqpoint{1.260474in}{1.289161in}}%
\pgfpathlineto{\pgfqpoint{1.277160in}{1.289286in}}%
\pgfpathlineto{\pgfqpoint{1.293846in}{1.289411in}}%
\pgfpathlineto{\pgfqpoint{1.310532in}{1.289537in}}%
\pgfpathlineto{\pgfqpoint{1.327218in}{1.289662in}}%
\pgfpathlineto{\pgfqpoint{1.343904in}{1.289787in}}%
\pgfpathlineto{\pgfqpoint{1.360590in}{1.289912in}}%
\pgfpathlineto{\pgfqpoint{1.377276in}{1.290038in}}%
\pgfpathlineto{\pgfqpoint{1.393963in}{1.290163in}}%
\pgfpathlineto{\pgfqpoint{1.410649in}{1.290288in}}%
\pgfpathlineto{\pgfqpoint{1.427335in}{1.290413in}}%
\pgfpathlineto{\pgfqpoint{1.444021in}{1.290538in}}%
\pgfpathlineto{\pgfqpoint{1.460707in}{1.290664in}}%
\pgfpathlineto{\pgfqpoint{1.477393in}{1.290789in}}%
\pgfpathlineto{\pgfqpoint{1.494079in}{1.290914in}}%
\pgfpathlineto{\pgfqpoint{1.510765in}{1.291039in}}%
\pgfpathlineto{\pgfqpoint{1.527452in}{1.291164in}}%
\pgfpathlineto{\pgfqpoint{1.544138in}{1.291290in}}%
\pgfpathlineto{\pgfqpoint{1.560824in}{1.291415in}}%
\pgfpathlineto{\pgfqpoint{1.577510in}{1.291540in}}%
\pgfpathlineto{\pgfqpoint{1.594196in}{1.291665in}}%
\pgfpathlineto{\pgfqpoint{1.610882in}{1.291790in}}%
\pgfpathlineto{\pgfqpoint{1.627568in}{1.291916in}}%
\pgfpathlineto{\pgfqpoint{1.644255in}{1.292041in}}%
\pgfpathlineto{\pgfqpoint{1.660941in}{1.292166in}}%
\pgfpathlineto{\pgfqpoint{1.677627in}{1.292291in}}%
\pgfpathlineto{\pgfqpoint{1.694313in}{1.292416in}}%
\pgfpathlineto{\pgfqpoint{1.710999in}{1.292542in}}%
\pgfpathlineto{\pgfqpoint{1.727685in}{1.292667in}}%
\pgfpathlineto{\pgfqpoint{1.744371in}{1.292792in}}%
\pgfpathlineto{\pgfqpoint{1.761057in}{1.292917in}}%
\pgfpathlineto{\pgfqpoint{1.777744in}{1.293043in}}%
\pgfpathlineto{\pgfqpoint{1.794430in}{1.293168in}}%
\pgfpathlineto{\pgfqpoint{1.811116in}{1.293293in}}%
\pgfpathlineto{\pgfqpoint{1.827802in}{1.293418in}}%
\pgfpathlineto{\pgfqpoint{1.844488in}{1.293543in}}%
\pgfpathlineto{\pgfqpoint{1.861174in}{1.293669in}}%
\pgfpathlineto{\pgfqpoint{1.877860in}{1.293794in}}%
\pgfpathlineto{\pgfqpoint{1.894546in}{1.293919in}}%
\pgfpathlineto{\pgfqpoint{1.911233in}{1.294044in}}%
\pgfpathlineto{\pgfqpoint{1.927919in}{1.294169in}}%
\pgfpathlineto{\pgfqpoint{1.944605in}{1.294295in}}%
\pgfpathlineto{\pgfqpoint{1.961291in}{1.294420in}}%
\pgfpathlineto{\pgfqpoint{1.977977in}{1.294545in}}%
\pgfpathlineto{\pgfqpoint{1.994663in}{1.294670in}}%
\pgfpathlineto{\pgfqpoint{2.011349in}{1.294795in}}%
\pgfpathlineto{\pgfqpoint{2.028035in}{1.294921in}}%
\pgfpathlineto{\pgfqpoint{2.044722in}{1.295046in}}%
\pgfpathlineto{\pgfqpoint{2.061408in}{1.295171in}}%
\pgfpathlineto{\pgfqpoint{2.078094in}{1.295296in}}%
\pgfpathlineto{\pgfqpoint{2.094780in}{1.295421in}}%
\pgfpathlineto{\pgfqpoint{2.111466in}{1.295547in}}%
\pgfpathlineto{\pgfqpoint{2.128152in}{1.295672in}}%
\pgfpathlineto{\pgfqpoint{2.144838in}{1.295797in}}%
\pgfpathlineto{\pgfqpoint{2.161525in}{1.295922in}}%
\pgfpathlineto{\pgfqpoint{2.178211in}{1.296048in}}%
\pgfusepath{stroke}%
\end{pgfscope}%
\begin{pgfscope}%
\pgfpathrectangle{\pgfqpoint{0.526284in}{0.473557in}}{\pgfqpoint{1.651927in}{1.704653in}}%
\pgfusepath{clip}%
\pgfsetrectcap%
\pgfsetroundjoin%
\pgfsetlinewidth{2.258437pt}%
\definecolor{currentstroke}{rgb}{0.580392,0.403922,0.741176}%
\pgfsetstrokecolor{currentstroke}%
\pgfsetdash{}{0pt}%
\pgfpathmoveto{\pgfqpoint{0.526284in}{0.916135in}}%
\pgfpathlineto{\pgfqpoint{0.542970in}{0.919443in}}%
\pgfpathlineto{\pgfqpoint{0.559656in}{0.922752in}}%
\pgfpathlineto{\pgfqpoint{0.576342in}{0.926060in}}%
\pgfpathlineto{\pgfqpoint{0.593028in}{0.929369in}}%
\pgfpathlineto{\pgfqpoint{0.609715in}{0.932677in}}%
\pgfpathlineto{\pgfqpoint{0.626401in}{0.935986in}}%
\pgfpathlineto{\pgfqpoint{0.643087in}{0.939294in}}%
\pgfpathlineto{\pgfqpoint{0.659773in}{0.942603in}}%
\pgfpathlineto{\pgfqpoint{0.676459in}{0.945912in}}%
\pgfpathlineto{\pgfqpoint{0.693145in}{0.949220in}}%
\pgfpathlineto{\pgfqpoint{0.709831in}{0.952529in}}%
\pgfpathlineto{\pgfqpoint{0.726517in}{0.955837in}}%
\pgfpathlineto{\pgfqpoint{0.743204in}{0.959146in}}%
\pgfpathlineto{\pgfqpoint{0.759890in}{0.962454in}}%
\pgfpathlineto{\pgfqpoint{0.776576in}{0.965763in}}%
\pgfpathlineto{\pgfqpoint{0.793262in}{0.969071in}}%
\pgfpathlineto{\pgfqpoint{0.809948in}{0.972380in}}%
\pgfpathlineto{\pgfqpoint{0.826634in}{0.975688in}}%
\pgfpathlineto{\pgfqpoint{0.843320in}{0.978997in}}%
\pgfpathlineto{\pgfqpoint{0.860006in}{0.982305in}}%
\pgfpathlineto{\pgfqpoint{0.876693in}{0.985614in}}%
\pgfpathlineto{\pgfqpoint{0.893379in}{0.988923in}}%
\pgfpathlineto{\pgfqpoint{0.910065in}{0.992231in}}%
\pgfpathlineto{\pgfqpoint{0.926751in}{0.995540in}}%
\pgfpathlineto{\pgfqpoint{0.943437in}{0.998848in}}%
\pgfpathlineto{\pgfqpoint{0.960123in}{1.002157in}}%
\pgfpathlineto{\pgfqpoint{0.976809in}{1.005465in}}%
\pgfpathlineto{\pgfqpoint{0.993495in}{1.008774in}}%
\pgfpathlineto{\pgfqpoint{1.010182in}{1.012082in}}%
\pgfpathlineto{\pgfqpoint{1.026868in}{1.015391in}}%
\pgfpathlineto{\pgfqpoint{1.043554in}{1.018699in}}%
\pgfpathlineto{\pgfqpoint{1.060240in}{1.022008in}}%
\pgfpathlineto{\pgfqpoint{1.076926in}{1.025316in}}%
\pgfpathlineto{\pgfqpoint{1.093612in}{1.028625in}}%
\pgfpathlineto{\pgfqpoint{1.110298in}{1.031934in}}%
\pgfpathlineto{\pgfqpoint{1.126985in}{1.035242in}}%
\pgfpathlineto{\pgfqpoint{1.143671in}{1.038551in}}%
\pgfpathlineto{\pgfqpoint{1.160357in}{1.041859in}}%
\pgfpathlineto{\pgfqpoint{1.177043in}{1.045168in}}%
\pgfpathlineto{\pgfqpoint{1.193729in}{1.048476in}}%
\pgfpathlineto{\pgfqpoint{1.210415in}{1.051785in}}%
\pgfpathlineto{\pgfqpoint{1.227101in}{1.055093in}}%
\pgfpathlineto{\pgfqpoint{1.243787in}{1.058402in}}%
\pgfpathlineto{\pgfqpoint{1.260474in}{1.061710in}}%
\pgfpathlineto{\pgfqpoint{1.277160in}{1.065019in}}%
\pgfpathlineto{\pgfqpoint{1.293846in}{1.068328in}}%
\pgfpathlineto{\pgfqpoint{1.310532in}{1.071636in}}%
\pgfpathlineto{\pgfqpoint{1.327218in}{1.074945in}}%
\pgfpathlineto{\pgfqpoint{1.343904in}{1.078253in}}%
\pgfpathlineto{\pgfqpoint{1.360590in}{1.081562in}}%
\pgfpathlineto{\pgfqpoint{1.377276in}{1.084870in}}%
\pgfpathlineto{\pgfqpoint{1.393963in}{1.088179in}}%
\pgfpathlineto{\pgfqpoint{1.410649in}{1.091487in}}%
\pgfpathlineto{\pgfqpoint{1.427335in}{1.094796in}}%
\pgfpathlineto{\pgfqpoint{1.444021in}{1.098104in}}%
\pgfpathlineto{\pgfqpoint{1.460707in}{1.101413in}}%
\pgfpathlineto{\pgfqpoint{1.477393in}{1.104721in}}%
\pgfpathlineto{\pgfqpoint{1.494079in}{1.108030in}}%
\pgfpathlineto{\pgfqpoint{1.510765in}{1.111339in}}%
\pgfpathlineto{\pgfqpoint{1.527452in}{1.114647in}}%
\pgfpathlineto{\pgfqpoint{1.544138in}{1.117956in}}%
\pgfpathlineto{\pgfqpoint{1.560824in}{1.121264in}}%
\pgfpathlineto{\pgfqpoint{1.577510in}{1.124573in}}%
\pgfpathlineto{\pgfqpoint{1.594196in}{1.127881in}}%
\pgfpathlineto{\pgfqpoint{1.610882in}{1.131190in}}%
\pgfpathlineto{\pgfqpoint{1.627568in}{1.134498in}}%
\pgfpathlineto{\pgfqpoint{1.644255in}{1.137807in}}%
\pgfpathlineto{\pgfqpoint{1.660941in}{1.141115in}}%
\pgfpathlineto{\pgfqpoint{1.677627in}{1.144424in}}%
\pgfpathlineto{\pgfqpoint{1.694313in}{1.147732in}}%
\pgfpathlineto{\pgfqpoint{1.710999in}{1.151041in}}%
\pgfpathlineto{\pgfqpoint{1.727685in}{1.154350in}}%
\pgfpathlineto{\pgfqpoint{1.744371in}{1.157658in}}%
\pgfpathlineto{\pgfqpoint{1.761057in}{1.160967in}}%
\pgfpathlineto{\pgfqpoint{1.777744in}{1.164275in}}%
\pgfpathlineto{\pgfqpoint{1.794430in}{1.167584in}}%
\pgfpathlineto{\pgfqpoint{1.811116in}{1.170892in}}%
\pgfpathlineto{\pgfqpoint{1.827802in}{1.174201in}}%
\pgfpathlineto{\pgfqpoint{1.844488in}{1.177509in}}%
\pgfpathlineto{\pgfqpoint{1.861174in}{1.180818in}}%
\pgfpathlineto{\pgfqpoint{1.877860in}{1.184126in}}%
\pgfpathlineto{\pgfqpoint{1.894546in}{1.187435in}}%
\pgfpathlineto{\pgfqpoint{1.911233in}{1.190744in}}%
\pgfpathlineto{\pgfqpoint{1.927919in}{1.194052in}}%
\pgfpathlineto{\pgfqpoint{1.944605in}{1.197361in}}%
\pgfpathlineto{\pgfqpoint{1.961291in}{1.200669in}}%
\pgfpathlineto{\pgfqpoint{1.977977in}{1.203978in}}%
\pgfpathlineto{\pgfqpoint{1.994663in}{1.207286in}}%
\pgfpathlineto{\pgfqpoint{2.011349in}{1.210595in}}%
\pgfpathlineto{\pgfqpoint{2.028035in}{1.213903in}}%
\pgfpathlineto{\pgfqpoint{2.044722in}{1.217212in}}%
\pgfpathlineto{\pgfqpoint{2.061408in}{1.220520in}}%
\pgfpathlineto{\pgfqpoint{2.078094in}{1.223829in}}%
\pgfpathlineto{\pgfqpoint{2.094780in}{1.227137in}}%
\pgfpathlineto{\pgfqpoint{2.111466in}{1.230446in}}%
\pgfpathlineto{\pgfqpoint{2.128152in}{1.233755in}}%
\pgfpathlineto{\pgfqpoint{2.144838in}{1.237063in}}%
\pgfpathlineto{\pgfqpoint{2.161525in}{1.240372in}}%
\pgfpathlineto{\pgfqpoint{2.178211in}{1.243680in}}%
\pgfusepath{stroke}%
\end{pgfscope}%
\begin{pgfscope}%
\pgfsetrectcap%
\pgfsetmiterjoin%
\pgfsetlinewidth{0.803000pt}%
\definecolor{currentstroke}{rgb}{0.000000,0.000000,0.000000}%
\pgfsetstrokecolor{currentstroke}%
\pgfsetdash{}{0pt}%
\pgfpathmoveto{\pgfqpoint{0.526284in}{0.473557in}}%
\pgfpathlineto{\pgfqpoint{0.526284in}{2.178211in}}%
\pgfusepath{stroke}%
\end{pgfscope}%
\begin{pgfscope}%
\pgfsetrectcap%
\pgfsetmiterjoin%
\pgfsetlinewidth{0.803000pt}%
\definecolor{currentstroke}{rgb}{0.000000,0.000000,0.000000}%
\pgfsetstrokecolor{currentstroke}%
\pgfsetdash{}{0pt}%
\pgfpathmoveto{\pgfqpoint{0.526284in}{0.473557in}}%
\pgfpathlineto{\pgfqpoint{2.178211in}{0.473557in}}%
\pgfusepath{stroke}%
\end{pgfscope}%
\end{pgfpicture}%
\makeatother%
\endgroup%
}} &
      \subfloat[\(\epsilon=0.01\)]{\resizebox{0.275\linewidth}{!}{%% Creator: Matplotlib, PGF backend
%%
%% To include the figure in your LaTeX document, write
%%   \input{<filename>.pgf}
%%
%% Make sure the required packages are loaded in your preamble
%%   \usepackage{pgf}
%%
%% and, on pdftex
%%   \usepackage[utf8]{inputenc}\DeclareUnicodeCharacter{2212}{-}
%%
%% or, on luatex and xetex
%%   \usepackage{unicode-math}
%%
%% Figures using additional raster images can only be included by \input if
%% they are in the same directory as the main LaTeX file. For loading figures
%% from other directories you can use the `import` package
%%   \usepackage{import}
%%
%% and then include the figures with
%%   \import{<path to file>}{<filename>.pgf}
%%
%% Matplotlib used the following preamble
%%   \usepackage[utf8]{inputenc}
%%   \usepackage[T1]{fontenc}
%%   \usepackage{amsmath}
%%   \newcommand*{\mat}[1]{\boldsymbol{#1}}
%%
\begingroup%
\makeatletter%
\begin{pgfpicture}%
\pgfpathrectangle{\pgfpointorigin}{\pgfqpoint{2.100000in}{2.278211in}}%
\pgfusepath{use as bounding box, clip}%
\begin{pgfscope}%
\pgfsetbuttcap%
\pgfsetmiterjoin%
\definecolor{currentfill}{rgb}{1.000000,1.000000,1.000000}%
\pgfsetfillcolor{currentfill}%
\pgfsetlinewidth{0.000000pt}%
\definecolor{currentstroke}{rgb}{1.000000,1.000000,1.000000}%
\pgfsetstrokecolor{currentstroke}%
\pgfsetstrokeopacity{0.000000}%
\pgfsetdash{}{0pt}%
\pgfpathmoveto{\pgfqpoint{0.000000in}{0.000000in}}%
\pgfpathlineto{\pgfqpoint{2.100000in}{0.000000in}}%
\pgfpathlineto{\pgfqpoint{2.100000in}{2.278211in}}%
\pgfpathlineto{\pgfqpoint{0.000000in}{2.278211in}}%
\pgfpathclose%
\pgfusepath{fill}%
\end{pgfscope}%
\begin{pgfscope}%
\pgfsetbuttcap%
\pgfsetmiterjoin%
\definecolor{currentfill}{rgb}{1.000000,1.000000,1.000000}%
\pgfsetfillcolor{currentfill}%
\pgfsetlinewidth{0.000000pt}%
\definecolor{currentstroke}{rgb}{0.000000,0.000000,0.000000}%
\pgfsetstrokecolor{currentstroke}%
\pgfsetstrokeopacity{0.000000}%
\pgfsetdash{}{0pt}%
\pgfpathmoveto{\pgfqpoint{0.341129in}{0.466613in}}%
\pgfpathlineto{\pgfqpoint{2.000000in}{0.466613in}}%
\pgfpathlineto{\pgfqpoint{2.000000in}{2.178211in}}%
\pgfpathlineto{\pgfqpoint{0.341129in}{2.178211in}}%
\pgfpathclose%
\pgfusepath{fill}%
\end{pgfscope}%
\begin{pgfscope}%
\pgfpathrectangle{\pgfqpoint{0.341129in}{0.466613in}}{\pgfqpoint{1.658871in}{1.711598in}}%
\pgfusepath{clip}%
\pgfsetroundcap%
\pgfsetroundjoin%
\pgfsetlinewidth{0.501875pt}%
\definecolor{currentstroke}{rgb}{0.800000,0.800000,0.800000}%
\pgfsetstrokecolor{currentstroke}%
\pgfsetdash{}{0pt}%
\pgfpathmoveto{\pgfqpoint{0.429247in}{0.466613in}}%
\pgfpathlineto{\pgfqpoint{0.429247in}{2.178211in}}%
\pgfusepath{stroke}%
\end{pgfscope}%
\begin{pgfscope}%
\definecolor{textcolor}{rgb}{0.150000,0.150000,0.150000}%
\pgfsetstrokecolor{textcolor}%
\pgfsetfillcolor{textcolor}%
\pgftext[x=0.429247in,y=0.376335in,,top]{\color{textcolor}\rmfamily\fontsize{8.000000}{9.600000}\selectfont \(\displaystyle {5}\)}%
\end{pgfscope}%
\begin{pgfscope}%
\pgfpathrectangle{\pgfqpoint{0.341129in}{0.466613in}}{\pgfqpoint{1.658871in}{1.711598in}}%
\pgfusepath{clip}%
\pgfsetroundcap%
\pgfsetroundjoin%
\pgfsetlinewidth{0.501875pt}%
\definecolor{currentstroke}{rgb}{0.800000,0.800000,0.800000}%
\pgfsetstrokecolor{currentstroke}%
\pgfsetdash{}{0pt}%
\pgfpathmoveto{\pgfqpoint{0.928330in}{0.466613in}}%
\pgfpathlineto{\pgfqpoint{0.928330in}{2.178211in}}%
\pgfusepath{stroke}%
\end{pgfscope}%
\begin{pgfscope}%
\definecolor{textcolor}{rgb}{0.150000,0.150000,0.150000}%
\pgfsetstrokecolor{textcolor}%
\pgfsetfillcolor{textcolor}%
\pgftext[x=0.928330in,y=0.376335in,,top]{\color{textcolor}\rmfamily\fontsize{8.000000}{9.600000}\selectfont \(\displaystyle {10}\)}%
\end{pgfscope}%
\begin{pgfscope}%
\pgfpathrectangle{\pgfqpoint{0.341129in}{0.466613in}}{\pgfqpoint{1.658871in}{1.711598in}}%
\pgfusepath{clip}%
\pgfsetroundcap%
\pgfsetroundjoin%
\pgfsetlinewidth{0.501875pt}%
\definecolor{currentstroke}{rgb}{0.800000,0.800000,0.800000}%
\pgfsetstrokecolor{currentstroke}%
\pgfsetdash{}{0pt}%
\pgfpathmoveto{\pgfqpoint{1.427413in}{0.466613in}}%
\pgfpathlineto{\pgfqpoint{1.427413in}{2.178211in}}%
\pgfusepath{stroke}%
\end{pgfscope}%
\begin{pgfscope}%
\definecolor{textcolor}{rgb}{0.150000,0.150000,0.150000}%
\pgfsetstrokecolor{textcolor}%
\pgfsetfillcolor{textcolor}%
\pgftext[x=1.427413in,y=0.376335in,,top]{\color{textcolor}\rmfamily\fontsize{8.000000}{9.600000}\selectfont \(\displaystyle {15}\)}%
\end{pgfscope}%
\begin{pgfscope}%
\pgfpathrectangle{\pgfqpoint{0.341129in}{0.466613in}}{\pgfqpoint{1.658871in}{1.711598in}}%
\pgfusepath{clip}%
\pgfsetroundcap%
\pgfsetroundjoin%
\pgfsetlinewidth{0.501875pt}%
\definecolor{currentstroke}{rgb}{0.800000,0.800000,0.800000}%
\pgfsetstrokecolor{currentstroke}%
\pgfsetdash{}{0pt}%
\pgfpathmoveto{\pgfqpoint{1.926497in}{0.466613in}}%
\pgfpathlineto{\pgfqpoint{1.926497in}{2.178211in}}%
\pgfusepath{stroke}%
\end{pgfscope}%
\begin{pgfscope}%
\definecolor{textcolor}{rgb}{0.150000,0.150000,0.150000}%
\pgfsetstrokecolor{textcolor}%
\pgfsetfillcolor{textcolor}%
\pgftext[x=1.926497in,y=0.376335in,,top]{\color{textcolor}\rmfamily\fontsize{8.000000}{9.600000}\selectfont \(\displaystyle {20}\)}%
\end{pgfscope}%
\begin{pgfscope}%
\definecolor{textcolor}{rgb}{0.150000,0.150000,0.150000}%
\pgfsetstrokecolor{textcolor}%
\pgfsetfillcolor{textcolor}%
\pgftext[x=1.170564in,y=0.222655in,,top]{\color{textcolor}\rmfamily\fontsize{10.000000}{12.000000}\selectfont Average degree}%
\end{pgfscope}%
\begin{pgfscope}%
\pgfpathrectangle{\pgfqpoint{0.341129in}{0.466613in}}{\pgfqpoint{1.658871in}{1.711598in}}%
\pgfusepath{clip}%
\pgfsetroundcap%
\pgfsetroundjoin%
\pgfsetlinewidth{0.501875pt}%
\definecolor{currentstroke}{rgb}{0.800000,0.800000,0.800000}%
\pgfsetstrokecolor{currentstroke}%
\pgfsetdash{}{0pt}%
\pgfpathmoveto{\pgfqpoint{0.341129in}{0.760562in}}%
\pgfpathlineto{\pgfqpoint{2.000000in}{0.760562in}}%
\pgfusepath{stroke}%
\end{pgfscope}%
\begin{pgfscope}%
\definecolor{textcolor}{rgb}{0.150000,0.150000,0.150000}%
\pgfsetstrokecolor{textcolor}%
\pgfsetfillcolor{textcolor}%
\pgftext[x=0.100000in, y=0.722300in, left, base]{\color{textcolor}\rmfamily\fontsize{8.000000}{9.600000}\selectfont \(\displaystyle {0.2}\)}%
\end{pgfscope}%
\begin{pgfscope}%
\pgfpathrectangle{\pgfqpoint{0.341129in}{0.466613in}}{\pgfqpoint{1.658871in}{1.711598in}}%
\pgfusepath{clip}%
\pgfsetroundcap%
\pgfsetroundjoin%
\pgfsetlinewidth{0.501875pt}%
\definecolor{currentstroke}{rgb}{0.800000,0.800000,0.800000}%
\pgfsetstrokecolor{currentstroke}%
\pgfsetdash{}{0pt}%
\pgfpathmoveto{\pgfqpoint{0.341129in}{1.102988in}}%
\pgfpathlineto{\pgfqpoint{2.000000in}{1.102988in}}%
\pgfusepath{stroke}%
\end{pgfscope}%
\begin{pgfscope}%
\definecolor{textcolor}{rgb}{0.150000,0.150000,0.150000}%
\pgfsetstrokecolor{textcolor}%
\pgfsetfillcolor{textcolor}%
\pgftext[x=0.100000in, y=1.064726in, left, base]{\color{textcolor}\rmfamily\fontsize{8.000000}{9.600000}\selectfont \(\displaystyle {0.4}\)}%
\end{pgfscope}%
\begin{pgfscope}%
\pgfpathrectangle{\pgfqpoint{0.341129in}{0.466613in}}{\pgfqpoint{1.658871in}{1.711598in}}%
\pgfusepath{clip}%
\pgfsetroundcap%
\pgfsetroundjoin%
\pgfsetlinewidth{0.501875pt}%
\definecolor{currentstroke}{rgb}{0.800000,0.800000,0.800000}%
\pgfsetstrokecolor{currentstroke}%
\pgfsetdash{}{0pt}%
\pgfpathmoveto{\pgfqpoint{0.341129in}{1.445415in}}%
\pgfpathlineto{\pgfqpoint{2.000000in}{1.445415in}}%
\pgfusepath{stroke}%
\end{pgfscope}%
\begin{pgfscope}%
\definecolor{textcolor}{rgb}{0.150000,0.150000,0.150000}%
\pgfsetstrokecolor{textcolor}%
\pgfsetfillcolor{textcolor}%
\pgftext[x=0.100000in, y=1.407153in, left, base]{\color{textcolor}\rmfamily\fontsize{8.000000}{9.600000}\selectfont \(\displaystyle {0.6}\)}%
\end{pgfscope}%
\begin{pgfscope}%
\pgfpathrectangle{\pgfqpoint{0.341129in}{0.466613in}}{\pgfqpoint{1.658871in}{1.711598in}}%
\pgfusepath{clip}%
\pgfsetroundcap%
\pgfsetroundjoin%
\pgfsetlinewidth{0.501875pt}%
\definecolor{currentstroke}{rgb}{0.800000,0.800000,0.800000}%
\pgfsetstrokecolor{currentstroke}%
\pgfsetdash{}{0pt}%
\pgfpathmoveto{\pgfqpoint{0.341129in}{1.787841in}}%
\pgfpathlineto{\pgfqpoint{2.000000in}{1.787841in}}%
\pgfusepath{stroke}%
\end{pgfscope}%
\begin{pgfscope}%
\definecolor{textcolor}{rgb}{0.150000,0.150000,0.150000}%
\pgfsetstrokecolor{textcolor}%
\pgfsetfillcolor{textcolor}%
\pgftext[x=0.100000in, y=1.749579in, left, base]{\color{textcolor}\rmfamily\fontsize{8.000000}{9.600000}\selectfont \(\displaystyle {0.8}\)}%
\end{pgfscope}%
\begin{pgfscope}%
\pgfpathrectangle{\pgfqpoint{0.341129in}{0.466613in}}{\pgfqpoint{1.658871in}{1.711598in}}%
\pgfusepath{clip}%
\pgfsetroundcap%
\pgfsetroundjoin%
\pgfsetlinewidth{0.501875pt}%
\definecolor{currentstroke}{rgb}{0.800000,0.800000,0.800000}%
\pgfsetstrokecolor{currentstroke}%
\pgfsetdash{}{0pt}%
\pgfpathmoveto{\pgfqpoint{0.341129in}{2.130268in}}%
\pgfpathlineto{\pgfqpoint{2.000000in}{2.130268in}}%
\pgfusepath{stroke}%
\end{pgfscope}%
\begin{pgfscope}%
\definecolor{textcolor}{rgb}{0.150000,0.150000,0.150000}%
\pgfsetstrokecolor{textcolor}%
\pgfsetfillcolor{textcolor}%
\pgftext[x=0.100000in, y=2.092005in, left, base]{\color{textcolor}\rmfamily\fontsize{8.000000}{9.600000}\selectfont \(\displaystyle {1.0}\)}%
\end{pgfscope}%
\begin{pgfscope}%
\pgfpathrectangle{\pgfqpoint{0.341129in}{0.466613in}}{\pgfqpoint{1.658871in}{1.711598in}}%
\pgfusepath{clip}%
\pgfsetbuttcap%
\pgfsetroundjoin%
\definecolor{currentfill}{rgb}{0.298039,0.447059,0.690196}%
\pgfsetfillcolor{currentfill}%
\pgfsetfillopacity{0.250000}%
\pgfsetlinewidth{1.003750pt}%
\definecolor{currentstroke}{rgb}{0.298039,0.447059,0.690196}%
\pgfsetstrokecolor{currentstroke}%
\pgfsetstrokeopacity{0.250000}%
\pgfsetdash{}{0pt}%
\pgfsys@defobject{currentmarker}{\pgfqpoint{-0.017010in}{-0.017010in}}{\pgfqpoint{0.017010in}{0.017010in}}{%
\pgfpathmoveto{\pgfqpoint{0.000000in}{-0.017010in}}%
\pgfpathcurveto{\pgfqpoint{0.004511in}{-0.017010in}}{\pgfqpoint{0.008838in}{-0.015218in}}{\pgfqpoint{0.012028in}{-0.012028in}}%
\pgfpathcurveto{\pgfqpoint{0.015218in}{-0.008838in}}{\pgfqpoint{0.017010in}{-0.004511in}}{\pgfqpoint{0.017010in}{0.000000in}}%
\pgfpathcurveto{\pgfqpoint{0.017010in}{0.004511in}}{\pgfqpoint{0.015218in}{0.008838in}}{\pgfqpoint{0.012028in}{0.012028in}}%
\pgfpathcurveto{\pgfqpoint{0.008838in}{0.015218in}}{\pgfqpoint{0.004511in}{0.017010in}}{\pgfqpoint{0.000000in}{0.017010in}}%
\pgfpathcurveto{\pgfqpoint{-0.004511in}{0.017010in}}{\pgfqpoint{-0.008838in}{0.015218in}}{\pgfqpoint{-0.012028in}{0.012028in}}%
\pgfpathcurveto{\pgfqpoint{-0.015218in}{0.008838in}}{\pgfqpoint{-0.017010in}{0.004511in}}{\pgfqpoint{-0.017010in}{0.000000in}}%
\pgfpathcurveto{\pgfqpoint{-0.017010in}{-0.004511in}}{\pgfqpoint{-0.015218in}{-0.008838in}}{\pgfqpoint{-0.012028in}{-0.012028in}}%
\pgfpathcurveto{\pgfqpoint{-0.008838in}{-0.015218in}}{\pgfqpoint{-0.004511in}{-0.017010in}}{\pgfqpoint{0.000000in}{-0.017010in}}%
\pgfpathclose%
\pgfusepath{stroke,fill}%
}%
\begin{pgfscope}%
\pgfsys@transformshift{1.684044in}{2.035875in}%
\pgfsys@useobject{currentmarker}{}%
\end{pgfscope}%
\begin{pgfscope}%
\pgfsys@transformshift{1.113006in}{2.042281in}%
\pgfsys@useobject{currentmarker}{}%
\end{pgfscope}%
\begin{pgfscope}%
\pgfsys@transformshift{0.938483in}{2.010996in}%
\pgfsys@useobject{currentmarker}{}%
\end{pgfscope}%
\begin{pgfscope}%
\pgfsys@transformshift{0.492780in}{1.966564in}%
\pgfsys@useobject{currentmarker}{}%
\end{pgfscope}%
\begin{pgfscope}%
\pgfsys@transformshift{0.529797in}{1.841809in}%
\pgfsys@useobject{currentmarker}{}%
\end{pgfscope}%
\begin{pgfscope}%
\pgfsys@transformshift{1.838155in}{2.032105in}%
\pgfsys@useobject{currentmarker}{}%
\end{pgfscope}%
\begin{pgfscope}%
\pgfsys@transformshift{0.477840in}{1.789150in}%
\pgfsys@useobject{currentmarker}{}%
\end{pgfscope}%
\begin{pgfscope}%
\pgfsys@transformshift{0.884452in}{2.013799in}%
\pgfsys@useobject{currentmarker}{}%
\end{pgfscope}%
\begin{pgfscope}%
\pgfsys@transformshift{1.509149in}{2.043502in}%
\pgfsys@useobject{currentmarker}{}%
\end{pgfscope}%
\begin{pgfscope}%
\pgfsys@transformshift{1.183184in}{1.952380in}%
\pgfsys@useobject{currentmarker}{}%
\end{pgfscope}%
\begin{pgfscope}%
\pgfsys@transformshift{1.140467in}{2.005003in}%
\pgfsys@useobject{currentmarker}{}%
\end{pgfscope}%
\begin{pgfscope}%
\pgfsys@transformshift{0.501776in}{1.967933in}%
\pgfsys@useobject{currentmarker}{}%
\end{pgfscope}%
\begin{pgfscope}%
\pgfsys@transformshift{0.461389in}{1.820851in}%
\pgfsys@useobject{currentmarker}{}%
\end{pgfscope}%
\begin{pgfscope}%
\pgfsys@transformshift{0.986905in}{1.985848in}%
\pgfsys@useobject{currentmarker}{}%
\end{pgfscope}%
\begin{pgfscope}%
\pgfsys@transformshift{0.969502in}{2.081259in}%
\pgfsys@useobject{currentmarker}{}%
\end{pgfscope}%
\begin{pgfscope}%
\pgfsys@transformshift{1.016218in}{1.967924in}%
\pgfsys@useobject{currentmarker}{}%
\end{pgfscope}%
\begin{pgfscope}%
\pgfsys@transformshift{0.413211in}{1.967707in}%
\pgfsys@useobject{currentmarker}{}%
\end{pgfscope}%
\begin{pgfscope}%
\pgfsys@transformshift{1.870040in}{2.048374in}%
\pgfsys@useobject{currentmarker}{}%
\end{pgfscope}%
\begin{pgfscope}%
\pgfsys@transformshift{1.031673in}{1.996205in}%
\pgfsys@useobject{currentmarker}{}%
\end{pgfscope}%
\begin{pgfscope}%
\pgfsys@transformshift{1.596314in}{2.000965in}%
\pgfsys@useobject{currentmarker}{}%
\end{pgfscope}%
\begin{pgfscope}%
\pgfsys@transformshift{0.496691in}{1.916916in}%
\pgfsys@useobject{currentmarker}{}%
\end{pgfscope}%
\begin{pgfscope}%
\pgfsys@transformshift{0.896936in}{1.944434in}%
\pgfsys@useobject{currentmarker}{}%
\end{pgfscope}%
\begin{pgfscope}%
\pgfsys@transformshift{0.442231in}{1.984713in}%
\pgfsys@useobject{currentmarker}{}%
\end{pgfscope}%
\begin{pgfscope}%
\pgfsys@transformshift{1.948436in}{2.077933in}%
\pgfsys@useobject{currentmarker}{}%
\end{pgfscope}%
\begin{pgfscope}%
\pgfsys@transformshift{0.607246in}{1.930821in}%
\pgfsys@useobject{currentmarker}{}%
\end{pgfscope}%
\begin{pgfscope}%
\pgfsys@transformshift{0.591373in}{1.784080in}%
\pgfsys@useobject{currentmarker}{}%
\end{pgfscope}%
\begin{pgfscope}%
\pgfsys@transformshift{0.495569in}{1.889508in}%
\pgfsys@useobject{currentmarker}{}%
\end{pgfscope}%
\begin{pgfscope}%
\pgfsys@transformshift{0.805700in}{1.987631in}%
\pgfsys@useobject{currentmarker}{}%
\end{pgfscope}%
\begin{pgfscope}%
\pgfsys@transformshift{0.472915in}{1.940626in}%
\pgfsys@useobject{currentmarker}{}%
\end{pgfscope}%
\begin{pgfscope}%
\pgfsys@transformshift{1.008191in}{1.973380in}%
\pgfsys@useobject{currentmarker}{}%
\end{pgfscope}%
\begin{pgfscope}%
\pgfsys@transformshift{0.795900in}{1.996031in}%
\pgfsys@useobject{currentmarker}{}%
\end{pgfscope}%
\begin{pgfscope}%
\pgfsys@transformshift{0.879596in}{1.971328in}%
\pgfsys@useobject{currentmarker}{}%
\end{pgfscope}%
\begin{pgfscope}%
\pgfsys@transformshift{0.851399in}{1.954185in}%
\pgfsys@useobject{currentmarker}{}%
\end{pgfscope}%
\begin{pgfscope}%
\pgfsys@transformshift{1.630362in}{2.014989in}%
\pgfsys@useobject{currentmarker}{}%
\end{pgfscope}%
\begin{pgfscope}%
\pgfsys@transformshift{0.497926in}{1.952627in}%
\pgfsys@useobject{currentmarker}{}%
\end{pgfscope}%
\begin{pgfscope}%
\pgfsys@transformshift{1.716622in}{1.871137in}%
\pgfsys@useobject{currentmarker}{}%
\end{pgfscope}%
\begin{pgfscope}%
\pgfsys@transformshift{0.524809in}{2.011034in}%
\pgfsys@useobject{currentmarker}{}%
\end{pgfscope}%
\begin{pgfscope}%
\pgfsys@transformshift{1.821229in}{2.072413in}%
\pgfsys@useobject{currentmarker}{}%
\end{pgfscope}%
\begin{pgfscope}%
\pgfsys@transformshift{0.925546in}{1.969361in}%
\pgfsys@useobject{currentmarker}{}%
\end{pgfscope}%
\begin{pgfscope}%
\pgfsys@transformshift{1.841684in}{1.855177in}%
\pgfsys@useobject{currentmarker}{}%
\end{pgfscope}%
\begin{pgfscope}%
\pgfsys@transformshift{1.522493in}{2.000107in}%
\pgfsys@useobject{currentmarker}{}%
\end{pgfscope}%
\begin{pgfscope}%
\pgfsys@transformshift{0.409285in}{2.008431in}%
\pgfsys@useobject{currentmarker}{}%
\end{pgfscope}%
\begin{pgfscope}%
\pgfsys@transformshift{0.896047in}{2.028268in}%
\pgfsys@useobject{currentmarker}{}%
\end{pgfscope}%
\begin{pgfscope}%
\pgfsys@transformshift{0.382462in}{1.982802in}%
\pgfsys@useobject{currentmarker}{}%
\end{pgfscope}%
\begin{pgfscope}%
\pgfsys@transformshift{0.855764in}{1.985058in}%
\pgfsys@useobject{currentmarker}{}%
\end{pgfscope}%
\begin{pgfscope}%
\pgfsys@transformshift{0.523733in}{1.855242in}%
\pgfsys@useobject{currentmarker}{}%
\end{pgfscope}%
\begin{pgfscope}%
\pgfsys@transformshift{0.470197in}{2.039092in}%
\pgfsys@useobject{currentmarker}{}%
\end{pgfscope}%
\begin{pgfscope}%
\pgfsys@transformshift{0.829991in}{1.956434in}%
\pgfsys@useobject{currentmarker}{}%
\end{pgfscope}%
\begin{pgfscope}%
\pgfsys@transformshift{0.933266in}{1.824049in}%
\pgfsys@useobject{currentmarker}{}%
\end{pgfscope}%
\begin{pgfscope}%
\pgfsys@transformshift{0.956789in}{2.033957in}%
\pgfsys@useobject{currentmarker}{}%
\end{pgfscope}%
\begin{pgfscope}%
\pgfsys@transformshift{0.936472in}{1.981897in}%
\pgfsys@useobject{currentmarker}{}%
\end{pgfscope}%
\begin{pgfscope}%
\pgfsys@transformshift{0.468794in}{2.021388in}%
\pgfsys@useobject{currentmarker}{}%
\end{pgfscope}%
\begin{pgfscope}%
\pgfsys@transformshift{0.402384in}{2.043176in}%
\pgfsys@useobject{currentmarker}{}%
\end{pgfscope}%
\begin{pgfscope}%
\pgfsys@transformshift{1.177456in}{1.958459in}%
\pgfsys@useobject{currentmarker}{}%
\end{pgfscope}%
\begin{pgfscope}%
\pgfsys@transformshift{0.416018in}{1.864783in}%
\pgfsys@useobject{currentmarker}{}%
\end{pgfscope}%
\begin{pgfscope}%
\pgfsys@transformshift{0.447008in}{2.036452in}%
\pgfsys@useobject{currentmarker}{}%
\end{pgfscope}%
\begin{pgfscope}%
\pgfsys@transformshift{0.549911in}{1.912859in}%
\pgfsys@useobject{currentmarker}{}%
\end{pgfscope}%
\begin{pgfscope}%
\pgfsys@transformshift{1.065128in}{2.064257in}%
\pgfsys@useobject{currentmarker}{}%
\end{pgfscope}%
\begin{pgfscope}%
\pgfsys@transformshift{0.450053in}{1.917745in}%
\pgfsys@useobject{currentmarker}{}%
\end{pgfscope}%
\begin{pgfscope}%
\pgfsys@transformshift{0.455142in}{1.896500in}%
\pgfsys@useobject{currentmarker}{}%
\end{pgfscope}%
\begin{pgfscope}%
\pgfsys@transformshift{0.484478in}{1.892179in}%
\pgfsys@useobject{currentmarker}{}%
\end{pgfscope}%
\begin{pgfscope}%
\pgfsys@transformshift{1.875283in}{2.044435in}%
\pgfsys@useobject{currentmarker}{}%
\end{pgfscope}%
\begin{pgfscope}%
\pgfsys@transformshift{0.375023in}{1.932976in}%
\pgfsys@useobject{currentmarker}{}%
\end{pgfscope}%
\begin{pgfscope}%
\pgfsys@transformshift{0.853143in}{1.971618in}%
\pgfsys@useobject{currentmarker}{}%
\end{pgfscope}%
\begin{pgfscope}%
\pgfsys@transformshift{0.436365in}{1.974012in}%
\pgfsys@useobject{currentmarker}{}%
\end{pgfscope}%
\begin{pgfscope}%
\pgfsys@transformshift{0.491797in}{1.900355in}%
\pgfsys@useobject{currentmarker}{}%
\end{pgfscope}%
\begin{pgfscope}%
\pgfsys@transformshift{0.902491in}{1.985718in}%
\pgfsys@useobject{currentmarker}{}%
\end{pgfscope}%
\begin{pgfscope}%
\pgfsys@transformshift{0.978414in}{1.958656in}%
\pgfsys@useobject{currentmarker}{}%
\end{pgfscope}%
\begin{pgfscope}%
\pgfsys@transformshift{0.944248in}{1.986369in}%
\pgfsys@useobject{currentmarker}{}%
\end{pgfscope}%
\begin{pgfscope}%
\pgfsys@transformshift{1.817503in}{1.807168in}%
\pgfsys@useobject{currentmarker}{}%
\end{pgfscope}%
\begin{pgfscope}%
\pgfsys@transformshift{1.721814in}{1.997032in}%
\pgfsys@useobject{currentmarker}{}%
\end{pgfscope}%
\begin{pgfscope}%
\pgfsys@transformshift{0.476494in}{1.998319in}%
\pgfsys@useobject{currentmarker}{}%
\end{pgfscope}%
\begin{pgfscope}%
\pgfsys@transformshift{0.942312in}{1.973383in}%
\pgfsys@useobject{currentmarker}{}%
\end{pgfscope}%
\begin{pgfscope}%
\pgfsys@transformshift{0.975555in}{2.055160in}%
\pgfsys@useobject{currentmarker}{}%
\end{pgfscope}%
\begin{pgfscope}%
\pgfsys@transformshift{1.642091in}{2.016878in}%
\pgfsys@useobject{currentmarker}{}%
\end{pgfscope}%
\begin{pgfscope}%
\pgfsys@transformshift{0.910804in}{1.957694in}%
\pgfsys@useobject{currentmarker}{}%
\end{pgfscope}%
\begin{pgfscope}%
\pgfsys@transformshift{0.507435in}{1.892127in}%
\pgfsys@useobject{currentmarker}{}%
\end{pgfscope}%
\begin{pgfscope}%
\pgfsys@transformshift{1.946413in}{1.816499in}%
\pgfsys@useobject{currentmarker}{}%
\end{pgfscope}%
\begin{pgfscope}%
\pgfsys@transformshift{1.639910in}{2.024456in}%
\pgfsys@useobject{currentmarker}{}%
\end{pgfscope}%
\begin{pgfscope}%
\pgfsys@transformshift{1.020077in}{2.023329in}%
\pgfsys@useobject{currentmarker}{}%
\end{pgfscope}%
\begin{pgfscope}%
\pgfsys@transformshift{1.837777in}{2.016484in}%
\pgfsys@useobject{currentmarker}{}%
\end{pgfscope}%
\begin{pgfscope}%
\pgfsys@transformshift{0.546434in}{1.867922in}%
\pgfsys@useobject{currentmarker}{}%
\end{pgfscope}%
\begin{pgfscope}%
\pgfsys@transformshift{0.507573in}{1.970320in}%
\pgfsys@useobject{currentmarker}{}%
\end{pgfscope}%
\begin{pgfscope}%
\pgfsys@transformshift{0.995047in}{1.959428in}%
\pgfsys@useobject{currentmarker}{}%
\end{pgfscope}%
\begin{pgfscope}%
\pgfsys@transformshift{0.507253in}{1.935437in}%
\pgfsys@useobject{currentmarker}{}%
\end{pgfscope}%
\begin{pgfscope}%
\pgfsys@transformshift{0.844423in}{2.042226in}%
\pgfsys@useobject{currentmarker}{}%
\end{pgfscope}%
\begin{pgfscope}%
\pgfsys@transformshift{1.175457in}{2.062096in}%
\pgfsys@useobject{currentmarker}{}%
\end{pgfscope}%
\begin{pgfscope}%
\pgfsys@transformshift{1.024973in}{1.892581in}%
\pgfsys@useobject{currentmarker}{}%
\end{pgfscope}%
\begin{pgfscope}%
\pgfsys@transformshift{1.150522in}{1.949093in}%
\pgfsys@useobject{currentmarker}{}%
\end{pgfscope}%
\begin{pgfscope}%
\pgfsys@transformshift{0.562256in}{2.012089in}%
\pgfsys@useobject{currentmarker}{}%
\end{pgfscope}%
\begin{pgfscope}%
\pgfsys@transformshift{1.482648in}{1.971849in}%
\pgfsys@useobject{currentmarker}{}%
\end{pgfscope}%
\begin{pgfscope}%
\pgfsys@transformshift{1.406514in}{2.010911in}%
\pgfsys@useobject{currentmarker}{}%
\end{pgfscope}%
\begin{pgfscope}%
\pgfsys@transformshift{1.080077in}{1.962338in}%
\pgfsys@useobject{currentmarker}{}%
\end{pgfscope}%
\begin{pgfscope}%
\pgfsys@transformshift{0.989979in}{2.052191in}%
\pgfsys@useobject{currentmarker}{}%
\end{pgfscope}%
\begin{pgfscope}%
\pgfsys@transformshift{0.987626in}{2.028019in}%
\pgfsys@useobject{currentmarker}{}%
\end{pgfscope}%
\begin{pgfscope}%
\pgfsys@transformshift{1.771689in}{2.072199in}%
\pgfsys@useobject{currentmarker}{}%
\end{pgfscope}%
\begin{pgfscope}%
\pgfsys@transformshift{0.891632in}{1.953362in}%
\pgfsys@useobject{currentmarker}{}%
\end{pgfscope}%
\begin{pgfscope}%
\pgfsys@transformshift{0.870836in}{2.012984in}%
\pgfsys@useobject{currentmarker}{}%
\end{pgfscope}%
\begin{pgfscope}%
\pgfsys@transformshift{1.667984in}{2.052091in}%
\pgfsys@useobject{currentmarker}{}%
\end{pgfscope}%
\begin{pgfscope}%
\pgfsys@transformshift{1.813990in}{2.084150in}%
\pgfsys@useobject{currentmarker}{}%
\end{pgfscope}%
\begin{pgfscope}%
\pgfsys@transformshift{0.403236in}{1.942209in}%
\pgfsys@useobject{currentmarker}{}%
\end{pgfscope}%
\begin{pgfscope}%
\pgfsys@transformshift{0.996537in}{2.043807in}%
\pgfsys@useobject{currentmarker}{}%
\end{pgfscope}%
\begin{pgfscope}%
\pgfsys@transformshift{1.043417in}{1.971641in}%
\pgfsys@useobject{currentmarker}{}%
\end{pgfscope}%
\begin{pgfscope}%
\pgfsys@transformshift{0.467968in}{1.858032in}%
\pgfsys@useobject{currentmarker}{}%
\end{pgfscope}%
\begin{pgfscope}%
\pgfsys@transformshift{1.915346in}{1.823510in}%
\pgfsys@useobject{currentmarker}{}%
\end{pgfscope}%
\begin{pgfscope}%
\pgfsys@transformshift{0.437600in}{2.009240in}%
\pgfsys@useobject{currentmarker}{}%
\end{pgfscope}%
\begin{pgfscope}%
\pgfsys@transformshift{0.438013in}{1.938684in}%
\pgfsys@useobject{currentmarker}{}%
\end{pgfscope}%
\begin{pgfscope}%
\pgfsys@transformshift{0.958228in}{1.801209in}%
\pgfsys@useobject{currentmarker}{}%
\end{pgfscope}%
\begin{pgfscope}%
\pgfsys@transformshift{1.012755in}{1.836123in}%
\pgfsys@useobject{currentmarker}{}%
\end{pgfscope}%
\begin{pgfscope}%
\pgfsys@transformshift{0.406225in}{1.871691in}%
\pgfsys@useobject{currentmarker}{}%
\end{pgfscope}%
\begin{pgfscope}%
\pgfsys@transformshift{0.473582in}{1.946951in}%
\pgfsys@useobject{currentmarker}{}%
\end{pgfscope}%
\begin{pgfscope}%
\pgfsys@transformshift{0.590756in}{1.844276in}%
\pgfsys@useobject{currentmarker}{}%
\end{pgfscope}%
\begin{pgfscope}%
\pgfsys@transformshift{0.918416in}{1.901615in}%
\pgfsys@useobject{currentmarker}{}%
\end{pgfscope}%
\begin{pgfscope}%
\pgfsys@transformshift{0.435231in}{1.961354in}%
\pgfsys@useobject{currentmarker}{}%
\end{pgfscope}%
\begin{pgfscope}%
\pgfsys@transformshift{1.100847in}{2.030241in}%
\pgfsys@useobject{currentmarker}{}%
\end{pgfscope}%
\begin{pgfscope}%
\pgfsys@transformshift{1.783869in}{2.073114in}%
\pgfsys@useobject{currentmarker}{}%
\end{pgfscope}%
\begin{pgfscope}%
\pgfsys@transformshift{0.881476in}{2.021972in}%
\pgfsys@useobject{currentmarker}{}%
\end{pgfscope}%
\begin{pgfscope}%
\pgfsys@transformshift{1.004126in}{1.860749in}%
\pgfsys@useobject{currentmarker}{}%
\end{pgfscope}%
\begin{pgfscope}%
\pgfsys@transformshift{1.494969in}{2.034785in}%
\pgfsys@useobject{currentmarker}{}%
\end{pgfscope}%
\begin{pgfscope}%
\pgfsys@transformshift{0.486970in}{1.831774in}%
\pgfsys@useobject{currentmarker}{}%
\end{pgfscope}%
\begin{pgfscope}%
\pgfsys@transformshift{0.439439in}{2.032296in}%
\pgfsys@useobject{currentmarker}{}%
\end{pgfscope}%
\begin{pgfscope}%
\pgfsys@transformshift{0.879557in}{1.981511in}%
\pgfsys@useobject{currentmarker}{}%
\end{pgfscope}%
\begin{pgfscope}%
\pgfsys@transformshift{0.852216in}{2.018595in}%
\pgfsys@useobject{currentmarker}{}%
\end{pgfscope}%
\begin{pgfscope}%
\pgfsys@transformshift{0.950782in}{2.045244in}%
\pgfsys@useobject{currentmarker}{}%
\end{pgfscope}%
\begin{pgfscope}%
\pgfsys@transformshift{0.502562in}{2.009367in}%
\pgfsys@useobject{currentmarker}{}%
\end{pgfscope}%
\begin{pgfscope}%
\pgfsys@transformshift{1.067095in}{2.005636in}%
\pgfsys@useobject{currentmarker}{}%
\end{pgfscope}%
\begin{pgfscope}%
\pgfsys@transformshift{0.478759in}{2.031151in}%
\pgfsys@useobject{currentmarker}{}%
\end{pgfscope}%
\begin{pgfscope}%
\pgfsys@transformshift{0.967133in}{2.011878in}%
\pgfsys@useobject{currentmarker}{}%
\end{pgfscope}%
\begin{pgfscope}%
\pgfsys@transformshift{0.508090in}{2.019576in}%
\pgfsys@useobject{currentmarker}{}%
\end{pgfscope}%
\begin{pgfscope}%
\pgfsys@transformshift{0.435594in}{1.905183in}%
\pgfsys@useobject{currentmarker}{}%
\end{pgfscope}%
\begin{pgfscope}%
\pgfsys@transformshift{0.980355in}{1.878627in}%
\pgfsys@useobject{currentmarker}{}%
\end{pgfscope}%
\begin{pgfscope}%
\pgfsys@transformshift{1.552149in}{2.009373in}%
\pgfsys@useobject{currentmarker}{}%
\end{pgfscope}%
\begin{pgfscope}%
\pgfsys@transformshift{0.875226in}{2.000884in}%
\pgfsys@useobject{currentmarker}{}%
\end{pgfscope}%
\begin{pgfscope}%
\pgfsys@transformshift{1.107273in}{2.019334in}%
\pgfsys@useobject{currentmarker}{}%
\end{pgfscope}%
\begin{pgfscope}%
\pgfsys@transformshift{0.523782in}{2.002489in}%
\pgfsys@useobject{currentmarker}{}%
\end{pgfscope}%
\begin{pgfscope}%
\pgfsys@transformshift{1.045306in}{2.008811in}%
\pgfsys@useobject{currentmarker}{}%
\end{pgfscope}%
\begin{pgfscope}%
\pgfsys@transformshift{1.526749in}{1.983528in}%
\pgfsys@useobject{currentmarker}{}%
\end{pgfscope}%
\begin{pgfscope}%
\pgfsys@transformshift{1.541666in}{2.057190in}%
\pgfsys@useobject{currentmarker}{}%
\end{pgfscope}%
\begin{pgfscope}%
\pgfsys@transformshift{1.574267in}{2.018599in}%
\pgfsys@useobject{currentmarker}{}%
\end{pgfscope}%
\begin{pgfscope}%
\pgfsys@transformshift{1.639614in}{2.023846in}%
\pgfsys@useobject{currentmarker}{}%
\end{pgfscope}%
\begin{pgfscope}%
\pgfsys@transformshift{0.881908in}{2.029191in}%
\pgfsys@useobject{currentmarker}{}%
\end{pgfscope}%
\begin{pgfscope}%
\pgfsys@transformshift{1.677526in}{2.053129in}%
\pgfsys@useobject{currentmarker}{}%
\end{pgfscope}%
\begin{pgfscope}%
\pgfsys@transformshift{0.483826in}{1.959344in}%
\pgfsys@useobject{currentmarker}{}%
\end{pgfscope}%
\begin{pgfscope}%
\pgfsys@transformshift{0.448297in}{1.987809in}%
\pgfsys@useobject{currentmarker}{}%
\end{pgfscope}%
\begin{pgfscope}%
\pgfsys@transformshift{0.976510in}{1.976583in}%
\pgfsys@useobject{currentmarker}{}%
\end{pgfscope}%
\begin{pgfscope}%
\pgfsys@transformshift{0.481727in}{1.976644in}%
\pgfsys@useobject{currentmarker}{}%
\end{pgfscope}%
\begin{pgfscope}%
\pgfsys@transformshift{1.808114in}{2.020556in}%
\pgfsys@useobject{currentmarker}{}%
\end{pgfscope}%
\begin{pgfscope}%
\pgfsys@transformshift{0.468390in}{1.883704in}%
\pgfsys@useobject{currentmarker}{}%
\end{pgfscope}%
\begin{pgfscope}%
\pgfsys@transformshift{0.458999in}{1.918727in}%
\pgfsys@useobject{currentmarker}{}%
\end{pgfscope}%
\begin{pgfscope}%
\pgfsys@transformshift{1.685818in}{1.994218in}%
\pgfsys@useobject{currentmarker}{}%
\end{pgfscope}%
\begin{pgfscope}%
\pgfsys@transformshift{0.635418in}{1.937019in}%
\pgfsys@useobject{currentmarker}{}%
\end{pgfscope}%
\begin{pgfscope}%
\pgfsys@transformshift{0.415977in}{1.867583in}%
\pgfsys@useobject{currentmarker}{}%
\end{pgfscope}%
\begin{pgfscope}%
\pgfsys@transformshift{0.341129in}{1.838886in}%
\pgfsys@useobject{currentmarker}{}%
\end{pgfscope}%
\begin{pgfscope}%
\pgfsys@transformshift{0.992976in}{2.010955in}%
\pgfsys@useobject{currentmarker}{}%
\end{pgfscope}%
\begin{pgfscope}%
\pgfsys@transformshift{0.883182in}{2.054321in}%
\pgfsys@useobject{currentmarker}{}%
\end{pgfscope}%
\begin{pgfscope}%
\pgfsys@transformshift{0.480881in}{2.026797in}%
\pgfsys@useobject{currentmarker}{}%
\end{pgfscope}%
\begin{pgfscope}%
\pgfsys@transformshift{1.244411in}{1.964549in}%
\pgfsys@useobject{currentmarker}{}%
\end{pgfscope}%
\begin{pgfscope}%
\pgfsys@transformshift{1.224203in}{1.980901in}%
\pgfsys@useobject{currentmarker}{}%
\end{pgfscope}%
\begin{pgfscope}%
\pgfsys@transformshift{0.449848in}{2.035481in}%
\pgfsys@useobject{currentmarker}{}%
\end{pgfscope}%
\begin{pgfscope}%
\pgfsys@transformshift{0.582527in}{1.852883in}%
\pgfsys@useobject{currentmarker}{}%
\end{pgfscope}%
\begin{pgfscope}%
\pgfsys@transformshift{1.714524in}{2.080492in}%
\pgfsys@useobject{currentmarker}{}%
\end{pgfscope}%
\begin{pgfscope}%
\pgfsys@transformshift{0.891264in}{1.975921in}%
\pgfsys@useobject{currentmarker}{}%
\end{pgfscope}%
\begin{pgfscope}%
\pgfsys@transformshift{1.857615in}{2.035096in}%
\pgfsys@useobject{currentmarker}{}%
\end{pgfscope}%
\begin{pgfscope}%
\pgfsys@transformshift{1.511327in}{2.032092in}%
\pgfsys@useobject{currentmarker}{}%
\end{pgfscope}%
\begin{pgfscope}%
\pgfsys@transformshift{1.748915in}{1.917796in}%
\pgfsys@useobject{currentmarker}{}%
\end{pgfscope}%
\begin{pgfscope}%
\pgfsys@transformshift{0.491937in}{1.956431in}%
\pgfsys@useobject{currentmarker}{}%
\end{pgfscope}%
\begin{pgfscope}%
\pgfsys@transformshift{0.556274in}{1.985023in}%
\pgfsys@useobject{currentmarker}{}%
\end{pgfscope}%
\begin{pgfscope}%
\pgfsys@transformshift{0.468664in}{2.039967in}%
\pgfsys@useobject{currentmarker}{}%
\end{pgfscope}%
\begin{pgfscope}%
\pgfsys@transformshift{1.550098in}{2.018460in}%
\pgfsys@useobject{currentmarker}{}%
\end{pgfscope}%
\begin{pgfscope}%
\pgfsys@transformshift{0.533554in}{1.988929in}%
\pgfsys@useobject{currentmarker}{}%
\end{pgfscope}%
\begin{pgfscope}%
\pgfsys@transformshift{1.009532in}{1.978883in}%
\pgfsys@useobject{currentmarker}{}%
\end{pgfscope}%
\begin{pgfscope}%
\pgfsys@transformshift{0.590179in}{1.920333in}%
\pgfsys@useobject{currentmarker}{}%
\end{pgfscope}%
\begin{pgfscope}%
\pgfsys@transformshift{0.432340in}{2.012681in}%
\pgfsys@useobject{currentmarker}{}%
\end{pgfscope}%
\begin{pgfscope}%
\pgfsys@transformshift{0.372906in}{1.929319in}%
\pgfsys@useobject{currentmarker}{}%
\end{pgfscope}%
\begin{pgfscope}%
\pgfsys@transformshift{0.459645in}{1.978767in}%
\pgfsys@useobject{currentmarker}{}%
\end{pgfscope}%
\begin{pgfscope}%
\pgfsys@transformshift{1.482135in}{1.966411in}%
\pgfsys@useobject{currentmarker}{}%
\end{pgfscope}%
\begin{pgfscope}%
\pgfsys@transformshift{0.622605in}{1.890328in}%
\pgfsys@useobject{currentmarker}{}%
\end{pgfscope}%
\begin{pgfscope}%
\pgfsys@transformshift{0.489705in}{1.964850in}%
\pgfsys@useobject{currentmarker}{}%
\end{pgfscope}%
\begin{pgfscope}%
\pgfsys@transformshift{1.102334in}{1.922019in}%
\pgfsys@useobject{currentmarker}{}%
\end{pgfscope}%
\begin{pgfscope}%
\pgfsys@transformshift{0.877887in}{1.985448in}%
\pgfsys@useobject{currentmarker}{}%
\end{pgfscope}%
\begin{pgfscope}%
\pgfsys@transformshift{0.519403in}{1.944886in}%
\pgfsys@useobject{currentmarker}{}%
\end{pgfscope}%
\begin{pgfscope}%
\pgfsys@transformshift{0.938030in}{2.010817in}%
\pgfsys@useobject{currentmarker}{}%
\end{pgfscope}%
\begin{pgfscope}%
\pgfsys@transformshift{0.961331in}{1.934212in}%
\pgfsys@useobject{currentmarker}{}%
\end{pgfscope}%
\begin{pgfscope}%
\pgfsys@transformshift{1.036665in}{2.053577in}%
\pgfsys@useobject{currentmarker}{}%
\end{pgfscope}%
\begin{pgfscope}%
\pgfsys@transformshift{0.628071in}{2.000401in}%
\pgfsys@useobject{currentmarker}{}%
\end{pgfscope}%
\begin{pgfscope}%
\pgfsys@transformshift{0.835422in}{2.017992in}%
\pgfsys@useobject{currentmarker}{}%
\end{pgfscope}%
\begin{pgfscope}%
\pgfsys@transformshift{1.066394in}{2.048400in}%
\pgfsys@useobject{currentmarker}{}%
\end{pgfscope}%
\begin{pgfscope}%
\pgfsys@transformshift{1.436938in}{1.968796in}%
\pgfsys@useobject{currentmarker}{}%
\end{pgfscope}%
\begin{pgfscope}%
\pgfsys@transformshift{1.603771in}{1.991051in}%
\pgfsys@useobject{currentmarker}{}%
\end{pgfscope}%
\begin{pgfscope}%
\pgfsys@transformshift{0.959028in}{2.031624in}%
\pgfsys@useobject{currentmarker}{}%
\end{pgfscope}%
\begin{pgfscope}%
\pgfsys@transformshift{0.432442in}{1.963267in}%
\pgfsys@useobject{currentmarker}{}%
\end{pgfscope}%
\begin{pgfscope}%
\pgfsys@transformshift{0.897917in}{1.974070in}%
\pgfsys@useobject{currentmarker}{}%
\end{pgfscope}%
\begin{pgfscope}%
\pgfsys@transformshift{0.894441in}{1.889290in}%
\pgfsys@useobject{currentmarker}{}%
\end{pgfscope}%
\begin{pgfscope}%
\pgfsys@transformshift{0.401578in}{2.002839in}%
\pgfsys@useobject{currentmarker}{}%
\end{pgfscope}%
\begin{pgfscope}%
\pgfsys@transformshift{0.411375in}{1.976238in}%
\pgfsys@useobject{currentmarker}{}%
\end{pgfscope}%
\begin{pgfscope}%
\pgfsys@transformshift{0.874473in}{1.998542in}%
\pgfsys@useobject{currentmarker}{}%
\end{pgfscope}%
\begin{pgfscope}%
\pgfsys@transformshift{0.989940in}{1.976915in}%
\pgfsys@useobject{currentmarker}{}%
\end{pgfscope}%
\begin{pgfscope}%
\pgfsys@transformshift{0.410574in}{1.965747in}%
\pgfsys@useobject{currentmarker}{}%
\end{pgfscope}%
\begin{pgfscope}%
\pgfsys@transformshift{0.887331in}{2.013639in}%
\pgfsys@useobject{currentmarker}{}%
\end{pgfscope}%
\begin{pgfscope}%
\pgfsys@transformshift{0.844030in}{2.004851in}%
\pgfsys@useobject{currentmarker}{}%
\end{pgfscope}%
\begin{pgfscope}%
\pgfsys@transformshift{0.923946in}{2.033458in}%
\pgfsys@useobject{currentmarker}{}%
\end{pgfscope}%
\begin{pgfscope}%
\pgfsys@transformshift{1.517373in}{2.013800in}%
\pgfsys@useobject{currentmarker}{}%
\end{pgfscope}%
\begin{pgfscope}%
\pgfsys@transformshift{1.544776in}{2.025531in}%
\pgfsys@useobject{currentmarker}{}%
\end{pgfscope}%
\begin{pgfscope}%
\pgfsys@transformshift{0.466736in}{1.864022in}%
\pgfsys@useobject{currentmarker}{}%
\end{pgfscope}%
\begin{pgfscope}%
\pgfsys@transformshift{0.539327in}{1.827491in}%
\pgfsys@useobject{currentmarker}{}%
\end{pgfscope}%
\begin{pgfscope}%
\pgfsys@transformshift{1.837625in}{2.051688in}%
\pgfsys@useobject{currentmarker}{}%
\end{pgfscope}%
\begin{pgfscope}%
\pgfsys@transformshift{0.442869in}{1.961095in}%
\pgfsys@useobject{currentmarker}{}%
\end{pgfscope}%
\begin{pgfscope}%
\pgfsys@transformshift{1.136179in}{2.051758in}%
\pgfsys@useobject{currentmarker}{}%
\end{pgfscope}%
\begin{pgfscope}%
\pgfsys@transformshift{0.540575in}{1.956472in}%
\pgfsys@useobject{currentmarker}{}%
\end{pgfscope}%
\begin{pgfscope}%
\pgfsys@transformshift{0.505594in}{1.920491in}%
\pgfsys@useobject{currentmarker}{}%
\end{pgfscope}%
\begin{pgfscope}%
\pgfsys@transformshift{0.464353in}{1.804968in}%
\pgfsys@useobject{currentmarker}{}%
\end{pgfscope}%
\begin{pgfscope}%
\pgfsys@transformshift{0.448024in}{1.972687in}%
\pgfsys@useobject{currentmarker}{}%
\end{pgfscope}%
\begin{pgfscope}%
\pgfsys@transformshift{1.133798in}{2.084452in}%
\pgfsys@useobject{currentmarker}{}%
\end{pgfscope}%
\begin{pgfscope}%
\pgfsys@transformshift{0.450472in}{1.940031in}%
\pgfsys@useobject{currentmarker}{}%
\end{pgfscope}%
\begin{pgfscope}%
\pgfsys@transformshift{0.915981in}{1.923742in}%
\pgfsys@useobject{currentmarker}{}%
\end{pgfscope}%
\begin{pgfscope}%
\pgfsys@transformshift{0.893518in}{1.946524in}%
\pgfsys@useobject{currentmarker}{}%
\end{pgfscope}%
\begin{pgfscope}%
\pgfsys@transformshift{0.839737in}{1.937529in}%
\pgfsys@useobject{currentmarker}{}%
\end{pgfscope}%
\begin{pgfscope}%
\pgfsys@transformshift{0.466023in}{1.919592in}%
\pgfsys@useobject{currentmarker}{}%
\end{pgfscope}%
\begin{pgfscope}%
\pgfsys@transformshift{1.451837in}{2.020622in}%
\pgfsys@useobject{currentmarker}{}%
\end{pgfscope}%
\begin{pgfscope}%
\pgfsys@transformshift{1.812241in}{1.876557in}%
\pgfsys@useobject{currentmarker}{}%
\end{pgfscope}%
\begin{pgfscope}%
\pgfsys@transformshift{0.470694in}{1.982610in}%
\pgfsys@useobject{currentmarker}{}%
\end{pgfscope}%
\begin{pgfscope}%
\pgfsys@transformshift{0.373662in}{1.985646in}%
\pgfsys@useobject{currentmarker}{}%
\end{pgfscope}%
\begin{pgfscope}%
\pgfsys@transformshift{0.403930in}{1.929366in}%
\pgfsys@useobject{currentmarker}{}%
\end{pgfscope}%
\begin{pgfscope}%
\pgfsys@transformshift{1.148968in}{2.080435in}%
\pgfsys@useobject{currentmarker}{}%
\end{pgfscope}%
\begin{pgfscope}%
\pgfsys@transformshift{0.466738in}{1.977282in}%
\pgfsys@useobject{currentmarker}{}%
\end{pgfscope}%
\begin{pgfscope}%
\pgfsys@transformshift{0.600280in}{2.026629in}%
\pgfsys@useobject{currentmarker}{}%
\end{pgfscope}%
\begin{pgfscope}%
\pgfsys@transformshift{0.430403in}{2.003002in}%
\pgfsys@useobject{currentmarker}{}%
\end{pgfscope}%
\begin{pgfscope}%
\pgfsys@transformshift{0.917645in}{1.933823in}%
\pgfsys@useobject{currentmarker}{}%
\end{pgfscope}%
\begin{pgfscope}%
\pgfsys@transformshift{1.972216in}{1.843439in}%
\pgfsys@useobject{currentmarker}{}%
\end{pgfscope}%
\begin{pgfscope}%
\pgfsys@transformshift{0.947192in}{1.953291in}%
\pgfsys@useobject{currentmarker}{}%
\end{pgfscope}%
\begin{pgfscope}%
\pgfsys@transformshift{0.945142in}{1.834715in}%
\pgfsys@useobject{currentmarker}{}%
\end{pgfscope}%
\begin{pgfscope}%
\pgfsys@transformshift{0.884523in}{1.976559in}%
\pgfsys@useobject{currentmarker}{}%
\end{pgfscope}%
\begin{pgfscope}%
\pgfsys@transformshift{0.868046in}{2.039586in}%
\pgfsys@useobject{currentmarker}{}%
\end{pgfscope}%
\begin{pgfscope}%
\pgfsys@transformshift{0.427823in}{1.961142in}%
\pgfsys@useobject{currentmarker}{}%
\end{pgfscope}%
\begin{pgfscope}%
\pgfsys@transformshift{1.240509in}{1.980366in}%
\pgfsys@useobject{currentmarker}{}%
\end{pgfscope}%
\begin{pgfscope}%
\pgfsys@transformshift{0.427367in}{2.074014in}%
\pgfsys@useobject{currentmarker}{}%
\end{pgfscope}%
\begin{pgfscope}%
\pgfsys@transformshift{0.877911in}{2.011605in}%
\pgfsys@useobject{currentmarker}{}%
\end{pgfscope}%
\begin{pgfscope}%
\pgfsys@transformshift{0.502381in}{1.860711in}%
\pgfsys@useobject{currentmarker}{}%
\end{pgfscope}%
\begin{pgfscope}%
\pgfsys@transformshift{0.897409in}{1.882994in}%
\pgfsys@useobject{currentmarker}{}%
\end{pgfscope}%
\begin{pgfscope}%
\pgfsys@transformshift{0.371513in}{1.858468in}%
\pgfsys@useobject{currentmarker}{}%
\end{pgfscope}%
\begin{pgfscope}%
\pgfsys@transformshift{0.424068in}{1.968746in}%
\pgfsys@useobject{currentmarker}{}%
\end{pgfscope}%
\begin{pgfscope}%
\pgfsys@transformshift{0.512683in}{1.758957in}%
\pgfsys@useobject{currentmarker}{}%
\end{pgfscope}%
\begin{pgfscope}%
\pgfsys@transformshift{0.423224in}{1.894487in}%
\pgfsys@useobject{currentmarker}{}%
\end{pgfscope}%
\begin{pgfscope}%
\pgfsys@transformshift{0.590216in}{1.890460in}%
\pgfsys@useobject{currentmarker}{}%
\end{pgfscope}%
\begin{pgfscope}%
\pgfsys@transformshift{1.015493in}{1.930369in}%
\pgfsys@useobject{currentmarker}{}%
\end{pgfscope}%
\begin{pgfscope}%
\pgfsys@transformshift{1.527374in}{2.025208in}%
\pgfsys@useobject{currentmarker}{}%
\end{pgfscope}%
\begin{pgfscope}%
\pgfsys@transformshift{1.794630in}{1.788124in}%
\pgfsys@useobject{currentmarker}{}%
\end{pgfscope}%
\begin{pgfscope}%
\pgfsys@transformshift{0.971721in}{1.987734in}%
\pgfsys@useobject{currentmarker}{}%
\end{pgfscope}%
\begin{pgfscope}%
\pgfsys@transformshift{0.402429in}{1.959504in}%
\pgfsys@useobject{currentmarker}{}%
\end{pgfscope}%
\begin{pgfscope}%
\pgfsys@transformshift{0.596209in}{1.908392in}%
\pgfsys@useobject{currentmarker}{}%
\end{pgfscope}%
\begin{pgfscope}%
\pgfsys@transformshift{0.596444in}{2.026270in}%
\pgfsys@useobject{currentmarker}{}%
\end{pgfscope}%
\begin{pgfscope}%
\pgfsys@transformshift{0.472945in}{2.016063in}%
\pgfsys@useobject{currentmarker}{}%
\end{pgfscope}%
\begin{pgfscope}%
\pgfsys@transformshift{0.505729in}{1.972211in}%
\pgfsys@useobject{currentmarker}{}%
\end{pgfscope}%
\begin{pgfscope}%
\pgfsys@transformshift{0.522392in}{1.962199in}%
\pgfsys@useobject{currentmarker}{}%
\end{pgfscope}%
\begin{pgfscope}%
\pgfsys@transformshift{0.958437in}{1.998790in}%
\pgfsys@useobject{currentmarker}{}%
\end{pgfscope}%
\begin{pgfscope}%
\pgfsys@transformshift{0.899312in}{2.004460in}%
\pgfsys@useobject{currentmarker}{}%
\end{pgfscope}%
\begin{pgfscope}%
\pgfsys@transformshift{0.477796in}{2.011428in}%
\pgfsys@useobject{currentmarker}{}%
\end{pgfscope}%
\begin{pgfscope}%
\pgfsys@transformshift{0.495644in}{1.967929in}%
\pgfsys@useobject{currentmarker}{}%
\end{pgfscope}%
\begin{pgfscope}%
\pgfsys@transformshift{0.461404in}{2.031057in}%
\pgfsys@useobject{currentmarker}{}%
\end{pgfscope}%
\begin{pgfscope}%
\pgfsys@transformshift{0.564846in}{1.990888in}%
\pgfsys@useobject{currentmarker}{}%
\end{pgfscope}%
\begin{pgfscope}%
\pgfsys@transformshift{0.438201in}{1.854848in}%
\pgfsys@useobject{currentmarker}{}%
\end{pgfscope}%
\begin{pgfscope}%
\pgfsys@transformshift{0.830917in}{2.011920in}%
\pgfsys@useobject{currentmarker}{}%
\end{pgfscope}%
\begin{pgfscope}%
\pgfsys@transformshift{0.470836in}{1.922485in}%
\pgfsys@useobject{currentmarker}{}%
\end{pgfscope}%
\begin{pgfscope}%
\pgfsys@transformshift{1.732430in}{2.036213in}%
\pgfsys@useobject{currentmarker}{}%
\end{pgfscope}%
\begin{pgfscope}%
\pgfsys@transformshift{0.880194in}{1.950029in}%
\pgfsys@useobject{currentmarker}{}%
\end{pgfscope}%
\begin{pgfscope}%
\pgfsys@transformshift{1.771020in}{1.855380in}%
\pgfsys@useobject{currentmarker}{}%
\end{pgfscope}%
\begin{pgfscope}%
\pgfsys@transformshift{0.408541in}{2.025881in}%
\pgfsys@useobject{currentmarker}{}%
\end{pgfscope}%
\begin{pgfscope}%
\pgfsys@transformshift{1.783013in}{1.858535in}%
\pgfsys@useobject{currentmarker}{}%
\end{pgfscope}%
\begin{pgfscope}%
\pgfsys@transformshift{1.734433in}{1.868614in}%
\pgfsys@useobject{currentmarker}{}%
\end{pgfscope}%
\begin{pgfscope}%
\pgfsys@transformshift{1.772028in}{2.033503in}%
\pgfsys@useobject{currentmarker}{}%
\end{pgfscope}%
\begin{pgfscope}%
\pgfsys@transformshift{0.482187in}{2.028166in}%
\pgfsys@useobject{currentmarker}{}%
\end{pgfscope}%
\begin{pgfscope}%
\pgfsys@transformshift{0.850634in}{1.866622in}%
\pgfsys@useobject{currentmarker}{}%
\end{pgfscope}%
\begin{pgfscope}%
\pgfsys@transformshift{1.489929in}{1.975941in}%
\pgfsys@useobject{currentmarker}{}%
\end{pgfscope}%
\begin{pgfscope}%
\pgfsys@transformshift{0.572833in}{1.873845in}%
\pgfsys@useobject{currentmarker}{}%
\end{pgfscope}%
\begin{pgfscope}%
\pgfsys@transformshift{0.459424in}{1.965924in}%
\pgfsys@useobject{currentmarker}{}%
\end{pgfscope}%
\begin{pgfscope}%
\pgfsys@transformshift{0.454844in}{1.972779in}%
\pgfsys@useobject{currentmarker}{}%
\end{pgfscope}%
\begin{pgfscope}%
\pgfsys@transformshift{1.529823in}{2.013721in}%
\pgfsys@useobject{currentmarker}{}%
\end{pgfscope}%
\begin{pgfscope}%
\pgfsys@transformshift{0.477367in}{2.042646in}%
\pgfsys@useobject{currentmarker}{}%
\end{pgfscope}%
\begin{pgfscope}%
\pgfsys@transformshift{1.049797in}{1.990806in}%
\pgfsys@useobject{currentmarker}{}%
\end{pgfscope}%
\begin{pgfscope}%
\pgfsys@transformshift{1.390925in}{2.087689in}%
\pgfsys@useobject{currentmarker}{}%
\end{pgfscope}%
\begin{pgfscope}%
\pgfsys@transformshift{1.615976in}{1.868969in}%
\pgfsys@useobject{currentmarker}{}%
\end{pgfscope}%
\begin{pgfscope}%
\pgfsys@transformshift{1.879685in}{1.885886in}%
\pgfsys@useobject{currentmarker}{}%
\end{pgfscope}%
\begin{pgfscope}%
\pgfsys@transformshift{0.474522in}{1.970276in}%
\pgfsys@useobject{currentmarker}{}%
\end{pgfscope}%
\begin{pgfscope}%
\pgfsys@transformshift{0.508097in}{2.039510in}%
\pgfsys@useobject{currentmarker}{}%
\end{pgfscope}%
\begin{pgfscope}%
\pgfsys@transformshift{0.846995in}{1.943929in}%
\pgfsys@useobject{currentmarker}{}%
\end{pgfscope}%
\begin{pgfscope}%
\pgfsys@transformshift{1.720538in}{2.025934in}%
\pgfsys@useobject{currentmarker}{}%
\end{pgfscope}%
\begin{pgfscope}%
\pgfsys@transformshift{1.561110in}{1.977443in}%
\pgfsys@useobject{currentmarker}{}%
\end{pgfscope}%
\begin{pgfscope}%
\pgfsys@transformshift{0.477188in}{2.019628in}%
\pgfsys@useobject{currentmarker}{}%
\end{pgfscope}%
\begin{pgfscope}%
\pgfsys@transformshift{0.462555in}{2.022772in}%
\pgfsys@useobject{currentmarker}{}%
\end{pgfscope}%
\begin{pgfscope}%
\pgfsys@transformshift{1.923906in}{2.016107in}%
\pgfsys@useobject{currentmarker}{}%
\end{pgfscope}%
\begin{pgfscope}%
\pgfsys@transformshift{0.990232in}{2.027617in}%
\pgfsys@useobject{currentmarker}{}%
\end{pgfscope}%
\begin{pgfscope}%
\pgfsys@transformshift{0.858173in}{1.979579in}%
\pgfsys@useobject{currentmarker}{}%
\end{pgfscope}%
\begin{pgfscope}%
\pgfsys@transformshift{0.955286in}{2.076142in}%
\pgfsys@useobject{currentmarker}{}%
\end{pgfscope}%
\begin{pgfscope}%
\pgfsys@transformshift{1.614084in}{2.036986in}%
\pgfsys@useobject{currentmarker}{}%
\end{pgfscope}%
\begin{pgfscope}%
\pgfsys@transformshift{0.839480in}{2.014193in}%
\pgfsys@useobject{currentmarker}{}%
\end{pgfscope}%
\begin{pgfscope}%
\pgfsys@transformshift{0.579938in}{1.982712in}%
\pgfsys@useobject{currentmarker}{}%
\end{pgfscope}%
\begin{pgfscope}%
\pgfsys@transformshift{1.021308in}{1.985571in}%
\pgfsys@useobject{currentmarker}{}%
\end{pgfscope}%
\begin{pgfscope}%
\pgfsys@transformshift{0.489542in}{2.040895in}%
\pgfsys@useobject{currentmarker}{}%
\end{pgfscope}%
\begin{pgfscope}%
\pgfsys@transformshift{0.482496in}{1.892284in}%
\pgfsys@useobject{currentmarker}{}%
\end{pgfscope}%
\begin{pgfscope}%
\pgfsys@transformshift{1.231263in}{1.971439in}%
\pgfsys@useobject{currentmarker}{}%
\end{pgfscope}%
\begin{pgfscope}%
\pgfsys@transformshift{0.612346in}{1.888968in}%
\pgfsys@useobject{currentmarker}{}%
\end{pgfscope}%
\begin{pgfscope}%
\pgfsys@transformshift{0.596517in}{1.825974in}%
\pgfsys@useobject{currentmarker}{}%
\end{pgfscope}%
\begin{pgfscope}%
\pgfsys@transformshift{1.011703in}{1.862801in}%
\pgfsys@useobject{currentmarker}{}%
\end{pgfscope}%
\begin{pgfscope}%
\pgfsys@transformshift{0.536110in}{1.817317in}%
\pgfsys@useobject{currentmarker}{}%
\end{pgfscope}%
\begin{pgfscope}%
\pgfsys@transformshift{0.473641in}{1.991567in}%
\pgfsys@useobject{currentmarker}{}%
\end{pgfscope}%
\begin{pgfscope}%
\pgfsys@transformshift{0.949604in}{1.887769in}%
\pgfsys@useobject{currentmarker}{}%
\end{pgfscope}%
\begin{pgfscope}%
\pgfsys@transformshift{1.891487in}{2.044231in}%
\pgfsys@useobject{currentmarker}{}%
\end{pgfscope}%
\begin{pgfscope}%
\pgfsys@transformshift{0.459142in}{2.014571in}%
\pgfsys@useobject{currentmarker}{}%
\end{pgfscope}%
\begin{pgfscope}%
\pgfsys@transformshift{0.993745in}{1.977051in}%
\pgfsys@useobject{currentmarker}{}%
\end{pgfscope}%
\begin{pgfscope}%
\pgfsys@transformshift{0.461010in}{2.013045in}%
\pgfsys@useobject{currentmarker}{}%
\end{pgfscope}%
\begin{pgfscope}%
\pgfsys@transformshift{0.452198in}{2.004363in}%
\pgfsys@useobject{currentmarker}{}%
\end{pgfscope}%
\begin{pgfscope}%
\pgfsys@transformshift{0.567084in}{1.994910in}%
\pgfsys@useobject{currentmarker}{}%
\end{pgfscope}%
\begin{pgfscope}%
\pgfsys@transformshift{0.925975in}{1.995046in}%
\pgfsys@useobject{currentmarker}{}%
\end{pgfscope}%
\begin{pgfscope}%
\pgfsys@transformshift{1.694241in}{1.787065in}%
\pgfsys@useobject{currentmarker}{}%
\end{pgfscope}%
\begin{pgfscope}%
\pgfsys@transformshift{0.589923in}{1.886487in}%
\pgfsys@useobject{currentmarker}{}%
\end{pgfscope}%
\begin{pgfscope}%
\pgfsys@transformshift{1.421991in}{1.993709in}%
\pgfsys@useobject{currentmarker}{}%
\end{pgfscope}%
\begin{pgfscope}%
\pgfsys@transformshift{0.569884in}{1.982389in}%
\pgfsys@useobject{currentmarker}{}%
\end{pgfscope}%
\begin{pgfscope}%
\pgfsys@transformshift{0.481143in}{1.683863in}%
\pgfsys@useobject{currentmarker}{}%
\end{pgfscope}%
\begin{pgfscope}%
\pgfsys@transformshift{0.598950in}{1.791169in}%
\pgfsys@useobject{currentmarker}{}%
\end{pgfscope}%
\begin{pgfscope}%
\pgfsys@transformshift{0.891632in}{1.916029in}%
\pgfsys@useobject{currentmarker}{}%
\end{pgfscope}%
\begin{pgfscope}%
\pgfsys@transformshift{1.518904in}{2.036051in}%
\pgfsys@useobject{currentmarker}{}%
\end{pgfscope}%
\begin{pgfscope}%
\pgfsys@transformshift{1.267712in}{1.979793in}%
\pgfsys@useobject{currentmarker}{}%
\end{pgfscope}%
\begin{pgfscope}%
\pgfsys@transformshift{1.736581in}{2.100411in}%
\pgfsys@useobject{currentmarker}{}%
\end{pgfscope}%
\begin{pgfscope}%
\pgfsys@transformshift{0.507870in}{1.987209in}%
\pgfsys@useobject{currentmarker}{}%
\end{pgfscope}%
\begin{pgfscope}%
\pgfsys@transformshift{0.876979in}{2.043224in}%
\pgfsys@useobject{currentmarker}{}%
\end{pgfscope}%
\begin{pgfscope}%
\pgfsys@transformshift{0.975052in}{2.053956in}%
\pgfsys@useobject{currentmarker}{}%
\end{pgfscope}%
\begin{pgfscope}%
\pgfsys@transformshift{0.581391in}{1.911221in}%
\pgfsys@useobject{currentmarker}{}%
\end{pgfscope}%
\begin{pgfscope}%
\pgfsys@transformshift{0.432905in}{1.960019in}%
\pgfsys@useobject{currentmarker}{}%
\end{pgfscope}%
\begin{pgfscope}%
\pgfsys@transformshift{0.887125in}{2.007180in}%
\pgfsys@useobject{currentmarker}{}%
\end{pgfscope}%
\begin{pgfscope}%
\pgfsys@transformshift{0.887888in}{1.907374in}%
\pgfsys@useobject{currentmarker}{}%
\end{pgfscope}%
\begin{pgfscope}%
\pgfsys@transformshift{1.047522in}{2.011627in}%
\pgfsys@useobject{currentmarker}{}%
\end{pgfscope}%
\begin{pgfscope}%
\pgfsys@transformshift{0.459882in}{2.052760in}%
\pgfsys@useobject{currentmarker}{}%
\end{pgfscope}%
\begin{pgfscope}%
\pgfsys@transformshift{0.869741in}{1.902037in}%
\pgfsys@useobject{currentmarker}{}%
\end{pgfscope}%
\begin{pgfscope}%
\pgfsys@transformshift{1.519451in}{2.016570in}%
\pgfsys@useobject{currentmarker}{}%
\end{pgfscope}%
\begin{pgfscope}%
\pgfsys@transformshift{1.704307in}{2.049332in}%
\pgfsys@useobject{currentmarker}{}%
\end{pgfscope}%
\begin{pgfscope}%
\pgfsys@transformshift{1.029054in}{2.009591in}%
\pgfsys@useobject{currentmarker}{}%
\end{pgfscope}%
\begin{pgfscope}%
\pgfsys@transformshift{0.983134in}{2.053035in}%
\pgfsys@useobject{currentmarker}{}%
\end{pgfscope}%
\begin{pgfscope}%
\pgfsys@transformshift{1.345503in}{2.003394in}%
\pgfsys@useobject{currentmarker}{}%
\end{pgfscope}%
\begin{pgfscope}%
\pgfsys@transformshift{0.480154in}{1.993950in}%
\pgfsys@useobject{currentmarker}{}%
\end{pgfscope}%
\begin{pgfscope}%
\pgfsys@transformshift{0.820400in}{1.870010in}%
\pgfsys@useobject{currentmarker}{}%
\end{pgfscope}%
\begin{pgfscope}%
\pgfsys@transformshift{0.890253in}{2.009958in}%
\pgfsys@useobject{currentmarker}{}%
\end{pgfscope}%
\begin{pgfscope}%
\pgfsys@transformshift{1.194868in}{2.075354in}%
\pgfsys@useobject{currentmarker}{}%
\end{pgfscope}%
\begin{pgfscope}%
\pgfsys@transformshift{0.938368in}{2.039920in}%
\pgfsys@useobject{currentmarker}{}%
\end{pgfscope}%
\begin{pgfscope}%
\pgfsys@transformshift{0.663011in}{1.893021in}%
\pgfsys@useobject{currentmarker}{}%
\end{pgfscope}%
\begin{pgfscope}%
\pgfsys@transformshift{0.916322in}{1.883759in}%
\pgfsys@useobject{currentmarker}{}%
\end{pgfscope}%
\begin{pgfscope}%
\pgfsys@transformshift{0.890291in}{2.007840in}%
\pgfsys@useobject{currentmarker}{}%
\end{pgfscope}%
\begin{pgfscope}%
\pgfsys@transformshift{0.909236in}{1.926533in}%
\pgfsys@useobject{currentmarker}{}%
\end{pgfscope}%
\begin{pgfscope}%
\pgfsys@transformshift{0.862619in}{1.920397in}%
\pgfsys@useobject{currentmarker}{}%
\end{pgfscope}%
\begin{pgfscope}%
\pgfsys@transformshift{0.597487in}{2.009900in}%
\pgfsys@useobject{currentmarker}{}%
\end{pgfscope}%
\begin{pgfscope}%
\pgfsys@transformshift{1.964345in}{1.846606in}%
\pgfsys@useobject{currentmarker}{}%
\end{pgfscope}%
\begin{pgfscope}%
\pgfsys@transformshift{1.692774in}{1.841912in}%
\pgfsys@useobject{currentmarker}{}%
\end{pgfscope}%
\begin{pgfscope}%
\pgfsys@transformshift{0.482531in}{1.989668in}%
\pgfsys@useobject{currentmarker}{}%
\end{pgfscope}%
\begin{pgfscope}%
\pgfsys@transformshift{0.804579in}{1.908110in}%
\pgfsys@useobject{currentmarker}{}%
\end{pgfscope}%
\begin{pgfscope}%
\pgfsys@transformshift{0.458204in}{1.955067in}%
\pgfsys@useobject{currentmarker}{}%
\end{pgfscope}%
\begin{pgfscope}%
\pgfsys@transformshift{1.029005in}{1.961944in}%
\pgfsys@useobject{currentmarker}{}%
\end{pgfscope}%
\begin{pgfscope}%
\pgfsys@transformshift{0.509849in}{1.896262in}%
\pgfsys@useobject{currentmarker}{}%
\end{pgfscope}%
\begin{pgfscope}%
\pgfsys@transformshift{1.034049in}{1.923603in}%
\pgfsys@useobject{currentmarker}{}%
\end{pgfscope}%
\begin{pgfscope}%
\pgfsys@transformshift{1.152583in}{2.076896in}%
\pgfsys@useobject{currentmarker}{}%
\end{pgfscope}%
\begin{pgfscope}%
\pgfsys@transformshift{1.044006in}{2.031916in}%
\pgfsys@useobject{currentmarker}{}%
\end{pgfscope}%
\begin{pgfscope}%
\pgfsys@transformshift{1.486794in}{2.068833in}%
\pgfsys@useobject{currentmarker}{}%
\end{pgfscope}%
\begin{pgfscope}%
\pgfsys@transformshift{0.870195in}{1.965489in}%
\pgfsys@useobject{currentmarker}{}%
\end{pgfscope}%
\begin{pgfscope}%
\pgfsys@transformshift{1.138693in}{2.025991in}%
\pgfsys@useobject{currentmarker}{}%
\end{pgfscope}%
\begin{pgfscope}%
\pgfsys@transformshift{0.847963in}{1.889322in}%
\pgfsys@useobject{currentmarker}{}%
\end{pgfscope}%
\begin{pgfscope}%
\pgfsys@transformshift{0.477983in}{1.972418in}%
\pgfsys@useobject{currentmarker}{}%
\end{pgfscope}%
\begin{pgfscope}%
\pgfsys@transformshift{0.492721in}{1.922245in}%
\pgfsys@useobject{currentmarker}{}%
\end{pgfscope}%
\begin{pgfscope}%
\pgfsys@transformshift{1.120030in}{2.062219in}%
\pgfsys@useobject{currentmarker}{}%
\end{pgfscope}%
\begin{pgfscope}%
\pgfsys@transformshift{0.895919in}{2.074103in}%
\pgfsys@useobject{currentmarker}{}%
\end{pgfscope}%
\begin{pgfscope}%
\pgfsys@transformshift{0.576124in}{1.862847in}%
\pgfsys@useobject{currentmarker}{}%
\end{pgfscope}%
\begin{pgfscope}%
\pgfsys@transformshift{0.471843in}{1.905948in}%
\pgfsys@useobject{currentmarker}{}%
\end{pgfscope}%
\begin{pgfscope}%
\pgfsys@transformshift{0.475316in}{1.893359in}%
\pgfsys@useobject{currentmarker}{}%
\end{pgfscope}%
\begin{pgfscope}%
\pgfsys@transformshift{1.653595in}{2.036840in}%
\pgfsys@useobject{currentmarker}{}%
\end{pgfscope}%
\begin{pgfscope}%
\pgfsys@transformshift{1.083484in}{2.050828in}%
\pgfsys@useobject{currentmarker}{}%
\end{pgfscope}%
\begin{pgfscope}%
\pgfsys@transformshift{0.929704in}{1.970788in}%
\pgfsys@useobject{currentmarker}{}%
\end{pgfscope}%
\begin{pgfscope}%
\pgfsys@transformshift{1.702011in}{2.005835in}%
\pgfsys@useobject{currentmarker}{}%
\end{pgfscope}%
\begin{pgfscope}%
\pgfsys@transformshift{0.861736in}{1.891492in}%
\pgfsys@useobject{currentmarker}{}%
\end{pgfscope}%
\begin{pgfscope}%
\pgfsys@transformshift{0.466758in}{1.935730in}%
\pgfsys@useobject{currentmarker}{}%
\end{pgfscope}%
\begin{pgfscope}%
\pgfsys@transformshift{0.895642in}{2.036040in}%
\pgfsys@useobject{currentmarker}{}%
\end{pgfscope}%
\begin{pgfscope}%
\pgfsys@transformshift{0.844526in}{1.996290in}%
\pgfsys@useobject{currentmarker}{}%
\end{pgfscope}%
\begin{pgfscope}%
\pgfsys@transformshift{0.903057in}{2.014994in}%
\pgfsys@useobject{currentmarker}{}%
\end{pgfscope}%
\begin{pgfscope}%
\pgfsys@transformshift{0.952685in}{2.039578in}%
\pgfsys@useobject{currentmarker}{}%
\end{pgfscope}%
\begin{pgfscope}%
\pgfsys@transformshift{0.576385in}{2.002018in}%
\pgfsys@useobject{currentmarker}{}%
\end{pgfscope}%
\begin{pgfscope}%
\pgfsys@transformshift{1.016020in}{1.987024in}%
\pgfsys@useobject{currentmarker}{}%
\end{pgfscope}%
\begin{pgfscope}%
\pgfsys@transformshift{0.494156in}{1.828227in}%
\pgfsys@useobject{currentmarker}{}%
\end{pgfscope}%
\begin{pgfscope}%
\pgfsys@transformshift{0.577397in}{1.886876in}%
\pgfsys@useobject{currentmarker}{}%
\end{pgfscope}%
\begin{pgfscope}%
\pgfsys@transformshift{0.445519in}{1.894889in}%
\pgfsys@useobject{currentmarker}{}%
\end{pgfscope}%
\begin{pgfscope}%
\pgfsys@transformshift{1.084282in}{1.940648in}%
\pgfsys@useobject{currentmarker}{}%
\end{pgfscope}%
\begin{pgfscope}%
\pgfsys@transformshift{1.594805in}{2.010727in}%
\pgfsys@useobject{currentmarker}{}%
\end{pgfscope}%
\begin{pgfscope}%
\pgfsys@transformshift{0.976071in}{2.011002in}%
\pgfsys@useobject{currentmarker}{}%
\end{pgfscope}%
\begin{pgfscope}%
\pgfsys@transformshift{0.954554in}{2.007710in}%
\pgfsys@useobject{currentmarker}{}%
\end{pgfscope}%
\begin{pgfscope}%
\pgfsys@transformshift{1.062412in}{2.001093in}%
\pgfsys@useobject{currentmarker}{}%
\end{pgfscope}%
\begin{pgfscope}%
\pgfsys@transformshift{0.403708in}{1.797011in}%
\pgfsys@useobject{currentmarker}{}%
\end{pgfscope}%
\begin{pgfscope}%
\pgfsys@transformshift{0.456382in}{1.956665in}%
\pgfsys@useobject{currentmarker}{}%
\end{pgfscope}%
\begin{pgfscope}%
\pgfsys@transformshift{0.425389in}{1.849496in}%
\pgfsys@useobject{currentmarker}{}%
\end{pgfscope}%
\begin{pgfscope}%
\pgfsys@transformshift{1.178319in}{1.871213in}%
\pgfsys@useobject{currentmarker}{}%
\end{pgfscope}%
\begin{pgfscope}%
\pgfsys@transformshift{0.868624in}{2.001621in}%
\pgfsys@useobject{currentmarker}{}%
\end{pgfscope}%
\begin{pgfscope}%
\pgfsys@transformshift{1.072938in}{1.958615in}%
\pgfsys@useobject{currentmarker}{}%
\end{pgfscope}%
\begin{pgfscope}%
\pgfsys@transformshift{0.842975in}{1.940849in}%
\pgfsys@useobject{currentmarker}{}%
\end{pgfscope}%
\begin{pgfscope}%
\pgfsys@transformshift{1.133541in}{2.048109in}%
\pgfsys@useobject{currentmarker}{}%
\end{pgfscope}%
\begin{pgfscope}%
\pgfsys@transformshift{0.985168in}{2.062582in}%
\pgfsys@useobject{currentmarker}{}%
\end{pgfscope}%
\begin{pgfscope}%
\pgfsys@transformshift{0.935163in}{2.049666in}%
\pgfsys@useobject{currentmarker}{}%
\end{pgfscope}%
\begin{pgfscope}%
\pgfsys@transformshift{0.852856in}{1.961811in}%
\pgfsys@useobject{currentmarker}{}%
\end{pgfscope}%
\begin{pgfscope}%
\pgfsys@transformshift{0.481479in}{1.911043in}%
\pgfsys@useobject{currentmarker}{}%
\end{pgfscope}%
\begin{pgfscope}%
\pgfsys@transformshift{0.483502in}{2.001655in}%
\pgfsys@useobject{currentmarker}{}%
\end{pgfscope}%
\begin{pgfscope}%
\pgfsys@transformshift{0.857024in}{1.879099in}%
\pgfsys@useobject{currentmarker}{}%
\end{pgfscope}%
\begin{pgfscope}%
\pgfsys@transformshift{0.541865in}{1.870736in}%
\pgfsys@useobject{currentmarker}{}%
\end{pgfscope}%
\begin{pgfscope}%
\pgfsys@transformshift{1.564506in}{2.050230in}%
\pgfsys@useobject{currentmarker}{}%
\end{pgfscope}%
\begin{pgfscope}%
\pgfsys@transformshift{0.926665in}{2.017250in}%
\pgfsys@useobject{currentmarker}{}%
\end{pgfscope}%
\begin{pgfscope}%
\pgfsys@transformshift{0.371504in}{1.965223in}%
\pgfsys@useobject{currentmarker}{}%
\end{pgfscope}%
\begin{pgfscope}%
\pgfsys@transformshift{0.869520in}{1.921908in}%
\pgfsys@useobject{currentmarker}{}%
\end{pgfscope}%
\begin{pgfscope}%
\pgfsys@transformshift{0.519844in}{1.932851in}%
\pgfsys@useobject{currentmarker}{}%
\end{pgfscope}%
\begin{pgfscope}%
\pgfsys@transformshift{0.608978in}{1.830993in}%
\pgfsys@useobject{currentmarker}{}%
\end{pgfscope}%
\begin{pgfscope}%
\pgfsys@transformshift{0.476814in}{1.912288in}%
\pgfsys@useobject{currentmarker}{}%
\end{pgfscope}%
\begin{pgfscope}%
\pgfsys@transformshift{0.933077in}{1.993333in}%
\pgfsys@useobject{currentmarker}{}%
\end{pgfscope}%
\begin{pgfscope}%
\pgfsys@transformshift{0.579351in}{1.963827in}%
\pgfsys@useobject{currentmarker}{}%
\end{pgfscope}%
\begin{pgfscope}%
\pgfsys@transformshift{0.459383in}{1.948637in}%
\pgfsys@useobject{currentmarker}{}%
\end{pgfscope}%
\begin{pgfscope}%
\pgfsys@transformshift{0.451992in}{1.890201in}%
\pgfsys@useobject{currentmarker}{}%
\end{pgfscope}%
\begin{pgfscope}%
\pgfsys@transformshift{0.448649in}{2.011148in}%
\pgfsys@useobject{currentmarker}{}%
\end{pgfscope}%
\begin{pgfscope}%
\pgfsys@transformshift{1.803636in}{2.064471in}%
\pgfsys@useobject{currentmarker}{}%
\end{pgfscope}%
\begin{pgfscope}%
\pgfsys@transformshift{1.680858in}{1.818542in}%
\pgfsys@useobject{currentmarker}{}%
\end{pgfscope}%
\begin{pgfscope}%
\pgfsys@transformshift{0.464450in}{1.979354in}%
\pgfsys@useobject{currentmarker}{}%
\end{pgfscope}%
\begin{pgfscope}%
\pgfsys@transformshift{0.467818in}{1.936108in}%
\pgfsys@useobject{currentmarker}{}%
\end{pgfscope}%
\begin{pgfscope}%
\pgfsys@transformshift{1.797702in}{2.083534in}%
\pgfsys@useobject{currentmarker}{}%
\end{pgfscope}%
\begin{pgfscope}%
\pgfsys@transformshift{1.800535in}{2.074620in}%
\pgfsys@useobject{currentmarker}{}%
\end{pgfscope}%
\begin{pgfscope}%
\pgfsys@transformshift{0.486053in}{1.936097in}%
\pgfsys@useobject{currentmarker}{}%
\end{pgfscope}%
\begin{pgfscope}%
\pgfsys@transformshift{0.604133in}{1.842328in}%
\pgfsys@useobject{currentmarker}{}%
\end{pgfscope}%
\begin{pgfscope}%
\pgfsys@transformshift{1.510193in}{1.989262in}%
\pgfsys@useobject{currentmarker}{}%
\end{pgfscope}%
\begin{pgfscope}%
\pgfsys@transformshift{0.969856in}{2.028947in}%
\pgfsys@useobject{currentmarker}{}%
\end{pgfscope}%
\begin{pgfscope}%
\pgfsys@transformshift{0.883908in}{2.031825in}%
\pgfsys@useobject{currentmarker}{}%
\end{pgfscope}%
\begin{pgfscope}%
\pgfsys@transformshift{0.538449in}{1.843638in}%
\pgfsys@useobject{currentmarker}{}%
\end{pgfscope}%
\begin{pgfscope}%
\pgfsys@transformshift{0.561398in}{1.900055in}%
\pgfsys@useobject{currentmarker}{}%
\end{pgfscope}%
\begin{pgfscope}%
\pgfsys@transformshift{0.984012in}{2.014873in}%
\pgfsys@useobject{currentmarker}{}%
\end{pgfscope}%
\begin{pgfscope}%
\pgfsys@transformshift{0.614232in}{1.996300in}%
\pgfsys@useobject{currentmarker}{}%
\end{pgfscope}%
\begin{pgfscope}%
\pgfsys@transformshift{1.178616in}{2.061807in}%
\pgfsys@useobject{currentmarker}{}%
\end{pgfscope}%
\begin{pgfscope}%
\pgfsys@transformshift{0.638320in}{1.902555in}%
\pgfsys@useobject{currentmarker}{}%
\end{pgfscope}%
\begin{pgfscope}%
\pgfsys@transformshift{1.571164in}{1.996646in}%
\pgfsys@useobject{currentmarker}{}%
\end{pgfscope}%
\begin{pgfscope}%
\pgfsys@transformshift{0.449499in}{1.984027in}%
\pgfsys@useobject{currentmarker}{}%
\end{pgfscope}%
\begin{pgfscope}%
\pgfsys@transformshift{1.000812in}{2.043058in}%
\pgfsys@useobject{currentmarker}{}%
\end{pgfscope}%
\begin{pgfscope}%
\pgfsys@transformshift{0.504036in}{1.954483in}%
\pgfsys@useobject{currentmarker}{}%
\end{pgfscope}%
\begin{pgfscope}%
\pgfsys@transformshift{1.192646in}{2.075238in}%
\pgfsys@useobject{currentmarker}{}%
\end{pgfscope}%
\begin{pgfscope}%
\pgfsys@transformshift{0.575539in}{1.827884in}%
\pgfsys@useobject{currentmarker}{}%
\end{pgfscope}%
\begin{pgfscope}%
\pgfsys@transformshift{0.517708in}{2.008348in}%
\pgfsys@useobject{currentmarker}{}%
\end{pgfscope}%
\begin{pgfscope}%
\pgfsys@transformshift{1.244210in}{1.821922in}%
\pgfsys@useobject{currentmarker}{}%
\end{pgfscope}%
\begin{pgfscope}%
\pgfsys@transformshift{0.615567in}{1.833762in}%
\pgfsys@useobject{currentmarker}{}%
\end{pgfscope}%
\begin{pgfscope}%
\pgfsys@transformshift{0.907672in}{1.987036in}%
\pgfsys@useobject{currentmarker}{}%
\end{pgfscope}%
\begin{pgfscope}%
\pgfsys@transformshift{2.000000in}{2.097321in}%
\pgfsys@useobject{currentmarker}{}%
\end{pgfscope}%
\begin{pgfscope}%
\pgfsys@transformshift{0.485263in}{1.966228in}%
\pgfsys@useobject{currentmarker}{}%
\end{pgfscope}%
\begin{pgfscope}%
\pgfsys@transformshift{0.433137in}{1.864650in}%
\pgfsys@useobject{currentmarker}{}%
\end{pgfscope}%
\begin{pgfscope}%
\pgfsys@transformshift{0.499449in}{2.023735in}%
\pgfsys@useobject{currentmarker}{}%
\end{pgfscope}%
\begin{pgfscope}%
\pgfsys@transformshift{0.868886in}{1.988674in}%
\pgfsys@useobject{currentmarker}{}%
\end{pgfscope}%
\begin{pgfscope}%
\pgfsys@transformshift{1.451865in}{2.053131in}%
\pgfsys@useobject{currentmarker}{}%
\end{pgfscope}%
\begin{pgfscope}%
\pgfsys@transformshift{0.534197in}{2.015209in}%
\pgfsys@useobject{currentmarker}{}%
\end{pgfscope}%
\begin{pgfscope}%
\pgfsys@transformshift{0.972750in}{2.013721in}%
\pgfsys@useobject{currentmarker}{}%
\end{pgfscope}%
\begin{pgfscope}%
\pgfsys@transformshift{0.996367in}{1.893217in}%
\pgfsys@useobject{currentmarker}{}%
\end{pgfscope}%
\begin{pgfscope}%
\pgfsys@transformshift{0.893513in}{1.930266in}%
\pgfsys@useobject{currentmarker}{}%
\end{pgfscope}%
\begin{pgfscope}%
\pgfsys@transformshift{0.481561in}{1.950838in}%
\pgfsys@useobject{currentmarker}{}%
\end{pgfscope}%
\begin{pgfscope}%
\pgfsys@transformshift{0.462083in}{1.969226in}%
\pgfsys@useobject{currentmarker}{}%
\end{pgfscope}%
\begin{pgfscope}%
\pgfsys@transformshift{0.947888in}{1.933776in}%
\pgfsys@useobject{currentmarker}{}%
\end{pgfscope}%
\begin{pgfscope}%
\pgfsys@transformshift{0.475131in}{1.979378in}%
\pgfsys@useobject{currentmarker}{}%
\end{pgfscope}%
\begin{pgfscope}%
\pgfsys@transformshift{0.944100in}{1.961379in}%
\pgfsys@useobject{currentmarker}{}%
\end{pgfscope}%
\begin{pgfscope}%
\pgfsys@transformshift{0.966481in}{1.969128in}%
\pgfsys@useobject{currentmarker}{}%
\end{pgfscope}%
\begin{pgfscope}%
\pgfsys@transformshift{1.021199in}{2.017833in}%
\pgfsys@useobject{currentmarker}{}%
\end{pgfscope}%
\begin{pgfscope}%
\pgfsys@transformshift{0.451598in}{2.003719in}%
\pgfsys@useobject{currentmarker}{}%
\end{pgfscope}%
\begin{pgfscope}%
\pgfsys@transformshift{0.461004in}{2.040366in}%
\pgfsys@useobject{currentmarker}{}%
\end{pgfscope}%
\begin{pgfscope}%
\pgfsys@transformshift{0.419847in}{1.963732in}%
\pgfsys@useobject{currentmarker}{}%
\end{pgfscope}%
\begin{pgfscope}%
\pgfsys@transformshift{0.498798in}{1.914104in}%
\pgfsys@useobject{currentmarker}{}%
\end{pgfscope}%
\begin{pgfscope}%
\pgfsys@transformshift{1.022778in}{1.938692in}%
\pgfsys@useobject{currentmarker}{}%
\end{pgfscope}%
\begin{pgfscope}%
\pgfsys@transformshift{1.618050in}{1.810593in}%
\pgfsys@useobject{currentmarker}{}%
\end{pgfscope}%
\begin{pgfscope}%
\pgfsys@transformshift{0.873499in}{2.005345in}%
\pgfsys@useobject{currentmarker}{}%
\end{pgfscope}%
\begin{pgfscope}%
\pgfsys@transformshift{1.717975in}{1.970817in}%
\pgfsys@useobject{currentmarker}{}%
\end{pgfscope}%
\begin{pgfscope}%
\pgfsys@transformshift{0.479704in}{2.028062in}%
\pgfsys@useobject{currentmarker}{}%
\end{pgfscope}%
\begin{pgfscope}%
\pgfsys@transformshift{0.921403in}{2.024912in}%
\pgfsys@useobject{currentmarker}{}%
\end{pgfscope}%
\begin{pgfscope}%
\pgfsys@transformshift{0.507519in}{1.959553in}%
\pgfsys@useobject{currentmarker}{}%
\end{pgfscope}%
\begin{pgfscope}%
\pgfsys@transformshift{1.242289in}{2.074117in}%
\pgfsys@useobject{currentmarker}{}%
\end{pgfscope}%
\begin{pgfscope}%
\pgfsys@transformshift{0.901785in}{1.970025in}%
\pgfsys@useobject{currentmarker}{}%
\end{pgfscope}%
\begin{pgfscope}%
\pgfsys@transformshift{1.303585in}{1.992254in}%
\pgfsys@useobject{currentmarker}{}%
\end{pgfscope}%
\begin{pgfscope}%
\pgfsys@transformshift{0.385710in}{1.989200in}%
\pgfsys@useobject{currentmarker}{}%
\end{pgfscope}%
\begin{pgfscope}%
\pgfsys@transformshift{0.487652in}{1.932830in}%
\pgfsys@useobject{currentmarker}{}%
\end{pgfscope}%
\begin{pgfscope}%
\pgfsys@transformshift{0.427840in}{1.956796in}%
\pgfsys@useobject{currentmarker}{}%
\end{pgfscope}%
\begin{pgfscope}%
\pgfsys@transformshift{0.479919in}{1.972791in}%
\pgfsys@useobject{currentmarker}{}%
\end{pgfscope}%
\begin{pgfscope}%
\pgfsys@transformshift{0.421076in}{1.711420in}%
\pgfsys@useobject{currentmarker}{}%
\end{pgfscope}%
\begin{pgfscope}%
\pgfsys@transformshift{0.930792in}{1.936381in}%
\pgfsys@useobject{currentmarker}{}%
\end{pgfscope}%
\begin{pgfscope}%
\pgfsys@transformshift{0.890478in}{2.050170in}%
\pgfsys@useobject{currentmarker}{}%
\end{pgfscope}%
\begin{pgfscope}%
\pgfsys@transformshift{0.528345in}{1.944560in}%
\pgfsys@useobject{currentmarker}{}%
\end{pgfscope}%
\begin{pgfscope}%
\pgfsys@transformshift{1.013284in}{1.947766in}%
\pgfsys@useobject{currentmarker}{}%
\end{pgfscope}%
\begin{pgfscope}%
\pgfsys@transformshift{0.586631in}{1.935518in}%
\pgfsys@useobject{currentmarker}{}%
\end{pgfscope}%
\begin{pgfscope}%
\pgfsys@transformshift{1.117579in}{1.979864in}%
\pgfsys@useobject{currentmarker}{}%
\end{pgfscope}%
\begin{pgfscope}%
\pgfsys@transformshift{0.440363in}{1.988283in}%
\pgfsys@useobject{currentmarker}{}%
\end{pgfscope}%
\begin{pgfscope}%
\pgfsys@transformshift{0.482005in}{2.009092in}%
\pgfsys@useobject{currentmarker}{}%
\end{pgfscope}%
\begin{pgfscope}%
\pgfsys@transformshift{0.439583in}{1.852331in}%
\pgfsys@useobject{currentmarker}{}%
\end{pgfscope}%
\begin{pgfscope}%
\pgfsys@transformshift{0.857963in}{1.951901in}%
\pgfsys@useobject{currentmarker}{}%
\end{pgfscope}%
\begin{pgfscope}%
\pgfsys@transformshift{1.001461in}{1.918350in}%
\pgfsys@useobject{currentmarker}{}%
\end{pgfscope}%
\begin{pgfscope}%
\pgfsys@transformshift{0.484673in}{1.966656in}%
\pgfsys@useobject{currentmarker}{}%
\end{pgfscope}%
\begin{pgfscope}%
\pgfsys@transformshift{0.458444in}{1.921266in}%
\pgfsys@useobject{currentmarker}{}%
\end{pgfscope}%
\begin{pgfscope}%
\pgfsys@transformshift{0.477639in}{2.054593in}%
\pgfsys@useobject{currentmarker}{}%
\end{pgfscope}%
\begin{pgfscope}%
\pgfsys@transformshift{0.908318in}{2.009289in}%
\pgfsys@useobject{currentmarker}{}%
\end{pgfscope}%
\begin{pgfscope}%
\pgfsys@transformshift{0.463108in}{1.902712in}%
\pgfsys@useobject{currentmarker}{}%
\end{pgfscope}%
\begin{pgfscope}%
\pgfsys@transformshift{1.555994in}{2.056174in}%
\pgfsys@useobject{currentmarker}{}%
\end{pgfscope}%
\begin{pgfscope}%
\pgfsys@transformshift{0.456229in}{1.815015in}%
\pgfsys@useobject{currentmarker}{}%
\end{pgfscope}%
\begin{pgfscope}%
\pgfsys@transformshift{0.909471in}{2.043044in}%
\pgfsys@useobject{currentmarker}{}%
\end{pgfscope}%
\end{pgfscope}%
\begin{pgfscope}%
\pgfpathrectangle{\pgfqpoint{0.341129in}{0.466613in}}{\pgfqpoint{1.658871in}{1.711598in}}%
\pgfusepath{clip}%
\pgfsetbuttcap%
\pgfsetroundjoin%
\definecolor{currentfill}{rgb}{0.298039,0.447059,0.690196}%
\pgfsetfillcolor{currentfill}%
\pgfsetfillopacity{0.150000}%
\pgfsetlinewidth{1.003750pt}%
\definecolor{currentstroke}{rgb}{1.000000,1.000000,1.000000}%
\pgfsetstrokecolor{currentstroke}%
\pgfsetstrokeopacity{0.150000}%
\pgfsetdash{}{0pt}%
\pgfsys@defobject{currentmarker}{\pgfqpoint{0.341129in}{1.934271in}}{\pgfqpoint{2.000000in}{2.031675in}}{%
\pgfpathmoveto{\pgfqpoint{0.341129in}{1.954845in}}%
\pgfpathlineto{\pgfqpoint{0.341129in}{1.934271in}}%
\pgfpathlineto{\pgfqpoint{0.357885in}{1.935208in}}%
\pgfpathlineto{\pgfqpoint{0.374641in}{1.936144in}}%
\pgfpathlineto{\pgfqpoint{0.391398in}{1.937075in}}%
\pgfpathlineto{\pgfqpoint{0.408154in}{1.938010in}}%
\pgfpathlineto{\pgfqpoint{0.424910in}{1.938945in}}%
\pgfpathlineto{\pgfqpoint{0.441666in}{1.939850in}}%
\pgfpathlineto{\pgfqpoint{0.458423in}{1.940768in}}%
\pgfpathlineto{\pgfqpoint{0.475179in}{1.941650in}}%
\pgfpathlineto{\pgfqpoint{0.491935in}{1.942549in}}%
\pgfpathlineto{\pgfqpoint{0.508691in}{1.943444in}}%
\pgfpathlineto{\pgfqpoint{0.525448in}{1.944245in}}%
\pgfpathlineto{\pgfqpoint{0.542204in}{1.945031in}}%
\pgfpathlineto{\pgfqpoint{0.558960in}{1.945844in}}%
\pgfpathlineto{\pgfqpoint{0.575717in}{1.946752in}}%
\pgfpathlineto{\pgfqpoint{0.592473in}{1.947652in}}%
\pgfpathlineto{\pgfqpoint{0.609229in}{1.948476in}}%
\pgfpathlineto{\pgfqpoint{0.625985in}{1.949348in}}%
\pgfpathlineto{\pgfqpoint{0.642742in}{1.950057in}}%
\pgfpathlineto{\pgfqpoint{0.659498in}{1.950838in}}%
\pgfpathlineto{\pgfqpoint{0.676254in}{1.951605in}}%
\pgfpathlineto{\pgfqpoint{0.693011in}{1.952458in}}%
\pgfpathlineto{\pgfqpoint{0.709767in}{1.953213in}}%
\pgfpathlineto{\pgfqpoint{0.726523in}{1.954060in}}%
\pgfpathlineto{\pgfqpoint{0.743279in}{1.954830in}}%
\pgfpathlineto{\pgfqpoint{0.760036in}{1.955578in}}%
\pgfpathlineto{\pgfqpoint{0.776792in}{1.956183in}}%
\pgfpathlineto{\pgfqpoint{0.793548in}{1.956815in}}%
\pgfpathlineto{\pgfqpoint{0.810304in}{1.957507in}}%
\pgfpathlineto{\pgfqpoint{0.827061in}{1.958113in}}%
\pgfpathlineto{\pgfqpoint{0.843817in}{1.958719in}}%
\pgfpathlineto{\pgfqpoint{0.860573in}{1.959343in}}%
\pgfpathlineto{\pgfqpoint{0.877330in}{1.960158in}}%
\pgfpathlineto{\pgfqpoint{0.894086in}{1.960890in}}%
\pgfpathlineto{\pgfqpoint{0.910842in}{1.961357in}}%
\pgfpathlineto{\pgfqpoint{0.927598in}{1.962032in}}%
\pgfpathlineto{\pgfqpoint{0.944355in}{1.962747in}}%
\pgfpathlineto{\pgfqpoint{0.961111in}{1.963357in}}%
\pgfpathlineto{\pgfqpoint{0.977867in}{1.964132in}}%
\pgfpathlineto{\pgfqpoint{0.994623in}{1.964734in}}%
\pgfpathlineto{\pgfqpoint{1.011380in}{1.965220in}}%
\pgfpathlineto{\pgfqpoint{1.028136in}{1.965665in}}%
\pgfpathlineto{\pgfqpoint{1.044892in}{1.966089in}}%
\pgfpathlineto{\pgfqpoint{1.061649in}{1.966639in}}%
\pgfpathlineto{\pgfqpoint{1.078405in}{1.967225in}}%
\pgfpathlineto{\pgfqpoint{1.095161in}{1.967779in}}%
\pgfpathlineto{\pgfqpoint{1.111917in}{1.968296in}}%
\pgfpathlineto{\pgfqpoint{1.128674in}{1.968814in}}%
\pgfpathlineto{\pgfqpoint{1.145430in}{1.969266in}}%
\pgfpathlineto{\pgfqpoint{1.162186in}{1.969725in}}%
\pgfpathlineto{\pgfqpoint{1.178942in}{1.970311in}}%
\pgfpathlineto{\pgfqpoint{1.195699in}{1.970897in}}%
\pgfpathlineto{\pgfqpoint{1.212455in}{1.971482in}}%
\pgfpathlineto{\pgfqpoint{1.229211in}{1.971900in}}%
\pgfpathlineto{\pgfqpoint{1.245968in}{1.972174in}}%
\pgfpathlineto{\pgfqpoint{1.262724in}{1.972735in}}%
\pgfpathlineto{\pgfqpoint{1.279480in}{1.973114in}}%
\pgfpathlineto{\pgfqpoint{1.296236in}{1.973461in}}%
\pgfpathlineto{\pgfqpoint{1.312993in}{1.973901in}}%
\pgfpathlineto{\pgfqpoint{1.329749in}{1.974384in}}%
\pgfpathlineto{\pgfqpoint{1.346505in}{1.974935in}}%
\pgfpathlineto{\pgfqpoint{1.363262in}{1.975348in}}%
\pgfpathlineto{\pgfqpoint{1.380018in}{1.975827in}}%
\pgfpathlineto{\pgfqpoint{1.396774in}{1.976310in}}%
\pgfpathlineto{\pgfqpoint{1.413530in}{1.976756in}}%
\pgfpathlineto{\pgfqpoint{1.430287in}{1.977272in}}%
\pgfpathlineto{\pgfqpoint{1.447043in}{1.977734in}}%
\pgfpathlineto{\pgfqpoint{1.463799in}{1.978123in}}%
\pgfpathlineto{\pgfqpoint{1.480555in}{1.978515in}}%
\pgfpathlineto{\pgfqpoint{1.497312in}{1.978906in}}%
\pgfpathlineto{\pgfqpoint{1.514068in}{1.979275in}}%
\pgfpathlineto{\pgfqpoint{1.530824in}{1.979618in}}%
\pgfpathlineto{\pgfqpoint{1.547581in}{1.980068in}}%
\pgfpathlineto{\pgfqpoint{1.564337in}{1.980458in}}%
\pgfpathlineto{\pgfqpoint{1.581093in}{1.980845in}}%
\pgfpathlineto{\pgfqpoint{1.597849in}{1.981227in}}%
\pgfpathlineto{\pgfqpoint{1.614606in}{1.981598in}}%
\pgfpathlineto{\pgfqpoint{1.631362in}{1.981940in}}%
\pgfpathlineto{\pgfqpoint{1.648118in}{1.982244in}}%
\pgfpathlineto{\pgfqpoint{1.664874in}{1.982548in}}%
\pgfpathlineto{\pgfqpoint{1.681631in}{1.982855in}}%
\pgfpathlineto{\pgfqpoint{1.698387in}{1.983173in}}%
\pgfpathlineto{\pgfqpoint{1.715143in}{1.983633in}}%
\pgfpathlineto{\pgfqpoint{1.731900in}{1.984094in}}%
\pgfpathlineto{\pgfqpoint{1.748656in}{1.984558in}}%
\pgfpathlineto{\pgfqpoint{1.765412in}{1.984943in}}%
\pgfpathlineto{\pgfqpoint{1.782168in}{1.985316in}}%
\pgfpathlineto{\pgfqpoint{1.798925in}{1.985697in}}%
\pgfpathlineto{\pgfqpoint{1.815681in}{1.986165in}}%
\pgfpathlineto{\pgfqpoint{1.832437in}{1.986634in}}%
\pgfpathlineto{\pgfqpoint{1.849193in}{1.987058in}}%
\pgfpathlineto{\pgfqpoint{1.865950in}{1.987447in}}%
\pgfpathlineto{\pgfqpoint{1.882706in}{1.987836in}}%
\pgfpathlineto{\pgfqpoint{1.899462in}{1.988226in}}%
\pgfpathlineto{\pgfqpoint{1.916219in}{1.988615in}}%
\pgfpathlineto{\pgfqpoint{1.932975in}{1.989005in}}%
\pgfpathlineto{\pgfqpoint{1.949731in}{1.989406in}}%
\pgfpathlineto{\pgfqpoint{1.966487in}{1.989852in}}%
\pgfpathlineto{\pgfqpoint{1.983244in}{1.990240in}}%
\pgfpathlineto{\pgfqpoint{2.000000in}{1.990662in}}%
\pgfpathlineto{\pgfqpoint{2.000000in}{2.031675in}}%
\pgfpathlineto{\pgfqpoint{2.000000in}{2.031675in}}%
\pgfpathlineto{\pgfqpoint{1.983244in}{2.030724in}}%
\pgfpathlineto{\pgfqpoint{1.966487in}{2.029740in}}%
\pgfpathlineto{\pgfqpoint{1.949731in}{2.028828in}}%
\pgfpathlineto{\pgfqpoint{1.932975in}{2.027930in}}%
\pgfpathlineto{\pgfqpoint{1.916219in}{2.027030in}}%
\pgfpathlineto{\pgfqpoint{1.899462in}{2.026128in}}%
\pgfpathlineto{\pgfqpoint{1.882706in}{2.025226in}}%
\pgfpathlineto{\pgfqpoint{1.865950in}{2.024314in}}%
\pgfpathlineto{\pgfqpoint{1.849193in}{2.023302in}}%
\pgfpathlineto{\pgfqpoint{1.832437in}{2.022290in}}%
\pgfpathlineto{\pgfqpoint{1.815681in}{2.021344in}}%
\pgfpathlineto{\pgfqpoint{1.798925in}{2.020351in}}%
\pgfpathlineto{\pgfqpoint{1.782168in}{2.019357in}}%
\pgfpathlineto{\pgfqpoint{1.765412in}{2.018474in}}%
\pgfpathlineto{\pgfqpoint{1.748656in}{2.017647in}}%
\pgfpathlineto{\pgfqpoint{1.731900in}{2.016819in}}%
\pgfpathlineto{\pgfqpoint{1.715143in}{2.015916in}}%
\pgfpathlineto{\pgfqpoint{1.698387in}{2.015009in}}%
\pgfpathlineto{\pgfqpoint{1.681631in}{2.014106in}}%
\pgfpathlineto{\pgfqpoint{1.664874in}{2.013202in}}%
\pgfpathlineto{\pgfqpoint{1.648118in}{2.012298in}}%
\pgfpathlineto{\pgfqpoint{1.631362in}{2.011382in}}%
\pgfpathlineto{\pgfqpoint{1.614606in}{2.010382in}}%
\pgfpathlineto{\pgfqpoint{1.597849in}{2.009381in}}%
\pgfpathlineto{\pgfqpoint{1.581093in}{2.008380in}}%
\pgfpathlineto{\pgfqpoint{1.564337in}{2.007379in}}%
\pgfpathlineto{\pgfqpoint{1.547581in}{2.006379in}}%
\pgfpathlineto{\pgfqpoint{1.530824in}{2.005378in}}%
\pgfpathlineto{\pgfqpoint{1.514068in}{2.004514in}}%
\pgfpathlineto{\pgfqpoint{1.497312in}{2.003696in}}%
\pgfpathlineto{\pgfqpoint{1.480555in}{2.002879in}}%
\pgfpathlineto{\pgfqpoint{1.463799in}{2.002062in}}%
\pgfpathlineto{\pgfqpoint{1.447043in}{2.001149in}}%
\pgfpathlineto{\pgfqpoint{1.430287in}{2.000179in}}%
\pgfpathlineto{\pgfqpoint{1.413530in}{1.999209in}}%
\pgfpathlineto{\pgfqpoint{1.396774in}{1.998313in}}%
\pgfpathlineto{\pgfqpoint{1.380018in}{1.997454in}}%
\pgfpathlineto{\pgfqpoint{1.363262in}{1.996595in}}%
\pgfpathlineto{\pgfqpoint{1.346505in}{1.995739in}}%
\pgfpathlineto{\pgfqpoint{1.329749in}{1.994884in}}%
\pgfpathlineto{\pgfqpoint{1.312993in}{1.994030in}}%
\pgfpathlineto{\pgfqpoint{1.296236in}{1.993173in}}%
\pgfpathlineto{\pgfqpoint{1.279480in}{1.992319in}}%
\pgfpathlineto{\pgfqpoint{1.262724in}{1.991402in}}%
\pgfpathlineto{\pgfqpoint{1.245968in}{1.990419in}}%
\pgfpathlineto{\pgfqpoint{1.229211in}{1.989469in}}%
\pgfpathlineto{\pgfqpoint{1.212455in}{1.988634in}}%
\pgfpathlineto{\pgfqpoint{1.195699in}{1.987818in}}%
\pgfpathlineto{\pgfqpoint{1.178942in}{1.986965in}}%
\pgfpathlineto{\pgfqpoint{1.162186in}{1.986134in}}%
\pgfpathlineto{\pgfqpoint{1.145430in}{1.985185in}}%
\pgfpathlineto{\pgfqpoint{1.128674in}{1.984371in}}%
\pgfpathlineto{\pgfqpoint{1.111917in}{1.983557in}}%
\pgfpathlineto{\pgfqpoint{1.095161in}{1.982680in}}%
\pgfpathlineto{\pgfqpoint{1.078405in}{1.981922in}}%
\pgfpathlineto{\pgfqpoint{1.061649in}{1.981114in}}%
\pgfpathlineto{\pgfqpoint{1.044892in}{1.980190in}}%
\pgfpathlineto{\pgfqpoint{1.028136in}{1.979365in}}%
\pgfpathlineto{\pgfqpoint{1.011380in}{1.978466in}}%
\pgfpathlineto{\pgfqpoint{0.994623in}{1.977665in}}%
\pgfpathlineto{\pgfqpoint{0.977867in}{1.976814in}}%
\pgfpathlineto{\pgfqpoint{0.961111in}{1.975927in}}%
\pgfpathlineto{\pgfqpoint{0.944355in}{1.975279in}}%
\pgfpathlineto{\pgfqpoint{0.927598in}{1.974477in}}%
\pgfpathlineto{\pgfqpoint{0.910842in}{1.973844in}}%
\pgfpathlineto{\pgfqpoint{0.894086in}{1.973276in}}%
\pgfpathlineto{\pgfqpoint{0.877330in}{1.972504in}}%
\pgfpathlineto{\pgfqpoint{0.860573in}{1.971666in}}%
\pgfpathlineto{\pgfqpoint{0.843817in}{1.971073in}}%
\pgfpathlineto{\pgfqpoint{0.827061in}{1.970353in}}%
\pgfpathlineto{\pgfqpoint{0.810304in}{1.969621in}}%
\pgfpathlineto{\pgfqpoint{0.793548in}{1.968964in}}%
\pgfpathlineto{\pgfqpoint{0.776792in}{1.968301in}}%
\pgfpathlineto{\pgfqpoint{0.760036in}{1.967555in}}%
\pgfpathlineto{\pgfqpoint{0.743279in}{1.967085in}}%
\pgfpathlineto{\pgfqpoint{0.726523in}{1.966447in}}%
\pgfpathlineto{\pgfqpoint{0.709767in}{1.965883in}}%
\pgfpathlineto{\pgfqpoint{0.693011in}{1.965076in}}%
\pgfpathlineto{\pgfqpoint{0.676254in}{1.964648in}}%
\pgfpathlineto{\pgfqpoint{0.659498in}{1.964064in}}%
\pgfpathlineto{\pgfqpoint{0.642742in}{1.963589in}}%
\pgfpathlineto{\pgfqpoint{0.625985in}{1.963101in}}%
\pgfpathlineto{\pgfqpoint{0.609229in}{1.962564in}}%
\pgfpathlineto{\pgfqpoint{0.592473in}{1.962085in}}%
\pgfpathlineto{\pgfqpoint{0.575717in}{1.961606in}}%
\pgfpathlineto{\pgfqpoint{0.558960in}{1.961130in}}%
\pgfpathlineto{\pgfqpoint{0.542204in}{1.960651in}}%
\pgfpathlineto{\pgfqpoint{0.525448in}{1.960176in}}%
\pgfpathlineto{\pgfqpoint{0.508691in}{1.959695in}}%
\pgfpathlineto{\pgfqpoint{0.491935in}{1.959021in}}%
\pgfpathlineto{\pgfqpoint{0.475179in}{1.958555in}}%
\pgfpathlineto{\pgfqpoint{0.458423in}{1.958161in}}%
\pgfpathlineto{\pgfqpoint{0.441666in}{1.957663in}}%
\pgfpathlineto{\pgfqpoint{0.424910in}{1.957128in}}%
\pgfpathlineto{\pgfqpoint{0.408154in}{1.956594in}}%
\pgfpathlineto{\pgfqpoint{0.391398in}{1.956115in}}%
\pgfpathlineto{\pgfqpoint{0.374641in}{1.955734in}}%
\pgfpathlineto{\pgfqpoint{0.357885in}{1.955282in}}%
\pgfpathlineto{\pgfqpoint{0.341129in}{1.954845in}}%
\pgfpathclose%
\pgfusepath{stroke,fill}%
}%
\begin{pgfscope}%
\pgfsys@transformshift{0.000000in}{0.000000in}%
\pgfsys@useobject{currentmarker}{}%
\end{pgfscope}%
\end{pgfscope}%
\begin{pgfscope}%
\pgfpathrectangle{\pgfqpoint{0.341129in}{0.466613in}}{\pgfqpoint{1.658871in}{1.711598in}}%
\pgfusepath{clip}%
\pgfsetbuttcap%
\pgfsetroundjoin%
\definecolor{currentfill}{rgb}{0.866667,0.517647,0.321569}%
\pgfsetfillcolor{currentfill}%
\pgfsetfillopacity{0.250000}%
\pgfsetlinewidth{1.003750pt}%
\definecolor{currentstroke}{rgb}{0.866667,0.517647,0.321569}%
\pgfsetstrokecolor{currentstroke}%
\pgfsetstrokeopacity{0.250000}%
\pgfsetdash{}{0pt}%
\pgfsys@defobject{currentmarker}{\pgfqpoint{-0.017010in}{-0.017010in}}{\pgfqpoint{0.017010in}{0.017010in}}{%
\pgfpathmoveto{\pgfqpoint{0.000000in}{-0.017010in}}%
\pgfpathcurveto{\pgfqpoint{0.004511in}{-0.017010in}}{\pgfqpoint{0.008838in}{-0.015218in}}{\pgfqpoint{0.012028in}{-0.012028in}}%
\pgfpathcurveto{\pgfqpoint{0.015218in}{-0.008838in}}{\pgfqpoint{0.017010in}{-0.004511in}}{\pgfqpoint{0.017010in}{0.000000in}}%
\pgfpathcurveto{\pgfqpoint{0.017010in}{0.004511in}}{\pgfqpoint{0.015218in}{0.008838in}}{\pgfqpoint{0.012028in}{0.012028in}}%
\pgfpathcurveto{\pgfqpoint{0.008838in}{0.015218in}}{\pgfqpoint{0.004511in}{0.017010in}}{\pgfqpoint{0.000000in}{0.017010in}}%
\pgfpathcurveto{\pgfqpoint{-0.004511in}{0.017010in}}{\pgfqpoint{-0.008838in}{0.015218in}}{\pgfqpoint{-0.012028in}{0.012028in}}%
\pgfpathcurveto{\pgfqpoint{-0.015218in}{0.008838in}}{\pgfqpoint{-0.017010in}{0.004511in}}{\pgfqpoint{-0.017010in}{0.000000in}}%
\pgfpathcurveto{\pgfqpoint{-0.017010in}{-0.004511in}}{\pgfqpoint{-0.015218in}{-0.008838in}}{\pgfqpoint{-0.012028in}{-0.012028in}}%
\pgfpathcurveto{\pgfqpoint{-0.008838in}{-0.015218in}}{\pgfqpoint{-0.004511in}{-0.017010in}}{\pgfqpoint{0.000000in}{-0.017010in}}%
\pgfpathclose%
\pgfusepath{stroke,fill}%
}%
\begin{pgfscope}%
\pgfsys@transformshift{1.684044in}{1.853878in}%
\pgfsys@useobject{currentmarker}{}%
\end{pgfscope}%
\begin{pgfscope}%
\pgfsys@transformshift{1.113006in}{1.870876in}%
\pgfsys@useobject{currentmarker}{}%
\end{pgfscope}%
\begin{pgfscope}%
\pgfsys@transformshift{0.938483in}{1.824262in}%
\pgfsys@useobject{currentmarker}{}%
\end{pgfscope}%
\begin{pgfscope}%
\pgfsys@transformshift{0.492780in}{1.773824in}%
\pgfsys@useobject{currentmarker}{}%
\end{pgfscope}%
\begin{pgfscope}%
\pgfsys@transformshift{0.529797in}{1.671481in}%
\pgfsys@useobject{currentmarker}{}%
\end{pgfscope}%
\begin{pgfscope}%
\pgfsys@transformshift{1.838155in}{1.850265in}%
\pgfsys@useobject{currentmarker}{}%
\end{pgfscope}%
\begin{pgfscope}%
\pgfsys@transformshift{0.477840in}{1.597136in}%
\pgfsys@useobject{currentmarker}{}%
\end{pgfscope}%
\begin{pgfscope}%
\pgfsys@transformshift{0.884452in}{1.884226in}%
\pgfsys@useobject{currentmarker}{}%
\end{pgfscope}%
\begin{pgfscope}%
\pgfsys@transformshift{1.509149in}{1.851611in}%
\pgfsys@useobject{currentmarker}{}%
\end{pgfscope}%
\begin{pgfscope}%
\pgfsys@transformshift{1.183184in}{1.777038in}%
\pgfsys@useobject{currentmarker}{}%
\end{pgfscope}%
\begin{pgfscope}%
\pgfsys@transformshift{1.140467in}{1.787371in}%
\pgfsys@useobject{currentmarker}{}%
\end{pgfscope}%
\begin{pgfscope}%
\pgfsys@transformshift{0.501776in}{1.797193in}%
\pgfsys@useobject{currentmarker}{}%
\end{pgfscope}%
\begin{pgfscope}%
\pgfsys@transformshift{0.461389in}{1.644097in}%
\pgfsys@useobject{currentmarker}{}%
\end{pgfscope}%
\begin{pgfscope}%
\pgfsys@transformshift{0.986905in}{1.797127in}%
\pgfsys@useobject{currentmarker}{}%
\end{pgfscope}%
\begin{pgfscope}%
\pgfsys@transformshift{0.969502in}{1.817884in}%
\pgfsys@useobject{currentmarker}{}%
\end{pgfscope}%
\begin{pgfscope}%
\pgfsys@transformshift{1.016218in}{1.767800in}%
\pgfsys@useobject{currentmarker}{}%
\end{pgfscope}%
\begin{pgfscope}%
\pgfsys@transformshift{0.413211in}{1.813012in}%
\pgfsys@useobject{currentmarker}{}%
\end{pgfscope}%
\begin{pgfscope}%
\pgfsys@transformshift{1.870040in}{1.869226in}%
\pgfsys@useobject{currentmarker}{}%
\end{pgfscope}%
\begin{pgfscope}%
\pgfsys@transformshift{1.031673in}{1.822569in}%
\pgfsys@useobject{currentmarker}{}%
\end{pgfscope}%
\begin{pgfscope}%
\pgfsys@transformshift{1.596314in}{1.800630in}%
\pgfsys@useobject{currentmarker}{}%
\end{pgfscope}%
\begin{pgfscope}%
\pgfsys@transformshift{0.496691in}{1.753535in}%
\pgfsys@useobject{currentmarker}{}%
\end{pgfscope}%
\begin{pgfscope}%
\pgfsys@transformshift{0.896936in}{1.754495in}%
\pgfsys@useobject{currentmarker}{}%
\end{pgfscope}%
\begin{pgfscope}%
\pgfsys@transformshift{0.442231in}{1.799971in}%
\pgfsys@useobject{currentmarker}{}%
\end{pgfscope}%
\begin{pgfscope}%
\pgfsys@transformshift{1.948436in}{1.914116in}%
\pgfsys@useobject{currentmarker}{}%
\end{pgfscope}%
\begin{pgfscope}%
\pgfsys@transformshift{0.607246in}{1.715833in}%
\pgfsys@useobject{currentmarker}{}%
\end{pgfscope}%
\begin{pgfscope}%
\pgfsys@transformshift{0.591373in}{1.542286in}%
\pgfsys@useobject{currentmarker}{}%
\end{pgfscope}%
\begin{pgfscope}%
\pgfsys@transformshift{0.495569in}{1.702933in}%
\pgfsys@useobject{currentmarker}{}%
\end{pgfscope}%
\begin{pgfscope}%
\pgfsys@transformshift{0.805700in}{1.736787in}%
\pgfsys@useobject{currentmarker}{}%
\end{pgfscope}%
\begin{pgfscope}%
\pgfsys@transformshift{0.472915in}{1.792532in}%
\pgfsys@useobject{currentmarker}{}%
\end{pgfscope}%
\begin{pgfscope}%
\pgfsys@transformshift{1.008191in}{1.780836in}%
\pgfsys@useobject{currentmarker}{}%
\end{pgfscope}%
\begin{pgfscope}%
\pgfsys@transformshift{0.795900in}{1.848792in}%
\pgfsys@useobject{currentmarker}{}%
\end{pgfscope}%
\begin{pgfscope}%
\pgfsys@transformshift{0.879596in}{1.827778in}%
\pgfsys@useobject{currentmarker}{}%
\end{pgfscope}%
\begin{pgfscope}%
\pgfsys@transformshift{0.851399in}{1.655878in}%
\pgfsys@useobject{currentmarker}{}%
\end{pgfscope}%
\begin{pgfscope}%
\pgfsys@transformshift{1.630362in}{1.829398in}%
\pgfsys@useobject{currentmarker}{}%
\end{pgfscope}%
\begin{pgfscope}%
\pgfsys@transformshift{0.497926in}{1.783384in}%
\pgfsys@useobject{currentmarker}{}%
\end{pgfscope}%
\begin{pgfscope}%
\pgfsys@transformshift{1.716622in}{1.645384in}%
\pgfsys@useobject{currentmarker}{}%
\end{pgfscope}%
\begin{pgfscope}%
\pgfsys@transformshift{0.524809in}{1.870159in}%
\pgfsys@useobject{currentmarker}{}%
\end{pgfscope}%
\begin{pgfscope}%
\pgfsys@transformshift{1.821229in}{1.904541in}%
\pgfsys@useobject{currentmarker}{}%
\end{pgfscope}%
\begin{pgfscope}%
\pgfsys@transformshift{0.925546in}{1.820690in}%
\pgfsys@useobject{currentmarker}{}%
\end{pgfscope}%
\begin{pgfscope}%
\pgfsys@transformshift{1.841684in}{1.632140in}%
\pgfsys@useobject{currentmarker}{}%
\end{pgfscope}%
\begin{pgfscope}%
\pgfsys@transformshift{1.522493in}{1.817563in}%
\pgfsys@useobject{currentmarker}{}%
\end{pgfscope}%
\begin{pgfscope}%
\pgfsys@transformshift{0.409285in}{1.855956in}%
\pgfsys@useobject{currentmarker}{}%
\end{pgfscope}%
\begin{pgfscope}%
\pgfsys@transformshift{0.896047in}{1.747612in}%
\pgfsys@useobject{currentmarker}{}%
\end{pgfscope}%
\begin{pgfscope}%
\pgfsys@transformshift{0.382462in}{1.828953in}%
\pgfsys@useobject{currentmarker}{}%
\end{pgfscope}%
\begin{pgfscope}%
\pgfsys@transformshift{0.855764in}{1.741755in}%
\pgfsys@useobject{currentmarker}{}%
\end{pgfscope}%
\begin{pgfscope}%
\pgfsys@transformshift{0.523733in}{1.642194in}%
\pgfsys@useobject{currentmarker}{}%
\end{pgfscope}%
\begin{pgfscope}%
\pgfsys@transformshift{0.470197in}{1.902907in}%
\pgfsys@useobject{currentmarker}{}%
\end{pgfscope}%
\begin{pgfscope}%
\pgfsys@transformshift{0.829991in}{1.808450in}%
\pgfsys@useobject{currentmarker}{}%
\end{pgfscope}%
\begin{pgfscope}%
\pgfsys@transformshift{0.933266in}{1.636699in}%
\pgfsys@useobject{currentmarker}{}%
\end{pgfscope}%
\begin{pgfscope}%
\pgfsys@transformshift{0.956789in}{1.911314in}%
\pgfsys@useobject{currentmarker}{}%
\end{pgfscope}%
\begin{pgfscope}%
\pgfsys@transformshift{0.936472in}{1.789964in}%
\pgfsys@useobject{currentmarker}{}%
\end{pgfscope}%
\begin{pgfscope}%
\pgfsys@transformshift{0.468794in}{1.886377in}%
\pgfsys@useobject{currentmarker}{}%
\end{pgfscope}%
\begin{pgfscope}%
\pgfsys@transformshift{0.402384in}{1.897348in}%
\pgfsys@useobject{currentmarker}{}%
\end{pgfscope}%
\begin{pgfscope}%
\pgfsys@transformshift{1.177456in}{1.771878in}%
\pgfsys@useobject{currentmarker}{}%
\end{pgfscope}%
\begin{pgfscope}%
\pgfsys@transformshift{0.416018in}{1.705976in}%
\pgfsys@useobject{currentmarker}{}%
\end{pgfscope}%
\begin{pgfscope}%
\pgfsys@transformshift{0.447008in}{1.886708in}%
\pgfsys@useobject{currentmarker}{}%
\end{pgfscope}%
\begin{pgfscope}%
\pgfsys@transformshift{0.549911in}{1.721435in}%
\pgfsys@useobject{currentmarker}{}%
\end{pgfscope}%
\begin{pgfscope}%
\pgfsys@transformshift{1.065128in}{1.895913in}%
\pgfsys@useobject{currentmarker}{}%
\end{pgfscope}%
\begin{pgfscope}%
\pgfsys@transformshift{0.450053in}{1.729974in}%
\pgfsys@useobject{currentmarker}{}%
\end{pgfscope}%
\begin{pgfscope}%
\pgfsys@transformshift{0.455142in}{1.730373in}%
\pgfsys@useobject{currentmarker}{}%
\end{pgfscope}%
\begin{pgfscope}%
\pgfsys@transformshift{0.484478in}{1.725728in}%
\pgfsys@useobject{currentmarker}{}%
\end{pgfscope}%
\begin{pgfscope}%
\pgfsys@transformshift{1.875283in}{1.877853in}%
\pgfsys@useobject{currentmarker}{}%
\end{pgfscope}%
\begin{pgfscope}%
\pgfsys@transformshift{0.375023in}{1.799180in}%
\pgfsys@useobject{currentmarker}{}%
\end{pgfscope}%
\begin{pgfscope}%
\pgfsys@transformshift{0.853143in}{1.719356in}%
\pgfsys@useobject{currentmarker}{}%
\end{pgfscope}%
\begin{pgfscope}%
\pgfsys@transformshift{0.436365in}{1.768455in}%
\pgfsys@useobject{currentmarker}{}%
\end{pgfscope}%
\begin{pgfscope}%
\pgfsys@transformshift{0.491797in}{1.705393in}%
\pgfsys@useobject{currentmarker}{}%
\end{pgfscope}%
\begin{pgfscope}%
\pgfsys@transformshift{0.902491in}{1.752481in}%
\pgfsys@useobject{currentmarker}{}%
\end{pgfscope}%
\begin{pgfscope}%
\pgfsys@transformshift{0.978414in}{1.770491in}%
\pgfsys@useobject{currentmarker}{}%
\end{pgfscope}%
\begin{pgfscope}%
\pgfsys@transformshift{0.944248in}{1.798518in}%
\pgfsys@useobject{currentmarker}{}%
\end{pgfscope}%
\begin{pgfscope}%
\pgfsys@transformshift{1.817503in}{1.565216in}%
\pgfsys@useobject{currentmarker}{}%
\end{pgfscope}%
\begin{pgfscope}%
\pgfsys@transformshift{1.721814in}{1.814271in}%
\pgfsys@useobject{currentmarker}{}%
\end{pgfscope}%
\begin{pgfscope}%
\pgfsys@transformshift{0.476494in}{1.877966in}%
\pgfsys@useobject{currentmarker}{}%
\end{pgfscope}%
\begin{pgfscope}%
\pgfsys@transformshift{0.942312in}{1.772440in}%
\pgfsys@useobject{currentmarker}{}%
\end{pgfscope}%
\begin{pgfscope}%
\pgfsys@transformshift{0.975555in}{1.797533in}%
\pgfsys@useobject{currentmarker}{}%
\end{pgfscope}%
\begin{pgfscope}%
\pgfsys@transformshift{1.642091in}{1.832804in}%
\pgfsys@useobject{currentmarker}{}%
\end{pgfscope}%
\begin{pgfscope}%
\pgfsys@transformshift{0.910804in}{1.804854in}%
\pgfsys@useobject{currentmarker}{}%
\end{pgfscope}%
\begin{pgfscope}%
\pgfsys@transformshift{0.507435in}{1.698788in}%
\pgfsys@useobject{currentmarker}{}%
\end{pgfscope}%
\begin{pgfscope}%
\pgfsys@transformshift{1.946413in}{1.589666in}%
\pgfsys@useobject{currentmarker}{}%
\end{pgfscope}%
\begin{pgfscope}%
\pgfsys@transformshift{1.639910in}{1.842256in}%
\pgfsys@useobject{currentmarker}{}%
\end{pgfscope}%
\begin{pgfscope}%
\pgfsys@transformshift{1.020077in}{1.841930in}%
\pgfsys@useobject{currentmarker}{}%
\end{pgfscope}%
\begin{pgfscope}%
\pgfsys@transformshift{1.837777in}{1.840432in}%
\pgfsys@useobject{currentmarker}{}%
\end{pgfscope}%
\begin{pgfscope}%
\pgfsys@transformshift{0.546434in}{1.664280in}%
\pgfsys@useobject{currentmarker}{}%
\end{pgfscope}%
\begin{pgfscope}%
\pgfsys@transformshift{0.507573in}{1.788815in}%
\pgfsys@useobject{currentmarker}{}%
\end{pgfscope}%
\begin{pgfscope}%
\pgfsys@transformshift{0.995047in}{1.751878in}%
\pgfsys@useobject{currentmarker}{}%
\end{pgfscope}%
\begin{pgfscope}%
\pgfsys@transformshift{0.507253in}{1.755752in}%
\pgfsys@useobject{currentmarker}{}%
\end{pgfscope}%
\begin{pgfscope}%
\pgfsys@transformshift{0.844423in}{1.806752in}%
\pgfsys@useobject{currentmarker}{}%
\end{pgfscope}%
\begin{pgfscope}%
\pgfsys@transformshift{1.175457in}{1.903431in}%
\pgfsys@useobject{currentmarker}{}%
\end{pgfscope}%
\begin{pgfscope}%
\pgfsys@transformshift{1.024973in}{1.667574in}%
\pgfsys@useobject{currentmarker}{}%
\end{pgfscope}%
\begin{pgfscope}%
\pgfsys@transformshift{1.150522in}{1.748656in}%
\pgfsys@useobject{currentmarker}{}%
\end{pgfscope}%
\begin{pgfscope}%
\pgfsys@transformshift{0.562256in}{1.801930in}%
\pgfsys@useobject{currentmarker}{}%
\end{pgfscope}%
\begin{pgfscope}%
\pgfsys@transformshift{1.482648in}{1.772463in}%
\pgfsys@useobject{currentmarker}{}%
\end{pgfscope}%
\begin{pgfscope}%
\pgfsys@transformshift{1.406514in}{1.785251in}%
\pgfsys@useobject{currentmarker}{}%
\end{pgfscope}%
\begin{pgfscope}%
\pgfsys@transformshift{1.080077in}{1.733285in}%
\pgfsys@useobject{currentmarker}{}%
\end{pgfscope}%
\begin{pgfscope}%
\pgfsys@transformshift{0.989979in}{1.934874in}%
\pgfsys@useobject{currentmarker}{}%
\end{pgfscope}%
\begin{pgfscope}%
\pgfsys@transformshift{0.987626in}{1.856735in}%
\pgfsys@useobject{currentmarker}{}%
\end{pgfscope}%
\begin{pgfscope}%
\pgfsys@transformshift{1.771689in}{1.911917in}%
\pgfsys@useobject{currentmarker}{}%
\end{pgfscope}%
\begin{pgfscope}%
\pgfsys@transformshift{0.891632in}{1.733270in}%
\pgfsys@useobject{currentmarker}{}%
\end{pgfscope}%
\begin{pgfscope}%
\pgfsys@transformshift{0.870836in}{1.880413in}%
\pgfsys@useobject{currentmarker}{}%
\end{pgfscope}%
\begin{pgfscope}%
\pgfsys@transformshift{1.667984in}{1.873402in}%
\pgfsys@useobject{currentmarker}{}%
\end{pgfscope}%
\begin{pgfscope}%
\pgfsys@transformshift{1.813990in}{1.923232in}%
\pgfsys@useobject{currentmarker}{}%
\end{pgfscope}%
\begin{pgfscope}%
\pgfsys@transformshift{0.403236in}{1.809894in}%
\pgfsys@useobject{currentmarker}{}%
\end{pgfscope}%
\begin{pgfscope}%
\pgfsys@transformshift{0.996537in}{1.865852in}%
\pgfsys@useobject{currentmarker}{}%
\end{pgfscope}%
\begin{pgfscope}%
\pgfsys@transformshift{1.043417in}{1.791330in}%
\pgfsys@useobject{currentmarker}{}%
\end{pgfscope}%
\begin{pgfscope}%
\pgfsys@transformshift{0.467968in}{1.682008in}%
\pgfsys@useobject{currentmarker}{}%
\end{pgfscope}%
\begin{pgfscope}%
\pgfsys@transformshift{1.915346in}{1.585657in}%
\pgfsys@useobject{currentmarker}{}%
\end{pgfscope}%
\begin{pgfscope}%
\pgfsys@transformshift{0.437600in}{1.828506in}%
\pgfsys@useobject{currentmarker}{}%
\end{pgfscope}%
\begin{pgfscope}%
\pgfsys@transformshift{0.438013in}{1.751950in}%
\pgfsys@useobject{currentmarker}{}%
\end{pgfscope}%
\begin{pgfscope}%
\pgfsys@transformshift{0.958228in}{1.618330in}%
\pgfsys@useobject{currentmarker}{}%
\end{pgfscope}%
\begin{pgfscope}%
\pgfsys@transformshift{1.012755in}{1.603112in}%
\pgfsys@useobject{currentmarker}{}%
\end{pgfscope}%
\begin{pgfscope}%
\pgfsys@transformshift{0.406225in}{1.711021in}%
\pgfsys@useobject{currentmarker}{}%
\end{pgfscope}%
\begin{pgfscope}%
\pgfsys@transformshift{0.473582in}{1.792207in}%
\pgfsys@useobject{currentmarker}{}%
\end{pgfscope}%
\begin{pgfscope}%
\pgfsys@transformshift{0.590756in}{1.637815in}%
\pgfsys@useobject{currentmarker}{}%
\end{pgfscope}%
\begin{pgfscope}%
\pgfsys@transformshift{0.918416in}{1.594812in}%
\pgfsys@useobject{currentmarker}{}%
\end{pgfscope}%
\begin{pgfscope}%
\pgfsys@transformshift{0.435231in}{1.763256in}%
\pgfsys@useobject{currentmarker}{}%
\end{pgfscope}%
\begin{pgfscope}%
\pgfsys@transformshift{1.100847in}{1.843999in}%
\pgfsys@useobject{currentmarker}{}%
\end{pgfscope}%
\begin{pgfscope}%
\pgfsys@transformshift{1.783869in}{1.906898in}%
\pgfsys@useobject{currentmarker}{}%
\end{pgfscope}%
\begin{pgfscope}%
\pgfsys@transformshift{0.881476in}{1.897486in}%
\pgfsys@useobject{currentmarker}{}%
\end{pgfscope}%
\begin{pgfscope}%
\pgfsys@transformshift{1.004126in}{1.635704in}%
\pgfsys@useobject{currentmarker}{}%
\end{pgfscope}%
\begin{pgfscope}%
\pgfsys@transformshift{1.494969in}{1.858499in}%
\pgfsys@useobject{currentmarker}{}%
\end{pgfscope}%
\begin{pgfscope}%
\pgfsys@transformshift{0.486970in}{1.661858in}%
\pgfsys@useobject{currentmarker}{}%
\end{pgfscope}%
\begin{pgfscope}%
\pgfsys@transformshift{0.439439in}{1.872697in}%
\pgfsys@useobject{currentmarker}{}%
\end{pgfscope}%
\begin{pgfscope}%
\pgfsys@transformshift{0.879557in}{1.837975in}%
\pgfsys@useobject{currentmarker}{}%
\end{pgfscope}%
\begin{pgfscope}%
\pgfsys@transformshift{0.852216in}{1.801933in}%
\pgfsys@useobject{currentmarker}{}%
\end{pgfscope}%
\begin{pgfscope}%
\pgfsys@transformshift{0.950782in}{1.762550in}%
\pgfsys@useobject{currentmarker}{}%
\end{pgfscope}%
\begin{pgfscope}%
\pgfsys@transformshift{0.502562in}{1.835901in}%
\pgfsys@useobject{currentmarker}{}%
\end{pgfscope}%
\begin{pgfscope}%
\pgfsys@transformshift{1.067095in}{1.840416in}%
\pgfsys@useobject{currentmarker}{}%
\end{pgfscope}%
\begin{pgfscope}%
\pgfsys@transformshift{0.478759in}{1.884457in}%
\pgfsys@useobject{currentmarker}{}%
\end{pgfscope}%
\begin{pgfscope}%
\pgfsys@transformshift{0.967133in}{1.826386in}%
\pgfsys@useobject{currentmarker}{}%
\end{pgfscope}%
\begin{pgfscope}%
\pgfsys@transformshift{0.508090in}{1.821193in}%
\pgfsys@useobject{currentmarker}{}%
\end{pgfscope}%
\begin{pgfscope}%
\pgfsys@transformshift{0.435594in}{1.741608in}%
\pgfsys@useobject{currentmarker}{}%
\end{pgfscope}%
\begin{pgfscope}%
\pgfsys@transformshift{0.980355in}{1.670583in}%
\pgfsys@useobject{currentmarker}{}%
\end{pgfscope}%
\begin{pgfscope}%
\pgfsys@transformshift{1.552149in}{1.813327in}%
\pgfsys@useobject{currentmarker}{}%
\end{pgfscope}%
\begin{pgfscope}%
\pgfsys@transformshift{0.875226in}{1.742375in}%
\pgfsys@useobject{currentmarker}{}%
\end{pgfscope}%
\begin{pgfscope}%
\pgfsys@transformshift{1.107273in}{1.820336in}%
\pgfsys@useobject{currentmarker}{}%
\end{pgfscope}%
\begin{pgfscope}%
\pgfsys@transformshift{0.523782in}{1.862860in}%
\pgfsys@useobject{currentmarker}{}%
\end{pgfscope}%
\begin{pgfscope}%
\pgfsys@transformshift{1.045306in}{1.837206in}%
\pgfsys@useobject{currentmarker}{}%
\end{pgfscope}%
\begin{pgfscope}%
\pgfsys@transformshift{1.526749in}{1.766356in}%
\pgfsys@useobject{currentmarker}{}%
\end{pgfscope}%
\begin{pgfscope}%
\pgfsys@transformshift{1.541666in}{1.866017in}%
\pgfsys@useobject{currentmarker}{}%
\end{pgfscope}%
\begin{pgfscope}%
\pgfsys@transformshift{1.574267in}{1.837566in}%
\pgfsys@useobject{currentmarker}{}%
\end{pgfscope}%
\begin{pgfscope}%
\pgfsys@transformshift{1.639614in}{1.826446in}%
\pgfsys@useobject{currentmarker}{}%
\end{pgfscope}%
\begin{pgfscope}%
\pgfsys@transformshift{0.881908in}{1.899346in}%
\pgfsys@useobject{currentmarker}{}%
\end{pgfscope}%
\begin{pgfscope}%
\pgfsys@transformshift{1.677526in}{1.880408in}%
\pgfsys@useobject{currentmarker}{}%
\end{pgfscope}%
\begin{pgfscope}%
\pgfsys@transformshift{0.483826in}{1.773211in}%
\pgfsys@useobject{currentmarker}{}%
\end{pgfscope}%
\begin{pgfscope}%
\pgfsys@transformshift{0.448297in}{1.808027in}%
\pgfsys@useobject{currentmarker}{}%
\end{pgfscope}%
\begin{pgfscope}%
\pgfsys@transformshift{0.976510in}{1.783030in}%
\pgfsys@useobject{currentmarker}{}%
\end{pgfscope}%
\begin{pgfscope}%
\pgfsys@transformshift{0.481727in}{1.778475in}%
\pgfsys@useobject{currentmarker}{}%
\end{pgfscope}%
\begin{pgfscope}%
\pgfsys@transformshift{1.808114in}{1.841508in}%
\pgfsys@useobject{currentmarker}{}%
\end{pgfscope}%
\begin{pgfscope}%
\pgfsys@transformshift{0.468390in}{1.738529in}%
\pgfsys@useobject{currentmarker}{}%
\end{pgfscope}%
\begin{pgfscope}%
\pgfsys@transformshift{0.458999in}{1.746376in}%
\pgfsys@useobject{currentmarker}{}%
\end{pgfscope}%
\begin{pgfscope}%
\pgfsys@transformshift{1.685818in}{1.806568in}%
\pgfsys@useobject{currentmarker}{}%
\end{pgfscope}%
\begin{pgfscope}%
\pgfsys@transformshift{0.635418in}{1.693719in}%
\pgfsys@useobject{currentmarker}{}%
\end{pgfscope}%
\begin{pgfscope}%
\pgfsys@transformshift{0.415977in}{1.687696in}%
\pgfsys@useobject{currentmarker}{}%
\end{pgfscope}%
\begin{pgfscope}%
\pgfsys@transformshift{0.341129in}{1.701703in}%
\pgfsys@useobject{currentmarker}{}%
\end{pgfscope}%
\begin{pgfscope}%
\pgfsys@transformshift{0.992976in}{1.826742in}%
\pgfsys@useobject{currentmarker}{}%
\end{pgfscope}%
\begin{pgfscope}%
\pgfsys@transformshift{0.883182in}{1.842315in}%
\pgfsys@useobject{currentmarker}{}%
\end{pgfscope}%
\begin{pgfscope}%
\pgfsys@transformshift{0.480881in}{1.861521in}%
\pgfsys@useobject{currentmarker}{}%
\end{pgfscope}%
\begin{pgfscope}%
\pgfsys@transformshift{1.244411in}{1.766810in}%
\pgfsys@useobject{currentmarker}{}%
\end{pgfscope}%
\begin{pgfscope}%
\pgfsys@transformshift{1.224203in}{1.792414in}%
\pgfsys@useobject{currentmarker}{}%
\end{pgfscope}%
\begin{pgfscope}%
\pgfsys@transformshift{0.449848in}{1.902063in}%
\pgfsys@useobject{currentmarker}{}%
\end{pgfscope}%
\begin{pgfscope}%
\pgfsys@transformshift{0.582527in}{1.661338in}%
\pgfsys@useobject{currentmarker}{}%
\end{pgfscope}%
\begin{pgfscope}%
\pgfsys@transformshift{1.714524in}{1.928243in}%
\pgfsys@useobject{currentmarker}{}%
\end{pgfscope}%
\begin{pgfscope}%
\pgfsys@transformshift{0.891264in}{1.833417in}%
\pgfsys@useobject{currentmarker}{}%
\end{pgfscope}%
\begin{pgfscope}%
\pgfsys@transformshift{1.857615in}{1.863186in}%
\pgfsys@useobject{currentmarker}{}%
\end{pgfscope}%
\begin{pgfscope}%
\pgfsys@transformshift{1.511327in}{1.828744in}%
\pgfsys@useobject{currentmarker}{}%
\end{pgfscope}%
\begin{pgfscope}%
\pgfsys@transformshift{1.748915in}{1.698133in}%
\pgfsys@useobject{currentmarker}{}%
\end{pgfscope}%
\begin{pgfscope}%
\pgfsys@transformshift{0.491937in}{1.799490in}%
\pgfsys@useobject{currentmarker}{}%
\end{pgfscope}%
\begin{pgfscope}%
\pgfsys@transformshift{0.556274in}{1.763487in}%
\pgfsys@useobject{currentmarker}{}%
\end{pgfscope}%
\begin{pgfscope}%
\pgfsys@transformshift{0.468664in}{1.902366in}%
\pgfsys@useobject{currentmarker}{}%
\end{pgfscope}%
\begin{pgfscope}%
\pgfsys@transformshift{1.550098in}{1.813917in}%
\pgfsys@useobject{currentmarker}{}%
\end{pgfscope}%
\begin{pgfscope}%
\pgfsys@transformshift{0.533554in}{1.789156in}%
\pgfsys@useobject{currentmarker}{}%
\end{pgfscope}%
\begin{pgfscope}%
\pgfsys@transformshift{1.009532in}{1.789108in}%
\pgfsys@useobject{currentmarker}{}%
\end{pgfscope}%
\begin{pgfscope}%
\pgfsys@transformshift{0.590179in}{1.687850in}%
\pgfsys@useobject{currentmarker}{}%
\end{pgfscope}%
\begin{pgfscope}%
\pgfsys@transformshift{0.432340in}{1.895095in}%
\pgfsys@useobject{currentmarker}{}%
\end{pgfscope}%
\begin{pgfscope}%
\pgfsys@transformshift{0.372906in}{1.788771in}%
\pgfsys@useobject{currentmarker}{}%
\end{pgfscope}%
\begin{pgfscope}%
\pgfsys@transformshift{0.459645in}{1.821694in}%
\pgfsys@useobject{currentmarker}{}%
\end{pgfscope}%
\begin{pgfscope}%
\pgfsys@transformshift{1.482135in}{1.760774in}%
\pgfsys@useobject{currentmarker}{}%
\end{pgfscope}%
\begin{pgfscope}%
\pgfsys@transformshift{0.622605in}{1.671690in}%
\pgfsys@useobject{currentmarker}{}%
\end{pgfscope}%
\begin{pgfscope}%
\pgfsys@transformshift{0.489705in}{1.819687in}%
\pgfsys@useobject{currentmarker}{}%
\end{pgfscope}%
\begin{pgfscope}%
\pgfsys@transformshift{1.102334in}{1.720226in}%
\pgfsys@useobject{currentmarker}{}%
\end{pgfscope}%
\begin{pgfscope}%
\pgfsys@transformshift{0.877887in}{1.726763in}%
\pgfsys@useobject{currentmarker}{}%
\end{pgfscope}%
\begin{pgfscope}%
\pgfsys@transformshift{0.519403in}{1.777832in}%
\pgfsys@useobject{currentmarker}{}%
\end{pgfscope}%
\begin{pgfscope}%
\pgfsys@transformshift{0.938030in}{1.739403in}%
\pgfsys@useobject{currentmarker}{}%
\end{pgfscope}%
\begin{pgfscope}%
\pgfsys@transformshift{0.961331in}{1.626364in}%
\pgfsys@useobject{currentmarker}{}%
\end{pgfscope}%
\begin{pgfscope}%
\pgfsys@transformshift{1.036665in}{1.899440in}%
\pgfsys@useobject{currentmarker}{}%
\end{pgfscope}%
\begin{pgfscope}%
\pgfsys@transformshift{0.628071in}{1.781383in}%
\pgfsys@useobject{currentmarker}{}%
\end{pgfscope}%
\begin{pgfscope}%
\pgfsys@transformshift{0.835422in}{1.883472in}%
\pgfsys@useobject{currentmarker}{}%
\end{pgfscope}%
\begin{pgfscope}%
\pgfsys@transformshift{1.066394in}{1.867558in}%
\pgfsys@useobject{currentmarker}{}%
\end{pgfscope}%
\begin{pgfscope}%
\pgfsys@transformshift{1.436938in}{1.767845in}%
\pgfsys@useobject{currentmarker}{}%
\end{pgfscope}%
\begin{pgfscope}%
\pgfsys@transformshift{1.603771in}{1.791464in}%
\pgfsys@useobject{currentmarker}{}%
\end{pgfscope}%
\begin{pgfscope}%
\pgfsys@transformshift{0.959028in}{1.764318in}%
\pgfsys@useobject{currentmarker}{}%
\end{pgfscope}%
\begin{pgfscope}%
\pgfsys@transformshift{0.432442in}{1.791753in}%
\pgfsys@useobject{currentmarker}{}%
\end{pgfscope}%
\begin{pgfscope}%
\pgfsys@transformshift{0.897917in}{1.819599in}%
\pgfsys@useobject{currentmarker}{}%
\end{pgfscope}%
\begin{pgfscope}%
\pgfsys@transformshift{0.894441in}{1.568461in}%
\pgfsys@useobject{currentmarker}{}%
\end{pgfscope}%
\begin{pgfscope}%
\pgfsys@transformshift{0.401578in}{1.881338in}%
\pgfsys@useobject{currentmarker}{}%
\end{pgfscope}%
\begin{pgfscope}%
\pgfsys@transformshift{0.411375in}{1.811524in}%
\pgfsys@useobject{currentmarker}{}%
\end{pgfscope}%
\begin{pgfscope}%
\pgfsys@transformshift{0.874473in}{1.737756in}%
\pgfsys@useobject{currentmarker}{}%
\end{pgfscope}%
\begin{pgfscope}%
\pgfsys@transformshift{0.989940in}{1.776186in}%
\pgfsys@useobject{currentmarker}{}%
\end{pgfscope}%
\begin{pgfscope}%
\pgfsys@transformshift{0.410574in}{1.851976in}%
\pgfsys@useobject{currentmarker}{}%
\end{pgfscope}%
\begin{pgfscope}%
\pgfsys@transformshift{0.887331in}{1.759701in}%
\pgfsys@useobject{currentmarker}{}%
\end{pgfscope}%
\begin{pgfscope}%
\pgfsys@transformshift{0.844030in}{1.752466in}%
\pgfsys@useobject{currentmarker}{}%
\end{pgfscope}%
\begin{pgfscope}%
\pgfsys@transformshift{0.923946in}{1.789607in}%
\pgfsys@useobject{currentmarker}{}%
\end{pgfscope}%
\begin{pgfscope}%
\pgfsys@transformshift{1.517373in}{1.830322in}%
\pgfsys@useobject{currentmarker}{}%
\end{pgfscope}%
\begin{pgfscope}%
\pgfsys@transformshift{1.544776in}{1.842049in}%
\pgfsys@useobject{currentmarker}{}%
\end{pgfscope}%
\begin{pgfscope}%
\pgfsys@transformshift{0.466736in}{1.647793in}%
\pgfsys@useobject{currentmarker}{}%
\end{pgfscope}%
\begin{pgfscope}%
\pgfsys@transformshift{0.539327in}{1.604716in}%
\pgfsys@useobject{currentmarker}{}%
\end{pgfscope}%
\begin{pgfscope}%
\pgfsys@transformshift{1.837625in}{1.875267in}%
\pgfsys@useobject{currentmarker}{}%
\end{pgfscope}%
\begin{pgfscope}%
\pgfsys@transformshift{0.442869in}{1.783365in}%
\pgfsys@useobject{currentmarker}{}%
\end{pgfscope}%
\begin{pgfscope}%
\pgfsys@transformshift{1.136179in}{1.887474in}%
\pgfsys@useobject{currentmarker}{}%
\end{pgfscope}%
\begin{pgfscope}%
\pgfsys@transformshift{0.540575in}{1.793836in}%
\pgfsys@useobject{currentmarker}{}%
\end{pgfscope}%
\begin{pgfscope}%
\pgfsys@transformshift{0.505594in}{1.710714in}%
\pgfsys@useobject{currentmarker}{}%
\end{pgfscope}%
\begin{pgfscope}%
\pgfsys@transformshift{0.464353in}{1.631359in}%
\pgfsys@useobject{currentmarker}{}%
\end{pgfscope}%
\begin{pgfscope}%
\pgfsys@transformshift{0.448024in}{1.792385in}%
\pgfsys@useobject{currentmarker}{}%
\end{pgfscope}%
\begin{pgfscope}%
\pgfsys@transformshift{1.133798in}{1.895245in}%
\pgfsys@useobject{currentmarker}{}%
\end{pgfscope}%
\begin{pgfscope}%
\pgfsys@transformshift{0.450472in}{1.762517in}%
\pgfsys@useobject{currentmarker}{}%
\end{pgfscope}%
\begin{pgfscope}%
\pgfsys@transformshift{0.915981in}{1.606382in}%
\pgfsys@useobject{currentmarker}{}%
\end{pgfscope}%
\begin{pgfscope}%
\pgfsys@transformshift{0.893518in}{1.802764in}%
\pgfsys@useobject{currentmarker}{}%
\end{pgfscope}%
\begin{pgfscope}%
\pgfsys@transformshift{0.839737in}{1.673821in}%
\pgfsys@useobject{currentmarker}{}%
\end{pgfscope}%
\begin{pgfscope}%
\pgfsys@transformshift{0.466023in}{1.781238in}%
\pgfsys@useobject{currentmarker}{}%
\end{pgfscope}%
\begin{pgfscope}%
\pgfsys@transformshift{1.451837in}{1.818982in}%
\pgfsys@useobject{currentmarker}{}%
\end{pgfscope}%
\begin{pgfscope}%
\pgfsys@transformshift{1.812241in}{1.656980in}%
\pgfsys@useobject{currentmarker}{}%
\end{pgfscope}%
\begin{pgfscope}%
\pgfsys@transformshift{0.470694in}{1.804335in}%
\pgfsys@useobject{currentmarker}{}%
\end{pgfscope}%
\begin{pgfscope}%
\pgfsys@transformshift{0.373662in}{1.833249in}%
\pgfsys@useobject{currentmarker}{}%
\end{pgfscope}%
\begin{pgfscope}%
\pgfsys@transformshift{0.403930in}{1.803244in}%
\pgfsys@useobject{currentmarker}{}%
\end{pgfscope}%
\begin{pgfscope}%
\pgfsys@transformshift{1.148968in}{1.923577in}%
\pgfsys@useobject{currentmarker}{}%
\end{pgfscope}%
\begin{pgfscope}%
\pgfsys@transformshift{0.466738in}{1.809417in}%
\pgfsys@useobject{currentmarker}{}%
\end{pgfscope}%
\begin{pgfscope}%
\pgfsys@transformshift{0.600280in}{1.859971in}%
\pgfsys@useobject{currentmarker}{}%
\end{pgfscope}%
\begin{pgfscope}%
\pgfsys@transformshift{0.430403in}{1.838121in}%
\pgfsys@useobject{currentmarker}{}%
\end{pgfscope}%
\begin{pgfscope}%
\pgfsys@transformshift{0.917645in}{1.778000in}%
\pgfsys@useobject{currentmarker}{}%
\end{pgfscope}%
\begin{pgfscope}%
\pgfsys@transformshift{1.972216in}{1.616473in}%
\pgfsys@useobject{currentmarker}{}%
\end{pgfscope}%
\begin{pgfscope}%
\pgfsys@transformshift{0.947192in}{1.807108in}%
\pgfsys@useobject{currentmarker}{}%
\end{pgfscope}%
\begin{pgfscope}%
\pgfsys@transformshift{0.945142in}{1.651043in}%
\pgfsys@useobject{currentmarker}{}%
\end{pgfscope}%
\begin{pgfscope}%
\pgfsys@transformshift{0.884523in}{1.706422in}%
\pgfsys@useobject{currentmarker}{}%
\end{pgfscope}%
\begin{pgfscope}%
\pgfsys@transformshift{0.868046in}{1.779314in}%
\pgfsys@useobject{currentmarker}{}%
\end{pgfscope}%
\begin{pgfscope}%
\pgfsys@transformshift{0.427823in}{1.807537in}%
\pgfsys@useobject{currentmarker}{}%
\end{pgfscope}%
\begin{pgfscope}%
\pgfsys@transformshift{1.240509in}{1.824030in}%
\pgfsys@useobject{currentmarker}{}%
\end{pgfscope}%
\begin{pgfscope}%
\pgfsys@transformshift{0.427367in}{1.922562in}%
\pgfsys@useobject{currentmarker}{}%
\end{pgfscope}%
\begin{pgfscope}%
\pgfsys@transformshift{0.877911in}{1.761497in}%
\pgfsys@useobject{currentmarker}{}%
\end{pgfscope}%
\begin{pgfscope}%
\pgfsys@transformshift{0.502381in}{1.654501in}%
\pgfsys@useobject{currentmarker}{}%
\end{pgfscope}%
\begin{pgfscope}%
\pgfsys@transformshift{0.897409in}{1.594435in}%
\pgfsys@useobject{currentmarker}{}%
\end{pgfscope}%
\begin{pgfscope}%
\pgfsys@transformshift{0.371513in}{1.693741in}%
\pgfsys@useobject{currentmarker}{}%
\end{pgfscope}%
\begin{pgfscope}%
\pgfsys@transformshift{0.424068in}{1.834225in}%
\pgfsys@useobject{currentmarker}{}%
\end{pgfscope}%
\begin{pgfscope}%
\pgfsys@transformshift{0.512683in}{1.526941in}%
\pgfsys@useobject{currentmarker}{}%
\end{pgfscope}%
\begin{pgfscope}%
\pgfsys@transformshift{0.423224in}{1.739869in}%
\pgfsys@useobject{currentmarker}{}%
\end{pgfscope}%
\begin{pgfscope}%
\pgfsys@transformshift{0.590216in}{1.688216in}%
\pgfsys@useobject{currentmarker}{}%
\end{pgfscope}%
\begin{pgfscope}%
\pgfsys@transformshift{1.015493in}{1.717077in}%
\pgfsys@useobject{currentmarker}{}%
\end{pgfscope}%
\begin{pgfscope}%
\pgfsys@transformshift{1.527374in}{1.834827in}%
\pgfsys@useobject{currentmarker}{}%
\end{pgfscope}%
\begin{pgfscope}%
\pgfsys@transformshift{1.794630in}{1.539164in}%
\pgfsys@useobject{currentmarker}{}%
\end{pgfscope}%
\begin{pgfscope}%
\pgfsys@transformshift{0.971721in}{1.820354in}%
\pgfsys@useobject{currentmarker}{}%
\end{pgfscope}%
\begin{pgfscope}%
\pgfsys@transformshift{0.402429in}{1.795930in}%
\pgfsys@useobject{currentmarker}{}%
\end{pgfscope}%
\begin{pgfscope}%
\pgfsys@transformshift{0.596209in}{1.667002in}%
\pgfsys@useobject{currentmarker}{}%
\end{pgfscope}%
\begin{pgfscope}%
\pgfsys@transformshift{0.596444in}{1.826485in}%
\pgfsys@useobject{currentmarker}{}%
\end{pgfscope}%
\begin{pgfscope}%
\pgfsys@transformshift{0.472945in}{1.830126in}%
\pgfsys@useobject{currentmarker}{}%
\end{pgfscope}%
\begin{pgfscope}%
\pgfsys@transformshift{0.505729in}{1.836818in}%
\pgfsys@useobject{currentmarker}{}%
\end{pgfscope}%
\begin{pgfscope}%
\pgfsys@transformshift{0.522392in}{1.767833in}%
\pgfsys@useobject{currentmarker}{}%
\end{pgfscope}%
\begin{pgfscope}%
\pgfsys@transformshift{0.958437in}{1.828689in}%
\pgfsys@useobject{currentmarker}{}%
\end{pgfscope}%
\begin{pgfscope}%
\pgfsys@transformshift{0.899312in}{1.868462in}%
\pgfsys@useobject{currentmarker}{}%
\end{pgfscope}%
\begin{pgfscope}%
\pgfsys@transformshift{0.477796in}{1.888001in}%
\pgfsys@useobject{currentmarker}{}%
\end{pgfscope}%
\begin{pgfscope}%
\pgfsys@transformshift{0.495644in}{1.793686in}%
\pgfsys@useobject{currentmarker}{}%
\end{pgfscope}%
\begin{pgfscope}%
\pgfsys@transformshift{0.461404in}{1.870968in}%
\pgfsys@useobject{currentmarker}{}%
\end{pgfscope}%
\begin{pgfscope}%
\pgfsys@transformshift{0.564846in}{1.822049in}%
\pgfsys@useobject{currentmarker}{}%
\end{pgfscope}%
\begin{pgfscope}%
\pgfsys@transformshift{0.438201in}{1.673372in}%
\pgfsys@useobject{currentmarker}{}%
\end{pgfscope}%
\begin{pgfscope}%
\pgfsys@transformshift{0.830917in}{1.875882in}%
\pgfsys@useobject{currentmarker}{}%
\end{pgfscope}%
\begin{pgfscope}%
\pgfsys@transformshift{0.470836in}{1.721628in}%
\pgfsys@useobject{currentmarker}{}%
\end{pgfscope}%
\begin{pgfscope}%
\pgfsys@transformshift{1.732430in}{1.847121in}%
\pgfsys@useobject{currentmarker}{}%
\end{pgfscope}%
\begin{pgfscope}%
\pgfsys@transformshift{0.880194in}{1.682577in}%
\pgfsys@useobject{currentmarker}{}%
\end{pgfscope}%
\begin{pgfscope}%
\pgfsys@transformshift{1.771020in}{1.627829in}%
\pgfsys@useobject{currentmarker}{}%
\end{pgfscope}%
\begin{pgfscope}%
\pgfsys@transformshift{0.408541in}{1.914038in}%
\pgfsys@useobject{currentmarker}{}%
\end{pgfscope}%
\begin{pgfscope}%
\pgfsys@transformshift{1.783013in}{1.632381in}%
\pgfsys@useobject{currentmarker}{}%
\end{pgfscope}%
\begin{pgfscope}%
\pgfsys@transformshift{1.734433in}{1.643701in}%
\pgfsys@useobject{currentmarker}{}%
\end{pgfscope}%
\begin{pgfscope}%
\pgfsys@transformshift{1.772028in}{1.891754in}%
\pgfsys@useobject{currentmarker}{}%
\end{pgfscope}%
\begin{pgfscope}%
\pgfsys@transformshift{0.482187in}{1.918927in}%
\pgfsys@useobject{currentmarker}{}%
\end{pgfscope}%
\begin{pgfscope}%
\pgfsys@transformshift{0.850634in}{1.585413in}%
\pgfsys@useobject{currentmarker}{}%
\end{pgfscope}%
\begin{pgfscope}%
\pgfsys@transformshift{1.489929in}{1.754640in}%
\pgfsys@useobject{currentmarker}{}%
\end{pgfscope}%
\begin{pgfscope}%
\pgfsys@transformshift{0.572833in}{1.671755in}%
\pgfsys@useobject{currentmarker}{}%
\end{pgfscope}%
\begin{pgfscope}%
\pgfsys@transformshift{0.459424in}{1.800711in}%
\pgfsys@useobject{currentmarker}{}%
\end{pgfscope}%
\begin{pgfscope}%
\pgfsys@transformshift{0.454844in}{1.786041in}%
\pgfsys@useobject{currentmarker}{}%
\end{pgfscope}%
\begin{pgfscope}%
\pgfsys@transformshift{1.529823in}{1.807215in}%
\pgfsys@useobject{currentmarker}{}%
\end{pgfscope}%
\begin{pgfscope}%
\pgfsys@transformshift{0.477367in}{1.907124in}%
\pgfsys@useobject{currentmarker}{}%
\end{pgfscope}%
\begin{pgfscope}%
\pgfsys@transformshift{1.049797in}{1.773046in}%
\pgfsys@useobject{currentmarker}{}%
\end{pgfscope}%
\begin{pgfscope}%
\pgfsys@transformshift{1.390925in}{1.960285in}%
\pgfsys@useobject{currentmarker}{}%
\end{pgfscope}%
\begin{pgfscope}%
\pgfsys@transformshift{1.615976in}{1.630735in}%
\pgfsys@useobject{currentmarker}{}%
\end{pgfscope}%
\begin{pgfscope}%
\pgfsys@transformshift{1.879685in}{1.667760in}%
\pgfsys@useobject{currentmarker}{}%
\end{pgfscope}%
\begin{pgfscope}%
\pgfsys@transformshift{0.474522in}{1.808092in}%
\pgfsys@useobject{currentmarker}{}%
\end{pgfscope}%
\begin{pgfscope}%
\pgfsys@transformshift{0.508097in}{1.898521in}%
\pgfsys@useobject{currentmarker}{}%
\end{pgfscope}%
\begin{pgfscope}%
\pgfsys@transformshift{0.846995in}{1.663276in}%
\pgfsys@useobject{currentmarker}{}%
\end{pgfscope}%
\begin{pgfscope}%
\pgfsys@transformshift{1.720538in}{1.853717in}%
\pgfsys@useobject{currentmarker}{}%
\end{pgfscope}%
\begin{pgfscope}%
\pgfsys@transformshift{1.561110in}{1.781498in}%
\pgfsys@useobject{currentmarker}{}%
\end{pgfscope}%
\begin{pgfscope}%
\pgfsys@transformshift{0.477188in}{1.887069in}%
\pgfsys@useobject{currentmarker}{}%
\end{pgfscope}%
\begin{pgfscope}%
\pgfsys@transformshift{0.462555in}{1.852859in}%
\pgfsys@useobject{currentmarker}{}%
\end{pgfscope}%
\begin{pgfscope}%
\pgfsys@transformshift{1.923906in}{1.851793in}%
\pgfsys@useobject{currentmarker}{}%
\end{pgfscope}%
\begin{pgfscope}%
\pgfsys@transformshift{0.990232in}{1.870146in}%
\pgfsys@useobject{currentmarker}{}%
\end{pgfscope}%
\begin{pgfscope}%
\pgfsys@transformshift{0.858173in}{1.679357in}%
\pgfsys@useobject{currentmarker}{}%
\end{pgfscope}%
\begin{pgfscope}%
\pgfsys@transformshift{0.955286in}{1.913521in}%
\pgfsys@useobject{currentmarker}{}%
\end{pgfscope}%
\begin{pgfscope}%
\pgfsys@transformshift{1.614084in}{1.864320in}%
\pgfsys@useobject{currentmarker}{}%
\end{pgfscope}%
\begin{pgfscope}%
\pgfsys@transformshift{0.839480in}{1.876600in}%
\pgfsys@useobject{currentmarker}{}%
\end{pgfscope}%
\begin{pgfscope}%
\pgfsys@transformshift{0.579938in}{1.826011in}%
\pgfsys@useobject{currentmarker}{}%
\end{pgfscope}%
\begin{pgfscope}%
\pgfsys@transformshift{1.021308in}{1.800030in}%
\pgfsys@useobject{currentmarker}{}%
\end{pgfscope}%
\begin{pgfscope}%
\pgfsys@transformshift{0.489542in}{1.875205in}%
\pgfsys@useobject{currentmarker}{}%
\end{pgfscope}%
\begin{pgfscope}%
\pgfsys@transformshift{0.482496in}{1.692648in}%
\pgfsys@useobject{currentmarker}{}%
\end{pgfscope}%
\begin{pgfscope}%
\pgfsys@transformshift{1.231263in}{1.792204in}%
\pgfsys@useobject{currentmarker}{}%
\end{pgfscope}%
\begin{pgfscope}%
\pgfsys@transformshift{0.612346in}{1.701551in}%
\pgfsys@useobject{currentmarker}{}%
\end{pgfscope}%
\begin{pgfscope}%
\pgfsys@transformshift{0.596517in}{1.628761in}%
\pgfsys@useobject{currentmarker}{}%
\end{pgfscope}%
\begin{pgfscope}%
\pgfsys@transformshift{1.011703in}{1.641101in}%
\pgfsys@useobject{currentmarker}{}%
\end{pgfscope}%
\begin{pgfscope}%
\pgfsys@transformshift{0.536110in}{1.632443in}%
\pgfsys@useobject{currentmarker}{}%
\end{pgfscope}%
\begin{pgfscope}%
\pgfsys@transformshift{0.473641in}{1.826021in}%
\pgfsys@useobject{currentmarker}{}%
\end{pgfscope}%
\begin{pgfscope}%
\pgfsys@transformshift{0.949604in}{1.707401in}%
\pgfsys@useobject{currentmarker}{}%
\end{pgfscope}%
\begin{pgfscope}%
\pgfsys@transformshift{1.891487in}{1.871047in}%
\pgfsys@useobject{currentmarker}{}%
\end{pgfscope}%
\begin{pgfscope}%
\pgfsys@transformshift{0.459142in}{1.866863in}%
\pgfsys@useobject{currentmarker}{}%
\end{pgfscope}%
\begin{pgfscope}%
\pgfsys@transformshift{0.993745in}{1.841028in}%
\pgfsys@useobject{currentmarker}{}%
\end{pgfscope}%
\begin{pgfscope}%
\pgfsys@transformshift{0.461010in}{1.842846in}%
\pgfsys@useobject{currentmarker}{}%
\end{pgfscope}%
\begin{pgfscope}%
\pgfsys@transformshift{0.452198in}{1.839528in}%
\pgfsys@useobject{currentmarker}{}%
\end{pgfscope}%
\begin{pgfscope}%
\pgfsys@transformshift{0.567084in}{1.814149in}%
\pgfsys@useobject{currentmarker}{}%
\end{pgfscope}%
\begin{pgfscope}%
\pgfsys@transformshift{0.925975in}{1.863146in}%
\pgfsys@useobject{currentmarker}{}%
\end{pgfscope}%
\begin{pgfscope}%
\pgfsys@transformshift{1.694241in}{1.540974in}%
\pgfsys@useobject{currentmarker}{}%
\end{pgfscope}%
\begin{pgfscope}%
\pgfsys@transformshift{0.589923in}{1.663022in}%
\pgfsys@useobject{currentmarker}{}%
\end{pgfscope}%
\begin{pgfscope}%
\pgfsys@transformshift{1.421991in}{1.796288in}%
\pgfsys@useobject{currentmarker}{}%
\end{pgfscope}%
\begin{pgfscope}%
\pgfsys@transformshift{0.569884in}{1.821477in}%
\pgfsys@useobject{currentmarker}{}%
\end{pgfscope}%
\begin{pgfscope}%
\pgfsys@transformshift{0.481143in}{1.465096in}%
\pgfsys@useobject{currentmarker}{}%
\end{pgfscope}%
\begin{pgfscope}%
\pgfsys@transformshift{0.598950in}{1.546932in}%
\pgfsys@useobject{currentmarker}{}%
\end{pgfscope}%
\begin{pgfscope}%
\pgfsys@transformshift{0.891632in}{1.626525in}%
\pgfsys@useobject{currentmarker}{}%
\end{pgfscope}%
\begin{pgfscope}%
\pgfsys@transformshift{1.518904in}{1.837483in}%
\pgfsys@useobject{currentmarker}{}%
\end{pgfscope}%
\begin{pgfscope}%
\pgfsys@transformshift{1.267712in}{1.784414in}%
\pgfsys@useobject{currentmarker}{}%
\end{pgfscope}%
\begin{pgfscope}%
\pgfsys@transformshift{1.736581in}{1.964223in}%
\pgfsys@useobject{currentmarker}{}%
\end{pgfscope}%
\begin{pgfscope}%
\pgfsys@transformshift{0.507870in}{1.793851in}%
\pgfsys@useobject{currentmarker}{}%
\end{pgfscope}%
\begin{pgfscope}%
\pgfsys@transformshift{0.876979in}{1.805955in}%
\pgfsys@useobject{currentmarker}{}%
\end{pgfscope}%
\begin{pgfscope}%
\pgfsys@transformshift{0.975052in}{1.886691in}%
\pgfsys@useobject{currentmarker}{}%
\end{pgfscope}%
\begin{pgfscope}%
\pgfsys@transformshift{0.581391in}{1.727132in}%
\pgfsys@useobject{currentmarker}{}%
\end{pgfscope}%
\begin{pgfscope}%
\pgfsys@transformshift{0.432905in}{1.798732in}%
\pgfsys@useobject{currentmarker}{}%
\end{pgfscope}%
\begin{pgfscope}%
\pgfsys@transformshift{0.887125in}{1.874545in}%
\pgfsys@useobject{currentmarker}{}%
\end{pgfscope}%
\begin{pgfscope}%
\pgfsys@transformshift{0.887888in}{1.591385in}%
\pgfsys@useobject{currentmarker}{}%
\end{pgfscope}%
\begin{pgfscope}%
\pgfsys@transformshift{1.047522in}{1.833555in}%
\pgfsys@useobject{currentmarker}{}%
\end{pgfscope}%
\begin{pgfscope}%
\pgfsys@transformshift{0.459882in}{1.936063in}%
\pgfsys@useobject{currentmarker}{}%
\end{pgfscope}%
\begin{pgfscope}%
\pgfsys@transformshift{0.869741in}{1.626180in}%
\pgfsys@useobject{currentmarker}{}%
\end{pgfscope}%
\begin{pgfscope}%
\pgfsys@transformshift{1.519451in}{1.804942in}%
\pgfsys@useobject{currentmarker}{}%
\end{pgfscope}%
\begin{pgfscope}%
\pgfsys@transformshift{1.704307in}{1.860094in}%
\pgfsys@useobject{currentmarker}{}%
\end{pgfscope}%
\begin{pgfscope}%
\pgfsys@transformshift{1.029054in}{1.830798in}%
\pgfsys@useobject{currentmarker}{}%
\end{pgfscope}%
\begin{pgfscope}%
\pgfsys@transformshift{0.983134in}{1.783155in}%
\pgfsys@useobject{currentmarker}{}%
\end{pgfscope}%
\begin{pgfscope}%
\pgfsys@transformshift{1.345503in}{1.853816in}%
\pgfsys@useobject{currentmarker}{}%
\end{pgfscope}%
\begin{pgfscope}%
\pgfsys@transformshift{0.480154in}{1.871917in}%
\pgfsys@useobject{currentmarker}{}%
\end{pgfscope}%
\begin{pgfscope}%
\pgfsys@transformshift{0.820400in}{1.627990in}%
\pgfsys@useobject{currentmarker}{}%
\end{pgfscope}%
\begin{pgfscope}%
\pgfsys@transformshift{0.890253in}{1.877717in}%
\pgfsys@useobject{currentmarker}{}%
\end{pgfscope}%
\begin{pgfscope}%
\pgfsys@transformshift{1.194868in}{1.938713in}%
\pgfsys@useobject{currentmarker}{}%
\end{pgfscope}%
\begin{pgfscope}%
\pgfsys@transformshift{0.938368in}{1.927769in}%
\pgfsys@useobject{currentmarker}{}%
\end{pgfscope}%
\begin{pgfscope}%
\pgfsys@transformshift{0.663011in}{1.642782in}%
\pgfsys@useobject{currentmarker}{}%
\end{pgfscope}%
\begin{pgfscope}%
\pgfsys@transformshift{0.916322in}{1.589403in}%
\pgfsys@useobject{currentmarker}{}%
\end{pgfscope}%
\begin{pgfscope}%
\pgfsys@transformshift{0.890291in}{1.781308in}%
\pgfsys@useobject{currentmarker}{}%
\end{pgfscope}%
\begin{pgfscope}%
\pgfsys@transformshift{0.909236in}{1.609222in}%
\pgfsys@useobject{currentmarker}{}%
\end{pgfscope}%
\begin{pgfscope}%
\pgfsys@transformshift{0.862619in}{1.672767in}%
\pgfsys@useobject{currentmarker}{}%
\end{pgfscope}%
\begin{pgfscope}%
\pgfsys@transformshift{0.597487in}{1.866490in}%
\pgfsys@useobject{currentmarker}{}%
\end{pgfscope}%
\begin{pgfscope}%
\pgfsys@transformshift{1.964345in}{1.617822in}%
\pgfsys@useobject{currentmarker}{}%
\end{pgfscope}%
\begin{pgfscope}%
\pgfsys@transformshift{1.692774in}{1.603060in}%
\pgfsys@useobject{currentmarker}{}%
\end{pgfscope}%
\begin{pgfscope}%
\pgfsys@transformshift{0.482531in}{1.839096in}%
\pgfsys@useobject{currentmarker}{}%
\end{pgfscope}%
\begin{pgfscope}%
\pgfsys@transformshift{0.804579in}{1.619719in}%
\pgfsys@useobject{currentmarker}{}%
\end{pgfscope}%
\begin{pgfscope}%
\pgfsys@transformshift{0.458204in}{1.779262in}%
\pgfsys@useobject{currentmarker}{}%
\end{pgfscope}%
\begin{pgfscope}%
\pgfsys@transformshift{1.029005in}{1.776192in}%
\pgfsys@useobject{currentmarker}{}%
\end{pgfscope}%
\begin{pgfscope}%
\pgfsys@transformshift{0.509849in}{1.673512in}%
\pgfsys@useobject{currentmarker}{}%
\end{pgfscope}%
\begin{pgfscope}%
\pgfsys@transformshift{1.034049in}{1.740667in}%
\pgfsys@useobject{currentmarker}{}%
\end{pgfscope}%
\begin{pgfscope}%
\pgfsys@transformshift{1.152583in}{1.896896in}%
\pgfsys@useobject{currentmarker}{}%
\end{pgfscope}%
\begin{pgfscope}%
\pgfsys@transformshift{1.044006in}{1.863007in}%
\pgfsys@useobject{currentmarker}{}%
\end{pgfscope}%
\begin{pgfscope}%
\pgfsys@transformshift{1.486794in}{1.873056in}%
\pgfsys@useobject{currentmarker}{}%
\end{pgfscope}%
\begin{pgfscope}%
\pgfsys@transformshift{0.870195in}{1.698673in}%
\pgfsys@useobject{currentmarker}{}%
\end{pgfscope}%
\begin{pgfscope}%
\pgfsys@transformshift{1.138693in}{1.857464in}%
\pgfsys@useobject{currentmarker}{}%
\end{pgfscope}%
\begin{pgfscope}%
\pgfsys@transformshift{0.847963in}{1.618692in}%
\pgfsys@useobject{currentmarker}{}%
\end{pgfscope}%
\begin{pgfscope}%
\pgfsys@transformshift{0.477983in}{1.822801in}%
\pgfsys@useobject{currentmarker}{}%
\end{pgfscope}%
\begin{pgfscope}%
\pgfsys@transformshift{0.492721in}{1.762168in}%
\pgfsys@useobject{currentmarker}{}%
\end{pgfscope}%
\begin{pgfscope}%
\pgfsys@transformshift{1.120030in}{1.854687in}%
\pgfsys@useobject{currentmarker}{}%
\end{pgfscope}%
\begin{pgfscope}%
\pgfsys@transformshift{0.895919in}{1.810307in}%
\pgfsys@useobject{currentmarker}{}%
\end{pgfscope}%
\begin{pgfscope}%
\pgfsys@transformshift{0.576124in}{1.651457in}%
\pgfsys@useobject{currentmarker}{}%
\end{pgfscope}%
\begin{pgfscope}%
\pgfsys@transformshift{0.471843in}{1.744981in}%
\pgfsys@useobject{currentmarker}{}%
\end{pgfscope}%
\begin{pgfscope}%
\pgfsys@transformshift{0.475316in}{1.711028in}%
\pgfsys@useobject{currentmarker}{}%
\end{pgfscope}%
\begin{pgfscope}%
\pgfsys@transformshift{1.653595in}{1.853019in}%
\pgfsys@useobject{currentmarker}{}%
\end{pgfscope}%
\begin{pgfscope}%
\pgfsys@transformshift{1.083484in}{1.855334in}%
\pgfsys@useobject{currentmarker}{}%
\end{pgfscope}%
\begin{pgfscope}%
\pgfsys@transformshift{0.929704in}{1.787198in}%
\pgfsys@useobject{currentmarker}{}%
\end{pgfscope}%
\begin{pgfscope}%
\pgfsys@transformshift{1.702011in}{1.836982in}%
\pgfsys@useobject{currentmarker}{}%
\end{pgfscope}%
\begin{pgfscope}%
\pgfsys@transformshift{0.861736in}{1.598110in}%
\pgfsys@useobject{currentmarker}{}%
\end{pgfscope}%
\begin{pgfscope}%
\pgfsys@transformshift{0.466758in}{1.742212in}%
\pgfsys@useobject{currentmarker}{}%
\end{pgfscope}%
\begin{pgfscope}%
\pgfsys@transformshift{0.895642in}{1.756967in}%
\pgfsys@useobject{currentmarker}{}%
\end{pgfscope}%
\begin{pgfscope}%
\pgfsys@transformshift{0.844526in}{1.776986in}%
\pgfsys@useobject{currentmarker}{}%
\end{pgfscope}%
\begin{pgfscope}%
\pgfsys@transformshift{0.903057in}{1.884061in}%
\pgfsys@useobject{currentmarker}{}%
\end{pgfscope}%
\begin{pgfscope}%
\pgfsys@transformshift{0.952685in}{1.755807in}%
\pgfsys@useobject{currentmarker}{}%
\end{pgfscope}%
\begin{pgfscope}%
\pgfsys@transformshift{0.576385in}{1.816056in}%
\pgfsys@useobject{currentmarker}{}%
\end{pgfscope}%
\begin{pgfscope}%
\pgfsys@transformshift{1.016020in}{1.791152in}%
\pgfsys@useobject{currentmarker}{}%
\end{pgfscope}%
\begin{pgfscope}%
\pgfsys@transformshift{0.494156in}{1.640214in}%
\pgfsys@useobject{currentmarker}{}%
\end{pgfscope}%
\begin{pgfscope}%
\pgfsys@transformshift{0.577397in}{1.653976in}%
\pgfsys@useobject{currentmarker}{}%
\end{pgfscope}%
\begin{pgfscope}%
\pgfsys@transformshift{0.445519in}{1.734631in}%
\pgfsys@useobject{currentmarker}{}%
\end{pgfscope}%
\begin{pgfscope}%
\pgfsys@transformshift{1.084282in}{1.759871in}%
\pgfsys@useobject{currentmarker}{}%
\end{pgfscope}%
\begin{pgfscope}%
\pgfsys@transformshift{1.594805in}{1.809346in}%
\pgfsys@useobject{currentmarker}{}%
\end{pgfscope}%
\begin{pgfscope}%
\pgfsys@transformshift{0.976071in}{1.824865in}%
\pgfsys@useobject{currentmarker}{}%
\end{pgfscope}%
\begin{pgfscope}%
\pgfsys@transformshift{0.954554in}{1.871110in}%
\pgfsys@useobject{currentmarker}{}%
\end{pgfscope}%
\begin{pgfscope}%
\pgfsys@transformshift{1.062412in}{1.873897in}%
\pgfsys@useobject{currentmarker}{}%
\end{pgfscope}%
\begin{pgfscope}%
\pgfsys@transformshift{0.403708in}{1.625250in}%
\pgfsys@useobject{currentmarker}{}%
\end{pgfscope}%
\begin{pgfscope}%
\pgfsys@transformshift{0.456382in}{1.787310in}%
\pgfsys@useobject{currentmarker}{}%
\end{pgfscope}%
\begin{pgfscope}%
\pgfsys@transformshift{0.425389in}{1.657540in}%
\pgfsys@useobject{currentmarker}{}%
\end{pgfscope}%
\begin{pgfscope}%
\pgfsys@transformshift{1.178319in}{1.671442in}%
\pgfsys@useobject{currentmarker}{}%
\end{pgfscope}%
\begin{pgfscope}%
\pgfsys@transformshift{0.868624in}{1.864556in}%
\pgfsys@useobject{currentmarker}{}%
\end{pgfscope}%
\begin{pgfscope}%
\pgfsys@transformshift{1.072938in}{1.774012in}%
\pgfsys@useobject{currentmarker}{}%
\end{pgfscope}%
\begin{pgfscope}%
\pgfsys@transformshift{0.842975in}{1.675131in}%
\pgfsys@useobject{currentmarker}{}%
\end{pgfscope}%
\begin{pgfscope}%
\pgfsys@transformshift{1.133541in}{1.846294in}%
\pgfsys@useobject{currentmarker}{}%
\end{pgfscope}%
\begin{pgfscope}%
\pgfsys@transformshift{0.985168in}{1.909805in}%
\pgfsys@useobject{currentmarker}{}%
\end{pgfscope}%
\begin{pgfscope}%
\pgfsys@transformshift{0.935163in}{1.930082in}%
\pgfsys@useobject{currentmarker}{}%
\end{pgfscope}%
\begin{pgfscope}%
\pgfsys@transformshift{0.852856in}{1.694758in}%
\pgfsys@useobject{currentmarker}{}%
\end{pgfscope}%
\begin{pgfscope}%
\pgfsys@transformshift{0.481479in}{1.684767in}%
\pgfsys@useobject{currentmarker}{}%
\end{pgfscope}%
\begin{pgfscope}%
\pgfsys@transformshift{0.483502in}{1.861505in}%
\pgfsys@useobject{currentmarker}{}%
\end{pgfscope}%
\begin{pgfscope}%
\pgfsys@transformshift{0.857024in}{1.585604in}%
\pgfsys@useobject{currentmarker}{}%
\end{pgfscope}%
\begin{pgfscope}%
\pgfsys@transformshift{0.541865in}{1.656028in}%
\pgfsys@useobject{currentmarker}{}%
\end{pgfscope}%
\begin{pgfscope}%
\pgfsys@transformshift{1.564506in}{1.867270in}%
\pgfsys@useobject{currentmarker}{}%
\end{pgfscope}%
\begin{pgfscope}%
\pgfsys@transformshift{0.926665in}{1.876514in}%
\pgfsys@useobject{currentmarker}{}%
\end{pgfscope}%
\begin{pgfscope}%
\pgfsys@transformshift{0.371504in}{1.821851in}%
\pgfsys@useobject{currentmarker}{}%
\end{pgfscope}%
\begin{pgfscope}%
\pgfsys@transformshift{0.869520in}{1.748987in}%
\pgfsys@useobject{currentmarker}{}%
\end{pgfscope}%
\begin{pgfscope}%
\pgfsys@transformshift{0.519844in}{1.709349in}%
\pgfsys@useobject{currentmarker}{}%
\end{pgfscope}%
\begin{pgfscope}%
\pgfsys@transformshift{0.608978in}{1.587894in}%
\pgfsys@useobject{currentmarker}{}%
\end{pgfscope}%
\begin{pgfscope}%
\pgfsys@transformshift{0.476814in}{1.737244in}%
\pgfsys@useobject{currentmarker}{}%
\end{pgfscope}%
\begin{pgfscope}%
\pgfsys@transformshift{0.933077in}{1.812006in}%
\pgfsys@useobject{currentmarker}{}%
\end{pgfscope}%
\begin{pgfscope}%
\pgfsys@transformshift{0.579351in}{1.781610in}%
\pgfsys@useobject{currentmarker}{}%
\end{pgfscope}%
\begin{pgfscope}%
\pgfsys@transformshift{0.459383in}{1.766339in}%
\pgfsys@useobject{currentmarker}{}%
\end{pgfscope}%
\begin{pgfscope}%
\pgfsys@transformshift{0.451992in}{1.742553in}%
\pgfsys@useobject{currentmarker}{}%
\end{pgfscope}%
\begin{pgfscope}%
\pgfsys@transformshift{0.448649in}{1.870017in}%
\pgfsys@useobject{currentmarker}{}%
\end{pgfscope}%
\begin{pgfscope}%
\pgfsys@transformshift{1.803636in}{1.914341in}%
\pgfsys@useobject{currentmarker}{}%
\end{pgfscope}%
\begin{pgfscope}%
\pgfsys@transformshift{1.680858in}{1.576975in}%
\pgfsys@useobject{currentmarker}{}%
\end{pgfscope}%
\begin{pgfscope}%
\pgfsys@transformshift{0.464450in}{1.783701in}%
\pgfsys@useobject{currentmarker}{}%
\end{pgfscope}%
\begin{pgfscope}%
\pgfsys@transformshift{0.467818in}{1.753583in}%
\pgfsys@useobject{currentmarker}{}%
\end{pgfscope}%
\begin{pgfscope}%
\pgfsys@transformshift{1.797702in}{1.935939in}%
\pgfsys@useobject{currentmarker}{}%
\end{pgfscope}%
\begin{pgfscope}%
\pgfsys@transformshift{1.800535in}{1.911926in}%
\pgfsys@useobject{currentmarker}{}%
\end{pgfscope}%
\begin{pgfscope}%
\pgfsys@transformshift{0.486053in}{1.738118in}%
\pgfsys@useobject{currentmarker}{}%
\end{pgfscope}%
\begin{pgfscope}%
\pgfsys@transformshift{0.604133in}{1.610316in}%
\pgfsys@useobject{currentmarker}{}%
\end{pgfscope}%
\begin{pgfscope}%
\pgfsys@transformshift{1.510193in}{1.790007in}%
\pgfsys@useobject{currentmarker}{}%
\end{pgfscope}%
\begin{pgfscope}%
\pgfsys@transformshift{0.969856in}{1.845263in}%
\pgfsys@useobject{currentmarker}{}%
\end{pgfscope}%
\begin{pgfscope}%
\pgfsys@transformshift{0.883908in}{1.906167in}%
\pgfsys@useobject{currentmarker}{}%
\end{pgfscope}%
\begin{pgfscope}%
\pgfsys@transformshift{0.538449in}{1.631533in}%
\pgfsys@useobject{currentmarker}{}%
\end{pgfscope}%
\begin{pgfscope}%
\pgfsys@transformshift{0.561398in}{1.703253in}%
\pgfsys@useobject{currentmarker}{}%
\end{pgfscope}%
\begin{pgfscope}%
\pgfsys@transformshift{0.984012in}{1.843107in}%
\pgfsys@useobject{currentmarker}{}%
\end{pgfscope}%
\begin{pgfscope}%
\pgfsys@transformshift{0.614232in}{1.827492in}%
\pgfsys@useobject{currentmarker}{}%
\end{pgfscope}%
\begin{pgfscope}%
\pgfsys@transformshift{1.178616in}{1.877332in}%
\pgfsys@useobject{currentmarker}{}%
\end{pgfscope}%
\begin{pgfscope}%
\pgfsys@transformshift{0.638320in}{1.665048in}%
\pgfsys@useobject{currentmarker}{}%
\end{pgfscope}%
\begin{pgfscope}%
\pgfsys@transformshift{1.571164in}{1.791625in}%
\pgfsys@useobject{currentmarker}{}%
\end{pgfscope}%
\begin{pgfscope}%
\pgfsys@transformshift{0.449499in}{1.856566in}%
\pgfsys@useobject{currentmarker}{}%
\end{pgfscope}%
\begin{pgfscope}%
\pgfsys@transformshift{1.000812in}{1.928867in}%
\pgfsys@useobject{currentmarker}{}%
\end{pgfscope}%
\begin{pgfscope}%
\pgfsys@transformshift{0.504036in}{1.763738in}%
\pgfsys@useobject{currentmarker}{}%
\end{pgfscope}%
\begin{pgfscope}%
\pgfsys@transformshift{1.192646in}{1.891322in}%
\pgfsys@useobject{currentmarker}{}%
\end{pgfscope}%
\begin{pgfscope}%
\pgfsys@transformshift{0.575539in}{1.621077in}%
\pgfsys@useobject{currentmarker}{}%
\end{pgfscope}%
\begin{pgfscope}%
\pgfsys@transformshift{0.517708in}{1.854897in}%
\pgfsys@useobject{currentmarker}{}%
\end{pgfscope}%
\begin{pgfscope}%
\pgfsys@transformshift{1.244210in}{1.618320in}%
\pgfsys@useobject{currentmarker}{}%
\end{pgfscope}%
\begin{pgfscope}%
\pgfsys@transformshift{0.615567in}{1.596637in}%
\pgfsys@useobject{currentmarker}{}%
\end{pgfscope}%
\begin{pgfscope}%
\pgfsys@transformshift{0.907672in}{1.838763in}%
\pgfsys@useobject{currentmarker}{}%
\end{pgfscope}%
\begin{pgfscope}%
\pgfsys@transformshift{2.000000in}{2.006407in}%
\pgfsys@useobject{currentmarker}{}%
\end{pgfscope}%
\begin{pgfscope}%
\pgfsys@transformshift{0.485263in}{1.800818in}%
\pgfsys@useobject{currentmarker}{}%
\end{pgfscope}%
\begin{pgfscope}%
\pgfsys@transformshift{0.433137in}{1.659247in}%
\pgfsys@useobject{currentmarker}{}%
\end{pgfscope}%
\begin{pgfscope}%
\pgfsys@transformshift{0.499449in}{1.837303in}%
\pgfsys@useobject{currentmarker}{}%
\end{pgfscope}%
\begin{pgfscope}%
\pgfsys@transformshift{0.868886in}{1.757756in}%
\pgfsys@useobject{currentmarker}{}%
\end{pgfscope}%
\begin{pgfscope}%
\pgfsys@transformshift{1.451865in}{1.844590in}%
\pgfsys@useobject{currentmarker}{}%
\end{pgfscope}%
\begin{pgfscope}%
\pgfsys@transformshift{0.534197in}{1.862265in}%
\pgfsys@useobject{currentmarker}{}%
\end{pgfscope}%
\begin{pgfscope}%
\pgfsys@transformshift{0.972750in}{1.849276in}%
\pgfsys@useobject{currentmarker}{}%
\end{pgfscope}%
\begin{pgfscope}%
\pgfsys@transformshift{0.996367in}{1.691411in}%
\pgfsys@useobject{currentmarker}{}%
\end{pgfscope}%
\begin{pgfscope}%
\pgfsys@transformshift{0.893513in}{1.725888in}%
\pgfsys@useobject{currentmarker}{}%
\end{pgfscope}%
\begin{pgfscope}%
\pgfsys@transformshift{0.481561in}{1.790297in}%
\pgfsys@useobject{currentmarker}{}%
\end{pgfscope}%
\begin{pgfscope}%
\pgfsys@transformshift{0.462083in}{1.805192in}%
\pgfsys@useobject{currentmarker}{}%
\end{pgfscope}%
\begin{pgfscope}%
\pgfsys@transformshift{0.947888in}{1.702712in}%
\pgfsys@useobject{currentmarker}{}%
\end{pgfscope}%
\begin{pgfscope}%
\pgfsys@transformshift{0.475131in}{1.802314in}%
\pgfsys@useobject{currentmarker}{}%
\end{pgfscope}%
\begin{pgfscope}%
\pgfsys@transformshift{0.944100in}{1.749088in}%
\pgfsys@useobject{currentmarker}{}%
\end{pgfscope}%
\begin{pgfscope}%
\pgfsys@transformshift{0.966481in}{1.785338in}%
\pgfsys@useobject{currentmarker}{}%
\end{pgfscope}%
\begin{pgfscope}%
\pgfsys@transformshift{1.021199in}{1.842153in}%
\pgfsys@useobject{currentmarker}{}%
\end{pgfscope}%
\begin{pgfscope}%
\pgfsys@transformshift{0.451598in}{1.834367in}%
\pgfsys@useobject{currentmarker}{}%
\end{pgfscope}%
\begin{pgfscope}%
\pgfsys@transformshift{0.461004in}{1.882816in}%
\pgfsys@useobject{currentmarker}{}%
\end{pgfscope}%
\begin{pgfscope}%
\pgfsys@transformshift{0.419847in}{1.823394in}%
\pgfsys@useobject{currentmarker}{}%
\end{pgfscope}%
\begin{pgfscope}%
\pgfsys@transformshift{0.498798in}{1.677898in}%
\pgfsys@useobject{currentmarker}{}%
\end{pgfscope}%
\begin{pgfscope}%
\pgfsys@transformshift{1.022778in}{1.732234in}%
\pgfsys@useobject{currentmarker}{}%
\end{pgfscope}%
\begin{pgfscope}%
\pgfsys@transformshift{1.618050in}{1.556256in}%
\pgfsys@useobject{currentmarker}{}%
\end{pgfscope}%
\begin{pgfscope}%
\pgfsys@transformshift{0.873499in}{1.868159in}%
\pgfsys@useobject{currentmarker}{}%
\end{pgfscope}%
\begin{pgfscope}%
\pgfsys@transformshift{1.717975in}{1.780727in}%
\pgfsys@useobject{currentmarker}{}%
\end{pgfscope}%
\begin{pgfscope}%
\pgfsys@transformshift{0.479704in}{1.916297in}%
\pgfsys@useobject{currentmarker}{}%
\end{pgfscope}%
\begin{pgfscope}%
\pgfsys@transformshift{0.921403in}{1.903746in}%
\pgfsys@useobject{currentmarker}{}%
\end{pgfscope}%
\begin{pgfscope}%
\pgfsys@transformshift{0.507519in}{1.795899in}%
\pgfsys@useobject{currentmarker}{}%
\end{pgfscope}%
\begin{pgfscope}%
\pgfsys@transformshift{1.242289in}{1.915682in}%
\pgfsys@useobject{currentmarker}{}%
\end{pgfscope}%
\begin{pgfscope}%
\pgfsys@transformshift{0.901785in}{1.694625in}%
\pgfsys@useobject{currentmarker}{}%
\end{pgfscope}%
\begin{pgfscope}%
\pgfsys@transformshift{1.303585in}{1.838219in}%
\pgfsys@useobject{currentmarker}{}%
\end{pgfscope}%
\begin{pgfscope}%
\pgfsys@transformshift{0.385710in}{1.817214in}%
\pgfsys@useobject{currentmarker}{}%
\end{pgfscope}%
\begin{pgfscope}%
\pgfsys@transformshift{0.487652in}{1.764339in}%
\pgfsys@useobject{currentmarker}{}%
\end{pgfscope}%
\begin{pgfscope}%
\pgfsys@transformshift{0.427840in}{1.762543in}%
\pgfsys@useobject{currentmarker}{}%
\end{pgfscope}%
\begin{pgfscope}%
\pgfsys@transformshift{0.479919in}{1.821305in}%
\pgfsys@useobject{currentmarker}{}%
\end{pgfscope}%
\begin{pgfscope}%
\pgfsys@transformshift{0.421076in}{1.473522in}%
\pgfsys@useobject{currentmarker}{}%
\end{pgfscope}%
\begin{pgfscope}%
\pgfsys@transformshift{0.930792in}{1.687402in}%
\pgfsys@useobject{currentmarker}{}%
\end{pgfscope}%
\begin{pgfscope}%
\pgfsys@transformshift{0.890478in}{1.928432in}%
\pgfsys@useobject{currentmarker}{}%
\end{pgfscope}%
\begin{pgfscope}%
\pgfsys@transformshift{0.528345in}{1.723264in}%
\pgfsys@useobject{currentmarker}{}%
\end{pgfscope}%
\begin{pgfscope}%
\pgfsys@transformshift{1.013284in}{1.743681in}%
\pgfsys@useobject{currentmarker}{}%
\end{pgfscope}%
\begin{pgfscope}%
\pgfsys@transformshift{0.586631in}{1.725600in}%
\pgfsys@useobject{currentmarker}{}%
\end{pgfscope}%
\begin{pgfscope}%
\pgfsys@transformshift{1.117579in}{1.764325in}%
\pgfsys@useobject{currentmarker}{}%
\end{pgfscope}%
\begin{pgfscope}%
\pgfsys@transformshift{0.440363in}{1.795331in}%
\pgfsys@useobject{currentmarker}{}%
\end{pgfscope}%
\begin{pgfscope}%
\pgfsys@transformshift{0.482005in}{1.880603in}%
\pgfsys@useobject{currentmarker}{}%
\end{pgfscope}%
\begin{pgfscope}%
\pgfsys@transformshift{0.439583in}{1.650477in}%
\pgfsys@useobject{currentmarker}{}%
\end{pgfscope}%
\begin{pgfscope}%
\pgfsys@transformshift{0.857963in}{1.799329in}%
\pgfsys@useobject{currentmarker}{}%
\end{pgfscope}%
\begin{pgfscope}%
\pgfsys@transformshift{1.001461in}{1.720511in}%
\pgfsys@useobject{currentmarker}{}%
\end{pgfscope}%
\begin{pgfscope}%
\pgfsys@transformshift{0.484673in}{1.806752in}%
\pgfsys@useobject{currentmarker}{}%
\end{pgfscope}%
\begin{pgfscope}%
\pgfsys@transformshift{0.458444in}{1.740723in}%
\pgfsys@useobject{currentmarker}{}%
\end{pgfscope}%
\begin{pgfscope}%
\pgfsys@transformshift{0.477639in}{1.952906in}%
\pgfsys@useobject{currentmarker}{}%
\end{pgfscope}%
\begin{pgfscope}%
\pgfsys@transformshift{0.908318in}{1.749866in}%
\pgfsys@useobject{currentmarker}{}%
\end{pgfscope}%
\begin{pgfscope}%
\pgfsys@transformshift{0.463108in}{1.722357in}%
\pgfsys@useobject{currentmarker}{}%
\end{pgfscope}%
\begin{pgfscope}%
\pgfsys@transformshift{1.555994in}{1.870109in}%
\pgfsys@useobject{currentmarker}{}%
\end{pgfscope}%
\begin{pgfscope}%
\pgfsys@transformshift{0.456229in}{1.603663in}%
\pgfsys@useobject{currentmarker}{}%
\end{pgfscope}%
\begin{pgfscope}%
\pgfsys@transformshift{0.909471in}{1.918379in}%
\pgfsys@useobject{currentmarker}{}%
\end{pgfscope}%
\end{pgfscope}%
\begin{pgfscope}%
\pgfpathrectangle{\pgfqpoint{0.341129in}{0.466613in}}{\pgfqpoint{1.658871in}{1.711598in}}%
\pgfusepath{clip}%
\pgfsetbuttcap%
\pgfsetroundjoin%
\definecolor{currentfill}{rgb}{0.866667,0.517647,0.321569}%
\pgfsetfillcolor{currentfill}%
\pgfsetfillopacity{0.150000}%
\pgfsetlinewidth{1.003750pt}%
\definecolor{currentstroke}{rgb}{1.000000,1.000000,1.000000}%
\pgfsetstrokecolor{currentstroke}%
\pgfsetstrokeopacity{0.150000}%
\pgfsetdash{}{0pt}%
\pgfsys@defobject{currentmarker}{\pgfqpoint{0.341129in}{1.750231in}}{\pgfqpoint{2.000000in}{1.830841in}}{%
\pgfpathmoveto{\pgfqpoint{0.341129in}{1.778359in}}%
\pgfpathlineto{\pgfqpoint{0.341129in}{1.750231in}}%
\pgfpathlineto{\pgfqpoint{0.357885in}{1.750885in}}%
\pgfpathlineto{\pgfqpoint{0.374641in}{1.751712in}}%
\pgfpathlineto{\pgfqpoint{0.391398in}{1.752580in}}%
\pgfpathlineto{\pgfqpoint{0.408154in}{1.753343in}}%
\pgfpathlineto{\pgfqpoint{0.424910in}{1.754107in}}%
\pgfpathlineto{\pgfqpoint{0.441666in}{1.754871in}}%
\pgfpathlineto{\pgfqpoint{0.458423in}{1.755635in}}%
\pgfpathlineto{\pgfqpoint{0.475179in}{1.756284in}}%
\pgfpathlineto{\pgfqpoint{0.491935in}{1.756933in}}%
\pgfpathlineto{\pgfqpoint{0.508691in}{1.757582in}}%
\pgfpathlineto{\pgfqpoint{0.525448in}{1.758232in}}%
\pgfpathlineto{\pgfqpoint{0.542204in}{1.758951in}}%
\pgfpathlineto{\pgfqpoint{0.558960in}{1.759675in}}%
\pgfpathlineto{\pgfqpoint{0.575717in}{1.760172in}}%
\pgfpathlineto{\pgfqpoint{0.592473in}{1.760808in}}%
\pgfpathlineto{\pgfqpoint{0.609229in}{1.761465in}}%
\pgfpathlineto{\pgfqpoint{0.625985in}{1.762101in}}%
\pgfpathlineto{\pgfqpoint{0.642742in}{1.762810in}}%
\pgfpathlineto{\pgfqpoint{0.659498in}{1.763532in}}%
\pgfpathlineto{\pgfqpoint{0.676254in}{1.764224in}}%
\pgfpathlineto{\pgfqpoint{0.693011in}{1.764869in}}%
\pgfpathlineto{\pgfqpoint{0.709767in}{1.765481in}}%
\pgfpathlineto{\pgfqpoint{0.726523in}{1.766080in}}%
\pgfpathlineto{\pgfqpoint{0.743279in}{1.766614in}}%
\pgfpathlineto{\pgfqpoint{0.760036in}{1.767191in}}%
\pgfpathlineto{\pgfqpoint{0.776792in}{1.767628in}}%
\pgfpathlineto{\pgfqpoint{0.793548in}{1.768125in}}%
\pgfpathlineto{\pgfqpoint{0.810304in}{1.768663in}}%
\pgfpathlineto{\pgfqpoint{0.827061in}{1.769195in}}%
\pgfpathlineto{\pgfqpoint{0.843817in}{1.769632in}}%
\pgfpathlineto{\pgfqpoint{0.860573in}{1.769951in}}%
\pgfpathlineto{\pgfqpoint{0.877330in}{1.770225in}}%
\pgfpathlineto{\pgfqpoint{0.894086in}{1.770648in}}%
\pgfpathlineto{\pgfqpoint{0.910842in}{1.770883in}}%
\pgfpathlineto{\pgfqpoint{0.927598in}{1.771207in}}%
\pgfpathlineto{\pgfqpoint{0.944355in}{1.771470in}}%
\pgfpathlineto{\pgfqpoint{0.961111in}{1.771806in}}%
\pgfpathlineto{\pgfqpoint{0.977867in}{1.772165in}}%
\pgfpathlineto{\pgfqpoint{0.994623in}{1.772510in}}%
\pgfpathlineto{\pgfqpoint{1.011380in}{1.772734in}}%
\pgfpathlineto{\pgfqpoint{1.028136in}{1.773004in}}%
\pgfpathlineto{\pgfqpoint{1.044892in}{1.773176in}}%
\pgfpathlineto{\pgfqpoint{1.061649in}{1.773467in}}%
\pgfpathlineto{\pgfqpoint{1.078405in}{1.773759in}}%
\pgfpathlineto{\pgfqpoint{1.095161in}{1.774017in}}%
\pgfpathlineto{\pgfqpoint{1.111917in}{1.774331in}}%
\pgfpathlineto{\pgfqpoint{1.128674in}{1.774504in}}%
\pgfpathlineto{\pgfqpoint{1.145430in}{1.774558in}}%
\pgfpathlineto{\pgfqpoint{1.162186in}{1.774616in}}%
\pgfpathlineto{\pgfqpoint{1.178942in}{1.774910in}}%
\pgfpathlineto{\pgfqpoint{1.195699in}{1.775174in}}%
\pgfpathlineto{\pgfqpoint{1.212455in}{1.775101in}}%
\pgfpathlineto{\pgfqpoint{1.229211in}{1.774951in}}%
\pgfpathlineto{\pgfqpoint{1.245968in}{1.775008in}}%
\pgfpathlineto{\pgfqpoint{1.262724in}{1.775162in}}%
\pgfpathlineto{\pgfqpoint{1.279480in}{1.775304in}}%
\pgfpathlineto{\pgfqpoint{1.296236in}{1.775556in}}%
\pgfpathlineto{\pgfqpoint{1.312993in}{1.775732in}}%
\pgfpathlineto{\pgfqpoint{1.329749in}{1.775782in}}%
\pgfpathlineto{\pgfqpoint{1.346505in}{1.775933in}}%
\pgfpathlineto{\pgfqpoint{1.363262in}{1.776084in}}%
\pgfpathlineto{\pgfqpoint{1.380018in}{1.776238in}}%
\pgfpathlineto{\pgfqpoint{1.396774in}{1.776392in}}%
\pgfpathlineto{\pgfqpoint{1.413530in}{1.776492in}}%
\pgfpathlineto{\pgfqpoint{1.430287in}{1.776502in}}%
\pgfpathlineto{\pgfqpoint{1.447043in}{1.776529in}}%
\pgfpathlineto{\pgfqpoint{1.463799in}{1.776629in}}%
\pgfpathlineto{\pgfqpoint{1.480555in}{1.776769in}}%
\pgfpathlineto{\pgfqpoint{1.497312in}{1.776908in}}%
\pgfpathlineto{\pgfqpoint{1.514068in}{1.777048in}}%
\pgfpathlineto{\pgfqpoint{1.530824in}{1.777187in}}%
\pgfpathlineto{\pgfqpoint{1.547581in}{1.777326in}}%
\pgfpathlineto{\pgfqpoint{1.564337in}{1.777466in}}%
\pgfpathlineto{\pgfqpoint{1.581093in}{1.777605in}}%
\pgfpathlineto{\pgfqpoint{1.597849in}{1.777745in}}%
\pgfpathlineto{\pgfqpoint{1.614606in}{1.777884in}}%
\pgfpathlineto{\pgfqpoint{1.631362in}{1.778023in}}%
\pgfpathlineto{\pgfqpoint{1.648118in}{1.778163in}}%
\pgfpathlineto{\pgfqpoint{1.664874in}{1.778302in}}%
\pgfpathlineto{\pgfqpoint{1.681631in}{1.778442in}}%
\pgfpathlineto{\pgfqpoint{1.698387in}{1.778563in}}%
\pgfpathlineto{\pgfqpoint{1.715143in}{1.778646in}}%
\pgfpathlineto{\pgfqpoint{1.731900in}{1.778664in}}%
\pgfpathlineto{\pgfqpoint{1.748656in}{1.778645in}}%
\pgfpathlineto{\pgfqpoint{1.765412in}{1.778748in}}%
\pgfpathlineto{\pgfqpoint{1.782168in}{1.778880in}}%
\pgfpathlineto{\pgfqpoint{1.798925in}{1.779011in}}%
\pgfpathlineto{\pgfqpoint{1.815681in}{1.779143in}}%
\pgfpathlineto{\pgfqpoint{1.832437in}{1.779274in}}%
\pgfpathlineto{\pgfqpoint{1.849193in}{1.779407in}}%
\pgfpathlineto{\pgfqpoint{1.865950in}{1.779542in}}%
\pgfpathlineto{\pgfqpoint{1.882706in}{1.779547in}}%
\pgfpathlineto{\pgfqpoint{1.899462in}{1.779532in}}%
\pgfpathlineto{\pgfqpoint{1.916219in}{1.779518in}}%
\pgfpathlineto{\pgfqpoint{1.932975in}{1.779503in}}%
\pgfpathlineto{\pgfqpoint{1.949731in}{1.779488in}}%
\pgfpathlineto{\pgfqpoint{1.966487in}{1.779611in}}%
\pgfpathlineto{\pgfqpoint{1.983244in}{1.779762in}}%
\pgfpathlineto{\pgfqpoint{2.000000in}{1.779912in}}%
\pgfpathlineto{\pgfqpoint{2.000000in}{1.830841in}}%
\pgfpathlineto{\pgfqpoint{2.000000in}{1.830841in}}%
\pgfpathlineto{\pgfqpoint{1.983244in}{1.830078in}}%
\pgfpathlineto{\pgfqpoint{1.966487in}{1.829309in}}%
\pgfpathlineto{\pgfqpoint{1.949731in}{1.828525in}}%
\pgfpathlineto{\pgfqpoint{1.932975in}{1.827772in}}%
\pgfpathlineto{\pgfqpoint{1.916219in}{1.827005in}}%
\pgfpathlineto{\pgfqpoint{1.899462in}{1.826237in}}%
\pgfpathlineto{\pgfqpoint{1.882706in}{1.825460in}}%
\pgfpathlineto{\pgfqpoint{1.865950in}{1.824631in}}%
\pgfpathlineto{\pgfqpoint{1.849193in}{1.823878in}}%
\pgfpathlineto{\pgfqpoint{1.832437in}{1.823165in}}%
\pgfpathlineto{\pgfqpoint{1.815681in}{1.822400in}}%
\pgfpathlineto{\pgfqpoint{1.798925in}{1.821633in}}%
\pgfpathlineto{\pgfqpoint{1.782168in}{1.820907in}}%
\pgfpathlineto{\pgfqpoint{1.765412in}{1.820305in}}%
\pgfpathlineto{\pgfqpoint{1.748656in}{1.819703in}}%
\pgfpathlineto{\pgfqpoint{1.731900in}{1.818992in}}%
\pgfpathlineto{\pgfqpoint{1.715143in}{1.818260in}}%
\pgfpathlineto{\pgfqpoint{1.698387in}{1.817544in}}%
\pgfpathlineto{\pgfqpoint{1.681631in}{1.816792in}}%
\pgfpathlineto{\pgfqpoint{1.664874in}{1.816157in}}%
\pgfpathlineto{\pgfqpoint{1.648118in}{1.815528in}}%
\pgfpathlineto{\pgfqpoint{1.631362in}{1.814906in}}%
\pgfpathlineto{\pgfqpoint{1.614606in}{1.814121in}}%
\pgfpathlineto{\pgfqpoint{1.597849in}{1.813541in}}%
\pgfpathlineto{\pgfqpoint{1.581093in}{1.812969in}}%
\pgfpathlineto{\pgfqpoint{1.564337in}{1.812170in}}%
\pgfpathlineto{\pgfqpoint{1.547581in}{1.811372in}}%
\pgfpathlineto{\pgfqpoint{1.530824in}{1.810773in}}%
\pgfpathlineto{\pgfqpoint{1.514068in}{1.810219in}}%
\pgfpathlineto{\pgfqpoint{1.497312in}{1.809588in}}%
\pgfpathlineto{\pgfqpoint{1.480555in}{1.809012in}}%
\pgfpathlineto{\pgfqpoint{1.463799in}{1.808332in}}%
\pgfpathlineto{\pgfqpoint{1.447043in}{1.807623in}}%
\pgfpathlineto{\pgfqpoint{1.430287in}{1.806915in}}%
\pgfpathlineto{\pgfqpoint{1.413530in}{1.806202in}}%
\pgfpathlineto{\pgfqpoint{1.396774in}{1.805500in}}%
\pgfpathlineto{\pgfqpoint{1.380018in}{1.804821in}}%
\pgfpathlineto{\pgfqpoint{1.363262in}{1.804071in}}%
\pgfpathlineto{\pgfqpoint{1.346505in}{1.803492in}}%
\pgfpathlineto{\pgfqpoint{1.329749in}{1.802920in}}%
\pgfpathlineto{\pgfqpoint{1.312993in}{1.802221in}}%
\pgfpathlineto{\pgfqpoint{1.296236in}{1.801434in}}%
\pgfpathlineto{\pgfqpoint{1.279480in}{1.800645in}}%
\pgfpathlineto{\pgfqpoint{1.262724in}{1.799977in}}%
\pgfpathlineto{\pgfqpoint{1.245968in}{1.799407in}}%
\pgfpathlineto{\pgfqpoint{1.229211in}{1.798902in}}%
\pgfpathlineto{\pgfqpoint{1.212455in}{1.798205in}}%
\pgfpathlineto{\pgfqpoint{1.195699in}{1.797454in}}%
\pgfpathlineto{\pgfqpoint{1.178942in}{1.796818in}}%
\pgfpathlineto{\pgfqpoint{1.162186in}{1.796156in}}%
\pgfpathlineto{\pgfqpoint{1.145430in}{1.795595in}}%
\pgfpathlineto{\pgfqpoint{1.128674in}{1.795090in}}%
\pgfpathlineto{\pgfqpoint{1.111917in}{1.794571in}}%
\pgfpathlineto{\pgfqpoint{1.095161in}{1.794029in}}%
\pgfpathlineto{\pgfqpoint{1.078405in}{1.793500in}}%
\pgfpathlineto{\pgfqpoint{1.061649in}{1.792965in}}%
\pgfpathlineto{\pgfqpoint{1.044892in}{1.792254in}}%
\pgfpathlineto{\pgfqpoint{1.028136in}{1.791538in}}%
\pgfpathlineto{\pgfqpoint{1.011380in}{1.791115in}}%
\pgfpathlineto{\pgfqpoint{0.994623in}{1.790731in}}%
\pgfpathlineto{\pgfqpoint{0.977867in}{1.790010in}}%
\pgfpathlineto{\pgfqpoint{0.961111in}{1.789548in}}%
\pgfpathlineto{\pgfqpoint{0.944355in}{1.789091in}}%
\pgfpathlineto{\pgfqpoint{0.927598in}{1.788633in}}%
\pgfpathlineto{\pgfqpoint{0.910842in}{1.788176in}}%
\pgfpathlineto{\pgfqpoint{0.894086in}{1.787710in}}%
\pgfpathlineto{\pgfqpoint{0.877330in}{1.787266in}}%
\pgfpathlineto{\pgfqpoint{0.860573in}{1.786844in}}%
\pgfpathlineto{\pgfqpoint{0.843817in}{1.786357in}}%
\pgfpathlineto{\pgfqpoint{0.827061in}{1.785898in}}%
\pgfpathlineto{\pgfqpoint{0.810304in}{1.785445in}}%
\pgfpathlineto{\pgfqpoint{0.793548in}{1.785078in}}%
\pgfpathlineto{\pgfqpoint{0.776792in}{1.784819in}}%
\pgfpathlineto{\pgfqpoint{0.760036in}{1.784310in}}%
\pgfpathlineto{\pgfqpoint{0.743279in}{1.783850in}}%
\pgfpathlineto{\pgfqpoint{0.726523in}{1.783512in}}%
\pgfpathlineto{\pgfqpoint{0.709767in}{1.783106in}}%
\pgfpathlineto{\pgfqpoint{0.693011in}{1.782730in}}%
\pgfpathlineto{\pgfqpoint{0.676254in}{1.782334in}}%
\pgfpathlineto{\pgfqpoint{0.659498in}{1.781858in}}%
\pgfpathlineto{\pgfqpoint{0.642742in}{1.781529in}}%
\pgfpathlineto{\pgfqpoint{0.625985in}{1.781216in}}%
\pgfpathlineto{\pgfqpoint{0.609229in}{1.780872in}}%
\pgfpathlineto{\pgfqpoint{0.592473in}{1.780591in}}%
\pgfpathlineto{\pgfqpoint{0.575717in}{1.780277in}}%
\pgfpathlineto{\pgfqpoint{0.558960in}{1.780097in}}%
\pgfpathlineto{\pgfqpoint{0.542204in}{1.779949in}}%
\pgfpathlineto{\pgfqpoint{0.525448in}{1.779755in}}%
\pgfpathlineto{\pgfqpoint{0.508691in}{1.779513in}}%
\pgfpathlineto{\pgfqpoint{0.491935in}{1.779278in}}%
\pgfpathlineto{\pgfqpoint{0.475179in}{1.779197in}}%
\pgfpathlineto{\pgfqpoint{0.458423in}{1.779168in}}%
\pgfpathlineto{\pgfqpoint{0.441666in}{1.779138in}}%
\pgfpathlineto{\pgfqpoint{0.424910in}{1.779076in}}%
\pgfpathlineto{\pgfqpoint{0.408154in}{1.778919in}}%
\pgfpathlineto{\pgfqpoint{0.391398in}{1.778702in}}%
\pgfpathlineto{\pgfqpoint{0.374641in}{1.778525in}}%
\pgfpathlineto{\pgfqpoint{0.357885in}{1.778442in}}%
\pgfpathlineto{\pgfqpoint{0.341129in}{1.778359in}}%
\pgfpathclose%
\pgfusepath{stroke,fill}%
}%
\begin{pgfscope}%
\pgfsys@transformshift{0.000000in}{0.000000in}%
\pgfsys@useobject{currentmarker}{}%
\end{pgfscope}%
\end{pgfscope}%
\begin{pgfscope}%
\pgfpathrectangle{\pgfqpoint{0.341129in}{0.466613in}}{\pgfqpoint{1.658871in}{1.711598in}}%
\pgfusepath{clip}%
\pgfsetbuttcap%
\pgfsetroundjoin%
\definecolor{currentfill}{rgb}{0.333333,0.658824,0.407843}%
\pgfsetfillcolor{currentfill}%
\pgfsetfillopacity{0.250000}%
\pgfsetlinewidth{1.003750pt}%
\definecolor{currentstroke}{rgb}{0.333333,0.658824,0.407843}%
\pgfsetstrokecolor{currentstroke}%
\pgfsetstrokeopacity{0.250000}%
\pgfsetdash{}{0pt}%
\pgfsys@defobject{currentmarker}{\pgfqpoint{-0.017010in}{-0.017010in}}{\pgfqpoint{0.017010in}{0.017010in}}{%
\pgfpathmoveto{\pgfqpoint{0.000000in}{-0.017010in}}%
\pgfpathcurveto{\pgfqpoint{0.004511in}{-0.017010in}}{\pgfqpoint{0.008838in}{-0.015218in}}{\pgfqpoint{0.012028in}{-0.012028in}}%
\pgfpathcurveto{\pgfqpoint{0.015218in}{-0.008838in}}{\pgfqpoint{0.017010in}{-0.004511in}}{\pgfqpoint{0.017010in}{0.000000in}}%
\pgfpathcurveto{\pgfqpoint{0.017010in}{0.004511in}}{\pgfqpoint{0.015218in}{0.008838in}}{\pgfqpoint{0.012028in}{0.012028in}}%
\pgfpathcurveto{\pgfqpoint{0.008838in}{0.015218in}}{\pgfqpoint{0.004511in}{0.017010in}}{\pgfqpoint{0.000000in}{0.017010in}}%
\pgfpathcurveto{\pgfqpoint{-0.004511in}{0.017010in}}{\pgfqpoint{-0.008838in}{0.015218in}}{\pgfqpoint{-0.012028in}{0.012028in}}%
\pgfpathcurveto{\pgfqpoint{-0.015218in}{0.008838in}}{\pgfqpoint{-0.017010in}{0.004511in}}{\pgfqpoint{-0.017010in}{0.000000in}}%
\pgfpathcurveto{\pgfqpoint{-0.017010in}{-0.004511in}}{\pgfqpoint{-0.015218in}{-0.008838in}}{\pgfqpoint{-0.012028in}{-0.012028in}}%
\pgfpathcurveto{\pgfqpoint{-0.008838in}{-0.015218in}}{\pgfqpoint{-0.004511in}{-0.017010in}}{\pgfqpoint{0.000000in}{-0.017010in}}%
\pgfpathclose%
\pgfusepath{stroke,fill}%
}%
\begin{pgfscope}%
\pgfsys@transformshift{1.684044in}{1.604386in}%
\pgfsys@useobject{currentmarker}{}%
\end{pgfscope}%
\begin{pgfscope}%
\pgfsys@transformshift{1.113006in}{1.615570in}%
\pgfsys@useobject{currentmarker}{}%
\end{pgfscope}%
\begin{pgfscope}%
\pgfsys@transformshift{0.938483in}{1.513186in}%
\pgfsys@useobject{currentmarker}{}%
\end{pgfscope}%
\begin{pgfscope}%
\pgfsys@transformshift{0.492780in}{1.416879in}%
\pgfsys@useobject{currentmarker}{}%
\end{pgfscope}%
\begin{pgfscope}%
\pgfsys@transformshift{0.529797in}{1.348529in}%
\pgfsys@useobject{currentmarker}{}%
\end{pgfscope}%
\begin{pgfscope}%
\pgfsys@transformshift{1.838155in}{1.611492in}%
\pgfsys@useobject{currentmarker}{}%
\end{pgfscope}%
\begin{pgfscope}%
\pgfsys@transformshift{0.477840in}{1.287294in}%
\pgfsys@useobject{currentmarker}{}%
\end{pgfscope}%
\begin{pgfscope}%
\pgfsys@transformshift{0.884452in}{1.671881in}%
\pgfsys@useobject{currentmarker}{}%
\end{pgfscope}%
\begin{pgfscope}%
\pgfsys@transformshift{1.509149in}{1.603966in}%
\pgfsys@useobject{currentmarker}{}%
\end{pgfscope}%
\begin{pgfscope}%
\pgfsys@transformshift{1.183184in}{1.518927in}%
\pgfsys@useobject{currentmarker}{}%
\end{pgfscope}%
\begin{pgfscope}%
\pgfsys@transformshift{1.140467in}{1.454959in}%
\pgfsys@useobject{currentmarker}{}%
\end{pgfscope}%
\begin{pgfscope}%
\pgfsys@transformshift{0.501776in}{1.483030in}%
\pgfsys@useobject{currentmarker}{}%
\end{pgfscope}%
\begin{pgfscope}%
\pgfsys@transformshift{0.461389in}{1.315171in}%
\pgfsys@useobject{currentmarker}{}%
\end{pgfscope}%
\begin{pgfscope}%
\pgfsys@transformshift{0.986905in}{1.518919in}%
\pgfsys@useobject{currentmarker}{}%
\end{pgfscope}%
\begin{pgfscope}%
\pgfsys@transformshift{0.969502in}{1.437366in}%
\pgfsys@useobject{currentmarker}{}%
\end{pgfscope}%
\begin{pgfscope}%
\pgfsys@transformshift{1.016218in}{1.447620in}%
\pgfsys@useobject{currentmarker}{}%
\end{pgfscope}%
\begin{pgfscope}%
\pgfsys@transformshift{0.413211in}{1.525909in}%
\pgfsys@useobject{currentmarker}{}%
\end{pgfscope}%
\begin{pgfscope}%
\pgfsys@transformshift{1.870040in}{1.618968in}%
\pgfsys@useobject{currentmarker}{}%
\end{pgfscope}%
\begin{pgfscope}%
\pgfsys@transformshift{1.031673in}{1.539905in}%
\pgfsys@useobject{currentmarker}{}%
\end{pgfscope}%
\begin{pgfscope}%
\pgfsys@transformshift{1.596314in}{1.548171in}%
\pgfsys@useobject{currentmarker}{}%
\end{pgfscope}%
\begin{pgfscope}%
\pgfsys@transformshift{0.496691in}{1.440595in}%
\pgfsys@useobject{currentmarker}{}%
\end{pgfscope}%
\begin{pgfscope}%
\pgfsys@transformshift{0.896936in}{1.440666in}%
\pgfsys@useobject{currentmarker}{}%
\end{pgfscope}%
\begin{pgfscope}%
\pgfsys@transformshift{0.442231in}{1.454745in}%
\pgfsys@useobject{currentmarker}{}%
\end{pgfscope}%
\begin{pgfscope}%
\pgfsys@transformshift{1.948436in}{1.680987in}%
\pgfsys@useobject{currentmarker}{}%
\end{pgfscope}%
\begin{pgfscope}%
\pgfsys@transformshift{0.607246in}{1.432636in}%
\pgfsys@useobject{currentmarker}{}%
\end{pgfscope}%
\begin{pgfscope}%
\pgfsys@transformshift{0.591373in}{1.163092in}%
\pgfsys@useobject{currentmarker}{}%
\end{pgfscope}%
\begin{pgfscope}%
\pgfsys@transformshift{0.495569in}{1.391788in}%
\pgfsys@useobject{currentmarker}{}%
\end{pgfscope}%
\begin{pgfscope}%
\pgfsys@transformshift{0.805700in}{1.349455in}%
\pgfsys@useobject{currentmarker}{}%
\end{pgfscope}%
\begin{pgfscope}%
\pgfsys@transformshift{0.472915in}{1.493328in}%
\pgfsys@useobject{currentmarker}{}%
\end{pgfscope}%
\begin{pgfscope}%
\pgfsys@transformshift{1.008191in}{1.501668in}%
\pgfsys@useobject{currentmarker}{}%
\end{pgfscope}%
\begin{pgfscope}%
\pgfsys@transformshift{0.795900in}{1.601000in}%
\pgfsys@useobject{currentmarker}{}%
\end{pgfscope}%
\begin{pgfscope}%
\pgfsys@transformshift{0.879596in}{1.612705in}%
\pgfsys@useobject{currentmarker}{}%
\end{pgfscope}%
\begin{pgfscope}%
\pgfsys@transformshift{0.851399in}{1.276415in}%
\pgfsys@useobject{currentmarker}{}%
\end{pgfscope}%
\begin{pgfscope}%
\pgfsys@transformshift{1.630362in}{1.593350in}%
\pgfsys@useobject{currentmarker}{}%
\end{pgfscope}%
\begin{pgfscope}%
\pgfsys@transformshift{0.497926in}{1.471706in}%
\pgfsys@useobject{currentmarker}{}%
\end{pgfscope}%
\begin{pgfscope}%
\pgfsys@transformshift{1.716622in}{1.323516in}%
\pgfsys@useobject{currentmarker}{}%
\end{pgfscope}%
\begin{pgfscope}%
\pgfsys@transformshift{0.524809in}{1.624478in}%
\pgfsys@useobject{currentmarker}{}%
\end{pgfscope}%
\begin{pgfscope}%
\pgfsys@transformshift{1.821229in}{1.649096in}%
\pgfsys@useobject{currentmarker}{}%
\end{pgfscope}%
\begin{pgfscope}%
\pgfsys@transformshift{0.925546in}{1.585633in}%
\pgfsys@useobject{currentmarker}{}%
\end{pgfscope}%
\begin{pgfscope}%
\pgfsys@transformshift{1.841684in}{1.317357in}%
\pgfsys@useobject{currentmarker}{}%
\end{pgfscope}%
\begin{pgfscope}%
\pgfsys@transformshift{1.522493in}{1.562821in}%
\pgfsys@useobject{currentmarker}{}%
\end{pgfscope}%
\begin{pgfscope}%
\pgfsys@transformshift{0.409285in}{1.532482in}%
\pgfsys@useobject{currentmarker}{}%
\end{pgfscope}%
\begin{pgfscope}%
\pgfsys@transformshift{0.896047in}{1.341515in}%
\pgfsys@useobject{currentmarker}{}%
\end{pgfscope}%
\begin{pgfscope}%
\pgfsys@transformshift{0.382462in}{1.505933in}%
\pgfsys@useobject{currentmarker}{}%
\end{pgfscope}%
\begin{pgfscope}%
\pgfsys@transformshift{0.855764in}{1.349124in}%
\pgfsys@useobject{currentmarker}{}%
\end{pgfscope}%
\begin{pgfscope}%
\pgfsys@transformshift{0.523733in}{1.276969in}%
\pgfsys@useobject{currentmarker}{}%
\end{pgfscope}%
\begin{pgfscope}%
\pgfsys@transformshift{0.470197in}{1.664582in}%
\pgfsys@useobject{currentmarker}{}%
\end{pgfscope}%
\begin{pgfscope}%
\pgfsys@transformshift{0.829991in}{1.554814in}%
\pgfsys@useobject{currentmarker}{}%
\end{pgfscope}%
\begin{pgfscope}%
\pgfsys@transformshift{0.933266in}{1.306707in}%
\pgfsys@useobject{currentmarker}{}%
\end{pgfscope}%
\begin{pgfscope}%
\pgfsys@transformshift{0.956789in}{1.712968in}%
\pgfsys@useobject{currentmarker}{}%
\end{pgfscope}%
\begin{pgfscope}%
\pgfsys@transformshift{0.936472in}{1.514444in}%
\pgfsys@useobject{currentmarker}{}%
\end{pgfscope}%
\begin{pgfscope}%
\pgfsys@transformshift{0.468794in}{1.631055in}%
\pgfsys@useobject{currentmarker}{}%
\end{pgfscope}%
\begin{pgfscope}%
\pgfsys@transformshift{0.402384in}{1.613793in}%
\pgfsys@useobject{currentmarker}{}%
\end{pgfscope}%
\begin{pgfscope}%
\pgfsys@transformshift{1.177456in}{1.489676in}%
\pgfsys@useobject{currentmarker}{}%
\end{pgfscope}%
\begin{pgfscope}%
\pgfsys@transformshift{0.416018in}{1.368569in}%
\pgfsys@useobject{currentmarker}{}%
\end{pgfscope}%
\begin{pgfscope}%
\pgfsys@transformshift{0.447008in}{1.634128in}%
\pgfsys@useobject{currentmarker}{}%
\end{pgfscope}%
\begin{pgfscope}%
\pgfsys@transformshift{0.549911in}{1.359375in}%
\pgfsys@useobject{currentmarker}{}%
\end{pgfscope}%
\begin{pgfscope}%
\pgfsys@transformshift{1.065128in}{1.623104in}%
\pgfsys@useobject{currentmarker}{}%
\end{pgfscope}%
\begin{pgfscope}%
\pgfsys@transformshift{0.450053in}{1.383877in}%
\pgfsys@useobject{currentmarker}{}%
\end{pgfscope}%
\begin{pgfscope}%
\pgfsys@transformshift{0.455142in}{1.400285in}%
\pgfsys@useobject{currentmarker}{}%
\end{pgfscope}%
\begin{pgfscope}%
\pgfsys@transformshift{0.484478in}{1.420215in}%
\pgfsys@useobject{currentmarker}{}%
\end{pgfscope}%
\begin{pgfscope}%
\pgfsys@transformshift{1.875283in}{1.721011in}%
\pgfsys@useobject{currentmarker}{}%
\end{pgfscope}%
\begin{pgfscope}%
\pgfsys@transformshift{0.375023in}{1.507777in}%
\pgfsys@useobject{currentmarker}{}%
\end{pgfscope}%
\begin{pgfscope}%
\pgfsys@transformshift{0.853143in}{1.310661in}%
\pgfsys@useobject{currentmarker}{}%
\end{pgfscope}%
\begin{pgfscope}%
\pgfsys@transformshift{0.436365in}{1.374891in}%
\pgfsys@useobject{currentmarker}{}%
\end{pgfscope}%
\begin{pgfscope}%
\pgfsys@transformshift{0.491797in}{1.328522in}%
\pgfsys@useobject{currentmarker}{}%
\end{pgfscope}%
\begin{pgfscope}%
\pgfsys@transformshift{0.902491in}{1.376423in}%
\pgfsys@useobject{currentmarker}{}%
\end{pgfscope}%
\begin{pgfscope}%
\pgfsys@transformshift{0.978414in}{1.505426in}%
\pgfsys@useobject{currentmarker}{}%
\end{pgfscope}%
\begin{pgfscope}%
\pgfsys@transformshift{0.944248in}{1.535808in}%
\pgfsys@useobject{currentmarker}{}%
\end{pgfscope}%
\begin{pgfscope}%
\pgfsys@transformshift{1.817503in}{1.228280in}%
\pgfsys@useobject{currentmarker}{}%
\end{pgfscope}%
\begin{pgfscope}%
\pgfsys@transformshift{1.721814in}{1.564469in}%
\pgfsys@useobject{currentmarker}{}%
\end{pgfscope}%
\begin{pgfscope}%
\pgfsys@transformshift{0.476494in}{1.685641in}%
\pgfsys@useobject{currentmarker}{}%
\end{pgfscope}%
\begin{pgfscope}%
\pgfsys@transformshift{0.942312in}{1.486255in}%
\pgfsys@useobject{currentmarker}{}%
\end{pgfscope}%
\begin{pgfscope}%
\pgfsys@transformshift{0.975555in}{1.472874in}%
\pgfsys@useobject{currentmarker}{}%
\end{pgfscope}%
\begin{pgfscope}%
\pgfsys@transformshift{1.642091in}{1.577801in}%
\pgfsys@useobject{currentmarker}{}%
\end{pgfscope}%
\begin{pgfscope}%
\pgfsys@transformshift{0.910804in}{1.571977in}%
\pgfsys@useobject{currentmarker}{}%
\end{pgfscope}%
\begin{pgfscope}%
\pgfsys@transformshift{0.507435in}{1.336925in}%
\pgfsys@useobject{currentmarker}{}%
\end{pgfscope}%
\begin{pgfscope}%
\pgfsys@transformshift{1.946413in}{1.293073in}%
\pgfsys@useobject{currentmarker}{}%
\end{pgfscope}%
\begin{pgfscope}%
\pgfsys@transformshift{1.639910in}{1.581876in}%
\pgfsys@useobject{currentmarker}{}%
\end{pgfscope}%
\begin{pgfscope}%
\pgfsys@transformshift{1.020077in}{1.546742in}%
\pgfsys@useobject{currentmarker}{}%
\end{pgfscope}%
\begin{pgfscope}%
\pgfsys@transformshift{1.837777in}{1.607863in}%
\pgfsys@useobject{currentmarker}{}%
\end{pgfscope}%
\begin{pgfscope}%
\pgfsys@transformshift{0.546434in}{1.309222in}%
\pgfsys@useobject{currentmarker}{}%
\end{pgfscope}%
\begin{pgfscope}%
\pgfsys@transformshift{0.507573in}{1.471006in}%
\pgfsys@useobject{currentmarker}{}%
\end{pgfscope}%
\begin{pgfscope}%
\pgfsys@transformshift{0.995047in}{1.448924in}%
\pgfsys@useobject{currentmarker}{}%
\end{pgfscope}%
\begin{pgfscope}%
\pgfsys@transformshift{0.507253in}{1.450818in}%
\pgfsys@useobject{currentmarker}{}%
\end{pgfscope}%
\begin{pgfscope}%
\pgfsys@transformshift{0.844423in}{1.427494in}%
\pgfsys@useobject{currentmarker}{}%
\end{pgfscope}%
\begin{pgfscope}%
\pgfsys@transformshift{1.175457in}{1.723651in}%
\pgfsys@useobject{currentmarker}{}%
\end{pgfscope}%
\begin{pgfscope}%
\pgfsys@transformshift{1.024973in}{1.330531in}%
\pgfsys@useobject{currentmarker}{}%
\end{pgfscope}%
\begin{pgfscope}%
\pgfsys@transformshift{1.150522in}{1.465268in}%
\pgfsys@useobject{currentmarker}{}%
\end{pgfscope}%
\begin{pgfscope}%
\pgfsys@transformshift{0.562256in}{1.498042in}%
\pgfsys@useobject{currentmarker}{}%
\end{pgfscope}%
\begin{pgfscope}%
\pgfsys@transformshift{1.482648in}{1.480125in}%
\pgfsys@useobject{currentmarker}{}%
\end{pgfscope}%
\begin{pgfscope}%
\pgfsys@transformshift{1.406514in}{1.505611in}%
\pgfsys@useobject{currentmarker}{}%
\end{pgfscope}%
\begin{pgfscope}%
\pgfsys@transformshift{1.080077in}{1.420347in}%
\pgfsys@useobject{currentmarker}{}%
\end{pgfscope}%
\begin{pgfscope}%
\pgfsys@transformshift{0.989979in}{1.741645in}%
\pgfsys@useobject{currentmarker}{}%
\end{pgfscope}%
\begin{pgfscope}%
\pgfsys@transformshift{0.987626in}{1.589132in}%
\pgfsys@useobject{currentmarker}{}%
\end{pgfscope}%
\begin{pgfscope}%
\pgfsys@transformshift{1.771689in}{1.725873in}%
\pgfsys@useobject{currentmarker}{}%
\end{pgfscope}%
\begin{pgfscope}%
\pgfsys@transformshift{0.891632in}{1.385061in}%
\pgfsys@useobject{currentmarker}{}%
\end{pgfscope}%
\begin{pgfscope}%
\pgfsys@transformshift{0.870836in}{1.640484in}%
\pgfsys@useobject{currentmarker}{}%
\end{pgfscope}%
\begin{pgfscope}%
\pgfsys@transformshift{1.667984in}{1.645310in}%
\pgfsys@useobject{currentmarker}{}%
\end{pgfscope}%
\begin{pgfscope}%
\pgfsys@transformshift{1.813990in}{1.770212in}%
\pgfsys@useobject{currentmarker}{}%
\end{pgfscope}%
\begin{pgfscope}%
\pgfsys@transformshift{0.403236in}{1.529337in}%
\pgfsys@useobject{currentmarker}{}%
\end{pgfscope}%
\begin{pgfscope}%
\pgfsys@transformshift{0.996537in}{1.564008in}%
\pgfsys@useobject{currentmarker}{}%
\end{pgfscope}%
\begin{pgfscope}%
\pgfsys@transformshift{1.043417in}{1.520038in}%
\pgfsys@useobject{currentmarker}{}%
\end{pgfscope}%
\begin{pgfscope}%
\pgfsys@transformshift{0.467968in}{1.396653in}%
\pgfsys@useobject{currentmarker}{}%
\end{pgfscope}%
\begin{pgfscope}%
\pgfsys@transformshift{1.915346in}{1.249974in}%
\pgfsys@useobject{currentmarker}{}%
\end{pgfscope}%
\begin{pgfscope}%
\pgfsys@transformshift{0.437600in}{1.534814in}%
\pgfsys@useobject{currentmarker}{}%
\end{pgfscope}%
\begin{pgfscope}%
\pgfsys@transformshift{0.438013in}{1.404350in}%
\pgfsys@useobject{currentmarker}{}%
\end{pgfscope}%
\begin{pgfscope}%
\pgfsys@transformshift{0.958228in}{1.313771in}%
\pgfsys@useobject{currentmarker}{}%
\end{pgfscope}%
\begin{pgfscope}%
\pgfsys@transformshift{1.012755in}{1.255940in}%
\pgfsys@useobject{currentmarker}{}%
\end{pgfscope}%
\begin{pgfscope}%
\pgfsys@transformshift{0.406225in}{1.379641in}%
\pgfsys@useobject{currentmarker}{}%
\end{pgfscope}%
\begin{pgfscope}%
\pgfsys@transformshift{0.473582in}{1.491067in}%
\pgfsys@useobject{currentmarker}{}%
\end{pgfscope}%
\begin{pgfscope}%
\pgfsys@transformshift{0.590756in}{1.282791in}%
\pgfsys@useobject{currentmarker}{}%
\end{pgfscope}%
\begin{pgfscope}%
\pgfsys@transformshift{0.918416in}{1.184177in}%
\pgfsys@useobject{currentmarker}{}%
\end{pgfscope}%
\begin{pgfscope}%
\pgfsys@transformshift{0.435231in}{1.416584in}%
\pgfsys@useobject{currentmarker}{}%
\end{pgfscope}%
\begin{pgfscope}%
\pgfsys@transformshift{1.100847in}{1.554159in}%
\pgfsys@useobject{currentmarker}{}%
\end{pgfscope}%
\begin{pgfscope}%
\pgfsys@transformshift{1.783869in}{1.683943in}%
\pgfsys@useobject{currentmarker}{}%
\end{pgfscope}%
\begin{pgfscope}%
\pgfsys@transformshift{0.881476in}{1.687763in}%
\pgfsys@useobject{currentmarker}{}%
\end{pgfscope}%
\begin{pgfscope}%
\pgfsys@transformshift{1.004126in}{1.279606in}%
\pgfsys@useobject{currentmarker}{}%
\end{pgfscope}%
\begin{pgfscope}%
\pgfsys@transformshift{1.494969in}{1.610839in}%
\pgfsys@useobject{currentmarker}{}%
\end{pgfscope}%
\begin{pgfscope}%
\pgfsys@transformshift{0.486970in}{1.346548in}%
\pgfsys@useobject{currentmarker}{}%
\end{pgfscope}%
\begin{pgfscope}%
\pgfsys@transformshift{0.439439in}{1.586684in}%
\pgfsys@useobject{currentmarker}{}%
\end{pgfscope}%
\begin{pgfscope}%
\pgfsys@transformshift{0.879557in}{1.607925in}%
\pgfsys@useobject{currentmarker}{}%
\end{pgfscope}%
\begin{pgfscope}%
\pgfsys@transformshift{0.852216in}{1.469420in}%
\pgfsys@useobject{currentmarker}{}%
\end{pgfscope}%
\begin{pgfscope}%
\pgfsys@transformshift{0.950782in}{1.448781in}%
\pgfsys@useobject{currentmarker}{}%
\end{pgfscope}%
\begin{pgfscope}%
\pgfsys@transformshift{0.502562in}{1.550546in}%
\pgfsys@useobject{currentmarker}{}%
\end{pgfscope}%
\begin{pgfscope}%
\pgfsys@transformshift{1.067095in}{1.560353in}%
\pgfsys@useobject{currentmarker}{}%
\end{pgfscope}%
\begin{pgfscope}%
\pgfsys@transformshift{0.478759in}{1.606930in}%
\pgfsys@useobject{currentmarker}{}%
\end{pgfscope}%
\begin{pgfscope}%
\pgfsys@transformshift{0.967133in}{1.541934in}%
\pgfsys@useobject{currentmarker}{}%
\end{pgfscope}%
\begin{pgfscope}%
\pgfsys@transformshift{0.508090in}{1.463959in}%
\pgfsys@useobject{currentmarker}{}%
\end{pgfscope}%
\begin{pgfscope}%
\pgfsys@transformshift{0.435594in}{1.443522in}%
\pgfsys@useobject{currentmarker}{}%
\end{pgfscope}%
\begin{pgfscope}%
\pgfsys@transformshift{0.980355in}{1.338376in}%
\pgfsys@useobject{currentmarker}{}%
\end{pgfscope}%
\begin{pgfscope}%
\pgfsys@transformshift{1.552149in}{1.522293in}%
\pgfsys@useobject{currentmarker}{}%
\end{pgfscope}%
\begin{pgfscope}%
\pgfsys@transformshift{0.875226in}{1.322526in}%
\pgfsys@useobject{currentmarker}{}%
\end{pgfscope}%
\begin{pgfscope}%
\pgfsys@transformshift{1.107273in}{1.524912in}%
\pgfsys@useobject{currentmarker}{}%
\end{pgfscope}%
\begin{pgfscope}%
\pgfsys@transformshift{0.523782in}{1.599574in}%
\pgfsys@useobject{currentmarker}{}%
\end{pgfscope}%
\begin{pgfscope}%
\pgfsys@transformshift{1.045306in}{1.632937in}%
\pgfsys@useobject{currentmarker}{}%
\end{pgfscope}%
\begin{pgfscope}%
\pgfsys@transformshift{1.526749in}{1.465758in}%
\pgfsys@useobject{currentmarker}{}%
\end{pgfscope}%
\begin{pgfscope}%
\pgfsys@transformshift{1.541666in}{1.601650in}%
\pgfsys@useobject{currentmarker}{}%
\end{pgfscope}%
\begin{pgfscope}%
\pgfsys@transformshift{1.574267in}{1.589769in}%
\pgfsys@useobject{currentmarker}{}%
\end{pgfscope}%
\begin{pgfscope}%
\pgfsys@transformshift{1.639614in}{1.575128in}%
\pgfsys@useobject{currentmarker}{}%
\end{pgfscope}%
\begin{pgfscope}%
\pgfsys@transformshift{0.881908in}{1.673947in}%
\pgfsys@useobject{currentmarker}{}%
\end{pgfscope}%
\begin{pgfscope}%
\pgfsys@transformshift{1.677526in}{1.674974in}%
\pgfsys@useobject{currentmarker}{}%
\end{pgfscope}%
\begin{pgfscope}%
\pgfsys@transformshift{0.483826in}{1.473371in}%
\pgfsys@useobject{currentmarker}{}%
\end{pgfscope}%
\begin{pgfscope}%
\pgfsys@transformshift{0.448297in}{1.508917in}%
\pgfsys@useobject{currentmarker}{}%
\end{pgfscope}%
\begin{pgfscope}%
\pgfsys@transformshift{0.976510in}{1.463634in}%
\pgfsys@useobject{currentmarker}{}%
\end{pgfscope}%
\begin{pgfscope}%
\pgfsys@transformshift{0.481727in}{1.418117in}%
\pgfsys@useobject{currentmarker}{}%
\end{pgfscope}%
\begin{pgfscope}%
\pgfsys@transformshift{1.808114in}{1.586026in}%
\pgfsys@useobject{currentmarker}{}%
\end{pgfscope}%
\begin{pgfscope}%
\pgfsys@transformshift{0.468390in}{1.435146in}%
\pgfsys@useobject{currentmarker}{}%
\end{pgfscope}%
\begin{pgfscope}%
\pgfsys@transformshift{0.458999in}{1.462355in}%
\pgfsys@useobject{currentmarker}{}%
\end{pgfscope}%
\begin{pgfscope}%
\pgfsys@transformshift{1.685818in}{1.553737in}%
\pgfsys@useobject{currentmarker}{}%
\end{pgfscope}%
\begin{pgfscope}%
\pgfsys@transformshift{0.635418in}{1.327380in}%
\pgfsys@useobject{currentmarker}{}%
\end{pgfscope}%
\begin{pgfscope}%
\pgfsys@transformshift{0.415977in}{1.345664in}%
\pgfsys@useobject{currentmarker}{}%
\end{pgfscope}%
\begin{pgfscope}%
\pgfsys@transformshift{0.341129in}{1.407131in}%
\pgfsys@useobject{currentmarker}{}%
\end{pgfscope}%
\begin{pgfscope}%
\pgfsys@transformshift{0.992976in}{1.533900in}%
\pgfsys@useobject{currentmarker}{}%
\end{pgfscope}%
\begin{pgfscope}%
\pgfsys@transformshift{0.883182in}{1.466294in}%
\pgfsys@useobject{currentmarker}{}%
\end{pgfscope}%
\begin{pgfscope}%
\pgfsys@transformshift{0.480881in}{1.571247in}%
\pgfsys@useobject{currentmarker}{}%
\end{pgfscope}%
\begin{pgfscope}%
\pgfsys@transformshift{1.244411in}{1.470424in}%
\pgfsys@useobject{currentmarker}{}%
\end{pgfscope}%
\begin{pgfscope}%
\pgfsys@transformshift{1.224203in}{1.516796in}%
\pgfsys@useobject{currentmarker}{}%
\end{pgfscope}%
\begin{pgfscope}%
\pgfsys@transformshift{0.449848in}{1.645582in}%
\pgfsys@useobject{currentmarker}{}%
\end{pgfscope}%
\begin{pgfscope}%
\pgfsys@transformshift{0.582527in}{1.307189in}%
\pgfsys@useobject{currentmarker}{}%
\end{pgfscope}%
\begin{pgfscope}%
\pgfsys@transformshift{1.714524in}{1.756530in}%
\pgfsys@useobject{currentmarker}{}%
\end{pgfscope}%
\begin{pgfscope}%
\pgfsys@transformshift{0.891264in}{1.599423in}%
\pgfsys@useobject{currentmarker}{}%
\end{pgfscope}%
\begin{pgfscope}%
\pgfsys@transformshift{1.857615in}{1.703566in}%
\pgfsys@useobject{currentmarker}{}%
\end{pgfscope}%
\begin{pgfscope}%
\pgfsys@transformshift{1.511327in}{1.567509in}%
\pgfsys@useobject{currentmarker}{}%
\end{pgfscope}%
\begin{pgfscope}%
\pgfsys@transformshift{1.748915in}{1.372758in}%
\pgfsys@useobject{currentmarker}{}%
\end{pgfscope}%
\begin{pgfscope}%
\pgfsys@transformshift{0.491937in}{1.503835in}%
\pgfsys@useobject{currentmarker}{}%
\end{pgfscope}%
\begin{pgfscope}%
\pgfsys@transformshift{0.556274in}{1.417246in}%
\pgfsys@useobject{currentmarker}{}%
\end{pgfscope}%
\begin{pgfscope}%
\pgfsys@transformshift{0.468664in}{1.626446in}%
\pgfsys@useobject{currentmarker}{}%
\end{pgfscope}%
\begin{pgfscope}%
\pgfsys@transformshift{1.550098in}{1.556922in}%
\pgfsys@useobject{currentmarker}{}%
\end{pgfscope}%
\begin{pgfscope}%
\pgfsys@transformshift{0.533554in}{1.416271in}%
\pgfsys@useobject{currentmarker}{}%
\end{pgfscope}%
\begin{pgfscope}%
\pgfsys@transformshift{1.009532in}{1.495837in}%
\pgfsys@useobject{currentmarker}{}%
\end{pgfscope}%
\begin{pgfscope}%
\pgfsys@transformshift{0.590179in}{1.291307in}%
\pgfsys@useobject{currentmarker}{}%
\end{pgfscope}%
\begin{pgfscope}%
\pgfsys@transformshift{0.432340in}{1.667493in}%
\pgfsys@useobject{currentmarker}{}%
\end{pgfscope}%
\begin{pgfscope}%
\pgfsys@transformshift{0.372906in}{1.487928in}%
\pgfsys@useobject{currentmarker}{}%
\end{pgfscope}%
\begin{pgfscope}%
\pgfsys@transformshift{0.459645in}{1.519389in}%
\pgfsys@useobject{currentmarker}{}%
\end{pgfscope}%
\begin{pgfscope}%
\pgfsys@transformshift{1.482135in}{1.466421in}%
\pgfsys@useobject{currentmarker}{}%
\end{pgfscope}%
\begin{pgfscope}%
\pgfsys@transformshift{0.622605in}{1.303139in}%
\pgfsys@useobject{currentmarker}{}%
\end{pgfscope}%
\begin{pgfscope}%
\pgfsys@transformshift{0.489705in}{1.537801in}%
\pgfsys@useobject{currentmarker}{}%
\end{pgfscope}%
\begin{pgfscope}%
\pgfsys@transformshift{1.102334in}{1.398170in}%
\pgfsys@useobject{currentmarker}{}%
\end{pgfscope}%
\begin{pgfscope}%
\pgfsys@transformshift{0.877887in}{1.358357in}%
\pgfsys@useobject{currentmarker}{}%
\end{pgfscope}%
\begin{pgfscope}%
\pgfsys@transformshift{0.519403in}{1.439494in}%
\pgfsys@useobject{currentmarker}{}%
\end{pgfscope}%
\begin{pgfscope}%
\pgfsys@transformshift{0.938030in}{1.364247in}%
\pgfsys@useobject{currentmarker}{}%
\end{pgfscope}%
\begin{pgfscope}%
\pgfsys@transformshift{0.961331in}{1.214445in}%
\pgfsys@useobject{currentmarker}{}%
\end{pgfscope}%
\begin{pgfscope}%
\pgfsys@transformshift{1.036665in}{1.636122in}%
\pgfsys@useobject{currentmarker}{}%
\end{pgfscope}%
\begin{pgfscope}%
\pgfsys@transformshift{0.628071in}{1.448644in}%
\pgfsys@useobject{currentmarker}{}%
\end{pgfscope}%
\begin{pgfscope}%
\pgfsys@transformshift{0.835422in}{1.654995in}%
\pgfsys@useobject{currentmarker}{}%
\end{pgfscope}%
\begin{pgfscope}%
\pgfsys@transformshift{1.066394in}{1.569414in}%
\pgfsys@useobject{currentmarker}{}%
\end{pgfscope}%
\begin{pgfscope}%
\pgfsys@transformshift{1.436938in}{1.481223in}%
\pgfsys@useobject{currentmarker}{}%
\end{pgfscope}%
\begin{pgfscope}%
\pgfsys@transformshift{1.603771in}{1.521121in}%
\pgfsys@useobject{currentmarker}{}%
\end{pgfscope}%
\begin{pgfscope}%
\pgfsys@transformshift{0.959028in}{1.373536in}%
\pgfsys@useobject{currentmarker}{}%
\end{pgfscope}%
\begin{pgfscope}%
\pgfsys@transformshift{0.432442in}{1.481824in}%
\pgfsys@useobject{currentmarker}{}%
\end{pgfscope}%
\begin{pgfscope}%
\pgfsys@transformshift{0.897917in}{1.577967in}%
\pgfsys@useobject{currentmarker}{}%
\end{pgfscope}%
\begin{pgfscope}%
\pgfsys@transformshift{0.894441in}{1.123165in}%
\pgfsys@useobject{currentmarker}{}%
\end{pgfscope}%
\begin{pgfscope}%
\pgfsys@transformshift{0.401578in}{1.625000in}%
\pgfsys@useobject{currentmarker}{}%
\end{pgfscope}%
\begin{pgfscope}%
\pgfsys@transformshift{0.411375in}{1.489220in}%
\pgfsys@useobject{currentmarker}{}%
\end{pgfscope}%
\begin{pgfscope}%
\pgfsys@transformshift{0.874473in}{1.327780in}%
\pgfsys@useobject{currentmarker}{}%
\end{pgfscope}%
\begin{pgfscope}%
\pgfsys@transformshift{0.989940in}{1.469697in}%
\pgfsys@useobject{currentmarker}{}%
\end{pgfscope}%
\begin{pgfscope}%
\pgfsys@transformshift{0.410574in}{1.667936in}%
\pgfsys@useobject{currentmarker}{}%
\end{pgfscope}%
\begin{pgfscope}%
\pgfsys@transformshift{0.887331in}{1.401679in}%
\pgfsys@useobject{currentmarker}{}%
\end{pgfscope}%
\begin{pgfscope}%
\pgfsys@transformshift{0.844030in}{1.348547in}%
\pgfsys@useobject{currentmarker}{}%
\end{pgfscope}%
\begin{pgfscope}%
\pgfsys@transformshift{0.923946in}{1.384711in}%
\pgfsys@useobject{currentmarker}{}%
\end{pgfscope}%
\begin{pgfscope}%
\pgfsys@transformshift{1.517373in}{1.558527in}%
\pgfsys@useobject{currentmarker}{}%
\end{pgfscope}%
\begin{pgfscope}%
\pgfsys@transformshift{1.544776in}{1.617817in}%
\pgfsys@useobject{currentmarker}{}%
\end{pgfscope}%
\begin{pgfscope}%
\pgfsys@transformshift{0.466736in}{1.326912in}%
\pgfsys@useobject{currentmarker}{}%
\end{pgfscope}%
\begin{pgfscope}%
\pgfsys@transformshift{0.539327in}{1.229530in}%
\pgfsys@useobject{currentmarker}{}%
\end{pgfscope}%
\begin{pgfscope}%
\pgfsys@transformshift{1.837625in}{1.624140in}%
\pgfsys@useobject{currentmarker}{}%
\end{pgfscope}%
\begin{pgfscope}%
\pgfsys@transformshift{0.442869in}{1.460160in}%
\pgfsys@useobject{currentmarker}{}%
\end{pgfscope}%
\begin{pgfscope}%
\pgfsys@transformshift{1.136179in}{1.687986in}%
\pgfsys@useobject{currentmarker}{}%
\end{pgfscope}%
\begin{pgfscope}%
\pgfsys@transformshift{0.540575in}{1.514487in}%
\pgfsys@useobject{currentmarker}{}%
\end{pgfscope}%
\begin{pgfscope}%
\pgfsys@transformshift{0.505594in}{1.310991in}%
\pgfsys@useobject{currentmarker}{}%
\end{pgfscope}%
\begin{pgfscope}%
\pgfsys@transformshift{0.464353in}{1.312782in}%
\pgfsys@useobject{currentmarker}{}%
\end{pgfscope}%
\begin{pgfscope}%
\pgfsys@transformshift{0.448024in}{1.460367in}%
\pgfsys@useobject{currentmarker}{}%
\end{pgfscope}%
\begin{pgfscope}%
\pgfsys@transformshift{1.133798in}{1.598844in}%
\pgfsys@useobject{currentmarker}{}%
\end{pgfscope}%
\begin{pgfscope}%
\pgfsys@transformshift{0.450472in}{1.420211in}%
\pgfsys@useobject{currentmarker}{}%
\end{pgfscope}%
\begin{pgfscope}%
\pgfsys@transformshift{0.915981in}{1.164978in}%
\pgfsys@useobject{currentmarker}{}%
\end{pgfscope}%
\begin{pgfscope}%
\pgfsys@transformshift{0.893518in}{1.566392in}%
\pgfsys@useobject{currentmarker}{}%
\end{pgfscope}%
\begin{pgfscope}%
\pgfsys@transformshift{0.839737in}{1.268702in}%
\pgfsys@useobject{currentmarker}{}%
\end{pgfscope}%
\begin{pgfscope}%
\pgfsys@transformshift{0.466023in}{1.515535in}%
\pgfsys@useobject{currentmarker}{}%
\end{pgfscope}%
\begin{pgfscope}%
\pgfsys@transformshift{1.451837in}{1.563220in}%
\pgfsys@useobject{currentmarker}{}%
\end{pgfscope}%
\begin{pgfscope}%
\pgfsys@transformshift{1.812241in}{1.345581in}%
\pgfsys@useobject{currentmarker}{}%
\end{pgfscope}%
\begin{pgfscope}%
\pgfsys@transformshift{0.470694in}{1.483342in}%
\pgfsys@useobject{currentmarker}{}%
\end{pgfscope}%
\begin{pgfscope}%
\pgfsys@transformshift{0.373662in}{1.506685in}%
\pgfsys@useobject{currentmarker}{}%
\end{pgfscope}%
\begin{pgfscope}%
\pgfsys@transformshift{0.403930in}{1.595645in}%
\pgfsys@useobject{currentmarker}{}%
\end{pgfscope}%
\begin{pgfscope}%
\pgfsys@transformshift{1.148968in}{1.708676in}%
\pgfsys@useobject{currentmarker}{}%
\end{pgfscope}%
\begin{pgfscope}%
\pgfsys@transformshift{0.466738in}{1.490892in}%
\pgfsys@useobject{currentmarker}{}%
\end{pgfscope}%
\begin{pgfscope}%
\pgfsys@transformshift{0.600280in}{1.571452in}%
\pgfsys@useobject{currentmarker}{}%
\end{pgfscope}%
\begin{pgfscope}%
\pgfsys@transformshift{0.430403in}{1.544093in}%
\pgfsys@useobject{currentmarker}{}%
\end{pgfscope}%
\begin{pgfscope}%
\pgfsys@transformshift{0.917645in}{1.544650in}%
\pgfsys@useobject{currentmarker}{}%
\end{pgfscope}%
\begin{pgfscope}%
\pgfsys@transformshift{1.972216in}{1.318866in}%
\pgfsys@useobject{currentmarker}{}%
\end{pgfscope}%
\begin{pgfscope}%
\pgfsys@transformshift{0.947192in}{1.576678in}%
\pgfsys@useobject{currentmarker}{}%
\end{pgfscope}%
\begin{pgfscope}%
\pgfsys@transformshift{0.945142in}{1.328970in}%
\pgfsys@useobject{currentmarker}{}%
\end{pgfscope}%
\begin{pgfscope}%
\pgfsys@transformshift{0.884523in}{1.350651in}%
\pgfsys@useobject{currentmarker}{}%
\end{pgfscope}%
\begin{pgfscope}%
\pgfsys@transformshift{0.868046in}{1.393679in}%
\pgfsys@useobject{currentmarker}{}%
\end{pgfscope}%
\begin{pgfscope}%
\pgfsys@transformshift{0.427823in}{1.546355in}%
\pgfsys@useobject{currentmarker}{}%
\end{pgfscope}%
\begin{pgfscope}%
\pgfsys@transformshift{1.240509in}{1.583247in}%
\pgfsys@useobject{currentmarker}{}%
\end{pgfscope}%
\begin{pgfscope}%
\pgfsys@transformshift{0.427367in}{1.733607in}%
\pgfsys@useobject{currentmarker}{}%
\end{pgfscope}%
\begin{pgfscope}%
\pgfsys@transformshift{0.877911in}{1.347984in}%
\pgfsys@useobject{currentmarker}{}%
\end{pgfscope}%
\begin{pgfscope}%
\pgfsys@transformshift{0.502381in}{1.325642in}%
\pgfsys@useobject{currentmarker}{}%
\end{pgfscope}%
\begin{pgfscope}%
\pgfsys@transformshift{0.897409in}{1.173501in}%
\pgfsys@useobject{currentmarker}{}%
\end{pgfscope}%
\begin{pgfscope}%
\pgfsys@transformshift{0.371513in}{1.379730in}%
\pgfsys@useobject{currentmarker}{}%
\end{pgfscope}%
\begin{pgfscope}%
\pgfsys@transformshift{0.424068in}{1.540110in}%
\pgfsys@useobject{currentmarker}{}%
\end{pgfscope}%
\begin{pgfscope}%
\pgfsys@transformshift{0.512683in}{1.123584in}%
\pgfsys@useobject{currentmarker}{}%
\end{pgfscope}%
\begin{pgfscope}%
\pgfsys@transformshift{0.423224in}{1.428819in}%
\pgfsys@useobject{currentmarker}{}%
\end{pgfscope}%
\begin{pgfscope}%
\pgfsys@transformshift{0.590216in}{1.332452in}%
\pgfsys@useobject{currentmarker}{}%
\end{pgfscope}%
\begin{pgfscope}%
\pgfsys@transformshift{1.015493in}{1.382847in}%
\pgfsys@useobject{currentmarker}{}%
\end{pgfscope}%
\begin{pgfscope}%
\pgfsys@transformshift{1.527374in}{1.575952in}%
\pgfsys@useobject{currentmarker}{}%
\end{pgfscope}%
\begin{pgfscope}%
\pgfsys@transformshift{1.794630in}{1.182647in}%
\pgfsys@useobject{currentmarker}{}%
\end{pgfscope}%
\begin{pgfscope}%
\pgfsys@transformshift{0.971721in}{1.578240in}%
\pgfsys@useobject{currentmarker}{}%
\end{pgfscope}%
\begin{pgfscope}%
\pgfsys@transformshift{0.402429in}{1.487207in}%
\pgfsys@useobject{currentmarker}{}%
\end{pgfscope}%
\begin{pgfscope}%
\pgfsys@transformshift{0.596209in}{1.321540in}%
\pgfsys@useobject{currentmarker}{}%
\end{pgfscope}%
\begin{pgfscope}%
\pgfsys@transformshift{0.596444in}{1.473987in}%
\pgfsys@useobject{currentmarker}{}%
\end{pgfscope}%
\begin{pgfscope}%
\pgfsys@transformshift{0.472945in}{1.479016in}%
\pgfsys@useobject{currentmarker}{}%
\end{pgfscope}%
\begin{pgfscope}%
\pgfsys@transformshift{0.505729in}{1.576334in}%
\pgfsys@useobject{currentmarker}{}%
\end{pgfscope}%
\begin{pgfscope}%
\pgfsys@transformshift{0.522392in}{1.430552in}%
\pgfsys@useobject{currentmarker}{}%
\end{pgfscope}%
\begin{pgfscope}%
\pgfsys@transformshift{0.958437in}{1.557151in}%
\pgfsys@useobject{currentmarker}{}%
\end{pgfscope}%
\begin{pgfscope}%
\pgfsys@transformshift{0.899312in}{1.670783in}%
\pgfsys@useobject{currentmarker}{}%
\end{pgfscope}%
\begin{pgfscope}%
\pgfsys@transformshift{0.477796in}{1.667000in}%
\pgfsys@useobject{currentmarker}{}%
\end{pgfscope}%
\begin{pgfscope}%
\pgfsys@transformshift{0.495644in}{1.492277in}%
\pgfsys@useobject{currentmarker}{}%
\end{pgfscope}%
\begin{pgfscope}%
\pgfsys@transformshift{0.461404in}{1.571833in}%
\pgfsys@useobject{currentmarker}{}%
\end{pgfscope}%
\begin{pgfscope}%
\pgfsys@transformshift{0.564846in}{1.505915in}%
\pgfsys@useobject{currentmarker}{}%
\end{pgfscope}%
\begin{pgfscope}%
\pgfsys@transformshift{0.438201in}{1.317212in}%
\pgfsys@useobject{currentmarker}{}%
\end{pgfscope}%
\begin{pgfscope}%
\pgfsys@transformshift{0.830917in}{1.626132in}%
\pgfsys@useobject{currentmarker}{}%
\end{pgfscope}%
\begin{pgfscope}%
\pgfsys@transformshift{0.470836in}{1.346926in}%
\pgfsys@useobject{currentmarker}{}%
\end{pgfscope}%
\begin{pgfscope}%
\pgfsys@transformshift{1.732430in}{1.592839in}%
\pgfsys@useobject{currentmarker}{}%
\end{pgfscope}%
\begin{pgfscope}%
\pgfsys@transformshift{0.880194in}{1.276970in}%
\pgfsys@useobject{currentmarker}{}%
\end{pgfscope}%
\begin{pgfscope}%
\pgfsys@transformshift{1.771020in}{1.307690in}%
\pgfsys@useobject{currentmarker}{}%
\end{pgfscope}%
\begin{pgfscope}%
\pgfsys@transformshift{0.408541in}{1.667983in}%
\pgfsys@useobject{currentmarker}{}%
\end{pgfscope}%
\begin{pgfscope}%
\pgfsys@transformshift{1.783013in}{1.309273in}%
\pgfsys@useobject{currentmarker}{}%
\end{pgfscope}%
\begin{pgfscope}%
\pgfsys@transformshift{1.734433in}{1.324880in}%
\pgfsys@useobject{currentmarker}{}%
\end{pgfscope}%
\begin{pgfscope}%
\pgfsys@transformshift{1.772028in}{1.751908in}%
\pgfsys@useobject{currentmarker}{}%
\end{pgfscope}%
\begin{pgfscope}%
\pgfsys@transformshift{0.482187in}{1.693863in}%
\pgfsys@useobject{currentmarker}{}%
\end{pgfscope}%
\begin{pgfscope}%
\pgfsys@transformshift{0.850634in}{1.173113in}%
\pgfsys@useobject{currentmarker}{}%
\end{pgfscope}%
\begin{pgfscope}%
\pgfsys@transformshift{1.489929in}{1.483645in}%
\pgfsys@useobject{currentmarker}{}%
\end{pgfscope}%
\begin{pgfscope}%
\pgfsys@transformshift{0.572833in}{1.358351in}%
\pgfsys@useobject{currentmarker}{}%
\end{pgfscope}%
\begin{pgfscope}%
\pgfsys@transformshift{0.459424in}{1.483327in}%
\pgfsys@useobject{currentmarker}{}%
\end{pgfscope}%
\begin{pgfscope}%
\pgfsys@transformshift{0.454844in}{1.450815in}%
\pgfsys@useobject{currentmarker}{}%
\end{pgfscope}%
\begin{pgfscope}%
\pgfsys@transformshift{1.529823in}{1.537107in}%
\pgfsys@useobject{currentmarker}{}%
\end{pgfscope}%
\begin{pgfscope}%
\pgfsys@transformshift{0.477367in}{1.616219in}%
\pgfsys@useobject{currentmarker}{}%
\end{pgfscope}%
\begin{pgfscope}%
\pgfsys@transformshift{1.049797in}{1.476670in}%
\pgfsys@useobject{currentmarker}{}%
\end{pgfscope}%
\begin{pgfscope}%
\pgfsys@transformshift{1.390925in}{1.832216in}%
\pgfsys@useobject{currentmarker}{}%
\end{pgfscope}%
\begin{pgfscope}%
\pgfsys@transformshift{1.615976in}{1.317544in}%
\pgfsys@useobject{currentmarker}{}%
\end{pgfscope}%
\begin{pgfscope}%
\pgfsys@transformshift{1.879685in}{1.351928in}%
\pgfsys@useobject{currentmarker}{}%
\end{pgfscope}%
\begin{pgfscope}%
\pgfsys@transformshift{0.474522in}{1.494684in}%
\pgfsys@useobject{currentmarker}{}%
\end{pgfscope}%
\begin{pgfscope}%
\pgfsys@transformshift{0.508097in}{1.619967in}%
\pgfsys@useobject{currentmarker}{}%
\end{pgfscope}%
\begin{pgfscope}%
\pgfsys@transformshift{0.846995in}{1.277633in}%
\pgfsys@useobject{currentmarker}{}%
\end{pgfscope}%
\begin{pgfscope}%
\pgfsys@transformshift{1.720538in}{1.649407in}%
\pgfsys@useobject{currentmarker}{}%
\end{pgfscope}%
\begin{pgfscope}%
\pgfsys@transformshift{1.561110in}{1.511367in}%
\pgfsys@useobject{currentmarker}{}%
\end{pgfscope}%
\begin{pgfscope}%
\pgfsys@transformshift{0.477188in}{1.635868in}%
\pgfsys@useobject{currentmarker}{}%
\end{pgfscope}%
\begin{pgfscope}%
\pgfsys@transformshift{0.462555in}{1.555511in}%
\pgfsys@useobject{currentmarker}{}%
\end{pgfscope}%
\begin{pgfscope}%
\pgfsys@transformshift{1.923906in}{1.679306in}%
\pgfsys@useobject{currentmarker}{}%
\end{pgfscope}%
\begin{pgfscope}%
\pgfsys@transformshift{0.990232in}{1.612599in}%
\pgfsys@useobject{currentmarker}{}%
\end{pgfscope}%
\begin{pgfscope}%
\pgfsys@transformshift{0.858173in}{1.311585in}%
\pgfsys@useobject{currentmarker}{}%
\end{pgfscope}%
\begin{pgfscope}%
\pgfsys@transformshift{0.955286in}{1.657939in}%
\pgfsys@useobject{currentmarker}{}%
\end{pgfscope}%
\begin{pgfscope}%
\pgfsys@transformshift{1.614084in}{1.677255in}%
\pgfsys@useobject{currentmarker}{}%
\end{pgfscope}%
\begin{pgfscope}%
\pgfsys@transformshift{0.839480in}{1.645083in}%
\pgfsys@useobject{currentmarker}{}%
\end{pgfscope}%
\begin{pgfscope}%
\pgfsys@transformshift{0.579938in}{1.537911in}%
\pgfsys@useobject{currentmarker}{}%
\end{pgfscope}%
\begin{pgfscope}%
\pgfsys@transformshift{1.021308in}{1.488693in}%
\pgfsys@useobject{currentmarker}{}%
\end{pgfscope}%
\begin{pgfscope}%
\pgfsys@transformshift{0.489542in}{1.537800in}%
\pgfsys@useobject{currentmarker}{}%
\end{pgfscope}%
\begin{pgfscope}%
\pgfsys@transformshift{0.482496in}{1.348078in}%
\pgfsys@useobject{currentmarker}{}%
\end{pgfscope}%
\begin{pgfscope}%
\pgfsys@transformshift{1.231263in}{1.520989in}%
\pgfsys@useobject{currentmarker}{}%
\end{pgfscope}%
\begin{pgfscope}%
\pgfsys@transformshift{0.612346in}{1.425892in}%
\pgfsys@useobject{currentmarker}{}%
\end{pgfscope}%
\begin{pgfscope}%
\pgfsys@transformshift{0.596517in}{1.322157in}%
\pgfsys@useobject{currentmarker}{}%
\end{pgfscope}%
\begin{pgfscope}%
\pgfsys@transformshift{1.011703in}{1.297711in}%
\pgfsys@useobject{currentmarker}{}%
\end{pgfscope}%
\begin{pgfscope}%
\pgfsys@transformshift{0.536110in}{1.320235in}%
\pgfsys@useobject{currentmarker}{}%
\end{pgfscope}%
\begin{pgfscope}%
\pgfsys@transformshift{0.473641in}{1.488964in}%
\pgfsys@useobject{currentmarker}{}%
\end{pgfscope}%
\begin{pgfscope}%
\pgfsys@transformshift{0.949604in}{1.395519in}%
\pgfsys@useobject{currentmarker}{}%
\end{pgfscope}%
\begin{pgfscope}%
\pgfsys@transformshift{1.891487in}{1.728531in}%
\pgfsys@useobject{currentmarker}{}%
\end{pgfscope}%
\begin{pgfscope}%
\pgfsys@transformshift{0.459142in}{1.608031in}%
\pgfsys@useobject{currentmarker}{}%
\end{pgfscope}%
\begin{pgfscope}%
\pgfsys@transformshift{0.993745in}{1.626422in}%
\pgfsys@useobject{currentmarker}{}%
\end{pgfscope}%
\begin{pgfscope}%
\pgfsys@transformshift{0.461010in}{1.592620in}%
\pgfsys@useobject{currentmarker}{}%
\end{pgfscope}%
\begin{pgfscope}%
\pgfsys@transformshift{0.452198in}{1.535536in}%
\pgfsys@useobject{currentmarker}{}%
\end{pgfscope}%
\begin{pgfscope}%
\pgfsys@transformshift{0.567084in}{1.498031in}%
\pgfsys@useobject{currentmarker}{}%
\end{pgfscope}%
\begin{pgfscope}%
\pgfsys@transformshift{0.925975in}{1.648938in}%
\pgfsys@useobject{currentmarker}{}%
\end{pgfscope}%
\begin{pgfscope}%
\pgfsys@transformshift{1.694241in}{1.195227in}%
\pgfsys@useobject{currentmarker}{}%
\end{pgfscope}%
\begin{pgfscope}%
\pgfsys@transformshift{0.589923in}{1.264753in}%
\pgfsys@useobject{currentmarker}{}%
\end{pgfscope}%
\begin{pgfscope}%
\pgfsys@transformshift{1.421991in}{1.515806in}%
\pgfsys@useobject{currentmarker}{}%
\end{pgfscope}%
\begin{pgfscope}%
\pgfsys@transformshift{0.569884in}{1.559808in}%
\pgfsys@useobject{currentmarker}{}%
\end{pgfscope}%
\begin{pgfscope}%
\pgfsys@transformshift{0.481143in}{1.071482in}%
\pgfsys@useobject{currentmarker}{}%
\end{pgfscope}%
\begin{pgfscope}%
\pgfsys@transformshift{0.598950in}{1.168543in}%
\pgfsys@useobject{currentmarker}{}%
\end{pgfscope}%
\begin{pgfscope}%
\pgfsys@transformshift{0.891632in}{1.235088in}%
\pgfsys@useobject{currentmarker}{}%
\end{pgfscope}%
\begin{pgfscope}%
\pgfsys@transformshift{1.518904in}{1.591859in}%
\pgfsys@useobject{currentmarker}{}%
\end{pgfscope}%
\begin{pgfscope}%
\pgfsys@transformshift{1.267712in}{1.497394in}%
\pgfsys@useobject{currentmarker}{}%
\end{pgfscope}%
\begin{pgfscope}%
\pgfsys@transformshift{1.736581in}{1.820008in}%
\pgfsys@useobject{currentmarker}{}%
\end{pgfscope}%
\begin{pgfscope}%
\pgfsys@transformshift{0.507870in}{1.438491in}%
\pgfsys@useobject{currentmarker}{}%
\end{pgfscope}%
\begin{pgfscope}%
\pgfsys@transformshift{0.876979in}{1.428494in}%
\pgfsys@useobject{currentmarker}{}%
\end{pgfscope}%
\begin{pgfscope}%
\pgfsys@transformshift{0.975052in}{1.620266in}%
\pgfsys@useobject{currentmarker}{}%
\end{pgfscope}%
\begin{pgfscope}%
\pgfsys@transformshift{0.581391in}{1.414532in}%
\pgfsys@useobject{currentmarker}{}%
\end{pgfscope}%
\begin{pgfscope}%
\pgfsys@transformshift{0.432905in}{1.478224in}%
\pgfsys@useobject{currentmarker}{}%
\end{pgfscope}%
\begin{pgfscope}%
\pgfsys@transformshift{0.887125in}{1.661528in}%
\pgfsys@useobject{currentmarker}{}%
\end{pgfscope}%
\begin{pgfscope}%
\pgfsys@transformshift{0.887888in}{1.202589in}%
\pgfsys@useobject{currentmarker}{}%
\end{pgfscope}%
\begin{pgfscope}%
\pgfsys@transformshift{1.047522in}{1.552929in}%
\pgfsys@useobject{currentmarker}{}%
\end{pgfscope}%
\begin{pgfscope}%
\pgfsys@transformshift{0.459882in}{1.757535in}%
\pgfsys@useobject{currentmarker}{}%
\end{pgfscope}%
\begin{pgfscope}%
\pgfsys@transformshift{0.869741in}{1.214669in}%
\pgfsys@useobject{currentmarker}{}%
\end{pgfscope}%
\begin{pgfscope}%
\pgfsys@transformshift{1.519451in}{1.519203in}%
\pgfsys@useobject{currentmarker}{}%
\end{pgfscope}%
\begin{pgfscope}%
\pgfsys@transformshift{1.704307in}{1.597269in}%
\pgfsys@useobject{currentmarker}{}%
\end{pgfscope}%
\begin{pgfscope}%
\pgfsys@transformshift{1.029054in}{1.533234in}%
\pgfsys@useobject{currentmarker}{}%
\end{pgfscope}%
\begin{pgfscope}%
\pgfsys@transformshift{0.983134in}{1.452531in}%
\pgfsys@useobject{currentmarker}{}%
\end{pgfscope}%
\begin{pgfscope}%
\pgfsys@transformshift{1.345503in}{1.620768in}%
\pgfsys@useobject{currentmarker}{}%
\end{pgfscope}%
\begin{pgfscope}%
\pgfsys@transformshift{0.480154in}{1.670298in}%
\pgfsys@useobject{currentmarker}{}%
\end{pgfscope}%
\begin{pgfscope}%
\pgfsys@transformshift{0.820400in}{1.263974in}%
\pgfsys@useobject{currentmarker}{}%
\end{pgfscope}%
\begin{pgfscope}%
\pgfsys@transformshift{0.890253in}{1.652967in}%
\pgfsys@useobject{currentmarker}{}%
\end{pgfscope}%
\begin{pgfscope}%
\pgfsys@transformshift{1.194868in}{1.729369in}%
\pgfsys@useobject{currentmarker}{}%
\end{pgfscope}%
\begin{pgfscope}%
\pgfsys@transformshift{0.938368in}{1.777388in}%
\pgfsys@useobject{currentmarker}{}%
\end{pgfscope}%
\begin{pgfscope}%
\pgfsys@transformshift{0.663011in}{1.253584in}%
\pgfsys@useobject{currentmarker}{}%
\end{pgfscope}%
\begin{pgfscope}%
\pgfsys@transformshift{0.916322in}{1.155424in}%
\pgfsys@useobject{currentmarker}{}%
\end{pgfscope}%
\begin{pgfscope}%
\pgfsys@transformshift{0.890291in}{1.426241in}%
\pgfsys@useobject{currentmarker}{}%
\end{pgfscope}%
\begin{pgfscope}%
\pgfsys@transformshift{0.909236in}{1.171602in}%
\pgfsys@useobject{currentmarker}{}%
\end{pgfscope}%
\begin{pgfscope}%
\pgfsys@transformshift{0.862619in}{1.273680in}%
\pgfsys@useobject{currentmarker}{}%
\end{pgfscope}%
\begin{pgfscope}%
\pgfsys@transformshift{0.597487in}{1.590281in}%
\pgfsys@useobject{currentmarker}{}%
\end{pgfscope}%
\begin{pgfscope}%
\pgfsys@transformshift{1.964345in}{1.325141in}%
\pgfsys@useobject{currentmarker}{}%
\end{pgfscope}%
\begin{pgfscope}%
\pgfsys@transformshift{1.692774in}{1.297537in}%
\pgfsys@useobject{currentmarker}{}%
\end{pgfscope}%
\begin{pgfscope}%
\pgfsys@transformshift{0.482531in}{1.549918in}%
\pgfsys@useobject{currentmarker}{}%
\end{pgfscope}%
\begin{pgfscope}%
\pgfsys@transformshift{0.804579in}{1.209072in}%
\pgfsys@useobject{currentmarker}{}%
\end{pgfscope}%
\begin{pgfscope}%
\pgfsys@transformshift{0.458204in}{1.458464in}%
\pgfsys@useobject{currentmarker}{}%
\end{pgfscope}%
\begin{pgfscope}%
\pgfsys@transformshift{1.029005in}{1.479446in}%
\pgfsys@useobject{currentmarker}{}%
\end{pgfscope}%
\begin{pgfscope}%
\pgfsys@transformshift{0.509849in}{1.262712in}%
\pgfsys@useobject{currentmarker}{}%
\end{pgfscope}%
\begin{pgfscope}%
\pgfsys@transformshift{1.034049in}{1.464033in}%
\pgfsys@useobject{currentmarker}{}%
\end{pgfscope}%
\begin{pgfscope}%
\pgfsys@transformshift{1.152583in}{1.598254in}%
\pgfsys@useobject{currentmarker}{}%
\end{pgfscope}%
\begin{pgfscope}%
\pgfsys@transformshift{1.044006in}{1.592420in}%
\pgfsys@useobject{currentmarker}{}%
\end{pgfscope}%
\begin{pgfscope}%
\pgfsys@transformshift{1.486794in}{1.626323in}%
\pgfsys@useobject{currentmarker}{}%
\end{pgfscope}%
\begin{pgfscope}%
\pgfsys@transformshift{0.870195in}{1.297183in}%
\pgfsys@useobject{currentmarker}{}%
\end{pgfscope}%
\begin{pgfscope}%
\pgfsys@transformshift{1.138693in}{1.595983in}%
\pgfsys@useobject{currentmarker}{}%
\end{pgfscope}%
\begin{pgfscope}%
\pgfsys@transformshift{0.847963in}{1.214719in}%
\pgfsys@useobject{currentmarker}{}%
\end{pgfscope}%
\begin{pgfscope}%
\pgfsys@transformshift{0.477983in}{1.535186in}%
\pgfsys@useobject{currentmarker}{}%
\end{pgfscope}%
\begin{pgfscope}%
\pgfsys@transformshift{0.492721in}{1.444552in}%
\pgfsys@useobject{currentmarker}{}%
\end{pgfscope}%
\begin{pgfscope}%
\pgfsys@transformshift{1.120030in}{1.578355in}%
\pgfsys@useobject{currentmarker}{}%
\end{pgfscope}%
\begin{pgfscope}%
\pgfsys@transformshift{0.895919in}{1.457010in}%
\pgfsys@useobject{currentmarker}{}%
\end{pgfscope}%
\begin{pgfscope}%
\pgfsys@transformshift{0.576124in}{1.286299in}%
\pgfsys@useobject{currentmarker}{}%
\end{pgfscope}%
\begin{pgfscope}%
\pgfsys@transformshift{0.471843in}{1.434037in}%
\pgfsys@useobject{currentmarker}{}%
\end{pgfscope}%
\begin{pgfscope}%
\pgfsys@transformshift{0.475316in}{1.362133in}%
\pgfsys@useobject{currentmarker}{}%
\end{pgfscope}%
\begin{pgfscope}%
\pgfsys@transformshift{1.653595in}{1.600380in}%
\pgfsys@useobject{currentmarker}{}%
\end{pgfscope}%
\begin{pgfscope}%
\pgfsys@transformshift{1.083484in}{1.533543in}%
\pgfsys@useobject{currentmarker}{}%
\end{pgfscope}%
\begin{pgfscope}%
\pgfsys@transformshift{0.929704in}{1.517529in}%
\pgfsys@useobject{currentmarker}{}%
\end{pgfscope}%
\begin{pgfscope}%
\pgfsys@transformshift{1.702011in}{1.667600in}%
\pgfsys@useobject{currentmarker}{}%
\end{pgfscope}%
\begin{pgfscope}%
\pgfsys@transformshift{0.861736in}{1.161262in}%
\pgfsys@useobject{currentmarker}{}%
\end{pgfscope}%
\begin{pgfscope}%
\pgfsys@transformshift{0.466758in}{1.395915in}%
\pgfsys@useobject{currentmarker}{}%
\end{pgfscope}%
\begin{pgfscope}%
\pgfsys@transformshift{0.895642in}{1.414540in}%
\pgfsys@useobject{currentmarker}{}%
\end{pgfscope}%
\begin{pgfscope}%
\pgfsys@transformshift{0.844526in}{1.471135in}%
\pgfsys@useobject{currentmarker}{}%
\end{pgfscope}%
\begin{pgfscope}%
\pgfsys@transformshift{0.903057in}{1.656013in}%
\pgfsys@useobject{currentmarker}{}%
\end{pgfscope}%
\begin{pgfscope}%
\pgfsys@transformshift{0.952685in}{1.375862in}%
\pgfsys@useobject{currentmarker}{}%
\end{pgfscope}%
\begin{pgfscope}%
\pgfsys@transformshift{0.576385in}{1.482607in}%
\pgfsys@useobject{currentmarker}{}%
\end{pgfscope}%
\begin{pgfscope}%
\pgfsys@transformshift{1.016020in}{1.549229in}%
\pgfsys@useobject{currentmarker}{}%
\end{pgfscope}%
\begin{pgfscope}%
\pgfsys@transformshift{0.494156in}{1.305933in}%
\pgfsys@useobject{currentmarker}{}%
\end{pgfscope}%
\begin{pgfscope}%
\pgfsys@transformshift{0.577397in}{1.254269in}%
\pgfsys@useobject{currentmarker}{}%
\end{pgfscope}%
\begin{pgfscope}%
\pgfsys@transformshift{0.445519in}{1.412236in}%
\pgfsys@useobject{currentmarker}{}%
\end{pgfscope}%
\begin{pgfscope}%
\pgfsys@transformshift{1.084282in}{1.484491in}%
\pgfsys@useobject{currentmarker}{}%
\end{pgfscope}%
\begin{pgfscope}%
\pgfsys@transformshift{1.594805in}{1.543972in}%
\pgfsys@useobject{currentmarker}{}%
\end{pgfscope}%
\begin{pgfscope}%
\pgfsys@transformshift{0.976071in}{1.555751in}%
\pgfsys@useobject{currentmarker}{}%
\end{pgfscope}%
\begin{pgfscope}%
\pgfsys@transformshift{0.954554in}{1.670164in}%
\pgfsys@useobject{currentmarker}{}%
\end{pgfscope}%
\begin{pgfscope}%
\pgfsys@transformshift{1.062412in}{1.705316in}%
\pgfsys@useobject{currentmarker}{}%
\end{pgfscope}%
\begin{pgfscope}%
\pgfsys@transformshift{0.403708in}{1.339211in}%
\pgfsys@useobject{currentmarker}{}%
\end{pgfscope}%
\begin{pgfscope}%
\pgfsys@transformshift{0.456382in}{1.460811in}%
\pgfsys@useobject{currentmarker}{}%
\end{pgfscope}%
\begin{pgfscope}%
\pgfsys@transformshift{0.425389in}{1.308582in}%
\pgfsys@useobject{currentmarker}{}%
\end{pgfscope}%
\begin{pgfscope}%
\pgfsys@transformshift{1.178319in}{1.346365in}%
\pgfsys@useobject{currentmarker}{}%
\end{pgfscope}%
\begin{pgfscope}%
\pgfsys@transformshift{0.868624in}{1.632518in}%
\pgfsys@useobject{currentmarker}{}%
\end{pgfscope}%
\begin{pgfscope}%
\pgfsys@transformshift{1.072938in}{1.503364in}%
\pgfsys@useobject{currentmarker}{}%
\end{pgfscope}%
\begin{pgfscope}%
\pgfsys@transformshift{0.842975in}{1.282793in}%
\pgfsys@useobject{currentmarker}{}%
\end{pgfscope}%
\begin{pgfscope}%
\pgfsys@transformshift{1.133541in}{1.565901in}%
\pgfsys@useobject{currentmarker}{}%
\end{pgfscope}%
\begin{pgfscope}%
\pgfsys@transformshift{0.985168in}{1.671507in}%
\pgfsys@useobject{currentmarker}{}%
\end{pgfscope}%
\begin{pgfscope}%
\pgfsys@transformshift{0.935163in}{1.750266in}%
\pgfsys@useobject{currentmarker}{}%
\end{pgfscope}%
\begin{pgfscope}%
\pgfsys@transformshift{0.852856in}{1.305743in}%
\pgfsys@useobject{currentmarker}{}%
\end{pgfscope}%
\begin{pgfscope}%
\pgfsys@transformshift{0.481479in}{1.295675in}%
\pgfsys@useobject{currentmarker}{}%
\end{pgfscope}%
\begin{pgfscope}%
\pgfsys@transformshift{0.483502in}{1.575224in}%
\pgfsys@useobject{currentmarker}{}%
\end{pgfscope}%
\begin{pgfscope}%
\pgfsys@transformshift{0.857024in}{1.128849in}%
\pgfsys@useobject{currentmarker}{}%
\end{pgfscope}%
\begin{pgfscope}%
\pgfsys@transformshift{0.541865in}{1.284162in}%
\pgfsys@useobject{currentmarker}{}%
\end{pgfscope}%
\begin{pgfscope}%
\pgfsys@transformshift{1.564506in}{1.629186in}%
\pgfsys@useobject{currentmarker}{}%
\end{pgfscope}%
\begin{pgfscope}%
\pgfsys@transformshift{0.926665in}{1.648337in}%
\pgfsys@useobject{currentmarker}{}%
\end{pgfscope}%
\begin{pgfscope}%
\pgfsys@transformshift{0.371504in}{1.526770in}%
\pgfsys@useobject{currentmarker}{}%
\end{pgfscope}%
\begin{pgfscope}%
\pgfsys@transformshift{0.869520in}{1.448404in}%
\pgfsys@useobject{currentmarker}{}%
\end{pgfscope}%
\begin{pgfscope}%
\pgfsys@transformshift{0.519844in}{1.314515in}%
\pgfsys@useobject{currentmarker}{}%
\end{pgfscope}%
\begin{pgfscope}%
\pgfsys@transformshift{0.608978in}{1.244374in}%
\pgfsys@useobject{currentmarker}{}%
\end{pgfscope}%
\begin{pgfscope}%
\pgfsys@transformshift{0.476814in}{1.416219in}%
\pgfsys@useobject{currentmarker}{}%
\end{pgfscope}%
\begin{pgfscope}%
\pgfsys@transformshift{0.933077in}{1.545297in}%
\pgfsys@useobject{currentmarker}{}%
\end{pgfscope}%
\begin{pgfscope}%
\pgfsys@transformshift{0.579351in}{1.483830in}%
\pgfsys@useobject{currentmarker}{}%
\end{pgfscope}%
\begin{pgfscope}%
\pgfsys@transformshift{0.459383in}{1.443812in}%
\pgfsys@useobject{currentmarker}{}%
\end{pgfscope}%
\begin{pgfscope}%
\pgfsys@transformshift{0.451992in}{1.465298in}%
\pgfsys@useobject{currentmarker}{}%
\end{pgfscope}%
\begin{pgfscope}%
\pgfsys@transformshift{0.448649in}{1.588618in}%
\pgfsys@useobject{currentmarker}{}%
\end{pgfscope}%
\begin{pgfscope}%
\pgfsys@transformshift{1.803636in}{1.712847in}%
\pgfsys@useobject{currentmarker}{}%
\end{pgfscope}%
\begin{pgfscope}%
\pgfsys@transformshift{1.680858in}{1.227498in}%
\pgfsys@useobject{currentmarker}{}%
\end{pgfscope}%
\begin{pgfscope}%
\pgfsys@transformshift{0.464450in}{1.434464in}%
\pgfsys@useobject{currentmarker}{}%
\end{pgfscope}%
\begin{pgfscope}%
\pgfsys@transformshift{0.467818in}{1.422629in}%
\pgfsys@useobject{currentmarker}{}%
\end{pgfscope}%
\begin{pgfscope}%
\pgfsys@transformshift{1.797702in}{1.746074in}%
\pgfsys@useobject{currentmarker}{}%
\end{pgfscope}%
\begin{pgfscope}%
\pgfsys@transformshift{1.800535in}{1.668087in}%
\pgfsys@useobject{currentmarker}{}%
\end{pgfscope}%
\begin{pgfscope}%
\pgfsys@transformshift{0.486053in}{1.385945in}%
\pgfsys@useobject{currentmarker}{}%
\end{pgfscope}%
\begin{pgfscope}%
\pgfsys@transformshift{0.604133in}{1.244300in}%
\pgfsys@useobject{currentmarker}{}%
\end{pgfscope}%
\begin{pgfscope}%
\pgfsys@transformshift{1.510193in}{1.511529in}%
\pgfsys@useobject{currentmarker}{}%
\end{pgfscope}%
\begin{pgfscope}%
\pgfsys@transformshift{0.969856in}{1.569106in}%
\pgfsys@useobject{currentmarker}{}%
\end{pgfscope}%
\begin{pgfscope}%
\pgfsys@transformshift{0.883908in}{1.718167in}%
\pgfsys@useobject{currentmarker}{}%
\end{pgfscope}%
\begin{pgfscope}%
\pgfsys@transformshift{0.538449in}{1.272291in}%
\pgfsys@useobject{currentmarker}{}%
\end{pgfscope}%
\begin{pgfscope}%
\pgfsys@transformshift{0.561398in}{1.399698in}%
\pgfsys@useobject{currentmarker}{}%
\end{pgfscope}%
\begin{pgfscope}%
\pgfsys@transformshift{0.984012in}{1.593969in}%
\pgfsys@useobject{currentmarker}{}%
\end{pgfscope}%
\begin{pgfscope}%
\pgfsys@transformshift{0.614232in}{1.576147in}%
\pgfsys@useobject{currentmarker}{}%
\end{pgfscope}%
\begin{pgfscope}%
\pgfsys@transformshift{1.178616in}{1.622142in}%
\pgfsys@useobject{currentmarker}{}%
\end{pgfscope}%
\begin{pgfscope}%
\pgfsys@transformshift{0.638320in}{1.303890in}%
\pgfsys@useobject{currentmarker}{}%
\end{pgfscope}%
\begin{pgfscope}%
\pgfsys@transformshift{1.571164in}{1.508121in}%
\pgfsys@useobject{currentmarker}{}%
\end{pgfscope}%
\begin{pgfscope}%
\pgfsys@transformshift{0.449499in}{1.612832in}%
\pgfsys@useobject{currentmarker}{}%
\end{pgfscope}%
\begin{pgfscope}%
\pgfsys@transformshift{1.000812in}{1.774126in}%
\pgfsys@useobject{currentmarker}{}%
\end{pgfscope}%
\begin{pgfscope}%
\pgfsys@transformshift{0.504036in}{1.424221in}%
\pgfsys@useobject{currentmarker}{}%
\end{pgfscope}%
\begin{pgfscope}%
\pgfsys@transformshift{1.192646in}{1.632723in}%
\pgfsys@useobject{currentmarker}{}%
\end{pgfscope}%
\begin{pgfscope}%
\pgfsys@transformshift{0.575539in}{1.258959in}%
\pgfsys@useobject{currentmarker}{}%
\end{pgfscope}%
\begin{pgfscope}%
\pgfsys@transformshift{0.517708in}{1.557455in}%
\pgfsys@useobject{currentmarker}{}%
\end{pgfscope}%
\begin{pgfscope}%
\pgfsys@transformshift{1.244210in}{1.285053in}%
\pgfsys@useobject{currentmarker}{}%
\end{pgfscope}%
\begin{pgfscope}%
\pgfsys@transformshift{0.615567in}{1.216999in}%
\pgfsys@useobject{currentmarker}{}%
\end{pgfscope}%
\begin{pgfscope}%
\pgfsys@transformshift{0.907672in}{1.591259in}%
\pgfsys@useobject{currentmarker}{}%
\end{pgfscope}%
\begin{pgfscope}%
\pgfsys@transformshift{2.000000in}{1.933366in}%
\pgfsys@useobject{currentmarker}{}%
\end{pgfscope}%
\begin{pgfscope}%
\pgfsys@transformshift{0.485263in}{1.493758in}%
\pgfsys@useobject{currentmarker}{}%
\end{pgfscope}%
\begin{pgfscope}%
\pgfsys@transformshift{0.433137in}{1.326054in}%
\pgfsys@useobject{currentmarker}{}%
\end{pgfscope}%
\begin{pgfscope}%
\pgfsys@transformshift{0.499449in}{1.521510in}%
\pgfsys@useobject{currentmarker}{}%
\end{pgfscope}%
\begin{pgfscope}%
\pgfsys@transformshift{0.868886in}{1.423280in}%
\pgfsys@useobject{currentmarker}{}%
\end{pgfscope}%
\begin{pgfscope}%
\pgfsys@transformshift{1.451865in}{1.590882in}%
\pgfsys@useobject{currentmarker}{}%
\end{pgfscope}%
\begin{pgfscope}%
\pgfsys@transformshift{0.534197in}{1.583317in}%
\pgfsys@useobject{currentmarker}{}%
\end{pgfscope}%
\begin{pgfscope}%
\pgfsys@transformshift{0.972750in}{1.577204in}%
\pgfsys@useobject{currentmarker}{}%
\end{pgfscope}%
\begin{pgfscope}%
\pgfsys@transformshift{0.996367in}{1.375284in}%
\pgfsys@useobject{currentmarker}{}%
\end{pgfscope}%
\begin{pgfscope}%
\pgfsys@transformshift{0.893513in}{1.393058in}%
\pgfsys@useobject{currentmarker}{}%
\end{pgfscope}%
\begin{pgfscope}%
\pgfsys@transformshift{0.481561in}{1.484323in}%
\pgfsys@useobject{currentmarker}{}%
\end{pgfscope}%
\begin{pgfscope}%
\pgfsys@transformshift{0.462083in}{1.523493in}%
\pgfsys@useobject{currentmarker}{}%
\end{pgfscope}%
\begin{pgfscope}%
\pgfsys@transformshift{0.947888in}{1.344824in}%
\pgfsys@useobject{currentmarker}{}%
\end{pgfscope}%
\begin{pgfscope}%
\pgfsys@transformshift{0.475131in}{1.449726in}%
\pgfsys@useobject{currentmarker}{}%
\end{pgfscope}%
\begin{pgfscope}%
\pgfsys@transformshift{0.944100in}{1.396056in}%
\pgfsys@useobject{currentmarker}{}%
\end{pgfscope}%
\begin{pgfscope}%
\pgfsys@transformshift{0.966481in}{1.496439in}%
\pgfsys@useobject{currentmarker}{}%
\end{pgfscope}%
\begin{pgfscope}%
\pgfsys@transformshift{1.021199in}{1.555825in}%
\pgfsys@useobject{currentmarker}{}%
\end{pgfscope}%
\begin{pgfscope}%
\pgfsys@transformshift{0.451598in}{1.530091in}%
\pgfsys@useobject{currentmarker}{}%
\end{pgfscope}%
\begin{pgfscope}%
\pgfsys@transformshift{0.461004in}{1.586854in}%
\pgfsys@useobject{currentmarker}{}%
\end{pgfscope}%
\begin{pgfscope}%
\pgfsys@transformshift{0.419847in}{1.538350in}%
\pgfsys@useobject{currentmarker}{}%
\end{pgfscope}%
\begin{pgfscope}%
\pgfsys@transformshift{0.498798in}{1.332895in}%
\pgfsys@useobject{currentmarker}{}%
\end{pgfscope}%
\begin{pgfscope}%
\pgfsys@transformshift{1.022778in}{1.421709in}%
\pgfsys@useobject{currentmarker}{}%
\end{pgfscope}%
\begin{pgfscope}%
\pgfsys@transformshift{1.618050in}{1.212466in}%
\pgfsys@useobject{currentmarker}{}%
\end{pgfscope}%
\begin{pgfscope}%
\pgfsys@transformshift{0.873499in}{1.646747in}%
\pgfsys@useobject{currentmarker}{}%
\end{pgfscope}%
\begin{pgfscope}%
\pgfsys@transformshift{1.717975in}{1.514944in}%
\pgfsys@useobject{currentmarker}{}%
\end{pgfscope}%
\begin{pgfscope}%
\pgfsys@transformshift{0.479704in}{1.703062in}%
\pgfsys@useobject{currentmarker}{}%
\end{pgfscope}%
\begin{pgfscope}%
\pgfsys@transformshift{0.921403in}{1.694434in}%
\pgfsys@useobject{currentmarker}{}%
\end{pgfscope}%
\begin{pgfscope}%
\pgfsys@transformshift{0.507519in}{1.478561in}%
\pgfsys@useobject{currentmarker}{}%
\end{pgfscope}%
\begin{pgfscope}%
\pgfsys@transformshift{1.242289in}{1.691080in}%
\pgfsys@useobject{currentmarker}{}%
\end{pgfscope}%
\begin{pgfscope}%
\pgfsys@transformshift{0.901785in}{1.309811in}%
\pgfsys@useobject{currentmarker}{}%
\end{pgfscope}%
\begin{pgfscope}%
\pgfsys@transformshift{1.303585in}{1.587737in}%
\pgfsys@useobject{currentmarker}{}%
\end{pgfscope}%
\begin{pgfscope}%
\pgfsys@transformshift{0.385710in}{1.483870in}%
\pgfsys@useobject{currentmarker}{}%
\end{pgfscope}%
\begin{pgfscope}%
\pgfsys@transformshift{0.487652in}{1.468399in}%
\pgfsys@useobject{currentmarker}{}%
\end{pgfscope}%
\begin{pgfscope}%
\pgfsys@transformshift{0.427840in}{1.355065in}%
\pgfsys@useobject{currentmarker}{}%
\end{pgfscope}%
\begin{pgfscope}%
\pgfsys@transformshift{0.479919in}{1.548716in}%
\pgfsys@useobject{currentmarker}{}%
\end{pgfscope}%
\begin{pgfscope}%
\pgfsys@transformshift{0.421076in}{1.061104in}%
\pgfsys@useobject{currentmarker}{}%
\end{pgfscope}%
\begin{pgfscope}%
\pgfsys@transformshift{0.930792in}{1.338566in}%
\pgfsys@useobject{currentmarker}{}%
\end{pgfscope}%
\begin{pgfscope}%
\pgfsys@transformshift{0.890478in}{1.744669in}%
\pgfsys@useobject{currentmarker}{}%
\end{pgfscope}%
\begin{pgfscope}%
\pgfsys@transformshift{0.528345in}{1.382908in}%
\pgfsys@useobject{currentmarker}{}%
\end{pgfscope}%
\begin{pgfscope}%
\pgfsys@transformshift{1.013284in}{1.465961in}%
\pgfsys@useobject{currentmarker}{}%
\end{pgfscope}%
\begin{pgfscope}%
\pgfsys@transformshift{0.586631in}{1.402159in}%
\pgfsys@useobject{currentmarker}{}%
\end{pgfscope}%
\begin{pgfscope}%
\pgfsys@transformshift{1.117579in}{1.476883in}%
\pgfsys@useobject{currentmarker}{}%
\end{pgfscope}%
\begin{pgfscope}%
\pgfsys@transformshift{0.440363in}{1.452592in}%
\pgfsys@useobject{currentmarker}{}%
\end{pgfscope}%
\begin{pgfscope}%
\pgfsys@transformshift{0.482005in}{1.633551in}%
\pgfsys@useobject{currentmarker}{}%
\end{pgfscope}%
\begin{pgfscope}%
\pgfsys@transformshift{0.439583in}{1.293352in}%
\pgfsys@useobject{currentmarker}{}%
\end{pgfscope}%
\begin{pgfscope}%
\pgfsys@transformshift{0.857963in}{1.521022in}%
\pgfsys@useobject{currentmarker}{}%
\end{pgfscope}%
\begin{pgfscope}%
\pgfsys@transformshift{1.001461in}{1.406064in}%
\pgfsys@useobject{currentmarker}{}%
\end{pgfscope}%
\begin{pgfscope}%
\pgfsys@transformshift{0.484673in}{1.537000in}%
\pgfsys@useobject{currentmarker}{}%
\end{pgfscope}%
\begin{pgfscope}%
\pgfsys@transformshift{0.458444in}{1.429553in}%
\pgfsys@useobject{currentmarker}{}%
\end{pgfscope}%
\begin{pgfscope}%
\pgfsys@transformshift{0.477639in}{1.740663in}%
\pgfsys@useobject{currentmarker}{}%
\end{pgfscope}%
\begin{pgfscope}%
\pgfsys@transformshift{0.908318in}{1.352320in}%
\pgfsys@useobject{currentmarker}{}%
\end{pgfscope}%
\begin{pgfscope}%
\pgfsys@transformshift{0.463108in}{1.399407in}%
\pgfsys@useobject{currentmarker}{}%
\end{pgfscope}%
\begin{pgfscope}%
\pgfsys@transformshift{1.555994in}{1.596659in}%
\pgfsys@useobject{currentmarker}{}%
\end{pgfscope}%
\begin{pgfscope}%
\pgfsys@transformshift{0.456229in}{1.262657in}%
\pgfsys@useobject{currentmarker}{}%
\end{pgfscope}%
\begin{pgfscope}%
\pgfsys@transformshift{0.909471in}{1.739252in}%
\pgfsys@useobject{currentmarker}{}%
\end{pgfscope}%
\end{pgfscope}%
\begin{pgfscope}%
\pgfpathrectangle{\pgfqpoint{0.341129in}{0.466613in}}{\pgfqpoint{1.658871in}{1.711598in}}%
\pgfusepath{clip}%
\pgfsetbuttcap%
\pgfsetroundjoin%
\definecolor{currentfill}{rgb}{0.333333,0.658824,0.407843}%
\pgfsetfillcolor{currentfill}%
\pgfsetfillopacity{0.150000}%
\pgfsetlinewidth{1.003750pt}%
\definecolor{currentstroke}{rgb}{1.000000,1.000000,1.000000}%
\pgfsetstrokecolor{currentstroke}%
\pgfsetstrokeopacity{0.150000}%
\pgfsetdash{}{0pt}%
\pgfsys@defobject{currentmarker}{\pgfqpoint{0.341129in}{1.418301in}}{\pgfqpoint{2.000000in}{1.600937in}}{%
\pgfpathmoveto{\pgfqpoint{0.341129in}{1.460233in}}%
\pgfpathlineto{\pgfqpoint{0.341129in}{1.418301in}}%
\pgfpathlineto{\pgfqpoint{0.357885in}{1.419888in}}%
\pgfpathlineto{\pgfqpoint{0.374641in}{1.421431in}}%
\pgfpathlineto{\pgfqpoint{0.391398in}{1.423125in}}%
\pgfpathlineto{\pgfqpoint{0.408154in}{1.424826in}}%
\pgfpathlineto{\pgfqpoint{0.424910in}{1.426593in}}%
\pgfpathlineto{\pgfqpoint{0.441666in}{1.428027in}}%
\pgfpathlineto{\pgfqpoint{0.458423in}{1.429539in}}%
\pgfpathlineto{\pgfqpoint{0.475179in}{1.431198in}}%
\pgfpathlineto{\pgfqpoint{0.491935in}{1.432695in}}%
\pgfpathlineto{\pgfqpoint{0.508691in}{1.434430in}}%
\pgfpathlineto{\pgfqpoint{0.525448in}{1.435730in}}%
\pgfpathlineto{\pgfqpoint{0.542204in}{1.437219in}}%
\pgfpathlineto{\pgfqpoint{0.558960in}{1.438505in}}%
\pgfpathlineto{\pgfqpoint{0.575717in}{1.439905in}}%
\pgfpathlineto{\pgfqpoint{0.592473in}{1.441501in}}%
\pgfpathlineto{\pgfqpoint{0.609229in}{1.443187in}}%
\pgfpathlineto{\pgfqpoint{0.625985in}{1.444856in}}%
\pgfpathlineto{\pgfqpoint{0.642742in}{1.446485in}}%
\pgfpathlineto{\pgfqpoint{0.659498in}{1.447911in}}%
\pgfpathlineto{\pgfqpoint{0.676254in}{1.449334in}}%
\pgfpathlineto{\pgfqpoint{0.693011in}{1.450761in}}%
\pgfpathlineto{\pgfqpoint{0.709767in}{1.452427in}}%
\pgfpathlineto{\pgfqpoint{0.726523in}{1.453978in}}%
\pgfpathlineto{\pgfqpoint{0.743279in}{1.455472in}}%
\pgfpathlineto{\pgfqpoint{0.760036in}{1.456794in}}%
\pgfpathlineto{\pgfqpoint{0.776792in}{1.458202in}}%
\pgfpathlineto{\pgfqpoint{0.793548in}{1.459303in}}%
\pgfpathlineto{\pgfqpoint{0.810304in}{1.460721in}}%
\pgfpathlineto{\pgfqpoint{0.827061in}{1.462001in}}%
\pgfpathlineto{\pgfqpoint{0.843817in}{1.462835in}}%
\pgfpathlineto{\pgfqpoint{0.860573in}{1.464435in}}%
\pgfpathlineto{\pgfqpoint{0.877330in}{1.465434in}}%
\pgfpathlineto{\pgfqpoint{0.894086in}{1.466731in}}%
\pgfpathlineto{\pgfqpoint{0.910842in}{1.467929in}}%
\pgfpathlineto{\pgfqpoint{0.927598in}{1.468997in}}%
\pgfpathlineto{\pgfqpoint{0.944355in}{1.470125in}}%
\pgfpathlineto{\pgfqpoint{0.961111in}{1.471257in}}%
\pgfpathlineto{\pgfqpoint{0.977867in}{1.472389in}}%
\pgfpathlineto{\pgfqpoint{0.994623in}{1.473526in}}%
\pgfpathlineto{\pgfqpoint{1.011380in}{1.474629in}}%
\pgfpathlineto{\pgfqpoint{1.028136in}{1.475786in}}%
\pgfpathlineto{\pgfqpoint{1.044892in}{1.476722in}}%
\pgfpathlineto{\pgfqpoint{1.061649in}{1.477731in}}%
\pgfpathlineto{\pgfqpoint{1.078405in}{1.478804in}}%
\pgfpathlineto{\pgfqpoint{1.095161in}{1.479971in}}%
\pgfpathlineto{\pgfqpoint{1.111917in}{1.481230in}}%
\pgfpathlineto{\pgfqpoint{1.128674in}{1.482235in}}%
\pgfpathlineto{\pgfqpoint{1.145430in}{1.483462in}}%
\pgfpathlineto{\pgfqpoint{1.162186in}{1.484588in}}%
\pgfpathlineto{\pgfqpoint{1.178942in}{1.485609in}}%
\pgfpathlineto{\pgfqpoint{1.195699in}{1.486656in}}%
\pgfpathlineto{\pgfqpoint{1.212455in}{1.487612in}}%
\pgfpathlineto{\pgfqpoint{1.229211in}{1.488482in}}%
\pgfpathlineto{\pgfqpoint{1.245968in}{1.489234in}}%
\pgfpathlineto{\pgfqpoint{1.262724in}{1.489993in}}%
\pgfpathlineto{\pgfqpoint{1.279480in}{1.490679in}}%
\pgfpathlineto{\pgfqpoint{1.296236in}{1.491351in}}%
\pgfpathlineto{\pgfqpoint{1.312993in}{1.492276in}}%
\pgfpathlineto{\pgfqpoint{1.329749in}{1.493037in}}%
\pgfpathlineto{\pgfqpoint{1.346505in}{1.493798in}}%
\pgfpathlineto{\pgfqpoint{1.363262in}{1.494562in}}%
\pgfpathlineto{\pgfqpoint{1.380018in}{1.495353in}}%
\pgfpathlineto{\pgfqpoint{1.396774in}{1.496327in}}%
\pgfpathlineto{\pgfqpoint{1.413530in}{1.497210in}}%
\pgfpathlineto{\pgfqpoint{1.430287in}{1.498092in}}%
\pgfpathlineto{\pgfqpoint{1.447043in}{1.498975in}}%
\pgfpathlineto{\pgfqpoint{1.463799in}{1.499778in}}%
\pgfpathlineto{\pgfqpoint{1.480555in}{1.500368in}}%
\pgfpathlineto{\pgfqpoint{1.497312in}{1.500970in}}%
\pgfpathlineto{\pgfqpoint{1.514068in}{1.502003in}}%
\pgfpathlineto{\pgfqpoint{1.530824in}{1.503041in}}%
\pgfpathlineto{\pgfqpoint{1.547581in}{1.504079in}}%
\pgfpathlineto{\pgfqpoint{1.564337in}{1.504737in}}%
\pgfpathlineto{\pgfqpoint{1.581093in}{1.505680in}}%
\pgfpathlineto{\pgfqpoint{1.597849in}{1.506346in}}%
\pgfpathlineto{\pgfqpoint{1.614606in}{1.507013in}}%
\pgfpathlineto{\pgfqpoint{1.631362in}{1.507683in}}%
\pgfpathlineto{\pgfqpoint{1.648118in}{1.508356in}}%
\pgfpathlineto{\pgfqpoint{1.664874in}{1.509029in}}%
\pgfpathlineto{\pgfqpoint{1.681631in}{1.509702in}}%
\pgfpathlineto{\pgfqpoint{1.698387in}{1.510389in}}%
\pgfpathlineto{\pgfqpoint{1.715143in}{1.511208in}}%
\pgfpathlineto{\pgfqpoint{1.731900in}{1.512029in}}%
\pgfpathlineto{\pgfqpoint{1.748656in}{1.512854in}}%
\pgfpathlineto{\pgfqpoint{1.765412in}{1.513621in}}%
\pgfpathlineto{\pgfqpoint{1.782168in}{1.514479in}}%
\pgfpathlineto{\pgfqpoint{1.798925in}{1.515310in}}%
\pgfpathlineto{\pgfqpoint{1.815681in}{1.516134in}}%
\pgfpathlineto{\pgfqpoint{1.832437in}{1.516958in}}%
\pgfpathlineto{\pgfqpoint{1.849193in}{1.517668in}}%
\pgfpathlineto{\pgfqpoint{1.865950in}{1.518346in}}%
\pgfpathlineto{\pgfqpoint{1.882706in}{1.519088in}}%
\pgfpathlineto{\pgfqpoint{1.899462in}{1.519964in}}%
\pgfpathlineto{\pgfqpoint{1.916219in}{1.520840in}}%
\pgfpathlineto{\pgfqpoint{1.932975in}{1.521717in}}%
\pgfpathlineto{\pgfqpoint{1.949731in}{1.522593in}}%
\pgfpathlineto{\pgfqpoint{1.966487in}{1.523470in}}%
\pgfpathlineto{\pgfqpoint{1.983244in}{1.524285in}}%
\pgfpathlineto{\pgfqpoint{2.000000in}{1.524972in}}%
\pgfpathlineto{\pgfqpoint{2.000000in}{1.600937in}}%
\pgfpathlineto{\pgfqpoint{2.000000in}{1.600937in}}%
\pgfpathlineto{\pgfqpoint{1.983244in}{1.599192in}}%
\pgfpathlineto{\pgfqpoint{1.966487in}{1.597511in}}%
\pgfpathlineto{\pgfqpoint{1.949731in}{1.595830in}}%
\pgfpathlineto{\pgfqpoint{1.932975in}{1.594149in}}%
\pgfpathlineto{\pgfqpoint{1.916219in}{1.592415in}}%
\pgfpathlineto{\pgfqpoint{1.899462in}{1.590661in}}%
\pgfpathlineto{\pgfqpoint{1.882706in}{1.588906in}}%
\pgfpathlineto{\pgfqpoint{1.865950in}{1.587152in}}%
\pgfpathlineto{\pgfqpoint{1.849193in}{1.585398in}}%
\pgfpathlineto{\pgfqpoint{1.832437in}{1.583643in}}%
\pgfpathlineto{\pgfqpoint{1.815681in}{1.581889in}}%
\pgfpathlineto{\pgfqpoint{1.798925in}{1.580132in}}%
\pgfpathlineto{\pgfqpoint{1.782168in}{1.578376in}}%
\pgfpathlineto{\pgfqpoint{1.765412in}{1.576620in}}%
\pgfpathlineto{\pgfqpoint{1.748656in}{1.574863in}}%
\pgfpathlineto{\pgfqpoint{1.731900in}{1.573106in}}%
\pgfpathlineto{\pgfqpoint{1.715143in}{1.571349in}}%
\pgfpathlineto{\pgfqpoint{1.698387in}{1.569591in}}%
\pgfpathlineto{\pgfqpoint{1.681631in}{1.567833in}}%
\pgfpathlineto{\pgfqpoint{1.664874in}{1.566075in}}%
\pgfpathlineto{\pgfqpoint{1.648118in}{1.564317in}}%
\pgfpathlineto{\pgfqpoint{1.631362in}{1.562559in}}%
\pgfpathlineto{\pgfqpoint{1.614606in}{1.560801in}}%
\pgfpathlineto{\pgfqpoint{1.597849in}{1.559061in}}%
\pgfpathlineto{\pgfqpoint{1.581093in}{1.557372in}}%
\pgfpathlineto{\pgfqpoint{1.564337in}{1.555684in}}%
\pgfpathlineto{\pgfqpoint{1.547581in}{1.553994in}}%
\pgfpathlineto{\pgfqpoint{1.530824in}{1.552491in}}%
\pgfpathlineto{\pgfqpoint{1.514068in}{1.550765in}}%
\pgfpathlineto{\pgfqpoint{1.497312in}{1.549012in}}%
\pgfpathlineto{\pgfqpoint{1.480555in}{1.547258in}}%
\pgfpathlineto{\pgfqpoint{1.463799in}{1.545691in}}%
\pgfpathlineto{\pgfqpoint{1.447043in}{1.544124in}}%
\pgfpathlineto{\pgfqpoint{1.430287in}{1.542557in}}%
\pgfpathlineto{\pgfqpoint{1.413530in}{1.541147in}}%
\pgfpathlineto{\pgfqpoint{1.396774in}{1.539394in}}%
\pgfpathlineto{\pgfqpoint{1.380018in}{1.537679in}}%
\pgfpathlineto{\pgfqpoint{1.363262in}{1.536182in}}%
\pgfpathlineto{\pgfqpoint{1.346505in}{1.534686in}}%
\pgfpathlineto{\pgfqpoint{1.329749in}{1.533153in}}%
\pgfpathlineto{\pgfqpoint{1.312993in}{1.531579in}}%
\pgfpathlineto{\pgfqpoint{1.296236in}{1.529957in}}%
\pgfpathlineto{\pgfqpoint{1.279480in}{1.528335in}}%
\pgfpathlineto{\pgfqpoint{1.262724in}{1.526516in}}%
\pgfpathlineto{\pgfqpoint{1.245968in}{1.524641in}}%
\pgfpathlineto{\pgfqpoint{1.229211in}{1.523244in}}%
\pgfpathlineto{\pgfqpoint{1.212455in}{1.521564in}}%
\pgfpathlineto{\pgfqpoint{1.195699in}{1.519817in}}%
\pgfpathlineto{\pgfqpoint{1.178942in}{1.518306in}}%
\pgfpathlineto{\pgfqpoint{1.162186in}{1.516593in}}%
\pgfpathlineto{\pgfqpoint{1.145430in}{1.514975in}}%
\pgfpathlineto{\pgfqpoint{1.128674in}{1.513255in}}%
\pgfpathlineto{\pgfqpoint{1.111917in}{1.511702in}}%
\pgfpathlineto{\pgfqpoint{1.095161in}{1.510157in}}%
\pgfpathlineto{\pgfqpoint{1.078405in}{1.508608in}}%
\pgfpathlineto{\pgfqpoint{1.061649in}{1.507157in}}%
\pgfpathlineto{\pgfqpoint{1.044892in}{1.505609in}}%
\pgfpathlineto{\pgfqpoint{1.028136in}{1.504040in}}%
\pgfpathlineto{\pgfqpoint{1.011380in}{1.502417in}}%
\pgfpathlineto{\pgfqpoint{0.994623in}{1.500878in}}%
\pgfpathlineto{\pgfqpoint{0.977867in}{1.499523in}}%
\pgfpathlineto{\pgfqpoint{0.961111in}{1.498222in}}%
\pgfpathlineto{\pgfqpoint{0.944355in}{1.496795in}}%
\pgfpathlineto{\pgfqpoint{0.927598in}{1.495554in}}%
\pgfpathlineto{\pgfqpoint{0.910842in}{1.494103in}}%
\pgfpathlineto{\pgfqpoint{0.894086in}{1.492654in}}%
\pgfpathlineto{\pgfqpoint{0.877330in}{1.491432in}}%
\pgfpathlineto{\pgfqpoint{0.860573in}{1.490179in}}%
\pgfpathlineto{\pgfqpoint{0.843817in}{1.488910in}}%
\pgfpathlineto{\pgfqpoint{0.827061in}{1.487765in}}%
\pgfpathlineto{\pgfqpoint{0.810304in}{1.486536in}}%
\pgfpathlineto{\pgfqpoint{0.793548in}{1.485363in}}%
\pgfpathlineto{\pgfqpoint{0.776792in}{1.484253in}}%
\pgfpathlineto{\pgfqpoint{0.760036in}{1.483216in}}%
\pgfpathlineto{\pgfqpoint{0.743279in}{1.482234in}}%
\pgfpathlineto{\pgfqpoint{0.726523in}{1.481473in}}%
\pgfpathlineto{\pgfqpoint{0.709767in}{1.480349in}}%
\pgfpathlineto{\pgfqpoint{0.693011in}{1.479060in}}%
\pgfpathlineto{\pgfqpoint{0.676254in}{1.477805in}}%
\pgfpathlineto{\pgfqpoint{0.659498in}{1.476488in}}%
\pgfpathlineto{\pgfqpoint{0.642742in}{1.475451in}}%
\pgfpathlineto{\pgfqpoint{0.625985in}{1.474590in}}%
\pgfpathlineto{\pgfqpoint{0.609229in}{1.473516in}}%
\pgfpathlineto{\pgfqpoint{0.592473in}{1.472550in}}%
\pgfpathlineto{\pgfqpoint{0.575717in}{1.471371in}}%
\pgfpathlineto{\pgfqpoint{0.558960in}{1.470287in}}%
\pgfpathlineto{\pgfqpoint{0.542204in}{1.469446in}}%
\pgfpathlineto{\pgfqpoint{0.525448in}{1.468699in}}%
\pgfpathlineto{\pgfqpoint{0.508691in}{1.467962in}}%
\pgfpathlineto{\pgfqpoint{0.491935in}{1.467205in}}%
\pgfpathlineto{\pgfqpoint{0.475179in}{1.466467in}}%
\pgfpathlineto{\pgfqpoint{0.458423in}{1.465701in}}%
\pgfpathlineto{\pgfqpoint{0.441666in}{1.464953in}}%
\pgfpathlineto{\pgfqpoint{0.424910in}{1.463850in}}%
\pgfpathlineto{\pgfqpoint{0.408154in}{1.463020in}}%
\pgfpathlineto{\pgfqpoint{0.391398in}{1.462432in}}%
\pgfpathlineto{\pgfqpoint{0.374641in}{1.461840in}}%
\pgfpathlineto{\pgfqpoint{0.357885in}{1.461170in}}%
\pgfpathlineto{\pgfqpoint{0.341129in}{1.460233in}}%
\pgfpathclose%
\pgfusepath{stroke,fill}%
}%
\begin{pgfscope}%
\pgfsys@transformshift{0.000000in}{0.000000in}%
\pgfsys@useobject{currentmarker}{}%
\end{pgfscope}%
\end{pgfscope}%
\begin{pgfscope}%
\pgfpathrectangle{\pgfqpoint{0.341129in}{0.466613in}}{\pgfqpoint{1.658871in}{1.711598in}}%
\pgfusepath{clip}%
\pgfsetbuttcap%
\pgfsetroundjoin%
\definecolor{currentfill}{rgb}{0.768627,0.305882,0.321569}%
\pgfsetfillcolor{currentfill}%
\pgfsetfillopacity{0.250000}%
\pgfsetlinewidth{1.003750pt}%
\definecolor{currentstroke}{rgb}{0.768627,0.305882,0.321569}%
\pgfsetstrokecolor{currentstroke}%
\pgfsetstrokeopacity{0.250000}%
\pgfsetdash{}{0pt}%
\pgfsys@defobject{currentmarker}{\pgfqpoint{-0.017010in}{-0.017010in}}{\pgfqpoint{0.017010in}{0.017010in}}{%
\pgfpathmoveto{\pgfqpoint{0.000000in}{-0.017010in}}%
\pgfpathcurveto{\pgfqpoint{0.004511in}{-0.017010in}}{\pgfqpoint{0.008838in}{-0.015218in}}{\pgfqpoint{0.012028in}{-0.012028in}}%
\pgfpathcurveto{\pgfqpoint{0.015218in}{-0.008838in}}{\pgfqpoint{0.017010in}{-0.004511in}}{\pgfqpoint{0.017010in}{0.000000in}}%
\pgfpathcurveto{\pgfqpoint{0.017010in}{0.004511in}}{\pgfqpoint{0.015218in}{0.008838in}}{\pgfqpoint{0.012028in}{0.012028in}}%
\pgfpathcurveto{\pgfqpoint{0.008838in}{0.015218in}}{\pgfqpoint{0.004511in}{0.017010in}}{\pgfqpoint{0.000000in}{0.017010in}}%
\pgfpathcurveto{\pgfqpoint{-0.004511in}{0.017010in}}{\pgfqpoint{-0.008838in}{0.015218in}}{\pgfqpoint{-0.012028in}{0.012028in}}%
\pgfpathcurveto{\pgfqpoint{-0.015218in}{0.008838in}}{\pgfqpoint{-0.017010in}{0.004511in}}{\pgfqpoint{-0.017010in}{0.000000in}}%
\pgfpathcurveto{\pgfqpoint{-0.017010in}{-0.004511in}}{\pgfqpoint{-0.015218in}{-0.008838in}}{\pgfqpoint{-0.012028in}{-0.012028in}}%
\pgfpathcurveto{\pgfqpoint{-0.008838in}{-0.015218in}}{\pgfqpoint{-0.004511in}{-0.017010in}}{\pgfqpoint{0.000000in}{-0.017010in}}%
\pgfpathclose%
\pgfusepath{stroke,fill}%
}%
\begin{pgfscope}%
\pgfsys@transformshift{1.684044in}{1.487621in}%
\pgfsys@useobject{currentmarker}{}%
\end{pgfscope}%
\begin{pgfscope}%
\pgfsys@transformshift{1.113006in}{1.427337in}%
\pgfsys@useobject{currentmarker}{}%
\end{pgfscope}%
\begin{pgfscope}%
\pgfsys@transformshift{0.938483in}{1.265604in}%
\pgfsys@useobject{currentmarker}{}%
\end{pgfscope}%
\begin{pgfscope}%
\pgfsys@transformshift{0.492780in}{1.152049in}%
\pgfsys@useobject{currentmarker}{}%
\end{pgfscope}%
\begin{pgfscope}%
\pgfsys@transformshift{0.529797in}{1.118678in}%
\pgfsys@useobject{currentmarker}{}%
\end{pgfscope}%
\begin{pgfscope}%
\pgfsys@transformshift{1.838155in}{1.458827in}%
\pgfsys@useobject{currentmarker}{}%
\end{pgfscope}%
\begin{pgfscope}%
\pgfsys@transformshift{0.477840in}{1.090915in}%
\pgfsys@useobject{currentmarker}{}%
\end{pgfscope}%
\begin{pgfscope}%
\pgfsys@transformshift{0.884452in}{1.521982in}%
\pgfsys@useobject{currentmarker}{}%
\end{pgfscope}%
\begin{pgfscope}%
\pgfsys@transformshift{1.509149in}{1.476897in}%
\pgfsys@useobject{currentmarker}{}%
\end{pgfscope}%
\begin{pgfscope}%
\pgfsys@transformshift{1.183184in}{1.326899in}%
\pgfsys@useobject{currentmarker}{}%
\end{pgfscope}%
\begin{pgfscope}%
\pgfsys@transformshift{1.140467in}{1.215456in}%
\pgfsys@useobject{currentmarker}{}%
\end{pgfscope}%
\begin{pgfscope}%
\pgfsys@transformshift{0.501776in}{1.251349in}%
\pgfsys@useobject{currentmarker}{}%
\end{pgfscope}%
\begin{pgfscope}%
\pgfsys@transformshift{0.461389in}{1.072476in}%
\pgfsys@useobject{currentmarker}{}%
\end{pgfscope}%
\begin{pgfscope}%
\pgfsys@transformshift{0.986905in}{1.332324in}%
\pgfsys@useobject{currentmarker}{}%
\end{pgfscope}%
\begin{pgfscope}%
\pgfsys@transformshift{0.969502in}{1.284761in}%
\pgfsys@useobject{currentmarker}{}%
\end{pgfscope}%
\begin{pgfscope}%
\pgfsys@transformshift{1.016218in}{1.228126in}%
\pgfsys@useobject{currentmarker}{}%
\end{pgfscope}%
\begin{pgfscope}%
\pgfsys@transformshift{0.413211in}{1.308942in}%
\pgfsys@useobject{currentmarker}{}%
\end{pgfscope}%
\begin{pgfscope}%
\pgfsys@transformshift{1.870040in}{1.456402in}%
\pgfsys@useobject{currentmarker}{}%
\end{pgfscope}%
\begin{pgfscope}%
\pgfsys@transformshift{1.031673in}{1.346079in}%
\pgfsys@useobject{currentmarker}{}%
\end{pgfscope}%
\begin{pgfscope}%
\pgfsys@transformshift{1.596314in}{1.433052in}%
\pgfsys@useobject{currentmarker}{}%
\end{pgfscope}%
\begin{pgfscope}%
\pgfsys@transformshift{0.496691in}{1.214484in}%
\pgfsys@useobject{currentmarker}{}%
\end{pgfscope}%
\begin{pgfscope}%
\pgfsys@transformshift{0.896936in}{1.206455in}%
\pgfsys@useobject{currentmarker}{}%
\end{pgfscope}%
\begin{pgfscope}%
\pgfsys@transformshift{0.442231in}{1.200958in}%
\pgfsys@useobject{currentmarker}{}%
\end{pgfscope}%
\begin{pgfscope}%
\pgfsys@transformshift{1.948436in}{1.524847in}%
\pgfsys@useobject{currentmarker}{}%
\end{pgfscope}%
\begin{pgfscope}%
\pgfsys@transformshift{0.607246in}{1.234917in}%
\pgfsys@useobject{currentmarker}{}%
\end{pgfscope}%
\begin{pgfscope}%
\pgfsys@transformshift{0.591373in}{0.924368in}%
\pgfsys@useobject{currentmarker}{}%
\end{pgfscope}%
\begin{pgfscope}%
\pgfsys@transformshift{0.495569in}{1.148236in}%
\pgfsys@useobject{currentmarker}{}%
\end{pgfscope}%
\begin{pgfscope}%
\pgfsys@transformshift{0.805700in}{1.083610in}%
\pgfsys@useobject{currentmarker}{}%
\end{pgfscope}%
\begin{pgfscope}%
\pgfsys@transformshift{0.472915in}{1.263815in}%
\pgfsys@useobject{currentmarker}{}%
\end{pgfscope}%
\begin{pgfscope}%
\pgfsys@transformshift{1.008191in}{1.342683in}%
\pgfsys@useobject{currentmarker}{}%
\end{pgfscope}%
\begin{pgfscope}%
\pgfsys@transformshift{0.795900in}{1.422431in}%
\pgfsys@useobject{currentmarker}{}%
\end{pgfscope}%
\begin{pgfscope}%
\pgfsys@transformshift{0.879596in}{1.468650in}%
\pgfsys@useobject{currentmarker}{}%
\end{pgfscope}%
\begin{pgfscope}%
\pgfsys@transformshift{0.851399in}{1.069346in}%
\pgfsys@useobject{currentmarker}{}%
\end{pgfscope}%
\begin{pgfscope}%
\pgfsys@transformshift{1.630362in}{1.481342in}%
\pgfsys@useobject{currentmarker}{}%
\end{pgfscope}%
\begin{pgfscope}%
\pgfsys@transformshift{0.497926in}{1.212673in}%
\pgfsys@useobject{currentmarker}{}%
\end{pgfscope}%
\begin{pgfscope}%
\pgfsys@transformshift{1.716622in}{1.127193in}%
\pgfsys@useobject{currentmarker}{}%
\end{pgfscope}%
\begin{pgfscope}%
\pgfsys@transformshift{0.524809in}{1.405953in}%
\pgfsys@useobject{currentmarker}{}%
\end{pgfscope}%
\begin{pgfscope}%
\pgfsys@transformshift{1.821229in}{1.481645in}%
\pgfsys@useobject{currentmarker}{}%
\end{pgfscope}%
\begin{pgfscope}%
\pgfsys@transformshift{0.925546in}{1.421390in}%
\pgfsys@useobject{currentmarker}{}%
\end{pgfscope}%
\begin{pgfscope}%
\pgfsys@transformshift{1.841684in}{1.126044in}%
\pgfsys@useobject{currentmarker}{}%
\end{pgfscope}%
\begin{pgfscope}%
\pgfsys@transformshift{1.522493in}{1.421956in}%
\pgfsys@useobject{currentmarker}{}%
\end{pgfscope}%
\begin{pgfscope}%
\pgfsys@transformshift{0.409285in}{1.323720in}%
\pgfsys@useobject{currentmarker}{}%
\end{pgfscope}%
\begin{pgfscope}%
\pgfsys@transformshift{0.896047in}{1.109957in}%
\pgfsys@useobject{currentmarker}{}%
\end{pgfscope}%
\begin{pgfscope}%
\pgfsys@transformshift{0.382462in}{1.292715in}%
\pgfsys@useobject{currentmarker}{}%
\end{pgfscope}%
\begin{pgfscope}%
\pgfsys@transformshift{0.855764in}{1.081361in}%
\pgfsys@useobject{currentmarker}{}%
\end{pgfscope}%
\begin{pgfscope}%
\pgfsys@transformshift{0.523733in}{1.060600in}%
\pgfsys@useobject{currentmarker}{}%
\end{pgfscope}%
\begin{pgfscope}%
\pgfsys@transformshift{0.470197in}{1.488579in}%
\pgfsys@useobject{currentmarker}{}%
\end{pgfscope}%
\begin{pgfscope}%
\pgfsys@transformshift{0.829991in}{1.378134in}%
\pgfsys@useobject{currentmarker}{}%
\end{pgfscope}%
\begin{pgfscope}%
\pgfsys@transformshift{0.933266in}{1.104103in}%
\pgfsys@useobject{currentmarker}{}%
\end{pgfscope}%
\begin{pgfscope}%
\pgfsys@transformshift{0.956789in}{1.576300in}%
\pgfsys@useobject{currentmarker}{}%
\end{pgfscope}%
\begin{pgfscope}%
\pgfsys@transformshift{0.936472in}{1.329957in}%
\pgfsys@useobject{currentmarker}{}%
\end{pgfscope}%
\begin{pgfscope}%
\pgfsys@transformshift{0.468794in}{1.441060in}%
\pgfsys@useobject{currentmarker}{}%
\end{pgfscope}%
\begin{pgfscope}%
\pgfsys@transformshift{0.402384in}{1.389649in}%
\pgfsys@useobject{currentmarker}{}%
\end{pgfscope}%
\begin{pgfscope}%
\pgfsys@transformshift{1.177456in}{1.340930in}%
\pgfsys@useobject{currentmarker}{}%
\end{pgfscope}%
\begin{pgfscope}%
\pgfsys@transformshift{0.416018in}{1.135908in}%
\pgfsys@useobject{currentmarker}{}%
\end{pgfscope}%
\begin{pgfscope}%
\pgfsys@transformshift{0.447008in}{1.425750in}%
\pgfsys@useobject{currentmarker}{}%
\end{pgfscope}%
\begin{pgfscope}%
\pgfsys@transformshift{0.549911in}{1.094038in}%
\pgfsys@useobject{currentmarker}{}%
\end{pgfscope}%
\begin{pgfscope}%
\pgfsys@transformshift{1.065128in}{1.405909in}%
\pgfsys@useobject{currentmarker}{}%
\end{pgfscope}%
\begin{pgfscope}%
\pgfsys@transformshift{0.450053in}{1.155991in}%
\pgfsys@useobject{currentmarker}{}%
\end{pgfscope}%
\begin{pgfscope}%
\pgfsys@transformshift{0.455142in}{1.167599in}%
\pgfsys@useobject{currentmarker}{}%
\end{pgfscope}%
\begin{pgfscope}%
\pgfsys@transformshift{0.484478in}{1.188448in}%
\pgfsys@useobject{currentmarker}{}%
\end{pgfscope}%
\begin{pgfscope}%
\pgfsys@transformshift{1.875283in}{1.661860in}%
\pgfsys@useobject{currentmarker}{}%
\end{pgfscope}%
\begin{pgfscope}%
\pgfsys@transformshift{0.375023in}{1.300280in}%
\pgfsys@useobject{currentmarker}{}%
\end{pgfscope}%
\begin{pgfscope}%
\pgfsys@transformshift{0.853143in}{1.045589in}%
\pgfsys@useobject{currentmarker}{}%
\end{pgfscope}%
\begin{pgfscope}%
\pgfsys@transformshift{0.436365in}{1.103323in}%
\pgfsys@useobject{currentmarker}{}%
\end{pgfscope}%
\begin{pgfscope}%
\pgfsys@transformshift{0.491797in}{1.054395in}%
\pgfsys@useobject{currentmarker}{}%
\end{pgfscope}%
\begin{pgfscope}%
\pgfsys@transformshift{0.902491in}{1.118134in}%
\pgfsys@useobject{currentmarker}{}%
\end{pgfscope}%
\begin{pgfscope}%
\pgfsys@transformshift{0.978414in}{1.324805in}%
\pgfsys@useobject{currentmarker}{}%
\end{pgfscope}%
\begin{pgfscope}%
\pgfsys@transformshift{0.944248in}{1.377459in}%
\pgfsys@useobject{currentmarker}{}%
\end{pgfscope}%
\begin{pgfscope}%
\pgfsys@transformshift{1.817503in}{1.046386in}%
\pgfsys@useobject{currentmarker}{}%
\end{pgfscope}%
\begin{pgfscope}%
\pgfsys@transformshift{1.721814in}{1.427247in}%
\pgfsys@useobject{currentmarker}{}%
\end{pgfscope}%
\begin{pgfscope}%
\pgfsys@transformshift{0.476494in}{1.534900in}%
\pgfsys@useobject{currentmarker}{}%
\end{pgfscope}%
\begin{pgfscope}%
\pgfsys@transformshift{0.942312in}{1.291442in}%
\pgfsys@useobject{currentmarker}{}%
\end{pgfscope}%
\begin{pgfscope}%
\pgfsys@transformshift{0.975555in}{1.315794in}%
\pgfsys@useobject{currentmarker}{}%
\end{pgfscope}%
\begin{pgfscope}%
\pgfsys@transformshift{1.642091in}{1.455330in}%
\pgfsys@useobject{currentmarker}{}%
\end{pgfscope}%
\begin{pgfscope}%
\pgfsys@transformshift{0.910804in}{1.427226in}%
\pgfsys@useobject{currentmarker}{}%
\end{pgfscope}%
\begin{pgfscope}%
\pgfsys@transformshift{0.507435in}{1.083965in}%
\pgfsys@useobject{currentmarker}{}%
\end{pgfscope}%
\begin{pgfscope}%
\pgfsys@transformshift{1.946413in}{1.115208in}%
\pgfsys@useobject{currentmarker}{}%
\end{pgfscope}%
\begin{pgfscope}%
\pgfsys@transformshift{1.639910in}{1.418632in}%
\pgfsys@useobject{currentmarker}{}%
\end{pgfscope}%
\begin{pgfscope}%
\pgfsys@transformshift{1.020077in}{1.333195in}%
\pgfsys@useobject{currentmarker}{}%
\end{pgfscope}%
\begin{pgfscope}%
\pgfsys@transformshift{1.837777in}{1.512835in}%
\pgfsys@useobject{currentmarker}{}%
\end{pgfscope}%
\begin{pgfscope}%
\pgfsys@transformshift{0.546434in}{1.048900in}%
\pgfsys@useobject{currentmarker}{}%
\end{pgfscope}%
\begin{pgfscope}%
\pgfsys@transformshift{0.507573in}{1.234563in}%
\pgfsys@useobject{currentmarker}{}%
\end{pgfscope}%
\begin{pgfscope}%
\pgfsys@transformshift{0.995047in}{1.236837in}%
\pgfsys@useobject{currentmarker}{}%
\end{pgfscope}%
\begin{pgfscope}%
\pgfsys@transformshift{0.507253in}{1.202368in}%
\pgfsys@useobject{currentmarker}{}%
\end{pgfscope}%
\begin{pgfscope}%
\pgfsys@transformshift{0.844423in}{1.194625in}%
\pgfsys@useobject{currentmarker}{}%
\end{pgfscope}%
\begin{pgfscope}%
\pgfsys@transformshift{1.175457in}{1.635571in}%
\pgfsys@useobject{currentmarker}{}%
\end{pgfscope}%
\begin{pgfscope}%
\pgfsys@transformshift{1.024973in}{1.136287in}%
\pgfsys@useobject{currentmarker}{}%
\end{pgfscope}%
\begin{pgfscope}%
\pgfsys@transformshift{1.150522in}{1.249559in}%
\pgfsys@useobject{currentmarker}{}%
\end{pgfscope}%
\begin{pgfscope}%
\pgfsys@transformshift{0.562256in}{1.264014in}%
\pgfsys@useobject{currentmarker}{}%
\end{pgfscope}%
\begin{pgfscope}%
\pgfsys@transformshift{1.482648in}{1.318721in}%
\pgfsys@useobject{currentmarker}{}%
\end{pgfscope}%
\begin{pgfscope}%
\pgfsys@transformshift{1.406514in}{1.365843in}%
\pgfsys@useobject{currentmarker}{}%
\end{pgfscope}%
\begin{pgfscope}%
\pgfsys@transformshift{1.080077in}{1.200560in}%
\pgfsys@useobject{currentmarker}{}%
\end{pgfscope}%
\begin{pgfscope}%
\pgfsys@transformshift{0.989979in}{1.587387in}%
\pgfsys@useobject{currentmarker}{}%
\end{pgfscope}%
\begin{pgfscope}%
\pgfsys@transformshift{0.987626in}{1.418323in}%
\pgfsys@useobject{currentmarker}{}%
\end{pgfscope}%
\begin{pgfscope}%
\pgfsys@transformshift{1.771689in}{1.635358in}%
\pgfsys@useobject{currentmarker}{}%
\end{pgfscope}%
\begin{pgfscope}%
\pgfsys@transformshift{0.891632in}{1.131720in}%
\pgfsys@useobject{currentmarker}{}%
\end{pgfscope}%
\begin{pgfscope}%
\pgfsys@transformshift{0.870836in}{1.470948in}%
\pgfsys@useobject{currentmarker}{}%
\end{pgfscope}%
\begin{pgfscope}%
\pgfsys@transformshift{1.667984in}{1.556885in}%
\pgfsys@useobject{currentmarker}{}%
\end{pgfscope}%
\begin{pgfscope}%
\pgfsys@transformshift{1.813990in}{1.754135in}%
\pgfsys@useobject{currentmarker}{}%
\end{pgfscope}%
\begin{pgfscope}%
\pgfsys@transformshift{0.403236in}{1.316163in}%
\pgfsys@useobject{currentmarker}{}%
\end{pgfscope}%
\begin{pgfscope}%
\pgfsys@transformshift{0.996537in}{1.333520in}%
\pgfsys@useobject{currentmarker}{}%
\end{pgfscope}%
\begin{pgfscope}%
\pgfsys@transformshift{1.043417in}{1.357168in}%
\pgfsys@useobject{currentmarker}{}%
\end{pgfscope}%
\begin{pgfscope}%
\pgfsys@transformshift{0.467968in}{1.167057in}%
\pgfsys@useobject{currentmarker}{}%
\end{pgfscope}%
\begin{pgfscope}%
\pgfsys@transformshift{1.915346in}{1.061252in}%
\pgfsys@useobject{currentmarker}{}%
\end{pgfscope}%
\begin{pgfscope}%
\pgfsys@transformshift{0.437600in}{1.295987in}%
\pgfsys@useobject{currentmarker}{}%
\end{pgfscope}%
\begin{pgfscope}%
\pgfsys@transformshift{0.438013in}{1.140820in}%
\pgfsys@useobject{currentmarker}{}%
\end{pgfscope}%
\begin{pgfscope}%
\pgfsys@transformshift{0.958228in}{1.113773in}%
\pgfsys@useobject{currentmarker}{}%
\end{pgfscope}%
\begin{pgfscope}%
\pgfsys@transformshift{1.012755in}{1.064531in}%
\pgfsys@useobject{currentmarker}{}%
\end{pgfscope}%
\begin{pgfscope}%
\pgfsys@transformshift{0.406225in}{1.140520in}%
\pgfsys@useobject{currentmarker}{}%
\end{pgfscope}%
\begin{pgfscope}%
\pgfsys@transformshift{0.473582in}{1.268904in}%
\pgfsys@useobject{currentmarker}{}%
\end{pgfscope}%
\begin{pgfscope}%
\pgfsys@transformshift{0.590756in}{1.041973in}%
\pgfsys@useobject{currentmarker}{}%
\end{pgfscope}%
\begin{pgfscope}%
\pgfsys@transformshift{0.918416in}{0.961047in}%
\pgfsys@useobject{currentmarker}{}%
\end{pgfscope}%
\begin{pgfscope}%
\pgfsys@transformshift{0.435231in}{1.174268in}%
\pgfsys@useobject{currentmarker}{}%
\end{pgfscope}%
\begin{pgfscope}%
\pgfsys@transformshift{1.100847in}{1.388876in}%
\pgfsys@useobject{currentmarker}{}%
\end{pgfscope}%
\begin{pgfscope}%
\pgfsys@transformshift{1.783869in}{1.581507in}%
\pgfsys@useobject{currentmarker}{}%
\end{pgfscope}%
\begin{pgfscope}%
\pgfsys@transformshift{0.881476in}{1.530408in}%
\pgfsys@useobject{currentmarker}{}%
\end{pgfscope}%
\begin{pgfscope}%
\pgfsys@transformshift{1.004126in}{1.076592in}%
\pgfsys@useobject{currentmarker}{}%
\end{pgfscope}%
\begin{pgfscope}%
\pgfsys@transformshift{1.494969in}{1.461289in}%
\pgfsys@useobject{currentmarker}{}%
\end{pgfscope}%
\begin{pgfscope}%
\pgfsys@transformshift{0.486970in}{1.138443in}%
\pgfsys@useobject{currentmarker}{}%
\end{pgfscope}%
\begin{pgfscope}%
\pgfsys@transformshift{0.439439in}{1.378099in}%
\pgfsys@useobject{currentmarker}{}%
\end{pgfscope}%
\begin{pgfscope}%
\pgfsys@transformshift{0.879557in}{1.442857in}%
\pgfsys@useobject{currentmarker}{}%
\end{pgfscope}%
\begin{pgfscope}%
\pgfsys@transformshift{0.852216in}{1.209871in}%
\pgfsys@useobject{currentmarker}{}%
\end{pgfscope}%
\begin{pgfscope}%
\pgfsys@transformshift{0.950782in}{1.337862in}%
\pgfsys@useobject{currentmarker}{}%
\end{pgfscope}%
\begin{pgfscope}%
\pgfsys@transformshift{0.502562in}{1.357555in}%
\pgfsys@useobject{currentmarker}{}%
\end{pgfscope}%
\begin{pgfscope}%
\pgfsys@transformshift{1.067095in}{1.348575in}%
\pgfsys@useobject{currentmarker}{}%
\end{pgfscope}%
\begin{pgfscope}%
\pgfsys@transformshift{0.478759in}{1.395887in}%
\pgfsys@useobject{currentmarker}{}%
\end{pgfscope}%
\begin{pgfscope}%
\pgfsys@transformshift{0.967133in}{1.353053in}%
\pgfsys@useobject{currentmarker}{}%
\end{pgfscope}%
\begin{pgfscope}%
\pgfsys@transformshift{0.508090in}{1.177885in}%
\pgfsys@useobject{currentmarker}{}%
\end{pgfscope}%
\begin{pgfscope}%
\pgfsys@transformshift{0.435594in}{1.222493in}%
\pgfsys@useobject{currentmarker}{}%
\end{pgfscope}%
\begin{pgfscope}%
\pgfsys@transformshift{0.980355in}{1.107836in}%
\pgfsys@useobject{currentmarker}{}%
\end{pgfscope}%
\begin{pgfscope}%
\pgfsys@transformshift{1.552149in}{1.353320in}%
\pgfsys@useobject{currentmarker}{}%
\end{pgfscope}%
\begin{pgfscope}%
\pgfsys@transformshift{0.875226in}{1.065316in}%
\pgfsys@useobject{currentmarker}{}%
\end{pgfscope}%
\begin{pgfscope}%
\pgfsys@transformshift{1.107273in}{1.340079in}%
\pgfsys@useobject{currentmarker}{}%
\end{pgfscope}%
\begin{pgfscope}%
\pgfsys@transformshift{0.523782in}{1.392964in}%
\pgfsys@useobject{currentmarker}{}%
\end{pgfscope}%
\begin{pgfscope}%
\pgfsys@transformshift{1.045306in}{1.495608in}%
\pgfsys@useobject{currentmarker}{}%
\end{pgfscope}%
\begin{pgfscope}%
\pgfsys@transformshift{1.526749in}{1.271264in}%
\pgfsys@useobject{currentmarker}{}%
\end{pgfscope}%
\begin{pgfscope}%
\pgfsys@transformshift{1.541666in}{1.466251in}%
\pgfsys@useobject{currentmarker}{}%
\end{pgfscope}%
\begin{pgfscope}%
\pgfsys@transformshift{1.574267in}{1.476564in}%
\pgfsys@useobject{currentmarker}{}%
\end{pgfscope}%
\begin{pgfscope}%
\pgfsys@transformshift{1.639614in}{1.453264in}%
\pgfsys@useobject{currentmarker}{}%
\end{pgfscope}%
\begin{pgfscope}%
\pgfsys@transformshift{0.881908in}{1.499793in}%
\pgfsys@useobject{currentmarker}{}%
\end{pgfscope}%
\begin{pgfscope}%
\pgfsys@transformshift{1.677526in}{1.601066in}%
\pgfsys@useobject{currentmarker}{}%
\end{pgfscope}%
\begin{pgfscope}%
\pgfsys@transformshift{0.483826in}{1.231471in}%
\pgfsys@useobject{currentmarker}{}%
\end{pgfscope}%
\begin{pgfscope}%
\pgfsys@transformshift{0.448297in}{1.308634in}%
\pgfsys@useobject{currentmarker}{}%
\end{pgfscope}%
\begin{pgfscope}%
\pgfsys@transformshift{0.976510in}{1.211948in}%
\pgfsys@useobject{currentmarker}{}%
\end{pgfscope}%
\begin{pgfscope}%
\pgfsys@transformshift{0.481727in}{1.158270in}%
\pgfsys@useobject{currentmarker}{}%
\end{pgfscope}%
\begin{pgfscope}%
\pgfsys@transformshift{1.808114in}{1.418390in}%
\pgfsys@useobject{currentmarker}{}%
\end{pgfscope}%
\begin{pgfscope}%
\pgfsys@transformshift{0.468390in}{1.224029in}%
\pgfsys@useobject{currentmarker}{}%
\end{pgfscope}%
\begin{pgfscope}%
\pgfsys@transformshift{0.458999in}{1.247864in}%
\pgfsys@useobject{currentmarker}{}%
\end{pgfscope}%
\begin{pgfscope}%
\pgfsys@transformshift{1.685818in}{1.425059in}%
\pgfsys@useobject{currentmarker}{}%
\end{pgfscope}%
\begin{pgfscope}%
\pgfsys@transformshift{0.635418in}{1.070873in}%
\pgfsys@useobject{currentmarker}{}%
\end{pgfscope}%
\begin{pgfscope}%
\pgfsys@transformshift{0.415977in}{1.103186in}%
\pgfsys@useobject{currentmarker}{}%
\end{pgfscope}%
\begin{pgfscope}%
\pgfsys@transformshift{0.341129in}{1.222093in}%
\pgfsys@useobject{currentmarker}{}%
\end{pgfscope}%
\begin{pgfscope}%
\pgfsys@transformshift{0.992976in}{1.331682in}%
\pgfsys@useobject{currentmarker}{}%
\end{pgfscope}%
\begin{pgfscope}%
\pgfsys@transformshift{0.883182in}{1.199121in}%
\pgfsys@useobject{currentmarker}{}%
\end{pgfscope}%
\begin{pgfscope}%
\pgfsys@transformshift{0.480881in}{1.350069in}%
\pgfsys@useobject{currentmarker}{}%
\end{pgfscope}%
\begin{pgfscope}%
\pgfsys@transformshift{1.244411in}{1.254044in}%
\pgfsys@useobject{currentmarker}{}%
\end{pgfscope}%
\begin{pgfscope}%
\pgfsys@transformshift{1.224203in}{1.356252in}%
\pgfsys@useobject{currentmarker}{}%
\end{pgfscope}%
\begin{pgfscope}%
\pgfsys@transformshift{0.449848in}{1.450035in}%
\pgfsys@useobject{currentmarker}{}%
\end{pgfscope}%
\begin{pgfscope}%
\pgfsys@transformshift{0.582527in}{1.046236in}%
\pgfsys@useobject{currentmarker}{}%
\end{pgfscope}%
\begin{pgfscope}%
\pgfsys@transformshift{1.714524in}{1.727609in}%
\pgfsys@useobject{currentmarker}{}%
\end{pgfscope}%
\begin{pgfscope}%
\pgfsys@transformshift{0.891264in}{1.441302in}%
\pgfsys@useobject{currentmarker}{}%
\end{pgfscope}%
\begin{pgfscope}%
\pgfsys@transformshift{1.857615in}{1.662876in}%
\pgfsys@useobject{currentmarker}{}%
\end{pgfscope}%
\begin{pgfscope}%
\pgfsys@transformshift{1.511327in}{1.416535in}%
\pgfsys@useobject{currentmarker}{}%
\end{pgfscope}%
\begin{pgfscope}%
\pgfsys@transformshift{1.748915in}{1.171075in}%
\pgfsys@useobject{currentmarker}{}%
\end{pgfscope}%
\begin{pgfscope}%
\pgfsys@transformshift{0.491937in}{1.277981in}%
\pgfsys@useobject{currentmarker}{}%
\end{pgfscope}%
\begin{pgfscope}%
\pgfsys@transformshift{0.556274in}{1.167835in}%
\pgfsys@useobject{currentmarker}{}%
\end{pgfscope}%
\begin{pgfscope}%
\pgfsys@transformshift{0.468664in}{1.420045in}%
\pgfsys@useobject{currentmarker}{}%
\end{pgfscope}%
\begin{pgfscope}%
\pgfsys@transformshift{1.550098in}{1.434573in}%
\pgfsys@useobject{currentmarker}{}%
\end{pgfscope}%
\begin{pgfscope}%
\pgfsys@transformshift{0.533554in}{1.165732in}%
\pgfsys@useobject{currentmarker}{}%
\end{pgfscope}%
\begin{pgfscope}%
\pgfsys@transformshift{1.009532in}{1.288450in}%
\pgfsys@useobject{currentmarker}{}%
\end{pgfscope}%
\begin{pgfscope}%
\pgfsys@transformshift{0.590179in}{1.041719in}%
\pgfsys@useobject{currentmarker}{}%
\end{pgfscope}%
\begin{pgfscope}%
\pgfsys@transformshift{0.432340in}{1.479087in}%
\pgfsys@useobject{currentmarker}{}%
\end{pgfscope}%
\begin{pgfscope}%
\pgfsys@transformshift{0.372906in}{1.250971in}%
\pgfsys@useobject{currentmarker}{}%
\end{pgfscope}%
\begin{pgfscope}%
\pgfsys@transformshift{0.459645in}{1.303805in}%
\pgfsys@useobject{currentmarker}{}%
\end{pgfscope}%
\begin{pgfscope}%
\pgfsys@transformshift{1.482135in}{1.305615in}%
\pgfsys@useobject{currentmarker}{}%
\end{pgfscope}%
\begin{pgfscope}%
\pgfsys@transformshift{0.622605in}{1.031449in}%
\pgfsys@useobject{currentmarker}{}%
\end{pgfscope}%
\begin{pgfscope}%
\pgfsys@transformshift{0.489705in}{1.339750in}%
\pgfsys@useobject{currentmarker}{}%
\end{pgfscope}%
\begin{pgfscope}%
\pgfsys@transformshift{1.102334in}{1.169121in}%
\pgfsys@useobject{currentmarker}{}%
\end{pgfscope}%
\begin{pgfscope}%
\pgfsys@transformshift{0.877887in}{1.135600in}%
\pgfsys@useobject{currentmarker}{}%
\end{pgfscope}%
\begin{pgfscope}%
\pgfsys@transformshift{0.519403in}{1.187150in}%
\pgfsys@useobject{currentmarker}{}%
\end{pgfscope}%
\begin{pgfscope}%
\pgfsys@transformshift{0.938030in}{1.169545in}%
\pgfsys@useobject{currentmarker}{}%
\end{pgfscope}%
\begin{pgfscope}%
\pgfsys@transformshift{0.961331in}{1.027789in}%
\pgfsys@useobject{currentmarker}{}%
\end{pgfscope}%
\begin{pgfscope}%
\pgfsys@transformshift{1.036665in}{1.440806in}%
\pgfsys@useobject{currentmarker}{}%
\end{pgfscope}%
\begin{pgfscope}%
\pgfsys@transformshift{0.628071in}{1.212076in}%
\pgfsys@useobject{currentmarker}{}%
\end{pgfscope}%
\begin{pgfscope}%
\pgfsys@transformshift{0.835422in}{1.488022in}%
\pgfsys@useobject{currentmarker}{}%
\end{pgfscope}%
\begin{pgfscope}%
\pgfsys@transformshift{1.066394in}{1.350204in}%
\pgfsys@useobject{currentmarker}{}%
\end{pgfscope}%
\begin{pgfscope}%
\pgfsys@transformshift{1.436938in}{1.333470in}%
\pgfsys@useobject{currentmarker}{}%
\end{pgfscope}%
\begin{pgfscope}%
\pgfsys@transformshift{1.603771in}{1.378040in}%
\pgfsys@useobject{currentmarker}{}%
\end{pgfscope}%
\begin{pgfscope}%
\pgfsys@transformshift{0.959028in}{1.170039in}%
\pgfsys@useobject{currentmarker}{}%
\end{pgfscope}%
\begin{pgfscope}%
\pgfsys@transformshift{0.432442in}{1.247120in}%
\pgfsys@useobject{currentmarker}{}%
\end{pgfscope}%
\begin{pgfscope}%
\pgfsys@transformshift{0.897917in}{1.416962in}%
\pgfsys@useobject{currentmarker}{}%
\end{pgfscope}%
\begin{pgfscope}%
\pgfsys@transformshift{0.894441in}{0.897817in}%
\pgfsys@useobject{currentmarker}{}%
\end{pgfscope}%
\begin{pgfscope}%
\pgfsys@transformshift{0.401578in}{1.436080in}%
\pgfsys@useobject{currentmarker}{}%
\end{pgfscope}%
\begin{pgfscope}%
\pgfsys@transformshift{0.411375in}{1.268414in}%
\pgfsys@useobject{currentmarker}{}%
\end{pgfscope}%
\begin{pgfscope}%
\pgfsys@transformshift{0.874473in}{1.085050in}%
\pgfsys@useobject{currentmarker}{}%
\end{pgfscope}%
\begin{pgfscope}%
\pgfsys@transformshift{0.989940in}{1.250945in}%
\pgfsys@useobject{currentmarker}{}%
\end{pgfscope}%
\begin{pgfscope}%
\pgfsys@transformshift{0.410574in}{1.472184in}%
\pgfsys@useobject{currentmarker}{}%
\end{pgfscope}%
\begin{pgfscope}%
\pgfsys@transformshift{0.887331in}{1.211480in}%
\pgfsys@useobject{currentmarker}{}%
\end{pgfscope}%
\begin{pgfscope}%
\pgfsys@transformshift{0.844030in}{1.081422in}%
\pgfsys@useobject{currentmarker}{}%
\end{pgfscope}%
\begin{pgfscope}%
\pgfsys@transformshift{0.923946in}{1.126552in}%
\pgfsys@useobject{currentmarker}{}%
\end{pgfscope}%
\begin{pgfscope}%
\pgfsys@transformshift{1.517373in}{1.403531in}%
\pgfsys@useobject{currentmarker}{}%
\end{pgfscope}%
\begin{pgfscope}%
\pgfsys@transformshift{1.544776in}{1.550746in}%
\pgfsys@useobject{currentmarker}{}%
\end{pgfscope}%
\begin{pgfscope}%
\pgfsys@transformshift{0.466736in}{1.116839in}%
\pgfsys@useobject{currentmarker}{}%
\end{pgfscope}%
\begin{pgfscope}%
\pgfsys@transformshift{0.539327in}{0.963250in}%
\pgfsys@useobject{currentmarker}{}%
\end{pgfscope}%
\begin{pgfscope}%
\pgfsys@transformshift{1.837625in}{1.458725in}%
\pgfsys@useobject{currentmarker}{}%
\end{pgfscope}%
\begin{pgfscope}%
\pgfsys@transformshift{0.442869in}{1.229769in}%
\pgfsys@useobject{currentmarker}{}%
\end{pgfscope}%
\begin{pgfscope}%
\pgfsys@transformshift{1.136179in}{1.574831in}%
\pgfsys@useobject{currentmarker}{}%
\end{pgfscope}%
\begin{pgfscope}%
\pgfsys@transformshift{0.540575in}{1.320600in}%
\pgfsys@useobject{currentmarker}{}%
\end{pgfscope}%
\begin{pgfscope}%
\pgfsys@transformshift{0.505594in}{1.042247in}%
\pgfsys@useobject{currentmarker}{}%
\end{pgfscope}%
\begin{pgfscope}%
\pgfsys@transformshift{0.464353in}{1.064058in}%
\pgfsys@useobject{currentmarker}{}%
\end{pgfscope}%
\begin{pgfscope}%
\pgfsys@transformshift{0.448024in}{1.184051in}%
\pgfsys@useobject{currentmarker}{}%
\end{pgfscope}%
\begin{pgfscope}%
\pgfsys@transformshift{1.133798in}{1.430971in}%
\pgfsys@useobject{currentmarker}{}%
\end{pgfscope}%
\begin{pgfscope}%
\pgfsys@transformshift{0.450472in}{1.165754in}%
\pgfsys@useobject{currentmarker}{}%
\end{pgfscope}%
\begin{pgfscope}%
\pgfsys@transformshift{0.915981in}{0.953298in}%
\pgfsys@useobject{currentmarker}{}%
\end{pgfscope}%
\begin{pgfscope}%
\pgfsys@transformshift{0.893518in}{1.402013in}%
\pgfsys@useobject{currentmarker}{}%
\end{pgfscope}%
\begin{pgfscope}%
\pgfsys@transformshift{0.839737in}{1.045586in}%
\pgfsys@useobject{currentmarker}{}%
\end{pgfscope}%
\begin{pgfscope}%
\pgfsys@transformshift{0.466023in}{1.331980in}%
\pgfsys@useobject{currentmarker}{}%
\end{pgfscope}%
\begin{pgfscope}%
\pgfsys@transformshift{1.451837in}{1.441183in}%
\pgfsys@useobject{currentmarker}{}%
\end{pgfscope}%
\begin{pgfscope}%
\pgfsys@transformshift{1.812241in}{1.154061in}%
\pgfsys@useobject{currentmarker}{}%
\end{pgfscope}%
\begin{pgfscope}%
\pgfsys@transformshift{0.470694in}{1.273955in}%
\pgfsys@useobject{currentmarker}{}%
\end{pgfscope}%
\begin{pgfscope}%
\pgfsys@transformshift{0.373662in}{1.300638in}%
\pgfsys@useobject{currentmarker}{}%
\end{pgfscope}%
\begin{pgfscope}%
\pgfsys@transformshift{0.403930in}{1.496310in}%
\pgfsys@useobject{currentmarker}{}%
\end{pgfscope}%
\begin{pgfscope}%
\pgfsys@transformshift{1.148968in}{1.581264in}%
\pgfsys@useobject{currentmarker}{}%
\end{pgfscope}%
\begin{pgfscope}%
\pgfsys@transformshift{0.466738in}{1.270482in}%
\pgfsys@useobject{currentmarker}{}%
\end{pgfscope}%
\begin{pgfscope}%
\pgfsys@transformshift{0.600280in}{1.344054in}%
\pgfsys@useobject{currentmarker}{}%
\end{pgfscope}%
\begin{pgfscope}%
\pgfsys@transformshift{0.430403in}{1.323415in}%
\pgfsys@useobject{currentmarker}{}%
\end{pgfscope}%
\begin{pgfscope}%
\pgfsys@transformshift{0.917645in}{1.390621in}%
\pgfsys@useobject{currentmarker}{}%
\end{pgfscope}%
\begin{pgfscope}%
\pgfsys@transformshift{1.972216in}{1.141139in}%
\pgfsys@useobject{currentmarker}{}%
\end{pgfscope}%
\begin{pgfscope}%
\pgfsys@transformshift{0.947192in}{1.410391in}%
\pgfsys@useobject{currentmarker}{}%
\end{pgfscope}%
\begin{pgfscope}%
\pgfsys@transformshift{0.945142in}{1.129262in}%
\pgfsys@useobject{currentmarker}{}%
\end{pgfscope}%
\begin{pgfscope}%
\pgfsys@transformshift{0.884523in}{1.135351in}%
\pgfsys@useobject{currentmarker}{}%
\end{pgfscope}%
\begin{pgfscope}%
\pgfsys@transformshift{0.868046in}{1.214543in}%
\pgfsys@useobject{currentmarker}{}%
\end{pgfscope}%
\begin{pgfscope}%
\pgfsys@transformshift{0.427823in}{1.352075in}%
\pgfsys@useobject{currentmarker}{}%
\end{pgfscope}%
\begin{pgfscope}%
\pgfsys@transformshift{1.240509in}{1.391172in}%
\pgfsys@useobject{currentmarker}{}%
\end{pgfscope}%
\begin{pgfscope}%
\pgfsys@transformshift{0.427367in}{1.576385in}%
\pgfsys@useobject{currentmarker}{}%
\end{pgfscope}%
\begin{pgfscope}%
\pgfsys@transformshift{0.877911in}{1.071753in}%
\pgfsys@useobject{currentmarker}{}%
\end{pgfscope}%
\begin{pgfscope}%
\pgfsys@transformshift{0.502381in}{1.077201in}%
\pgfsys@useobject{currentmarker}{}%
\end{pgfscope}%
\begin{pgfscope}%
\pgfsys@transformshift{0.897409in}{0.919237in}%
\pgfsys@useobject{currentmarker}{}%
\end{pgfscope}%
\begin{pgfscope}%
\pgfsys@transformshift{0.371513in}{1.155289in}%
\pgfsys@useobject{currentmarker}{}%
\end{pgfscope}%
\begin{pgfscope}%
\pgfsys@transformshift{0.424068in}{1.302890in}%
\pgfsys@useobject{currentmarker}{}%
\end{pgfscope}%
\begin{pgfscope}%
\pgfsys@transformshift{0.512683in}{0.892849in}%
\pgfsys@useobject{currentmarker}{}%
\end{pgfscope}%
\begin{pgfscope}%
\pgfsys@transformshift{0.423224in}{1.200747in}%
\pgfsys@useobject{currentmarker}{}%
\end{pgfscope}%
\begin{pgfscope}%
\pgfsys@transformshift{0.590216in}{1.082301in}%
\pgfsys@useobject{currentmarker}{}%
\end{pgfscope}%
\begin{pgfscope}%
\pgfsys@transformshift{1.015493in}{1.134772in}%
\pgfsys@useobject{currentmarker}{}%
\end{pgfscope}%
\begin{pgfscope}%
\pgfsys@transformshift{1.527374in}{1.457841in}%
\pgfsys@useobject{currentmarker}{}%
\end{pgfscope}%
\begin{pgfscope}%
\pgfsys@transformshift{1.794630in}{1.009026in}%
\pgfsys@useobject{currentmarker}{}%
\end{pgfscope}%
\begin{pgfscope}%
\pgfsys@transformshift{0.971721in}{1.386013in}%
\pgfsys@useobject{currentmarker}{}%
\end{pgfscope}%
\begin{pgfscope}%
\pgfsys@transformshift{0.402429in}{1.247239in}%
\pgfsys@useobject{currentmarker}{}%
\end{pgfscope}%
\begin{pgfscope}%
\pgfsys@transformshift{0.596209in}{1.083764in}%
\pgfsys@useobject{currentmarker}{}%
\end{pgfscope}%
\begin{pgfscope}%
\pgfsys@transformshift{0.596444in}{1.220013in}%
\pgfsys@useobject{currentmarker}{}%
\end{pgfscope}%
\begin{pgfscope}%
\pgfsys@transformshift{0.472945in}{1.233616in}%
\pgfsys@useobject{currentmarker}{}%
\end{pgfscope}%
\begin{pgfscope}%
\pgfsys@transformshift{0.505729in}{1.356762in}%
\pgfsys@useobject{currentmarker}{}%
\end{pgfscope}%
\begin{pgfscope}%
\pgfsys@transformshift{0.522392in}{1.193311in}%
\pgfsys@useobject{currentmarker}{}%
\end{pgfscope}%
\begin{pgfscope}%
\pgfsys@transformshift{0.958437in}{1.369494in}%
\pgfsys@useobject{currentmarker}{}%
\end{pgfscope}%
\begin{pgfscope}%
\pgfsys@transformshift{0.899312in}{1.529486in}%
\pgfsys@useobject{currentmarker}{}%
\end{pgfscope}%
\begin{pgfscope}%
\pgfsys@transformshift{0.477796in}{1.484361in}%
\pgfsys@useobject{currentmarker}{}%
\end{pgfscope}%
\begin{pgfscope}%
\pgfsys@transformshift{0.495644in}{1.295847in}%
\pgfsys@useobject{currentmarker}{}%
\end{pgfscope}%
\begin{pgfscope}%
\pgfsys@transformshift{0.461404in}{1.379425in}%
\pgfsys@useobject{currentmarker}{}%
\end{pgfscope}%
\begin{pgfscope}%
\pgfsys@transformshift{0.564846in}{1.261890in}%
\pgfsys@useobject{currentmarker}{}%
\end{pgfscope}%
\begin{pgfscope}%
\pgfsys@transformshift{0.438201in}{1.060657in}%
\pgfsys@useobject{currentmarker}{}%
\end{pgfscope}%
\begin{pgfscope}%
\pgfsys@transformshift{0.830917in}{1.437509in}%
\pgfsys@useobject{currentmarker}{}%
\end{pgfscope}%
\begin{pgfscope}%
\pgfsys@transformshift{0.470836in}{1.127368in}%
\pgfsys@useobject{currentmarker}{}%
\end{pgfscope}%
\begin{pgfscope}%
\pgfsys@transformshift{1.732430in}{1.430452in}%
\pgfsys@useobject{currentmarker}{}%
\end{pgfscope}%
\begin{pgfscope}%
\pgfsys@transformshift{0.880194in}{1.036652in}%
\pgfsys@useobject{currentmarker}{}%
\end{pgfscope}%
\begin{pgfscope}%
\pgfsys@transformshift{1.771020in}{1.112657in}%
\pgfsys@useobject{currentmarker}{}%
\end{pgfscope}%
\begin{pgfscope}%
\pgfsys@transformshift{0.408541in}{1.484373in}%
\pgfsys@useobject{currentmarker}{}%
\end{pgfscope}%
\begin{pgfscope}%
\pgfsys@transformshift{1.783013in}{1.110510in}%
\pgfsys@useobject{currentmarker}{}%
\end{pgfscope}%
\begin{pgfscope}%
\pgfsys@transformshift{1.734433in}{1.130040in}%
\pgfsys@useobject{currentmarker}{}%
\end{pgfscope}%
\begin{pgfscope}%
\pgfsys@transformshift{1.772028in}{1.705700in}%
\pgfsys@useobject{currentmarker}{}%
\end{pgfscope}%
\begin{pgfscope}%
\pgfsys@transformshift{0.482187in}{1.542357in}%
\pgfsys@useobject{currentmarker}{}%
\end{pgfscope}%
\begin{pgfscope}%
\pgfsys@transformshift{0.850634in}{0.930793in}%
\pgfsys@useobject{currentmarker}{}%
\end{pgfscope}%
\begin{pgfscope}%
\pgfsys@transformshift{1.489929in}{1.351484in}%
\pgfsys@useobject{currentmarker}{}%
\end{pgfscope}%
\begin{pgfscope}%
\pgfsys@transformshift{0.572833in}{1.162555in}%
\pgfsys@useobject{currentmarker}{}%
\end{pgfscope}%
\begin{pgfscope}%
\pgfsys@transformshift{0.459424in}{1.255506in}%
\pgfsys@useobject{currentmarker}{}%
\end{pgfscope}%
\begin{pgfscope}%
\pgfsys@transformshift{0.454844in}{1.225830in}%
\pgfsys@useobject{currentmarker}{}%
\end{pgfscope}%
\begin{pgfscope}%
\pgfsys@transformshift{1.529823in}{1.393384in}%
\pgfsys@useobject{currentmarker}{}%
\end{pgfscope}%
\begin{pgfscope}%
\pgfsys@transformshift{0.477367in}{1.395996in}%
\pgfsys@useobject{currentmarker}{}%
\end{pgfscope}%
\begin{pgfscope}%
\pgfsys@transformshift{1.049797in}{1.288212in}%
\pgfsys@useobject{currentmarker}{}%
\end{pgfscope}%
\begin{pgfscope}%
\pgfsys@transformshift{1.390925in}{1.800615in}%
\pgfsys@useobject{currentmarker}{}%
\end{pgfscope}%
\begin{pgfscope}%
\pgfsys@transformshift{1.615976in}{1.131333in}%
\pgfsys@useobject{currentmarker}{}%
\end{pgfscope}%
\begin{pgfscope}%
\pgfsys@transformshift{1.879685in}{1.156900in}%
\pgfsys@useobject{currentmarker}{}%
\end{pgfscope}%
\begin{pgfscope}%
\pgfsys@transformshift{0.474522in}{1.240450in}%
\pgfsys@useobject{currentmarker}{}%
\end{pgfscope}%
\begin{pgfscope}%
\pgfsys@transformshift{0.508097in}{1.412479in}%
\pgfsys@useobject{currentmarker}{}%
\end{pgfscope}%
\begin{pgfscope}%
\pgfsys@transformshift{0.846995in}{1.048332in}%
\pgfsys@useobject{currentmarker}{}%
\end{pgfscope}%
\begin{pgfscope}%
\pgfsys@transformshift{1.720538in}{1.549828in}%
\pgfsys@useobject{currentmarker}{}%
\end{pgfscope}%
\begin{pgfscope}%
\pgfsys@transformshift{1.561110in}{1.359412in}%
\pgfsys@useobject{currentmarker}{}%
\end{pgfscope}%
\begin{pgfscope}%
\pgfsys@transformshift{0.477188in}{1.441379in}%
\pgfsys@useobject{currentmarker}{}%
\end{pgfscope}%
\begin{pgfscope}%
\pgfsys@transformshift{0.462555in}{1.328383in}%
\pgfsys@useobject{currentmarker}{}%
\end{pgfscope}%
\begin{pgfscope}%
\pgfsys@transformshift{1.923906in}{1.604214in}%
\pgfsys@useobject{currentmarker}{}%
\end{pgfscope}%
\begin{pgfscope}%
\pgfsys@transformshift{0.990232in}{1.425142in}%
\pgfsys@useobject{currentmarker}{}%
\end{pgfscope}%
\begin{pgfscope}%
\pgfsys@transformshift{0.858173in}{1.119905in}%
\pgfsys@useobject{currentmarker}{}%
\end{pgfscope}%
\begin{pgfscope}%
\pgfsys@transformshift{0.955286in}{1.497259in}%
\pgfsys@useobject{currentmarker}{}%
\end{pgfscope}%
\begin{pgfscope}%
\pgfsys@transformshift{1.614084in}{1.578339in}%
\pgfsys@useobject{currentmarker}{}%
\end{pgfscope}%
\begin{pgfscope}%
\pgfsys@transformshift{0.839480in}{1.476984in}%
\pgfsys@useobject{currentmarker}{}%
\end{pgfscope}%
\begin{pgfscope}%
\pgfsys@transformshift{0.579938in}{1.325724in}%
\pgfsys@useobject{currentmarker}{}%
\end{pgfscope}%
\begin{pgfscope}%
\pgfsys@transformshift{1.021308in}{1.264045in}%
\pgfsys@useobject{currentmarker}{}%
\end{pgfscope}%
\begin{pgfscope}%
\pgfsys@transformshift{0.489542in}{1.303825in}%
\pgfsys@useobject{currentmarker}{}%
\end{pgfscope}%
\begin{pgfscope}%
\pgfsys@transformshift{0.482496in}{1.095996in}%
\pgfsys@useobject{currentmarker}{}%
\end{pgfscope}%
\begin{pgfscope}%
\pgfsys@transformshift{1.231263in}{1.323961in}%
\pgfsys@useobject{currentmarker}{}%
\end{pgfscope}%
\begin{pgfscope}%
\pgfsys@transformshift{0.612346in}{1.239647in}%
\pgfsys@useobject{currentmarker}{}%
\end{pgfscope}%
\begin{pgfscope}%
\pgfsys@transformshift{0.596517in}{1.107225in}%
\pgfsys@useobject{currentmarker}{}%
\end{pgfscope}%
\begin{pgfscope}%
\pgfsys@transformshift{1.011703in}{1.100291in}%
\pgfsys@useobject{currentmarker}{}%
\end{pgfscope}%
\begin{pgfscope}%
\pgfsys@transformshift{0.536110in}{1.102691in}%
\pgfsys@useobject{currentmarker}{}%
\end{pgfscope}%
\begin{pgfscope}%
\pgfsys@transformshift{0.473641in}{1.251831in}%
\pgfsys@useobject{currentmarker}{}%
\end{pgfscope}%
\begin{pgfscope}%
\pgfsys@transformshift{0.949604in}{1.207898in}%
\pgfsys@useobject{currentmarker}{}%
\end{pgfscope}%
\begin{pgfscope}%
\pgfsys@transformshift{1.891487in}{1.688427in}%
\pgfsys@useobject{currentmarker}{}%
\end{pgfscope}%
\begin{pgfscope}%
\pgfsys@transformshift{0.459142in}{1.404075in}%
\pgfsys@useobject{currentmarker}{}%
\end{pgfscope}%
\begin{pgfscope}%
\pgfsys@transformshift{0.993745in}{1.484413in}%
\pgfsys@useobject{currentmarker}{}%
\end{pgfscope}%
\begin{pgfscope}%
\pgfsys@transformshift{0.461010in}{1.333377in}%
\pgfsys@useobject{currentmarker}{}%
\end{pgfscope}%
\begin{pgfscope}%
\pgfsys@transformshift{0.452198in}{1.316032in}%
\pgfsys@useobject{currentmarker}{}%
\end{pgfscope}%
\begin{pgfscope}%
\pgfsys@transformshift{0.567084in}{1.262109in}%
\pgfsys@useobject{currentmarker}{}%
\end{pgfscope}%
\begin{pgfscope}%
\pgfsys@transformshift{0.925975in}{1.500296in}%
\pgfsys@useobject{currentmarker}{}%
\end{pgfscope}%
\begin{pgfscope}%
\pgfsys@transformshift{1.694241in}{1.013319in}%
\pgfsys@useobject{currentmarker}{}%
\end{pgfscope}%
\begin{pgfscope}%
\pgfsys@transformshift{0.589923in}{0.976090in}%
\pgfsys@useobject{currentmarker}{}%
\end{pgfscope}%
\begin{pgfscope}%
\pgfsys@transformshift{1.421991in}{1.369284in}%
\pgfsys@useobject{currentmarker}{}%
\end{pgfscope}%
\begin{pgfscope}%
\pgfsys@transformshift{0.569884in}{1.338992in}%
\pgfsys@useobject{currentmarker}{}%
\end{pgfscope}%
\begin{pgfscope}%
\pgfsys@transformshift{0.481143in}{0.848914in}%
\pgfsys@useobject{currentmarker}{}%
\end{pgfscope}%
\begin{pgfscope}%
\pgfsys@transformshift{0.598950in}{0.931032in}%
\pgfsys@useobject{currentmarker}{}%
\end{pgfscope}%
\begin{pgfscope}%
\pgfsys@transformshift{0.891632in}{1.010774in}%
\pgfsys@useobject{currentmarker}{}%
\end{pgfscope}%
\begin{pgfscope}%
\pgfsys@transformshift{1.518904in}{1.456998in}%
\pgfsys@useobject{currentmarker}{}%
\end{pgfscope}%
\begin{pgfscope}%
\pgfsys@transformshift{1.267712in}{1.283228in}%
\pgfsys@useobject{currentmarker}{}%
\end{pgfscope}%
\begin{pgfscope}%
\pgfsys@transformshift{1.736581in}{1.797052in}%
\pgfsys@useobject{currentmarker}{}%
\end{pgfscope}%
\begin{pgfscope}%
\pgfsys@transformshift{0.507870in}{1.192874in}%
\pgfsys@useobject{currentmarker}{}%
\end{pgfscope}%
\begin{pgfscope}%
\pgfsys@transformshift{0.876979in}{1.171973in}%
\pgfsys@useobject{currentmarker}{}%
\end{pgfscope}%
\begin{pgfscope}%
\pgfsys@transformshift{0.975052in}{1.407525in}%
\pgfsys@useobject{currentmarker}{}%
\end{pgfscope}%
\begin{pgfscope}%
\pgfsys@transformshift{0.581391in}{1.189702in}%
\pgfsys@useobject{currentmarker}{}%
\end{pgfscope}%
\begin{pgfscope}%
\pgfsys@transformshift{0.432905in}{1.236292in}%
\pgfsys@useobject{currentmarker}{}%
\end{pgfscope}%
\begin{pgfscope}%
\pgfsys@transformshift{0.887125in}{1.507604in}%
\pgfsys@useobject{currentmarker}{}%
\end{pgfscope}%
\begin{pgfscope}%
\pgfsys@transformshift{0.887888in}{0.993421in}%
\pgfsys@useobject{currentmarker}{}%
\end{pgfscope}%
\begin{pgfscope}%
\pgfsys@transformshift{1.047522in}{1.352737in}%
\pgfsys@useobject{currentmarker}{}%
\end{pgfscope}%
\begin{pgfscope}%
\pgfsys@transformshift{0.459882in}{1.641709in}%
\pgfsys@useobject{currentmarker}{}%
\end{pgfscope}%
\begin{pgfscope}%
\pgfsys@transformshift{0.869741in}{0.976271in}%
\pgfsys@useobject{currentmarker}{}%
\end{pgfscope}%
\begin{pgfscope}%
\pgfsys@transformshift{1.519451in}{1.368911in}%
\pgfsys@useobject{currentmarker}{}%
\end{pgfscope}%
\begin{pgfscope}%
\pgfsys@transformshift{1.704307in}{1.431722in}%
\pgfsys@useobject{currentmarker}{}%
\end{pgfscope}%
\begin{pgfscope}%
\pgfsys@transformshift{1.029054in}{1.321096in}%
\pgfsys@useobject{currentmarker}{}%
\end{pgfscope}%
\begin{pgfscope}%
\pgfsys@transformshift{0.983134in}{1.337550in}%
\pgfsys@useobject{currentmarker}{}%
\end{pgfscope}%
\begin{pgfscope}%
\pgfsys@transformshift{1.345503in}{1.434797in}%
\pgfsys@useobject{currentmarker}{}%
\end{pgfscope}%
\begin{pgfscope}%
\pgfsys@transformshift{0.480154in}{1.520920in}%
\pgfsys@useobject{currentmarker}{}%
\end{pgfscope}%
\begin{pgfscope}%
\pgfsys@transformshift{0.820400in}{1.016039in}%
\pgfsys@useobject{currentmarker}{}%
\end{pgfscope}%
\begin{pgfscope}%
\pgfsys@transformshift{0.890253in}{1.500797in}%
\pgfsys@useobject{currentmarker}{}%
\end{pgfscope}%
\begin{pgfscope}%
\pgfsys@transformshift{1.194868in}{1.582416in}%
\pgfsys@useobject{currentmarker}{}%
\end{pgfscope}%
\begin{pgfscope}%
\pgfsys@transformshift{0.938368in}{1.694059in}%
\pgfsys@useobject{currentmarker}{}%
\end{pgfscope}%
\begin{pgfscope}%
\pgfsys@transformshift{0.663011in}{1.005039in}%
\pgfsys@useobject{currentmarker}{}%
\end{pgfscope}%
\begin{pgfscope}%
\pgfsys@transformshift{0.916322in}{0.946409in}%
\pgfsys@useobject{currentmarker}{}%
\end{pgfscope}%
\begin{pgfscope}%
\pgfsys@transformshift{0.890291in}{1.188555in}%
\pgfsys@useobject{currentmarker}{}%
\end{pgfscope}%
\begin{pgfscope}%
\pgfsys@transformshift{0.909236in}{0.954110in}%
\pgfsys@useobject{currentmarker}{}%
\end{pgfscope}%
\begin{pgfscope}%
\pgfsys@transformshift{0.862619in}{1.014159in}%
\pgfsys@useobject{currentmarker}{}%
\end{pgfscope}%
\begin{pgfscope}%
\pgfsys@transformshift{0.597487in}{1.362886in}%
\pgfsys@useobject{currentmarker}{}%
\end{pgfscope}%
\begin{pgfscope}%
\pgfsys@transformshift{1.964345in}{1.141644in}%
\pgfsys@useobject{currentmarker}{}%
\end{pgfscope}%
\begin{pgfscope}%
\pgfsys@transformshift{1.692774in}{1.109522in}%
\pgfsys@useobject{currentmarker}{}%
\end{pgfscope}%
\begin{pgfscope}%
\pgfsys@transformshift{0.482531in}{1.322066in}%
\pgfsys@useobject{currentmarker}{}%
\end{pgfscope}%
\begin{pgfscope}%
\pgfsys@transformshift{0.804579in}{0.981671in}%
\pgfsys@useobject{currentmarker}{}%
\end{pgfscope}%
\begin{pgfscope}%
\pgfsys@transformshift{0.458204in}{1.207746in}%
\pgfsys@useobject{currentmarker}{}%
\end{pgfscope}%
\begin{pgfscope}%
\pgfsys@transformshift{1.029005in}{1.261818in}%
\pgfsys@useobject{currentmarker}{}%
\end{pgfscope}%
\begin{pgfscope}%
\pgfsys@transformshift{0.509849in}{0.966337in}%
\pgfsys@useobject{currentmarker}{}%
\end{pgfscope}%
\begin{pgfscope}%
\pgfsys@transformshift{1.034049in}{1.271159in}%
\pgfsys@useobject{currentmarker}{}%
\end{pgfscope}%
\begin{pgfscope}%
\pgfsys@transformshift{1.152583in}{1.361312in}%
\pgfsys@useobject{currentmarker}{}%
\end{pgfscope}%
\begin{pgfscope}%
\pgfsys@transformshift{1.044006in}{1.410445in}%
\pgfsys@useobject{currentmarker}{}%
\end{pgfscope}%
\begin{pgfscope}%
\pgfsys@transformshift{1.486794in}{1.535659in}%
\pgfsys@useobject{currentmarker}{}%
\end{pgfscope}%
\begin{pgfscope}%
\pgfsys@transformshift{0.870195in}{1.067367in}%
\pgfsys@useobject{currentmarker}{}%
\end{pgfscope}%
\begin{pgfscope}%
\pgfsys@transformshift{1.138693in}{1.425613in}%
\pgfsys@useobject{currentmarker}{}%
\end{pgfscope}%
\begin{pgfscope}%
\pgfsys@transformshift{0.847963in}{0.961714in}%
\pgfsys@useobject{currentmarker}{}%
\end{pgfscope}%
\begin{pgfscope}%
\pgfsys@transformshift{0.477983in}{1.323590in}%
\pgfsys@useobject{currentmarker}{}%
\end{pgfscope}%
\begin{pgfscope}%
\pgfsys@transformshift{0.492721in}{1.203550in}%
\pgfsys@useobject{currentmarker}{}%
\end{pgfscope}%
\begin{pgfscope}%
\pgfsys@transformshift{1.120030in}{1.414248in}%
\pgfsys@useobject{currentmarker}{}%
\end{pgfscope}%
\begin{pgfscope}%
\pgfsys@transformshift{0.895919in}{1.291414in}%
\pgfsys@useobject{currentmarker}{}%
\end{pgfscope}%
\begin{pgfscope}%
\pgfsys@transformshift{0.576124in}{1.049759in}%
\pgfsys@useobject{currentmarker}{}%
\end{pgfscope}%
\begin{pgfscope}%
\pgfsys@transformshift{0.471843in}{1.222970in}%
\pgfsys@useobject{currentmarker}{}%
\end{pgfscope}%
\begin{pgfscope}%
\pgfsys@transformshift{0.475316in}{1.105818in}%
\pgfsys@useobject{currentmarker}{}%
\end{pgfscope}%
\begin{pgfscope}%
\pgfsys@transformshift{1.653595in}{1.466013in}%
\pgfsys@useobject{currentmarker}{}%
\end{pgfscope}%
\begin{pgfscope}%
\pgfsys@transformshift{1.083484in}{1.307948in}%
\pgfsys@useobject{currentmarker}{}%
\end{pgfscope}%
\begin{pgfscope}%
\pgfsys@transformshift{0.929704in}{1.340190in}%
\pgfsys@useobject{currentmarker}{}%
\end{pgfscope}%
\begin{pgfscope}%
\pgfsys@transformshift{1.702011in}{1.588117in}%
\pgfsys@useobject{currentmarker}{}%
\end{pgfscope}%
\begin{pgfscope}%
\pgfsys@transformshift{0.861736in}{0.903142in}%
\pgfsys@useobject{currentmarker}{}%
\end{pgfscope}%
\begin{pgfscope}%
\pgfsys@transformshift{0.466758in}{1.168785in}%
\pgfsys@useobject{currentmarker}{}%
\end{pgfscope}%
\begin{pgfscope}%
\pgfsys@transformshift{0.895642in}{1.263776in}%
\pgfsys@useobject{currentmarker}{}%
\end{pgfscope}%
\begin{pgfscope}%
\pgfsys@transformshift{0.844526in}{1.252565in}%
\pgfsys@useobject{currentmarker}{}%
\end{pgfscope}%
\begin{pgfscope}%
\pgfsys@transformshift{0.903057in}{1.490866in}%
\pgfsys@useobject{currentmarker}{}%
\end{pgfscope}%
\begin{pgfscope}%
\pgfsys@transformshift{0.952685in}{1.171810in}%
\pgfsys@useobject{currentmarker}{}%
\end{pgfscope}%
\begin{pgfscope}%
\pgfsys@transformshift{0.576385in}{1.273560in}%
\pgfsys@useobject{currentmarker}{}%
\end{pgfscope}%
\begin{pgfscope}%
\pgfsys@transformshift{1.016020in}{1.385189in}%
\pgfsys@useobject{currentmarker}{}%
\end{pgfscope}%
\begin{pgfscope}%
\pgfsys@transformshift{0.494156in}{1.059888in}%
\pgfsys@useobject{currentmarker}{}%
\end{pgfscope}%
\begin{pgfscope}%
\pgfsys@transformshift{0.577397in}{1.031859in}%
\pgfsys@useobject{currentmarker}{}%
\end{pgfscope}%
\begin{pgfscope}%
\pgfsys@transformshift{0.445519in}{1.180613in}%
\pgfsys@useobject{currentmarker}{}%
\end{pgfscope}%
\begin{pgfscope}%
\pgfsys@transformshift{1.084282in}{1.294459in}%
\pgfsys@useobject{currentmarker}{}%
\end{pgfscope}%
\begin{pgfscope}%
\pgfsys@transformshift{1.594805in}{1.376634in}%
\pgfsys@useobject{currentmarker}{}%
\end{pgfscope}%
\begin{pgfscope}%
\pgfsys@transformshift{0.976071in}{1.361664in}%
\pgfsys@useobject{currentmarker}{}%
\end{pgfscope}%
\begin{pgfscope}%
\pgfsys@transformshift{0.954554in}{1.531928in}%
\pgfsys@useobject{currentmarker}{}%
\end{pgfscope}%
\begin{pgfscope}%
\pgfsys@transformshift{1.062412in}{1.586640in}%
\pgfsys@useobject{currentmarker}{}%
\end{pgfscope}%
\begin{pgfscope}%
\pgfsys@transformshift{0.403708in}{1.130497in}%
\pgfsys@useobject{currentmarker}{}%
\end{pgfscope}%
\begin{pgfscope}%
\pgfsys@transformshift{0.456382in}{1.216600in}%
\pgfsys@useobject{currentmarker}{}%
\end{pgfscope}%
\begin{pgfscope}%
\pgfsys@transformshift{0.425389in}{1.056461in}%
\pgfsys@useobject{currentmarker}{}%
\end{pgfscope}%
\begin{pgfscope}%
\pgfsys@transformshift{1.178319in}{1.137978in}%
\pgfsys@useobject{currentmarker}{}%
\end{pgfscope}%
\begin{pgfscope}%
\pgfsys@transformshift{0.868624in}{1.470723in}%
\pgfsys@useobject{currentmarker}{}%
\end{pgfscope}%
\begin{pgfscope}%
\pgfsys@transformshift{1.072938in}{1.318761in}%
\pgfsys@useobject{currentmarker}{}%
\end{pgfscope}%
\begin{pgfscope}%
\pgfsys@transformshift{0.842975in}{1.046531in}%
\pgfsys@useobject{currentmarker}{}%
\end{pgfscope}%
\begin{pgfscope}%
\pgfsys@transformshift{1.133541in}{1.391786in}%
\pgfsys@useobject{currentmarker}{}%
\end{pgfscope}%
\begin{pgfscope}%
\pgfsys@transformshift{0.985168in}{1.497243in}%
\pgfsys@useobject{currentmarker}{}%
\end{pgfscope}%
\begin{pgfscope}%
\pgfsys@transformshift{0.935163in}{1.618225in}%
\pgfsys@useobject{currentmarker}{}%
\end{pgfscope}%
\begin{pgfscope}%
\pgfsys@transformshift{0.852856in}{1.072802in}%
\pgfsys@useobject{currentmarker}{}%
\end{pgfscope}%
\begin{pgfscope}%
\pgfsys@transformshift{0.481479in}{1.039272in}%
\pgfsys@useobject{currentmarker}{}%
\end{pgfscope}%
\begin{pgfscope}%
\pgfsys@transformshift{0.483502in}{1.359445in}%
\pgfsys@useobject{currentmarker}{}%
\end{pgfscope}%
\begin{pgfscope}%
\pgfsys@transformshift{0.857024in}{0.886518in}%
\pgfsys@useobject{currentmarker}{}%
\end{pgfscope}%
\begin{pgfscope}%
\pgfsys@transformshift{0.541865in}{1.025738in}%
\pgfsys@useobject{currentmarker}{}%
\end{pgfscope}%
\begin{pgfscope}%
\pgfsys@transformshift{1.564506in}{1.513075in}%
\pgfsys@useobject{currentmarker}{}%
\end{pgfscope}%
\begin{pgfscope}%
\pgfsys@transformshift{0.926665in}{1.485507in}%
\pgfsys@useobject{currentmarker}{}%
\end{pgfscope}%
\begin{pgfscope}%
\pgfsys@transformshift{0.371504in}{1.316713in}%
\pgfsys@useobject{currentmarker}{}%
\end{pgfscope}%
\begin{pgfscope}%
\pgfsys@transformshift{0.869520in}{1.238657in}%
\pgfsys@useobject{currentmarker}{}%
\end{pgfscope}%
\begin{pgfscope}%
\pgfsys@transformshift{0.519844in}{1.032321in}%
\pgfsys@useobject{currentmarker}{}%
\end{pgfscope}%
\begin{pgfscope}%
\pgfsys@transformshift{0.608978in}{1.017183in}%
\pgfsys@useobject{currentmarker}{}%
\end{pgfscope}%
\begin{pgfscope}%
\pgfsys@transformshift{0.476814in}{1.186349in}%
\pgfsys@useobject{currentmarker}{}%
\end{pgfscope}%
\begin{pgfscope}%
\pgfsys@transformshift{0.933077in}{1.352872in}%
\pgfsys@useobject{currentmarker}{}%
\end{pgfscope}%
\begin{pgfscope}%
\pgfsys@transformshift{0.579351in}{1.290964in}%
\pgfsys@useobject{currentmarker}{}%
\end{pgfscope}%
\begin{pgfscope}%
\pgfsys@transformshift{0.459383in}{1.204087in}%
\pgfsys@useobject{currentmarker}{}%
\end{pgfscope}%
\begin{pgfscope}%
\pgfsys@transformshift{0.451992in}{1.255055in}%
\pgfsys@useobject{currentmarker}{}%
\end{pgfscope}%
\begin{pgfscope}%
\pgfsys@transformshift{0.448649in}{1.384474in}%
\pgfsys@useobject{currentmarker}{}%
\end{pgfscope}%
\begin{pgfscope}%
\pgfsys@transformshift{1.803636in}{1.618892in}%
\pgfsys@useobject{currentmarker}{}%
\end{pgfscope}%
\begin{pgfscope}%
\pgfsys@transformshift{1.680858in}{1.034231in}%
\pgfsys@useobject{currentmarker}{}%
\end{pgfscope}%
\begin{pgfscope}%
\pgfsys@transformshift{0.464450in}{1.169364in}%
\pgfsys@useobject{currentmarker}{}%
\end{pgfscope}%
\begin{pgfscope}%
\pgfsys@transformshift{0.467818in}{1.188755in}%
\pgfsys@useobject{currentmarker}{}%
\end{pgfscope}%
\begin{pgfscope}%
\pgfsys@transformshift{1.797702in}{1.678275in}%
\pgfsys@useobject{currentmarker}{}%
\end{pgfscope}%
\begin{pgfscope}%
\pgfsys@transformshift{1.800535in}{1.507720in}%
\pgfsys@useobject{currentmarker}{}%
\end{pgfscope}%
\begin{pgfscope}%
\pgfsys@transformshift{0.486053in}{1.136188in}%
\pgfsys@useobject{currentmarker}{}%
\end{pgfscope}%
\begin{pgfscope}%
\pgfsys@transformshift{0.604133in}{1.030561in}%
\pgfsys@useobject{currentmarker}{}%
\end{pgfscope}%
\begin{pgfscope}%
\pgfsys@transformshift{1.510193in}{1.362772in}%
\pgfsys@useobject{currentmarker}{}%
\end{pgfscope}%
\begin{pgfscope}%
\pgfsys@transformshift{0.969856in}{1.380789in}%
\pgfsys@useobject{currentmarker}{}%
\end{pgfscope}%
\begin{pgfscope}%
\pgfsys@transformshift{0.883908in}{1.591399in}%
\pgfsys@useobject{currentmarker}{}%
\end{pgfscope}%
\begin{pgfscope}%
\pgfsys@transformshift{0.538449in}{1.041077in}%
\pgfsys@useobject{currentmarker}{}%
\end{pgfscope}%
\begin{pgfscope}%
\pgfsys@transformshift{0.561398in}{1.175597in}%
\pgfsys@useobject{currentmarker}{}%
\end{pgfscope}%
\begin{pgfscope}%
\pgfsys@transformshift{0.984012in}{1.437014in}%
\pgfsys@useobject{currentmarker}{}%
\end{pgfscope}%
\begin{pgfscope}%
\pgfsys@transformshift{0.614232in}{1.406096in}%
\pgfsys@useobject{currentmarker}{}%
\end{pgfscope}%
\begin{pgfscope}%
\pgfsys@transformshift{1.178616in}{1.449350in}%
\pgfsys@useobject{currentmarker}{}%
\end{pgfscope}%
\begin{pgfscope}%
\pgfsys@transformshift{0.638320in}{1.085971in}%
\pgfsys@useobject{currentmarker}{}%
\end{pgfscope}%
\begin{pgfscope}%
\pgfsys@transformshift{1.571164in}{1.339966in}%
\pgfsys@useobject{currentmarker}{}%
\end{pgfscope}%
\begin{pgfscope}%
\pgfsys@transformshift{0.449499in}{1.448212in}%
\pgfsys@useobject{currentmarker}{}%
\end{pgfscope}%
\begin{pgfscope}%
\pgfsys@transformshift{1.000812in}{1.673053in}%
\pgfsys@useobject{currentmarker}{}%
\end{pgfscope}%
\begin{pgfscope}%
\pgfsys@transformshift{0.504036in}{1.221009in}%
\pgfsys@useobject{currentmarker}{}%
\end{pgfscope}%
\begin{pgfscope}%
\pgfsys@transformshift{1.192646in}{1.481908in}%
\pgfsys@useobject{currentmarker}{}%
\end{pgfscope}%
\begin{pgfscope}%
\pgfsys@transformshift{0.575539in}{1.010543in}%
\pgfsys@useobject{currentmarker}{}%
\end{pgfscope}%
\begin{pgfscope}%
\pgfsys@transformshift{0.517708in}{1.328855in}%
\pgfsys@useobject{currentmarker}{}%
\end{pgfscope}%
\begin{pgfscope}%
\pgfsys@transformshift{1.244210in}{1.079428in}%
\pgfsys@useobject{currentmarker}{}%
\end{pgfscope}%
\begin{pgfscope}%
\pgfsys@transformshift{0.615567in}{0.960872in}%
\pgfsys@useobject{currentmarker}{}%
\end{pgfscope}%
\begin{pgfscope}%
\pgfsys@transformshift{0.907672in}{1.421787in}%
\pgfsys@useobject{currentmarker}{}%
\end{pgfscope}%
\begin{pgfscope}%
\pgfsys@transformshift{2.000000in}{1.918446in}%
\pgfsys@useobject{currentmarker}{}%
\end{pgfscope}%
\begin{pgfscope}%
\pgfsys@transformshift{0.485263in}{1.274430in}%
\pgfsys@useobject{currentmarker}{}%
\end{pgfscope}%
\begin{pgfscope}%
\pgfsys@transformshift{0.433137in}{1.100580in}%
\pgfsys@useobject{currentmarker}{}%
\end{pgfscope}%
\begin{pgfscope}%
\pgfsys@transformshift{0.499449in}{1.297981in}%
\pgfsys@useobject{currentmarker}{}%
\end{pgfscope}%
\begin{pgfscope}%
\pgfsys@transformshift{0.868886in}{1.185245in}%
\pgfsys@useobject{currentmarker}{}%
\end{pgfscope}%
\begin{pgfscope}%
\pgfsys@transformshift{1.451865in}{1.471195in}%
\pgfsys@useobject{currentmarker}{}%
\end{pgfscope}%
\begin{pgfscope}%
\pgfsys@transformshift{0.534197in}{1.353199in}%
\pgfsys@useobject{currentmarker}{}%
\end{pgfscope}%
\begin{pgfscope}%
\pgfsys@transformshift{0.972750in}{1.380567in}%
\pgfsys@useobject{currentmarker}{}%
\end{pgfscope}%
\begin{pgfscope}%
\pgfsys@transformshift{0.996367in}{1.147894in}%
\pgfsys@useobject{currentmarker}{}%
\end{pgfscope}%
\begin{pgfscope}%
\pgfsys@transformshift{0.893513in}{1.132618in}%
\pgfsys@useobject{currentmarker}{}%
\end{pgfscope}%
\begin{pgfscope}%
\pgfsys@transformshift{0.481561in}{1.284117in}%
\pgfsys@useobject{currentmarker}{}%
\end{pgfscope}%
\begin{pgfscope}%
\pgfsys@transformshift{0.462083in}{1.314088in}%
\pgfsys@useobject{currentmarker}{}%
\end{pgfscope}%
\begin{pgfscope}%
\pgfsys@transformshift{0.947888in}{1.106811in}%
\pgfsys@useobject{currentmarker}{}%
\end{pgfscope}%
\begin{pgfscope}%
\pgfsys@transformshift{0.475131in}{1.207995in}%
\pgfsys@useobject{currentmarker}{}%
\end{pgfscope}%
\begin{pgfscope}%
\pgfsys@transformshift{0.944100in}{1.146536in}%
\pgfsys@useobject{currentmarker}{}%
\end{pgfscope}%
\begin{pgfscope}%
\pgfsys@transformshift{0.966481in}{1.303906in}%
\pgfsys@useobject{currentmarker}{}%
\end{pgfscope}%
\begin{pgfscope}%
\pgfsys@transformshift{1.021199in}{1.340841in}%
\pgfsys@useobject{currentmarker}{}%
\end{pgfscope}%
\begin{pgfscope}%
\pgfsys@transformshift{0.451598in}{1.311422in}%
\pgfsys@useobject{currentmarker}{}%
\end{pgfscope}%
\begin{pgfscope}%
\pgfsys@transformshift{0.461004in}{1.356760in}%
\pgfsys@useobject{currentmarker}{}%
\end{pgfscope}%
\begin{pgfscope}%
\pgfsys@transformshift{0.419847in}{1.315680in}%
\pgfsys@useobject{currentmarker}{}%
\end{pgfscope}%
\begin{pgfscope}%
\pgfsys@transformshift{0.498798in}{1.103847in}%
\pgfsys@useobject{currentmarker}{}%
\end{pgfscope}%
\begin{pgfscope}%
\pgfsys@transformshift{1.022778in}{1.221958in}%
\pgfsys@useobject{currentmarker}{}%
\end{pgfscope}%
\begin{pgfscope}%
\pgfsys@transformshift{1.618050in}{1.038964in}%
\pgfsys@useobject{currentmarker}{}%
\end{pgfscope}%
\begin{pgfscope}%
\pgfsys@transformshift{0.873499in}{1.506327in}%
\pgfsys@useobject{currentmarker}{}%
\end{pgfscope}%
\begin{pgfscope}%
\pgfsys@transformshift{1.717975in}{1.354639in}%
\pgfsys@useobject{currentmarker}{}%
\end{pgfscope}%
\begin{pgfscope}%
\pgfsys@transformshift{0.479704in}{1.541297in}%
\pgfsys@useobject{currentmarker}{}%
\end{pgfscope}%
\begin{pgfscope}%
\pgfsys@transformshift{0.921403in}{1.548783in}%
\pgfsys@useobject{currentmarker}{}%
\end{pgfscope}%
\begin{pgfscope}%
\pgfsys@transformshift{0.507519in}{1.228927in}%
\pgfsys@useobject{currentmarker}{}%
\end{pgfscope}%
\begin{pgfscope}%
\pgfsys@transformshift{1.242289in}{1.580904in}%
\pgfsys@useobject{currentmarker}{}%
\end{pgfscope}%
\begin{pgfscope}%
\pgfsys@transformshift{0.901785in}{1.078349in}%
\pgfsys@useobject{currentmarker}{}%
\end{pgfscope}%
\begin{pgfscope}%
\pgfsys@transformshift{1.303585in}{1.396466in}%
\pgfsys@useobject{currentmarker}{}%
\end{pgfscope}%
\begin{pgfscope}%
\pgfsys@transformshift{0.385710in}{1.262607in}%
\pgfsys@useobject{currentmarker}{}%
\end{pgfscope}%
\begin{pgfscope}%
\pgfsys@transformshift{0.487652in}{1.245040in}%
\pgfsys@useobject{currentmarker}{}%
\end{pgfscope}%
\begin{pgfscope}%
\pgfsys@transformshift{0.427840in}{1.095760in}%
\pgfsys@useobject{currentmarker}{}%
\end{pgfscope}%
\begin{pgfscope}%
\pgfsys@transformshift{0.479919in}{1.338604in}%
\pgfsys@useobject{currentmarker}{}%
\end{pgfscope}%
\begin{pgfscope}%
\pgfsys@transformshift{0.421076in}{0.842494in}%
\pgfsys@useobject{currentmarker}{}%
\end{pgfscope}%
\begin{pgfscope}%
\pgfsys@transformshift{0.930792in}{1.118722in}%
\pgfsys@useobject{currentmarker}{}%
\end{pgfscope}%
\begin{pgfscope}%
\pgfsys@transformshift{0.890478in}{1.628281in}%
\pgfsys@useobject{currentmarker}{}%
\end{pgfscope}%
\begin{pgfscope}%
\pgfsys@transformshift{0.528345in}{1.149965in}%
\pgfsys@useobject{currentmarker}{}%
\end{pgfscope}%
\begin{pgfscope}%
\pgfsys@transformshift{1.013284in}{1.282084in}%
\pgfsys@useobject{currentmarker}{}%
\end{pgfscope}%
\begin{pgfscope}%
\pgfsys@transformshift{0.586631in}{1.180007in}%
\pgfsys@useobject{currentmarker}{}%
\end{pgfscope}%
\begin{pgfscope}%
\pgfsys@transformshift{1.117579in}{1.301271in}%
\pgfsys@useobject{currentmarker}{}%
\end{pgfscope}%
\begin{pgfscope}%
\pgfsys@transformshift{0.440363in}{1.192028in}%
\pgfsys@useobject{currentmarker}{}%
\end{pgfscope}%
\begin{pgfscope}%
\pgfsys@transformshift{0.482005in}{1.442385in}%
\pgfsys@useobject{currentmarker}{}%
\end{pgfscope}%
\begin{pgfscope}%
\pgfsys@transformshift{0.439583in}{1.051128in}%
\pgfsys@useobject{currentmarker}{}%
\end{pgfscope}%
\begin{pgfscope}%
\pgfsys@transformshift{0.857963in}{1.317906in}%
\pgfsys@useobject{currentmarker}{}%
\end{pgfscope}%
\begin{pgfscope}%
\pgfsys@transformshift{1.001461in}{1.180427in}%
\pgfsys@useobject{currentmarker}{}%
\end{pgfscope}%
\begin{pgfscope}%
\pgfsys@transformshift{0.484673in}{1.351604in}%
\pgfsys@useobject{currentmarker}{}%
\end{pgfscope}%
\begin{pgfscope}%
\pgfsys@transformshift{0.458444in}{1.213087in}%
\pgfsys@useobject{currentmarker}{}%
\end{pgfscope}%
\begin{pgfscope}%
\pgfsys@transformshift{0.477639in}{1.614736in}%
\pgfsys@useobject{currentmarker}{}%
\end{pgfscope}%
\begin{pgfscope}%
\pgfsys@transformshift{0.908318in}{1.136241in}%
\pgfsys@useobject{currentmarker}{}%
\end{pgfscope}%
\begin{pgfscope}%
\pgfsys@transformshift{0.463108in}{1.178807in}%
\pgfsys@useobject{currentmarker}{}%
\end{pgfscope}%
\begin{pgfscope}%
\pgfsys@transformshift{1.555994in}{1.461762in}%
\pgfsys@useobject{currentmarker}{}%
\end{pgfscope}%
\begin{pgfscope}%
\pgfsys@transformshift{0.456229in}{1.021112in}%
\pgfsys@useobject{currentmarker}{}%
\end{pgfscope}%
\begin{pgfscope}%
\pgfsys@transformshift{0.909471in}{1.616146in}%
\pgfsys@useobject{currentmarker}{}%
\end{pgfscope}%
\end{pgfscope}%
\begin{pgfscope}%
\pgfpathrectangle{\pgfqpoint{0.341129in}{0.466613in}}{\pgfqpoint{1.658871in}{1.711598in}}%
\pgfusepath{clip}%
\pgfsetbuttcap%
\pgfsetroundjoin%
\definecolor{currentfill}{rgb}{0.768627,0.305882,0.321569}%
\pgfsetfillcolor{currentfill}%
\pgfsetfillopacity{0.150000}%
\pgfsetlinewidth{1.003750pt}%
\definecolor{currentstroke}{rgb}{1.000000,1.000000,1.000000}%
\pgfsetstrokecolor{currentstroke}%
\pgfsetstrokeopacity{0.150000}%
\pgfsetdash{}{0pt}%
\pgfsys@defobject{currentmarker}{\pgfqpoint{0.341129in}{1.177307in}}{\pgfqpoint{2.000000in}{1.496758in}}{%
\pgfpathmoveto{\pgfqpoint{0.341129in}{1.223222in}}%
\pgfpathlineto{\pgfqpoint{0.341129in}{1.177307in}}%
\pgfpathlineto{\pgfqpoint{0.357885in}{1.180470in}}%
\pgfpathlineto{\pgfqpoint{0.374641in}{1.183506in}}%
\pgfpathlineto{\pgfqpoint{0.391398in}{1.186263in}}%
\pgfpathlineto{\pgfqpoint{0.408154in}{1.189468in}}%
\pgfpathlineto{\pgfqpoint{0.424910in}{1.192815in}}%
\pgfpathlineto{\pgfqpoint{0.441666in}{1.196084in}}%
\pgfpathlineto{\pgfqpoint{0.458423in}{1.198898in}}%
\pgfpathlineto{\pgfqpoint{0.475179in}{1.201764in}}%
\pgfpathlineto{\pgfqpoint{0.491935in}{1.204922in}}%
\pgfpathlineto{\pgfqpoint{0.508691in}{1.207844in}}%
\pgfpathlineto{\pgfqpoint{0.525448in}{1.210671in}}%
\pgfpathlineto{\pgfqpoint{0.542204in}{1.213538in}}%
\pgfpathlineto{\pgfqpoint{0.558960in}{1.216047in}}%
\pgfpathlineto{\pgfqpoint{0.575717in}{1.218786in}}%
\pgfpathlineto{\pgfqpoint{0.592473in}{1.221670in}}%
\pgfpathlineto{\pgfqpoint{0.609229in}{1.224809in}}%
\pgfpathlineto{\pgfqpoint{0.625985in}{1.227753in}}%
\pgfpathlineto{\pgfqpoint{0.642742in}{1.230731in}}%
\pgfpathlineto{\pgfqpoint{0.659498in}{1.233600in}}%
\pgfpathlineto{\pgfqpoint{0.676254in}{1.236442in}}%
\pgfpathlineto{\pgfqpoint{0.693011in}{1.238969in}}%
\pgfpathlineto{\pgfqpoint{0.709767in}{1.241757in}}%
\pgfpathlineto{\pgfqpoint{0.726523in}{1.244296in}}%
\pgfpathlineto{\pgfqpoint{0.743279in}{1.246793in}}%
\pgfpathlineto{\pgfqpoint{0.760036in}{1.249203in}}%
\pgfpathlineto{\pgfqpoint{0.776792in}{1.251903in}}%
\pgfpathlineto{\pgfqpoint{0.793548in}{1.254362in}}%
\pgfpathlineto{\pgfqpoint{0.810304in}{1.256845in}}%
\pgfpathlineto{\pgfqpoint{0.827061in}{1.259273in}}%
\pgfpathlineto{\pgfqpoint{0.843817in}{1.261664in}}%
\pgfpathlineto{\pgfqpoint{0.860573in}{1.264073in}}%
\pgfpathlineto{\pgfqpoint{0.877330in}{1.266401in}}%
\pgfpathlineto{\pgfqpoint{0.894086in}{1.268743in}}%
\pgfpathlineto{\pgfqpoint{0.910842in}{1.270933in}}%
\pgfpathlineto{\pgfqpoint{0.927598in}{1.273117in}}%
\pgfpathlineto{\pgfqpoint{0.944355in}{1.275414in}}%
\pgfpathlineto{\pgfqpoint{0.961111in}{1.277705in}}%
\pgfpathlineto{\pgfqpoint{0.977867in}{1.279727in}}%
\pgfpathlineto{\pgfqpoint{0.994623in}{1.281906in}}%
\pgfpathlineto{\pgfqpoint{1.011380in}{1.283938in}}%
\pgfpathlineto{\pgfqpoint{1.028136in}{1.286070in}}%
\pgfpathlineto{\pgfqpoint{1.044892in}{1.288315in}}%
\pgfpathlineto{\pgfqpoint{1.061649in}{1.290490in}}%
\pgfpathlineto{\pgfqpoint{1.078405in}{1.292766in}}%
\pgfpathlineto{\pgfqpoint{1.095161in}{1.294864in}}%
\pgfpathlineto{\pgfqpoint{1.111917in}{1.296964in}}%
\pgfpathlineto{\pgfqpoint{1.128674in}{1.298803in}}%
\pgfpathlineto{\pgfqpoint{1.145430in}{1.300587in}}%
\pgfpathlineto{\pgfqpoint{1.162186in}{1.302431in}}%
\pgfpathlineto{\pgfqpoint{1.178942in}{1.304317in}}%
\pgfpathlineto{\pgfqpoint{1.195699in}{1.306391in}}%
\pgfpathlineto{\pgfqpoint{1.212455in}{1.308440in}}%
\pgfpathlineto{\pgfqpoint{1.229211in}{1.310333in}}%
\pgfpathlineto{\pgfqpoint{1.245968in}{1.312296in}}%
\pgfpathlineto{\pgfqpoint{1.262724in}{1.314107in}}%
\pgfpathlineto{\pgfqpoint{1.279480in}{1.315991in}}%
\pgfpathlineto{\pgfqpoint{1.296236in}{1.317875in}}%
\pgfpathlineto{\pgfqpoint{1.312993in}{1.319765in}}%
\pgfpathlineto{\pgfqpoint{1.329749in}{1.321653in}}%
\pgfpathlineto{\pgfqpoint{1.346505in}{1.323536in}}%
\pgfpathlineto{\pgfqpoint{1.363262in}{1.325424in}}%
\pgfpathlineto{\pgfqpoint{1.380018in}{1.327343in}}%
\pgfpathlineto{\pgfqpoint{1.396774in}{1.329304in}}%
\pgfpathlineto{\pgfqpoint{1.413530in}{1.331426in}}%
\pgfpathlineto{\pgfqpoint{1.430287in}{1.333460in}}%
\pgfpathlineto{\pgfqpoint{1.447043in}{1.335317in}}%
\pgfpathlineto{\pgfqpoint{1.463799in}{1.337077in}}%
\pgfpathlineto{\pgfqpoint{1.480555in}{1.339039in}}%
\pgfpathlineto{\pgfqpoint{1.497312in}{1.340897in}}%
\pgfpathlineto{\pgfqpoint{1.514068in}{1.342713in}}%
\pgfpathlineto{\pgfqpoint{1.530824in}{1.344531in}}%
\pgfpathlineto{\pgfqpoint{1.547581in}{1.346175in}}%
\pgfpathlineto{\pgfqpoint{1.564337in}{1.348054in}}%
\pgfpathlineto{\pgfqpoint{1.581093in}{1.349942in}}%
\pgfpathlineto{\pgfqpoint{1.597849in}{1.351799in}}%
\pgfpathlineto{\pgfqpoint{1.614606in}{1.353644in}}%
\pgfpathlineto{\pgfqpoint{1.631362in}{1.355488in}}%
\pgfpathlineto{\pgfqpoint{1.648118in}{1.357251in}}%
\pgfpathlineto{\pgfqpoint{1.664874in}{1.359064in}}%
\pgfpathlineto{\pgfqpoint{1.681631in}{1.360878in}}%
\pgfpathlineto{\pgfqpoint{1.698387in}{1.362691in}}%
\pgfpathlineto{\pgfqpoint{1.715143in}{1.364507in}}%
\pgfpathlineto{\pgfqpoint{1.731900in}{1.366326in}}%
\pgfpathlineto{\pgfqpoint{1.748656in}{1.368127in}}%
\pgfpathlineto{\pgfqpoint{1.765412in}{1.369891in}}%
\pgfpathlineto{\pgfqpoint{1.782168in}{1.371783in}}%
\pgfpathlineto{\pgfqpoint{1.798925in}{1.373686in}}%
\pgfpathlineto{\pgfqpoint{1.815681in}{1.375600in}}%
\pgfpathlineto{\pgfqpoint{1.832437in}{1.377515in}}%
\pgfpathlineto{\pgfqpoint{1.849193in}{1.379429in}}%
\pgfpathlineto{\pgfqpoint{1.865950in}{1.381102in}}%
\pgfpathlineto{\pgfqpoint{1.882706in}{1.382743in}}%
\pgfpathlineto{\pgfqpoint{1.899462in}{1.384666in}}%
\pgfpathlineto{\pgfqpoint{1.916219in}{1.386311in}}%
\pgfpathlineto{\pgfqpoint{1.932975in}{1.388120in}}%
\pgfpathlineto{\pgfqpoint{1.949731in}{1.389930in}}%
\pgfpathlineto{\pgfqpoint{1.966487in}{1.391744in}}%
\pgfpathlineto{\pgfqpoint{1.983244in}{1.393559in}}%
\pgfpathlineto{\pgfqpoint{2.000000in}{1.395374in}}%
\pgfpathlineto{\pgfqpoint{2.000000in}{1.496758in}}%
\pgfpathlineto{\pgfqpoint{2.000000in}{1.496758in}}%
\pgfpathlineto{\pgfqpoint{1.983244in}{1.493741in}}%
\pgfpathlineto{\pgfqpoint{1.966487in}{1.490777in}}%
\pgfpathlineto{\pgfqpoint{1.949731in}{1.487812in}}%
\pgfpathlineto{\pgfqpoint{1.932975in}{1.484848in}}%
\pgfpathlineto{\pgfqpoint{1.916219in}{1.481884in}}%
\pgfpathlineto{\pgfqpoint{1.899462in}{1.478919in}}%
\pgfpathlineto{\pgfqpoint{1.882706in}{1.475955in}}%
\pgfpathlineto{\pgfqpoint{1.865950in}{1.472991in}}%
\pgfpathlineto{\pgfqpoint{1.849193in}{1.469954in}}%
\pgfpathlineto{\pgfqpoint{1.832437in}{1.466974in}}%
\pgfpathlineto{\pgfqpoint{1.815681in}{1.464009in}}%
\pgfpathlineto{\pgfqpoint{1.798925in}{1.460948in}}%
\pgfpathlineto{\pgfqpoint{1.782168in}{1.457887in}}%
\pgfpathlineto{\pgfqpoint{1.765412in}{1.454826in}}%
\pgfpathlineto{\pgfqpoint{1.748656in}{1.451760in}}%
\pgfpathlineto{\pgfqpoint{1.731900in}{1.448694in}}%
\pgfpathlineto{\pgfqpoint{1.715143in}{1.445593in}}%
\pgfpathlineto{\pgfqpoint{1.698387in}{1.442406in}}%
\pgfpathlineto{\pgfqpoint{1.681631in}{1.439216in}}%
\pgfpathlineto{\pgfqpoint{1.664874in}{1.436052in}}%
\pgfpathlineto{\pgfqpoint{1.648118in}{1.432972in}}%
\pgfpathlineto{\pgfqpoint{1.631362in}{1.429897in}}%
\pgfpathlineto{\pgfqpoint{1.614606in}{1.426817in}}%
\pgfpathlineto{\pgfqpoint{1.597849in}{1.423735in}}%
\pgfpathlineto{\pgfqpoint{1.581093in}{1.420644in}}%
\pgfpathlineto{\pgfqpoint{1.564337in}{1.417288in}}%
\pgfpathlineto{\pgfqpoint{1.547581in}{1.413855in}}%
\pgfpathlineto{\pgfqpoint{1.530824in}{1.410522in}}%
\pgfpathlineto{\pgfqpoint{1.514068in}{1.407331in}}%
\pgfpathlineto{\pgfqpoint{1.497312in}{1.404140in}}%
\pgfpathlineto{\pgfqpoint{1.480555in}{1.400949in}}%
\pgfpathlineto{\pgfqpoint{1.463799in}{1.397878in}}%
\pgfpathlineto{\pgfqpoint{1.447043in}{1.394868in}}%
\pgfpathlineto{\pgfqpoint{1.430287in}{1.391859in}}%
\pgfpathlineto{\pgfqpoint{1.413530in}{1.388591in}}%
\pgfpathlineto{\pgfqpoint{1.396774in}{1.385320in}}%
\pgfpathlineto{\pgfqpoint{1.380018in}{1.382043in}}%
\pgfpathlineto{\pgfqpoint{1.363262in}{1.378975in}}%
\pgfpathlineto{\pgfqpoint{1.346505in}{1.376138in}}%
\pgfpathlineto{\pgfqpoint{1.329749in}{1.373342in}}%
\pgfpathlineto{\pgfqpoint{1.312993in}{1.370004in}}%
\pgfpathlineto{\pgfqpoint{1.296236in}{1.366661in}}%
\pgfpathlineto{\pgfqpoint{1.279480in}{1.363318in}}%
\pgfpathlineto{\pgfqpoint{1.262724in}{1.360106in}}%
\pgfpathlineto{\pgfqpoint{1.245968in}{1.357001in}}%
\pgfpathlineto{\pgfqpoint{1.229211in}{1.354094in}}%
\pgfpathlineto{\pgfqpoint{1.212455in}{1.351187in}}%
\pgfpathlineto{\pgfqpoint{1.195699in}{1.348279in}}%
\pgfpathlineto{\pgfqpoint{1.178942in}{1.345365in}}%
\pgfpathlineto{\pgfqpoint{1.162186in}{1.342459in}}%
\pgfpathlineto{\pgfqpoint{1.145430in}{1.339607in}}%
\pgfpathlineto{\pgfqpoint{1.128674in}{1.336867in}}%
\pgfpathlineto{\pgfqpoint{1.111917in}{1.334142in}}%
\pgfpathlineto{\pgfqpoint{1.095161in}{1.331271in}}%
\pgfpathlineto{\pgfqpoint{1.078405in}{1.328356in}}%
\pgfpathlineto{\pgfqpoint{1.061649in}{1.325674in}}%
\pgfpathlineto{\pgfqpoint{1.044892in}{1.322801in}}%
\pgfpathlineto{\pgfqpoint{1.028136in}{1.319818in}}%
\pgfpathlineto{\pgfqpoint{1.011380in}{1.316973in}}%
\pgfpathlineto{\pgfqpoint{0.994623in}{1.314196in}}%
\pgfpathlineto{\pgfqpoint{0.977867in}{1.311329in}}%
\pgfpathlineto{\pgfqpoint{0.961111in}{1.308416in}}%
\pgfpathlineto{\pgfqpoint{0.944355in}{1.305571in}}%
\pgfpathlineto{\pgfqpoint{0.927598in}{1.302718in}}%
\pgfpathlineto{\pgfqpoint{0.910842in}{1.299905in}}%
\pgfpathlineto{\pgfqpoint{0.894086in}{1.297331in}}%
\pgfpathlineto{\pgfqpoint{0.877330in}{1.294649in}}%
\pgfpathlineto{\pgfqpoint{0.860573in}{1.291920in}}%
\pgfpathlineto{\pgfqpoint{0.843817in}{1.289608in}}%
\pgfpathlineto{\pgfqpoint{0.827061in}{1.287101in}}%
\pgfpathlineto{\pgfqpoint{0.810304in}{1.284645in}}%
\pgfpathlineto{\pgfqpoint{0.793548in}{1.282293in}}%
\pgfpathlineto{\pgfqpoint{0.776792in}{1.279986in}}%
\pgfpathlineto{\pgfqpoint{0.760036in}{1.277498in}}%
\pgfpathlineto{\pgfqpoint{0.743279in}{1.274852in}}%
\pgfpathlineto{\pgfqpoint{0.726523in}{1.272572in}}%
\pgfpathlineto{\pgfqpoint{0.709767in}{1.270374in}}%
\pgfpathlineto{\pgfqpoint{0.693011in}{1.268080in}}%
\pgfpathlineto{\pgfqpoint{0.676254in}{1.265606in}}%
\pgfpathlineto{\pgfqpoint{0.659498in}{1.263387in}}%
\pgfpathlineto{\pgfqpoint{0.642742in}{1.261230in}}%
\pgfpathlineto{\pgfqpoint{0.625985in}{1.259125in}}%
\pgfpathlineto{\pgfqpoint{0.609229in}{1.257072in}}%
\pgfpathlineto{\pgfqpoint{0.592473in}{1.254821in}}%
\pgfpathlineto{\pgfqpoint{0.575717in}{1.252577in}}%
\pgfpathlineto{\pgfqpoint{0.558960in}{1.250404in}}%
\pgfpathlineto{\pgfqpoint{0.542204in}{1.248098in}}%
\pgfpathlineto{\pgfqpoint{0.525448in}{1.245914in}}%
\pgfpathlineto{\pgfqpoint{0.508691in}{1.243738in}}%
\pgfpathlineto{\pgfqpoint{0.491935in}{1.241557in}}%
\pgfpathlineto{\pgfqpoint{0.475179in}{1.239373in}}%
\pgfpathlineto{\pgfqpoint{0.458423in}{1.237255in}}%
\pgfpathlineto{\pgfqpoint{0.441666in}{1.235090in}}%
\pgfpathlineto{\pgfqpoint{0.424910in}{1.232963in}}%
\pgfpathlineto{\pgfqpoint{0.408154in}{1.231027in}}%
\pgfpathlineto{\pgfqpoint{0.391398in}{1.229109in}}%
\pgfpathlineto{\pgfqpoint{0.374641in}{1.227127in}}%
\pgfpathlineto{\pgfqpoint{0.357885in}{1.225104in}}%
\pgfpathlineto{\pgfqpoint{0.341129in}{1.223222in}}%
\pgfpathclose%
\pgfusepath{stroke,fill}%
}%
\begin{pgfscope}%
\pgfsys@transformshift{0.000000in}{0.000000in}%
\pgfsys@useobject{currentmarker}{}%
\end{pgfscope}%
\end{pgfscope}%
\begin{pgfscope}%
\pgfpathrectangle{\pgfqpoint{0.341129in}{0.466613in}}{\pgfqpoint{1.658871in}{1.711598in}}%
\pgfusepath{clip}%
\pgfsetbuttcap%
\pgfsetroundjoin%
\definecolor{currentfill}{rgb}{0.505882,0.447059,0.701961}%
\pgfsetfillcolor{currentfill}%
\pgfsetfillopacity{0.250000}%
\pgfsetlinewidth{1.003750pt}%
\definecolor{currentstroke}{rgb}{0.505882,0.447059,0.701961}%
\pgfsetstrokecolor{currentstroke}%
\pgfsetstrokeopacity{0.250000}%
\pgfsetdash{}{0pt}%
\pgfsys@defobject{currentmarker}{\pgfqpoint{-0.017010in}{-0.017010in}}{\pgfqpoint{0.017010in}{0.017010in}}{%
\pgfpathmoveto{\pgfqpoint{0.000000in}{-0.017010in}}%
\pgfpathcurveto{\pgfqpoint{0.004511in}{-0.017010in}}{\pgfqpoint{0.008838in}{-0.015218in}}{\pgfqpoint{0.012028in}{-0.012028in}}%
\pgfpathcurveto{\pgfqpoint{0.015218in}{-0.008838in}}{\pgfqpoint{0.017010in}{-0.004511in}}{\pgfqpoint{0.017010in}{0.000000in}}%
\pgfpathcurveto{\pgfqpoint{0.017010in}{0.004511in}}{\pgfqpoint{0.015218in}{0.008838in}}{\pgfqpoint{0.012028in}{0.012028in}}%
\pgfpathcurveto{\pgfqpoint{0.008838in}{0.015218in}}{\pgfqpoint{0.004511in}{0.017010in}}{\pgfqpoint{0.000000in}{0.017010in}}%
\pgfpathcurveto{\pgfqpoint{-0.004511in}{0.017010in}}{\pgfqpoint{-0.008838in}{0.015218in}}{\pgfqpoint{-0.012028in}{0.012028in}}%
\pgfpathcurveto{\pgfqpoint{-0.015218in}{0.008838in}}{\pgfqpoint{-0.017010in}{0.004511in}}{\pgfqpoint{-0.017010in}{0.000000in}}%
\pgfpathcurveto{\pgfqpoint{-0.017010in}{-0.004511in}}{\pgfqpoint{-0.015218in}{-0.008838in}}{\pgfqpoint{-0.012028in}{-0.012028in}}%
\pgfpathcurveto{\pgfqpoint{-0.008838in}{-0.015218in}}{\pgfqpoint{-0.004511in}{-0.017010in}}{\pgfqpoint{0.000000in}{-0.017010in}}%
\pgfpathclose%
\pgfusepath{stroke,fill}%
}%
\begin{pgfscope}%
\pgfsys@transformshift{1.684044in}{1.400393in}%
\pgfsys@useobject{currentmarker}{}%
\end{pgfscope}%
\begin{pgfscope}%
\pgfsys@transformshift{1.113006in}{1.183090in}%
\pgfsys@useobject{currentmarker}{}%
\end{pgfscope}%
\begin{pgfscope}%
\pgfsys@transformshift{0.938483in}{0.849586in}%
\pgfsys@useobject{currentmarker}{}%
\end{pgfscope}%
\begin{pgfscope}%
\pgfsys@transformshift{0.492780in}{0.764066in}%
\pgfsys@useobject{currentmarker}{}%
\end{pgfscope}%
\begin{pgfscope}%
\pgfsys@transformshift{0.529797in}{0.734372in}%
\pgfsys@useobject{currentmarker}{}%
\end{pgfscope}%
\begin{pgfscope}%
\pgfsys@transformshift{1.838155in}{1.348406in}%
\pgfsys@useobject{currentmarker}{}%
\end{pgfscope}%
\begin{pgfscope}%
\pgfsys@transformshift{0.477840in}{0.804347in}%
\pgfsys@useobject{currentmarker}{}%
\end{pgfscope}%
\begin{pgfscope}%
\pgfsys@transformshift{0.884452in}{1.300277in}%
\pgfsys@useobject{currentmarker}{}%
\end{pgfscope}%
\begin{pgfscope}%
\pgfsys@transformshift{1.509149in}{1.379272in}%
\pgfsys@useobject{currentmarker}{}%
\end{pgfscope}%
\begin{pgfscope}%
\pgfsys@transformshift{1.183184in}{0.983944in}%
\pgfsys@useobject{currentmarker}{}%
\end{pgfscope}%
\begin{pgfscope}%
\pgfsys@transformshift{1.140467in}{0.848880in}%
\pgfsys@useobject{currentmarker}{}%
\end{pgfscope}%
\begin{pgfscope}%
\pgfsys@transformshift{0.501776in}{0.805848in}%
\pgfsys@useobject{currentmarker}{}%
\end{pgfscope}%
\begin{pgfscope}%
\pgfsys@transformshift{0.461389in}{0.692826in}%
\pgfsys@useobject{currentmarker}{}%
\end{pgfscope}%
\begin{pgfscope}%
\pgfsys@transformshift{0.986905in}{1.053761in}%
\pgfsys@useobject{currentmarker}{}%
\end{pgfscope}%
\begin{pgfscope}%
\pgfsys@transformshift{0.969502in}{1.160445in}%
\pgfsys@useobject{currentmarker}{}%
\end{pgfscope}%
\begin{pgfscope}%
\pgfsys@transformshift{1.016218in}{0.864508in}%
\pgfsys@useobject{currentmarker}{}%
\end{pgfscope}%
\begin{pgfscope}%
\pgfsys@transformshift{0.413211in}{0.973333in}%
\pgfsys@useobject{currentmarker}{}%
\end{pgfscope}%
\begin{pgfscope}%
\pgfsys@transformshift{1.870040in}{1.313571in}%
\pgfsys@useobject{currentmarker}{}%
\end{pgfscope}%
\begin{pgfscope}%
\pgfsys@transformshift{1.031673in}{1.084010in}%
\pgfsys@useobject{currentmarker}{}%
\end{pgfscope}%
\begin{pgfscope}%
\pgfsys@transformshift{1.596314in}{1.356227in}%
\pgfsys@useobject{currentmarker}{}%
\end{pgfscope}%
\begin{pgfscope}%
\pgfsys@transformshift{0.496691in}{0.775731in}%
\pgfsys@useobject{currentmarker}{}%
\end{pgfscope}%
\begin{pgfscope}%
\pgfsys@transformshift{0.896936in}{0.832111in}%
\pgfsys@useobject{currentmarker}{}%
\end{pgfscope}%
\begin{pgfscope}%
\pgfsys@transformshift{0.442231in}{0.820277in}%
\pgfsys@useobject{currentmarker}{}%
\end{pgfscope}%
\begin{pgfscope}%
\pgfsys@transformshift{1.948436in}{1.395740in}%
\pgfsys@useobject{currentmarker}{}%
\end{pgfscope}%
\begin{pgfscope}%
\pgfsys@transformshift{0.607246in}{0.974168in}%
\pgfsys@useobject{currentmarker}{}%
\end{pgfscope}%
\begin{pgfscope}%
\pgfsys@transformshift{0.591373in}{0.616944in}%
\pgfsys@useobject{currentmarker}{}%
\end{pgfscope}%
\begin{pgfscope}%
\pgfsys@transformshift{0.495569in}{0.767265in}%
\pgfsys@useobject{currentmarker}{}%
\end{pgfscope}%
\begin{pgfscope}%
\pgfsys@transformshift{0.805700in}{0.804976in}%
\pgfsys@useobject{currentmarker}{}%
\end{pgfscope}%
\begin{pgfscope}%
\pgfsys@transformshift{0.472915in}{0.842651in}%
\pgfsys@useobject{currentmarker}{}%
\end{pgfscope}%
\begin{pgfscope}%
\pgfsys@transformshift{1.008191in}{1.130423in}%
\pgfsys@useobject{currentmarker}{}%
\end{pgfscope}%
\begin{pgfscope}%
\pgfsys@transformshift{0.795900in}{1.161884in}%
\pgfsys@useobject{currentmarker}{}%
\end{pgfscope}%
\begin{pgfscope}%
\pgfsys@transformshift{0.879596in}{1.229862in}%
\pgfsys@useobject{currentmarker}{}%
\end{pgfscope}%
\begin{pgfscope}%
\pgfsys@transformshift{0.851399in}{0.815551in}%
\pgfsys@useobject{currentmarker}{}%
\end{pgfscope}%
\begin{pgfscope}%
\pgfsys@transformshift{1.630362in}{1.394211in}%
\pgfsys@useobject{currentmarker}{}%
\end{pgfscope}%
\begin{pgfscope}%
\pgfsys@transformshift{0.497926in}{0.789889in}%
\pgfsys@useobject{currentmarker}{}%
\end{pgfscope}%
\begin{pgfscope}%
\pgfsys@transformshift{1.716622in}{0.877609in}%
\pgfsys@useobject{currentmarker}{}%
\end{pgfscope}%
\begin{pgfscope}%
\pgfsys@transformshift{0.524809in}{1.018124in}%
\pgfsys@useobject{currentmarker}{}%
\end{pgfscope}%
\begin{pgfscope}%
\pgfsys@transformshift{1.821229in}{1.347705in}%
\pgfsys@useobject{currentmarker}{}%
\end{pgfscope}%
\begin{pgfscope}%
\pgfsys@transformshift{0.925546in}{1.167178in}%
\pgfsys@useobject{currentmarker}{}%
\end{pgfscope}%
\begin{pgfscope}%
\pgfsys@transformshift{1.841684in}{0.880890in}%
\pgfsys@useobject{currentmarker}{}%
\end{pgfscope}%
\begin{pgfscope}%
\pgfsys@transformshift{1.522493in}{1.292137in}%
\pgfsys@useobject{currentmarker}{}%
\end{pgfscope}%
\begin{pgfscope}%
\pgfsys@transformshift{0.409285in}{0.981721in}%
\pgfsys@useobject{currentmarker}{}%
\end{pgfscope}%
\begin{pgfscope}%
\pgfsys@transformshift{0.896047in}{0.879350in}%
\pgfsys@useobject{currentmarker}{}%
\end{pgfscope}%
\begin{pgfscope}%
\pgfsys@transformshift{0.382462in}{0.940329in}%
\pgfsys@useobject{currentmarker}{}%
\end{pgfscope}%
\begin{pgfscope}%
\pgfsys@transformshift{0.855764in}{0.805618in}%
\pgfsys@useobject{currentmarker}{}%
\end{pgfscope}%
\begin{pgfscope}%
\pgfsys@transformshift{0.523733in}{0.729130in}%
\pgfsys@useobject{currentmarker}{}%
\end{pgfscope}%
\begin{pgfscope}%
\pgfsys@transformshift{0.470197in}{1.150423in}%
\pgfsys@useobject{currentmarker}{}%
\end{pgfscope}%
\begin{pgfscope}%
\pgfsys@transformshift{0.829991in}{1.117976in}%
\pgfsys@useobject{currentmarker}{}%
\end{pgfscope}%
\begin{pgfscope}%
\pgfsys@transformshift{0.933266in}{0.778003in}%
\pgfsys@useobject{currentmarker}{}%
\end{pgfscope}%
\begin{pgfscope}%
\pgfsys@transformshift{0.956789in}{1.378383in}%
\pgfsys@useobject{currentmarker}{}%
\end{pgfscope}%
\begin{pgfscope}%
\pgfsys@transformshift{0.936472in}{1.068958in}%
\pgfsys@useobject{currentmarker}{}%
\end{pgfscope}%
\begin{pgfscope}%
\pgfsys@transformshift{0.468794in}{1.076857in}%
\pgfsys@useobject{currentmarker}{}%
\end{pgfscope}%
\begin{pgfscope}%
\pgfsys@transformshift{0.402384in}{0.988621in}%
\pgfsys@useobject{currentmarker}{}%
\end{pgfscope}%
\begin{pgfscope}%
\pgfsys@transformshift{1.177456in}{1.142170in}%
\pgfsys@useobject{currentmarker}{}%
\end{pgfscope}%
\begin{pgfscope}%
\pgfsys@transformshift{0.416018in}{0.719340in}%
\pgfsys@useobject{currentmarker}{}%
\end{pgfscope}%
\begin{pgfscope}%
\pgfsys@transformshift{0.447008in}{1.208350in}%
\pgfsys@useobject{currentmarker}{}%
\end{pgfscope}%
\begin{pgfscope}%
\pgfsys@transformshift{0.549911in}{0.682895in}%
\pgfsys@useobject{currentmarker}{}%
\end{pgfscope}%
\begin{pgfscope}%
\pgfsys@transformshift{1.065128in}{1.040034in}%
\pgfsys@useobject{currentmarker}{}%
\end{pgfscope}%
\begin{pgfscope}%
\pgfsys@transformshift{0.450053in}{0.716436in}%
\pgfsys@useobject{currentmarker}{}%
\end{pgfscope}%
\begin{pgfscope}%
\pgfsys@transformshift{0.455142in}{0.766082in}%
\pgfsys@useobject{currentmarker}{}%
\end{pgfscope}%
\begin{pgfscope}%
\pgfsys@transformshift{0.484478in}{0.776744in}%
\pgfsys@useobject{currentmarker}{}%
\end{pgfscope}%
\begin{pgfscope}%
\pgfsys@transformshift{1.875283in}{1.635460in}%
\pgfsys@useobject{currentmarker}{}%
\end{pgfscope}%
\begin{pgfscope}%
\pgfsys@transformshift{0.375023in}{1.034957in}%
\pgfsys@useobject{currentmarker}{}%
\end{pgfscope}%
\begin{pgfscope}%
\pgfsys@transformshift{0.853143in}{0.793819in}%
\pgfsys@useobject{currentmarker}{}%
\end{pgfscope}%
\begin{pgfscope}%
\pgfsys@transformshift{0.436365in}{0.762400in}%
\pgfsys@useobject{currentmarker}{}%
\end{pgfscope}%
\begin{pgfscope}%
\pgfsys@transformshift{0.491797in}{0.662365in}%
\pgfsys@useobject{currentmarker}{}%
\end{pgfscope}%
\begin{pgfscope}%
\pgfsys@transformshift{0.902491in}{0.866469in}%
\pgfsys@useobject{currentmarker}{}%
\end{pgfscope}%
\begin{pgfscope}%
\pgfsys@transformshift{0.978414in}{1.109819in}%
\pgfsys@useobject{currentmarker}{}%
\end{pgfscope}%
\begin{pgfscope}%
\pgfsys@transformshift{0.944248in}{1.150272in}%
\pgfsys@useobject{currentmarker}{}%
\end{pgfscope}%
\begin{pgfscope}%
\pgfsys@transformshift{1.817503in}{0.869055in}%
\pgfsys@useobject{currentmarker}{}%
\end{pgfscope}%
\begin{pgfscope}%
\pgfsys@transformshift{1.721814in}{1.296186in}%
\pgfsys@useobject{currentmarker}{}%
\end{pgfscope}%
\begin{pgfscope}%
\pgfsys@transformshift{0.476494in}{1.221422in}%
\pgfsys@useobject{currentmarker}{}%
\end{pgfscope}%
\begin{pgfscope}%
\pgfsys@transformshift{0.942312in}{1.030925in}%
\pgfsys@useobject{currentmarker}{}%
\end{pgfscope}%
\begin{pgfscope}%
\pgfsys@transformshift{0.975555in}{1.179308in}%
\pgfsys@useobject{currentmarker}{}%
\end{pgfscope}%
\begin{pgfscope}%
\pgfsys@transformshift{1.642091in}{1.376178in}%
\pgfsys@useobject{currentmarker}{}%
\end{pgfscope}%
\begin{pgfscope}%
\pgfsys@transformshift{0.910804in}{1.203174in}%
\pgfsys@useobject{currentmarker}{}%
\end{pgfscope}%
\begin{pgfscope}%
\pgfsys@transformshift{0.507435in}{0.694186in}%
\pgfsys@useobject{currentmarker}{}%
\end{pgfscope}%
\begin{pgfscope}%
\pgfsys@transformshift{1.946413in}{0.910674in}%
\pgfsys@useobject{currentmarker}{}%
\end{pgfscope}%
\begin{pgfscope}%
\pgfsys@transformshift{1.639910in}{1.264912in}%
\pgfsys@useobject{currentmarker}{}%
\end{pgfscope}%
\begin{pgfscope}%
\pgfsys@transformshift{1.020077in}{0.980450in}%
\pgfsys@useobject{currentmarker}{}%
\end{pgfscope}%
\begin{pgfscope}%
\pgfsys@transformshift{1.837777in}{1.446065in}%
\pgfsys@useobject{currentmarker}{}%
\end{pgfscope}%
\begin{pgfscope}%
\pgfsys@transformshift{0.546434in}{0.665502in}%
\pgfsys@useobject{currentmarker}{}%
\end{pgfscope}%
\begin{pgfscope}%
\pgfsys@transformshift{0.507573in}{0.876420in}%
\pgfsys@useobject{currentmarker}{}%
\end{pgfscope}%
\begin{pgfscope}%
\pgfsys@transformshift{0.995047in}{0.884265in}%
\pgfsys@useobject{currentmarker}{}%
\end{pgfscope}%
\begin{pgfscope}%
\pgfsys@transformshift{0.507253in}{0.771777in}%
\pgfsys@useobject{currentmarker}{}%
\end{pgfscope}%
\begin{pgfscope}%
\pgfsys@transformshift{0.844423in}{0.914610in}%
\pgfsys@useobject{currentmarker}{}%
\end{pgfscope}%
\begin{pgfscope}%
\pgfsys@transformshift{1.175457in}{1.555032in}%
\pgfsys@useobject{currentmarker}{}%
\end{pgfscope}%
\begin{pgfscope}%
\pgfsys@transformshift{1.024973in}{0.850378in}%
\pgfsys@useobject{currentmarker}{}%
\end{pgfscope}%
\begin{pgfscope}%
\pgfsys@transformshift{1.150522in}{0.916365in}%
\pgfsys@useobject{currentmarker}{}%
\end{pgfscope}%
\begin{pgfscope}%
\pgfsys@transformshift{0.562256in}{0.894342in}%
\pgfsys@useobject{currentmarker}{}%
\end{pgfscope}%
\begin{pgfscope}%
\pgfsys@transformshift{1.482648in}{1.165553in}%
\pgfsys@useobject{currentmarker}{}%
\end{pgfscope}%
\begin{pgfscope}%
\pgfsys@transformshift{1.406514in}{1.261924in}%
\pgfsys@useobject{currentmarker}{}%
\end{pgfscope}%
\begin{pgfscope}%
\pgfsys@transformshift{1.080077in}{0.849420in}%
\pgfsys@useobject{currentmarker}{}%
\end{pgfscope}%
\begin{pgfscope}%
\pgfsys@transformshift{0.989979in}{1.357758in}%
\pgfsys@useobject{currentmarker}{}%
\end{pgfscope}%
\begin{pgfscope}%
\pgfsys@transformshift{0.987626in}{1.235414in}%
\pgfsys@useobject{currentmarker}{}%
\end{pgfscope}%
\begin{pgfscope}%
\pgfsys@transformshift{1.771689in}{1.566566in}%
\pgfsys@useobject{currentmarker}{}%
\end{pgfscope}%
\begin{pgfscope}%
\pgfsys@transformshift{0.891632in}{0.724422in}%
\pgfsys@useobject{currentmarker}{}%
\end{pgfscope}%
\begin{pgfscope}%
\pgfsys@transformshift{0.870836in}{1.171236in}%
\pgfsys@useobject{currentmarker}{}%
\end{pgfscope}%
\begin{pgfscope}%
\pgfsys@transformshift{1.667984in}{1.514052in}%
\pgfsys@useobject{currentmarker}{}%
\end{pgfscope}%
\begin{pgfscope}%
\pgfsys@transformshift{1.813990in}{1.734108in}%
\pgfsys@useobject{currentmarker}{}%
\end{pgfscope}%
\begin{pgfscope}%
\pgfsys@transformshift{0.403236in}{0.894715in}%
\pgfsys@useobject{currentmarker}{}%
\end{pgfscope}%
\begin{pgfscope}%
\pgfsys@transformshift{0.996537in}{0.986803in}%
\pgfsys@useobject{currentmarker}{}%
\end{pgfscope}%
\begin{pgfscope}%
\pgfsys@transformshift{1.043417in}{1.128774in}%
\pgfsys@useobject{currentmarker}{}%
\end{pgfscope}%
\begin{pgfscope}%
\pgfsys@transformshift{0.467968in}{0.758703in}%
\pgfsys@useobject{currentmarker}{}%
\end{pgfscope}%
\begin{pgfscope}%
\pgfsys@transformshift{1.915346in}{0.830339in}%
\pgfsys@useobject{currentmarker}{}%
\end{pgfscope}%
\begin{pgfscope}%
\pgfsys@transformshift{0.437600in}{1.026499in}%
\pgfsys@useobject{currentmarker}{}%
\end{pgfscope}%
\begin{pgfscope}%
\pgfsys@transformshift{0.438013in}{0.785136in}%
\pgfsys@useobject{currentmarker}{}%
\end{pgfscope}%
\begin{pgfscope}%
\pgfsys@transformshift{0.958228in}{0.811674in}%
\pgfsys@useobject{currentmarker}{}%
\end{pgfscope}%
\begin{pgfscope}%
\pgfsys@transformshift{1.012755in}{0.778883in}%
\pgfsys@useobject{currentmarker}{}%
\end{pgfscope}%
\begin{pgfscope}%
\pgfsys@transformshift{0.406225in}{0.750144in}%
\pgfsys@useobject{currentmarker}{}%
\end{pgfscope}%
\begin{pgfscope}%
\pgfsys@transformshift{0.473582in}{0.853793in}%
\pgfsys@useobject{currentmarker}{}%
\end{pgfscope}%
\begin{pgfscope}%
\pgfsys@transformshift{0.590756in}{0.682177in}%
\pgfsys@useobject{currentmarker}{}%
\end{pgfscope}%
\begin{pgfscope}%
\pgfsys@transformshift{0.918416in}{0.716102in}%
\pgfsys@useobject{currentmarker}{}%
\end{pgfscope}%
\begin{pgfscope}%
\pgfsys@transformshift{0.435231in}{0.815216in}%
\pgfsys@useobject{currentmarker}{}%
\end{pgfscope}%
\begin{pgfscope}%
\pgfsys@transformshift{1.100847in}{1.181676in}%
\pgfsys@useobject{currentmarker}{}%
\end{pgfscope}%
\begin{pgfscope}%
\pgfsys@transformshift{1.783869in}{1.485008in}%
\pgfsys@useobject{currentmarker}{}%
\end{pgfscope}%
\begin{pgfscope}%
\pgfsys@transformshift{0.881476in}{1.296155in}%
\pgfsys@useobject{currentmarker}{}%
\end{pgfscope}%
\begin{pgfscope}%
\pgfsys@transformshift{1.004126in}{0.775778in}%
\pgfsys@useobject{currentmarker}{}%
\end{pgfscope}%
\begin{pgfscope}%
\pgfsys@transformshift{1.494969in}{1.320093in}%
\pgfsys@useobject{currentmarker}{}%
\end{pgfscope}%
\begin{pgfscope}%
\pgfsys@transformshift{0.486970in}{0.787749in}%
\pgfsys@useobject{currentmarker}{}%
\end{pgfscope}%
\begin{pgfscope}%
\pgfsys@transformshift{0.439439in}{0.934857in}%
\pgfsys@useobject{currentmarker}{}%
\end{pgfscope}%
\begin{pgfscope}%
\pgfsys@transformshift{0.879557in}{1.202368in}%
\pgfsys@useobject{currentmarker}{}%
\end{pgfscope}%
\begin{pgfscope}%
\pgfsys@transformshift{0.852216in}{0.917183in}%
\pgfsys@useobject{currentmarker}{}%
\end{pgfscope}%
\begin{pgfscope}%
\pgfsys@transformshift{0.950782in}{1.241185in}%
\pgfsys@useobject{currentmarker}{}%
\end{pgfscope}%
\begin{pgfscope}%
\pgfsys@transformshift{0.502562in}{0.979835in}%
\pgfsys@useobject{currentmarker}{}%
\end{pgfscope}%
\begin{pgfscope}%
\pgfsys@transformshift{1.067095in}{1.012165in}%
\pgfsys@useobject{currentmarker}{}%
\end{pgfscope}%
\begin{pgfscope}%
\pgfsys@transformshift{0.478759in}{1.003688in}%
\pgfsys@useobject{currentmarker}{}%
\end{pgfscope}%
\begin{pgfscope}%
\pgfsys@transformshift{0.967133in}{1.131411in}%
\pgfsys@useobject{currentmarker}{}%
\end{pgfscope}%
\begin{pgfscope}%
\pgfsys@transformshift{0.508090in}{0.755243in}%
\pgfsys@useobject{currentmarker}{}%
\end{pgfscope}%
\begin{pgfscope}%
\pgfsys@transformshift{0.435594in}{0.876416in}%
\pgfsys@useobject{currentmarker}{}%
\end{pgfscope}%
\begin{pgfscope}%
\pgfsys@transformshift{0.980355in}{0.752436in}%
\pgfsys@useobject{currentmarker}{}%
\end{pgfscope}%
\begin{pgfscope}%
\pgfsys@transformshift{1.552149in}{1.198701in}%
\pgfsys@useobject{currentmarker}{}%
\end{pgfscope}%
\begin{pgfscope}%
\pgfsys@transformshift{0.875226in}{0.799533in}%
\pgfsys@useobject{currentmarker}{}%
\end{pgfscope}%
\begin{pgfscope}%
\pgfsys@transformshift{1.107273in}{1.105924in}%
\pgfsys@useobject{currentmarker}{}%
\end{pgfscope}%
\begin{pgfscope}%
\pgfsys@transformshift{0.523782in}{1.032298in}%
\pgfsys@useobject{currentmarker}{}%
\end{pgfscope}%
\begin{pgfscope}%
\pgfsys@transformshift{1.045306in}{1.274777in}%
\pgfsys@useobject{currentmarker}{}%
\end{pgfscope}%
\begin{pgfscope}%
\pgfsys@transformshift{1.526749in}{1.050533in}%
\pgfsys@useobject{currentmarker}{}%
\end{pgfscope}%
\begin{pgfscope}%
\pgfsys@transformshift{1.541666in}{1.368268in}%
\pgfsys@useobject{currentmarker}{}%
\end{pgfscope}%
\begin{pgfscope}%
\pgfsys@transformshift{1.574267in}{1.380848in}%
\pgfsys@useobject{currentmarker}{}%
\end{pgfscope}%
\begin{pgfscope}%
\pgfsys@transformshift{1.639614in}{1.368848in}%
\pgfsys@useobject{currentmarker}{}%
\end{pgfscope}%
\begin{pgfscope}%
\pgfsys@transformshift{0.881908in}{1.225333in}%
\pgfsys@useobject{currentmarker}{}%
\end{pgfscope}%
\begin{pgfscope}%
\pgfsys@transformshift{1.677526in}{1.568891in}%
\pgfsys@useobject{currentmarker}{}%
\end{pgfscope}%
\begin{pgfscope}%
\pgfsys@transformshift{0.483826in}{0.856308in}%
\pgfsys@useobject{currentmarker}{}%
\end{pgfscope}%
\begin{pgfscope}%
\pgfsys@transformshift{0.448297in}{0.939082in}%
\pgfsys@useobject{currentmarker}{}%
\end{pgfscope}%
\begin{pgfscope}%
\pgfsys@transformshift{0.976510in}{0.816378in}%
\pgfsys@useobject{currentmarker}{}%
\end{pgfscope}%
\begin{pgfscope}%
\pgfsys@transformshift{0.481727in}{0.772783in}%
\pgfsys@useobject{currentmarker}{}%
\end{pgfscope}%
\begin{pgfscope}%
\pgfsys@transformshift{1.808114in}{1.272907in}%
\pgfsys@useobject{currentmarker}{}%
\end{pgfscope}%
\begin{pgfscope}%
\pgfsys@transformshift{0.468390in}{0.825513in}%
\pgfsys@useobject{currentmarker}{}%
\end{pgfscope}%
\begin{pgfscope}%
\pgfsys@transformshift{0.458999in}{0.845010in}%
\pgfsys@useobject{currentmarker}{}%
\end{pgfscope}%
\begin{pgfscope}%
\pgfsys@transformshift{1.685818in}{1.315133in}%
\pgfsys@useobject{currentmarker}{}%
\end{pgfscope}%
\begin{pgfscope}%
\pgfsys@transformshift{0.635418in}{0.753193in}%
\pgfsys@useobject{currentmarker}{}%
\end{pgfscope}%
\begin{pgfscope}%
\pgfsys@transformshift{0.415977in}{0.668006in}%
\pgfsys@useobject{currentmarker}{}%
\end{pgfscope}%
\begin{pgfscope}%
\pgfsys@transformshift{0.341129in}{0.889238in}%
\pgfsys@useobject{currentmarker}{}%
\end{pgfscope}%
\begin{pgfscope}%
\pgfsys@transformshift{0.992976in}{1.047546in}%
\pgfsys@useobject{currentmarker}{}%
\end{pgfscope}%
\begin{pgfscope}%
\pgfsys@transformshift{0.883182in}{0.852044in}%
\pgfsys@useobject{currentmarker}{}%
\end{pgfscope}%
\begin{pgfscope}%
\pgfsys@transformshift{0.480881in}{0.942435in}%
\pgfsys@useobject{currentmarker}{}%
\end{pgfscope}%
\begin{pgfscope}%
\pgfsys@transformshift{1.244411in}{0.869448in}%
\pgfsys@useobject{currentmarker}{}%
\end{pgfscope}%
\begin{pgfscope}%
\pgfsys@transformshift{1.224203in}{1.106037in}%
\pgfsys@useobject{currentmarker}{}%
\end{pgfscope}%
\begin{pgfscope}%
\pgfsys@transformshift{0.449848in}{1.118680in}%
\pgfsys@useobject{currentmarker}{}%
\end{pgfscope}%
\begin{pgfscope}%
\pgfsys@transformshift{0.582527in}{0.650148in}%
\pgfsys@useobject{currentmarker}{}%
\end{pgfscope}%
\begin{pgfscope}%
\pgfsys@transformshift{1.714524in}{1.712593in}%
\pgfsys@useobject{currentmarker}{}%
\end{pgfscope}%
\begin{pgfscope}%
\pgfsys@transformshift{0.891264in}{1.199631in}%
\pgfsys@useobject{currentmarker}{}%
\end{pgfscope}%
\begin{pgfscope}%
\pgfsys@transformshift{1.857615in}{1.648129in}%
\pgfsys@useobject{currentmarker}{}%
\end{pgfscope}%
\begin{pgfscope}%
\pgfsys@transformshift{1.511327in}{1.299700in}%
\pgfsys@useobject{currentmarker}{}%
\end{pgfscope}%
\begin{pgfscope}%
\pgfsys@transformshift{1.748915in}{0.887206in}%
\pgfsys@useobject{currentmarker}{}%
\end{pgfscope}%
\begin{pgfscope}%
\pgfsys@transformshift{0.491937in}{0.872511in}%
\pgfsys@useobject{currentmarker}{}%
\end{pgfscope}%
\begin{pgfscope}%
\pgfsys@transformshift{0.556274in}{0.872944in}%
\pgfsys@useobject{currentmarker}{}%
\end{pgfscope}%
\begin{pgfscope}%
\pgfsys@transformshift{0.468664in}{1.081775in}%
\pgfsys@useobject{currentmarker}{}%
\end{pgfscope}%
\begin{pgfscope}%
\pgfsys@transformshift{1.550098in}{1.331925in}%
\pgfsys@useobject{currentmarker}{}%
\end{pgfscope}%
\begin{pgfscope}%
\pgfsys@transformshift{0.533554in}{0.779334in}%
\pgfsys@useobject{currentmarker}{}%
\end{pgfscope}%
\begin{pgfscope}%
\pgfsys@transformshift{1.009532in}{0.988055in}%
\pgfsys@useobject{currentmarker}{}%
\end{pgfscope}%
\begin{pgfscope}%
\pgfsys@transformshift{0.590179in}{0.693383in}%
\pgfsys@useobject{currentmarker}{}%
\end{pgfscope}%
\begin{pgfscope}%
\pgfsys@transformshift{0.432340in}{1.098713in}%
\pgfsys@useobject{currentmarker}{}%
\end{pgfscope}%
\begin{pgfscope}%
\pgfsys@transformshift{0.372906in}{0.918766in}%
\pgfsys@useobject{currentmarker}{}%
\end{pgfscope}%
\begin{pgfscope}%
\pgfsys@transformshift{0.459645in}{0.887962in}%
\pgfsys@useobject{currentmarker}{}%
\end{pgfscope}%
\begin{pgfscope}%
\pgfsys@transformshift{1.482135in}{1.146494in}%
\pgfsys@useobject{currentmarker}{}%
\end{pgfscope}%
\begin{pgfscope}%
\pgfsys@transformshift{0.622605in}{0.601406in}%
\pgfsys@useobject{currentmarker}{}%
\end{pgfscope}%
\begin{pgfscope}%
\pgfsys@transformshift{0.489705in}{1.024668in}%
\pgfsys@useobject{currentmarker}{}%
\end{pgfscope}%
\begin{pgfscope}%
\pgfsys@transformshift{1.102334in}{0.796139in}%
\pgfsys@useobject{currentmarker}{}%
\end{pgfscope}%
\begin{pgfscope}%
\pgfsys@transformshift{0.877887in}{0.849831in}%
\pgfsys@useobject{currentmarker}{}%
\end{pgfscope}%
\begin{pgfscope}%
\pgfsys@transformshift{0.519403in}{0.736032in}%
\pgfsys@useobject{currentmarker}{}%
\end{pgfscope}%
\begin{pgfscope}%
\pgfsys@transformshift{0.938030in}{0.995665in}%
\pgfsys@useobject{currentmarker}{}%
\end{pgfscope}%
\begin{pgfscope}%
\pgfsys@transformshift{0.961331in}{0.829047in}%
\pgfsys@useobject{currentmarker}{}%
\end{pgfscope}%
\begin{pgfscope}%
\pgfsys@transformshift{1.036665in}{1.125165in}%
\pgfsys@useobject{currentmarker}{}%
\end{pgfscope}%
\begin{pgfscope}%
\pgfsys@transformshift{0.628071in}{0.888463in}%
\pgfsys@useobject{currentmarker}{}%
\end{pgfscope}%
\begin{pgfscope}%
\pgfsys@transformshift{0.835422in}{1.252349in}%
\pgfsys@useobject{currentmarker}{}%
\end{pgfscope}%
\begin{pgfscope}%
\pgfsys@transformshift{1.066394in}{1.067944in}%
\pgfsys@useobject{currentmarker}{}%
\end{pgfscope}%
\begin{pgfscope}%
\pgfsys@transformshift{1.436938in}{1.193219in}%
\pgfsys@useobject{currentmarker}{}%
\end{pgfscope}%
\begin{pgfscope}%
\pgfsys@transformshift{1.603771in}{1.263093in}%
\pgfsys@useobject{currentmarker}{}%
\end{pgfscope}%
\begin{pgfscope}%
\pgfsys@transformshift{0.959028in}{0.977234in}%
\pgfsys@useobject{currentmarker}{}%
\end{pgfscope}%
\begin{pgfscope}%
\pgfsys@transformshift{0.432442in}{0.840903in}%
\pgfsys@useobject{currentmarker}{}%
\end{pgfscope}%
\begin{pgfscope}%
\pgfsys@transformshift{0.897917in}{1.177549in}%
\pgfsys@useobject{currentmarker}{}%
\end{pgfscope}%
\begin{pgfscope}%
\pgfsys@transformshift{0.894441in}{0.670195in}%
\pgfsys@useobject{currentmarker}{}%
\end{pgfscope}%
\begin{pgfscope}%
\pgfsys@transformshift{0.401578in}{1.029347in}%
\pgfsys@useobject{currentmarker}{}%
\end{pgfscope}%
\begin{pgfscope}%
\pgfsys@transformshift{0.411375in}{0.949671in}%
\pgfsys@useobject{currentmarker}{}%
\end{pgfscope}%
\begin{pgfscope}%
\pgfsys@transformshift{0.874473in}{0.853569in}%
\pgfsys@useobject{currentmarker}{}%
\end{pgfscope}%
\begin{pgfscope}%
\pgfsys@transformshift{0.989940in}{0.882186in}%
\pgfsys@useobject{currentmarker}{}%
\end{pgfscope}%
\begin{pgfscope}%
\pgfsys@transformshift{0.410574in}{1.211739in}%
\pgfsys@useobject{currentmarker}{}%
\end{pgfscope}%
\begin{pgfscope}%
\pgfsys@transformshift{0.887331in}{1.032130in}%
\pgfsys@useobject{currentmarker}{}%
\end{pgfscope}%
\begin{pgfscope}%
\pgfsys@transformshift{0.844030in}{0.819469in}%
\pgfsys@useobject{currentmarker}{}%
\end{pgfscope}%
\begin{pgfscope}%
\pgfsys@transformshift{0.923946in}{0.899749in}%
\pgfsys@useobject{currentmarker}{}%
\end{pgfscope}%
\begin{pgfscope}%
\pgfsys@transformshift{1.517373in}{1.274367in}%
\pgfsys@useobject{currentmarker}{}%
\end{pgfscope}%
\begin{pgfscope}%
\pgfsys@transformshift{1.544776in}{1.453499in}%
\pgfsys@useobject{currentmarker}{}%
\end{pgfscope}%
\begin{pgfscope}%
\pgfsys@transformshift{0.466736in}{0.885221in}%
\pgfsys@useobject{currentmarker}{}%
\end{pgfscope}%
\begin{pgfscope}%
\pgfsys@transformshift{0.539327in}{0.582225in}%
\pgfsys@useobject{currentmarker}{}%
\end{pgfscope}%
\begin{pgfscope}%
\pgfsys@transformshift{1.837625in}{1.323887in}%
\pgfsys@useobject{currentmarker}{}%
\end{pgfscope}%
\begin{pgfscope}%
\pgfsys@transformshift{0.442869in}{0.797293in}%
\pgfsys@useobject{currentmarker}{}%
\end{pgfscope}%
\begin{pgfscope}%
\pgfsys@transformshift{1.136179in}{1.405797in}%
\pgfsys@useobject{currentmarker}{}%
\end{pgfscope}%
\begin{pgfscope}%
\pgfsys@transformshift{0.540575in}{0.957063in}%
\pgfsys@useobject{currentmarker}{}%
\end{pgfscope}%
\begin{pgfscope}%
\pgfsys@transformshift{0.505594in}{0.669659in}%
\pgfsys@useobject{currentmarker}{}%
\end{pgfscope}%
\begin{pgfscope}%
\pgfsys@transformshift{0.464353in}{0.691703in}%
\pgfsys@useobject{currentmarker}{}%
\end{pgfscope}%
\begin{pgfscope}%
\pgfsys@transformshift{0.448024in}{0.803658in}%
\pgfsys@useobject{currentmarker}{}%
\end{pgfscope}%
\begin{pgfscope}%
\pgfsys@transformshift{1.133798in}{1.205041in}%
\pgfsys@useobject{currentmarker}{}%
\end{pgfscope}%
\begin{pgfscope}%
\pgfsys@transformshift{0.450472in}{0.727725in}%
\pgfsys@useobject{currentmarker}{}%
\end{pgfscope}%
\begin{pgfscope}%
\pgfsys@transformshift{0.915981in}{0.727118in}%
\pgfsys@useobject{currentmarker}{}%
\end{pgfscope}%
\begin{pgfscope}%
\pgfsys@transformshift{0.893518in}{1.118595in}%
\pgfsys@useobject{currentmarker}{}%
\end{pgfscope}%
\begin{pgfscope}%
\pgfsys@transformshift{0.839737in}{0.862274in}%
\pgfsys@useobject{currentmarker}{}%
\end{pgfscope}%
\begin{pgfscope}%
\pgfsys@transformshift{0.466023in}{0.969980in}%
\pgfsys@useobject{currentmarker}{}%
\end{pgfscope}%
\begin{pgfscope}%
\pgfsys@transformshift{1.451837in}{1.364034in}%
\pgfsys@useobject{currentmarker}{}%
\end{pgfscope}%
\begin{pgfscope}%
\pgfsys@transformshift{1.812241in}{0.898925in}%
\pgfsys@useobject{currentmarker}{}%
\end{pgfscope}%
\begin{pgfscope}%
\pgfsys@transformshift{0.470694in}{0.846292in}%
\pgfsys@useobject{currentmarker}{}%
\end{pgfscope}%
\begin{pgfscope}%
\pgfsys@transformshift{0.373662in}{0.906428in}%
\pgfsys@useobject{currentmarker}{}%
\end{pgfscope}%
\begin{pgfscope}%
\pgfsys@transformshift{0.403930in}{1.251879in}%
\pgfsys@useobject{currentmarker}{}%
\end{pgfscope}%
\begin{pgfscope}%
\pgfsys@transformshift{1.148968in}{1.432334in}%
\pgfsys@useobject{currentmarker}{}%
\end{pgfscope}%
\begin{pgfscope}%
\pgfsys@transformshift{0.466738in}{0.914756in}%
\pgfsys@useobject{currentmarker}{}%
\end{pgfscope}%
\begin{pgfscope}%
\pgfsys@transformshift{0.600280in}{0.956451in}%
\pgfsys@useobject{currentmarker}{}%
\end{pgfscope}%
\begin{pgfscope}%
\pgfsys@transformshift{0.430403in}{0.947260in}%
\pgfsys@useobject{currentmarker}{}%
\end{pgfscope}%
\begin{pgfscope}%
\pgfsys@transformshift{0.917645in}{1.150991in}%
\pgfsys@useobject{currentmarker}{}%
\end{pgfscope}%
\begin{pgfscope}%
\pgfsys@transformshift{1.972216in}{0.932008in}%
\pgfsys@useobject{currentmarker}{}%
\end{pgfscope}%
\begin{pgfscope}%
\pgfsys@transformshift{0.947192in}{1.102919in}%
\pgfsys@useobject{currentmarker}{}%
\end{pgfscope}%
\begin{pgfscope}%
\pgfsys@transformshift{0.945142in}{0.820881in}%
\pgfsys@useobject{currentmarker}{}%
\end{pgfscope}%
\begin{pgfscope}%
\pgfsys@transformshift{0.884523in}{0.859270in}%
\pgfsys@useobject{currentmarker}{}%
\end{pgfscope}%
\begin{pgfscope}%
\pgfsys@transformshift{0.868046in}{1.045270in}%
\pgfsys@useobject{currentmarker}{}%
\end{pgfscope}%
\begin{pgfscope}%
\pgfsys@transformshift{0.427823in}{1.038977in}%
\pgfsys@useobject{currentmarker}{}%
\end{pgfscope}%
\begin{pgfscope}%
\pgfsys@transformshift{1.240509in}{1.013864in}%
\pgfsys@useobject{currentmarker}{}%
\end{pgfscope}%
\begin{pgfscope}%
\pgfsys@transformshift{0.427367in}{1.009521in}%
\pgfsys@useobject{currentmarker}{}%
\end{pgfscope}%
\begin{pgfscope}%
\pgfsys@transformshift{0.877911in}{0.787463in}%
\pgfsys@useobject{currentmarker}{}%
\end{pgfscope}%
\begin{pgfscope}%
\pgfsys@transformshift{0.502381in}{0.681402in}%
\pgfsys@useobject{currentmarker}{}%
\end{pgfscope}%
\begin{pgfscope}%
\pgfsys@transformshift{0.897409in}{0.635702in}%
\pgfsys@useobject{currentmarker}{}%
\end{pgfscope}%
\begin{pgfscope}%
\pgfsys@transformshift{0.371513in}{0.819658in}%
\pgfsys@useobject{currentmarker}{}%
\end{pgfscope}%
\begin{pgfscope}%
\pgfsys@transformshift{0.424068in}{0.885826in}%
\pgfsys@useobject{currentmarker}{}%
\end{pgfscope}%
\begin{pgfscope}%
\pgfsys@transformshift{0.512683in}{0.586059in}%
\pgfsys@useobject{currentmarker}{}%
\end{pgfscope}%
\begin{pgfscope}%
\pgfsys@transformshift{0.423224in}{0.787677in}%
\pgfsys@useobject{currentmarker}{}%
\end{pgfscope}%
\begin{pgfscope}%
\pgfsys@transformshift{0.590216in}{0.674819in}%
\pgfsys@useobject{currentmarker}{}%
\end{pgfscope}%
\begin{pgfscope}%
\pgfsys@transformshift{1.015493in}{0.766359in}%
\pgfsys@useobject{currentmarker}{}%
\end{pgfscope}%
\begin{pgfscope}%
\pgfsys@transformshift{1.527374in}{1.365401in}%
\pgfsys@useobject{currentmarker}{}%
\end{pgfscope}%
\begin{pgfscope}%
\pgfsys@transformshift{1.794630in}{0.818412in}%
\pgfsys@useobject{currentmarker}{}%
\end{pgfscope}%
\begin{pgfscope}%
\pgfsys@transformshift{0.971721in}{1.065313in}%
\pgfsys@useobject{currentmarker}{}%
\end{pgfscope}%
\begin{pgfscope}%
\pgfsys@transformshift{0.402429in}{0.817184in}%
\pgfsys@useobject{currentmarker}{}%
\end{pgfscope}%
\begin{pgfscope}%
\pgfsys@transformshift{0.596209in}{0.783834in}%
\pgfsys@useobject{currentmarker}{}%
\end{pgfscope}%
\begin{pgfscope}%
\pgfsys@transformshift{0.596444in}{0.856021in}%
\pgfsys@useobject{currentmarker}{}%
\end{pgfscope}%
\begin{pgfscope}%
\pgfsys@transformshift{0.472945in}{0.883450in}%
\pgfsys@useobject{currentmarker}{}%
\end{pgfscope}%
\begin{pgfscope}%
\pgfsys@transformshift{0.505729in}{0.971186in}%
\pgfsys@useobject{currentmarker}{}%
\end{pgfscope}%
\begin{pgfscope}%
\pgfsys@transformshift{0.522392in}{0.778855in}%
\pgfsys@useobject{currentmarker}{}%
\end{pgfscope}%
\begin{pgfscope}%
\pgfsys@transformshift{0.958437in}{1.084691in}%
\pgfsys@useobject{currentmarker}{}%
\end{pgfscope}%
\begin{pgfscope}%
\pgfsys@transformshift{0.899312in}{1.280587in}%
\pgfsys@useobject{currentmarker}{}%
\end{pgfscope}%
\begin{pgfscope}%
\pgfsys@transformshift{0.477796in}{1.154528in}%
\pgfsys@useobject{currentmarker}{}%
\end{pgfscope}%
\begin{pgfscope}%
\pgfsys@transformshift{0.495644in}{0.866732in}%
\pgfsys@useobject{currentmarker}{}%
\end{pgfscope}%
\begin{pgfscope}%
\pgfsys@transformshift{0.461404in}{1.064507in}%
\pgfsys@useobject{currentmarker}{}%
\end{pgfscope}%
\begin{pgfscope}%
\pgfsys@transformshift{0.564846in}{0.860019in}%
\pgfsys@useobject{currentmarker}{}%
\end{pgfscope}%
\begin{pgfscope}%
\pgfsys@transformshift{0.438201in}{0.666768in}%
\pgfsys@useobject{currentmarker}{}%
\end{pgfscope}%
\begin{pgfscope}%
\pgfsys@transformshift{0.830917in}{1.148717in}%
\pgfsys@useobject{currentmarker}{}%
\end{pgfscope}%
\begin{pgfscope}%
\pgfsys@transformshift{0.470836in}{0.871795in}%
\pgfsys@useobject{currentmarker}{}%
\end{pgfscope}%
\begin{pgfscope}%
\pgfsys@transformshift{1.732430in}{1.293788in}%
\pgfsys@useobject{currentmarker}{}%
\end{pgfscope}%
\begin{pgfscope}%
\pgfsys@transformshift{0.880194in}{0.817928in}%
\pgfsys@useobject{currentmarker}{}%
\end{pgfscope}%
\begin{pgfscope}%
\pgfsys@transformshift{1.771020in}{0.879999in}%
\pgfsys@useobject{currentmarker}{}%
\end{pgfscope}%
\begin{pgfscope}%
\pgfsys@transformshift{0.408541in}{1.135796in}%
\pgfsys@useobject{currentmarker}{}%
\end{pgfscope}%
\begin{pgfscope}%
\pgfsys@transformshift{1.783013in}{0.869936in}%
\pgfsys@useobject{currentmarker}{}%
\end{pgfscope}%
\begin{pgfscope}%
\pgfsys@transformshift{1.734433in}{0.884771in}%
\pgfsys@useobject{currentmarker}{}%
\end{pgfscope}%
\begin{pgfscope}%
\pgfsys@transformshift{1.772028in}{1.678112in}%
\pgfsys@useobject{currentmarker}{}%
\end{pgfscope}%
\begin{pgfscope}%
\pgfsys@transformshift{0.482187in}{1.247029in}%
\pgfsys@useobject{currentmarker}{}%
\end{pgfscope}%
\begin{pgfscope}%
\pgfsys@transformshift{0.850634in}{0.641639in}%
\pgfsys@useobject{currentmarker}{}%
\end{pgfscope}%
\begin{pgfscope}%
\pgfsys@transformshift{1.489929in}{1.247090in}%
\pgfsys@useobject{currentmarker}{}%
\end{pgfscope}%
\begin{pgfscope}%
\pgfsys@transformshift{0.572833in}{0.795480in}%
\pgfsys@useobject{currentmarker}{}%
\end{pgfscope}%
\begin{pgfscope}%
\pgfsys@transformshift{0.459424in}{0.905080in}%
\pgfsys@useobject{currentmarker}{}%
\end{pgfscope}%
\begin{pgfscope}%
\pgfsys@transformshift{0.454844in}{0.908602in}%
\pgfsys@useobject{currentmarker}{}%
\end{pgfscope}%
\begin{pgfscope}%
\pgfsys@transformshift{1.529823in}{1.267818in}%
\pgfsys@useobject{currentmarker}{}%
\end{pgfscope}%
\begin{pgfscope}%
\pgfsys@transformshift{0.477367in}{1.014549in}%
\pgfsys@useobject{currentmarker}{}%
\end{pgfscope}%
\begin{pgfscope}%
\pgfsys@transformshift{1.049797in}{1.017134in}%
\pgfsys@useobject{currentmarker}{}%
\end{pgfscope}%
\begin{pgfscope}%
\pgfsys@transformshift{1.390925in}{1.775666in}%
\pgfsys@useobject{currentmarker}{}%
\end{pgfscope}%
\begin{pgfscope}%
\pgfsys@transformshift{1.615976in}{0.920289in}%
\pgfsys@useobject{currentmarker}{}%
\end{pgfscope}%
\begin{pgfscope}%
\pgfsys@transformshift{1.879685in}{0.905589in}%
\pgfsys@useobject{currentmarker}{}%
\end{pgfscope}%
\begin{pgfscope}%
\pgfsys@transformshift{0.474522in}{0.807376in}%
\pgfsys@useobject{currentmarker}{}%
\end{pgfscope}%
\begin{pgfscope}%
\pgfsys@transformshift{0.508097in}{1.020336in}%
\pgfsys@useobject{currentmarker}{}%
\end{pgfscope}%
\begin{pgfscope}%
\pgfsys@transformshift{0.846995in}{0.805812in}%
\pgfsys@useobject{currentmarker}{}%
\end{pgfscope}%
\begin{pgfscope}%
\pgfsys@transformshift{1.720538in}{1.485511in}%
\pgfsys@useobject{currentmarker}{}%
\end{pgfscope}%
\begin{pgfscope}%
\pgfsys@transformshift{1.561110in}{1.226139in}%
\pgfsys@useobject{currentmarker}{}%
\end{pgfscope}%
\begin{pgfscope}%
\pgfsys@transformshift{0.477188in}{1.084409in}%
\pgfsys@useobject{currentmarker}{}%
\end{pgfscope}%
\begin{pgfscope}%
\pgfsys@transformshift{0.462555in}{0.975554in}%
\pgfsys@useobject{currentmarker}{}%
\end{pgfscope}%
\begin{pgfscope}%
\pgfsys@transformshift{1.923906in}{1.551429in}%
\pgfsys@useobject{currentmarker}{}%
\end{pgfscope}%
\begin{pgfscope}%
\pgfsys@transformshift{0.990232in}{1.166492in}%
\pgfsys@useobject{currentmarker}{}%
\end{pgfscope}%
\begin{pgfscope}%
\pgfsys@transformshift{0.858173in}{0.922740in}%
\pgfsys@useobject{currentmarker}{}%
\end{pgfscope}%
\begin{pgfscope}%
\pgfsys@transformshift{0.955286in}{1.306969in}%
\pgfsys@useobject{currentmarker}{}%
\end{pgfscope}%
\begin{pgfscope}%
\pgfsys@transformshift{1.614084in}{1.497703in}%
\pgfsys@useobject{currentmarker}{}%
\end{pgfscope}%
\begin{pgfscope}%
\pgfsys@transformshift{0.839480in}{1.236987in}%
\pgfsys@useobject{currentmarker}{}%
\end{pgfscope}%
\begin{pgfscope}%
\pgfsys@transformshift{0.579938in}{0.959274in}%
\pgfsys@useobject{currentmarker}{}%
\end{pgfscope}%
\begin{pgfscope}%
\pgfsys@transformshift{1.021308in}{0.890897in}%
\pgfsys@useobject{currentmarker}{}%
\end{pgfscope}%
\begin{pgfscope}%
\pgfsys@transformshift{0.489542in}{0.930269in}%
\pgfsys@useobject{currentmarker}{}%
\end{pgfscope}%
\begin{pgfscope}%
\pgfsys@transformshift{0.482496in}{0.683753in}%
\pgfsys@useobject{currentmarker}{}%
\end{pgfscope}%
\begin{pgfscope}%
\pgfsys@transformshift{1.231263in}{0.943074in}%
\pgfsys@useobject{currentmarker}{}%
\end{pgfscope}%
\begin{pgfscope}%
\pgfsys@transformshift{0.612346in}{0.936265in}%
\pgfsys@useobject{currentmarker}{}%
\end{pgfscope}%
\begin{pgfscope}%
\pgfsys@transformshift{0.596517in}{0.767110in}%
\pgfsys@useobject{currentmarker}{}%
\end{pgfscope}%
\begin{pgfscope}%
\pgfsys@transformshift{1.011703in}{0.810567in}%
\pgfsys@useobject{currentmarker}{}%
\end{pgfscope}%
\begin{pgfscope}%
\pgfsys@transformshift{0.536110in}{0.747792in}%
\pgfsys@useobject{currentmarker}{}%
\end{pgfscope}%
\begin{pgfscope}%
\pgfsys@transformshift{0.473641in}{0.831255in}%
\pgfsys@useobject{currentmarker}{}%
\end{pgfscope}%
\begin{pgfscope}%
\pgfsys@transformshift{0.949604in}{0.901217in}%
\pgfsys@useobject{currentmarker}{}%
\end{pgfscope}%
\begin{pgfscope}%
\pgfsys@transformshift{1.891487in}{1.668583in}%
\pgfsys@useobject{currentmarker}{}%
\end{pgfscope}%
\begin{pgfscope}%
\pgfsys@transformshift{0.459142in}{1.032290in}%
\pgfsys@useobject{currentmarker}{}%
\end{pgfscope}%
\begin{pgfscope}%
\pgfsys@transformshift{0.993745in}{1.266114in}%
\pgfsys@useobject{currentmarker}{}%
\end{pgfscope}%
\begin{pgfscope}%
\pgfsys@transformshift{0.461010in}{1.022285in}%
\pgfsys@useobject{currentmarker}{}%
\end{pgfscope}%
\begin{pgfscope}%
\pgfsys@transformshift{0.452198in}{0.833123in}%
\pgfsys@useobject{currentmarker}{}%
\end{pgfscope}%
\begin{pgfscope}%
\pgfsys@transformshift{0.567084in}{0.899315in}%
\pgfsys@useobject{currentmarker}{}%
\end{pgfscope}%
\begin{pgfscope}%
\pgfsys@transformshift{0.925975in}{1.270112in}%
\pgfsys@useobject{currentmarker}{}%
\end{pgfscope}%
\begin{pgfscope}%
\pgfsys@transformshift{1.694241in}{0.812136in}%
\pgfsys@useobject{currentmarker}{}%
\end{pgfscope}%
\begin{pgfscope}%
\pgfsys@transformshift{0.589923in}{0.604278in}%
\pgfsys@useobject{currentmarker}{}%
\end{pgfscope}%
\begin{pgfscope}%
\pgfsys@transformshift{1.421991in}{1.265321in}%
\pgfsys@useobject{currentmarker}{}%
\end{pgfscope}%
\begin{pgfscope}%
\pgfsys@transformshift{0.569884in}{0.962529in}%
\pgfsys@useobject{currentmarker}{}%
\end{pgfscope}%
\begin{pgfscope}%
\pgfsys@transformshift{0.481143in}{0.544413in}%
\pgfsys@useobject{currentmarker}{}%
\end{pgfscope}%
\begin{pgfscope}%
\pgfsys@transformshift{0.598950in}{0.596888in}%
\pgfsys@useobject{currentmarker}{}%
\end{pgfscope}%
\begin{pgfscope}%
\pgfsys@transformshift{0.891632in}{0.755347in}%
\pgfsys@useobject{currentmarker}{}%
\end{pgfscope}%
\begin{pgfscope}%
\pgfsys@transformshift{1.518904in}{1.352853in}%
\pgfsys@useobject{currentmarker}{}%
\end{pgfscope}%
\begin{pgfscope}%
\pgfsys@transformshift{1.267712in}{0.898977in}%
\pgfsys@useobject{currentmarker}{}%
\end{pgfscope}%
\begin{pgfscope}%
\pgfsys@transformshift{1.736581in}{1.759871in}%
\pgfsys@useobject{currentmarker}{}%
\end{pgfscope}%
\begin{pgfscope}%
\pgfsys@transformshift{0.507870in}{0.778722in}%
\pgfsys@useobject{currentmarker}{}%
\end{pgfscope}%
\begin{pgfscope}%
\pgfsys@transformshift{0.876979in}{0.906774in}%
\pgfsys@useobject{currentmarker}{}%
\end{pgfscope}%
\begin{pgfscope}%
\pgfsys@transformshift{0.975052in}{1.109821in}%
\pgfsys@useobject{currentmarker}{}%
\end{pgfscope}%
\begin{pgfscope}%
\pgfsys@transformshift{0.581391in}{0.799643in}%
\pgfsys@useobject{currentmarker}{}%
\end{pgfscope}%
\begin{pgfscope}%
\pgfsys@transformshift{0.432905in}{0.793095in}%
\pgfsys@useobject{currentmarker}{}%
\end{pgfscope}%
\begin{pgfscope}%
\pgfsys@transformshift{0.887125in}{1.251236in}%
\pgfsys@useobject{currentmarker}{}%
\end{pgfscope}%
\begin{pgfscope}%
\pgfsys@transformshift{0.887888in}{0.765754in}%
\pgfsys@useobject{currentmarker}{}%
\end{pgfscope}%
\begin{pgfscope}%
\pgfsys@transformshift{1.047522in}{1.065855in}%
\pgfsys@useobject{currentmarker}{}%
\end{pgfscope}%
\begin{pgfscope}%
\pgfsys@transformshift{0.459882in}{1.340388in}%
\pgfsys@useobject{currentmarker}{}%
\end{pgfscope}%
\begin{pgfscope}%
\pgfsys@transformshift{0.869741in}{0.754729in}%
\pgfsys@useobject{currentmarker}{}%
\end{pgfscope}%
\begin{pgfscope}%
\pgfsys@transformshift{1.519451in}{1.228861in}%
\pgfsys@useobject{currentmarker}{}%
\end{pgfscope}%
\begin{pgfscope}%
\pgfsys@transformshift{1.704307in}{1.295959in}%
\pgfsys@useobject{currentmarker}{}%
\end{pgfscope}%
\begin{pgfscope}%
\pgfsys@transformshift{1.029054in}{0.938317in}%
\pgfsys@useobject{currentmarker}{}%
\end{pgfscope}%
\begin{pgfscope}%
\pgfsys@transformshift{0.983134in}{1.239057in}%
\pgfsys@useobject{currentmarker}{}%
\end{pgfscope}%
\begin{pgfscope}%
\pgfsys@transformshift{1.345503in}{1.049835in}%
\pgfsys@useobject{currentmarker}{}%
\end{pgfscope}%
\begin{pgfscope}%
\pgfsys@transformshift{0.480154in}{1.214818in}%
\pgfsys@useobject{currentmarker}{}%
\end{pgfscope}%
\begin{pgfscope}%
\pgfsys@transformshift{0.820400in}{0.759478in}%
\pgfsys@useobject{currentmarker}{}%
\end{pgfscope}%
\begin{pgfscope}%
\pgfsys@transformshift{0.890253in}{1.262146in}%
\pgfsys@useobject{currentmarker}{}%
\end{pgfscope}%
\begin{pgfscope}%
\pgfsys@transformshift{1.194868in}{1.390862in}%
\pgfsys@useobject{currentmarker}{}%
\end{pgfscope}%
\begin{pgfscope}%
\pgfsys@transformshift{0.938368in}{1.567422in}%
\pgfsys@useobject{currentmarker}{}%
\end{pgfscope}%
\begin{pgfscope}%
\pgfsys@transformshift{0.663011in}{0.649169in}%
\pgfsys@useobject{currentmarker}{}%
\end{pgfscope}%
\begin{pgfscope}%
\pgfsys@transformshift{0.916322in}{0.727041in}%
\pgfsys@useobject{currentmarker}{}%
\end{pgfscope}%
\begin{pgfscope}%
\pgfsys@transformshift{0.890291in}{0.921657in}%
\pgfsys@useobject{currentmarker}{}%
\end{pgfscope}%
\begin{pgfscope}%
\pgfsys@transformshift{0.909236in}{0.731934in}%
\pgfsys@useobject{currentmarker}{}%
\end{pgfscope}%
\begin{pgfscope}%
\pgfsys@transformshift{0.862619in}{0.665348in}%
\pgfsys@useobject{currentmarker}{}%
\end{pgfscope}%
\begin{pgfscope}%
\pgfsys@transformshift{0.597487in}{0.979346in}%
\pgfsys@useobject{currentmarker}{}%
\end{pgfscope}%
\begin{pgfscope}%
\pgfsys@transformshift{1.964345in}{0.919529in}%
\pgfsys@useobject{currentmarker}{}%
\end{pgfscope}%
\begin{pgfscope}%
\pgfsys@transformshift{1.692774in}{0.876338in}%
\pgfsys@useobject{currentmarker}{}%
\end{pgfscope}%
\begin{pgfscope}%
\pgfsys@transformshift{0.482531in}{0.923698in}%
\pgfsys@useobject{currentmarker}{}%
\end{pgfscope}%
\begin{pgfscope}%
\pgfsys@transformshift{0.804579in}{0.786283in}%
\pgfsys@useobject{currentmarker}{}%
\end{pgfscope}%
\begin{pgfscope}%
\pgfsys@transformshift{0.458204in}{0.826534in}%
\pgfsys@useobject{currentmarker}{}%
\end{pgfscope}%
\begin{pgfscope}%
\pgfsys@transformshift{1.029005in}{0.917603in}%
\pgfsys@useobject{currentmarker}{}%
\end{pgfscope}%
\begin{pgfscope}%
\pgfsys@transformshift{0.509849in}{0.584143in}%
\pgfsys@useobject{currentmarker}{}%
\end{pgfscope}%
\begin{pgfscope}%
\pgfsys@transformshift{1.034049in}{0.982965in}%
\pgfsys@useobject{currentmarker}{}%
\end{pgfscope}%
\begin{pgfscope}%
\pgfsys@transformshift{1.152583in}{1.025957in}%
\pgfsys@useobject{currentmarker}{}%
\end{pgfscope}%
\begin{pgfscope}%
\pgfsys@transformshift{1.044006in}{1.157914in}%
\pgfsys@useobject{currentmarker}{}%
\end{pgfscope}%
\begin{pgfscope}%
\pgfsys@transformshift{1.486794in}{1.445327in}%
\pgfsys@useobject{currentmarker}{}%
\end{pgfscope}%
\begin{pgfscope}%
\pgfsys@transformshift{0.870195in}{0.754127in}%
\pgfsys@useobject{currentmarker}{}%
\end{pgfscope}%
\begin{pgfscope}%
\pgfsys@transformshift{1.138693in}{1.237600in}%
\pgfsys@useobject{currentmarker}{}%
\end{pgfscope}%
\begin{pgfscope}%
\pgfsys@transformshift{0.847963in}{0.715666in}%
\pgfsys@useobject{currentmarker}{}%
\end{pgfscope}%
\begin{pgfscope}%
\pgfsys@transformshift{0.477983in}{0.940103in}%
\pgfsys@useobject{currentmarker}{}%
\end{pgfscope}%
\begin{pgfscope}%
\pgfsys@transformshift{0.492721in}{0.747421in}%
\pgfsys@useobject{currentmarker}{}%
\end{pgfscope}%
\begin{pgfscope}%
\pgfsys@transformshift{1.120030in}{1.201453in}%
\pgfsys@useobject{currentmarker}{}%
\end{pgfscope}%
\begin{pgfscope}%
\pgfsys@transformshift{0.895919in}{1.119294in}%
\pgfsys@useobject{currentmarker}{}%
\end{pgfscope}%
\begin{pgfscope}%
\pgfsys@transformshift{0.576124in}{0.710070in}%
\pgfsys@useobject{currentmarker}{}%
\end{pgfscope}%
\begin{pgfscope}%
\pgfsys@transformshift{0.471843in}{0.841886in}%
\pgfsys@useobject{currentmarker}{}%
\end{pgfscope}%
\begin{pgfscope}%
\pgfsys@transformshift{0.475316in}{0.719730in}%
\pgfsys@useobject{currentmarker}{}%
\end{pgfscope}%
\begin{pgfscope}%
\pgfsys@transformshift{1.653595in}{1.367920in}%
\pgfsys@useobject{currentmarker}{}%
\end{pgfscope}%
\begin{pgfscope}%
\pgfsys@transformshift{1.083484in}{0.994846in}%
\pgfsys@useobject{currentmarker}{}%
\end{pgfscope}%
\begin{pgfscope}%
\pgfsys@transformshift{0.929704in}{1.084987in}%
\pgfsys@useobject{currentmarker}{}%
\end{pgfscope}%
\begin{pgfscope}%
\pgfsys@transformshift{1.702011in}{1.546991in}%
\pgfsys@useobject{currentmarker}{}%
\end{pgfscope}%
\begin{pgfscope}%
\pgfsys@transformshift{0.861736in}{0.603716in}%
\pgfsys@useobject{currentmarker}{}%
\end{pgfscope}%
\begin{pgfscope}%
\pgfsys@transformshift{0.466758in}{0.712488in}%
\pgfsys@useobject{currentmarker}{}%
\end{pgfscope}%
\begin{pgfscope}%
\pgfsys@transformshift{0.895642in}{1.145491in}%
\pgfsys@useobject{currentmarker}{}%
\end{pgfscope}%
\begin{pgfscope}%
\pgfsys@transformshift{0.844526in}{0.924221in}%
\pgfsys@useobject{currentmarker}{}%
\end{pgfscope}%
\begin{pgfscope}%
\pgfsys@transformshift{0.903057in}{1.260450in}%
\pgfsys@useobject{currentmarker}{}%
\end{pgfscope}%
\begin{pgfscope}%
\pgfsys@transformshift{0.952685in}{0.969953in}%
\pgfsys@useobject{currentmarker}{}%
\end{pgfscope}%
\begin{pgfscope}%
\pgfsys@transformshift{0.576385in}{0.951654in}%
\pgfsys@useobject{currentmarker}{}%
\end{pgfscope}%
\begin{pgfscope}%
\pgfsys@transformshift{1.016020in}{1.155363in}%
\pgfsys@useobject{currentmarker}{}%
\end{pgfscope}%
\begin{pgfscope}%
\pgfsys@transformshift{0.494156in}{0.696081in}%
\pgfsys@useobject{currentmarker}{}%
\end{pgfscope}%
\begin{pgfscope}%
\pgfsys@transformshift{0.577397in}{0.706638in}%
\pgfsys@useobject{currentmarker}{}%
\end{pgfscope}%
\begin{pgfscope}%
\pgfsys@transformshift{0.445519in}{0.769952in}%
\pgfsys@useobject{currentmarker}{}%
\end{pgfscope}%
\begin{pgfscope}%
\pgfsys@transformshift{1.084282in}{1.046020in}%
\pgfsys@useobject{currentmarker}{}%
\end{pgfscope}%
\begin{pgfscope}%
\pgfsys@transformshift{1.594805in}{1.182827in}%
\pgfsys@useobject{currentmarker}{}%
\end{pgfscope}%
\begin{pgfscope}%
\pgfsys@transformshift{0.976071in}{1.044640in}%
\pgfsys@useobject{currentmarker}{}%
\end{pgfscope}%
\begin{pgfscope}%
\pgfsys@transformshift{0.954554in}{1.324438in}%
\pgfsys@useobject{currentmarker}{}%
\end{pgfscope}%
\begin{pgfscope}%
\pgfsys@transformshift{1.062412in}{1.408929in}%
\pgfsys@useobject{currentmarker}{}%
\end{pgfscope}%
\begin{pgfscope}%
\pgfsys@transformshift{0.403708in}{0.743181in}%
\pgfsys@useobject{currentmarker}{}%
\end{pgfscope}%
\begin{pgfscope}%
\pgfsys@transformshift{0.456382in}{0.786576in}%
\pgfsys@useobject{currentmarker}{}%
\end{pgfscope}%
\begin{pgfscope}%
\pgfsys@transformshift{0.425389in}{0.677706in}%
\pgfsys@useobject{currentmarker}{}%
\end{pgfscope}%
\begin{pgfscope}%
\pgfsys@transformshift{1.178319in}{0.782450in}%
\pgfsys@useobject{currentmarker}{}%
\end{pgfscope}%
\begin{pgfscope}%
\pgfsys@transformshift{0.868624in}{1.223032in}%
\pgfsys@useobject{currentmarker}{}%
\end{pgfscope}%
\begin{pgfscope}%
\pgfsys@transformshift{1.072938in}{1.041966in}%
\pgfsys@useobject{currentmarker}{}%
\end{pgfscope}%
\begin{pgfscope}%
\pgfsys@transformshift{0.842975in}{0.737447in}%
\pgfsys@useobject{currentmarker}{}%
\end{pgfscope}%
\begin{pgfscope}%
\pgfsys@transformshift{1.133541in}{1.167924in}%
\pgfsys@useobject{currentmarker}{}%
\end{pgfscope}%
\begin{pgfscope}%
\pgfsys@transformshift{0.985168in}{1.233160in}%
\pgfsys@useobject{currentmarker}{}%
\end{pgfscope}%
\begin{pgfscope}%
\pgfsys@transformshift{0.935163in}{1.436798in}%
\pgfsys@useobject{currentmarker}{}%
\end{pgfscope}%
\begin{pgfscope}%
\pgfsys@transformshift{0.852856in}{0.768599in}%
\pgfsys@useobject{currentmarker}{}%
\end{pgfscope}%
\begin{pgfscope}%
\pgfsys@transformshift{0.481479in}{0.711076in}%
\pgfsys@useobject{currentmarker}{}%
\end{pgfscope}%
\begin{pgfscope}%
\pgfsys@transformshift{0.483502in}{0.944124in}%
\pgfsys@useobject{currentmarker}{}%
\end{pgfscope}%
\begin{pgfscope}%
\pgfsys@transformshift{0.857024in}{0.622559in}%
\pgfsys@useobject{currentmarker}{}%
\end{pgfscope}%
\begin{pgfscope}%
\pgfsys@transformshift{0.541865in}{0.676007in}%
\pgfsys@useobject{currentmarker}{}%
\end{pgfscope}%
\begin{pgfscope}%
\pgfsys@transformshift{1.564506in}{1.412295in}%
\pgfsys@useobject{currentmarker}{}%
\end{pgfscope}%
\begin{pgfscope}%
\pgfsys@transformshift{0.926665in}{1.254116in}%
\pgfsys@useobject{currentmarker}{}%
\end{pgfscope}%
\begin{pgfscope}%
\pgfsys@transformshift{0.371504in}{0.942445in}%
\pgfsys@useobject{currentmarker}{}%
\end{pgfscope}%
\begin{pgfscope}%
\pgfsys@transformshift{0.869520in}{0.923663in}%
\pgfsys@useobject{currentmarker}{}%
\end{pgfscope}%
\begin{pgfscope}%
\pgfsys@transformshift{0.519844in}{0.644602in}%
\pgfsys@useobject{currentmarker}{}%
\end{pgfscope}%
\begin{pgfscope}%
\pgfsys@transformshift{0.608978in}{0.695537in}%
\pgfsys@useobject{currentmarker}{}%
\end{pgfscope}%
\begin{pgfscope}%
\pgfsys@transformshift{0.476814in}{0.808518in}%
\pgfsys@useobject{currentmarker}{}%
\end{pgfscope}%
\begin{pgfscope}%
\pgfsys@transformshift{0.933077in}{1.051436in}%
\pgfsys@useobject{currentmarker}{}%
\end{pgfscope}%
\begin{pgfscope}%
\pgfsys@transformshift{0.579351in}{0.932446in}%
\pgfsys@useobject{currentmarker}{}%
\end{pgfscope}%
\begin{pgfscope}%
\pgfsys@transformshift{0.459383in}{0.775386in}%
\pgfsys@useobject{currentmarker}{}%
\end{pgfscope}%
\begin{pgfscope}%
\pgfsys@transformshift{0.451992in}{0.865131in}%
\pgfsys@useobject{currentmarker}{}%
\end{pgfscope}%
\begin{pgfscope}%
\pgfsys@transformshift{0.448649in}{0.964533in}%
\pgfsys@useobject{currentmarker}{}%
\end{pgfscope}%
\begin{pgfscope}%
\pgfsys@transformshift{1.803636in}{1.535408in}%
\pgfsys@useobject{currentmarker}{}%
\end{pgfscope}%
\begin{pgfscope}%
\pgfsys@transformshift{1.680858in}{0.821256in}%
\pgfsys@useobject{currentmarker}{}%
\end{pgfscope}%
\begin{pgfscope}%
\pgfsys@transformshift{0.464450in}{0.759360in}%
\pgfsys@useobject{currentmarker}{}%
\end{pgfscope}%
\begin{pgfscope}%
\pgfsys@transformshift{0.467818in}{0.779176in}%
\pgfsys@useobject{currentmarker}{}%
\end{pgfscope}%
\begin{pgfscope}%
\pgfsys@transformshift{1.797702in}{1.648282in}%
\pgfsys@useobject{currentmarker}{}%
\end{pgfscope}%
\begin{pgfscope}%
\pgfsys@transformshift{1.800535in}{1.354942in}%
\pgfsys@useobject{currentmarker}{}%
\end{pgfscope}%
\begin{pgfscope}%
\pgfsys@transformshift{0.486053in}{0.763075in}%
\pgfsys@useobject{currentmarker}{}%
\end{pgfscope}%
\begin{pgfscope}%
\pgfsys@transformshift{0.604133in}{0.728778in}%
\pgfsys@useobject{currentmarker}{}%
\end{pgfscope}%
\begin{pgfscope}%
\pgfsys@transformshift{1.510193in}{1.228264in}%
\pgfsys@useobject{currentmarker}{}%
\end{pgfscope}%
\begin{pgfscope}%
\pgfsys@transformshift{0.969856in}{1.103999in}%
\pgfsys@useobject{currentmarker}{}%
\end{pgfscope}%
\begin{pgfscope}%
\pgfsys@transformshift{0.883908in}{1.426865in}%
\pgfsys@useobject{currentmarker}{}%
\end{pgfscope}%
\begin{pgfscope}%
\pgfsys@transformshift{0.538449in}{0.685656in}%
\pgfsys@useobject{currentmarker}{}%
\end{pgfscope}%
\begin{pgfscope}%
\pgfsys@transformshift{0.561398in}{0.793810in}%
\pgfsys@useobject{currentmarker}{}%
\end{pgfscope}%
\begin{pgfscope}%
\pgfsys@transformshift{0.984012in}{1.193845in}%
\pgfsys@useobject{currentmarker}{}%
\end{pgfscope}%
\begin{pgfscope}%
\pgfsys@transformshift{0.614232in}{1.138990in}%
\pgfsys@useobject{currentmarker}{}%
\end{pgfscope}%
\begin{pgfscope}%
\pgfsys@transformshift{1.178616in}{1.218757in}%
\pgfsys@useobject{currentmarker}{}%
\end{pgfscope}%
\begin{pgfscope}%
\pgfsys@transformshift{0.638320in}{0.729711in}%
\pgfsys@useobject{currentmarker}{}%
\end{pgfscope}%
\begin{pgfscope}%
\pgfsys@transformshift{1.571164in}{1.207471in}%
\pgfsys@useobject{currentmarker}{}%
\end{pgfscope}%
\begin{pgfscope}%
\pgfsys@transformshift{0.449499in}{1.107384in}%
\pgfsys@useobject{currentmarker}{}%
\end{pgfscope}%
\begin{pgfscope}%
\pgfsys@transformshift{1.000812in}{1.548277in}%
\pgfsys@useobject{currentmarker}{}%
\end{pgfscope}%
\begin{pgfscope}%
\pgfsys@transformshift{0.504036in}{0.881076in}%
\pgfsys@useobject{currentmarker}{}%
\end{pgfscope}%
\begin{pgfscope}%
\pgfsys@transformshift{1.192646in}{1.247721in}%
\pgfsys@useobject{currentmarker}{}%
\end{pgfscope}%
\begin{pgfscope}%
\pgfsys@transformshift{0.575539in}{0.679322in}%
\pgfsys@useobject{currentmarker}{}%
\end{pgfscope}%
\begin{pgfscope}%
\pgfsys@transformshift{0.517708in}{0.915799in}%
\pgfsys@useobject{currentmarker}{}%
\end{pgfscope}%
\begin{pgfscope}%
\pgfsys@transformshift{1.244210in}{0.736781in}%
\pgfsys@useobject{currentmarker}{}%
\end{pgfscope}%
\begin{pgfscope}%
\pgfsys@transformshift{0.615567in}{0.636259in}%
\pgfsys@useobject{currentmarker}{}%
\end{pgfscope}%
\begin{pgfscope}%
\pgfsys@transformshift{0.907672in}{1.110689in}%
\pgfsys@useobject{currentmarker}{}%
\end{pgfscope}%
\begin{pgfscope}%
\pgfsys@transformshift{2.000000in}{1.901818in}%
\pgfsys@useobject{currentmarker}{}%
\end{pgfscope}%
\begin{pgfscope}%
\pgfsys@transformshift{0.485263in}{0.889235in}%
\pgfsys@useobject{currentmarker}{}%
\end{pgfscope}%
\begin{pgfscope}%
\pgfsys@transformshift{0.433137in}{0.688437in}%
\pgfsys@useobject{currentmarker}{}%
\end{pgfscope}%
\begin{pgfscope}%
\pgfsys@transformshift{0.499449in}{0.957457in}%
\pgfsys@useobject{currentmarker}{}%
\end{pgfscope}%
\begin{pgfscope}%
\pgfsys@transformshift{0.868886in}{0.923162in}%
\pgfsys@useobject{currentmarker}{}%
\end{pgfscope}%
\begin{pgfscope}%
\pgfsys@transformshift{1.451865in}{1.399520in}%
\pgfsys@useobject{currentmarker}{}%
\end{pgfscope}%
\begin{pgfscope}%
\pgfsys@transformshift{0.534197in}{0.943478in}%
\pgfsys@useobject{currentmarker}{}%
\end{pgfscope}%
\begin{pgfscope}%
\pgfsys@transformshift{0.972750in}{1.072814in}%
\pgfsys@useobject{currentmarker}{}%
\end{pgfscope}%
\begin{pgfscope}%
\pgfsys@transformshift{0.996367in}{0.784178in}%
\pgfsys@useobject{currentmarker}{}%
\end{pgfscope}%
\begin{pgfscope}%
\pgfsys@transformshift{0.893513in}{0.728239in}%
\pgfsys@useobject{currentmarker}{}%
\end{pgfscope}%
\begin{pgfscope}%
\pgfsys@transformshift{0.481561in}{0.917704in}%
\pgfsys@useobject{currentmarker}{}%
\end{pgfscope}%
\begin{pgfscope}%
\pgfsys@transformshift{0.462083in}{1.005964in}%
\pgfsys@useobject{currentmarker}{}%
\end{pgfscope}%
\begin{pgfscope}%
\pgfsys@transformshift{0.947888in}{0.735371in}%
\pgfsys@useobject{currentmarker}{}%
\end{pgfscope}%
\begin{pgfscope}%
\pgfsys@transformshift{0.475131in}{0.873113in}%
\pgfsys@useobject{currentmarker}{}%
\end{pgfscope}%
\begin{pgfscope}%
\pgfsys@transformshift{0.944100in}{0.760635in}%
\pgfsys@useobject{currentmarker}{}%
\end{pgfscope}%
\begin{pgfscope}%
\pgfsys@transformshift{0.966481in}{1.013617in}%
\pgfsys@useobject{currentmarker}{}%
\end{pgfscope}%
\begin{pgfscope}%
\pgfsys@transformshift{1.021199in}{0.959586in}%
\pgfsys@useobject{currentmarker}{}%
\end{pgfscope}%
\begin{pgfscope}%
\pgfsys@transformshift{0.451598in}{0.941080in}%
\pgfsys@useobject{currentmarker}{}%
\end{pgfscope}%
\begin{pgfscope}%
\pgfsys@transformshift{0.461004in}{0.928170in}%
\pgfsys@useobject{currentmarker}{}%
\end{pgfscope}%
\begin{pgfscope}%
\pgfsys@transformshift{0.419847in}{0.862853in}%
\pgfsys@useobject{currentmarker}{}%
\end{pgfscope}%
\begin{pgfscope}%
\pgfsys@transformshift{0.498798in}{0.714466in}%
\pgfsys@useobject{currentmarker}{}%
\end{pgfscope}%
\begin{pgfscope}%
\pgfsys@transformshift{1.022778in}{0.864692in}%
\pgfsys@useobject{currentmarker}{}%
\end{pgfscope}%
\begin{pgfscope}%
\pgfsys@transformshift{1.618050in}{0.851249in}%
\pgfsys@useobject{currentmarker}{}%
\end{pgfscope}%
\begin{pgfscope}%
\pgfsys@transformshift{0.873499in}{1.314158in}%
\pgfsys@useobject{currentmarker}{}%
\end{pgfscope}%
\begin{pgfscope}%
\pgfsys@transformshift{1.717975in}{1.156581in}%
\pgfsys@useobject{currentmarker}{}%
\end{pgfscope}%
\begin{pgfscope}%
\pgfsys@transformshift{0.479704in}{1.219238in}%
\pgfsys@useobject{currentmarker}{}%
\end{pgfscope}%
\begin{pgfscope}%
\pgfsys@transformshift{0.921403in}{1.328838in}%
\pgfsys@useobject{currentmarker}{}%
\end{pgfscope}%
\begin{pgfscope}%
\pgfsys@transformshift{0.507519in}{0.857176in}%
\pgfsys@useobject{currentmarker}{}%
\end{pgfscope}%
\begin{pgfscope}%
\pgfsys@transformshift{1.242289in}{1.474977in}%
\pgfsys@useobject{currentmarker}{}%
\end{pgfscope}%
\begin{pgfscope}%
\pgfsys@transformshift{0.901785in}{0.803523in}%
\pgfsys@useobject{currentmarker}{}%
\end{pgfscope}%
\begin{pgfscope}%
\pgfsys@transformshift{1.303585in}{1.007537in}%
\pgfsys@useobject{currentmarker}{}%
\end{pgfscope}%
\begin{pgfscope}%
\pgfsys@transformshift{0.385710in}{0.885783in}%
\pgfsys@useobject{currentmarker}{}%
\end{pgfscope}%
\begin{pgfscope}%
\pgfsys@transformshift{0.487652in}{0.837204in}%
\pgfsys@useobject{currentmarker}{}%
\end{pgfscope}%
\begin{pgfscope}%
\pgfsys@transformshift{0.427840in}{0.684668in}%
\pgfsys@useobject{currentmarker}{}%
\end{pgfscope}%
\begin{pgfscope}%
\pgfsys@transformshift{0.479919in}{0.984282in}%
\pgfsys@useobject{currentmarker}{}%
\end{pgfscope}%
\begin{pgfscope}%
\pgfsys@transformshift{0.421076in}{0.562344in}%
\pgfsys@useobject{currentmarker}{}%
\end{pgfscope}%
\begin{pgfscope}%
\pgfsys@transformshift{0.930792in}{0.843785in}%
\pgfsys@useobject{currentmarker}{}%
\end{pgfscope}%
\begin{pgfscope}%
\pgfsys@transformshift{0.890478in}{1.454928in}%
\pgfsys@useobject{currentmarker}{}%
\end{pgfscope}%
\begin{pgfscope}%
\pgfsys@transformshift{0.528345in}{0.863963in}%
\pgfsys@useobject{currentmarker}{}%
\end{pgfscope}%
\begin{pgfscope}%
\pgfsys@transformshift{1.013284in}{0.982357in}%
\pgfsys@useobject{currentmarker}{}%
\end{pgfscope}%
\begin{pgfscope}%
\pgfsys@transformshift{0.586631in}{0.841398in}%
\pgfsys@useobject{currentmarker}{}%
\end{pgfscope}%
\begin{pgfscope}%
\pgfsys@transformshift{1.117579in}{1.047666in}%
\pgfsys@useobject{currentmarker}{}%
\end{pgfscope}%
\begin{pgfscope}%
\pgfsys@transformshift{0.440363in}{0.823285in}%
\pgfsys@useobject{currentmarker}{}%
\end{pgfscope}%
\begin{pgfscope}%
\pgfsys@transformshift{0.482005in}{1.074157in}%
\pgfsys@useobject{currentmarker}{}%
\end{pgfscope}%
\begin{pgfscope}%
\pgfsys@transformshift{0.439583in}{0.674852in}%
\pgfsys@useobject{currentmarker}{}%
\end{pgfscope}%
\begin{pgfscope}%
\pgfsys@transformshift{0.857963in}{1.000856in}%
\pgfsys@useobject{currentmarker}{}%
\end{pgfscope}%
\begin{pgfscope}%
\pgfsys@transformshift{1.001461in}{0.813812in}%
\pgfsys@useobject{currentmarker}{}%
\end{pgfscope}%
\begin{pgfscope}%
\pgfsys@transformshift{0.484673in}{1.025307in}%
\pgfsys@useobject{currentmarker}{}%
\end{pgfscope}%
\begin{pgfscope}%
\pgfsys@transformshift{0.458444in}{0.834739in}%
\pgfsys@useobject{currentmarker}{}%
\end{pgfscope}%
\begin{pgfscope}%
\pgfsys@transformshift{0.477639in}{1.381801in}%
\pgfsys@useobject{currentmarker}{}%
\end{pgfscope}%
\begin{pgfscope}%
\pgfsys@transformshift{0.908318in}{0.949598in}%
\pgfsys@useobject{currentmarker}{}%
\end{pgfscope}%
\begin{pgfscope}%
\pgfsys@transformshift{0.463108in}{0.754500in}%
\pgfsys@useobject{currentmarker}{}%
\end{pgfscope}%
\begin{pgfscope}%
\pgfsys@transformshift{1.555994in}{1.356879in}%
\pgfsys@useobject{currentmarker}{}%
\end{pgfscope}%
\begin{pgfscope}%
\pgfsys@transformshift{0.456229in}{0.717403in}%
\pgfsys@useobject{currentmarker}{}%
\end{pgfscope}%
\begin{pgfscope}%
\pgfsys@transformshift{0.909471in}{1.452903in}%
\pgfsys@useobject{currentmarker}{}%
\end{pgfscope}%
\end{pgfscope}%
\begin{pgfscope}%
\pgfpathrectangle{\pgfqpoint{0.341129in}{0.466613in}}{\pgfqpoint{1.658871in}{1.711598in}}%
\pgfusepath{clip}%
\pgfsetbuttcap%
\pgfsetroundjoin%
\definecolor{currentfill}{rgb}{0.505882,0.447059,0.701961}%
\pgfsetfillcolor{currentfill}%
\pgfsetfillopacity{0.150000}%
\pgfsetlinewidth{1.003750pt}%
\definecolor{currentstroke}{rgb}{1.000000,1.000000,1.000000}%
\pgfsetstrokecolor{currentstroke}%
\pgfsetstrokeopacity{0.150000}%
\pgfsetdash{}{0pt}%
\pgfsys@defobject{currentmarker}{\pgfqpoint{0.341129in}{0.784074in}}{\pgfqpoint{2.000000in}{1.432328in}}{%
\pgfpathmoveto{\pgfqpoint{0.341129in}{0.839047in}}%
\pgfpathlineto{\pgfqpoint{0.341129in}{0.784074in}}%
\pgfpathlineto{\pgfqpoint{0.357885in}{0.790364in}}%
\pgfpathlineto{\pgfqpoint{0.374641in}{0.796666in}}%
\pgfpathlineto{\pgfqpoint{0.391398in}{0.802958in}}%
\pgfpathlineto{\pgfqpoint{0.408154in}{0.809167in}}%
\pgfpathlineto{\pgfqpoint{0.424910in}{0.815499in}}%
\pgfpathlineto{\pgfqpoint{0.441666in}{0.822021in}}%
\pgfpathlineto{\pgfqpoint{0.458423in}{0.828191in}}%
\pgfpathlineto{\pgfqpoint{0.475179in}{0.834359in}}%
\pgfpathlineto{\pgfqpoint{0.491935in}{0.840468in}}%
\pgfpathlineto{\pgfqpoint{0.508691in}{0.846611in}}%
\pgfpathlineto{\pgfqpoint{0.525448in}{0.853148in}}%
\pgfpathlineto{\pgfqpoint{0.542204in}{0.859486in}}%
\pgfpathlineto{\pgfqpoint{0.558960in}{0.865741in}}%
\pgfpathlineto{\pgfqpoint{0.575717in}{0.872215in}}%
\pgfpathlineto{\pgfqpoint{0.592473in}{0.878343in}}%
\pgfpathlineto{\pgfqpoint{0.609229in}{0.884712in}}%
\pgfpathlineto{\pgfqpoint{0.625985in}{0.890640in}}%
\pgfpathlineto{\pgfqpoint{0.642742in}{0.896178in}}%
\pgfpathlineto{\pgfqpoint{0.659498in}{0.902079in}}%
\pgfpathlineto{\pgfqpoint{0.676254in}{0.907858in}}%
\pgfpathlineto{\pgfqpoint{0.693011in}{0.913632in}}%
\pgfpathlineto{\pgfqpoint{0.709767in}{0.919415in}}%
\pgfpathlineto{\pgfqpoint{0.726523in}{0.925175in}}%
\pgfpathlineto{\pgfqpoint{0.743279in}{0.930970in}}%
\pgfpathlineto{\pgfqpoint{0.760036in}{0.936628in}}%
\pgfpathlineto{\pgfqpoint{0.776792in}{0.942373in}}%
\pgfpathlineto{\pgfqpoint{0.793548in}{0.947929in}}%
\pgfpathlineto{\pgfqpoint{0.810304in}{0.953156in}}%
\pgfpathlineto{\pgfqpoint{0.827061in}{0.958864in}}%
\pgfpathlineto{\pgfqpoint{0.843817in}{0.964246in}}%
\pgfpathlineto{\pgfqpoint{0.860573in}{0.969742in}}%
\pgfpathlineto{\pgfqpoint{0.877330in}{0.975350in}}%
\pgfpathlineto{\pgfqpoint{0.894086in}{0.980651in}}%
\pgfpathlineto{\pgfqpoint{0.910842in}{0.985485in}}%
\pgfpathlineto{\pgfqpoint{0.927598in}{0.990900in}}%
\pgfpathlineto{\pgfqpoint{0.944355in}{0.996324in}}%
\pgfpathlineto{\pgfqpoint{0.961111in}{1.001674in}}%
\pgfpathlineto{\pgfqpoint{0.977867in}{1.007134in}}%
\pgfpathlineto{\pgfqpoint{0.994623in}{1.012358in}}%
\pgfpathlineto{\pgfqpoint{1.011380in}{1.017645in}}%
\pgfpathlineto{\pgfqpoint{1.028136in}{1.022999in}}%
\pgfpathlineto{\pgfqpoint{1.044892in}{1.028227in}}%
\pgfpathlineto{\pgfqpoint{1.061649in}{1.033596in}}%
\pgfpathlineto{\pgfqpoint{1.078405in}{1.038965in}}%
\pgfpathlineto{\pgfqpoint{1.095161in}{1.044111in}}%
\pgfpathlineto{\pgfqpoint{1.111917in}{1.049161in}}%
\pgfpathlineto{\pgfqpoint{1.128674in}{1.054415in}}%
\pgfpathlineto{\pgfqpoint{1.145430in}{1.059515in}}%
\pgfpathlineto{\pgfqpoint{1.162186in}{1.064332in}}%
\pgfpathlineto{\pgfqpoint{1.178942in}{1.069655in}}%
\pgfpathlineto{\pgfqpoint{1.195699in}{1.074581in}}%
\pgfpathlineto{\pgfqpoint{1.212455in}{1.079836in}}%
\pgfpathlineto{\pgfqpoint{1.229211in}{1.084916in}}%
\pgfpathlineto{\pgfqpoint{1.245968in}{1.090080in}}%
\pgfpathlineto{\pgfqpoint{1.262724in}{1.094977in}}%
\pgfpathlineto{\pgfqpoint{1.279480in}{1.100012in}}%
\pgfpathlineto{\pgfqpoint{1.296236in}{1.104919in}}%
\pgfpathlineto{\pgfqpoint{1.312993in}{1.109797in}}%
\pgfpathlineto{\pgfqpoint{1.329749in}{1.114686in}}%
\pgfpathlineto{\pgfqpoint{1.346505in}{1.119576in}}%
\pgfpathlineto{\pgfqpoint{1.363262in}{1.124466in}}%
\pgfpathlineto{\pgfqpoint{1.380018in}{1.129356in}}%
\pgfpathlineto{\pgfqpoint{1.396774in}{1.134249in}}%
\pgfpathlineto{\pgfqpoint{1.413530in}{1.139142in}}%
\pgfpathlineto{\pgfqpoint{1.430287in}{1.144035in}}%
\pgfpathlineto{\pgfqpoint{1.447043in}{1.148923in}}%
\pgfpathlineto{\pgfqpoint{1.463799in}{1.153807in}}%
\pgfpathlineto{\pgfqpoint{1.480555in}{1.158718in}}%
\pgfpathlineto{\pgfqpoint{1.497312in}{1.163737in}}%
\pgfpathlineto{\pgfqpoint{1.514068in}{1.168757in}}%
\pgfpathlineto{\pgfqpoint{1.530824in}{1.173712in}}%
\pgfpathlineto{\pgfqpoint{1.547581in}{1.178691in}}%
\pgfpathlineto{\pgfqpoint{1.564337in}{1.183824in}}%
\pgfpathlineto{\pgfqpoint{1.581093in}{1.188849in}}%
\pgfpathlineto{\pgfqpoint{1.597849in}{1.193874in}}%
\pgfpathlineto{\pgfqpoint{1.614606in}{1.198851in}}%
\pgfpathlineto{\pgfqpoint{1.631362in}{1.203771in}}%
\pgfpathlineto{\pgfqpoint{1.648118in}{1.208660in}}%
\pgfpathlineto{\pgfqpoint{1.664874in}{1.213550in}}%
\pgfpathlineto{\pgfqpoint{1.681631in}{1.218446in}}%
\pgfpathlineto{\pgfqpoint{1.698387in}{1.223342in}}%
\pgfpathlineto{\pgfqpoint{1.715143in}{1.228238in}}%
\pgfpathlineto{\pgfqpoint{1.731900in}{1.233134in}}%
\pgfpathlineto{\pgfqpoint{1.748656in}{1.238030in}}%
\pgfpathlineto{\pgfqpoint{1.765412in}{1.242926in}}%
\pgfpathlineto{\pgfqpoint{1.782168in}{1.247822in}}%
\pgfpathlineto{\pgfqpoint{1.798925in}{1.252723in}}%
\pgfpathlineto{\pgfqpoint{1.815681in}{1.257626in}}%
\pgfpathlineto{\pgfqpoint{1.832437in}{1.262528in}}%
\pgfpathlineto{\pgfqpoint{1.849193in}{1.267431in}}%
\pgfpathlineto{\pgfqpoint{1.865950in}{1.272381in}}%
\pgfpathlineto{\pgfqpoint{1.882706in}{1.277533in}}%
\pgfpathlineto{\pgfqpoint{1.899462in}{1.282685in}}%
\pgfpathlineto{\pgfqpoint{1.916219in}{1.287837in}}%
\pgfpathlineto{\pgfqpoint{1.932975in}{1.292922in}}%
\pgfpathlineto{\pgfqpoint{1.949731in}{1.297853in}}%
\pgfpathlineto{\pgfqpoint{1.966487in}{1.302785in}}%
\pgfpathlineto{\pgfqpoint{1.983244in}{1.307654in}}%
\pgfpathlineto{\pgfqpoint{2.000000in}{1.312448in}}%
\pgfpathlineto{\pgfqpoint{2.000000in}{1.432328in}}%
\pgfpathlineto{\pgfqpoint{2.000000in}{1.432328in}}%
\pgfpathlineto{\pgfqpoint{1.983244in}{1.425953in}}%
\pgfpathlineto{\pgfqpoint{1.966487in}{1.419579in}}%
\pgfpathlineto{\pgfqpoint{1.949731in}{1.413204in}}%
\pgfpathlineto{\pgfqpoint{1.932975in}{1.406819in}}%
\pgfpathlineto{\pgfqpoint{1.916219in}{1.400469in}}%
\pgfpathlineto{\pgfqpoint{1.899462in}{1.394255in}}%
\pgfpathlineto{\pgfqpoint{1.882706in}{1.388041in}}%
\pgfpathlineto{\pgfqpoint{1.865950in}{1.381827in}}%
\pgfpathlineto{\pgfqpoint{1.849193in}{1.375613in}}%
\pgfpathlineto{\pgfqpoint{1.832437in}{1.369136in}}%
\pgfpathlineto{\pgfqpoint{1.815681in}{1.362515in}}%
\pgfpathlineto{\pgfqpoint{1.798925in}{1.355894in}}%
\pgfpathlineto{\pgfqpoint{1.782168in}{1.349519in}}%
\pgfpathlineto{\pgfqpoint{1.765412in}{1.343148in}}%
\pgfpathlineto{\pgfqpoint{1.748656in}{1.336777in}}%
\pgfpathlineto{\pgfqpoint{1.731900in}{1.330406in}}%
\pgfpathlineto{\pgfqpoint{1.715143in}{1.324035in}}%
\pgfpathlineto{\pgfqpoint{1.698387in}{1.317629in}}%
\pgfpathlineto{\pgfqpoint{1.681631in}{1.311169in}}%
\pgfpathlineto{\pgfqpoint{1.664874in}{1.304651in}}%
\pgfpathlineto{\pgfqpoint{1.648118in}{1.298133in}}%
\pgfpathlineto{\pgfqpoint{1.631362in}{1.291614in}}%
\pgfpathlineto{\pgfqpoint{1.614606in}{1.285096in}}%
\pgfpathlineto{\pgfqpoint{1.597849in}{1.278576in}}%
\pgfpathlineto{\pgfqpoint{1.581093in}{1.272280in}}%
\pgfpathlineto{\pgfqpoint{1.564337in}{1.266052in}}%
\pgfpathlineto{\pgfqpoint{1.547581in}{1.259610in}}%
\pgfpathlineto{\pgfqpoint{1.530824in}{1.253281in}}%
\pgfpathlineto{\pgfqpoint{1.514068in}{1.247077in}}%
\pgfpathlineto{\pgfqpoint{1.497312in}{1.240872in}}%
\pgfpathlineto{\pgfqpoint{1.480555in}{1.234665in}}%
\pgfpathlineto{\pgfqpoint{1.463799in}{1.228456in}}%
\pgfpathlineto{\pgfqpoint{1.447043in}{1.222135in}}%
\pgfpathlineto{\pgfqpoint{1.430287in}{1.215770in}}%
\pgfpathlineto{\pgfqpoint{1.413530in}{1.209406in}}%
\pgfpathlineto{\pgfqpoint{1.396774in}{1.203041in}}%
\pgfpathlineto{\pgfqpoint{1.380018in}{1.196770in}}%
\pgfpathlineto{\pgfqpoint{1.363262in}{1.190735in}}%
\pgfpathlineto{\pgfqpoint{1.346505in}{1.184699in}}%
\pgfpathlineto{\pgfqpoint{1.329749in}{1.178405in}}%
\pgfpathlineto{\pgfqpoint{1.312993in}{1.172163in}}%
\pgfpathlineto{\pgfqpoint{1.296236in}{1.165799in}}%
\pgfpathlineto{\pgfqpoint{1.279480in}{1.159436in}}%
\pgfpathlineto{\pgfqpoint{1.262724in}{1.153064in}}%
\pgfpathlineto{\pgfqpoint{1.245968in}{1.146812in}}%
\pgfpathlineto{\pgfqpoint{1.229211in}{1.140355in}}%
\pgfpathlineto{\pgfqpoint{1.212455in}{1.133836in}}%
\pgfpathlineto{\pgfqpoint{1.195699in}{1.127423in}}%
\pgfpathlineto{\pgfqpoint{1.178942in}{1.121050in}}%
\pgfpathlineto{\pgfqpoint{1.162186in}{1.114677in}}%
\pgfpathlineto{\pgfqpoint{1.145430in}{1.108304in}}%
\pgfpathlineto{\pgfqpoint{1.128674in}{1.101931in}}%
\pgfpathlineto{\pgfqpoint{1.111917in}{1.095600in}}%
\pgfpathlineto{\pgfqpoint{1.095161in}{1.089356in}}%
\pgfpathlineto{\pgfqpoint{1.078405in}{1.083139in}}%
\pgfpathlineto{\pgfqpoint{1.061649in}{1.076925in}}%
\pgfpathlineto{\pgfqpoint{1.044892in}{1.070801in}}%
\pgfpathlineto{\pgfqpoint{1.028136in}{1.064678in}}%
\pgfpathlineto{\pgfqpoint{1.011380in}{1.058555in}}%
\pgfpathlineto{\pgfqpoint{0.994623in}{1.052452in}}%
\pgfpathlineto{\pgfqpoint{0.977867in}{1.046349in}}%
\pgfpathlineto{\pgfqpoint{0.961111in}{1.040194in}}%
\pgfpathlineto{\pgfqpoint{0.944355in}{1.034186in}}%
\pgfpathlineto{\pgfqpoint{0.927598in}{1.028260in}}%
\pgfpathlineto{\pgfqpoint{0.910842in}{1.022092in}}%
\pgfpathlineto{\pgfqpoint{0.894086in}{1.016208in}}%
\pgfpathlineto{\pgfqpoint{0.877330in}{1.010363in}}%
\pgfpathlineto{\pgfqpoint{0.860573in}{1.004373in}}%
\pgfpathlineto{\pgfqpoint{0.843817in}{0.998679in}}%
\pgfpathlineto{\pgfqpoint{0.827061in}{0.992795in}}%
\pgfpathlineto{\pgfqpoint{0.810304in}{0.986937in}}%
\pgfpathlineto{\pgfqpoint{0.793548in}{0.981483in}}%
\pgfpathlineto{\pgfqpoint{0.776792in}{0.975258in}}%
\pgfpathlineto{\pgfqpoint{0.760036in}{0.969621in}}%
\pgfpathlineto{\pgfqpoint{0.743279in}{0.963966in}}%
\pgfpathlineto{\pgfqpoint{0.726523in}{0.958789in}}%
\pgfpathlineto{\pgfqpoint{0.709767in}{0.953070in}}%
\pgfpathlineto{\pgfqpoint{0.693011in}{0.947679in}}%
\pgfpathlineto{\pgfqpoint{0.676254in}{0.942377in}}%
\pgfpathlineto{\pgfqpoint{0.659498in}{0.937133in}}%
\pgfpathlineto{\pgfqpoint{0.642742in}{0.931445in}}%
\pgfpathlineto{\pgfqpoint{0.625985in}{0.926107in}}%
\pgfpathlineto{\pgfqpoint{0.609229in}{0.920802in}}%
\pgfpathlineto{\pgfqpoint{0.592473in}{0.915510in}}%
\pgfpathlineto{\pgfqpoint{0.575717in}{0.910294in}}%
\pgfpathlineto{\pgfqpoint{0.558960in}{0.904975in}}%
\pgfpathlineto{\pgfqpoint{0.542204in}{0.899870in}}%
\pgfpathlineto{\pgfqpoint{0.525448in}{0.894764in}}%
\pgfpathlineto{\pgfqpoint{0.508691in}{0.889671in}}%
\pgfpathlineto{\pgfqpoint{0.491935in}{0.884577in}}%
\pgfpathlineto{\pgfqpoint{0.475179in}{0.879478in}}%
\pgfpathlineto{\pgfqpoint{0.458423in}{0.874459in}}%
\pgfpathlineto{\pgfqpoint{0.441666in}{0.869350in}}%
\pgfpathlineto{\pgfqpoint{0.424910in}{0.864205in}}%
\pgfpathlineto{\pgfqpoint{0.408154in}{0.859086in}}%
\pgfpathlineto{\pgfqpoint{0.391398in}{0.853990in}}%
\pgfpathlineto{\pgfqpoint{0.374641in}{0.848983in}}%
\pgfpathlineto{\pgfqpoint{0.357885in}{0.844023in}}%
\pgfpathlineto{\pgfqpoint{0.341129in}{0.839047in}}%
\pgfpathclose%
\pgfusepath{stroke,fill}%
}%
\begin{pgfscope}%
\pgfsys@transformshift{0.000000in}{0.000000in}%
\pgfsys@useobject{currentmarker}{}%
\end{pgfscope}%
\end{pgfscope}%
\begin{pgfscope}%
\pgfpathrectangle{\pgfqpoint{0.341129in}{0.466613in}}{\pgfqpoint{1.658871in}{1.711598in}}%
\pgfusepath{clip}%
\pgfsetroundcap%
\pgfsetroundjoin%
\pgfsetlinewidth{1.505625pt}%
\definecolor{currentstroke}{rgb}{0.298039,0.447059,0.690196}%
\pgfsetstrokecolor{currentstroke}%
\pgfsetdash{}{0pt}%
\pgfpathmoveto{\pgfqpoint{0.341129in}{1.944615in}}%
\pgfpathlineto{\pgfqpoint{0.357885in}{1.945294in}}%
\pgfpathlineto{\pgfqpoint{0.374641in}{1.945973in}}%
\pgfpathlineto{\pgfqpoint{0.391398in}{1.946652in}}%
\pgfpathlineto{\pgfqpoint{0.408154in}{1.947330in}}%
\pgfpathlineto{\pgfqpoint{0.424910in}{1.948009in}}%
\pgfpathlineto{\pgfqpoint{0.441666in}{1.948688in}}%
\pgfpathlineto{\pgfqpoint{0.458423in}{1.949367in}}%
\pgfpathlineto{\pgfqpoint{0.475179in}{1.950046in}}%
\pgfpathlineto{\pgfqpoint{0.491935in}{1.950725in}}%
\pgfpathlineto{\pgfqpoint{0.508691in}{1.951404in}}%
\pgfpathlineto{\pgfqpoint{0.525448in}{1.952083in}}%
\pgfpathlineto{\pgfqpoint{0.542204in}{1.952762in}}%
\pgfpathlineto{\pgfqpoint{0.558960in}{1.953441in}}%
\pgfpathlineto{\pgfqpoint{0.575717in}{1.954120in}}%
\pgfpathlineto{\pgfqpoint{0.592473in}{1.954799in}}%
\pgfpathlineto{\pgfqpoint{0.609229in}{1.955477in}}%
\pgfpathlineto{\pgfqpoint{0.625985in}{1.956156in}}%
\pgfpathlineto{\pgfqpoint{0.642742in}{1.956835in}}%
\pgfpathlineto{\pgfqpoint{0.659498in}{1.957514in}}%
\pgfpathlineto{\pgfqpoint{0.676254in}{1.958193in}}%
\pgfpathlineto{\pgfqpoint{0.693011in}{1.958872in}}%
\pgfpathlineto{\pgfqpoint{0.709767in}{1.959551in}}%
\pgfpathlineto{\pgfqpoint{0.726523in}{1.960230in}}%
\pgfpathlineto{\pgfqpoint{0.743279in}{1.960909in}}%
\pgfpathlineto{\pgfqpoint{0.760036in}{1.961588in}}%
\pgfpathlineto{\pgfqpoint{0.776792in}{1.962267in}}%
\pgfpathlineto{\pgfqpoint{0.793548in}{1.962946in}}%
\pgfpathlineto{\pgfqpoint{0.810304in}{1.963624in}}%
\pgfpathlineto{\pgfqpoint{0.827061in}{1.964303in}}%
\pgfpathlineto{\pgfqpoint{0.843817in}{1.964982in}}%
\pgfpathlineto{\pgfqpoint{0.860573in}{1.965661in}}%
\pgfpathlineto{\pgfqpoint{0.877330in}{1.966340in}}%
\pgfpathlineto{\pgfqpoint{0.894086in}{1.967019in}}%
\pgfpathlineto{\pgfqpoint{0.910842in}{1.967698in}}%
\pgfpathlineto{\pgfqpoint{0.927598in}{1.968377in}}%
\pgfpathlineto{\pgfqpoint{0.944355in}{1.969056in}}%
\pgfpathlineto{\pgfqpoint{0.961111in}{1.969735in}}%
\pgfpathlineto{\pgfqpoint{0.977867in}{1.970414in}}%
\pgfpathlineto{\pgfqpoint{0.994623in}{1.971093in}}%
\pgfpathlineto{\pgfqpoint{1.011380in}{1.971771in}}%
\pgfpathlineto{\pgfqpoint{1.028136in}{1.972450in}}%
\pgfpathlineto{\pgfqpoint{1.044892in}{1.973129in}}%
\pgfpathlineto{\pgfqpoint{1.061649in}{1.973808in}}%
\pgfpathlineto{\pgfqpoint{1.078405in}{1.974487in}}%
\pgfpathlineto{\pgfqpoint{1.095161in}{1.975166in}}%
\pgfpathlineto{\pgfqpoint{1.111917in}{1.975845in}}%
\pgfpathlineto{\pgfqpoint{1.128674in}{1.976524in}}%
\pgfpathlineto{\pgfqpoint{1.145430in}{1.977203in}}%
\pgfpathlineto{\pgfqpoint{1.162186in}{1.977882in}}%
\pgfpathlineto{\pgfqpoint{1.178942in}{1.978561in}}%
\pgfpathlineto{\pgfqpoint{1.195699in}{1.979240in}}%
\pgfpathlineto{\pgfqpoint{1.212455in}{1.979919in}}%
\pgfpathlineto{\pgfqpoint{1.229211in}{1.980597in}}%
\pgfpathlineto{\pgfqpoint{1.245968in}{1.981276in}}%
\pgfpathlineto{\pgfqpoint{1.262724in}{1.981955in}}%
\pgfpathlineto{\pgfqpoint{1.279480in}{1.982634in}}%
\pgfpathlineto{\pgfqpoint{1.296236in}{1.983313in}}%
\pgfpathlineto{\pgfqpoint{1.312993in}{1.983992in}}%
\pgfpathlineto{\pgfqpoint{1.329749in}{1.984671in}}%
\pgfpathlineto{\pgfqpoint{1.346505in}{1.985350in}}%
\pgfpathlineto{\pgfqpoint{1.363262in}{1.986029in}}%
\pgfpathlineto{\pgfqpoint{1.380018in}{1.986708in}}%
\pgfpathlineto{\pgfqpoint{1.396774in}{1.987387in}}%
\pgfpathlineto{\pgfqpoint{1.413530in}{1.988066in}}%
\pgfpathlineto{\pgfqpoint{1.430287in}{1.988744in}}%
\pgfpathlineto{\pgfqpoint{1.447043in}{1.989423in}}%
\pgfpathlineto{\pgfqpoint{1.463799in}{1.990102in}}%
\pgfpathlineto{\pgfqpoint{1.480555in}{1.990781in}}%
\pgfpathlineto{\pgfqpoint{1.497312in}{1.991460in}}%
\pgfpathlineto{\pgfqpoint{1.514068in}{1.992139in}}%
\pgfpathlineto{\pgfqpoint{1.530824in}{1.992818in}}%
\pgfpathlineto{\pgfqpoint{1.547581in}{1.993497in}}%
\pgfpathlineto{\pgfqpoint{1.564337in}{1.994176in}}%
\pgfpathlineto{\pgfqpoint{1.581093in}{1.994855in}}%
\pgfpathlineto{\pgfqpoint{1.597849in}{1.995534in}}%
\pgfpathlineto{\pgfqpoint{1.614606in}{1.996213in}}%
\pgfpathlineto{\pgfqpoint{1.631362in}{1.996891in}}%
\pgfpathlineto{\pgfqpoint{1.648118in}{1.997570in}}%
\pgfpathlineto{\pgfqpoint{1.664874in}{1.998249in}}%
\pgfpathlineto{\pgfqpoint{1.681631in}{1.998928in}}%
\pgfpathlineto{\pgfqpoint{1.698387in}{1.999607in}}%
\pgfpathlineto{\pgfqpoint{1.715143in}{2.000286in}}%
\pgfpathlineto{\pgfqpoint{1.731900in}{2.000965in}}%
\pgfpathlineto{\pgfqpoint{1.748656in}{2.001644in}}%
\pgfpathlineto{\pgfqpoint{1.765412in}{2.002323in}}%
\pgfpathlineto{\pgfqpoint{1.782168in}{2.003002in}}%
\pgfpathlineto{\pgfqpoint{1.798925in}{2.003681in}}%
\pgfpathlineto{\pgfqpoint{1.815681in}{2.004360in}}%
\pgfpathlineto{\pgfqpoint{1.832437in}{2.005038in}}%
\pgfpathlineto{\pgfqpoint{1.849193in}{2.005717in}}%
\pgfpathlineto{\pgfqpoint{1.865950in}{2.006396in}}%
\pgfpathlineto{\pgfqpoint{1.882706in}{2.007075in}}%
\pgfpathlineto{\pgfqpoint{1.899462in}{2.007754in}}%
\pgfpathlineto{\pgfqpoint{1.916219in}{2.008433in}}%
\pgfpathlineto{\pgfqpoint{1.932975in}{2.009112in}}%
\pgfpathlineto{\pgfqpoint{1.949731in}{2.009791in}}%
\pgfpathlineto{\pgfqpoint{1.966487in}{2.010470in}}%
\pgfpathlineto{\pgfqpoint{1.983244in}{2.011149in}}%
\pgfpathlineto{\pgfqpoint{2.000000in}{2.011828in}}%
\pgfusepath{stroke}%
\end{pgfscope}%
\begin{pgfscope}%
\pgfpathrectangle{\pgfqpoint{0.341129in}{0.466613in}}{\pgfqpoint{1.658871in}{1.711598in}}%
\pgfusepath{clip}%
\pgfsetroundcap%
\pgfsetroundjoin%
\pgfsetlinewidth{1.505625pt}%
\definecolor{currentstroke}{rgb}{0.866667,0.517647,0.321569}%
\pgfsetstrokecolor{currentstroke}%
\pgfsetdash{}{0pt}%
\pgfpathmoveto{\pgfqpoint{0.341129in}{1.764670in}}%
\pgfpathlineto{\pgfqpoint{0.357885in}{1.765092in}}%
\pgfpathlineto{\pgfqpoint{0.374641in}{1.765514in}}%
\pgfpathlineto{\pgfqpoint{0.391398in}{1.765935in}}%
\pgfpathlineto{\pgfqpoint{0.408154in}{1.766357in}}%
\pgfpathlineto{\pgfqpoint{0.424910in}{1.766778in}}%
\pgfpathlineto{\pgfqpoint{0.441666in}{1.767200in}}%
\pgfpathlineto{\pgfqpoint{0.458423in}{1.767621in}}%
\pgfpathlineto{\pgfqpoint{0.475179in}{1.768043in}}%
\pgfpathlineto{\pgfqpoint{0.491935in}{1.768464in}}%
\pgfpathlineto{\pgfqpoint{0.508691in}{1.768886in}}%
\pgfpathlineto{\pgfqpoint{0.525448in}{1.769307in}}%
\pgfpathlineto{\pgfqpoint{0.542204in}{1.769729in}}%
\pgfpathlineto{\pgfqpoint{0.558960in}{1.770151in}}%
\pgfpathlineto{\pgfqpoint{0.575717in}{1.770572in}}%
\pgfpathlineto{\pgfqpoint{0.592473in}{1.770994in}}%
\pgfpathlineto{\pgfqpoint{0.609229in}{1.771415in}}%
\pgfpathlineto{\pgfqpoint{0.625985in}{1.771837in}}%
\pgfpathlineto{\pgfqpoint{0.642742in}{1.772258in}}%
\pgfpathlineto{\pgfqpoint{0.659498in}{1.772680in}}%
\pgfpathlineto{\pgfqpoint{0.676254in}{1.773101in}}%
\pgfpathlineto{\pgfqpoint{0.693011in}{1.773523in}}%
\pgfpathlineto{\pgfqpoint{0.709767in}{1.773945in}}%
\pgfpathlineto{\pgfqpoint{0.726523in}{1.774366in}}%
\pgfpathlineto{\pgfqpoint{0.743279in}{1.774788in}}%
\pgfpathlineto{\pgfqpoint{0.760036in}{1.775209in}}%
\pgfpathlineto{\pgfqpoint{0.776792in}{1.775631in}}%
\pgfpathlineto{\pgfqpoint{0.793548in}{1.776052in}}%
\pgfpathlineto{\pgfqpoint{0.810304in}{1.776474in}}%
\pgfpathlineto{\pgfqpoint{0.827061in}{1.776895in}}%
\pgfpathlineto{\pgfqpoint{0.843817in}{1.777317in}}%
\pgfpathlineto{\pgfqpoint{0.860573in}{1.777738in}}%
\pgfpathlineto{\pgfqpoint{0.877330in}{1.778160in}}%
\pgfpathlineto{\pgfqpoint{0.894086in}{1.778582in}}%
\pgfpathlineto{\pgfqpoint{0.910842in}{1.779003in}}%
\pgfpathlineto{\pgfqpoint{0.927598in}{1.779425in}}%
\pgfpathlineto{\pgfqpoint{0.944355in}{1.779846in}}%
\pgfpathlineto{\pgfqpoint{0.961111in}{1.780268in}}%
\pgfpathlineto{\pgfqpoint{0.977867in}{1.780689in}}%
\pgfpathlineto{\pgfqpoint{0.994623in}{1.781111in}}%
\pgfpathlineto{\pgfqpoint{1.011380in}{1.781532in}}%
\pgfpathlineto{\pgfqpoint{1.028136in}{1.781954in}}%
\pgfpathlineto{\pgfqpoint{1.044892in}{1.782376in}}%
\pgfpathlineto{\pgfqpoint{1.061649in}{1.782797in}}%
\pgfpathlineto{\pgfqpoint{1.078405in}{1.783219in}}%
\pgfpathlineto{\pgfqpoint{1.095161in}{1.783640in}}%
\pgfpathlineto{\pgfqpoint{1.111917in}{1.784062in}}%
\pgfpathlineto{\pgfqpoint{1.128674in}{1.784483in}}%
\pgfpathlineto{\pgfqpoint{1.145430in}{1.784905in}}%
\pgfpathlineto{\pgfqpoint{1.162186in}{1.785326in}}%
\pgfpathlineto{\pgfqpoint{1.178942in}{1.785748in}}%
\pgfpathlineto{\pgfqpoint{1.195699in}{1.786170in}}%
\pgfpathlineto{\pgfqpoint{1.212455in}{1.786591in}}%
\pgfpathlineto{\pgfqpoint{1.229211in}{1.787013in}}%
\pgfpathlineto{\pgfqpoint{1.245968in}{1.787434in}}%
\pgfpathlineto{\pgfqpoint{1.262724in}{1.787856in}}%
\pgfpathlineto{\pgfqpoint{1.279480in}{1.788277in}}%
\pgfpathlineto{\pgfqpoint{1.296236in}{1.788699in}}%
\pgfpathlineto{\pgfqpoint{1.312993in}{1.789120in}}%
\pgfpathlineto{\pgfqpoint{1.329749in}{1.789542in}}%
\pgfpathlineto{\pgfqpoint{1.346505in}{1.789963in}}%
\pgfpathlineto{\pgfqpoint{1.363262in}{1.790385in}}%
\pgfpathlineto{\pgfqpoint{1.380018in}{1.790807in}}%
\pgfpathlineto{\pgfqpoint{1.396774in}{1.791228in}}%
\pgfpathlineto{\pgfqpoint{1.413530in}{1.791650in}}%
\pgfpathlineto{\pgfqpoint{1.430287in}{1.792071in}}%
\pgfpathlineto{\pgfqpoint{1.447043in}{1.792493in}}%
\pgfpathlineto{\pgfqpoint{1.463799in}{1.792914in}}%
\pgfpathlineto{\pgfqpoint{1.480555in}{1.793336in}}%
\pgfpathlineto{\pgfqpoint{1.497312in}{1.793757in}}%
\pgfpathlineto{\pgfqpoint{1.514068in}{1.794179in}}%
\pgfpathlineto{\pgfqpoint{1.530824in}{1.794601in}}%
\pgfpathlineto{\pgfqpoint{1.547581in}{1.795022in}}%
\pgfpathlineto{\pgfqpoint{1.564337in}{1.795444in}}%
\pgfpathlineto{\pgfqpoint{1.581093in}{1.795865in}}%
\pgfpathlineto{\pgfqpoint{1.597849in}{1.796287in}}%
\pgfpathlineto{\pgfqpoint{1.614606in}{1.796708in}}%
\pgfpathlineto{\pgfqpoint{1.631362in}{1.797130in}}%
\pgfpathlineto{\pgfqpoint{1.648118in}{1.797551in}}%
\pgfpathlineto{\pgfqpoint{1.664874in}{1.797973in}}%
\pgfpathlineto{\pgfqpoint{1.681631in}{1.798394in}}%
\pgfpathlineto{\pgfqpoint{1.698387in}{1.798816in}}%
\pgfpathlineto{\pgfqpoint{1.715143in}{1.799238in}}%
\pgfpathlineto{\pgfqpoint{1.731900in}{1.799659in}}%
\pgfpathlineto{\pgfqpoint{1.748656in}{1.800081in}}%
\pgfpathlineto{\pgfqpoint{1.765412in}{1.800502in}}%
\pgfpathlineto{\pgfqpoint{1.782168in}{1.800924in}}%
\pgfpathlineto{\pgfqpoint{1.798925in}{1.801345in}}%
\pgfpathlineto{\pgfqpoint{1.815681in}{1.801767in}}%
\pgfpathlineto{\pgfqpoint{1.832437in}{1.802188in}}%
\pgfpathlineto{\pgfqpoint{1.849193in}{1.802610in}}%
\pgfpathlineto{\pgfqpoint{1.865950in}{1.803032in}}%
\pgfpathlineto{\pgfqpoint{1.882706in}{1.803453in}}%
\pgfpathlineto{\pgfqpoint{1.899462in}{1.803875in}}%
\pgfpathlineto{\pgfqpoint{1.916219in}{1.804296in}}%
\pgfpathlineto{\pgfqpoint{1.932975in}{1.804718in}}%
\pgfpathlineto{\pgfqpoint{1.949731in}{1.805139in}}%
\pgfpathlineto{\pgfqpoint{1.966487in}{1.805561in}}%
\pgfpathlineto{\pgfqpoint{1.983244in}{1.805982in}}%
\pgfpathlineto{\pgfqpoint{2.000000in}{1.806404in}}%
\pgfusepath{stroke}%
\end{pgfscope}%
\begin{pgfscope}%
\pgfpathrectangle{\pgfqpoint{0.341129in}{0.466613in}}{\pgfqpoint{1.658871in}{1.711598in}}%
\pgfusepath{clip}%
\pgfsetroundcap%
\pgfsetroundjoin%
\pgfsetlinewidth{1.505625pt}%
\definecolor{currentstroke}{rgb}{0.333333,0.658824,0.407843}%
\pgfsetstrokecolor{currentstroke}%
\pgfsetdash{}{0pt}%
\pgfpathmoveto{\pgfqpoint{0.341129in}{1.438275in}}%
\pgfpathlineto{\pgfqpoint{0.357885in}{1.439539in}}%
\pgfpathlineto{\pgfqpoint{0.374641in}{1.440804in}}%
\pgfpathlineto{\pgfqpoint{0.391398in}{1.442068in}}%
\pgfpathlineto{\pgfqpoint{0.408154in}{1.443333in}}%
\pgfpathlineto{\pgfqpoint{0.424910in}{1.444597in}}%
\pgfpathlineto{\pgfqpoint{0.441666in}{1.445862in}}%
\pgfpathlineto{\pgfqpoint{0.458423in}{1.447126in}}%
\pgfpathlineto{\pgfqpoint{0.475179in}{1.448391in}}%
\pgfpathlineto{\pgfqpoint{0.491935in}{1.449655in}}%
\pgfpathlineto{\pgfqpoint{0.508691in}{1.450920in}}%
\pgfpathlineto{\pgfqpoint{0.525448in}{1.452184in}}%
\pgfpathlineto{\pgfqpoint{0.542204in}{1.453449in}}%
\pgfpathlineto{\pgfqpoint{0.558960in}{1.454713in}}%
\pgfpathlineto{\pgfqpoint{0.575717in}{1.455978in}}%
\pgfpathlineto{\pgfqpoint{0.592473in}{1.457242in}}%
\pgfpathlineto{\pgfqpoint{0.609229in}{1.458507in}}%
\pgfpathlineto{\pgfqpoint{0.625985in}{1.459771in}}%
\pgfpathlineto{\pgfqpoint{0.642742in}{1.461036in}}%
\pgfpathlineto{\pgfqpoint{0.659498in}{1.462300in}}%
\pgfpathlineto{\pgfqpoint{0.676254in}{1.463565in}}%
\pgfpathlineto{\pgfqpoint{0.693011in}{1.464829in}}%
\pgfpathlineto{\pgfqpoint{0.709767in}{1.466094in}}%
\pgfpathlineto{\pgfqpoint{0.726523in}{1.467358in}}%
\pgfpathlineto{\pgfqpoint{0.743279in}{1.468623in}}%
\pgfpathlineto{\pgfqpoint{0.760036in}{1.469887in}}%
\pgfpathlineto{\pgfqpoint{0.776792in}{1.471152in}}%
\pgfpathlineto{\pgfqpoint{0.793548in}{1.472416in}}%
\pgfpathlineto{\pgfqpoint{0.810304in}{1.473681in}}%
\pgfpathlineto{\pgfqpoint{0.827061in}{1.474945in}}%
\pgfpathlineto{\pgfqpoint{0.843817in}{1.476210in}}%
\pgfpathlineto{\pgfqpoint{0.860573in}{1.477474in}}%
\pgfpathlineto{\pgfqpoint{0.877330in}{1.478739in}}%
\pgfpathlineto{\pgfqpoint{0.894086in}{1.480004in}}%
\pgfpathlineto{\pgfqpoint{0.910842in}{1.481268in}}%
\pgfpathlineto{\pgfqpoint{0.927598in}{1.482533in}}%
\pgfpathlineto{\pgfqpoint{0.944355in}{1.483797in}}%
\pgfpathlineto{\pgfqpoint{0.961111in}{1.485062in}}%
\pgfpathlineto{\pgfqpoint{0.977867in}{1.486326in}}%
\pgfpathlineto{\pgfqpoint{0.994623in}{1.487591in}}%
\pgfpathlineto{\pgfqpoint{1.011380in}{1.488855in}}%
\pgfpathlineto{\pgfqpoint{1.028136in}{1.490120in}}%
\pgfpathlineto{\pgfqpoint{1.044892in}{1.491384in}}%
\pgfpathlineto{\pgfqpoint{1.061649in}{1.492649in}}%
\pgfpathlineto{\pgfqpoint{1.078405in}{1.493913in}}%
\pgfpathlineto{\pgfqpoint{1.095161in}{1.495178in}}%
\pgfpathlineto{\pgfqpoint{1.111917in}{1.496442in}}%
\pgfpathlineto{\pgfqpoint{1.128674in}{1.497707in}}%
\pgfpathlineto{\pgfqpoint{1.145430in}{1.498971in}}%
\pgfpathlineto{\pgfqpoint{1.162186in}{1.500236in}}%
\pgfpathlineto{\pgfqpoint{1.178942in}{1.501500in}}%
\pgfpathlineto{\pgfqpoint{1.195699in}{1.502765in}}%
\pgfpathlineto{\pgfqpoint{1.212455in}{1.504029in}}%
\pgfpathlineto{\pgfqpoint{1.229211in}{1.505294in}}%
\pgfpathlineto{\pgfqpoint{1.245968in}{1.506558in}}%
\pgfpathlineto{\pgfqpoint{1.262724in}{1.507823in}}%
\pgfpathlineto{\pgfqpoint{1.279480in}{1.509087in}}%
\pgfpathlineto{\pgfqpoint{1.296236in}{1.510352in}}%
\pgfpathlineto{\pgfqpoint{1.312993in}{1.511616in}}%
\pgfpathlineto{\pgfqpoint{1.329749in}{1.512881in}}%
\pgfpathlineto{\pgfqpoint{1.346505in}{1.514145in}}%
\pgfpathlineto{\pgfqpoint{1.363262in}{1.515410in}}%
\pgfpathlineto{\pgfqpoint{1.380018in}{1.516674in}}%
\pgfpathlineto{\pgfqpoint{1.396774in}{1.517939in}}%
\pgfpathlineto{\pgfqpoint{1.413530in}{1.519203in}}%
\pgfpathlineto{\pgfqpoint{1.430287in}{1.520468in}}%
\pgfpathlineto{\pgfqpoint{1.447043in}{1.521732in}}%
\pgfpathlineto{\pgfqpoint{1.463799in}{1.522997in}}%
\pgfpathlineto{\pgfqpoint{1.480555in}{1.524261in}}%
\pgfpathlineto{\pgfqpoint{1.497312in}{1.525526in}}%
\pgfpathlineto{\pgfqpoint{1.514068in}{1.526790in}}%
\pgfpathlineto{\pgfqpoint{1.530824in}{1.528055in}}%
\pgfpathlineto{\pgfqpoint{1.547581in}{1.529319in}}%
\pgfpathlineto{\pgfqpoint{1.564337in}{1.530584in}}%
\pgfpathlineto{\pgfqpoint{1.581093in}{1.531848in}}%
\pgfpathlineto{\pgfqpoint{1.597849in}{1.533113in}}%
\pgfpathlineto{\pgfqpoint{1.614606in}{1.534377in}}%
\pgfpathlineto{\pgfqpoint{1.631362in}{1.535642in}}%
\pgfpathlineto{\pgfqpoint{1.648118in}{1.536906in}}%
\pgfpathlineto{\pgfqpoint{1.664874in}{1.538171in}}%
\pgfpathlineto{\pgfqpoint{1.681631in}{1.539435in}}%
\pgfpathlineto{\pgfqpoint{1.698387in}{1.540700in}}%
\pgfpathlineto{\pgfqpoint{1.715143in}{1.541964in}}%
\pgfpathlineto{\pgfqpoint{1.731900in}{1.543229in}}%
\pgfpathlineto{\pgfqpoint{1.748656in}{1.544493in}}%
\pgfpathlineto{\pgfqpoint{1.765412in}{1.545758in}}%
\pgfpathlineto{\pgfqpoint{1.782168in}{1.547022in}}%
\pgfpathlineto{\pgfqpoint{1.798925in}{1.548287in}}%
\pgfpathlineto{\pgfqpoint{1.815681in}{1.549551in}}%
\pgfpathlineto{\pgfqpoint{1.832437in}{1.550816in}}%
\pgfpathlineto{\pgfqpoint{1.849193in}{1.552080in}}%
\pgfpathlineto{\pgfqpoint{1.865950in}{1.553345in}}%
\pgfpathlineto{\pgfqpoint{1.882706in}{1.554609in}}%
\pgfpathlineto{\pgfqpoint{1.899462in}{1.555874in}}%
\pgfpathlineto{\pgfqpoint{1.916219in}{1.557138in}}%
\pgfpathlineto{\pgfqpoint{1.932975in}{1.558403in}}%
\pgfpathlineto{\pgfqpoint{1.949731in}{1.559667in}}%
\pgfpathlineto{\pgfqpoint{1.966487in}{1.560932in}}%
\pgfpathlineto{\pgfqpoint{1.983244in}{1.562196in}}%
\pgfpathlineto{\pgfqpoint{2.000000in}{1.563461in}}%
\pgfusepath{stroke}%
\end{pgfscope}%
\begin{pgfscope}%
\pgfpathrectangle{\pgfqpoint{0.341129in}{0.466613in}}{\pgfqpoint{1.658871in}{1.711598in}}%
\pgfusepath{clip}%
\pgfsetroundcap%
\pgfsetroundjoin%
\pgfsetlinewidth{1.505625pt}%
\definecolor{currentstroke}{rgb}{0.768627,0.305882,0.321569}%
\pgfsetstrokecolor{currentstroke}%
\pgfsetdash{}{0pt}%
\pgfpathmoveto{\pgfqpoint{0.341129in}{1.201584in}}%
\pgfpathlineto{\pgfqpoint{0.357885in}{1.204066in}}%
\pgfpathlineto{\pgfqpoint{0.374641in}{1.206549in}}%
\pgfpathlineto{\pgfqpoint{0.391398in}{1.209032in}}%
\pgfpathlineto{\pgfqpoint{0.408154in}{1.211515in}}%
\pgfpathlineto{\pgfqpoint{0.424910in}{1.213998in}}%
\pgfpathlineto{\pgfqpoint{0.441666in}{1.216481in}}%
\pgfpathlineto{\pgfqpoint{0.458423in}{1.218963in}}%
\pgfpathlineto{\pgfqpoint{0.475179in}{1.221446in}}%
\pgfpathlineto{\pgfqpoint{0.491935in}{1.223929in}}%
\pgfpathlineto{\pgfqpoint{0.508691in}{1.226412in}}%
\pgfpathlineto{\pgfqpoint{0.525448in}{1.228895in}}%
\pgfpathlineto{\pgfqpoint{0.542204in}{1.231377in}}%
\pgfpathlineto{\pgfqpoint{0.558960in}{1.233860in}}%
\pgfpathlineto{\pgfqpoint{0.575717in}{1.236343in}}%
\pgfpathlineto{\pgfqpoint{0.592473in}{1.238826in}}%
\pgfpathlineto{\pgfqpoint{0.609229in}{1.241309in}}%
\pgfpathlineto{\pgfqpoint{0.625985in}{1.243792in}}%
\pgfpathlineto{\pgfqpoint{0.642742in}{1.246274in}}%
\pgfpathlineto{\pgfqpoint{0.659498in}{1.248757in}}%
\pgfpathlineto{\pgfqpoint{0.676254in}{1.251240in}}%
\pgfpathlineto{\pgfqpoint{0.693011in}{1.253723in}}%
\pgfpathlineto{\pgfqpoint{0.709767in}{1.256206in}}%
\pgfpathlineto{\pgfqpoint{0.726523in}{1.258689in}}%
\pgfpathlineto{\pgfqpoint{0.743279in}{1.261171in}}%
\pgfpathlineto{\pgfqpoint{0.760036in}{1.263654in}}%
\pgfpathlineto{\pgfqpoint{0.776792in}{1.266137in}}%
\pgfpathlineto{\pgfqpoint{0.793548in}{1.268620in}}%
\pgfpathlineto{\pgfqpoint{0.810304in}{1.271103in}}%
\pgfpathlineto{\pgfqpoint{0.827061in}{1.273586in}}%
\pgfpathlineto{\pgfqpoint{0.843817in}{1.276068in}}%
\pgfpathlineto{\pgfqpoint{0.860573in}{1.278551in}}%
\pgfpathlineto{\pgfqpoint{0.877330in}{1.281034in}}%
\pgfpathlineto{\pgfqpoint{0.894086in}{1.283517in}}%
\pgfpathlineto{\pgfqpoint{0.910842in}{1.286000in}}%
\pgfpathlineto{\pgfqpoint{0.927598in}{1.288482in}}%
\pgfpathlineto{\pgfqpoint{0.944355in}{1.290965in}}%
\pgfpathlineto{\pgfqpoint{0.961111in}{1.293448in}}%
\pgfpathlineto{\pgfqpoint{0.977867in}{1.295931in}}%
\pgfpathlineto{\pgfqpoint{0.994623in}{1.298414in}}%
\pgfpathlineto{\pgfqpoint{1.011380in}{1.300897in}}%
\pgfpathlineto{\pgfqpoint{1.028136in}{1.303379in}}%
\pgfpathlineto{\pgfqpoint{1.044892in}{1.305862in}}%
\pgfpathlineto{\pgfqpoint{1.061649in}{1.308345in}}%
\pgfpathlineto{\pgfqpoint{1.078405in}{1.310828in}}%
\pgfpathlineto{\pgfqpoint{1.095161in}{1.313311in}}%
\pgfpathlineto{\pgfqpoint{1.111917in}{1.315794in}}%
\pgfpathlineto{\pgfqpoint{1.128674in}{1.318276in}}%
\pgfpathlineto{\pgfqpoint{1.145430in}{1.320759in}}%
\pgfpathlineto{\pgfqpoint{1.162186in}{1.323242in}}%
\pgfpathlineto{\pgfqpoint{1.178942in}{1.325725in}}%
\pgfpathlineto{\pgfqpoint{1.195699in}{1.328208in}}%
\pgfpathlineto{\pgfqpoint{1.212455in}{1.330691in}}%
\pgfpathlineto{\pgfqpoint{1.229211in}{1.333173in}}%
\pgfpathlineto{\pgfqpoint{1.245968in}{1.335656in}}%
\pgfpathlineto{\pgfqpoint{1.262724in}{1.338139in}}%
\pgfpathlineto{\pgfqpoint{1.279480in}{1.340622in}}%
\pgfpathlineto{\pgfqpoint{1.296236in}{1.343105in}}%
\pgfpathlineto{\pgfqpoint{1.312993in}{1.345587in}}%
\pgfpathlineto{\pgfqpoint{1.329749in}{1.348070in}}%
\pgfpathlineto{\pgfqpoint{1.346505in}{1.350553in}}%
\pgfpathlineto{\pgfqpoint{1.363262in}{1.353036in}}%
\pgfpathlineto{\pgfqpoint{1.380018in}{1.355519in}}%
\pgfpathlineto{\pgfqpoint{1.396774in}{1.358002in}}%
\pgfpathlineto{\pgfqpoint{1.413530in}{1.360484in}}%
\pgfpathlineto{\pgfqpoint{1.430287in}{1.362967in}}%
\pgfpathlineto{\pgfqpoint{1.447043in}{1.365450in}}%
\pgfpathlineto{\pgfqpoint{1.463799in}{1.367933in}}%
\pgfpathlineto{\pgfqpoint{1.480555in}{1.370416in}}%
\pgfpathlineto{\pgfqpoint{1.497312in}{1.372899in}}%
\pgfpathlineto{\pgfqpoint{1.514068in}{1.375381in}}%
\pgfpathlineto{\pgfqpoint{1.530824in}{1.377864in}}%
\pgfpathlineto{\pgfqpoint{1.547581in}{1.380347in}}%
\pgfpathlineto{\pgfqpoint{1.564337in}{1.382830in}}%
\pgfpathlineto{\pgfqpoint{1.581093in}{1.385313in}}%
\pgfpathlineto{\pgfqpoint{1.597849in}{1.387796in}}%
\pgfpathlineto{\pgfqpoint{1.614606in}{1.390278in}}%
\pgfpathlineto{\pgfqpoint{1.631362in}{1.392761in}}%
\pgfpathlineto{\pgfqpoint{1.648118in}{1.395244in}}%
\pgfpathlineto{\pgfqpoint{1.664874in}{1.397727in}}%
\pgfpathlineto{\pgfqpoint{1.681631in}{1.400210in}}%
\pgfpathlineto{\pgfqpoint{1.698387in}{1.402693in}}%
\pgfpathlineto{\pgfqpoint{1.715143in}{1.405175in}}%
\pgfpathlineto{\pgfqpoint{1.731900in}{1.407658in}}%
\pgfpathlineto{\pgfqpoint{1.748656in}{1.410141in}}%
\pgfpathlineto{\pgfqpoint{1.765412in}{1.412624in}}%
\pgfpathlineto{\pgfqpoint{1.782168in}{1.415107in}}%
\pgfpathlineto{\pgfqpoint{1.798925in}{1.417589in}}%
\pgfpathlineto{\pgfqpoint{1.815681in}{1.420072in}}%
\pgfpathlineto{\pgfqpoint{1.832437in}{1.422555in}}%
\pgfpathlineto{\pgfqpoint{1.849193in}{1.425038in}}%
\pgfpathlineto{\pgfqpoint{1.865950in}{1.427521in}}%
\pgfpathlineto{\pgfqpoint{1.882706in}{1.430004in}}%
\pgfpathlineto{\pgfqpoint{1.899462in}{1.432486in}}%
\pgfpathlineto{\pgfqpoint{1.916219in}{1.434969in}}%
\pgfpathlineto{\pgfqpoint{1.932975in}{1.437452in}}%
\pgfpathlineto{\pgfqpoint{1.949731in}{1.439935in}}%
\pgfpathlineto{\pgfqpoint{1.966487in}{1.442418in}}%
\pgfpathlineto{\pgfqpoint{1.983244in}{1.444901in}}%
\pgfpathlineto{\pgfqpoint{2.000000in}{1.447383in}}%
\pgfusepath{stroke}%
\end{pgfscope}%
\begin{pgfscope}%
\pgfpathrectangle{\pgfqpoint{0.341129in}{0.466613in}}{\pgfqpoint{1.658871in}{1.711598in}}%
\pgfusepath{clip}%
\pgfsetroundcap%
\pgfsetroundjoin%
\pgfsetlinewidth{1.505625pt}%
\definecolor{currentstroke}{rgb}{0.505882,0.447059,0.701961}%
\pgfsetstrokecolor{currentstroke}%
\pgfsetdash{}{0pt}%
\pgfpathmoveto{\pgfqpoint{0.341129in}{0.811188in}}%
\pgfpathlineto{\pgfqpoint{0.357885in}{0.816872in}}%
\pgfpathlineto{\pgfqpoint{0.374641in}{0.822556in}}%
\pgfpathlineto{\pgfqpoint{0.391398in}{0.828241in}}%
\pgfpathlineto{\pgfqpoint{0.408154in}{0.833925in}}%
\pgfpathlineto{\pgfqpoint{0.424910in}{0.839609in}}%
\pgfpathlineto{\pgfqpoint{0.441666in}{0.845294in}}%
\pgfpathlineto{\pgfqpoint{0.458423in}{0.850978in}}%
\pgfpathlineto{\pgfqpoint{0.475179in}{0.856662in}}%
\pgfpathlineto{\pgfqpoint{0.491935in}{0.862347in}}%
\pgfpathlineto{\pgfqpoint{0.508691in}{0.868031in}}%
\pgfpathlineto{\pgfqpoint{0.525448in}{0.873715in}}%
\pgfpathlineto{\pgfqpoint{0.542204in}{0.879400in}}%
\pgfpathlineto{\pgfqpoint{0.558960in}{0.885084in}}%
\pgfpathlineto{\pgfqpoint{0.575717in}{0.890768in}}%
\pgfpathlineto{\pgfqpoint{0.592473in}{0.896453in}}%
\pgfpathlineto{\pgfqpoint{0.609229in}{0.902137in}}%
\pgfpathlineto{\pgfqpoint{0.625985in}{0.907821in}}%
\pgfpathlineto{\pgfqpoint{0.642742in}{0.913506in}}%
\pgfpathlineto{\pgfqpoint{0.659498in}{0.919190in}}%
\pgfpathlineto{\pgfqpoint{0.676254in}{0.924874in}}%
\pgfpathlineto{\pgfqpoint{0.693011in}{0.930559in}}%
\pgfpathlineto{\pgfqpoint{0.709767in}{0.936243in}}%
\pgfpathlineto{\pgfqpoint{0.726523in}{0.941927in}}%
\pgfpathlineto{\pgfqpoint{0.743279in}{0.947612in}}%
\pgfpathlineto{\pgfqpoint{0.760036in}{0.953296in}}%
\pgfpathlineto{\pgfqpoint{0.776792in}{0.958980in}}%
\pgfpathlineto{\pgfqpoint{0.793548in}{0.964665in}}%
\pgfpathlineto{\pgfqpoint{0.810304in}{0.970349in}}%
\pgfpathlineto{\pgfqpoint{0.827061in}{0.976034in}}%
\pgfpathlineto{\pgfqpoint{0.843817in}{0.981718in}}%
\pgfpathlineto{\pgfqpoint{0.860573in}{0.987402in}}%
\pgfpathlineto{\pgfqpoint{0.877330in}{0.993087in}}%
\pgfpathlineto{\pgfqpoint{0.894086in}{0.998771in}}%
\pgfpathlineto{\pgfqpoint{0.910842in}{1.004455in}}%
\pgfpathlineto{\pgfqpoint{0.927598in}{1.010140in}}%
\pgfpathlineto{\pgfqpoint{0.944355in}{1.015824in}}%
\pgfpathlineto{\pgfqpoint{0.961111in}{1.021508in}}%
\pgfpathlineto{\pgfqpoint{0.977867in}{1.027193in}}%
\pgfpathlineto{\pgfqpoint{0.994623in}{1.032877in}}%
\pgfpathlineto{\pgfqpoint{1.011380in}{1.038561in}}%
\pgfpathlineto{\pgfqpoint{1.028136in}{1.044246in}}%
\pgfpathlineto{\pgfqpoint{1.044892in}{1.049930in}}%
\pgfpathlineto{\pgfqpoint{1.061649in}{1.055614in}}%
\pgfpathlineto{\pgfqpoint{1.078405in}{1.061299in}}%
\pgfpathlineto{\pgfqpoint{1.095161in}{1.066983in}}%
\pgfpathlineto{\pgfqpoint{1.111917in}{1.072667in}}%
\pgfpathlineto{\pgfqpoint{1.128674in}{1.078352in}}%
\pgfpathlineto{\pgfqpoint{1.145430in}{1.084036in}}%
\pgfpathlineto{\pgfqpoint{1.162186in}{1.089720in}}%
\pgfpathlineto{\pgfqpoint{1.178942in}{1.095405in}}%
\pgfpathlineto{\pgfqpoint{1.195699in}{1.101089in}}%
\pgfpathlineto{\pgfqpoint{1.212455in}{1.106773in}}%
\pgfpathlineto{\pgfqpoint{1.229211in}{1.112458in}}%
\pgfpathlineto{\pgfqpoint{1.245968in}{1.118142in}}%
\pgfpathlineto{\pgfqpoint{1.262724in}{1.123826in}}%
\pgfpathlineto{\pgfqpoint{1.279480in}{1.129511in}}%
\pgfpathlineto{\pgfqpoint{1.296236in}{1.135195in}}%
\pgfpathlineto{\pgfqpoint{1.312993in}{1.140879in}}%
\pgfpathlineto{\pgfqpoint{1.329749in}{1.146564in}}%
\pgfpathlineto{\pgfqpoint{1.346505in}{1.152248in}}%
\pgfpathlineto{\pgfqpoint{1.363262in}{1.157933in}}%
\pgfpathlineto{\pgfqpoint{1.380018in}{1.163617in}}%
\pgfpathlineto{\pgfqpoint{1.396774in}{1.169301in}}%
\pgfpathlineto{\pgfqpoint{1.413530in}{1.174986in}}%
\pgfpathlineto{\pgfqpoint{1.430287in}{1.180670in}}%
\pgfpathlineto{\pgfqpoint{1.447043in}{1.186354in}}%
\pgfpathlineto{\pgfqpoint{1.463799in}{1.192039in}}%
\pgfpathlineto{\pgfqpoint{1.480555in}{1.197723in}}%
\pgfpathlineto{\pgfqpoint{1.497312in}{1.203407in}}%
\pgfpathlineto{\pgfqpoint{1.514068in}{1.209092in}}%
\pgfpathlineto{\pgfqpoint{1.530824in}{1.214776in}}%
\pgfpathlineto{\pgfqpoint{1.547581in}{1.220460in}}%
\pgfpathlineto{\pgfqpoint{1.564337in}{1.226145in}}%
\pgfpathlineto{\pgfqpoint{1.581093in}{1.231829in}}%
\pgfpathlineto{\pgfqpoint{1.597849in}{1.237513in}}%
\pgfpathlineto{\pgfqpoint{1.614606in}{1.243198in}}%
\pgfpathlineto{\pgfqpoint{1.631362in}{1.248882in}}%
\pgfpathlineto{\pgfqpoint{1.648118in}{1.254566in}}%
\pgfpathlineto{\pgfqpoint{1.664874in}{1.260251in}}%
\pgfpathlineto{\pgfqpoint{1.681631in}{1.265935in}}%
\pgfpathlineto{\pgfqpoint{1.698387in}{1.271619in}}%
\pgfpathlineto{\pgfqpoint{1.715143in}{1.277304in}}%
\pgfpathlineto{\pgfqpoint{1.731900in}{1.282988in}}%
\pgfpathlineto{\pgfqpoint{1.748656in}{1.288672in}}%
\pgfpathlineto{\pgfqpoint{1.765412in}{1.294357in}}%
\pgfpathlineto{\pgfqpoint{1.782168in}{1.300041in}}%
\pgfpathlineto{\pgfqpoint{1.798925in}{1.305725in}}%
\pgfpathlineto{\pgfqpoint{1.815681in}{1.311410in}}%
\pgfpathlineto{\pgfqpoint{1.832437in}{1.317094in}}%
\pgfpathlineto{\pgfqpoint{1.849193in}{1.322778in}}%
\pgfpathlineto{\pgfqpoint{1.865950in}{1.328463in}}%
\pgfpathlineto{\pgfqpoint{1.882706in}{1.334147in}}%
\pgfpathlineto{\pgfqpoint{1.899462in}{1.339832in}}%
\pgfpathlineto{\pgfqpoint{1.916219in}{1.345516in}}%
\pgfpathlineto{\pgfqpoint{1.932975in}{1.351200in}}%
\pgfpathlineto{\pgfqpoint{1.949731in}{1.356885in}}%
\pgfpathlineto{\pgfqpoint{1.966487in}{1.362569in}}%
\pgfpathlineto{\pgfqpoint{1.983244in}{1.368253in}}%
\pgfpathlineto{\pgfqpoint{2.000000in}{1.373938in}}%
\pgfusepath{stroke}%
\end{pgfscope}%
\begin{pgfscope}%
\pgfsetrectcap%
\pgfsetmiterjoin%
\pgfsetlinewidth{0.752812pt}%
\definecolor{currentstroke}{rgb}{0.700000,0.700000,0.700000}%
\pgfsetstrokecolor{currentstroke}%
\pgfsetdash{}{0pt}%
\pgfpathmoveto{\pgfqpoint{0.341129in}{0.466613in}}%
\pgfpathlineto{\pgfqpoint{0.341129in}{2.178211in}}%
\pgfusepath{stroke}%
\end{pgfscope}%
\begin{pgfscope}%
\pgfsetrectcap%
\pgfsetmiterjoin%
\pgfsetlinewidth{0.752812pt}%
\definecolor{currentstroke}{rgb}{0.700000,0.700000,0.700000}%
\pgfsetstrokecolor{currentstroke}%
\pgfsetdash{}{0pt}%
\pgfpathmoveto{\pgfqpoint{0.341129in}{0.466613in}}%
\pgfpathlineto{\pgfqpoint{2.000000in}{0.466613in}}%
\pgfusepath{stroke}%
\end{pgfscope}%
\end{pgfpicture}%
\makeatother%
\endgroup%
}}  &
      \subfloat[\(\epsilon=0.1\)]{\resizebox{0.275\linewidth}{!}{%% Creator: Matplotlib, PGF backend
%%
%% To include the figure in your LaTeX document, write
%%   \input{<filename>.pgf}
%%
%% Make sure the required packages are loaded in your preamble
%%   \usepackage{pgf}
%%
%% and, on pdftex
%%   \usepackage[utf8]{inputenc}\DeclareUnicodeCharacter{2212}{-}
%%
%% or, on luatex and xetex
%%   \usepackage{unicode-math}
%%
%% Figures using additional raster images can only be included by \input if
%% they are in the same directory as the main LaTeX file. For loading figures
%% from other directories you can use the `import` package
%%   \usepackage{import}
%%
%% and then include the figures with
%%   \import{<path to file>}{<filename>.pgf}
%%
%% Matplotlib used the following preamble
%%   \usepackage[utf8]{inputenc}
%%   \usepackage[T1]{fontenc}
%%   \usepackage{amsmath}
%%   \newcommand*{\mat}[1]{\boldsymbol{#1}}
%%
\begingroup%
\makeatletter%
\begin{pgfpicture}%
\pgfpathrectangle{\pgfpointorigin}{\pgfqpoint{2.100000in}{2.278211in}}%
\pgfusepath{use as bounding box, clip}%
\begin{pgfscope}%
\pgfsetbuttcap%
\pgfsetmiterjoin%
\definecolor{currentfill}{rgb}{1.000000,1.000000,1.000000}%
\pgfsetfillcolor{currentfill}%
\pgfsetlinewidth{0.000000pt}%
\definecolor{currentstroke}{rgb}{1.000000,1.000000,1.000000}%
\pgfsetstrokecolor{currentstroke}%
\pgfsetstrokeopacity{0.000000}%
\pgfsetdash{}{0pt}%
\pgfpathmoveto{\pgfqpoint{0.000000in}{0.000000in}}%
\pgfpathlineto{\pgfqpoint{2.100000in}{0.000000in}}%
\pgfpathlineto{\pgfqpoint{2.100000in}{2.278211in}}%
\pgfpathlineto{\pgfqpoint{0.000000in}{2.278211in}}%
\pgfpathclose%
\pgfusepath{fill}%
\end{pgfscope}%
\begin{pgfscope}%
\pgfsetbuttcap%
\pgfsetmiterjoin%
\definecolor{currentfill}{rgb}{1.000000,1.000000,1.000000}%
\pgfsetfillcolor{currentfill}%
\pgfsetlinewidth{0.000000pt}%
\definecolor{currentstroke}{rgb}{0.000000,0.000000,0.000000}%
\pgfsetstrokecolor{currentstroke}%
\pgfsetstrokeopacity{0.000000}%
\pgfsetdash{}{0pt}%
\pgfpathmoveto{\pgfqpoint{0.341129in}{0.466613in}}%
\pgfpathlineto{\pgfqpoint{2.000000in}{0.466613in}}%
\pgfpathlineto{\pgfqpoint{2.000000in}{2.178211in}}%
\pgfpathlineto{\pgfqpoint{0.341129in}{2.178211in}}%
\pgfpathclose%
\pgfusepath{fill}%
\end{pgfscope}%
\begin{pgfscope}%
\pgfpathrectangle{\pgfqpoint{0.341129in}{0.466613in}}{\pgfqpoint{1.658871in}{1.711598in}}%
\pgfusepath{clip}%
\pgfsetroundcap%
\pgfsetroundjoin%
\pgfsetlinewidth{0.501875pt}%
\definecolor{currentstroke}{rgb}{0.800000,0.800000,0.800000}%
\pgfsetstrokecolor{currentstroke}%
\pgfsetdash{}{0pt}%
\pgfpathmoveto{\pgfqpoint{0.556536in}{0.466613in}}%
\pgfpathlineto{\pgfqpoint{0.556536in}{2.178211in}}%
\pgfusepath{stroke}%
\end{pgfscope}%
\begin{pgfscope}%
\definecolor{textcolor}{rgb}{0.150000,0.150000,0.150000}%
\pgfsetstrokecolor{textcolor}%
\pgfsetfillcolor{textcolor}%
\pgftext[x=0.556536in,y=0.376335in,,top]{\color{textcolor}\rmfamily\fontsize{8.000000}{9.600000}\selectfont \(\displaystyle {0.4}\)}%
\end{pgfscope}%
\begin{pgfscope}%
\pgfpathrectangle{\pgfqpoint{0.341129in}{0.466613in}}{\pgfqpoint{1.658871in}{1.711598in}}%
\pgfusepath{clip}%
\pgfsetroundcap%
\pgfsetroundjoin%
\pgfsetlinewidth{0.501875pt}%
\definecolor{currentstroke}{rgb}{0.800000,0.800000,0.800000}%
\pgfsetstrokecolor{currentstroke}%
\pgfsetdash{}{0pt}%
\pgfpathmoveto{\pgfqpoint{1.374840in}{0.466613in}}%
\pgfpathlineto{\pgfqpoint{1.374840in}{2.178211in}}%
\pgfusepath{stroke}%
\end{pgfscope}%
\begin{pgfscope}%
\definecolor{textcolor}{rgb}{0.150000,0.150000,0.150000}%
\pgfsetstrokecolor{textcolor}%
\pgfsetfillcolor{textcolor}%
\pgftext[x=1.374840in,y=0.376335in,,top]{\color{textcolor}\rmfamily\fontsize{8.000000}{9.600000}\selectfont \(\displaystyle {0.6}\)}%
\end{pgfscope}%
\begin{pgfscope}%
\definecolor{textcolor}{rgb}{0.150000,0.150000,0.150000}%
\pgfsetstrokecolor{textcolor}%
\pgfsetfillcolor{textcolor}%
\pgftext[x=1.170564in,y=0.222655in,,top]{\color{textcolor}\rmfamily\fontsize{10.000000}{12.000000}\selectfont Frac. train labels}%
\end{pgfscope}%
\begin{pgfscope}%
\pgfpathrectangle{\pgfqpoint{0.341129in}{0.466613in}}{\pgfqpoint{1.658871in}{1.711598in}}%
\pgfusepath{clip}%
\pgfsetroundcap%
\pgfsetroundjoin%
\pgfsetlinewidth{0.501875pt}%
\definecolor{currentstroke}{rgb}{0.800000,0.800000,0.800000}%
\pgfsetstrokecolor{currentstroke}%
\pgfsetdash{}{0pt}%
\pgfpathmoveto{\pgfqpoint{0.341129in}{0.760562in}}%
\pgfpathlineto{\pgfqpoint{2.000000in}{0.760562in}}%
\pgfusepath{stroke}%
\end{pgfscope}%
\begin{pgfscope}%
\definecolor{textcolor}{rgb}{0.150000,0.150000,0.150000}%
\pgfsetstrokecolor{textcolor}%
\pgfsetfillcolor{textcolor}%
\pgftext[x=0.100000in, y=0.722300in, left, base]{\color{textcolor}\rmfamily\fontsize{8.000000}{9.600000}\selectfont \(\displaystyle {0.2}\)}%
\end{pgfscope}%
\begin{pgfscope}%
\pgfpathrectangle{\pgfqpoint{0.341129in}{0.466613in}}{\pgfqpoint{1.658871in}{1.711598in}}%
\pgfusepath{clip}%
\pgfsetroundcap%
\pgfsetroundjoin%
\pgfsetlinewidth{0.501875pt}%
\definecolor{currentstroke}{rgb}{0.800000,0.800000,0.800000}%
\pgfsetstrokecolor{currentstroke}%
\pgfsetdash{}{0pt}%
\pgfpathmoveto{\pgfqpoint{0.341129in}{1.102988in}}%
\pgfpathlineto{\pgfqpoint{2.000000in}{1.102988in}}%
\pgfusepath{stroke}%
\end{pgfscope}%
\begin{pgfscope}%
\definecolor{textcolor}{rgb}{0.150000,0.150000,0.150000}%
\pgfsetstrokecolor{textcolor}%
\pgfsetfillcolor{textcolor}%
\pgftext[x=0.100000in, y=1.064726in, left, base]{\color{textcolor}\rmfamily\fontsize{8.000000}{9.600000}\selectfont \(\displaystyle {0.4}\)}%
\end{pgfscope}%
\begin{pgfscope}%
\pgfpathrectangle{\pgfqpoint{0.341129in}{0.466613in}}{\pgfqpoint{1.658871in}{1.711598in}}%
\pgfusepath{clip}%
\pgfsetroundcap%
\pgfsetroundjoin%
\pgfsetlinewidth{0.501875pt}%
\definecolor{currentstroke}{rgb}{0.800000,0.800000,0.800000}%
\pgfsetstrokecolor{currentstroke}%
\pgfsetdash{}{0pt}%
\pgfpathmoveto{\pgfqpoint{0.341129in}{1.445415in}}%
\pgfpathlineto{\pgfqpoint{2.000000in}{1.445415in}}%
\pgfusepath{stroke}%
\end{pgfscope}%
\begin{pgfscope}%
\definecolor{textcolor}{rgb}{0.150000,0.150000,0.150000}%
\pgfsetstrokecolor{textcolor}%
\pgfsetfillcolor{textcolor}%
\pgftext[x=0.100000in, y=1.407153in, left, base]{\color{textcolor}\rmfamily\fontsize{8.000000}{9.600000}\selectfont \(\displaystyle {0.6}\)}%
\end{pgfscope}%
\begin{pgfscope}%
\pgfpathrectangle{\pgfqpoint{0.341129in}{0.466613in}}{\pgfqpoint{1.658871in}{1.711598in}}%
\pgfusepath{clip}%
\pgfsetroundcap%
\pgfsetroundjoin%
\pgfsetlinewidth{0.501875pt}%
\definecolor{currentstroke}{rgb}{0.800000,0.800000,0.800000}%
\pgfsetstrokecolor{currentstroke}%
\pgfsetdash{}{0pt}%
\pgfpathmoveto{\pgfqpoint{0.341129in}{1.787841in}}%
\pgfpathlineto{\pgfqpoint{2.000000in}{1.787841in}}%
\pgfusepath{stroke}%
\end{pgfscope}%
\begin{pgfscope}%
\definecolor{textcolor}{rgb}{0.150000,0.150000,0.150000}%
\pgfsetstrokecolor{textcolor}%
\pgfsetfillcolor{textcolor}%
\pgftext[x=0.100000in, y=1.749579in, left, base]{\color{textcolor}\rmfamily\fontsize{8.000000}{9.600000}\selectfont \(\displaystyle {0.8}\)}%
\end{pgfscope}%
\begin{pgfscope}%
\pgfpathrectangle{\pgfqpoint{0.341129in}{0.466613in}}{\pgfqpoint{1.658871in}{1.711598in}}%
\pgfusepath{clip}%
\pgfsetroundcap%
\pgfsetroundjoin%
\pgfsetlinewidth{0.501875pt}%
\definecolor{currentstroke}{rgb}{0.800000,0.800000,0.800000}%
\pgfsetstrokecolor{currentstroke}%
\pgfsetdash{}{0pt}%
\pgfpathmoveto{\pgfqpoint{0.341129in}{2.130268in}}%
\pgfpathlineto{\pgfqpoint{2.000000in}{2.130268in}}%
\pgfusepath{stroke}%
\end{pgfscope}%
\begin{pgfscope}%
\definecolor{textcolor}{rgb}{0.150000,0.150000,0.150000}%
\pgfsetstrokecolor{textcolor}%
\pgfsetfillcolor{textcolor}%
\pgftext[x=0.100000in, y=2.092005in, left, base]{\color{textcolor}\rmfamily\fontsize{8.000000}{9.600000}\selectfont \(\displaystyle {1.0}\)}%
\end{pgfscope}%
\begin{pgfscope}%
\pgfpathrectangle{\pgfqpoint{0.341129in}{0.466613in}}{\pgfqpoint{1.658871in}{1.711598in}}%
\pgfusepath{clip}%
\pgfsetbuttcap%
\pgfsetroundjoin%
\definecolor{currentfill}{rgb}{0.298039,0.447059,0.690196}%
\pgfsetfillcolor{currentfill}%
\pgfsetfillopacity{0.250000}%
\pgfsetlinewidth{1.003750pt}%
\definecolor{currentstroke}{rgb}{0.298039,0.447059,0.690196}%
\pgfsetstrokecolor{currentstroke}%
\pgfsetstrokeopacity{0.250000}%
\pgfsetdash{}{0pt}%
\pgfsys@defobject{currentmarker}{\pgfqpoint{-0.017010in}{-0.017010in}}{\pgfqpoint{0.017010in}{0.017010in}}{%
\pgfpathmoveto{\pgfqpoint{0.000000in}{-0.017010in}}%
\pgfpathcurveto{\pgfqpoint{0.004511in}{-0.017010in}}{\pgfqpoint{0.008838in}{-0.015218in}}{\pgfqpoint{0.012028in}{-0.012028in}}%
\pgfpathcurveto{\pgfqpoint{0.015218in}{-0.008838in}}{\pgfqpoint{0.017010in}{-0.004511in}}{\pgfqpoint{0.017010in}{0.000000in}}%
\pgfpathcurveto{\pgfqpoint{0.017010in}{0.004511in}}{\pgfqpoint{0.015218in}{0.008838in}}{\pgfqpoint{0.012028in}{0.012028in}}%
\pgfpathcurveto{\pgfqpoint{0.008838in}{0.015218in}}{\pgfqpoint{0.004511in}{0.017010in}}{\pgfqpoint{0.000000in}{0.017010in}}%
\pgfpathcurveto{\pgfqpoint{-0.004511in}{0.017010in}}{\pgfqpoint{-0.008838in}{0.015218in}}{\pgfqpoint{-0.012028in}{0.012028in}}%
\pgfpathcurveto{\pgfqpoint{-0.015218in}{0.008838in}}{\pgfqpoint{-0.017010in}{0.004511in}}{\pgfqpoint{-0.017010in}{0.000000in}}%
\pgfpathcurveto{\pgfqpoint{-0.017010in}{-0.004511in}}{\pgfqpoint{-0.015218in}{-0.008838in}}{\pgfqpoint{-0.012028in}{-0.012028in}}%
\pgfpathcurveto{\pgfqpoint{-0.008838in}{-0.015218in}}{\pgfqpoint{-0.004511in}{-0.017010in}}{\pgfqpoint{0.000000in}{-0.017010in}}%
\pgfpathclose%
\pgfusepath{stroke,fill}%
}%
\begin{pgfscope}%
\pgfsys@transformshift{0.667456in}{2.035875in}%
\pgfsys@useobject{currentmarker}{}%
\end{pgfscope}%
\begin{pgfscope}%
\pgfsys@transformshift{0.872203in}{2.042281in}%
\pgfsys@useobject{currentmarker}{}%
\end{pgfscope}%
\begin{pgfscope}%
\pgfsys@transformshift{1.174456in}{2.010996in}%
\pgfsys@useobject{currentmarker}{}%
\end{pgfscope}%
\begin{pgfscope}%
\pgfsys@transformshift{1.594005in}{1.966564in}%
\pgfsys@useobject{currentmarker}{}%
\end{pgfscope}%
\begin{pgfscope}%
\pgfsys@transformshift{1.217413in}{1.841809in}%
\pgfsys@useobject{currentmarker}{}%
\end{pgfscope}%
\begin{pgfscope}%
\pgfsys@transformshift{0.553973in}{2.032105in}%
\pgfsys@useobject{currentmarker}{}%
\end{pgfscope}%
\begin{pgfscope}%
\pgfsys@transformshift{1.639332in}{1.789150in}%
\pgfsys@useobject{currentmarker}{}%
\end{pgfscope}%
\begin{pgfscope}%
\pgfsys@transformshift{0.709729in}{2.013799in}%
\pgfsys@useobject{currentmarker}{}%
\end{pgfscope}%
\begin{pgfscope}%
\pgfsys@transformshift{1.202352in}{2.043502in}%
\pgfsys@useobject{currentmarker}{}%
\end{pgfscope}%
\begin{pgfscope}%
\pgfsys@transformshift{0.671018in}{1.952380in}%
\pgfsys@useobject{currentmarker}{}%
\end{pgfscope}%
\begin{pgfscope}%
\pgfsys@transformshift{1.442003in}{2.005003in}%
\pgfsys@useobject{currentmarker}{}%
\end{pgfscope}%
\begin{pgfscope}%
\pgfsys@transformshift{1.398849in}{1.967933in}%
\pgfsys@useobject{currentmarker}{}%
\end{pgfscope}%
\begin{pgfscope}%
\pgfsys@transformshift{1.552752in}{1.820851in}%
\pgfsys@useobject{currentmarker}{}%
\end{pgfscope}%
\begin{pgfscope}%
\pgfsys@transformshift{1.177147in}{1.985848in}%
\pgfsys@useobject{currentmarker}{}%
\end{pgfscope}%
\begin{pgfscope}%
\pgfsys@transformshift{1.887294in}{2.081259in}%
\pgfsys@useobject{currentmarker}{}%
\end{pgfscope}%
\begin{pgfscope}%
\pgfsys@transformshift{1.188970in}{1.967924in}%
\pgfsys@useobject{currentmarker}{}%
\end{pgfscope}%
\begin{pgfscope}%
\pgfsys@transformshift{1.534509in}{1.967707in}%
\pgfsys@useobject{currentmarker}{}%
\end{pgfscope}%
\begin{pgfscope}%
\pgfsys@transformshift{0.588572in}{2.048374in}%
\pgfsys@useobject{currentmarker}{}%
\end{pgfscope}%
\begin{pgfscope}%
\pgfsys@transformshift{0.952296in}{1.996205in}%
\pgfsys@useobject{currentmarker}{}%
\end{pgfscope}%
\begin{pgfscope}%
\pgfsys@transformshift{0.705522in}{2.000965in}%
\pgfsys@useobject{currentmarker}{}%
\end{pgfscope}%
\begin{pgfscope}%
\pgfsys@transformshift{1.121384in}{1.916916in}%
\pgfsys@useobject{currentmarker}{}%
\end{pgfscope}%
\begin{pgfscope}%
\pgfsys@transformshift{1.103574in}{1.944434in}%
\pgfsys@useobject{currentmarker}{}%
\end{pgfscope}%
\begin{pgfscope}%
\pgfsys@transformshift{1.817005in}{1.984713in}%
\pgfsys@useobject{currentmarker}{}%
\end{pgfscope}%
\begin{pgfscope}%
\pgfsys@transformshift{0.496492in}{2.077933in}%
\pgfsys@useobject{currentmarker}{}%
\end{pgfscope}%
\begin{pgfscope}%
\pgfsys@transformshift{1.711311in}{1.930821in}%
\pgfsys@useobject{currentmarker}{}%
\end{pgfscope}%
\begin{pgfscope}%
\pgfsys@transformshift{1.750408in}{1.784080in}%
\pgfsys@useobject{currentmarker}{}%
\end{pgfscope}%
\begin{pgfscope}%
\pgfsys@transformshift{1.333406in}{1.889508in}%
\pgfsys@useobject{currentmarker}{}%
\end{pgfscope}%
\begin{pgfscope}%
\pgfsys@transformshift{1.724223in}{1.987631in}%
\pgfsys@useobject{currentmarker}{}%
\end{pgfscope}%
\begin{pgfscope}%
\pgfsys@transformshift{1.256999in}{1.940626in}%
\pgfsys@useobject{currentmarker}{}%
\end{pgfscope}%
\begin{pgfscope}%
\pgfsys@transformshift{1.047649in}{1.973380in}%
\pgfsys@useobject{currentmarker}{}%
\end{pgfscope}%
\begin{pgfscope}%
\pgfsys@transformshift{0.878614in}{1.996031in}%
\pgfsys@useobject{currentmarker}{}%
\end{pgfscope}%
\begin{pgfscope}%
\pgfsys@transformshift{0.761954in}{1.971328in}%
\pgfsys@useobject{currentmarker}{}%
\end{pgfscope}%
\begin{pgfscope}%
\pgfsys@transformshift{1.796268in}{1.954185in}%
\pgfsys@useobject{currentmarker}{}%
\end{pgfscope}%
\begin{pgfscope}%
\pgfsys@transformshift{0.726378in}{2.014989in}%
\pgfsys@useobject{currentmarker}{}%
\end{pgfscope}%
\begin{pgfscope}%
\pgfsys@transformshift{1.408196in}{1.952627in}%
\pgfsys@useobject{currentmarker}{}%
\end{pgfscope}%
\begin{pgfscope}%
\pgfsys@transformshift{0.612217in}{1.871137in}%
\pgfsys@useobject{currentmarker}{}%
\end{pgfscope}%
\begin{pgfscope}%
\pgfsys@transformshift{1.180902in}{2.011034in}%
\pgfsys@useobject{currentmarker}{}%
\end{pgfscope}%
\begin{pgfscope}%
\pgfsys@transformshift{0.627344in}{2.072413in}%
\pgfsys@useobject{currentmarker}{}%
\end{pgfscope}%
\begin{pgfscope}%
\pgfsys@transformshift{0.628282in}{1.969361in}%
\pgfsys@useobject{currentmarker}{}%
\end{pgfscope}%
\begin{pgfscope}%
\pgfsys@transformshift{0.437992in}{1.855177in}%
\pgfsys@useobject{currentmarker}{}%
\end{pgfscope}%
\begin{pgfscope}%
\pgfsys@transformshift{0.758691in}{2.000107in}%
\pgfsys@useobject{currentmarker}{}%
\end{pgfscope}%
\begin{pgfscope}%
\pgfsys@transformshift{1.614220in}{2.008431in}%
\pgfsys@useobject{currentmarker}{}%
\end{pgfscope}%
\begin{pgfscope}%
\pgfsys@transformshift{1.846563in}{2.028268in}%
\pgfsys@useobject{currentmarker}{}%
\end{pgfscope}%
\begin{pgfscope}%
\pgfsys@transformshift{1.460173in}{1.982802in}%
\pgfsys@useobject{currentmarker}{}%
\end{pgfscope}%
\begin{pgfscope}%
\pgfsys@transformshift{1.637112in}{1.985058in}%
\pgfsys@useobject{currentmarker}{}%
\end{pgfscope}%
\begin{pgfscope}%
\pgfsys@transformshift{1.615512in}{1.855242in}%
\pgfsys@useobject{currentmarker}{}%
\end{pgfscope}%
\begin{pgfscope}%
\pgfsys@transformshift{1.242112in}{2.039092in}%
\pgfsys@useobject{currentmarker}{}%
\end{pgfscope}%
\begin{pgfscope}%
\pgfsys@transformshift{0.688214in}{1.956434in}%
\pgfsys@useobject{currentmarker}{}%
\end{pgfscope}%
\begin{pgfscope}%
\pgfsys@transformshift{0.911315in}{1.824049in}%
\pgfsys@useobject{currentmarker}{}%
\end{pgfscope}%
\begin{pgfscope}%
\pgfsys@transformshift{0.551462in}{2.033957in}%
\pgfsys@useobject{currentmarker}{}%
\end{pgfscope}%
\begin{pgfscope}%
\pgfsys@transformshift{1.069973in}{1.981897in}%
\pgfsys@useobject{currentmarker}{}%
\end{pgfscope}%
\begin{pgfscope}%
\pgfsys@transformshift{1.211180in}{2.021388in}%
\pgfsys@useobject{currentmarker}{}%
\end{pgfscope}%
\begin{pgfscope}%
\pgfsys@transformshift{1.485757in}{2.043176in}%
\pgfsys@useobject{currentmarker}{}%
\end{pgfscope}%
\begin{pgfscope}%
\pgfsys@transformshift{0.743119in}{1.958459in}%
\pgfsys@useobject{currentmarker}{}%
\end{pgfscope}%
\begin{pgfscope}%
\pgfsys@transformshift{1.407385in}{1.864783in}%
\pgfsys@useobject{currentmarker}{}%
\end{pgfscope}%
\begin{pgfscope}%
\pgfsys@transformshift{1.618799in}{2.036452in}%
\pgfsys@useobject{currentmarker}{}%
\end{pgfscope}%
\begin{pgfscope}%
\pgfsys@transformshift{1.477965in}{1.912859in}%
\pgfsys@useobject{currentmarker}{}%
\end{pgfscope}%
\begin{pgfscope}%
\pgfsys@transformshift{1.156480in}{2.064257in}%
\pgfsys@useobject{currentmarker}{}%
\end{pgfscope}%
\begin{pgfscope}%
\pgfsys@transformshift{1.583657in}{1.917745in}%
\pgfsys@useobject{currentmarker}{}%
\end{pgfscope}%
\begin{pgfscope}%
\pgfsys@transformshift{1.440624in}{1.896500in}%
\pgfsys@useobject{currentmarker}{}%
\end{pgfscope}%
\begin{pgfscope}%
\pgfsys@transformshift{1.318935in}{1.892179in}%
\pgfsys@useobject{currentmarker}{}%
\end{pgfscope}%
\begin{pgfscope}%
\pgfsys@transformshift{0.547443in}{2.044435in}%
\pgfsys@useobject{currentmarker}{}%
\end{pgfscope}%
\begin{pgfscope}%
\pgfsys@transformshift{1.436824in}{1.932976in}%
\pgfsys@useobject{currentmarker}{}%
\end{pgfscope}%
\begin{pgfscope}%
\pgfsys@transformshift{1.658179in}{1.971618in}%
\pgfsys@useobject{currentmarker}{}%
\end{pgfscope}%
\begin{pgfscope}%
\pgfsys@transformshift{1.709942in}{1.974012in}%
\pgfsys@useobject{currentmarker}{}%
\end{pgfscope}%
\begin{pgfscope}%
\pgfsys@transformshift{1.638310in}{1.900355in}%
\pgfsys@useobject{currentmarker}{}%
\end{pgfscope}%
\begin{pgfscope}%
\pgfsys@transformshift{1.723779in}{1.985718in}%
\pgfsys@useobject{currentmarker}{}%
\end{pgfscope}%
\begin{pgfscope}%
\pgfsys@transformshift{1.124544in}{1.958656in}%
\pgfsys@useobject{currentmarker}{}%
\end{pgfscope}%
\begin{pgfscope}%
\pgfsys@transformshift{1.267442in}{1.986369in}%
\pgfsys@useobject{currentmarker}{}%
\end{pgfscope}%
\begin{pgfscope}%
\pgfsys@transformshift{0.562064in}{1.807168in}%
\pgfsys@useobject{currentmarker}{}%
\end{pgfscope}%
\begin{pgfscope}%
\pgfsys@transformshift{0.642722in}{1.997032in}%
\pgfsys@useobject{currentmarker}{}%
\end{pgfscope}%
\begin{pgfscope}%
\pgfsys@transformshift{1.150269in}{1.998319in}%
\pgfsys@useobject{currentmarker}{}%
\end{pgfscope}%
\begin{pgfscope}%
\pgfsys@transformshift{1.173281in}{1.973383in}%
\pgfsys@useobject{currentmarker}{}%
\end{pgfscope}%
\begin{pgfscope}%
\pgfsys@transformshift{1.891719in}{2.055160in}%
\pgfsys@useobject{currentmarker}{}%
\end{pgfscope}%
\begin{pgfscope}%
\pgfsys@transformshift{0.694241in}{2.016878in}%
\pgfsys@useobject{currentmarker}{}%
\end{pgfscope}%
\begin{pgfscope}%
\pgfsys@transformshift{0.790322in}{1.957694in}%
\pgfsys@useobject{currentmarker}{}%
\end{pgfscope}%
\begin{pgfscope}%
\pgfsys@transformshift{1.424550in}{1.892127in}%
\pgfsys@useobject{currentmarker}{}%
\end{pgfscope}%
\begin{pgfscope}%
\pgfsys@transformshift{0.384790in}{1.816499in}%
\pgfsys@useobject{currentmarker}{}%
\end{pgfscope}%
\begin{pgfscope}%
\pgfsys@transformshift{0.729116in}{2.024456in}%
\pgfsys@useobject{currentmarker}{}%
\end{pgfscope}%
\begin{pgfscope}%
\pgfsys@transformshift{1.217637in}{2.023329in}%
\pgfsys@useobject{currentmarker}{}%
\end{pgfscope}%
\begin{pgfscope}%
\pgfsys@transformshift{0.452540in}{2.016484in}%
\pgfsys@useobject{currentmarker}{}%
\end{pgfscope}%
\begin{pgfscope}%
\pgfsys@transformshift{1.493704in}{1.867922in}%
\pgfsys@useobject{currentmarker}{}%
\end{pgfscope}%
\begin{pgfscope}%
\pgfsys@transformshift{1.543302in}{1.970320in}%
\pgfsys@useobject{currentmarker}{}%
\end{pgfscope}%
\begin{pgfscope}%
\pgfsys@transformshift{1.218014in}{1.959428in}%
\pgfsys@useobject{currentmarker}{}%
\end{pgfscope}%
\begin{pgfscope}%
\pgfsys@transformshift{1.521861in}{1.935437in}%
\pgfsys@useobject{currentmarker}{}%
\end{pgfscope}%
\begin{pgfscope}%
\pgfsys@transformshift{1.764286in}{2.042226in}%
\pgfsys@useobject{currentmarker}{}%
\end{pgfscope}%
\begin{pgfscope}%
\pgfsys@transformshift{0.924889in}{2.062096in}%
\pgfsys@useobject{currentmarker}{}%
\end{pgfscope}%
\begin{pgfscope}%
\pgfsys@transformshift{1.222866in}{1.892581in}%
\pgfsys@useobject{currentmarker}{}%
\end{pgfscope}%
\begin{pgfscope}%
\pgfsys@transformshift{1.382222in}{1.949093in}%
\pgfsys@useobject{currentmarker}{}%
\end{pgfscope}%
\begin{pgfscope}%
\pgfsys@transformshift{1.643945in}{2.012089in}%
\pgfsys@useobject{currentmarker}{}%
\end{pgfscope}%
\begin{pgfscope}%
\pgfsys@transformshift{0.801834in}{1.971849in}%
\pgfsys@useobject{currentmarker}{}%
\end{pgfscope}%
\begin{pgfscope}%
\pgfsys@transformshift{1.258384in}{2.010911in}%
\pgfsys@useobject{currentmarker}{}%
\end{pgfscope}%
\begin{pgfscope}%
\pgfsys@transformshift{1.394783in}{1.962338in}%
\pgfsys@useobject{currentmarker}{}%
\end{pgfscope}%
\begin{pgfscope}%
\pgfsys@transformshift{0.341129in}{2.052191in}%
\pgfsys@useobject{currentmarker}{}%
\end{pgfscope}%
\begin{pgfscope}%
\pgfsys@transformshift{1.069581in}{2.028019in}%
\pgfsys@useobject{currentmarker}{}%
\end{pgfscope}%
\begin{pgfscope}%
\pgfsys@transformshift{0.789606in}{2.072199in}%
\pgfsys@useobject{currentmarker}{}%
\end{pgfscope}%
\begin{pgfscope}%
\pgfsys@transformshift{1.273287in}{1.953362in}%
\pgfsys@useobject{currentmarker}{}%
\end{pgfscope}%
\begin{pgfscope}%
\pgfsys@transformshift{0.646998in}{2.012984in}%
\pgfsys@useobject{currentmarker}{}%
\end{pgfscope}%
\begin{pgfscope}%
\pgfsys@transformshift{0.761065in}{2.052091in}%
\pgfsys@useobject{currentmarker}{}%
\end{pgfscope}%
\begin{pgfscope}%
\pgfsys@transformshift{0.628168in}{2.084150in}%
\pgfsys@useobject{currentmarker}{}%
\end{pgfscope}%
\begin{pgfscope}%
\pgfsys@transformshift{1.214800in}{1.942209in}%
\pgfsys@useobject{currentmarker}{}%
\end{pgfscope}%
\begin{pgfscope}%
\pgfsys@transformshift{1.051231in}{2.043807in}%
\pgfsys@useobject{currentmarker}{}%
\end{pgfscope}%
\begin{pgfscope}%
\pgfsys@transformshift{1.010738in}{1.971641in}%
\pgfsys@useobject{currentmarker}{}%
\end{pgfscope}%
\begin{pgfscope}%
\pgfsys@transformshift{1.387136in}{1.858032in}%
\pgfsys@useobject{currentmarker}{}%
\end{pgfscope}%
\begin{pgfscope}%
\pgfsys@transformshift{0.522501in}{1.823510in}%
\pgfsys@useobject{currentmarker}{}%
\end{pgfscope}%
\begin{pgfscope}%
\pgfsys@transformshift{1.737851in}{2.009240in}%
\pgfsys@useobject{currentmarker}{}%
\end{pgfscope}%
\begin{pgfscope}%
\pgfsys@transformshift{1.622475in}{1.938684in}%
\pgfsys@useobject{currentmarker}{}%
\end{pgfscope}%
\begin{pgfscope}%
\pgfsys@transformshift{0.776054in}{1.801209in}%
\pgfsys@useobject{currentmarker}{}%
\end{pgfscope}%
\begin{pgfscope}%
\pgfsys@transformshift{1.260864in}{1.836123in}%
\pgfsys@useobject{currentmarker}{}%
\end{pgfscope}%
\begin{pgfscope}%
\pgfsys@transformshift{1.587232in}{1.871691in}%
\pgfsys@useobject{currentmarker}{}%
\end{pgfscope}%
\begin{pgfscope}%
\pgfsys@transformshift{1.339693in}{1.946951in}%
\pgfsys@useobject{currentmarker}{}%
\end{pgfscope}%
\begin{pgfscope}%
\pgfsys@transformshift{1.448928in}{1.844276in}%
\pgfsys@useobject{currentmarker}{}%
\end{pgfscope}%
\begin{pgfscope}%
\pgfsys@transformshift{1.848941in}{1.901615in}%
\pgfsys@useobject{currentmarker}{}%
\end{pgfscope}%
\begin{pgfscope}%
\pgfsys@transformshift{1.645721in}{1.961354in}%
\pgfsys@useobject{currentmarker}{}%
\end{pgfscope}%
\begin{pgfscope}%
\pgfsys@transformshift{0.980835in}{2.030241in}%
\pgfsys@useobject{currentmarker}{}%
\end{pgfscope}%
\begin{pgfscope}%
\pgfsys@transformshift{0.704938in}{2.073114in}%
\pgfsys@useobject{currentmarker}{}%
\end{pgfscope}%
\begin{pgfscope}%
\pgfsys@transformshift{0.622306in}{2.021972in}%
\pgfsys@useobject{currentmarker}{}%
\end{pgfscope}%
\begin{pgfscope}%
\pgfsys@transformshift{1.213744in}{1.860749in}%
\pgfsys@useobject{currentmarker}{}%
\end{pgfscope}%
\begin{pgfscope}%
\pgfsys@transformshift{0.934075in}{2.034785in}%
\pgfsys@useobject{currentmarker}{}%
\end{pgfscope}%
\begin{pgfscope}%
\pgfsys@transformshift{1.307417in}{1.831774in}%
\pgfsys@useobject{currentmarker}{}%
\end{pgfscope}%
\begin{pgfscope}%
\pgfsys@transformshift{1.605349in}{2.032296in}%
\pgfsys@useobject{currentmarker}{}%
\end{pgfscope}%
\begin{pgfscope}%
\pgfsys@transformshift{0.750394in}{1.981511in}%
\pgfsys@useobject{currentmarker}{}%
\end{pgfscope}%
\begin{pgfscope}%
\pgfsys@transformshift{1.771258in}{2.018595in}%
\pgfsys@useobject{currentmarker}{}%
\end{pgfscope}%
\begin{pgfscope}%
\pgfsys@transformshift{1.939172in}{2.045244in}%
\pgfsys@useobject{currentmarker}{}%
\end{pgfscope}%
\begin{pgfscope}%
\pgfsys@transformshift{1.599583in}{2.009367in}%
\pgfsys@useobject{currentmarker}{}%
\end{pgfscope}%
\begin{pgfscope}%
\pgfsys@transformshift{0.881257in}{2.005636in}%
\pgfsys@useobject{currentmarker}{}%
\end{pgfscope}%
\begin{pgfscope}%
\pgfsys@transformshift{1.201768in}{2.031151in}%
\pgfsys@useobject{currentmarker}{}%
\end{pgfscope}%
\begin{pgfscope}%
\pgfsys@transformshift{1.244134in}{2.011878in}%
\pgfsys@useobject{currentmarker}{}%
\end{pgfscope}%
\begin{pgfscope}%
\pgfsys@transformshift{1.750150in}{2.019576in}%
\pgfsys@useobject{currentmarker}{}%
\end{pgfscope}%
\begin{pgfscope}%
\pgfsys@transformshift{1.541153in}{1.905183in}%
\pgfsys@useobject{currentmarker}{}%
\end{pgfscope}%
\begin{pgfscope}%
\pgfsys@transformshift{0.964663in}{1.878627in}%
\pgfsys@useobject{currentmarker}{}%
\end{pgfscope}%
\begin{pgfscope}%
\pgfsys@transformshift{0.818109in}{2.009373in}%
\pgfsys@useobject{currentmarker}{}%
\end{pgfscope}%
\begin{pgfscope}%
\pgfsys@transformshift{1.755264in}{2.000884in}%
\pgfsys@useobject{currentmarker}{}%
\end{pgfscope}%
\begin{pgfscope}%
\pgfsys@transformshift{1.450480in}{2.019334in}%
\pgfsys@useobject{currentmarker}{}%
\end{pgfscope}%
\begin{pgfscope}%
\pgfsys@transformshift{1.246595in}{2.002489in}%
\pgfsys@useobject{currentmarker}{}%
\end{pgfscope}%
\begin{pgfscope}%
\pgfsys@transformshift{1.110348in}{2.008811in}%
\pgfsys@useobject{currentmarker}{}%
\end{pgfscope}%
\begin{pgfscope}%
\pgfsys@transformshift{0.853154in}{1.983528in}%
\pgfsys@useobject{currentmarker}{}%
\end{pgfscope}%
\begin{pgfscope}%
\pgfsys@transformshift{0.803206in}{2.057190in}%
\pgfsys@useobject{currentmarker}{}%
\end{pgfscope}%
\begin{pgfscope}%
\pgfsys@transformshift{0.800584in}{2.018599in}%
\pgfsys@useobject{currentmarker}{}%
\end{pgfscope}%
\begin{pgfscope}%
\pgfsys@transformshift{0.828786in}{2.023846in}%
\pgfsys@useobject{currentmarker}{}%
\end{pgfscope}%
\begin{pgfscope}%
\pgfsys@transformshift{0.802472in}{2.029191in}%
\pgfsys@useobject{currentmarker}{}%
\end{pgfscope}%
\begin{pgfscope}%
\pgfsys@transformshift{0.778610in}{2.053129in}%
\pgfsys@useobject{currentmarker}{}%
\end{pgfscope}%
\begin{pgfscope}%
\pgfsys@transformshift{1.678430in}{1.959344in}%
\pgfsys@useobject{currentmarker}{}%
\end{pgfscope}%
\begin{pgfscope}%
\pgfsys@transformshift{1.678534in}{1.987809in}%
\pgfsys@useobject{currentmarker}{}%
\end{pgfscope}%
\begin{pgfscope}%
\pgfsys@transformshift{1.164510in}{1.976583in}%
\pgfsys@useobject{currentmarker}{}%
\end{pgfscope}%
\begin{pgfscope}%
\pgfsys@transformshift{1.533885in}{1.976644in}%
\pgfsys@useobject{currentmarker}{}%
\end{pgfscope}%
\begin{pgfscope}%
\pgfsys@transformshift{0.532239in}{2.020556in}%
\pgfsys@useobject{currentmarker}{}%
\end{pgfscope}%
\begin{pgfscope}%
\pgfsys@transformshift{1.059630in}{1.883704in}%
\pgfsys@useobject{currentmarker}{}%
\end{pgfscope}%
\begin{pgfscope}%
\pgfsys@transformshift{1.471016in}{1.918727in}%
\pgfsys@useobject{currentmarker}{}%
\end{pgfscope}%
\begin{pgfscope}%
\pgfsys@transformshift{0.764422in}{1.994218in}%
\pgfsys@useobject{currentmarker}{}%
\end{pgfscope}%
\begin{pgfscope}%
\pgfsys@transformshift{1.724626in}{1.937019in}%
\pgfsys@useobject{currentmarker}{}%
\end{pgfscope}%
\begin{pgfscope}%
\pgfsys@transformshift{1.557866in}{1.867583in}%
\pgfsys@useobject{currentmarker}{}%
\end{pgfscope}%
\begin{pgfscope}%
\pgfsys@transformshift{1.322665in}{1.838886in}%
\pgfsys@useobject{currentmarker}{}%
\end{pgfscope}%
\begin{pgfscope}%
\pgfsys@transformshift{1.043939in}{2.010955in}%
\pgfsys@useobject{currentmarker}{}%
\end{pgfscope}%
\begin{pgfscope}%
\pgfsys@transformshift{1.615078in}{2.054321in}%
\pgfsys@useobject{currentmarker}{}%
\end{pgfscope}%
\begin{pgfscope}%
\pgfsys@transformshift{1.467591in}{2.026797in}%
\pgfsys@useobject{currentmarker}{}%
\end{pgfscope}%
\begin{pgfscope}%
\pgfsys@transformshift{0.991846in}{1.964549in}%
\pgfsys@useobject{currentmarker}{}%
\end{pgfscope}%
\begin{pgfscope}%
\pgfsys@transformshift{0.697891in}{1.980901in}%
\pgfsys@useobject{currentmarker}{}%
\end{pgfscope}%
\begin{pgfscope}%
\pgfsys@transformshift{1.294350in}{2.035481in}%
\pgfsys@useobject{currentmarker}{}%
\end{pgfscope}%
\begin{pgfscope}%
\pgfsys@transformshift{1.409354in}{1.852883in}%
\pgfsys@useobject{currentmarker}{}%
\end{pgfscope}%
\begin{pgfscope}%
\pgfsys@transformshift{0.733011in}{2.080492in}%
\pgfsys@useobject{currentmarker}{}%
\end{pgfscope}%
\begin{pgfscope}%
\pgfsys@transformshift{0.634257in}{1.975921in}%
\pgfsys@useobject{currentmarker}{}%
\end{pgfscope}%
\begin{pgfscope}%
\pgfsys@transformshift{0.526826in}{2.035096in}%
\pgfsys@useobject{currentmarker}{}%
\end{pgfscope}%
\begin{pgfscope}%
\pgfsys@transformshift{1.225515in}{2.032092in}%
\pgfsys@useobject{currentmarker}{}%
\end{pgfscope}%
\begin{pgfscope}%
\pgfsys@transformshift{0.578368in}{1.917796in}%
\pgfsys@useobject{currentmarker}{}%
\end{pgfscope}%
\begin{pgfscope}%
\pgfsys@transformshift{1.515054in}{1.956431in}%
\pgfsys@useobject{currentmarker}{}%
\end{pgfscope}%
\begin{pgfscope}%
\pgfsys@transformshift{1.989373in}{1.985023in}%
\pgfsys@useobject{currentmarker}{}%
\end{pgfscope}%
\begin{pgfscope}%
\pgfsys@transformshift{1.360084in}{2.039967in}%
\pgfsys@useobject{currentmarker}{}%
\end{pgfscope}%
\begin{pgfscope}%
\pgfsys@transformshift{0.851541in}{2.018460in}%
\pgfsys@useobject{currentmarker}{}%
\end{pgfscope}%
\begin{pgfscope}%
\pgfsys@transformshift{1.688966in}{1.988929in}%
\pgfsys@useobject{currentmarker}{}%
\end{pgfscope}%
\begin{pgfscope}%
\pgfsys@transformshift{1.144955in}{1.978883in}%
\pgfsys@useobject{currentmarker}{}%
\end{pgfscope}%
\begin{pgfscope}%
\pgfsys@transformshift{1.681418in}{1.920333in}%
\pgfsys@useobject{currentmarker}{}%
\end{pgfscope}%
\begin{pgfscope}%
\pgfsys@transformshift{1.133992in}{2.012681in}%
\pgfsys@useobject{currentmarker}{}%
\end{pgfscope}%
\begin{pgfscope}%
\pgfsys@transformshift{1.480556in}{1.929319in}%
\pgfsys@useobject{currentmarker}{}%
\end{pgfscope}%
\begin{pgfscope}%
\pgfsys@transformshift{1.339321in}{1.978767in}%
\pgfsys@useobject{currentmarker}{}%
\end{pgfscope}%
\begin{pgfscope}%
\pgfsys@transformshift{0.874156in}{1.966411in}%
\pgfsys@useobject{currentmarker}{}%
\end{pgfscope}%
\begin{pgfscope}%
\pgfsys@transformshift{1.592077in}{1.890328in}%
\pgfsys@useobject{currentmarker}{}%
\end{pgfscope}%
\begin{pgfscope}%
\pgfsys@transformshift{1.288980in}{1.964850in}%
\pgfsys@useobject{currentmarker}{}%
\end{pgfscope}%
\begin{pgfscope}%
\pgfsys@transformshift{0.987975in}{1.922019in}%
\pgfsys@useobject{currentmarker}{}%
\end{pgfscope}%
\begin{pgfscope}%
\pgfsys@transformshift{1.714750in}{1.985448in}%
\pgfsys@useobject{currentmarker}{}%
\end{pgfscope}%
\begin{pgfscope}%
\pgfsys@transformshift{1.365090in}{1.944886in}%
\pgfsys@useobject{currentmarker}{}%
\end{pgfscope}%
\begin{pgfscope}%
\pgfsys@transformshift{1.930143in}{2.010817in}%
\pgfsys@useobject{currentmarker}{}%
\end{pgfscope}%
\begin{pgfscope}%
\pgfsys@transformshift{1.850728in}{1.934212in}%
\pgfsys@useobject{currentmarker}{}%
\end{pgfscope}%
\begin{pgfscope}%
\pgfsys@transformshift{0.856167in}{2.053577in}%
\pgfsys@useobject{currentmarker}{}%
\end{pgfscope}%
\begin{pgfscope}%
\pgfsys@transformshift{1.724113in}{2.000401in}%
\pgfsys@useobject{currentmarker}{}%
\end{pgfscope}%
\begin{pgfscope}%
\pgfsys@transformshift{0.746656in}{2.017992in}%
\pgfsys@useobject{currentmarker}{}%
\end{pgfscope}%
\begin{pgfscope}%
\pgfsys@transformshift{0.981125in}{2.048400in}%
\pgfsys@useobject{currentmarker}{}%
\end{pgfscope}%
\begin{pgfscope}%
\pgfsys@transformshift{0.840360in}{1.968796in}%
\pgfsys@useobject{currentmarker}{}%
\end{pgfscope}%
\begin{pgfscope}%
\pgfsys@transformshift{0.780491in}{1.991051in}%
\pgfsys@useobject{currentmarker}{}%
\end{pgfscope}%
\begin{pgfscope}%
\pgfsys@transformshift{1.873816in}{2.031624in}%
\pgfsys@useobject{currentmarker}{}%
\end{pgfscope}%
\begin{pgfscope}%
\pgfsys@transformshift{1.594601in}{1.963267in}%
\pgfsys@useobject{currentmarker}{}%
\end{pgfscope}%
\begin{pgfscope}%
\pgfsys@transformshift{0.815736in}{1.974070in}%
\pgfsys@useobject{currentmarker}{}%
\end{pgfscope}%
\begin{pgfscope}%
\pgfsys@transformshift{1.864796in}{1.889290in}%
\pgfsys@useobject{currentmarker}{}%
\end{pgfscope}%
\begin{pgfscope}%
\pgfsys@transformshift{1.367631in}{2.002839in}%
\pgfsys@useobject{currentmarker}{}%
\end{pgfscope}%
\begin{pgfscope}%
\pgfsys@transformshift{1.653345in}{1.976238in}%
\pgfsys@useobject{currentmarker}{}%
\end{pgfscope}%
\begin{pgfscope}%
\pgfsys@transformshift{1.768545in}{1.998542in}%
\pgfsys@useobject{currentmarker}{}%
\end{pgfscope}%
\begin{pgfscope}%
\pgfsys@transformshift{1.244730in}{1.976915in}%
\pgfsys@useobject{currentmarker}{}%
\end{pgfscope}%
\begin{pgfscope}%
\pgfsys@transformshift{0.782047in}{1.965747in}%
\pgfsys@useobject{currentmarker}{}%
\end{pgfscope}%
\begin{pgfscope}%
\pgfsys@transformshift{1.846677in}{2.013639in}%
\pgfsys@useobject{currentmarker}{}%
\end{pgfscope}%
\begin{pgfscope}%
\pgfsys@transformshift{1.775739in}{2.004851in}%
\pgfsys@useobject{currentmarker}{}%
\end{pgfscope}%
\begin{pgfscope}%
\pgfsys@transformshift{1.703721in}{2.033458in}%
\pgfsys@useobject{currentmarker}{}%
\end{pgfscope}%
\begin{pgfscope}%
\pgfsys@transformshift{0.834121in}{2.013800in}%
\pgfsys@useobject{currentmarker}{}%
\end{pgfscope}%
\begin{pgfscope}%
\pgfsys@transformshift{1.119982in}{2.025531in}%
\pgfsys@useobject{currentmarker}{}%
\end{pgfscope}%
\begin{pgfscope}%
\pgfsys@transformshift{1.891923in}{1.864022in}%
\pgfsys@useobject{currentmarker}{}%
\end{pgfscope}%
\begin{pgfscope}%
\pgfsys@transformshift{1.690465in}{1.827491in}%
\pgfsys@useobject{currentmarker}{}%
\end{pgfscope}%
\begin{pgfscope}%
\pgfsys@transformshift{0.559984in}{2.051688in}%
\pgfsys@useobject{currentmarker}{}%
\end{pgfscope}%
\begin{pgfscope}%
\pgfsys@transformshift{1.447613in}{1.961095in}%
\pgfsys@useobject{currentmarker}{}%
\end{pgfscope}%
\begin{pgfscope}%
\pgfsys@transformshift{0.897928in}{2.051758in}%
\pgfsys@useobject{currentmarker}{}%
\end{pgfscope}%
\begin{pgfscope}%
\pgfsys@transformshift{1.417775in}{1.956472in}%
\pgfsys@useobject{currentmarker}{}%
\end{pgfscope}%
\begin{pgfscope}%
\pgfsys@transformshift{1.673252in}{1.920491in}%
\pgfsys@useobject{currentmarker}{}%
\end{pgfscope}%
\begin{pgfscope}%
\pgfsys@transformshift{1.412143in}{1.804968in}%
\pgfsys@useobject{currentmarker}{}%
\end{pgfscope}%
\begin{pgfscope}%
\pgfsys@transformshift{1.641912in}{1.972687in}%
\pgfsys@useobject{currentmarker}{}%
\end{pgfscope}%
\begin{pgfscope}%
\pgfsys@transformshift{1.388318in}{2.084452in}%
\pgfsys@useobject{currentmarker}{}%
\end{pgfscope}%
\begin{pgfscope}%
\pgfsys@transformshift{1.495487in}{1.940031in}%
\pgfsys@useobject{currentmarker}{}%
\end{pgfscope}%
\begin{pgfscope}%
\pgfsys@transformshift{1.918935in}{1.923742in}%
\pgfsys@useobject{currentmarker}{}%
\end{pgfscope}%
\begin{pgfscope}%
\pgfsys@transformshift{0.550025in}{1.946524in}%
\pgfsys@useobject{currentmarker}{}%
\end{pgfscope}%
\begin{pgfscope}%
\pgfsys@transformshift{1.821318in}{1.937529in}%
\pgfsys@useobject{currentmarker}{}%
\end{pgfscope}%
\begin{pgfscope}%
\pgfsys@transformshift{0.956955in}{1.919592in}%
\pgfsys@useobject{currentmarker}{}%
\end{pgfscope}%
\begin{pgfscope}%
\pgfsys@transformshift{0.895984in}{2.020622in}%
\pgfsys@useobject{currentmarker}{}%
\end{pgfscope}%
\begin{pgfscope}%
\pgfsys@transformshift{0.467116in}{1.876557in}%
\pgfsys@useobject{currentmarker}{}%
\end{pgfscope}%
\begin{pgfscope}%
\pgfsys@transformshift{1.606478in}{1.982610in}%
\pgfsys@useobject{currentmarker}{}%
\end{pgfscope}%
\begin{pgfscope}%
\pgfsys@transformshift{1.423682in}{1.985646in}%
\pgfsys@useobject{currentmarker}{}%
\end{pgfscope}%
\begin{pgfscope}%
\pgfsys@transformshift{1.436312in}{1.929366in}%
\pgfsys@useobject{currentmarker}{}%
\end{pgfscope}%
\begin{pgfscope}%
\pgfsys@transformshift{0.970257in}{2.080435in}%
\pgfsys@useobject{currentmarker}{}%
\end{pgfscope}%
\begin{pgfscope}%
\pgfsys@transformshift{1.482495in}{1.977282in}%
\pgfsys@useobject{currentmarker}{}%
\end{pgfscope}%
\begin{pgfscope}%
\pgfsys@transformshift{1.363658in}{2.026629in}%
\pgfsys@useobject{currentmarker}{}%
\end{pgfscope}%
\begin{pgfscope}%
\pgfsys@transformshift{1.566350in}{2.003002in}%
\pgfsys@useobject{currentmarker}{}%
\end{pgfscope}%
\begin{pgfscope}%
\pgfsys@transformshift{0.737168in}{1.933823in}%
\pgfsys@useobject{currentmarker}{}%
\end{pgfscope}%
\begin{pgfscope}%
\pgfsys@transformshift{0.416424in}{1.843439in}%
\pgfsys@useobject{currentmarker}{}%
\end{pgfscope}%
\begin{pgfscope}%
\pgfsys@transformshift{0.524334in}{1.953291in}%
\pgfsys@useobject{currentmarker}{}%
\end{pgfscope}%
\begin{pgfscope}%
\pgfsys@transformshift{0.739043in}{1.834715in}%
\pgfsys@useobject{currentmarker}{}%
\end{pgfscope}%
\begin{pgfscope}%
\pgfsys@transformshift{1.840100in}{1.976559in}%
\pgfsys@useobject{currentmarker}{}%
\end{pgfscope}%
\begin{pgfscope}%
\pgfsys@transformshift{1.824568in}{2.039586in}%
\pgfsys@useobject{currentmarker}{}%
\end{pgfscope}%
\begin{pgfscope}%
\pgfsys@transformshift{1.470009in}{1.961142in}%
\pgfsys@useobject{currentmarker}{}%
\end{pgfscope}%
\begin{pgfscope}%
\pgfsys@transformshift{0.471982in}{1.980366in}%
\pgfsys@useobject{currentmarker}{}%
\end{pgfscope}%
\begin{pgfscope}%
\pgfsys@transformshift{1.573451in}{2.074014in}%
\pgfsys@useobject{currentmarker}{}%
\end{pgfscope}%
\begin{pgfscope}%
\pgfsys@transformshift{1.738247in}{2.011605in}%
\pgfsys@useobject{currentmarker}{}%
\end{pgfscope}%
\begin{pgfscope}%
\pgfsys@transformshift{1.607428in}{1.860711in}%
\pgfsys@useobject{currentmarker}{}%
\end{pgfscope}%
\begin{pgfscope}%
\pgfsys@transformshift{1.759927in}{1.882994in}%
\pgfsys@useobject{currentmarker}{}%
\end{pgfscope}%
\begin{pgfscope}%
\pgfsys@transformshift{1.551624in}{1.858468in}%
\pgfsys@useobject{currentmarker}{}%
\end{pgfscope}%
\begin{pgfscope}%
\pgfsys@transformshift{1.356607in}{1.968746in}%
\pgfsys@useobject{currentmarker}{}%
\end{pgfscope}%
\begin{pgfscope}%
\pgfsys@transformshift{1.649686in}{1.758957in}%
\pgfsys@useobject{currentmarker}{}%
\end{pgfscope}%
\begin{pgfscope}%
\pgfsys@transformshift{1.305691in}{1.894487in}%
\pgfsys@useobject{currentmarker}{}%
\end{pgfscope}%
\begin{pgfscope}%
\pgfsys@transformshift{1.393587in}{1.890460in}%
\pgfsys@useobject{currentmarker}{}%
\end{pgfscope}%
\begin{pgfscope}%
\pgfsys@transformshift{1.153074in}{1.930369in}%
\pgfsys@useobject{currentmarker}{}%
\end{pgfscope}%
\begin{pgfscope}%
\pgfsys@transformshift{1.150217in}{2.025208in}%
\pgfsys@useobject{currentmarker}{}%
\end{pgfscope}%
\begin{pgfscope}%
\pgfsys@transformshift{0.541563in}{1.788124in}%
\pgfsys@useobject{currentmarker}{}%
\end{pgfscope}%
\begin{pgfscope}%
\pgfsys@transformshift{0.907947in}{1.987734in}%
\pgfsys@useobject{currentmarker}{}%
\end{pgfscope}%
\begin{pgfscope}%
\pgfsys@transformshift{1.404883in}{1.959504in}%
\pgfsys@useobject{currentmarker}{}%
\end{pgfscope}%
\begin{pgfscope}%
\pgfsys@transformshift{1.766444in}{1.908392in}%
\pgfsys@useobject{currentmarker}{}%
\end{pgfscope}%
\begin{pgfscope}%
\pgfsys@transformshift{1.727091in}{2.026270in}%
\pgfsys@useobject{currentmarker}{}%
\end{pgfscope}%
\begin{pgfscope}%
\pgfsys@transformshift{1.794191in}{2.016063in}%
\pgfsys@useobject{currentmarker}{}%
\end{pgfscope}%
\begin{pgfscope}%
\pgfsys@transformshift{1.006287in}{1.972211in}%
\pgfsys@useobject{currentmarker}{}%
\end{pgfscope}%
\begin{pgfscope}%
\pgfsys@transformshift{1.581569in}{1.962199in}%
\pgfsys@useobject{currentmarker}{}%
\end{pgfscope}%
\begin{pgfscope}%
\pgfsys@transformshift{0.941329in}{1.998790in}%
\pgfsys@useobject{currentmarker}{}%
\end{pgfscope}%
\begin{pgfscope}%
\pgfsys@transformshift{0.691952in}{2.004460in}%
\pgfsys@useobject{currentmarker}{}%
\end{pgfscope}%
\begin{pgfscope}%
\pgfsys@transformshift{1.057616in}{2.011428in}%
\pgfsys@useobject{currentmarker}{}%
\end{pgfscope}%
\begin{pgfscope}%
\pgfsys@transformshift{1.673517in}{1.967929in}%
\pgfsys@useobject{currentmarker}{}%
\end{pgfscope}%
\begin{pgfscope}%
\pgfsys@transformshift{1.477276in}{2.031057in}%
\pgfsys@useobject{currentmarker}{}%
\end{pgfscope}%
\begin{pgfscope}%
\pgfsys@transformshift{1.418685in}{1.990888in}%
\pgfsys@useobject{currentmarker}{}%
\end{pgfscope}%
\begin{pgfscope}%
\pgfsys@transformshift{1.590325in}{1.854848in}%
\pgfsys@useobject{currentmarker}{}%
\end{pgfscope}%
\begin{pgfscope}%
\pgfsys@transformshift{0.779671in}{2.011920in}%
\pgfsys@useobject{currentmarker}{}%
\end{pgfscope}%
\begin{pgfscope}%
\pgfsys@transformshift{1.674297in}{1.922485in}%
\pgfsys@useobject{currentmarker}{}%
\end{pgfscope}%
\begin{pgfscope}%
\pgfsys@transformshift{0.676244in}{2.036213in}%
\pgfsys@useobject{currentmarker}{}%
\end{pgfscope}%
\begin{pgfscope}%
\pgfsys@transformshift{1.715152in}{1.950029in}%
\pgfsys@useobject{currentmarker}{}%
\end{pgfscope}%
\begin{pgfscope}%
\pgfsys@transformshift{0.491079in}{1.855380in}%
\pgfsys@useobject{currentmarker}{}%
\end{pgfscope}%
\begin{pgfscope}%
\pgfsys@transformshift{1.084356in}{2.025881in}%
\pgfsys@useobject{currentmarker}{}%
\end{pgfscope}%
\begin{pgfscope}%
\pgfsys@transformshift{0.525316in}{1.858535in}%
\pgfsys@useobject{currentmarker}{}%
\end{pgfscope}%
\begin{pgfscope}%
\pgfsys@transformshift{0.530939in}{1.868614in}%
\pgfsys@useobject{currentmarker}{}%
\end{pgfscope}%
\begin{pgfscope}%
\pgfsys@transformshift{0.632029in}{2.033503in}%
\pgfsys@useobject{currentmarker}{}%
\end{pgfscope}%
\begin{pgfscope}%
\pgfsys@transformshift{0.952115in}{2.028166in}%
\pgfsys@useobject{currentmarker}{}%
\end{pgfscope}%
\begin{pgfscope}%
\pgfsys@transformshift{1.781361in}{1.866622in}%
\pgfsys@useobject{currentmarker}{}%
\end{pgfscope}%
\begin{pgfscope}%
\pgfsys@transformshift{0.893435in}{1.975941in}%
\pgfsys@useobject{currentmarker}{}%
\end{pgfscope}%
\begin{pgfscope}%
\pgfsys@transformshift{1.630080in}{1.873845in}%
\pgfsys@useobject{currentmarker}{}%
\end{pgfscope}%
\begin{pgfscope}%
\pgfsys@transformshift{1.495001in}{1.965924in}%
\pgfsys@useobject{currentmarker}{}%
\end{pgfscope}%
\begin{pgfscope}%
\pgfsys@transformshift{1.807197in}{1.972779in}%
\pgfsys@useobject{currentmarker}{}%
\end{pgfscope}%
\begin{pgfscope}%
\pgfsys@transformshift{0.872166in}{2.013721in}%
\pgfsys@useobject{currentmarker}{}%
\end{pgfscope}%
\begin{pgfscope}%
\pgfsys@transformshift{1.447090in}{2.042646in}%
\pgfsys@useobject{currentmarker}{}%
\end{pgfscope}%
\begin{pgfscope}%
\pgfsys@transformshift{1.534740in}{1.990806in}%
\pgfsys@useobject{currentmarker}{}%
\end{pgfscope}%
\begin{pgfscope}%
\pgfsys@transformshift{0.648228in}{2.087689in}%
\pgfsys@useobject{currentmarker}{}%
\end{pgfscope}%
\begin{pgfscope}%
\pgfsys@transformshift{0.939750in}{1.868969in}%
\pgfsys@useobject{currentmarker}{}%
\end{pgfscope}%
\begin{pgfscope}%
\pgfsys@transformshift{0.437103in}{1.885886in}%
\pgfsys@useobject{currentmarker}{}%
\end{pgfscope}%
\begin{pgfscope}%
\pgfsys@transformshift{1.534965in}{1.970276in}%
\pgfsys@useobject{currentmarker}{}%
\end{pgfscope}%
\begin{pgfscope}%
\pgfsys@transformshift{1.227246in}{2.039510in}%
\pgfsys@useobject{currentmarker}{}%
\end{pgfscope}%
\begin{pgfscope}%
\pgfsys@transformshift{1.816498in}{1.943929in}%
\pgfsys@useobject{currentmarker}{}%
\end{pgfscope}%
\begin{pgfscope}%
\pgfsys@transformshift{0.733769in}{2.025934in}%
\pgfsys@useobject{currentmarker}{}%
\end{pgfscope}%
\begin{pgfscope}%
\pgfsys@transformshift{0.664484in}{1.977443in}%
\pgfsys@useobject{currentmarker}{}%
\end{pgfscope}%
\begin{pgfscope}%
\pgfsys@transformshift{1.076644in}{2.019628in}%
\pgfsys@useobject{currentmarker}{}%
\end{pgfscope}%
\begin{pgfscope}%
\pgfsys@transformshift{1.548521in}{2.022772in}%
\pgfsys@useobject{currentmarker}{}%
\end{pgfscope}%
\begin{pgfscope}%
\pgfsys@transformshift{0.536243in}{2.016107in}%
\pgfsys@useobject{currentmarker}{}%
\end{pgfscope}%
\begin{pgfscope}%
\pgfsys@transformshift{0.889736in}{2.027617in}%
\pgfsys@useobject{currentmarker}{}%
\end{pgfscope}%
\begin{pgfscope}%
\pgfsys@transformshift{1.818644in}{1.979579in}%
\pgfsys@useobject{currentmarker}{}%
\end{pgfscope}%
\begin{pgfscope}%
\pgfsys@transformshift{1.073205in}{2.076142in}%
\pgfsys@useobject{currentmarker}{}%
\end{pgfscope}%
\begin{pgfscope}%
\pgfsys@transformshift{0.882269in}{2.036986in}%
\pgfsys@useobject{currentmarker}{}%
\end{pgfscope}%
\begin{pgfscope}%
\pgfsys@transformshift{0.755753in}{2.014193in}%
\pgfsys@useobject{currentmarker}{}%
\end{pgfscope}%
\begin{pgfscope}%
\pgfsys@transformshift{1.318495in}{1.982712in}%
\pgfsys@useobject{currentmarker}{}%
\end{pgfscope}%
\begin{pgfscope}%
\pgfsys@transformshift{1.102293in}{1.985571in}%
\pgfsys@useobject{currentmarker}{}%
\end{pgfscope}%
\begin{pgfscope}%
\pgfsys@transformshift{1.814698in}{2.040895in}%
\pgfsys@useobject{currentmarker}{}%
\end{pgfscope}%
\begin{pgfscope}%
\pgfsys@transformshift{1.595111in}{1.892284in}%
\pgfsys@useobject{currentmarker}{}%
\end{pgfscope}%
\begin{pgfscope}%
\pgfsys@transformshift{0.726840in}{1.971439in}%
\pgfsys@useobject{currentmarker}{}%
\end{pgfscope}%
\begin{pgfscope}%
\pgfsys@transformshift{1.492602in}{1.888968in}%
\pgfsys@useobject{currentmarker}{}%
\end{pgfscope}%
\begin{pgfscope}%
\pgfsys@transformshift{1.393179in}{1.825974in}%
\pgfsys@useobject{currentmarker}{}%
\end{pgfscope}%
\begin{pgfscope}%
\pgfsys@transformshift{1.166943in}{1.862801in}%
\pgfsys@useobject{currentmarker}{}%
\end{pgfscope}%
\begin{pgfscope}%
\pgfsys@transformshift{1.444452in}{1.817317in}%
\pgfsys@useobject{currentmarker}{}%
\end{pgfscope}%
\begin{pgfscope}%
\pgfsys@transformshift{1.693005in}{1.991567in}%
\pgfsys@useobject{currentmarker}{}%
\end{pgfscope}%
\begin{pgfscope}%
\pgfsys@transformshift{0.817674in}{1.887769in}%
\pgfsys@useobject{currentmarker}{}%
\end{pgfscope}%
\begin{pgfscope}%
\pgfsys@transformshift{0.521183in}{2.044231in}%
\pgfsys@useobject{currentmarker}{}%
\end{pgfscope}%
\begin{pgfscope}%
\pgfsys@transformshift{1.298494in}{2.014571in}%
\pgfsys@useobject{currentmarker}{}%
\end{pgfscope}%
\begin{pgfscope}%
\pgfsys@transformshift{0.537740in}{1.977051in}%
\pgfsys@useobject{currentmarker}{}%
\end{pgfscope}%
\begin{pgfscope}%
\pgfsys@transformshift{1.711514in}{2.013045in}%
\pgfsys@useobject{currentmarker}{}%
\end{pgfscope}%
\begin{pgfscope}%
\pgfsys@transformshift{1.607575in}{2.004363in}%
\pgfsys@useobject{currentmarker}{}%
\end{pgfscope}%
\begin{pgfscope}%
\pgfsys@transformshift{1.569340in}{1.994910in}%
\pgfsys@useobject{currentmarker}{}%
\end{pgfscope}%
\begin{pgfscope}%
\pgfsys@transformshift{0.423151in}{1.995046in}%
\pgfsys@useobject{currentmarker}{}%
\end{pgfscope}%
\begin{pgfscope}%
\pgfsys@transformshift{0.636198in}{1.787065in}%
\pgfsys@useobject{currentmarker}{}%
\end{pgfscope}%
\begin{pgfscope}%
\pgfsys@transformshift{1.742823in}{1.886487in}%
\pgfsys@useobject{currentmarker}{}%
\end{pgfscope}%
\begin{pgfscope}%
\pgfsys@transformshift{0.849878in}{1.993709in}%
\pgfsys@useobject{currentmarker}{}%
\end{pgfscope}%
\begin{pgfscope}%
\pgfsys@transformshift{1.353468in}{1.982389in}%
\pgfsys@useobject{currentmarker}{}%
\end{pgfscope}%
\begin{pgfscope}%
\pgfsys@transformshift{1.622982in}{1.683863in}%
\pgfsys@useobject{currentmarker}{}%
\end{pgfscope}%
\begin{pgfscope}%
\pgfsys@transformshift{1.599783in}{1.791169in}%
\pgfsys@useobject{currentmarker}{}%
\end{pgfscope}%
\begin{pgfscope}%
\pgfsys@transformshift{1.895187in}{1.916029in}%
\pgfsys@useobject{currentmarker}{}%
\end{pgfscope}%
\begin{pgfscope}%
\pgfsys@transformshift{1.129530in}{2.036051in}%
\pgfsys@useobject{currentmarker}{}%
\end{pgfscope}%
\begin{pgfscope}%
\pgfsys@transformshift{0.991382in}{1.979793in}%
\pgfsys@useobject{currentmarker}{}%
\end{pgfscope}%
\begin{pgfscope}%
\pgfsys@transformshift{0.795664in}{2.100411in}%
\pgfsys@useobject{currentmarker}{}%
\end{pgfscope}%
\begin{pgfscope}%
\pgfsys@transformshift{1.780025in}{1.987209in}%
\pgfsys@useobject{currentmarker}{}%
\end{pgfscope}%
\begin{pgfscope}%
\pgfsys@transformshift{1.712749in}{2.043224in}%
\pgfsys@useobject{currentmarker}{}%
\end{pgfscope}%
\begin{pgfscope}%
\pgfsys@transformshift{1.145211in}{2.053956in}%
\pgfsys@useobject{currentmarker}{}%
\end{pgfscope}%
\begin{pgfscope}%
\pgfsys@transformshift{1.328349in}{1.911221in}%
\pgfsys@useobject{currentmarker}{}%
\end{pgfscope}%
\begin{pgfscope}%
\pgfsys@transformshift{1.405073in}{1.960019in}%
\pgfsys@useobject{currentmarker}{}%
\end{pgfscope}%
\begin{pgfscope}%
\pgfsys@transformshift{0.669867in}{2.007180in}%
\pgfsys@useobject{currentmarker}{}%
\end{pgfscope}%
\begin{pgfscope}%
\pgfsys@transformshift{1.898744in}{1.907374in}%
\pgfsys@useobject{currentmarker}{}%
\end{pgfscope}%
\begin{pgfscope}%
\pgfsys@transformshift{0.874940in}{2.011627in}%
\pgfsys@useobject{currentmarker}{}%
\end{pgfscope}%
\begin{pgfscope}%
\pgfsys@transformshift{1.459937in}{2.052760in}%
\pgfsys@useobject{currentmarker}{}%
\end{pgfscope}%
\begin{pgfscope}%
\pgfsys@transformshift{1.895928in}{1.902037in}%
\pgfsys@useobject{currentmarker}{}%
\end{pgfscope}%
\begin{pgfscope}%
\pgfsys@transformshift{0.853418in}{2.016570in}%
\pgfsys@useobject{currentmarker}{}%
\end{pgfscope}%
\begin{pgfscope}%
\pgfsys@transformshift{0.717208in}{2.049332in}%
\pgfsys@useobject{currentmarker}{}%
\end{pgfscope}%
\begin{pgfscope}%
\pgfsys@transformshift{1.075298in}{2.009591in}%
\pgfsys@useobject{currentmarker}{}%
\end{pgfscope}%
\begin{pgfscope}%
\pgfsys@transformshift{1.920218in}{2.053035in}%
\pgfsys@useobject{currentmarker}{}%
\end{pgfscope}%
\begin{pgfscope}%
\pgfsys@transformshift{0.377829in}{2.003394in}%
\pgfsys@useobject{currentmarker}{}%
\end{pgfscope}%
\begin{pgfscope}%
\pgfsys@transformshift{1.155745in}{1.993950in}%
\pgfsys@useobject{currentmarker}{}%
\end{pgfscope}%
\begin{pgfscope}%
\pgfsys@transformshift{1.712524in}{1.870010in}%
\pgfsys@useobject{currentmarker}{}%
\end{pgfscope}%
\begin{pgfscope}%
\pgfsys@transformshift{0.534596in}{2.009958in}%
\pgfsys@useobject{currentmarker}{}%
\end{pgfscope}%
\begin{pgfscope}%
\pgfsys@transformshift{0.622741in}{2.075354in}%
\pgfsys@useobject{currentmarker}{}%
\end{pgfscope}%
\begin{pgfscope}%
\pgfsys@transformshift{0.575527in}{2.039920in}%
\pgfsys@useobject{currentmarker}{}%
\end{pgfscope}%
\begin{pgfscope}%
\pgfsys@transformshift{1.804863in}{1.893021in}%
\pgfsys@useobject{currentmarker}{}%
\end{pgfscope}%
\begin{pgfscope}%
\pgfsys@transformshift{1.749745in}{1.883759in}%
\pgfsys@useobject{currentmarker}{}%
\end{pgfscope}%
\begin{pgfscope}%
\pgfsys@transformshift{1.674396in}{2.007840in}%
\pgfsys@useobject{currentmarker}{}%
\end{pgfscope}%
\begin{pgfscope}%
\pgfsys@transformshift{1.859027in}{1.926533in}%
\pgfsys@useobject{currentmarker}{}%
\end{pgfscope}%
\begin{pgfscope}%
\pgfsys@transformshift{1.603827in}{1.920397in}%
\pgfsys@useobject{currentmarker}{}%
\end{pgfscope}%
\begin{pgfscope}%
\pgfsys@transformshift{1.255917in}{2.009900in}%
\pgfsys@useobject{currentmarker}{}%
\end{pgfscope}%
\begin{pgfscope}%
\pgfsys@transformshift{0.419746in}{1.846606in}%
\pgfsys@useobject{currentmarker}{}%
\end{pgfscope}%
\begin{pgfscope}%
\pgfsys@transformshift{0.903589in}{1.841912in}%
\pgfsys@useobject{currentmarker}{}%
\end{pgfscope}%
\begin{pgfscope}%
\pgfsys@transformshift{1.411837in}{1.989668in}%
\pgfsys@useobject{currentmarker}{}%
\end{pgfscope}%
\begin{pgfscope}%
\pgfsys@transformshift{1.888179in}{1.908110in}%
\pgfsys@useobject{currentmarker}{}%
\end{pgfscope}%
\begin{pgfscope}%
\pgfsys@transformshift{1.463948in}{1.955067in}%
\pgfsys@useobject{currentmarker}{}%
\end{pgfscope}%
\begin{pgfscope}%
\pgfsys@transformshift{1.037147in}{1.961944in}%
\pgfsys@useobject{currentmarker}{}%
\end{pgfscope}%
\begin{pgfscope}%
\pgfsys@transformshift{1.744144in}{1.896262in}%
\pgfsys@useobject{currentmarker}{}%
\end{pgfscope}%
\begin{pgfscope}%
\pgfsys@transformshift{0.867355in}{1.923603in}%
\pgfsys@useobject{currentmarker}{}%
\end{pgfscope}%
\begin{pgfscope}%
\pgfsys@transformshift{1.331798in}{2.076896in}%
\pgfsys@useobject{currentmarker}{}%
\end{pgfscope}%
\begin{pgfscope}%
\pgfsys@transformshift{0.968745in}{2.031916in}%
\pgfsys@useobject{currentmarker}{}%
\end{pgfscope}%
\begin{pgfscope}%
\pgfsys@transformshift{1.193049in}{2.068833in}%
\pgfsys@useobject{currentmarker}{}%
\end{pgfscope}%
\begin{pgfscope}%
\pgfsys@transformshift{1.664246in}{1.965489in}%
\pgfsys@useobject{currentmarker}{}%
\end{pgfscope}%
\begin{pgfscope}%
\pgfsys@transformshift{0.835587in}{2.025991in}%
\pgfsys@useobject{currentmarker}{}%
\end{pgfscope}%
\begin{pgfscope}%
\pgfsys@transformshift{1.701974in}{1.889322in}%
\pgfsys@useobject{currentmarker}{}%
\end{pgfscope}%
\begin{pgfscope}%
\pgfsys@transformshift{1.269033in}{1.972418in}%
\pgfsys@useobject{currentmarker}{}%
\end{pgfscope}%
\begin{pgfscope}%
\pgfsys@transformshift{1.329352in}{1.922245in}%
\pgfsys@useobject{currentmarker}{}%
\end{pgfscope}%
\begin{pgfscope}%
\pgfsys@transformshift{1.471661in}{2.062219in}%
\pgfsys@useobject{currentmarker}{}%
\end{pgfscope}%
\begin{pgfscope}%
\pgfsys@transformshift{1.916157in}{2.074103in}%
\pgfsys@useobject{currentmarker}{}%
\end{pgfscope}%
\begin{pgfscope}%
\pgfsys@transformshift{1.533379in}{1.862847in}%
\pgfsys@useobject{currentmarker}{}%
\end{pgfscope}%
\begin{pgfscope}%
\pgfsys@transformshift{1.358583in}{1.905948in}%
\pgfsys@useobject{currentmarker}{}%
\end{pgfscope}%
\begin{pgfscope}%
\pgfsys@transformshift{1.599245in}{1.893359in}%
\pgfsys@useobject{currentmarker}{}%
\end{pgfscope}%
\begin{pgfscope}%
\pgfsys@transformshift{0.703172in}{2.036840in}%
\pgfsys@useobject{currentmarker}{}%
\end{pgfscope}%
\begin{pgfscope}%
\pgfsys@transformshift{1.320412in}{2.050828in}%
\pgfsys@useobject{currentmarker}{}%
\end{pgfscope}%
\begin{pgfscope}%
\pgfsys@transformshift{1.136406in}{1.970788in}%
\pgfsys@useobject{currentmarker}{}%
\end{pgfscope}%
\begin{pgfscope}%
\pgfsys@transformshift{0.721527in}{2.005835in}%
\pgfsys@useobject{currentmarker}{}%
\end{pgfscope}%
\begin{pgfscope}%
\pgfsys@transformshift{1.722543in}{1.891492in}%
\pgfsys@useobject{currentmarker}{}%
\end{pgfscope}%
\begin{pgfscope}%
\pgfsys@transformshift{1.835288in}{1.935730in}%
\pgfsys@useobject{currentmarker}{}%
\end{pgfscope}%
\begin{pgfscope}%
\pgfsys@transformshift{1.984248in}{2.036040in}%
\pgfsys@useobject{currentmarker}{}%
\end{pgfscope}%
\begin{pgfscope}%
\pgfsys@transformshift{1.658536in}{1.996290in}%
\pgfsys@useobject{currentmarker}{}%
\end{pgfscope}%
\begin{pgfscope}%
\pgfsys@transformshift{0.561607in}{2.014994in}%
\pgfsys@useobject{currentmarker}{}%
\end{pgfscope}%
\begin{pgfscope}%
\pgfsys@transformshift{1.906539in}{2.039578in}%
\pgfsys@useobject{currentmarker}{}%
\end{pgfscope}%
\begin{pgfscope}%
\pgfsys@transformshift{1.521333in}{2.002018in}%
\pgfsys@useobject{currentmarker}{}%
\end{pgfscope}%
\begin{pgfscope}%
\pgfsys@transformshift{1.136396in}{1.987024in}%
\pgfsys@useobject{currentmarker}{}%
\end{pgfscope}%
\begin{pgfscope}%
\pgfsys@transformshift{1.435818in}{1.828227in}%
\pgfsys@useobject{currentmarker}{}%
\end{pgfscope}%
\begin{pgfscope}%
\pgfsys@transformshift{1.643271in}{1.886876in}%
\pgfsys@useobject{currentmarker}{}%
\end{pgfscope}%
\begin{pgfscope}%
\pgfsys@transformshift{1.445064in}{1.894889in}%
\pgfsys@useobject{currentmarker}{}%
\end{pgfscope}%
\begin{pgfscope}%
\pgfsys@transformshift{0.658007in}{1.940648in}%
\pgfsys@useobject{currentmarker}{}%
\end{pgfscope}%
\begin{pgfscope}%
\pgfsys@transformshift{0.786972in}{2.010727in}%
\pgfsys@useobject{currentmarker}{}%
\end{pgfscope}%
\begin{pgfscope}%
\pgfsys@transformshift{1.102575in}{2.011002in}%
\pgfsys@useobject{currentmarker}{}%
\end{pgfscope}%
\begin{pgfscope}%
\pgfsys@transformshift{0.532598in}{2.007710in}%
\pgfsys@useobject{currentmarker}{}%
\end{pgfscope}%
\begin{pgfscope}%
\pgfsys@transformshift{0.357574in}{2.001093in}%
\pgfsys@useobject{currentmarker}{}%
\end{pgfscope}%
\begin{pgfscope}%
\pgfsys@transformshift{1.683669in}{1.797011in}%
\pgfsys@useobject{currentmarker}{}%
\end{pgfscope}%
\begin{pgfscope}%
\pgfsys@transformshift{1.612445in}{1.956665in}%
\pgfsys@useobject{currentmarker}{}%
\end{pgfscope}%
\begin{pgfscope}%
\pgfsys@transformshift{1.620066in}{1.849496in}%
\pgfsys@useobject{currentmarker}{}%
\end{pgfscope}%
\begin{pgfscope}%
\pgfsys@transformshift{0.680343in}{1.871213in}%
\pgfsys@useobject{currentmarker}{}%
\end{pgfscope}%
\begin{pgfscope}%
\pgfsys@transformshift{0.675258in}{2.001621in}%
\pgfsys@useobject{currentmarker}{}%
\end{pgfscope}%
\begin{pgfscope}%
\pgfsys@transformshift{0.963927in}{1.958615in}%
\pgfsys@useobject{currentmarker}{}%
\end{pgfscope}%
\begin{pgfscope}%
\pgfsys@transformshift{1.684562in}{1.940849in}%
\pgfsys@useobject{currentmarker}{}%
\end{pgfscope}%
\begin{pgfscope}%
\pgfsys@transformshift{1.473245in}{2.048109in}%
\pgfsys@useobject{currentmarker}{}%
\end{pgfscope}%
\begin{pgfscope}%
\pgfsys@transformshift{1.020383in}{2.062582in}%
\pgfsys@useobject{currentmarker}{}%
\end{pgfscope}%
\begin{pgfscope}%
\pgfsys@transformshift{0.681597in}{2.049666in}%
\pgfsys@useobject{currentmarker}{}%
\end{pgfscope}%
\begin{pgfscope}%
\pgfsys@transformshift{1.730115in}{1.961811in}%
\pgfsys@useobject{currentmarker}{}%
\end{pgfscope}%
\begin{pgfscope}%
\pgfsys@transformshift{1.808933in}{1.911043in}%
\pgfsys@useobject{currentmarker}{}%
\end{pgfscope}%
\begin{pgfscope}%
\pgfsys@transformshift{1.251675in}{2.001655in}%
\pgfsys@useobject{currentmarker}{}%
\end{pgfscope}%
\begin{pgfscope}%
\pgfsys@transformshift{1.717946in}{1.879099in}%
\pgfsys@useobject{currentmarker}{}%
\end{pgfscope}%
\begin{pgfscope}%
\pgfsys@transformshift{1.597312in}{1.870736in}%
\pgfsys@useobject{currentmarker}{}%
\end{pgfscope}%
\begin{pgfscope}%
\pgfsys@transformshift{1.148136in}{2.050230in}%
\pgfsys@useobject{currentmarker}{}%
\end{pgfscope}%
\begin{pgfscope}%
\pgfsys@transformshift{0.720143in}{2.017250in}%
\pgfsys@useobject{currentmarker}{}%
\end{pgfscope}%
\begin{pgfscope}%
\pgfsys@transformshift{1.574970in}{1.965223in}%
\pgfsys@useobject{currentmarker}{}%
\end{pgfscope}%
\begin{pgfscope}%
\pgfsys@transformshift{0.869972in}{1.921908in}%
\pgfsys@useobject{currentmarker}{}%
\end{pgfscope}%
\begin{pgfscope}%
\pgfsys@transformshift{1.713193in}{1.932851in}%
\pgfsys@useobject{currentmarker}{}%
\end{pgfscope}%
\begin{pgfscope}%
\pgfsys@transformshift{1.673692in}{1.830993in}%
\pgfsys@useobject{currentmarker}{}%
\end{pgfscope}%
\begin{pgfscope}%
\pgfsys@transformshift{1.344861in}{1.912288in}%
\pgfsys@useobject{currentmarker}{}%
\end{pgfscope}%
\begin{pgfscope}%
\pgfsys@transformshift{1.182101in}{1.993333in}%
\pgfsys@useobject{currentmarker}{}%
\end{pgfscope}%
\begin{pgfscope}%
\pgfsys@transformshift{1.449668in}{1.963827in}%
\pgfsys@useobject{currentmarker}{}%
\end{pgfscope}%
\begin{pgfscope}%
\pgfsys@transformshift{1.561680in}{1.948637in}%
\pgfsys@useobject{currentmarker}{}%
\end{pgfscope}%
\begin{pgfscope}%
\pgfsys@transformshift{1.192063in}{1.890201in}%
\pgfsys@useobject{currentmarker}{}%
\end{pgfscope}%
\begin{pgfscope}%
\pgfsys@transformshift{1.264374in}{2.011148in}%
\pgfsys@useobject{currentmarker}{}%
\end{pgfscope}%
\begin{pgfscope}%
\pgfsys@transformshift{0.727375in}{2.064471in}%
\pgfsys@useobject{currentmarker}{}%
\end{pgfscope}%
\begin{pgfscope}%
\pgfsys@transformshift{0.687912in}{1.818542in}%
\pgfsys@useobject{currentmarker}{}%
\end{pgfscope}%
\begin{pgfscope}%
\pgfsys@transformshift{1.823596in}{1.979354in}%
\pgfsys@useobject{currentmarker}{}%
\end{pgfscope}%
\begin{pgfscope}%
\pgfsys@transformshift{1.420153in}{1.936108in}%
\pgfsys@useobject{currentmarker}{}%
\end{pgfscope}%
\begin{pgfscope}%
\pgfsys@transformshift{0.672265in}{2.083534in}%
\pgfsys@useobject{currentmarker}{}%
\end{pgfscope}%
\begin{pgfscope}%
\pgfsys@transformshift{0.574267in}{2.074620in}%
\pgfsys@useobject{currentmarker}{}%
\end{pgfscope}%
\begin{pgfscope}%
\pgfsys@transformshift{1.570297in}{1.936097in}%
\pgfsys@useobject{currentmarker}{}%
\end{pgfscope}%
\begin{pgfscope}%
\pgfsys@transformshift{1.650981in}{1.842328in}%
\pgfsys@useobject{currentmarker}{}%
\end{pgfscope}%
\begin{pgfscope}%
\pgfsys@transformshift{0.894282in}{1.989262in}%
\pgfsys@useobject{currentmarker}{}%
\end{pgfscope}%
\begin{pgfscope}%
\pgfsys@transformshift{1.157746in}{2.028947in}%
\pgfsys@useobject{currentmarker}{}%
\end{pgfscope}%
\begin{pgfscope}%
\pgfsys@transformshift{0.608353in}{2.031825in}%
\pgfsys@useobject{currentmarker}{}%
\end{pgfscope}%
\begin{pgfscope}%
\pgfsys@transformshift{1.631665in}{1.843638in}%
\pgfsys@useobject{currentmarker}{}%
\end{pgfscope}%
\begin{pgfscope}%
\pgfsys@transformshift{1.419963in}{1.900055in}%
\pgfsys@useobject{currentmarker}{}%
\end{pgfscope}%
\begin{pgfscope}%
\pgfsys@transformshift{1.180003in}{2.014873in}%
\pgfsys@useobject{currentmarker}{}%
\end{pgfscope}%
\begin{pgfscope}%
\pgfsys@transformshift{1.469789in}{1.996300in}%
\pgfsys@useobject{currentmarker}{}%
\end{pgfscope}%
\begin{pgfscope}%
\pgfsys@transformshift{1.493240in}{2.061807in}%
\pgfsys@useobject{currentmarker}{}%
\end{pgfscope}%
\begin{pgfscope}%
\pgfsys@transformshift{1.725815in}{1.902555in}%
\pgfsys@useobject{currentmarker}{}%
\end{pgfscope}%
\begin{pgfscope}%
\pgfsys@transformshift{1.002068in}{1.996646in}%
\pgfsys@useobject{currentmarker}{}%
\end{pgfscope}%
\begin{pgfscope}%
\pgfsys@transformshift{1.111774in}{1.984027in}%
\pgfsys@useobject{currentmarker}{}%
\end{pgfscope}%
\begin{pgfscope}%
\pgfsys@transformshift{0.361781in}{2.043058in}%
\pgfsys@useobject{currentmarker}{}%
\end{pgfscope}%
\begin{pgfscope}%
\pgfsys@transformshift{1.608152in}{1.954483in}%
\pgfsys@useobject{currentmarker}{}%
\end{pgfscope}%
\begin{pgfscope}%
\pgfsys@transformshift{1.468569in}{2.075238in}%
\pgfsys@useobject{currentmarker}{}%
\end{pgfscope}%
\begin{pgfscope}%
\pgfsys@transformshift{1.646778in}{1.827884in}%
\pgfsys@useobject{currentmarker}{}%
\end{pgfscope}%
\begin{pgfscope}%
\pgfsys@transformshift{1.356664in}{2.008348in}%
\pgfsys@useobject{currentmarker}{}%
\end{pgfscope}%
\begin{pgfscope}%
\pgfsys@transformshift{0.657110in}{1.821922in}%
\pgfsys@useobject{currentmarker}{}%
\end{pgfscope}%
\begin{pgfscope}%
\pgfsys@transformshift{1.700456in}{1.833762in}%
\pgfsys@useobject{currentmarker}{}%
\end{pgfscope}%
\begin{pgfscope}%
\pgfsys@transformshift{0.741052in}{1.987036in}%
\pgfsys@useobject{currentmarker}{}%
\end{pgfscope}%
\begin{pgfscope}%
\pgfsys@transformshift{0.576055in}{2.097321in}%
\pgfsys@useobject{currentmarker}{}%
\end{pgfscope}%
\begin{pgfscope}%
\pgfsys@transformshift{1.509849in}{1.966228in}%
\pgfsys@useobject{currentmarker}{}%
\end{pgfscope}%
\begin{pgfscope}%
\pgfsys@transformshift{1.838065in}{1.864650in}%
\pgfsys@useobject{currentmarker}{}%
\end{pgfscope}%
\begin{pgfscope}%
\pgfsys@transformshift{1.699823in}{2.023735in}%
\pgfsys@useobject{currentmarker}{}%
\end{pgfscope}%
\begin{pgfscope}%
\pgfsys@transformshift{1.575775in}{1.988674in}%
\pgfsys@useobject{currentmarker}{}%
\end{pgfscope}%
\begin{pgfscope}%
\pgfsys@transformshift{1.000652in}{2.053131in}%
\pgfsys@useobject{currentmarker}{}%
\end{pgfscope}%
\begin{pgfscope}%
\pgfsys@transformshift{1.275329in}{2.015209in}%
\pgfsys@useobject{currentmarker}{}%
\end{pgfscope}%
\begin{pgfscope}%
\pgfsys@transformshift{0.787302in}{2.013721in}%
\pgfsys@useobject{currentmarker}{}%
\end{pgfscope}%
\begin{pgfscope}%
\pgfsys@transformshift{0.941308in}{1.893217in}%
\pgfsys@useobject{currentmarker}{}%
\end{pgfscope}%
\begin{pgfscope}%
\pgfsys@transformshift{1.041866in}{1.930266in}%
\pgfsys@useobject{currentmarker}{}%
\end{pgfscope}%
\begin{pgfscope}%
\pgfsys@transformshift{1.471835in}{1.950838in}%
\pgfsys@useobject{currentmarker}{}%
\end{pgfscope}%
\begin{pgfscope}%
\pgfsys@transformshift{1.225077in}{1.969226in}%
\pgfsys@useobject{currentmarker}{}%
\end{pgfscope}%
\begin{pgfscope}%
\pgfsys@transformshift{1.294631in}{1.933776in}%
\pgfsys@useobject{currentmarker}{}%
\end{pgfscope}%
\begin{pgfscope}%
\pgfsys@transformshift{1.597060in}{1.979378in}%
\pgfsys@useobject{currentmarker}{}%
\end{pgfscope}%
\begin{pgfscope}%
\pgfsys@transformshift{1.136873in}{1.961379in}%
\pgfsys@useobject{currentmarker}{}%
\end{pgfscope}%
\begin{pgfscope}%
\pgfsys@transformshift{1.080421in}{1.969128in}%
\pgfsys@useobject{currentmarker}{}%
\end{pgfscope}%
\begin{pgfscope}%
\pgfsys@transformshift{1.185638in}{2.017833in}%
\pgfsys@useobject{currentmarker}{}%
\end{pgfscope}%
\begin{pgfscope}%
\pgfsys@transformshift{1.407062in}{2.003719in}%
\pgfsys@useobject{currentmarker}{}%
\end{pgfscope}%
\begin{pgfscope}%
\pgfsys@transformshift{1.504034in}{2.040366in}%
\pgfsys@useobject{currentmarker}{}%
\end{pgfscope}%
\begin{pgfscope}%
\pgfsys@transformshift{1.396930in}{1.963732in}%
\pgfsys@useobject{currentmarker}{}%
\end{pgfscope}%
\begin{pgfscope}%
\pgfsys@transformshift{2.000000in}{1.914104in}%
\pgfsys@useobject{currentmarker}{}%
\end{pgfscope}%
\begin{pgfscope}%
\pgfsys@transformshift{1.200619in}{1.938692in}%
\pgfsys@useobject{currentmarker}{}%
\end{pgfscope}%
\begin{pgfscope}%
\pgfsys@transformshift{0.950093in}{1.810593in}%
\pgfsys@useobject{currentmarker}{}%
\end{pgfscope}%
\begin{pgfscope}%
\pgfsys@transformshift{0.632457in}{2.005345in}%
\pgfsys@useobject{currentmarker}{}%
\end{pgfscope}%
\begin{pgfscope}%
\pgfsys@transformshift{0.673561in}{1.970817in}%
\pgfsys@useobject{currentmarker}{}%
\end{pgfscope}%
\begin{pgfscope}%
\pgfsys@transformshift{0.694799in}{2.028062in}%
\pgfsys@useobject{currentmarker}{}%
\end{pgfscope}%
\begin{pgfscope}%
\pgfsys@transformshift{0.518198in}{2.024912in}%
\pgfsys@useobject{currentmarker}{}%
\end{pgfscope}%
\begin{pgfscope}%
\pgfsys@transformshift{1.324442in}{1.959553in}%
\pgfsys@useobject{currentmarker}{}%
\end{pgfscope}%
\begin{pgfscope}%
\pgfsys@transformshift{0.827155in}{2.074117in}%
\pgfsys@useobject{currentmarker}{}%
\end{pgfscope}%
\begin{pgfscope}%
\pgfsys@transformshift{1.894588in}{1.970025in}%
\pgfsys@useobject{currentmarker}{}%
\end{pgfscope}%
\begin{pgfscope}%
\pgfsys@transformshift{0.446321in}{1.992254in}%
\pgfsys@useobject{currentmarker}{}%
\end{pgfscope}%
\begin{pgfscope}%
\pgfsys@transformshift{1.741761in}{1.989200in}%
\pgfsys@useobject{currentmarker}{}%
\end{pgfscope}%
\begin{pgfscope}%
\pgfsys@transformshift{1.307351in}{1.932830in}%
\pgfsys@useobject{currentmarker}{}%
\end{pgfscope}%
\begin{pgfscope}%
\pgfsys@transformshift{1.835875in}{1.956796in}%
\pgfsys@useobject{currentmarker}{}%
\end{pgfscope}%
\begin{pgfscope}%
\pgfsys@transformshift{1.237368in}{1.972791in}%
\pgfsys@useobject{currentmarker}{}%
\end{pgfscope}%
\begin{pgfscope}%
\pgfsys@transformshift{1.875271in}{1.711420in}%
\pgfsys@useobject{currentmarker}{}%
\end{pgfscope}%
\begin{pgfscope}%
\pgfsys@transformshift{1.669995in}{1.936381in}%
\pgfsys@useobject{currentmarker}{}%
\end{pgfscope}%
\begin{pgfscope}%
\pgfsys@transformshift{0.628877in}{2.050170in}%
\pgfsys@useobject{currentmarker}{}%
\end{pgfscope}%
\begin{pgfscope}%
\pgfsys@transformshift{1.644915in}{1.944560in}%
\pgfsys@useobject{currentmarker}{}%
\end{pgfscope}%
\begin{pgfscope}%
\pgfsys@transformshift{1.179959in}{1.947766in}%
\pgfsys@useobject{currentmarker}{}%
\end{pgfscope}%
\begin{pgfscope}%
\pgfsys@transformshift{1.753339in}{1.935518in}%
\pgfsys@useobject{currentmarker}{}%
\end{pgfscope}%
\begin{pgfscope}%
\pgfsys@transformshift{1.477915in}{1.979864in}%
\pgfsys@useobject{currentmarker}{}%
\end{pgfscope}%
\begin{pgfscope}%
\pgfsys@transformshift{1.742609in}{1.988283in}%
\pgfsys@useobject{currentmarker}{}%
\end{pgfscope}%
\begin{pgfscope}%
\pgfsys@transformshift{1.094044in}{2.009092in}%
\pgfsys@useobject{currentmarker}{}%
\end{pgfscope}%
\begin{pgfscope}%
\pgfsys@transformshift{1.645687in}{1.852331in}%
\pgfsys@useobject{currentmarker}{}%
\end{pgfscope}%
\begin{pgfscope}%
\pgfsys@transformshift{0.717249in}{1.951901in}%
\pgfsys@useobject{currentmarker}{}%
\end{pgfscope}%
\begin{pgfscope}%
\pgfsys@transformshift{0.963925in}{1.918350in}%
\pgfsys@useobject{currentmarker}{}%
\end{pgfscope}%
\begin{pgfscope}%
\pgfsys@transformshift{1.342874in}{1.966656in}%
\pgfsys@useobject{currentmarker}{}%
\end{pgfscope}%
\begin{pgfscope}%
\pgfsys@transformshift{1.522619in}{1.921266in}%
\pgfsys@useobject{currentmarker}{}%
\end{pgfscope}%
\begin{pgfscope}%
\pgfsys@transformshift{0.649847in}{2.054593in}%
\pgfsys@useobject{currentmarker}{}%
\end{pgfscope}%
\begin{pgfscope}%
\pgfsys@transformshift{1.826185in}{2.009289in}%
\pgfsys@useobject{currentmarker}{}%
\end{pgfscope}%
\begin{pgfscope}%
\pgfsys@transformshift{1.619806in}{1.902712in}%
\pgfsys@useobject{currentmarker}{}%
\end{pgfscope}%
\begin{pgfscope}%
\pgfsys@transformshift{1.136702in}{2.056174in}%
\pgfsys@useobject{currentmarker}{}%
\end{pgfscope}%
\begin{pgfscope}%
\pgfsys@transformshift{1.894425in}{1.815015in}%
\pgfsys@useobject{currentmarker}{}%
\end{pgfscope}%
\begin{pgfscope}%
\pgfsys@transformshift{0.668971in}{2.043044in}%
\pgfsys@useobject{currentmarker}{}%
\end{pgfscope}%
\end{pgfscope}%
\begin{pgfscope}%
\pgfpathrectangle{\pgfqpoint{0.341129in}{0.466613in}}{\pgfqpoint{1.658871in}{1.711598in}}%
\pgfusepath{clip}%
\pgfsetbuttcap%
\pgfsetroundjoin%
\definecolor{currentfill}{rgb}{0.298039,0.447059,0.690196}%
\pgfsetfillcolor{currentfill}%
\pgfsetfillopacity{0.150000}%
\pgfsetlinewidth{1.003750pt}%
\definecolor{currentstroke}{rgb}{1.000000,1.000000,1.000000}%
\pgfsetstrokecolor{currentstroke}%
\pgfsetstrokeopacity{0.150000}%
\pgfsetdash{}{0pt}%
\pgfsys@defobject{currentmarker}{\pgfqpoint{0.341129in}{1.926600in}}{\pgfqpoint{2.000000in}{2.015390in}}{%
\pgfpathmoveto{\pgfqpoint{0.341129in}{2.015390in}}%
\pgfpathlineto{\pgfqpoint{0.341129in}{1.983741in}}%
\pgfpathlineto{\pgfqpoint{0.357885in}{1.983360in}}%
\pgfpathlineto{\pgfqpoint{0.374641in}{1.982980in}}%
\pgfpathlineto{\pgfqpoint{0.391398in}{1.982601in}}%
\pgfpathlineto{\pgfqpoint{0.408154in}{1.982222in}}%
\pgfpathlineto{\pgfqpoint{0.424910in}{1.981934in}}%
\pgfpathlineto{\pgfqpoint{0.441666in}{1.981598in}}%
\pgfpathlineto{\pgfqpoint{0.458423in}{1.981220in}}%
\pgfpathlineto{\pgfqpoint{0.475179in}{1.980844in}}%
\pgfpathlineto{\pgfqpoint{0.491935in}{1.980460in}}%
\pgfpathlineto{\pgfqpoint{0.508691in}{1.980047in}}%
\pgfpathlineto{\pgfqpoint{0.525448in}{1.979701in}}%
\pgfpathlineto{\pgfqpoint{0.542204in}{1.979335in}}%
\pgfpathlineto{\pgfqpoint{0.558960in}{1.978959in}}%
\pgfpathlineto{\pgfqpoint{0.575717in}{1.978584in}}%
\pgfpathlineto{\pgfqpoint{0.592473in}{1.978201in}}%
\pgfpathlineto{\pgfqpoint{0.609229in}{1.977794in}}%
\pgfpathlineto{\pgfqpoint{0.625985in}{1.977414in}}%
\pgfpathlineto{\pgfqpoint{0.642742in}{1.977063in}}%
\pgfpathlineto{\pgfqpoint{0.659498in}{1.976659in}}%
\pgfpathlineto{\pgfqpoint{0.676254in}{1.976254in}}%
\pgfpathlineto{\pgfqpoint{0.693011in}{1.975851in}}%
\pgfpathlineto{\pgfqpoint{0.709767in}{1.975437in}}%
\pgfpathlineto{\pgfqpoint{0.726523in}{1.975010in}}%
\pgfpathlineto{\pgfqpoint{0.743279in}{1.974584in}}%
\pgfpathlineto{\pgfqpoint{0.760036in}{1.974193in}}%
\pgfpathlineto{\pgfqpoint{0.776792in}{1.973822in}}%
\pgfpathlineto{\pgfqpoint{0.793548in}{1.973340in}}%
\pgfpathlineto{\pgfqpoint{0.810304in}{1.972920in}}%
\pgfpathlineto{\pgfqpoint{0.827061in}{1.972500in}}%
\pgfpathlineto{\pgfqpoint{0.843817in}{1.972080in}}%
\pgfpathlineto{\pgfqpoint{0.860573in}{1.971660in}}%
\pgfpathlineto{\pgfqpoint{0.877330in}{1.971285in}}%
\pgfpathlineto{\pgfqpoint{0.894086in}{1.970998in}}%
\pgfpathlineto{\pgfqpoint{0.910842in}{1.970630in}}%
\pgfpathlineto{\pgfqpoint{0.927598in}{1.970172in}}%
\pgfpathlineto{\pgfqpoint{0.944355in}{1.969764in}}%
\pgfpathlineto{\pgfqpoint{0.961111in}{1.969294in}}%
\pgfpathlineto{\pgfqpoint{0.977867in}{1.968954in}}%
\pgfpathlineto{\pgfqpoint{0.994623in}{1.968550in}}%
\pgfpathlineto{\pgfqpoint{1.011380in}{1.968145in}}%
\pgfpathlineto{\pgfqpoint{1.028136in}{1.967694in}}%
\pgfpathlineto{\pgfqpoint{1.044892in}{1.967042in}}%
\pgfpathlineto{\pgfqpoint{1.061649in}{1.966619in}}%
\pgfpathlineto{\pgfqpoint{1.078405in}{1.966092in}}%
\pgfpathlineto{\pgfqpoint{1.095161in}{1.965633in}}%
\pgfpathlineto{\pgfqpoint{1.111917in}{1.965125in}}%
\pgfpathlineto{\pgfqpoint{1.128674in}{1.964616in}}%
\pgfpathlineto{\pgfqpoint{1.145430in}{1.964065in}}%
\pgfpathlineto{\pgfqpoint{1.162186in}{1.963576in}}%
\pgfpathlineto{\pgfqpoint{1.178942in}{1.962963in}}%
\pgfpathlineto{\pgfqpoint{1.195699in}{1.962323in}}%
\pgfpathlineto{\pgfqpoint{1.212455in}{1.961702in}}%
\pgfpathlineto{\pgfqpoint{1.229211in}{1.961119in}}%
\pgfpathlineto{\pgfqpoint{1.245968in}{1.960559in}}%
\pgfpathlineto{\pgfqpoint{1.262724in}{1.959922in}}%
\pgfpathlineto{\pgfqpoint{1.279480in}{1.959269in}}%
\pgfpathlineto{\pgfqpoint{1.296236in}{1.958568in}}%
\pgfpathlineto{\pgfqpoint{1.312993in}{1.957890in}}%
\pgfpathlineto{\pgfqpoint{1.329749in}{1.957214in}}%
\pgfpathlineto{\pgfqpoint{1.346505in}{1.956576in}}%
\pgfpathlineto{\pgfqpoint{1.363262in}{1.956061in}}%
\pgfpathlineto{\pgfqpoint{1.380018in}{1.955430in}}%
\pgfpathlineto{\pgfqpoint{1.396774in}{1.954740in}}%
\pgfpathlineto{\pgfqpoint{1.413530in}{1.954140in}}%
\pgfpathlineto{\pgfqpoint{1.430287in}{1.953499in}}%
\pgfpathlineto{\pgfqpoint{1.447043in}{1.952844in}}%
\pgfpathlineto{\pgfqpoint{1.463799in}{1.952137in}}%
\pgfpathlineto{\pgfqpoint{1.480555in}{1.951389in}}%
\pgfpathlineto{\pgfqpoint{1.497312in}{1.950733in}}%
\pgfpathlineto{\pgfqpoint{1.514068in}{1.950029in}}%
\pgfpathlineto{\pgfqpoint{1.530824in}{1.949323in}}%
\pgfpathlineto{\pgfqpoint{1.547581in}{1.948613in}}%
\pgfpathlineto{\pgfqpoint{1.564337in}{1.947824in}}%
\pgfpathlineto{\pgfqpoint{1.581093in}{1.947023in}}%
\pgfpathlineto{\pgfqpoint{1.597849in}{1.946196in}}%
\pgfpathlineto{\pgfqpoint{1.614606in}{1.945324in}}%
\pgfpathlineto{\pgfqpoint{1.631362in}{1.944469in}}%
\pgfpathlineto{\pgfqpoint{1.648118in}{1.943618in}}%
\pgfpathlineto{\pgfqpoint{1.664874in}{1.942768in}}%
\pgfpathlineto{\pgfqpoint{1.681631in}{1.941974in}}%
\pgfpathlineto{\pgfqpoint{1.698387in}{1.941130in}}%
\pgfpathlineto{\pgfqpoint{1.715143in}{1.940301in}}%
\pgfpathlineto{\pgfqpoint{1.731900in}{1.939396in}}%
\pgfpathlineto{\pgfqpoint{1.748656in}{1.938586in}}%
\pgfpathlineto{\pgfqpoint{1.765412in}{1.937821in}}%
\pgfpathlineto{\pgfqpoint{1.782168in}{1.937056in}}%
\pgfpathlineto{\pgfqpoint{1.798925in}{1.936293in}}%
\pgfpathlineto{\pgfqpoint{1.815681in}{1.935423in}}%
\pgfpathlineto{\pgfqpoint{1.832437in}{1.934623in}}%
\pgfpathlineto{\pgfqpoint{1.849193in}{1.933884in}}%
\pgfpathlineto{\pgfqpoint{1.865950in}{1.933152in}}%
\pgfpathlineto{\pgfqpoint{1.882706in}{1.932353in}}%
\pgfpathlineto{\pgfqpoint{1.899462in}{1.931451in}}%
\pgfpathlineto{\pgfqpoint{1.916219in}{1.930581in}}%
\pgfpathlineto{\pgfqpoint{1.932975in}{1.929678in}}%
\pgfpathlineto{\pgfqpoint{1.949731in}{1.928812in}}%
\pgfpathlineto{\pgfqpoint{1.966487in}{1.928070in}}%
\pgfpathlineto{\pgfqpoint{1.983244in}{1.927335in}}%
\pgfpathlineto{\pgfqpoint{2.000000in}{1.926600in}}%
\pgfpathlineto{\pgfqpoint{2.000000in}{1.952043in}}%
\pgfpathlineto{\pgfqpoint{2.000000in}{1.952043in}}%
\pgfpathlineto{\pgfqpoint{1.983244in}{1.952334in}}%
\pgfpathlineto{\pgfqpoint{1.966487in}{1.952661in}}%
\pgfpathlineto{\pgfqpoint{1.949731in}{1.953055in}}%
\pgfpathlineto{\pgfqpoint{1.932975in}{1.953515in}}%
\pgfpathlineto{\pgfqpoint{1.916219in}{1.953789in}}%
\pgfpathlineto{\pgfqpoint{1.899462in}{1.954127in}}%
\pgfpathlineto{\pgfqpoint{1.882706in}{1.954493in}}%
\pgfpathlineto{\pgfqpoint{1.865950in}{1.954896in}}%
\pgfpathlineto{\pgfqpoint{1.849193in}{1.955226in}}%
\pgfpathlineto{\pgfqpoint{1.832437in}{1.955565in}}%
\pgfpathlineto{\pgfqpoint{1.815681in}{1.955963in}}%
\pgfpathlineto{\pgfqpoint{1.798925in}{1.956334in}}%
\pgfpathlineto{\pgfqpoint{1.782168in}{1.956700in}}%
\pgfpathlineto{\pgfqpoint{1.765412in}{1.957078in}}%
\pgfpathlineto{\pgfqpoint{1.748656in}{1.957479in}}%
\pgfpathlineto{\pgfqpoint{1.731900in}{1.957806in}}%
\pgfpathlineto{\pgfqpoint{1.715143in}{1.958217in}}%
\pgfpathlineto{\pgfqpoint{1.698387in}{1.958716in}}%
\pgfpathlineto{\pgfqpoint{1.681631in}{1.959217in}}%
\pgfpathlineto{\pgfqpoint{1.664874in}{1.959723in}}%
\pgfpathlineto{\pgfqpoint{1.648118in}{1.960228in}}%
\pgfpathlineto{\pgfqpoint{1.631362in}{1.960734in}}%
\pgfpathlineto{\pgfqpoint{1.614606in}{1.961235in}}%
\pgfpathlineto{\pgfqpoint{1.597849in}{1.961670in}}%
\pgfpathlineto{\pgfqpoint{1.581093in}{1.961994in}}%
\pgfpathlineto{\pgfqpoint{1.564337in}{1.962466in}}%
\pgfpathlineto{\pgfqpoint{1.547581in}{1.963032in}}%
\pgfpathlineto{\pgfqpoint{1.530824in}{1.963590in}}%
\pgfpathlineto{\pgfqpoint{1.514068in}{1.964019in}}%
\pgfpathlineto{\pgfqpoint{1.497312in}{1.964608in}}%
\pgfpathlineto{\pgfqpoint{1.480555in}{1.965134in}}%
\pgfpathlineto{\pgfqpoint{1.463799in}{1.965705in}}%
\pgfpathlineto{\pgfqpoint{1.447043in}{1.966271in}}%
\pgfpathlineto{\pgfqpoint{1.430287in}{1.966796in}}%
\pgfpathlineto{\pgfqpoint{1.413530in}{1.967347in}}%
\pgfpathlineto{\pgfqpoint{1.396774in}{1.967767in}}%
\pgfpathlineto{\pgfqpoint{1.380018in}{1.968179in}}%
\pgfpathlineto{\pgfqpoint{1.363262in}{1.968736in}}%
\pgfpathlineto{\pgfqpoint{1.346505in}{1.969295in}}%
\pgfpathlineto{\pgfqpoint{1.329749in}{1.969958in}}%
\pgfpathlineto{\pgfqpoint{1.312993in}{1.970642in}}%
\pgfpathlineto{\pgfqpoint{1.296236in}{1.971298in}}%
\pgfpathlineto{\pgfqpoint{1.279480in}{1.972060in}}%
\pgfpathlineto{\pgfqpoint{1.262724in}{1.972666in}}%
\pgfpathlineto{\pgfqpoint{1.245968in}{1.973233in}}%
\pgfpathlineto{\pgfqpoint{1.229211in}{1.973800in}}%
\pgfpathlineto{\pgfqpoint{1.212455in}{1.974425in}}%
\pgfpathlineto{\pgfqpoint{1.195699in}{1.975107in}}%
\pgfpathlineto{\pgfqpoint{1.178942in}{1.975664in}}%
\pgfpathlineto{\pgfqpoint{1.162186in}{1.976416in}}%
\pgfpathlineto{\pgfqpoint{1.145430in}{1.977039in}}%
\pgfpathlineto{\pgfqpoint{1.128674in}{1.977678in}}%
\pgfpathlineto{\pgfqpoint{1.111917in}{1.978305in}}%
\pgfpathlineto{\pgfqpoint{1.095161in}{1.979068in}}%
\pgfpathlineto{\pgfqpoint{1.078405in}{1.979628in}}%
\pgfpathlineto{\pgfqpoint{1.061649in}{1.980452in}}%
\pgfpathlineto{\pgfqpoint{1.044892in}{1.981201in}}%
\pgfpathlineto{\pgfqpoint{1.028136in}{1.981931in}}%
\pgfpathlineto{\pgfqpoint{1.011380in}{1.982664in}}%
\pgfpathlineto{\pgfqpoint{0.994623in}{1.983437in}}%
\pgfpathlineto{\pgfqpoint{0.977867in}{1.984205in}}%
\pgfpathlineto{\pgfqpoint{0.961111in}{1.984983in}}%
\pgfpathlineto{\pgfqpoint{0.944355in}{1.985759in}}%
\pgfpathlineto{\pgfqpoint{0.927598in}{1.986532in}}%
\pgfpathlineto{\pgfqpoint{0.910842in}{1.987355in}}%
\pgfpathlineto{\pgfqpoint{0.894086in}{1.988079in}}%
\pgfpathlineto{\pgfqpoint{0.877330in}{1.988819in}}%
\pgfpathlineto{\pgfqpoint{0.860573in}{1.989526in}}%
\pgfpathlineto{\pgfqpoint{0.843817in}{1.990231in}}%
\pgfpathlineto{\pgfqpoint{0.827061in}{1.990926in}}%
\pgfpathlineto{\pgfqpoint{0.810304in}{1.991721in}}%
\pgfpathlineto{\pgfqpoint{0.793548in}{1.992480in}}%
\pgfpathlineto{\pgfqpoint{0.776792in}{1.993309in}}%
\pgfpathlineto{\pgfqpoint{0.760036in}{1.994064in}}%
\pgfpathlineto{\pgfqpoint{0.743279in}{1.994858in}}%
\pgfpathlineto{\pgfqpoint{0.726523in}{1.995691in}}%
\pgfpathlineto{\pgfqpoint{0.709767in}{1.996499in}}%
\pgfpathlineto{\pgfqpoint{0.693011in}{1.997362in}}%
\pgfpathlineto{\pgfqpoint{0.676254in}{1.998157in}}%
\pgfpathlineto{\pgfqpoint{0.659498in}{1.999031in}}%
\pgfpathlineto{\pgfqpoint{0.642742in}{1.999874in}}%
\pgfpathlineto{\pgfqpoint{0.625985in}{2.000712in}}%
\pgfpathlineto{\pgfqpoint{0.609229in}{2.001545in}}%
\pgfpathlineto{\pgfqpoint{0.592473in}{2.002381in}}%
\pgfpathlineto{\pgfqpoint{0.575717in}{2.003218in}}%
\pgfpathlineto{\pgfqpoint{0.558960in}{2.004055in}}%
\pgfpathlineto{\pgfqpoint{0.542204in}{2.004891in}}%
\pgfpathlineto{\pgfqpoint{0.525448in}{2.005725in}}%
\pgfpathlineto{\pgfqpoint{0.508691in}{2.006582in}}%
\pgfpathlineto{\pgfqpoint{0.491935in}{2.007481in}}%
\pgfpathlineto{\pgfqpoint{0.475179in}{2.008390in}}%
\pgfpathlineto{\pgfqpoint{0.458423in}{2.009274in}}%
\pgfpathlineto{\pgfqpoint{0.441666in}{2.010172in}}%
\pgfpathlineto{\pgfqpoint{0.424910in}{2.011053in}}%
\pgfpathlineto{\pgfqpoint{0.408154in}{2.011957in}}%
\pgfpathlineto{\pgfqpoint{0.391398in}{2.012831in}}%
\pgfpathlineto{\pgfqpoint{0.374641in}{2.013645in}}%
\pgfpathlineto{\pgfqpoint{0.357885in}{2.014509in}}%
\pgfpathlineto{\pgfqpoint{0.341129in}{2.015390in}}%
\pgfpathclose%
\pgfusepath{stroke,fill}%
}%
\begin{pgfscope}%
\pgfsys@transformshift{0.000000in}{0.000000in}%
\pgfsys@useobject{currentmarker}{}%
\end{pgfscope}%
\end{pgfscope}%
\begin{pgfscope}%
\pgfpathrectangle{\pgfqpoint{0.341129in}{0.466613in}}{\pgfqpoint{1.658871in}{1.711598in}}%
\pgfusepath{clip}%
\pgfsetbuttcap%
\pgfsetroundjoin%
\definecolor{currentfill}{rgb}{0.866667,0.517647,0.321569}%
\pgfsetfillcolor{currentfill}%
\pgfsetfillopacity{0.250000}%
\pgfsetlinewidth{1.003750pt}%
\definecolor{currentstroke}{rgb}{0.866667,0.517647,0.321569}%
\pgfsetstrokecolor{currentstroke}%
\pgfsetstrokeopacity{0.250000}%
\pgfsetdash{}{0pt}%
\pgfsys@defobject{currentmarker}{\pgfqpoint{-0.017010in}{-0.017010in}}{\pgfqpoint{0.017010in}{0.017010in}}{%
\pgfpathmoveto{\pgfqpoint{0.000000in}{-0.017010in}}%
\pgfpathcurveto{\pgfqpoint{0.004511in}{-0.017010in}}{\pgfqpoint{0.008838in}{-0.015218in}}{\pgfqpoint{0.012028in}{-0.012028in}}%
\pgfpathcurveto{\pgfqpoint{0.015218in}{-0.008838in}}{\pgfqpoint{0.017010in}{-0.004511in}}{\pgfqpoint{0.017010in}{0.000000in}}%
\pgfpathcurveto{\pgfqpoint{0.017010in}{0.004511in}}{\pgfqpoint{0.015218in}{0.008838in}}{\pgfqpoint{0.012028in}{0.012028in}}%
\pgfpathcurveto{\pgfqpoint{0.008838in}{0.015218in}}{\pgfqpoint{0.004511in}{0.017010in}}{\pgfqpoint{0.000000in}{0.017010in}}%
\pgfpathcurveto{\pgfqpoint{-0.004511in}{0.017010in}}{\pgfqpoint{-0.008838in}{0.015218in}}{\pgfqpoint{-0.012028in}{0.012028in}}%
\pgfpathcurveto{\pgfqpoint{-0.015218in}{0.008838in}}{\pgfqpoint{-0.017010in}{0.004511in}}{\pgfqpoint{-0.017010in}{0.000000in}}%
\pgfpathcurveto{\pgfqpoint{-0.017010in}{-0.004511in}}{\pgfqpoint{-0.015218in}{-0.008838in}}{\pgfqpoint{-0.012028in}{-0.012028in}}%
\pgfpathcurveto{\pgfqpoint{-0.008838in}{-0.015218in}}{\pgfqpoint{-0.004511in}{-0.017010in}}{\pgfqpoint{0.000000in}{-0.017010in}}%
\pgfpathclose%
\pgfusepath{stroke,fill}%
}%
\begin{pgfscope}%
\pgfsys@transformshift{0.667456in}{1.853878in}%
\pgfsys@useobject{currentmarker}{}%
\end{pgfscope}%
\begin{pgfscope}%
\pgfsys@transformshift{0.872203in}{1.870876in}%
\pgfsys@useobject{currentmarker}{}%
\end{pgfscope}%
\begin{pgfscope}%
\pgfsys@transformshift{1.174456in}{1.824262in}%
\pgfsys@useobject{currentmarker}{}%
\end{pgfscope}%
\begin{pgfscope}%
\pgfsys@transformshift{1.594005in}{1.773824in}%
\pgfsys@useobject{currentmarker}{}%
\end{pgfscope}%
\begin{pgfscope}%
\pgfsys@transformshift{1.217413in}{1.671481in}%
\pgfsys@useobject{currentmarker}{}%
\end{pgfscope}%
\begin{pgfscope}%
\pgfsys@transformshift{0.553973in}{1.850265in}%
\pgfsys@useobject{currentmarker}{}%
\end{pgfscope}%
\begin{pgfscope}%
\pgfsys@transformshift{1.639332in}{1.597136in}%
\pgfsys@useobject{currentmarker}{}%
\end{pgfscope}%
\begin{pgfscope}%
\pgfsys@transformshift{0.709729in}{1.884226in}%
\pgfsys@useobject{currentmarker}{}%
\end{pgfscope}%
\begin{pgfscope}%
\pgfsys@transformshift{1.202352in}{1.851611in}%
\pgfsys@useobject{currentmarker}{}%
\end{pgfscope}%
\begin{pgfscope}%
\pgfsys@transformshift{0.671018in}{1.777038in}%
\pgfsys@useobject{currentmarker}{}%
\end{pgfscope}%
\begin{pgfscope}%
\pgfsys@transformshift{1.442003in}{1.787371in}%
\pgfsys@useobject{currentmarker}{}%
\end{pgfscope}%
\begin{pgfscope}%
\pgfsys@transformshift{1.398849in}{1.797193in}%
\pgfsys@useobject{currentmarker}{}%
\end{pgfscope}%
\begin{pgfscope}%
\pgfsys@transformshift{1.552752in}{1.644097in}%
\pgfsys@useobject{currentmarker}{}%
\end{pgfscope}%
\begin{pgfscope}%
\pgfsys@transformshift{1.177147in}{1.797127in}%
\pgfsys@useobject{currentmarker}{}%
\end{pgfscope}%
\begin{pgfscope}%
\pgfsys@transformshift{1.887294in}{1.817884in}%
\pgfsys@useobject{currentmarker}{}%
\end{pgfscope}%
\begin{pgfscope}%
\pgfsys@transformshift{1.188970in}{1.767800in}%
\pgfsys@useobject{currentmarker}{}%
\end{pgfscope}%
\begin{pgfscope}%
\pgfsys@transformshift{1.534509in}{1.813012in}%
\pgfsys@useobject{currentmarker}{}%
\end{pgfscope}%
\begin{pgfscope}%
\pgfsys@transformshift{0.588572in}{1.869226in}%
\pgfsys@useobject{currentmarker}{}%
\end{pgfscope}%
\begin{pgfscope}%
\pgfsys@transformshift{0.952296in}{1.822569in}%
\pgfsys@useobject{currentmarker}{}%
\end{pgfscope}%
\begin{pgfscope}%
\pgfsys@transformshift{0.705522in}{1.800630in}%
\pgfsys@useobject{currentmarker}{}%
\end{pgfscope}%
\begin{pgfscope}%
\pgfsys@transformshift{1.121384in}{1.753535in}%
\pgfsys@useobject{currentmarker}{}%
\end{pgfscope}%
\begin{pgfscope}%
\pgfsys@transformshift{1.103574in}{1.754495in}%
\pgfsys@useobject{currentmarker}{}%
\end{pgfscope}%
\begin{pgfscope}%
\pgfsys@transformshift{1.817005in}{1.799971in}%
\pgfsys@useobject{currentmarker}{}%
\end{pgfscope}%
\begin{pgfscope}%
\pgfsys@transformshift{0.496492in}{1.914116in}%
\pgfsys@useobject{currentmarker}{}%
\end{pgfscope}%
\begin{pgfscope}%
\pgfsys@transformshift{1.711311in}{1.715833in}%
\pgfsys@useobject{currentmarker}{}%
\end{pgfscope}%
\begin{pgfscope}%
\pgfsys@transformshift{1.750408in}{1.542286in}%
\pgfsys@useobject{currentmarker}{}%
\end{pgfscope}%
\begin{pgfscope}%
\pgfsys@transformshift{1.333406in}{1.702933in}%
\pgfsys@useobject{currentmarker}{}%
\end{pgfscope}%
\begin{pgfscope}%
\pgfsys@transformshift{1.724223in}{1.736787in}%
\pgfsys@useobject{currentmarker}{}%
\end{pgfscope}%
\begin{pgfscope}%
\pgfsys@transformshift{1.256999in}{1.792532in}%
\pgfsys@useobject{currentmarker}{}%
\end{pgfscope}%
\begin{pgfscope}%
\pgfsys@transformshift{1.047649in}{1.780836in}%
\pgfsys@useobject{currentmarker}{}%
\end{pgfscope}%
\begin{pgfscope}%
\pgfsys@transformshift{0.878614in}{1.848792in}%
\pgfsys@useobject{currentmarker}{}%
\end{pgfscope}%
\begin{pgfscope}%
\pgfsys@transformshift{0.761954in}{1.827778in}%
\pgfsys@useobject{currentmarker}{}%
\end{pgfscope}%
\begin{pgfscope}%
\pgfsys@transformshift{1.796268in}{1.655878in}%
\pgfsys@useobject{currentmarker}{}%
\end{pgfscope}%
\begin{pgfscope}%
\pgfsys@transformshift{0.726378in}{1.829398in}%
\pgfsys@useobject{currentmarker}{}%
\end{pgfscope}%
\begin{pgfscope}%
\pgfsys@transformshift{1.408196in}{1.783384in}%
\pgfsys@useobject{currentmarker}{}%
\end{pgfscope}%
\begin{pgfscope}%
\pgfsys@transformshift{0.612217in}{1.645384in}%
\pgfsys@useobject{currentmarker}{}%
\end{pgfscope}%
\begin{pgfscope}%
\pgfsys@transformshift{1.180902in}{1.870159in}%
\pgfsys@useobject{currentmarker}{}%
\end{pgfscope}%
\begin{pgfscope}%
\pgfsys@transformshift{0.627344in}{1.904541in}%
\pgfsys@useobject{currentmarker}{}%
\end{pgfscope}%
\begin{pgfscope}%
\pgfsys@transformshift{0.628282in}{1.820690in}%
\pgfsys@useobject{currentmarker}{}%
\end{pgfscope}%
\begin{pgfscope}%
\pgfsys@transformshift{0.437992in}{1.632140in}%
\pgfsys@useobject{currentmarker}{}%
\end{pgfscope}%
\begin{pgfscope}%
\pgfsys@transformshift{0.758691in}{1.817563in}%
\pgfsys@useobject{currentmarker}{}%
\end{pgfscope}%
\begin{pgfscope}%
\pgfsys@transformshift{1.614220in}{1.855956in}%
\pgfsys@useobject{currentmarker}{}%
\end{pgfscope}%
\begin{pgfscope}%
\pgfsys@transformshift{1.846563in}{1.747612in}%
\pgfsys@useobject{currentmarker}{}%
\end{pgfscope}%
\begin{pgfscope}%
\pgfsys@transformshift{1.460173in}{1.828953in}%
\pgfsys@useobject{currentmarker}{}%
\end{pgfscope}%
\begin{pgfscope}%
\pgfsys@transformshift{1.637112in}{1.741755in}%
\pgfsys@useobject{currentmarker}{}%
\end{pgfscope}%
\begin{pgfscope}%
\pgfsys@transformshift{1.615512in}{1.642194in}%
\pgfsys@useobject{currentmarker}{}%
\end{pgfscope}%
\begin{pgfscope}%
\pgfsys@transformshift{1.242112in}{1.902907in}%
\pgfsys@useobject{currentmarker}{}%
\end{pgfscope}%
\begin{pgfscope}%
\pgfsys@transformshift{0.688214in}{1.808450in}%
\pgfsys@useobject{currentmarker}{}%
\end{pgfscope}%
\begin{pgfscope}%
\pgfsys@transformshift{0.911315in}{1.636699in}%
\pgfsys@useobject{currentmarker}{}%
\end{pgfscope}%
\begin{pgfscope}%
\pgfsys@transformshift{0.551462in}{1.911314in}%
\pgfsys@useobject{currentmarker}{}%
\end{pgfscope}%
\begin{pgfscope}%
\pgfsys@transformshift{1.069973in}{1.789964in}%
\pgfsys@useobject{currentmarker}{}%
\end{pgfscope}%
\begin{pgfscope}%
\pgfsys@transformshift{1.211180in}{1.886377in}%
\pgfsys@useobject{currentmarker}{}%
\end{pgfscope}%
\begin{pgfscope}%
\pgfsys@transformshift{1.485757in}{1.897348in}%
\pgfsys@useobject{currentmarker}{}%
\end{pgfscope}%
\begin{pgfscope}%
\pgfsys@transformshift{0.743119in}{1.771878in}%
\pgfsys@useobject{currentmarker}{}%
\end{pgfscope}%
\begin{pgfscope}%
\pgfsys@transformshift{1.407385in}{1.705976in}%
\pgfsys@useobject{currentmarker}{}%
\end{pgfscope}%
\begin{pgfscope}%
\pgfsys@transformshift{1.618799in}{1.886708in}%
\pgfsys@useobject{currentmarker}{}%
\end{pgfscope}%
\begin{pgfscope}%
\pgfsys@transformshift{1.477965in}{1.721435in}%
\pgfsys@useobject{currentmarker}{}%
\end{pgfscope}%
\begin{pgfscope}%
\pgfsys@transformshift{1.156480in}{1.895913in}%
\pgfsys@useobject{currentmarker}{}%
\end{pgfscope}%
\begin{pgfscope}%
\pgfsys@transformshift{1.583657in}{1.729974in}%
\pgfsys@useobject{currentmarker}{}%
\end{pgfscope}%
\begin{pgfscope}%
\pgfsys@transformshift{1.440624in}{1.730373in}%
\pgfsys@useobject{currentmarker}{}%
\end{pgfscope}%
\begin{pgfscope}%
\pgfsys@transformshift{1.318935in}{1.725728in}%
\pgfsys@useobject{currentmarker}{}%
\end{pgfscope}%
\begin{pgfscope}%
\pgfsys@transformshift{0.547443in}{1.877853in}%
\pgfsys@useobject{currentmarker}{}%
\end{pgfscope}%
\begin{pgfscope}%
\pgfsys@transformshift{1.436824in}{1.799180in}%
\pgfsys@useobject{currentmarker}{}%
\end{pgfscope}%
\begin{pgfscope}%
\pgfsys@transformshift{1.658179in}{1.719356in}%
\pgfsys@useobject{currentmarker}{}%
\end{pgfscope}%
\begin{pgfscope}%
\pgfsys@transformshift{1.709942in}{1.768455in}%
\pgfsys@useobject{currentmarker}{}%
\end{pgfscope}%
\begin{pgfscope}%
\pgfsys@transformshift{1.638310in}{1.705393in}%
\pgfsys@useobject{currentmarker}{}%
\end{pgfscope}%
\begin{pgfscope}%
\pgfsys@transformshift{1.723779in}{1.752481in}%
\pgfsys@useobject{currentmarker}{}%
\end{pgfscope}%
\begin{pgfscope}%
\pgfsys@transformshift{1.124544in}{1.770491in}%
\pgfsys@useobject{currentmarker}{}%
\end{pgfscope}%
\begin{pgfscope}%
\pgfsys@transformshift{1.267442in}{1.798518in}%
\pgfsys@useobject{currentmarker}{}%
\end{pgfscope}%
\begin{pgfscope}%
\pgfsys@transformshift{0.562064in}{1.565216in}%
\pgfsys@useobject{currentmarker}{}%
\end{pgfscope}%
\begin{pgfscope}%
\pgfsys@transformshift{0.642722in}{1.814271in}%
\pgfsys@useobject{currentmarker}{}%
\end{pgfscope}%
\begin{pgfscope}%
\pgfsys@transformshift{1.150269in}{1.877966in}%
\pgfsys@useobject{currentmarker}{}%
\end{pgfscope}%
\begin{pgfscope}%
\pgfsys@transformshift{1.173281in}{1.772440in}%
\pgfsys@useobject{currentmarker}{}%
\end{pgfscope}%
\begin{pgfscope}%
\pgfsys@transformshift{1.891719in}{1.797533in}%
\pgfsys@useobject{currentmarker}{}%
\end{pgfscope}%
\begin{pgfscope}%
\pgfsys@transformshift{0.694241in}{1.832804in}%
\pgfsys@useobject{currentmarker}{}%
\end{pgfscope}%
\begin{pgfscope}%
\pgfsys@transformshift{0.790322in}{1.804854in}%
\pgfsys@useobject{currentmarker}{}%
\end{pgfscope}%
\begin{pgfscope}%
\pgfsys@transformshift{1.424550in}{1.698788in}%
\pgfsys@useobject{currentmarker}{}%
\end{pgfscope}%
\begin{pgfscope}%
\pgfsys@transformshift{0.384790in}{1.589666in}%
\pgfsys@useobject{currentmarker}{}%
\end{pgfscope}%
\begin{pgfscope}%
\pgfsys@transformshift{0.729116in}{1.842256in}%
\pgfsys@useobject{currentmarker}{}%
\end{pgfscope}%
\begin{pgfscope}%
\pgfsys@transformshift{1.217637in}{1.841930in}%
\pgfsys@useobject{currentmarker}{}%
\end{pgfscope}%
\begin{pgfscope}%
\pgfsys@transformshift{0.452540in}{1.840432in}%
\pgfsys@useobject{currentmarker}{}%
\end{pgfscope}%
\begin{pgfscope}%
\pgfsys@transformshift{1.493704in}{1.664280in}%
\pgfsys@useobject{currentmarker}{}%
\end{pgfscope}%
\begin{pgfscope}%
\pgfsys@transformshift{1.543302in}{1.788815in}%
\pgfsys@useobject{currentmarker}{}%
\end{pgfscope}%
\begin{pgfscope}%
\pgfsys@transformshift{1.218014in}{1.751878in}%
\pgfsys@useobject{currentmarker}{}%
\end{pgfscope}%
\begin{pgfscope}%
\pgfsys@transformshift{1.521861in}{1.755752in}%
\pgfsys@useobject{currentmarker}{}%
\end{pgfscope}%
\begin{pgfscope}%
\pgfsys@transformshift{1.764286in}{1.806752in}%
\pgfsys@useobject{currentmarker}{}%
\end{pgfscope}%
\begin{pgfscope}%
\pgfsys@transformshift{0.924889in}{1.903431in}%
\pgfsys@useobject{currentmarker}{}%
\end{pgfscope}%
\begin{pgfscope}%
\pgfsys@transformshift{1.222866in}{1.667574in}%
\pgfsys@useobject{currentmarker}{}%
\end{pgfscope}%
\begin{pgfscope}%
\pgfsys@transformshift{1.382222in}{1.748656in}%
\pgfsys@useobject{currentmarker}{}%
\end{pgfscope}%
\begin{pgfscope}%
\pgfsys@transformshift{1.643945in}{1.801930in}%
\pgfsys@useobject{currentmarker}{}%
\end{pgfscope}%
\begin{pgfscope}%
\pgfsys@transformshift{0.801834in}{1.772463in}%
\pgfsys@useobject{currentmarker}{}%
\end{pgfscope}%
\begin{pgfscope}%
\pgfsys@transformshift{1.258384in}{1.785251in}%
\pgfsys@useobject{currentmarker}{}%
\end{pgfscope}%
\begin{pgfscope}%
\pgfsys@transformshift{1.394783in}{1.733285in}%
\pgfsys@useobject{currentmarker}{}%
\end{pgfscope}%
\begin{pgfscope}%
\pgfsys@transformshift{0.341129in}{1.934874in}%
\pgfsys@useobject{currentmarker}{}%
\end{pgfscope}%
\begin{pgfscope}%
\pgfsys@transformshift{1.069581in}{1.856735in}%
\pgfsys@useobject{currentmarker}{}%
\end{pgfscope}%
\begin{pgfscope}%
\pgfsys@transformshift{0.789606in}{1.911917in}%
\pgfsys@useobject{currentmarker}{}%
\end{pgfscope}%
\begin{pgfscope}%
\pgfsys@transformshift{1.273287in}{1.733270in}%
\pgfsys@useobject{currentmarker}{}%
\end{pgfscope}%
\begin{pgfscope}%
\pgfsys@transformshift{0.646998in}{1.880413in}%
\pgfsys@useobject{currentmarker}{}%
\end{pgfscope}%
\begin{pgfscope}%
\pgfsys@transformshift{0.761065in}{1.873402in}%
\pgfsys@useobject{currentmarker}{}%
\end{pgfscope}%
\begin{pgfscope}%
\pgfsys@transformshift{0.628168in}{1.923232in}%
\pgfsys@useobject{currentmarker}{}%
\end{pgfscope}%
\begin{pgfscope}%
\pgfsys@transformshift{1.214800in}{1.809894in}%
\pgfsys@useobject{currentmarker}{}%
\end{pgfscope}%
\begin{pgfscope}%
\pgfsys@transformshift{1.051231in}{1.865852in}%
\pgfsys@useobject{currentmarker}{}%
\end{pgfscope}%
\begin{pgfscope}%
\pgfsys@transformshift{1.010738in}{1.791330in}%
\pgfsys@useobject{currentmarker}{}%
\end{pgfscope}%
\begin{pgfscope}%
\pgfsys@transformshift{1.387136in}{1.682008in}%
\pgfsys@useobject{currentmarker}{}%
\end{pgfscope}%
\begin{pgfscope}%
\pgfsys@transformshift{0.522501in}{1.585657in}%
\pgfsys@useobject{currentmarker}{}%
\end{pgfscope}%
\begin{pgfscope}%
\pgfsys@transformshift{1.737851in}{1.828506in}%
\pgfsys@useobject{currentmarker}{}%
\end{pgfscope}%
\begin{pgfscope}%
\pgfsys@transformshift{1.622475in}{1.751950in}%
\pgfsys@useobject{currentmarker}{}%
\end{pgfscope}%
\begin{pgfscope}%
\pgfsys@transformshift{0.776054in}{1.618330in}%
\pgfsys@useobject{currentmarker}{}%
\end{pgfscope}%
\begin{pgfscope}%
\pgfsys@transformshift{1.260864in}{1.603112in}%
\pgfsys@useobject{currentmarker}{}%
\end{pgfscope}%
\begin{pgfscope}%
\pgfsys@transformshift{1.587232in}{1.711021in}%
\pgfsys@useobject{currentmarker}{}%
\end{pgfscope}%
\begin{pgfscope}%
\pgfsys@transformshift{1.339693in}{1.792207in}%
\pgfsys@useobject{currentmarker}{}%
\end{pgfscope}%
\begin{pgfscope}%
\pgfsys@transformshift{1.448928in}{1.637815in}%
\pgfsys@useobject{currentmarker}{}%
\end{pgfscope}%
\begin{pgfscope}%
\pgfsys@transformshift{1.848941in}{1.594812in}%
\pgfsys@useobject{currentmarker}{}%
\end{pgfscope}%
\begin{pgfscope}%
\pgfsys@transformshift{1.645721in}{1.763256in}%
\pgfsys@useobject{currentmarker}{}%
\end{pgfscope}%
\begin{pgfscope}%
\pgfsys@transformshift{0.980835in}{1.843999in}%
\pgfsys@useobject{currentmarker}{}%
\end{pgfscope}%
\begin{pgfscope}%
\pgfsys@transformshift{0.704938in}{1.906898in}%
\pgfsys@useobject{currentmarker}{}%
\end{pgfscope}%
\begin{pgfscope}%
\pgfsys@transformshift{0.622306in}{1.897486in}%
\pgfsys@useobject{currentmarker}{}%
\end{pgfscope}%
\begin{pgfscope}%
\pgfsys@transformshift{1.213744in}{1.635704in}%
\pgfsys@useobject{currentmarker}{}%
\end{pgfscope}%
\begin{pgfscope}%
\pgfsys@transformshift{0.934075in}{1.858499in}%
\pgfsys@useobject{currentmarker}{}%
\end{pgfscope}%
\begin{pgfscope}%
\pgfsys@transformshift{1.307417in}{1.661858in}%
\pgfsys@useobject{currentmarker}{}%
\end{pgfscope}%
\begin{pgfscope}%
\pgfsys@transformshift{1.605349in}{1.872697in}%
\pgfsys@useobject{currentmarker}{}%
\end{pgfscope}%
\begin{pgfscope}%
\pgfsys@transformshift{0.750394in}{1.837975in}%
\pgfsys@useobject{currentmarker}{}%
\end{pgfscope}%
\begin{pgfscope}%
\pgfsys@transformshift{1.771258in}{1.801933in}%
\pgfsys@useobject{currentmarker}{}%
\end{pgfscope}%
\begin{pgfscope}%
\pgfsys@transformshift{1.939172in}{1.762550in}%
\pgfsys@useobject{currentmarker}{}%
\end{pgfscope}%
\begin{pgfscope}%
\pgfsys@transformshift{1.599583in}{1.835901in}%
\pgfsys@useobject{currentmarker}{}%
\end{pgfscope}%
\begin{pgfscope}%
\pgfsys@transformshift{0.881257in}{1.840416in}%
\pgfsys@useobject{currentmarker}{}%
\end{pgfscope}%
\begin{pgfscope}%
\pgfsys@transformshift{1.201768in}{1.884457in}%
\pgfsys@useobject{currentmarker}{}%
\end{pgfscope}%
\begin{pgfscope}%
\pgfsys@transformshift{1.244134in}{1.826386in}%
\pgfsys@useobject{currentmarker}{}%
\end{pgfscope}%
\begin{pgfscope}%
\pgfsys@transformshift{1.750150in}{1.821193in}%
\pgfsys@useobject{currentmarker}{}%
\end{pgfscope}%
\begin{pgfscope}%
\pgfsys@transformshift{1.541153in}{1.741608in}%
\pgfsys@useobject{currentmarker}{}%
\end{pgfscope}%
\begin{pgfscope}%
\pgfsys@transformshift{0.964663in}{1.670583in}%
\pgfsys@useobject{currentmarker}{}%
\end{pgfscope}%
\begin{pgfscope}%
\pgfsys@transformshift{0.818109in}{1.813327in}%
\pgfsys@useobject{currentmarker}{}%
\end{pgfscope}%
\begin{pgfscope}%
\pgfsys@transformshift{1.755264in}{1.742375in}%
\pgfsys@useobject{currentmarker}{}%
\end{pgfscope}%
\begin{pgfscope}%
\pgfsys@transformshift{1.450480in}{1.820336in}%
\pgfsys@useobject{currentmarker}{}%
\end{pgfscope}%
\begin{pgfscope}%
\pgfsys@transformshift{1.246595in}{1.862860in}%
\pgfsys@useobject{currentmarker}{}%
\end{pgfscope}%
\begin{pgfscope}%
\pgfsys@transformshift{1.110348in}{1.837206in}%
\pgfsys@useobject{currentmarker}{}%
\end{pgfscope}%
\begin{pgfscope}%
\pgfsys@transformshift{0.853154in}{1.766356in}%
\pgfsys@useobject{currentmarker}{}%
\end{pgfscope}%
\begin{pgfscope}%
\pgfsys@transformshift{0.803206in}{1.866017in}%
\pgfsys@useobject{currentmarker}{}%
\end{pgfscope}%
\begin{pgfscope}%
\pgfsys@transformshift{0.800584in}{1.837566in}%
\pgfsys@useobject{currentmarker}{}%
\end{pgfscope}%
\begin{pgfscope}%
\pgfsys@transformshift{0.828786in}{1.826446in}%
\pgfsys@useobject{currentmarker}{}%
\end{pgfscope}%
\begin{pgfscope}%
\pgfsys@transformshift{0.802472in}{1.899346in}%
\pgfsys@useobject{currentmarker}{}%
\end{pgfscope}%
\begin{pgfscope}%
\pgfsys@transformshift{0.778610in}{1.880408in}%
\pgfsys@useobject{currentmarker}{}%
\end{pgfscope}%
\begin{pgfscope}%
\pgfsys@transformshift{1.678430in}{1.773211in}%
\pgfsys@useobject{currentmarker}{}%
\end{pgfscope}%
\begin{pgfscope}%
\pgfsys@transformshift{1.678534in}{1.808027in}%
\pgfsys@useobject{currentmarker}{}%
\end{pgfscope}%
\begin{pgfscope}%
\pgfsys@transformshift{1.164510in}{1.783030in}%
\pgfsys@useobject{currentmarker}{}%
\end{pgfscope}%
\begin{pgfscope}%
\pgfsys@transformshift{1.533885in}{1.778475in}%
\pgfsys@useobject{currentmarker}{}%
\end{pgfscope}%
\begin{pgfscope}%
\pgfsys@transformshift{0.532239in}{1.841508in}%
\pgfsys@useobject{currentmarker}{}%
\end{pgfscope}%
\begin{pgfscope}%
\pgfsys@transformshift{1.059630in}{1.738529in}%
\pgfsys@useobject{currentmarker}{}%
\end{pgfscope}%
\begin{pgfscope}%
\pgfsys@transformshift{1.471016in}{1.746376in}%
\pgfsys@useobject{currentmarker}{}%
\end{pgfscope}%
\begin{pgfscope}%
\pgfsys@transformshift{0.764422in}{1.806568in}%
\pgfsys@useobject{currentmarker}{}%
\end{pgfscope}%
\begin{pgfscope}%
\pgfsys@transformshift{1.724626in}{1.693719in}%
\pgfsys@useobject{currentmarker}{}%
\end{pgfscope}%
\begin{pgfscope}%
\pgfsys@transformshift{1.557866in}{1.687696in}%
\pgfsys@useobject{currentmarker}{}%
\end{pgfscope}%
\begin{pgfscope}%
\pgfsys@transformshift{1.322665in}{1.701703in}%
\pgfsys@useobject{currentmarker}{}%
\end{pgfscope}%
\begin{pgfscope}%
\pgfsys@transformshift{1.043939in}{1.826742in}%
\pgfsys@useobject{currentmarker}{}%
\end{pgfscope}%
\begin{pgfscope}%
\pgfsys@transformshift{1.615078in}{1.842315in}%
\pgfsys@useobject{currentmarker}{}%
\end{pgfscope}%
\begin{pgfscope}%
\pgfsys@transformshift{1.467591in}{1.861521in}%
\pgfsys@useobject{currentmarker}{}%
\end{pgfscope}%
\begin{pgfscope}%
\pgfsys@transformshift{0.991846in}{1.766810in}%
\pgfsys@useobject{currentmarker}{}%
\end{pgfscope}%
\begin{pgfscope}%
\pgfsys@transformshift{0.697891in}{1.792414in}%
\pgfsys@useobject{currentmarker}{}%
\end{pgfscope}%
\begin{pgfscope}%
\pgfsys@transformshift{1.294350in}{1.902063in}%
\pgfsys@useobject{currentmarker}{}%
\end{pgfscope}%
\begin{pgfscope}%
\pgfsys@transformshift{1.409354in}{1.661338in}%
\pgfsys@useobject{currentmarker}{}%
\end{pgfscope}%
\begin{pgfscope}%
\pgfsys@transformshift{0.733011in}{1.928243in}%
\pgfsys@useobject{currentmarker}{}%
\end{pgfscope}%
\begin{pgfscope}%
\pgfsys@transformshift{0.634257in}{1.833417in}%
\pgfsys@useobject{currentmarker}{}%
\end{pgfscope}%
\begin{pgfscope}%
\pgfsys@transformshift{0.526826in}{1.863186in}%
\pgfsys@useobject{currentmarker}{}%
\end{pgfscope}%
\begin{pgfscope}%
\pgfsys@transformshift{1.225515in}{1.828744in}%
\pgfsys@useobject{currentmarker}{}%
\end{pgfscope}%
\begin{pgfscope}%
\pgfsys@transformshift{0.578368in}{1.698133in}%
\pgfsys@useobject{currentmarker}{}%
\end{pgfscope}%
\begin{pgfscope}%
\pgfsys@transformshift{1.515054in}{1.799490in}%
\pgfsys@useobject{currentmarker}{}%
\end{pgfscope}%
\begin{pgfscope}%
\pgfsys@transformshift{1.989373in}{1.763487in}%
\pgfsys@useobject{currentmarker}{}%
\end{pgfscope}%
\begin{pgfscope}%
\pgfsys@transformshift{1.360084in}{1.902366in}%
\pgfsys@useobject{currentmarker}{}%
\end{pgfscope}%
\begin{pgfscope}%
\pgfsys@transformshift{0.851541in}{1.813917in}%
\pgfsys@useobject{currentmarker}{}%
\end{pgfscope}%
\begin{pgfscope}%
\pgfsys@transformshift{1.688966in}{1.789156in}%
\pgfsys@useobject{currentmarker}{}%
\end{pgfscope}%
\begin{pgfscope}%
\pgfsys@transformshift{1.144955in}{1.789108in}%
\pgfsys@useobject{currentmarker}{}%
\end{pgfscope}%
\begin{pgfscope}%
\pgfsys@transformshift{1.681418in}{1.687850in}%
\pgfsys@useobject{currentmarker}{}%
\end{pgfscope}%
\begin{pgfscope}%
\pgfsys@transformshift{1.133992in}{1.895095in}%
\pgfsys@useobject{currentmarker}{}%
\end{pgfscope}%
\begin{pgfscope}%
\pgfsys@transformshift{1.480556in}{1.788771in}%
\pgfsys@useobject{currentmarker}{}%
\end{pgfscope}%
\begin{pgfscope}%
\pgfsys@transformshift{1.339321in}{1.821694in}%
\pgfsys@useobject{currentmarker}{}%
\end{pgfscope}%
\begin{pgfscope}%
\pgfsys@transformshift{0.874156in}{1.760774in}%
\pgfsys@useobject{currentmarker}{}%
\end{pgfscope}%
\begin{pgfscope}%
\pgfsys@transformshift{1.592077in}{1.671690in}%
\pgfsys@useobject{currentmarker}{}%
\end{pgfscope}%
\begin{pgfscope}%
\pgfsys@transformshift{1.288980in}{1.819687in}%
\pgfsys@useobject{currentmarker}{}%
\end{pgfscope}%
\begin{pgfscope}%
\pgfsys@transformshift{0.987975in}{1.720226in}%
\pgfsys@useobject{currentmarker}{}%
\end{pgfscope}%
\begin{pgfscope}%
\pgfsys@transformshift{1.714750in}{1.726763in}%
\pgfsys@useobject{currentmarker}{}%
\end{pgfscope}%
\begin{pgfscope}%
\pgfsys@transformshift{1.365090in}{1.777832in}%
\pgfsys@useobject{currentmarker}{}%
\end{pgfscope}%
\begin{pgfscope}%
\pgfsys@transformshift{1.930143in}{1.739403in}%
\pgfsys@useobject{currentmarker}{}%
\end{pgfscope}%
\begin{pgfscope}%
\pgfsys@transformshift{1.850728in}{1.626364in}%
\pgfsys@useobject{currentmarker}{}%
\end{pgfscope}%
\begin{pgfscope}%
\pgfsys@transformshift{0.856167in}{1.899440in}%
\pgfsys@useobject{currentmarker}{}%
\end{pgfscope}%
\begin{pgfscope}%
\pgfsys@transformshift{1.724113in}{1.781383in}%
\pgfsys@useobject{currentmarker}{}%
\end{pgfscope}%
\begin{pgfscope}%
\pgfsys@transformshift{0.746656in}{1.883472in}%
\pgfsys@useobject{currentmarker}{}%
\end{pgfscope}%
\begin{pgfscope}%
\pgfsys@transformshift{0.981125in}{1.867558in}%
\pgfsys@useobject{currentmarker}{}%
\end{pgfscope}%
\begin{pgfscope}%
\pgfsys@transformshift{0.840360in}{1.767845in}%
\pgfsys@useobject{currentmarker}{}%
\end{pgfscope}%
\begin{pgfscope}%
\pgfsys@transformshift{0.780491in}{1.791464in}%
\pgfsys@useobject{currentmarker}{}%
\end{pgfscope}%
\begin{pgfscope}%
\pgfsys@transformshift{1.873816in}{1.764318in}%
\pgfsys@useobject{currentmarker}{}%
\end{pgfscope}%
\begin{pgfscope}%
\pgfsys@transformshift{1.594601in}{1.791753in}%
\pgfsys@useobject{currentmarker}{}%
\end{pgfscope}%
\begin{pgfscope}%
\pgfsys@transformshift{0.815736in}{1.819599in}%
\pgfsys@useobject{currentmarker}{}%
\end{pgfscope}%
\begin{pgfscope}%
\pgfsys@transformshift{1.864796in}{1.568461in}%
\pgfsys@useobject{currentmarker}{}%
\end{pgfscope}%
\begin{pgfscope}%
\pgfsys@transformshift{1.367631in}{1.881338in}%
\pgfsys@useobject{currentmarker}{}%
\end{pgfscope}%
\begin{pgfscope}%
\pgfsys@transformshift{1.653345in}{1.811524in}%
\pgfsys@useobject{currentmarker}{}%
\end{pgfscope}%
\begin{pgfscope}%
\pgfsys@transformshift{1.768545in}{1.737756in}%
\pgfsys@useobject{currentmarker}{}%
\end{pgfscope}%
\begin{pgfscope}%
\pgfsys@transformshift{1.244730in}{1.776186in}%
\pgfsys@useobject{currentmarker}{}%
\end{pgfscope}%
\begin{pgfscope}%
\pgfsys@transformshift{0.782047in}{1.851976in}%
\pgfsys@useobject{currentmarker}{}%
\end{pgfscope}%
\begin{pgfscope}%
\pgfsys@transformshift{1.846677in}{1.759701in}%
\pgfsys@useobject{currentmarker}{}%
\end{pgfscope}%
\begin{pgfscope}%
\pgfsys@transformshift{1.775739in}{1.752466in}%
\pgfsys@useobject{currentmarker}{}%
\end{pgfscope}%
\begin{pgfscope}%
\pgfsys@transformshift{1.703721in}{1.789607in}%
\pgfsys@useobject{currentmarker}{}%
\end{pgfscope}%
\begin{pgfscope}%
\pgfsys@transformshift{0.834121in}{1.830322in}%
\pgfsys@useobject{currentmarker}{}%
\end{pgfscope}%
\begin{pgfscope}%
\pgfsys@transformshift{1.119982in}{1.842049in}%
\pgfsys@useobject{currentmarker}{}%
\end{pgfscope}%
\begin{pgfscope}%
\pgfsys@transformshift{1.891923in}{1.647793in}%
\pgfsys@useobject{currentmarker}{}%
\end{pgfscope}%
\begin{pgfscope}%
\pgfsys@transformshift{1.690465in}{1.604716in}%
\pgfsys@useobject{currentmarker}{}%
\end{pgfscope}%
\begin{pgfscope}%
\pgfsys@transformshift{0.559984in}{1.875267in}%
\pgfsys@useobject{currentmarker}{}%
\end{pgfscope}%
\begin{pgfscope}%
\pgfsys@transformshift{1.447613in}{1.783365in}%
\pgfsys@useobject{currentmarker}{}%
\end{pgfscope}%
\begin{pgfscope}%
\pgfsys@transformshift{0.897928in}{1.887474in}%
\pgfsys@useobject{currentmarker}{}%
\end{pgfscope}%
\begin{pgfscope}%
\pgfsys@transformshift{1.417775in}{1.793836in}%
\pgfsys@useobject{currentmarker}{}%
\end{pgfscope}%
\begin{pgfscope}%
\pgfsys@transformshift{1.673252in}{1.710714in}%
\pgfsys@useobject{currentmarker}{}%
\end{pgfscope}%
\begin{pgfscope}%
\pgfsys@transformshift{1.412143in}{1.631359in}%
\pgfsys@useobject{currentmarker}{}%
\end{pgfscope}%
\begin{pgfscope}%
\pgfsys@transformshift{1.641912in}{1.792385in}%
\pgfsys@useobject{currentmarker}{}%
\end{pgfscope}%
\begin{pgfscope}%
\pgfsys@transformshift{1.388318in}{1.895245in}%
\pgfsys@useobject{currentmarker}{}%
\end{pgfscope}%
\begin{pgfscope}%
\pgfsys@transformshift{1.495487in}{1.762517in}%
\pgfsys@useobject{currentmarker}{}%
\end{pgfscope}%
\begin{pgfscope}%
\pgfsys@transformshift{1.918935in}{1.606382in}%
\pgfsys@useobject{currentmarker}{}%
\end{pgfscope}%
\begin{pgfscope}%
\pgfsys@transformshift{0.550025in}{1.802764in}%
\pgfsys@useobject{currentmarker}{}%
\end{pgfscope}%
\begin{pgfscope}%
\pgfsys@transformshift{1.821318in}{1.673821in}%
\pgfsys@useobject{currentmarker}{}%
\end{pgfscope}%
\begin{pgfscope}%
\pgfsys@transformshift{0.956955in}{1.781238in}%
\pgfsys@useobject{currentmarker}{}%
\end{pgfscope}%
\begin{pgfscope}%
\pgfsys@transformshift{0.895984in}{1.818982in}%
\pgfsys@useobject{currentmarker}{}%
\end{pgfscope}%
\begin{pgfscope}%
\pgfsys@transformshift{0.467116in}{1.656980in}%
\pgfsys@useobject{currentmarker}{}%
\end{pgfscope}%
\begin{pgfscope}%
\pgfsys@transformshift{1.606478in}{1.804335in}%
\pgfsys@useobject{currentmarker}{}%
\end{pgfscope}%
\begin{pgfscope}%
\pgfsys@transformshift{1.423682in}{1.833249in}%
\pgfsys@useobject{currentmarker}{}%
\end{pgfscope}%
\begin{pgfscope}%
\pgfsys@transformshift{1.436312in}{1.803244in}%
\pgfsys@useobject{currentmarker}{}%
\end{pgfscope}%
\begin{pgfscope}%
\pgfsys@transformshift{0.970257in}{1.923577in}%
\pgfsys@useobject{currentmarker}{}%
\end{pgfscope}%
\begin{pgfscope}%
\pgfsys@transformshift{1.482495in}{1.809417in}%
\pgfsys@useobject{currentmarker}{}%
\end{pgfscope}%
\begin{pgfscope}%
\pgfsys@transformshift{1.363658in}{1.859971in}%
\pgfsys@useobject{currentmarker}{}%
\end{pgfscope}%
\begin{pgfscope}%
\pgfsys@transformshift{1.566350in}{1.838121in}%
\pgfsys@useobject{currentmarker}{}%
\end{pgfscope}%
\begin{pgfscope}%
\pgfsys@transformshift{0.737168in}{1.778000in}%
\pgfsys@useobject{currentmarker}{}%
\end{pgfscope}%
\begin{pgfscope}%
\pgfsys@transformshift{0.416424in}{1.616473in}%
\pgfsys@useobject{currentmarker}{}%
\end{pgfscope}%
\begin{pgfscope}%
\pgfsys@transformshift{0.524334in}{1.807108in}%
\pgfsys@useobject{currentmarker}{}%
\end{pgfscope}%
\begin{pgfscope}%
\pgfsys@transformshift{0.739043in}{1.651043in}%
\pgfsys@useobject{currentmarker}{}%
\end{pgfscope}%
\begin{pgfscope}%
\pgfsys@transformshift{1.840100in}{1.706422in}%
\pgfsys@useobject{currentmarker}{}%
\end{pgfscope}%
\begin{pgfscope}%
\pgfsys@transformshift{1.824568in}{1.779314in}%
\pgfsys@useobject{currentmarker}{}%
\end{pgfscope}%
\begin{pgfscope}%
\pgfsys@transformshift{1.470009in}{1.807537in}%
\pgfsys@useobject{currentmarker}{}%
\end{pgfscope}%
\begin{pgfscope}%
\pgfsys@transformshift{0.471982in}{1.824030in}%
\pgfsys@useobject{currentmarker}{}%
\end{pgfscope}%
\begin{pgfscope}%
\pgfsys@transformshift{1.573451in}{1.922562in}%
\pgfsys@useobject{currentmarker}{}%
\end{pgfscope}%
\begin{pgfscope}%
\pgfsys@transformshift{1.738247in}{1.761497in}%
\pgfsys@useobject{currentmarker}{}%
\end{pgfscope}%
\begin{pgfscope}%
\pgfsys@transformshift{1.607428in}{1.654501in}%
\pgfsys@useobject{currentmarker}{}%
\end{pgfscope}%
\begin{pgfscope}%
\pgfsys@transformshift{1.759927in}{1.594435in}%
\pgfsys@useobject{currentmarker}{}%
\end{pgfscope}%
\begin{pgfscope}%
\pgfsys@transformshift{1.551624in}{1.693741in}%
\pgfsys@useobject{currentmarker}{}%
\end{pgfscope}%
\begin{pgfscope}%
\pgfsys@transformshift{1.356607in}{1.834225in}%
\pgfsys@useobject{currentmarker}{}%
\end{pgfscope}%
\begin{pgfscope}%
\pgfsys@transformshift{1.649686in}{1.526941in}%
\pgfsys@useobject{currentmarker}{}%
\end{pgfscope}%
\begin{pgfscope}%
\pgfsys@transformshift{1.305691in}{1.739869in}%
\pgfsys@useobject{currentmarker}{}%
\end{pgfscope}%
\begin{pgfscope}%
\pgfsys@transformshift{1.393587in}{1.688216in}%
\pgfsys@useobject{currentmarker}{}%
\end{pgfscope}%
\begin{pgfscope}%
\pgfsys@transformshift{1.153074in}{1.717077in}%
\pgfsys@useobject{currentmarker}{}%
\end{pgfscope}%
\begin{pgfscope}%
\pgfsys@transformshift{1.150217in}{1.834827in}%
\pgfsys@useobject{currentmarker}{}%
\end{pgfscope}%
\begin{pgfscope}%
\pgfsys@transformshift{0.541563in}{1.539164in}%
\pgfsys@useobject{currentmarker}{}%
\end{pgfscope}%
\begin{pgfscope}%
\pgfsys@transformshift{0.907947in}{1.820354in}%
\pgfsys@useobject{currentmarker}{}%
\end{pgfscope}%
\begin{pgfscope}%
\pgfsys@transformshift{1.404883in}{1.795930in}%
\pgfsys@useobject{currentmarker}{}%
\end{pgfscope}%
\begin{pgfscope}%
\pgfsys@transformshift{1.766444in}{1.667002in}%
\pgfsys@useobject{currentmarker}{}%
\end{pgfscope}%
\begin{pgfscope}%
\pgfsys@transformshift{1.727091in}{1.826485in}%
\pgfsys@useobject{currentmarker}{}%
\end{pgfscope}%
\begin{pgfscope}%
\pgfsys@transformshift{1.794191in}{1.830126in}%
\pgfsys@useobject{currentmarker}{}%
\end{pgfscope}%
\begin{pgfscope}%
\pgfsys@transformshift{1.006287in}{1.836818in}%
\pgfsys@useobject{currentmarker}{}%
\end{pgfscope}%
\begin{pgfscope}%
\pgfsys@transformshift{1.581569in}{1.767833in}%
\pgfsys@useobject{currentmarker}{}%
\end{pgfscope}%
\begin{pgfscope}%
\pgfsys@transformshift{0.941329in}{1.828689in}%
\pgfsys@useobject{currentmarker}{}%
\end{pgfscope}%
\begin{pgfscope}%
\pgfsys@transformshift{0.691952in}{1.868462in}%
\pgfsys@useobject{currentmarker}{}%
\end{pgfscope}%
\begin{pgfscope}%
\pgfsys@transformshift{1.057616in}{1.888001in}%
\pgfsys@useobject{currentmarker}{}%
\end{pgfscope}%
\begin{pgfscope}%
\pgfsys@transformshift{1.673517in}{1.793686in}%
\pgfsys@useobject{currentmarker}{}%
\end{pgfscope}%
\begin{pgfscope}%
\pgfsys@transformshift{1.477276in}{1.870968in}%
\pgfsys@useobject{currentmarker}{}%
\end{pgfscope}%
\begin{pgfscope}%
\pgfsys@transformshift{1.418685in}{1.822049in}%
\pgfsys@useobject{currentmarker}{}%
\end{pgfscope}%
\begin{pgfscope}%
\pgfsys@transformshift{1.590325in}{1.673372in}%
\pgfsys@useobject{currentmarker}{}%
\end{pgfscope}%
\begin{pgfscope}%
\pgfsys@transformshift{0.779671in}{1.875882in}%
\pgfsys@useobject{currentmarker}{}%
\end{pgfscope}%
\begin{pgfscope}%
\pgfsys@transformshift{1.674297in}{1.721628in}%
\pgfsys@useobject{currentmarker}{}%
\end{pgfscope}%
\begin{pgfscope}%
\pgfsys@transformshift{0.676244in}{1.847121in}%
\pgfsys@useobject{currentmarker}{}%
\end{pgfscope}%
\begin{pgfscope}%
\pgfsys@transformshift{1.715152in}{1.682577in}%
\pgfsys@useobject{currentmarker}{}%
\end{pgfscope}%
\begin{pgfscope}%
\pgfsys@transformshift{0.491079in}{1.627829in}%
\pgfsys@useobject{currentmarker}{}%
\end{pgfscope}%
\begin{pgfscope}%
\pgfsys@transformshift{1.084356in}{1.914038in}%
\pgfsys@useobject{currentmarker}{}%
\end{pgfscope}%
\begin{pgfscope}%
\pgfsys@transformshift{0.525316in}{1.632381in}%
\pgfsys@useobject{currentmarker}{}%
\end{pgfscope}%
\begin{pgfscope}%
\pgfsys@transformshift{0.530939in}{1.643701in}%
\pgfsys@useobject{currentmarker}{}%
\end{pgfscope}%
\begin{pgfscope}%
\pgfsys@transformshift{0.632029in}{1.891754in}%
\pgfsys@useobject{currentmarker}{}%
\end{pgfscope}%
\begin{pgfscope}%
\pgfsys@transformshift{0.952115in}{1.918927in}%
\pgfsys@useobject{currentmarker}{}%
\end{pgfscope}%
\begin{pgfscope}%
\pgfsys@transformshift{1.781361in}{1.585413in}%
\pgfsys@useobject{currentmarker}{}%
\end{pgfscope}%
\begin{pgfscope}%
\pgfsys@transformshift{0.893435in}{1.754640in}%
\pgfsys@useobject{currentmarker}{}%
\end{pgfscope}%
\begin{pgfscope}%
\pgfsys@transformshift{1.630080in}{1.671755in}%
\pgfsys@useobject{currentmarker}{}%
\end{pgfscope}%
\begin{pgfscope}%
\pgfsys@transformshift{1.495001in}{1.800711in}%
\pgfsys@useobject{currentmarker}{}%
\end{pgfscope}%
\begin{pgfscope}%
\pgfsys@transformshift{1.807197in}{1.786041in}%
\pgfsys@useobject{currentmarker}{}%
\end{pgfscope}%
\begin{pgfscope}%
\pgfsys@transformshift{0.872166in}{1.807215in}%
\pgfsys@useobject{currentmarker}{}%
\end{pgfscope}%
\begin{pgfscope}%
\pgfsys@transformshift{1.447090in}{1.907124in}%
\pgfsys@useobject{currentmarker}{}%
\end{pgfscope}%
\begin{pgfscope}%
\pgfsys@transformshift{1.534740in}{1.773046in}%
\pgfsys@useobject{currentmarker}{}%
\end{pgfscope}%
\begin{pgfscope}%
\pgfsys@transformshift{0.648228in}{1.960285in}%
\pgfsys@useobject{currentmarker}{}%
\end{pgfscope}%
\begin{pgfscope}%
\pgfsys@transformshift{0.939750in}{1.630735in}%
\pgfsys@useobject{currentmarker}{}%
\end{pgfscope}%
\begin{pgfscope}%
\pgfsys@transformshift{0.437103in}{1.667760in}%
\pgfsys@useobject{currentmarker}{}%
\end{pgfscope}%
\begin{pgfscope}%
\pgfsys@transformshift{1.534965in}{1.808092in}%
\pgfsys@useobject{currentmarker}{}%
\end{pgfscope}%
\begin{pgfscope}%
\pgfsys@transformshift{1.227246in}{1.898521in}%
\pgfsys@useobject{currentmarker}{}%
\end{pgfscope}%
\begin{pgfscope}%
\pgfsys@transformshift{1.816498in}{1.663276in}%
\pgfsys@useobject{currentmarker}{}%
\end{pgfscope}%
\begin{pgfscope}%
\pgfsys@transformshift{0.733769in}{1.853717in}%
\pgfsys@useobject{currentmarker}{}%
\end{pgfscope}%
\begin{pgfscope}%
\pgfsys@transformshift{0.664484in}{1.781498in}%
\pgfsys@useobject{currentmarker}{}%
\end{pgfscope}%
\begin{pgfscope}%
\pgfsys@transformshift{1.076644in}{1.887069in}%
\pgfsys@useobject{currentmarker}{}%
\end{pgfscope}%
\begin{pgfscope}%
\pgfsys@transformshift{1.548521in}{1.852859in}%
\pgfsys@useobject{currentmarker}{}%
\end{pgfscope}%
\begin{pgfscope}%
\pgfsys@transformshift{0.536243in}{1.851793in}%
\pgfsys@useobject{currentmarker}{}%
\end{pgfscope}%
\begin{pgfscope}%
\pgfsys@transformshift{0.889736in}{1.870146in}%
\pgfsys@useobject{currentmarker}{}%
\end{pgfscope}%
\begin{pgfscope}%
\pgfsys@transformshift{1.818644in}{1.679357in}%
\pgfsys@useobject{currentmarker}{}%
\end{pgfscope}%
\begin{pgfscope}%
\pgfsys@transformshift{1.073205in}{1.913521in}%
\pgfsys@useobject{currentmarker}{}%
\end{pgfscope}%
\begin{pgfscope}%
\pgfsys@transformshift{0.882269in}{1.864320in}%
\pgfsys@useobject{currentmarker}{}%
\end{pgfscope}%
\begin{pgfscope}%
\pgfsys@transformshift{0.755753in}{1.876600in}%
\pgfsys@useobject{currentmarker}{}%
\end{pgfscope}%
\begin{pgfscope}%
\pgfsys@transformshift{1.318495in}{1.826011in}%
\pgfsys@useobject{currentmarker}{}%
\end{pgfscope}%
\begin{pgfscope}%
\pgfsys@transformshift{1.102293in}{1.800030in}%
\pgfsys@useobject{currentmarker}{}%
\end{pgfscope}%
\begin{pgfscope}%
\pgfsys@transformshift{1.814698in}{1.875205in}%
\pgfsys@useobject{currentmarker}{}%
\end{pgfscope}%
\begin{pgfscope}%
\pgfsys@transformshift{1.595111in}{1.692648in}%
\pgfsys@useobject{currentmarker}{}%
\end{pgfscope}%
\begin{pgfscope}%
\pgfsys@transformshift{0.726840in}{1.792204in}%
\pgfsys@useobject{currentmarker}{}%
\end{pgfscope}%
\begin{pgfscope}%
\pgfsys@transformshift{1.492602in}{1.701551in}%
\pgfsys@useobject{currentmarker}{}%
\end{pgfscope}%
\begin{pgfscope}%
\pgfsys@transformshift{1.393179in}{1.628761in}%
\pgfsys@useobject{currentmarker}{}%
\end{pgfscope}%
\begin{pgfscope}%
\pgfsys@transformshift{1.166943in}{1.641101in}%
\pgfsys@useobject{currentmarker}{}%
\end{pgfscope}%
\begin{pgfscope}%
\pgfsys@transformshift{1.444452in}{1.632443in}%
\pgfsys@useobject{currentmarker}{}%
\end{pgfscope}%
\begin{pgfscope}%
\pgfsys@transformshift{1.693005in}{1.826021in}%
\pgfsys@useobject{currentmarker}{}%
\end{pgfscope}%
\begin{pgfscope}%
\pgfsys@transformshift{0.817674in}{1.707401in}%
\pgfsys@useobject{currentmarker}{}%
\end{pgfscope}%
\begin{pgfscope}%
\pgfsys@transformshift{0.521183in}{1.871047in}%
\pgfsys@useobject{currentmarker}{}%
\end{pgfscope}%
\begin{pgfscope}%
\pgfsys@transformshift{1.298494in}{1.866863in}%
\pgfsys@useobject{currentmarker}{}%
\end{pgfscope}%
\begin{pgfscope}%
\pgfsys@transformshift{0.537740in}{1.841028in}%
\pgfsys@useobject{currentmarker}{}%
\end{pgfscope}%
\begin{pgfscope}%
\pgfsys@transformshift{1.711514in}{1.842846in}%
\pgfsys@useobject{currentmarker}{}%
\end{pgfscope}%
\begin{pgfscope}%
\pgfsys@transformshift{1.607575in}{1.839528in}%
\pgfsys@useobject{currentmarker}{}%
\end{pgfscope}%
\begin{pgfscope}%
\pgfsys@transformshift{1.569340in}{1.814149in}%
\pgfsys@useobject{currentmarker}{}%
\end{pgfscope}%
\begin{pgfscope}%
\pgfsys@transformshift{0.423151in}{1.863146in}%
\pgfsys@useobject{currentmarker}{}%
\end{pgfscope}%
\begin{pgfscope}%
\pgfsys@transformshift{0.636198in}{1.540974in}%
\pgfsys@useobject{currentmarker}{}%
\end{pgfscope}%
\begin{pgfscope}%
\pgfsys@transformshift{1.742823in}{1.663022in}%
\pgfsys@useobject{currentmarker}{}%
\end{pgfscope}%
\begin{pgfscope}%
\pgfsys@transformshift{0.849878in}{1.796288in}%
\pgfsys@useobject{currentmarker}{}%
\end{pgfscope}%
\begin{pgfscope}%
\pgfsys@transformshift{1.353468in}{1.821477in}%
\pgfsys@useobject{currentmarker}{}%
\end{pgfscope}%
\begin{pgfscope}%
\pgfsys@transformshift{1.622982in}{1.465096in}%
\pgfsys@useobject{currentmarker}{}%
\end{pgfscope}%
\begin{pgfscope}%
\pgfsys@transformshift{1.599783in}{1.546932in}%
\pgfsys@useobject{currentmarker}{}%
\end{pgfscope}%
\begin{pgfscope}%
\pgfsys@transformshift{1.895187in}{1.626525in}%
\pgfsys@useobject{currentmarker}{}%
\end{pgfscope}%
\begin{pgfscope}%
\pgfsys@transformshift{1.129530in}{1.837483in}%
\pgfsys@useobject{currentmarker}{}%
\end{pgfscope}%
\begin{pgfscope}%
\pgfsys@transformshift{0.991382in}{1.784414in}%
\pgfsys@useobject{currentmarker}{}%
\end{pgfscope}%
\begin{pgfscope}%
\pgfsys@transformshift{0.795664in}{1.964223in}%
\pgfsys@useobject{currentmarker}{}%
\end{pgfscope}%
\begin{pgfscope}%
\pgfsys@transformshift{1.780025in}{1.793851in}%
\pgfsys@useobject{currentmarker}{}%
\end{pgfscope}%
\begin{pgfscope}%
\pgfsys@transformshift{1.712749in}{1.805955in}%
\pgfsys@useobject{currentmarker}{}%
\end{pgfscope}%
\begin{pgfscope}%
\pgfsys@transformshift{1.145211in}{1.886691in}%
\pgfsys@useobject{currentmarker}{}%
\end{pgfscope}%
\begin{pgfscope}%
\pgfsys@transformshift{1.328349in}{1.727132in}%
\pgfsys@useobject{currentmarker}{}%
\end{pgfscope}%
\begin{pgfscope}%
\pgfsys@transformshift{1.405073in}{1.798732in}%
\pgfsys@useobject{currentmarker}{}%
\end{pgfscope}%
\begin{pgfscope}%
\pgfsys@transformshift{0.669867in}{1.874545in}%
\pgfsys@useobject{currentmarker}{}%
\end{pgfscope}%
\begin{pgfscope}%
\pgfsys@transformshift{1.898744in}{1.591385in}%
\pgfsys@useobject{currentmarker}{}%
\end{pgfscope}%
\begin{pgfscope}%
\pgfsys@transformshift{0.874940in}{1.833555in}%
\pgfsys@useobject{currentmarker}{}%
\end{pgfscope}%
\begin{pgfscope}%
\pgfsys@transformshift{1.459937in}{1.936063in}%
\pgfsys@useobject{currentmarker}{}%
\end{pgfscope}%
\begin{pgfscope}%
\pgfsys@transformshift{1.895928in}{1.626180in}%
\pgfsys@useobject{currentmarker}{}%
\end{pgfscope}%
\begin{pgfscope}%
\pgfsys@transformshift{0.853418in}{1.804942in}%
\pgfsys@useobject{currentmarker}{}%
\end{pgfscope}%
\begin{pgfscope}%
\pgfsys@transformshift{0.717208in}{1.860094in}%
\pgfsys@useobject{currentmarker}{}%
\end{pgfscope}%
\begin{pgfscope}%
\pgfsys@transformshift{1.075298in}{1.830798in}%
\pgfsys@useobject{currentmarker}{}%
\end{pgfscope}%
\begin{pgfscope}%
\pgfsys@transformshift{1.920218in}{1.783155in}%
\pgfsys@useobject{currentmarker}{}%
\end{pgfscope}%
\begin{pgfscope}%
\pgfsys@transformshift{0.377829in}{1.853816in}%
\pgfsys@useobject{currentmarker}{}%
\end{pgfscope}%
\begin{pgfscope}%
\pgfsys@transformshift{1.155745in}{1.871917in}%
\pgfsys@useobject{currentmarker}{}%
\end{pgfscope}%
\begin{pgfscope}%
\pgfsys@transformshift{1.712524in}{1.627990in}%
\pgfsys@useobject{currentmarker}{}%
\end{pgfscope}%
\begin{pgfscope}%
\pgfsys@transformshift{0.534596in}{1.877717in}%
\pgfsys@useobject{currentmarker}{}%
\end{pgfscope}%
\begin{pgfscope}%
\pgfsys@transformshift{0.622741in}{1.938713in}%
\pgfsys@useobject{currentmarker}{}%
\end{pgfscope}%
\begin{pgfscope}%
\pgfsys@transformshift{0.575527in}{1.927769in}%
\pgfsys@useobject{currentmarker}{}%
\end{pgfscope}%
\begin{pgfscope}%
\pgfsys@transformshift{1.804863in}{1.642782in}%
\pgfsys@useobject{currentmarker}{}%
\end{pgfscope}%
\begin{pgfscope}%
\pgfsys@transformshift{1.749745in}{1.589403in}%
\pgfsys@useobject{currentmarker}{}%
\end{pgfscope}%
\begin{pgfscope}%
\pgfsys@transformshift{1.674396in}{1.781308in}%
\pgfsys@useobject{currentmarker}{}%
\end{pgfscope}%
\begin{pgfscope}%
\pgfsys@transformshift{1.859027in}{1.609222in}%
\pgfsys@useobject{currentmarker}{}%
\end{pgfscope}%
\begin{pgfscope}%
\pgfsys@transformshift{1.603827in}{1.672767in}%
\pgfsys@useobject{currentmarker}{}%
\end{pgfscope}%
\begin{pgfscope}%
\pgfsys@transformshift{1.255917in}{1.866490in}%
\pgfsys@useobject{currentmarker}{}%
\end{pgfscope}%
\begin{pgfscope}%
\pgfsys@transformshift{0.419746in}{1.617822in}%
\pgfsys@useobject{currentmarker}{}%
\end{pgfscope}%
\begin{pgfscope}%
\pgfsys@transformshift{0.903589in}{1.603060in}%
\pgfsys@useobject{currentmarker}{}%
\end{pgfscope}%
\begin{pgfscope}%
\pgfsys@transformshift{1.411837in}{1.839096in}%
\pgfsys@useobject{currentmarker}{}%
\end{pgfscope}%
\begin{pgfscope}%
\pgfsys@transformshift{1.888179in}{1.619719in}%
\pgfsys@useobject{currentmarker}{}%
\end{pgfscope}%
\begin{pgfscope}%
\pgfsys@transformshift{1.463948in}{1.779262in}%
\pgfsys@useobject{currentmarker}{}%
\end{pgfscope}%
\begin{pgfscope}%
\pgfsys@transformshift{1.037147in}{1.776192in}%
\pgfsys@useobject{currentmarker}{}%
\end{pgfscope}%
\begin{pgfscope}%
\pgfsys@transformshift{1.744144in}{1.673512in}%
\pgfsys@useobject{currentmarker}{}%
\end{pgfscope}%
\begin{pgfscope}%
\pgfsys@transformshift{0.867355in}{1.740667in}%
\pgfsys@useobject{currentmarker}{}%
\end{pgfscope}%
\begin{pgfscope}%
\pgfsys@transformshift{1.331798in}{1.896896in}%
\pgfsys@useobject{currentmarker}{}%
\end{pgfscope}%
\begin{pgfscope}%
\pgfsys@transformshift{0.968745in}{1.863007in}%
\pgfsys@useobject{currentmarker}{}%
\end{pgfscope}%
\begin{pgfscope}%
\pgfsys@transformshift{1.193049in}{1.873056in}%
\pgfsys@useobject{currentmarker}{}%
\end{pgfscope}%
\begin{pgfscope}%
\pgfsys@transformshift{1.664246in}{1.698673in}%
\pgfsys@useobject{currentmarker}{}%
\end{pgfscope}%
\begin{pgfscope}%
\pgfsys@transformshift{0.835587in}{1.857464in}%
\pgfsys@useobject{currentmarker}{}%
\end{pgfscope}%
\begin{pgfscope}%
\pgfsys@transformshift{1.701974in}{1.618692in}%
\pgfsys@useobject{currentmarker}{}%
\end{pgfscope}%
\begin{pgfscope}%
\pgfsys@transformshift{1.269033in}{1.822801in}%
\pgfsys@useobject{currentmarker}{}%
\end{pgfscope}%
\begin{pgfscope}%
\pgfsys@transformshift{1.329352in}{1.762168in}%
\pgfsys@useobject{currentmarker}{}%
\end{pgfscope}%
\begin{pgfscope}%
\pgfsys@transformshift{1.471661in}{1.854687in}%
\pgfsys@useobject{currentmarker}{}%
\end{pgfscope}%
\begin{pgfscope}%
\pgfsys@transformshift{1.916157in}{1.810307in}%
\pgfsys@useobject{currentmarker}{}%
\end{pgfscope}%
\begin{pgfscope}%
\pgfsys@transformshift{1.533379in}{1.651457in}%
\pgfsys@useobject{currentmarker}{}%
\end{pgfscope}%
\begin{pgfscope}%
\pgfsys@transformshift{1.358583in}{1.744981in}%
\pgfsys@useobject{currentmarker}{}%
\end{pgfscope}%
\begin{pgfscope}%
\pgfsys@transformshift{1.599245in}{1.711028in}%
\pgfsys@useobject{currentmarker}{}%
\end{pgfscope}%
\begin{pgfscope}%
\pgfsys@transformshift{0.703172in}{1.853019in}%
\pgfsys@useobject{currentmarker}{}%
\end{pgfscope}%
\begin{pgfscope}%
\pgfsys@transformshift{1.320412in}{1.855334in}%
\pgfsys@useobject{currentmarker}{}%
\end{pgfscope}%
\begin{pgfscope}%
\pgfsys@transformshift{1.136406in}{1.787198in}%
\pgfsys@useobject{currentmarker}{}%
\end{pgfscope}%
\begin{pgfscope}%
\pgfsys@transformshift{0.721527in}{1.836982in}%
\pgfsys@useobject{currentmarker}{}%
\end{pgfscope}%
\begin{pgfscope}%
\pgfsys@transformshift{1.722543in}{1.598110in}%
\pgfsys@useobject{currentmarker}{}%
\end{pgfscope}%
\begin{pgfscope}%
\pgfsys@transformshift{1.835288in}{1.742212in}%
\pgfsys@useobject{currentmarker}{}%
\end{pgfscope}%
\begin{pgfscope}%
\pgfsys@transformshift{1.984248in}{1.756967in}%
\pgfsys@useobject{currentmarker}{}%
\end{pgfscope}%
\begin{pgfscope}%
\pgfsys@transformshift{1.658536in}{1.776986in}%
\pgfsys@useobject{currentmarker}{}%
\end{pgfscope}%
\begin{pgfscope}%
\pgfsys@transformshift{0.561607in}{1.884061in}%
\pgfsys@useobject{currentmarker}{}%
\end{pgfscope}%
\begin{pgfscope}%
\pgfsys@transformshift{1.906539in}{1.755807in}%
\pgfsys@useobject{currentmarker}{}%
\end{pgfscope}%
\begin{pgfscope}%
\pgfsys@transformshift{1.521333in}{1.816056in}%
\pgfsys@useobject{currentmarker}{}%
\end{pgfscope}%
\begin{pgfscope}%
\pgfsys@transformshift{1.136396in}{1.791152in}%
\pgfsys@useobject{currentmarker}{}%
\end{pgfscope}%
\begin{pgfscope}%
\pgfsys@transformshift{1.435818in}{1.640214in}%
\pgfsys@useobject{currentmarker}{}%
\end{pgfscope}%
\begin{pgfscope}%
\pgfsys@transformshift{1.643271in}{1.653976in}%
\pgfsys@useobject{currentmarker}{}%
\end{pgfscope}%
\begin{pgfscope}%
\pgfsys@transformshift{1.445064in}{1.734631in}%
\pgfsys@useobject{currentmarker}{}%
\end{pgfscope}%
\begin{pgfscope}%
\pgfsys@transformshift{0.658007in}{1.759871in}%
\pgfsys@useobject{currentmarker}{}%
\end{pgfscope}%
\begin{pgfscope}%
\pgfsys@transformshift{0.786972in}{1.809346in}%
\pgfsys@useobject{currentmarker}{}%
\end{pgfscope}%
\begin{pgfscope}%
\pgfsys@transformshift{1.102575in}{1.824865in}%
\pgfsys@useobject{currentmarker}{}%
\end{pgfscope}%
\begin{pgfscope}%
\pgfsys@transformshift{0.532598in}{1.871110in}%
\pgfsys@useobject{currentmarker}{}%
\end{pgfscope}%
\begin{pgfscope}%
\pgfsys@transformshift{0.357574in}{1.873897in}%
\pgfsys@useobject{currentmarker}{}%
\end{pgfscope}%
\begin{pgfscope}%
\pgfsys@transformshift{1.683669in}{1.625250in}%
\pgfsys@useobject{currentmarker}{}%
\end{pgfscope}%
\begin{pgfscope}%
\pgfsys@transformshift{1.612445in}{1.787310in}%
\pgfsys@useobject{currentmarker}{}%
\end{pgfscope}%
\begin{pgfscope}%
\pgfsys@transformshift{1.620066in}{1.657540in}%
\pgfsys@useobject{currentmarker}{}%
\end{pgfscope}%
\begin{pgfscope}%
\pgfsys@transformshift{0.680343in}{1.671442in}%
\pgfsys@useobject{currentmarker}{}%
\end{pgfscope}%
\begin{pgfscope}%
\pgfsys@transformshift{0.675258in}{1.864556in}%
\pgfsys@useobject{currentmarker}{}%
\end{pgfscope}%
\begin{pgfscope}%
\pgfsys@transformshift{0.963927in}{1.774012in}%
\pgfsys@useobject{currentmarker}{}%
\end{pgfscope}%
\begin{pgfscope}%
\pgfsys@transformshift{1.684562in}{1.675131in}%
\pgfsys@useobject{currentmarker}{}%
\end{pgfscope}%
\begin{pgfscope}%
\pgfsys@transformshift{1.473245in}{1.846294in}%
\pgfsys@useobject{currentmarker}{}%
\end{pgfscope}%
\begin{pgfscope}%
\pgfsys@transformshift{1.020383in}{1.909805in}%
\pgfsys@useobject{currentmarker}{}%
\end{pgfscope}%
\begin{pgfscope}%
\pgfsys@transformshift{0.681597in}{1.930082in}%
\pgfsys@useobject{currentmarker}{}%
\end{pgfscope}%
\begin{pgfscope}%
\pgfsys@transformshift{1.730115in}{1.694758in}%
\pgfsys@useobject{currentmarker}{}%
\end{pgfscope}%
\begin{pgfscope}%
\pgfsys@transformshift{1.808933in}{1.684767in}%
\pgfsys@useobject{currentmarker}{}%
\end{pgfscope}%
\begin{pgfscope}%
\pgfsys@transformshift{1.251675in}{1.861505in}%
\pgfsys@useobject{currentmarker}{}%
\end{pgfscope}%
\begin{pgfscope}%
\pgfsys@transformshift{1.717946in}{1.585604in}%
\pgfsys@useobject{currentmarker}{}%
\end{pgfscope}%
\begin{pgfscope}%
\pgfsys@transformshift{1.597312in}{1.656028in}%
\pgfsys@useobject{currentmarker}{}%
\end{pgfscope}%
\begin{pgfscope}%
\pgfsys@transformshift{1.148136in}{1.867270in}%
\pgfsys@useobject{currentmarker}{}%
\end{pgfscope}%
\begin{pgfscope}%
\pgfsys@transformshift{0.720143in}{1.876514in}%
\pgfsys@useobject{currentmarker}{}%
\end{pgfscope}%
\begin{pgfscope}%
\pgfsys@transformshift{1.574970in}{1.821851in}%
\pgfsys@useobject{currentmarker}{}%
\end{pgfscope}%
\begin{pgfscope}%
\pgfsys@transformshift{0.869972in}{1.748987in}%
\pgfsys@useobject{currentmarker}{}%
\end{pgfscope}%
\begin{pgfscope}%
\pgfsys@transformshift{1.713193in}{1.709349in}%
\pgfsys@useobject{currentmarker}{}%
\end{pgfscope}%
\begin{pgfscope}%
\pgfsys@transformshift{1.673692in}{1.587894in}%
\pgfsys@useobject{currentmarker}{}%
\end{pgfscope}%
\begin{pgfscope}%
\pgfsys@transformshift{1.344861in}{1.737244in}%
\pgfsys@useobject{currentmarker}{}%
\end{pgfscope}%
\begin{pgfscope}%
\pgfsys@transformshift{1.182101in}{1.812006in}%
\pgfsys@useobject{currentmarker}{}%
\end{pgfscope}%
\begin{pgfscope}%
\pgfsys@transformshift{1.449668in}{1.781610in}%
\pgfsys@useobject{currentmarker}{}%
\end{pgfscope}%
\begin{pgfscope}%
\pgfsys@transformshift{1.561680in}{1.766339in}%
\pgfsys@useobject{currentmarker}{}%
\end{pgfscope}%
\begin{pgfscope}%
\pgfsys@transformshift{1.192063in}{1.742553in}%
\pgfsys@useobject{currentmarker}{}%
\end{pgfscope}%
\begin{pgfscope}%
\pgfsys@transformshift{1.264374in}{1.870017in}%
\pgfsys@useobject{currentmarker}{}%
\end{pgfscope}%
\begin{pgfscope}%
\pgfsys@transformshift{0.727375in}{1.914341in}%
\pgfsys@useobject{currentmarker}{}%
\end{pgfscope}%
\begin{pgfscope}%
\pgfsys@transformshift{0.687912in}{1.576975in}%
\pgfsys@useobject{currentmarker}{}%
\end{pgfscope}%
\begin{pgfscope}%
\pgfsys@transformshift{1.823596in}{1.783701in}%
\pgfsys@useobject{currentmarker}{}%
\end{pgfscope}%
\begin{pgfscope}%
\pgfsys@transformshift{1.420153in}{1.753583in}%
\pgfsys@useobject{currentmarker}{}%
\end{pgfscope}%
\begin{pgfscope}%
\pgfsys@transformshift{0.672265in}{1.935939in}%
\pgfsys@useobject{currentmarker}{}%
\end{pgfscope}%
\begin{pgfscope}%
\pgfsys@transformshift{0.574267in}{1.911926in}%
\pgfsys@useobject{currentmarker}{}%
\end{pgfscope}%
\begin{pgfscope}%
\pgfsys@transformshift{1.570297in}{1.738118in}%
\pgfsys@useobject{currentmarker}{}%
\end{pgfscope}%
\begin{pgfscope}%
\pgfsys@transformshift{1.650981in}{1.610316in}%
\pgfsys@useobject{currentmarker}{}%
\end{pgfscope}%
\begin{pgfscope}%
\pgfsys@transformshift{0.894282in}{1.790007in}%
\pgfsys@useobject{currentmarker}{}%
\end{pgfscope}%
\begin{pgfscope}%
\pgfsys@transformshift{1.157746in}{1.845263in}%
\pgfsys@useobject{currentmarker}{}%
\end{pgfscope}%
\begin{pgfscope}%
\pgfsys@transformshift{0.608353in}{1.906167in}%
\pgfsys@useobject{currentmarker}{}%
\end{pgfscope}%
\begin{pgfscope}%
\pgfsys@transformshift{1.631665in}{1.631533in}%
\pgfsys@useobject{currentmarker}{}%
\end{pgfscope}%
\begin{pgfscope}%
\pgfsys@transformshift{1.419963in}{1.703253in}%
\pgfsys@useobject{currentmarker}{}%
\end{pgfscope}%
\begin{pgfscope}%
\pgfsys@transformshift{1.180003in}{1.843107in}%
\pgfsys@useobject{currentmarker}{}%
\end{pgfscope}%
\begin{pgfscope}%
\pgfsys@transformshift{1.469789in}{1.827492in}%
\pgfsys@useobject{currentmarker}{}%
\end{pgfscope}%
\begin{pgfscope}%
\pgfsys@transformshift{1.493240in}{1.877332in}%
\pgfsys@useobject{currentmarker}{}%
\end{pgfscope}%
\begin{pgfscope}%
\pgfsys@transformshift{1.725815in}{1.665048in}%
\pgfsys@useobject{currentmarker}{}%
\end{pgfscope}%
\begin{pgfscope}%
\pgfsys@transformshift{1.002068in}{1.791625in}%
\pgfsys@useobject{currentmarker}{}%
\end{pgfscope}%
\begin{pgfscope}%
\pgfsys@transformshift{1.111774in}{1.856566in}%
\pgfsys@useobject{currentmarker}{}%
\end{pgfscope}%
\begin{pgfscope}%
\pgfsys@transformshift{0.361781in}{1.928867in}%
\pgfsys@useobject{currentmarker}{}%
\end{pgfscope}%
\begin{pgfscope}%
\pgfsys@transformshift{1.608152in}{1.763738in}%
\pgfsys@useobject{currentmarker}{}%
\end{pgfscope}%
\begin{pgfscope}%
\pgfsys@transformshift{1.468569in}{1.891322in}%
\pgfsys@useobject{currentmarker}{}%
\end{pgfscope}%
\begin{pgfscope}%
\pgfsys@transformshift{1.646778in}{1.621077in}%
\pgfsys@useobject{currentmarker}{}%
\end{pgfscope}%
\begin{pgfscope}%
\pgfsys@transformshift{1.356664in}{1.854897in}%
\pgfsys@useobject{currentmarker}{}%
\end{pgfscope}%
\begin{pgfscope}%
\pgfsys@transformshift{0.657110in}{1.618320in}%
\pgfsys@useobject{currentmarker}{}%
\end{pgfscope}%
\begin{pgfscope}%
\pgfsys@transformshift{1.700456in}{1.596637in}%
\pgfsys@useobject{currentmarker}{}%
\end{pgfscope}%
\begin{pgfscope}%
\pgfsys@transformshift{0.741052in}{1.838763in}%
\pgfsys@useobject{currentmarker}{}%
\end{pgfscope}%
\begin{pgfscope}%
\pgfsys@transformshift{0.576055in}{2.006407in}%
\pgfsys@useobject{currentmarker}{}%
\end{pgfscope}%
\begin{pgfscope}%
\pgfsys@transformshift{1.509849in}{1.800818in}%
\pgfsys@useobject{currentmarker}{}%
\end{pgfscope}%
\begin{pgfscope}%
\pgfsys@transformshift{1.838065in}{1.659247in}%
\pgfsys@useobject{currentmarker}{}%
\end{pgfscope}%
\begin{pgfscope}%
\pgfsys@transformshift{1.699823in}{1.837303in}%
\pgfsys@useobject{currentmarker}{}%
\end{pgfscope}%
\begin{pgfscope}%
\pgfsys@transformshift{1.575775in}{1.757756in}%
\pgfsys@useobject{currentmarker}{}%
\end{pgfscope}%
\begin{pgfscope}%
\pgfsys@transformshift{1.000652in}{1.844590in}%
\pgfsys@useobject{currentmarker}{}%
\end{pgfscope}%
\begin{pgfscope}%
\pgfsys@transformshift{1.275329in}{1.862265in}%
\pgfsys@useobject{currentmarker}{}%
\end{pgfscope}%
\begin{pgfscope}%
\pgfsys@transformshift{0.787302in}{1.849276in}%
\pgfsys@useobject{currentmarker}{}%
\end{pgfscope}%
\begin{pgfscope}%
\pgfsys@transformshift{0.941308in}{1.691411in}%
\pgfsys@useobject{currentmarker}{}%
\end{pgfscope}%
\begin{pgfscope}%
\pgfsys@transformshift{1.041866in}{1.725888in}%
\pgfsys@useobject{currentmarker}{}%
\end{pgfscope}%
\begin{pgfscope}%
\pgfsys@transformshift{1.471835in}{1.790297in}%
\pgfsys@useobject{currentmarker}{}%
\end{pgfscope}%
\begin{pgfscope}%
\pgfsys@transformshift{1.225077in}{1.805192in}%
\pgfsys@useobject{currentmarker}{}%
\end{pgfscope}%
\begin{pgfscope}%
\pgfsys@transformshift{1.294631in}{1.702712in}%
\pgfsys@useobject{currentmarker}{}%
\end{pgfscope}%
\begin{pgfscope}%
\pgfsys@transformshift{1.597060in}{1.802314in}%
\pgfsys@useobject{currentmarker}{}%
\end{pgfscope}%
\begin{pgfscope}%
\pgfsys@transformshift{1.136873in}{1.749088in}%
\pgfsys@useobject{currentmarker}{}%
\end{pgfscope}%
\begin{pgfscope}%
\pgfsys@transformshift{1.080421in}{1.785338in}%
\pgfsys@useobject{currentmarker}{}%
\end{pgfscope}%
\begin{pgfscope}%
\pgfsys@transformshift{1.185638in}{1.842153in}%
\pgfsys@useobject{currentmarker}{}%
\end{pgfscope}%
\begin{pgfscope}%
\pgfsys@transformshift{1.407062in}{1.834367in}%
\pgfsys@useobject{currentmarker}{}%
\end{pgfscope}%
\begin{pgfscope}%
\pgfsys@transformshift{1.504034in}{1.882816in}%
\pgfsys@useobject{currentmarker}{}%
\end{pgfscope}%
\begin{pgfscope}%
\pgfsys@transformshift{1.396930in}{1.823394in}%
\pgfsys@useobject{currentmarker}{}%
\end{pgfscope}%
\begin{pgfscope}%
\pgfsys@transformshift{2.000000in}{1.677898in}%
\pgfsys@useobject{currentmarker}{}%
\end{pgfscope}%
\begin{pgfscope}%
\pgfsys@transformshift{1.200619in}{1.732234in}%
\pgfsys@useobject{currentmarker}{}%
\end{pgfscope}%
\begin{pgfscope}%
\pgfsys@transformshift{0.950093in}{1.556256in}%
\pgfsys@useobject{currentmarker}{}%
\end{pgfscope}%
\begin{pgfscope}%
\pgfsys@transformshift{0.632457in}{1.868159in}%
\pgfsys@useobject{currentmarker}{}%
\end{pgfscope}%
\begin{pgfscope}%
\pgfsys@transformshift{0.673561in}{1.780727in}%
\pgfsys@useobject{currentmarker}{}%
\end{pgfscope}%
\begin{pgfscope}%
\pgfsys@transformshift{0.694799in}{1.916297in}%
\pgfsys@useobject{currentmarker}{}%
\end{pgfscope}%
\begin{pgfscope}%
\pgfsys@transformshift{0.518198in}{1.903746in}%
\pgfsys@useobject{currentmarker}{}%
\end{pgfscope}%
\begin{pgfscope}%
\pgfsys@transformshift{1.324442in}{1.795899in}%
\pgfsys@useobject{currentmarker}{}%
\end{pgfscope}%
\begin{pgfscope}%
\pgfsys@transformshift{0.827155in}{1.915682in}%
\pgfsys@useobject{currentmarker}{}%
\end{pgfscope}%
\begin{pgfscope}%
\pgfsys@transformshift{1.894588in}{1.694625in}%
\pgfsys@useobject{currentmarker}{}%
\end{pgfscope}%
\begin{pgfscope}%
\pgfsys@transformshift{0.446321in}{1.838219in}%
\pgfsys@useobject{currentmarker}{}%
\end{pgfscope}%
\begin{pgfscope}%
\pgfsys@transformshift{1.741761in}{1.817214in}%
\pgfsys@useobject{currentmarker}{}%
\end{pgfscope}%
\begin{pgfscope}%
\pgfsys@transformshift{1.307351in}{1.764339in}%
\pgfsys@useobject{currentmarker}{}%
\end{pgfscope}%
\begin{pgfscope}%
\pgfsys@transformshift{1.835875in}{1.762543in}%
\pgfsys@useobject{currentmarker}{}%
\end{pgfscope}%
\begin{pgfscope}%
\pgfsys@transformshift{1.237368in}{1.821305in}%
\pgfsys@useobject{currentmarker}{}%
\end{pgfscope}%
\begin{pgfscope}%
\pgfsys@transformshift{1.875271in}{1.473522in}%
\pgfsys@useobject{currentmarker}{}%
\end{pgfscope}%
\begin{pgfscope}%
\pgfsys@transformshift{1.669995in}{1.687402in}%
\pgfsys@useobject{currentmarker}{}%
\end{pgfscope}%
\begin{pgfscope}%
\pgfsys@transformshift{0.628877in}{1.928432in}%
\pgfsys@useobject{currentmarker}{}%
\end{pgfscope}%
\begin{pgfscope}%
\pgfsys@transformshift{1.644915in}{1.723264in}%
\pgfsys@useobject{currentmarker}{}%
\end{pgfscope}%
\begin{pgfscope}%
\pgfsys@transformshift{1.179959in}{1.743681in}%
\pgfsys@useobject{currentmarker}{}%
\end{pgfscope}%
\begin{pgfscope}%
\pgfsys@transformshift{1.753339in}{1.725600in}%
\pgfsys@useobject{currentmarker}{}%
\end{pgfscope}%
\begin{pgfscope}%
\pgfsys@transformshift{1.477915in}{1.764325in}%
\pgfsys@useobject{currentmarker}{}%
\end{pgfscope}%
\begin{pgfscope}%
\pgfsys@transformshift{1.742609in}{1.795331in}%
\pgfsys@useobject{currentmarker}{}%
\end{pgfscope}%
\begin{pgfscope}%
\pgfsys@transformshift{1.094044in}{1.880603in}%
\pgfsys@useobject{currentmarker}{}%
\end{pgfscope}%
\begin{pgfscope}%
\pgfsys@transformshift{1.645687in}{1.650477in}%
\pgfsys@useobject{currentmarker}{}%
\end{pgfscope}%
\begin{pgfscope}%
\pgfsys@transformshift{0.717249in}{1.799329in}%
\pgfsys@useobject{currentmarker}{}%
\end{pgfscope}%
\begin{pgfscope}%
\pgfsys@transformshift{0.963925in}{1.720511in}%
\pgfsys@useobject{currentmarker}{}%
\end{pgfscope}%
\begin{pgfscope}%
\pgfsys@transformshift{1.342874in}{1.806752in}%
\pgfsys@useobject{currentmarker}{}%
\end{pgfscope}%
\begin{pgfscope}%
\pgfsys@transformshift{1.522619in}{1.740723in}%
\pgfsys@useobject{currentmarker}{}%
\end{pgfscope}%
\begin{pgfscope}%
\pgfsys@transformshift{0.649847in}{1.952906in}%
\pgfsys@useobject{currentmarker}{}%
\end{pgfscope}%
\begin{pgfscope}%
\pgfsys@transformshift{1.826185in}{1.749866in}%
\pgfsys@useobject{currentmarker}{}%
\end{pgfscope}%
\begin{pgfscope}%
\pgfsys@transformshift{1.619806in}{1.722357in}%
\pgfsys@useobject{currentmarker}{}%
\end{pgfscope}%
\begin{pgfscope}%
\pgfsys@transformshift{1.136702in}{1.870109in}%
\pgfsys@useobject{currentmarker}{}%
\end{pgfscope}%
\begin{pgfscope}%
\pgfsys@transformshift{1.894425in}{1.603663in}%
\pgfsys@useobject{currentmarker}{}%
\end{pgfscope}%
\begin{pgfscope}%
\pgfsys@transformshift{0.668971in}{1.918379in}%
\pgfsys@useobject{currentmarker}{}%
\end{pgfscope}%
\end{pgfscope}%
\begin{pgfscope}%
\pgfpathrectangle{\pgfqpoint{0.341129in}{0.466613in}}{\pgfqpoint{1.658871in}{1.711598in}}%
\pgfusepath{clip}%
\pgfsetbuttcap%
\pgfsetroundjoin%
\definecolor{currentfill}{rgb}{0.866667,0.517647,0.321569}%
\pgfsetfillcolor{currentfill}%
\pgfsetfillopacity{0.150000}%
\pgfsetlinewidth{1.003750pt}%
\definecolor{currentstroke}{rgb}{1.000000,1.000000,1.000000}%
\pgfsetstrokecolor{currentstroke}%
\pgfsetstrokeopacity{0.150000}%
\pgfsetdash{}{0pt}%
\pgfsys@defobject{currentmarker}{\pgfqpoint{0.341129in}{1.697848in}}{\pgfqpoint{2.000000in}{1.876323in}}{%
\pgfpathmoveto{\pgfqpoint{0.341129in}{1.876323in}}%
\pgfpathlineto{\pgfqpoint{0.341129in}{1.833915in}}%
\pgfpathlineto{\pgfqpoint{0.357885in}{1.832816in}}%
\pgfpathlineto{\pgfqpoint{0.374641in}{1.831717in}}%
\pgfpathlineto{\pgfqpoint{0.391398in}{1.830619in}}%
\pgfpathlineto{\pgfqpoint{0.408154in}{1.829520in}}%
\pgfpathlineto{\pgfqpoint{0.424910in}{1.828341in}}%
\pgfpathlineto{\pgfqpoint{0.441666in}{1.827156in}}%
\pgfpathlineto{\pgfqpoint{0.458423in}{1.825970in}}%
\pgfpathlineto{\pgfqpoint{0.475179in}{1.824784in}}%
\pgfpathlineto{\pgfqpoint{0.491935in}{1.823599in}}%
\pgfpathlineto{\pgfqpoint{0.508691in}{1.822413in}}%
\pgfpathlineto{\pgfqpoint{0.525448in}{1.821227in}}%
\pgfpathlineto{\pgfqpoint{0.542204in}{1.820109in}}%
\pgfpathlineto{\pgfqpoint{0.558960in}{1.819062in}}%
\pgfpathlineto{\pgfqpoint{0.575717in}{1.818015in}}%
\pgfpathlineto{\pgfqpoint{0.592473in}{1.816968in}}%
\pgfpathlineto{\pgfqpoint{0.609229in}{1.815789in}}%
\pgfpathlineto{\pgfqpoint{0.625985in}{1.814811in}}%
\pgfpathlineto{\pgfqpoint{0.642742in}{1.813678in}}%
\pgfpathlineto{\pgfqpoint{0.659498in}{1.812498in}}%
\pgfpathlineto{\pgfqpoint{0.676254in}{1.811353in}}%
\pgfpathlineto{\pgfqpoint{0.693011in}{1.810260in}}%
\pgfpathlineto{\pgfqpoint{0.709767in}{1.809135in}}%
\pgfpathlineto{\pgfqpoint{0.726523in}{1.808088in}}%
\pgfpathlineto{\pgfqpoint{0.743279in}{1.807038in}}%
\pgfpathlineto{\pgfqpoint{0.760036in}{1.805988in}}%
\pgfpathlineto{\pgfqpoint{0.776792in}{1.804937in}}%
\pgfpathlineto{\pgfqpoint{0.793548in}{1.803888in}}%
\pgfpathlineto{\pgfqpoint{0.810304in}{1.802831in}}%
\pgfpathlineto{\pgfqpoint{0.827061in}{1.801674in}}%
\pgfpathlineto{\pgfqpoint{0.843817in}{1.800491in}}%
\pgfpathlineto{\pgfqpoint{0.860573in}{1.799207in}}%
\pgfpathlineto{\pgfqpoint{0.877330in}{1.797913in}}%
\pgfpathlineto{\pgfqpoint{0.894086in}{1.796570in}}%
\pgfpathlineto{\pgfqpoint{0.910842in}{1.795227in}}%
\pgfpathlineto{\pgfqpoint{0.927598in}{1.794054in}}%
\pgfpathlineto{\pgfqpoint{0.944355in}{1.792754in}}%
\pgfpathlineto{\pgfqpoint{0.961111in}{1.791503in}}%
\pgfpathlineto{\pgfqpoint{0.977867in}{1.790390in}}%
\pgfpathlineto{\pgfqpoint{0.994623in}{1.789384in}}%
\pgfpathlineto{\pgfqpoint{1.011380in}{1.788405in}}%
\pgfpathlineto{\pgfqpoint{1.028136in}{1.787099in}}%
\pgfpathlineto{\pgfqpoint{1.044892in}{1.785753in}}%
\pgfpathlineto{\pgfqpoint{1.061649in}{1.784404in}}%
\pgfpathlineto{\pgfqpoint{1.078405in}{1.783165in}}%
\pgfpathlineto{\pgfqpoint{1.095161in}{1.781890in}}%
\pgfpathlineto{\pgfqpoint{1.111917in}{1.780600in}}%
\pgfpathlineto{\pgfqpoint{1.128674in}{1.779370in}}%
\pgfpathlineto{\pgfqpoint{1.145430in}{1.777948in}}%
\pgfpathlineto{\pgfqpoint{1.162186in}{1.776563in}}%
\pgfpathlineto{\pgfqpoint{1.178942in}{1.775408in}}%
\pgfpathlineto{\pgfqpoint{1.195699in}{1.774084in}}%
\pgfpathlineto{\pgfqpoint{1.212455in}{1.772938in}}%
\pgfpathlineto{\pgfqpoint{1.229211in}{1.771792in}}%
\pgfpathlineto{\pgfqpoint{1.245968in}{1.770499in}}%
\pgfpathlineto{\pgfqpoint{1.262724in}{1.769188in}}%
\pgfpathlineto{\pgfqpoint{1.279480in}{1.767764in}}%
\pgfpathlineto{\pgfqpoint{1.296236in}{1.766403in}}%
\pgfpathlineto{\pgfqpoint{1.312993in}{1.764902in}}%
\pgfpathlineto{\pgfqpoint{1.329749in}{1.763297in}}%
\pgfpathlineto{\pgfqpoint{1.346505in}{1.761698in}}%
\pgfpathlineto{\pgfqpoint{1.363262in}{1.760221in}}%
\pgfpathlineto{\pgfqpoint{1.380018in}{1.758837in}}%
\pgfpathlineto{\pgfqpoint{1.396774in}{1.757367in}}%
\pgfpathlineto{\pgfqpoint{1.413530in}{1.756041in}}%
\pgfpathlineto{\pgfqpoint{1.430287in}{1.754454in}}%
\pgfpathlineto{\pgfqpoint{1.447043in}{1.752889in}}%
\pgfpathlineto{\pgfqpoint{1.463799in}{1.751302in}}%
\pgfpathlineto{\pgfqpoint{1.480555in}{1.749698in}}%
\pgfpathlineto{\pgfqpoint{1.497312in}{1.748161in}}%
\pgfpathlineto{\pgfqpoint{1.514068in}{1.746584in}}%
\pgfpathlineto{\pgfqpoint{1.530824in}{1.744970in}}%
\pgfpathlineto{\pgfqpoint{1.547581in}{1.743334in}}%
\pgfpathlineto{\pgfqpoint{1.564337in}{1.741681in}}%
\pgfpathlineto{\pgfqpoint{1.581093in}{1.740074in}}%
\pgfpathlineto{\pgfqpoint{1.597849in}{1.738396in}}%
\pgfpathlineto{\pgfqpoint{1.614606in}{1.736712in}}%
\pgfpathlineto{\pgfqpoint{1.631362in}{1.735164in}}%
\pgfpathlineto{\pgfqpoint{1.648118in}{1.733559in}}%
\pgfpathlineto{\pgfqpoint{1.664874in}{1.731952in}}%
\pgfpathlineto{\pgfqpoint{1.681631in}{1.730252in}}%
\pgfpathlineto{\pgfqpoint{1.698387in}{1.728656in}}%
\pgfpathlineto{\pgfqpoint{1.715143in}{1.727002in}}%
\pgfpathlineto{\pgfqpoint{1.731900in}{1.725290in}}%
\pgfpathlineto{\pgfqpoint{1.748656in}{1.723578in}}%
\pgfpathlineto{\pgfqpoint{1.765412in}{1.721823in}}%
\pgfpathlineto{\pgfqpoint{1.782168in}{1.720065in}}%
\pgfpathlineto{\pgfqpoint{1.798925in}{1.718359in}}%
\pgfpathlineto{\pgfqpoint{1.815681in}{1.716675in}}%
\pgfpathlineto{\pgfqpoint{1.832437in}{1.714941in}}%
\pgfpathlineto{\pgfqpoint{1.849193in}{1.713222in}}%
\pgfpathlineto{\pgfqpoint{1.865950in}{1.711439in}}%
\pgfpathlineto{\pgfqpoint{1.882706in}{1.709709in}}%
\pgfpathlineto{\pgfqpoint{1.899462in}{1.708068in}}%
\pgfpathlineto{\pgfqpoint{1.916219in}{1.706397in}}%
\pgfpathlineto{\pgfqpoint{1.932975in}{1.704691in}}%
\pgfpathlineto{\pgfqpoint{1.949731in}{1.702985in}}%
\pgfpathlineto{\pgfqpoint{1.966487in}{1.701273in}}%
\pgfpathlineto{\pgfqpoint{1.983244in}{1.699561in}}%
\pgfpathlineto{\pgfqpoint{2.000000in}{1.697848in}}%
\pgfpathlineto{\pgfqpoint{2.000000in}{1.729271in}}%
\pgfpathlineto{\pgfqpoint{2.000000in}{1.729271in}}%
\pgfpathlineto{\pgfqpoint{1.983244in}{1.730381in}}%
\pgfpathlineto{\pgfqpoint{1.966487in}{1.731470in}}%
\pgfpathlineto{\pgfqpoint{1.949731in}{1.732559in}}%
\pgfpathlineto{\pgfqpoint{1.932975in}{1.733688in}}%
\pgfpathlineto{\pgfqpoint{1.916219in}{1.734834in}}%
\pgfpathlineto{\pgfqpoint{1.899462in}{1.735947in}}%
\pgfpathlineto{\pgfqpoint{1.882706in}{1.737061in}}%
\pgfpathlineto{\pgfqpoint{1.865950in}{1.738175in}}%
\pgfpathlineto{\pgfqpoint{1.849193in}{1.739378in}}%
\pgfpathlineto{\pgfqpoint{1.832437in}{1.740591in}}%
\pgfpathlineto{\pgfqpoint{1.815681in}{1.741592in}}%
\pgfpathlineto{\pgfqpoint{1.798925in}{1.742740in}}%
\pgfpathlineto{\pgfqpoint{1.782168in}{1.743904in}}%
\pgfpathlineto{\pgfqpoint{1.765412in}{1.745072in}}%
\pgfpathlineto{\pgfqpoint{1.748656in}{1.746281in}}%
\pgfpathlineto{\pgfqpoint{1.731900in}{1.747540in}}%
\pgfpathlineto{\pgfqpoint{1.715143in}{1.748748in}}%
\pgfpathlineto{\pgfqpoint{1.698387in}{1.749942in}}%
\pgfpathlineto{\pgfqpoint{1.681631in}{1.751164in}}%
\pgfpathlineto{\pgfqpoint{1.664874in}{1.752434in}}%
\pgfpathlineto{\pgfqpoint{1.648118in}{1.753729in}}%
\pgfpathlineto{\pgfqpoint{1.631362in}{1.754908in}}%
\pgfpathlineto{\pgfqpoint{1.614606in}{1.756089in}}%
\pgfpathlineto{\pgfqpoint{1.597849in}{1.757392in}}%
\pgfpathlineto{\pgfqpoint{1.581093in}{1.758527in}}%
\pgfpathlineto{\pgfqpoint{1.564337in}{1.759789in}}%
\pgfpathlineto{\pgfqpoint{1.547581in}{1.761063in}}%
\pgfpathlineto{\pgfqpoint{1.530824in}{1.762444in}}%
\pgfpathlineto{\pgfqpoint{1.514068in}{1.763664in}}%
\pgfpathlineto{\pgfqpoint{1.497312in}{1.764980in}}%
\pgfpathlineto{\pgfqpoint{1.480555in}{1.766254in}}%
\pgfpathlineto{\pgfqpoint{1.463799in}{1.767621in}}%
\pgfpathlineto{\pgfqpoint{1.447043in}{1.768943in}}%
\pgfpathlineto{\pgfqpoint{1.430287in}{1.770276in}}%
\pgfpathlineto{\pgfqpoint{1.413530in}{1.771565in}}%
\pgfpathlineto{\pgfqpoint{1.396774in}{1.772907in}}%
\pgfpathlineto{\pgfqpoint{1.380018in}{1.774363in}}%
\pgfpathlineto{\pgfqpoint{1.363262in}{1.775713in}}%
\pgfpathlineto{\pgfqpoint{1.346505in}{1.777086in}}%
\pgfpathlineto{\pgfqpoint{1.329749in}{1.778477in}}%
\pgfpathlineto{\pgfqpoint{1.312993in}{1.780046in}}%
\pgfpathlineto{\pgfqpoint{1.296236in}{1.781556in}}%
\pgfpathlineto{\pgfqpoint{1.279480in}{1.783020in}}%
\pgfpathlineto{\pgfqpoint{1.262724in}{1.784483in}}%
\pgfpathlineto{\pgfqpoint{1.245968in}{1.785823in}}%
\pgfpathlineto{\pgfqpoint{1.229211in}{1.787088in}}%
\pgfpathlineto{\pgfqpoint{1.212455in}{1.788753in}}%
\pgfpathlineto{\pgfqpoint{1.195699in}{1.790339in}}%
\pgfpathlineto{\pgfqpoint{1.178942in}{1.791806in}}%
\pgfpathlineto{\pgfqpoint{1.162186in}{1.793304in}}%
\pgfpathlineto{\pgfqpoint{1.145430in}{1.794865in}}%
\pgfpathlineto{\pgfqpoint{1.128674in}{1.796469in}}%
\pgfpathlineto{\pgfqpoint{1.111917in}{1.798076in}}%
\pgfpathlineto{\pgfqpoint{1.095161in}{1.799790in}}%
\pgfpathlineto{\pgfqpoint{1.078405in}{1.801547in}}%
\pgfpathlineto{\pgfqpoint{1.061649in}{1.803024in}}%
\pgfpathlineto{\pgfqpoint{1.044892in}{1.804736in}}%
\pgfpathlineto{\pgfqpoint{1.028136in}{1.806505in}}%
\pgfpathlineto{\pgfqpoint{1.011380in}{1.808372in}}%
\pgfpathlineto{\pgfqpoint{0.994623in}{1.810195in}}%
\pgfpathlineto{\pgfqpoint{0.977867in}{1.811829in}}%
\pgfpathlineto{\pgfqpoint{0.961111in}{1.813383in}}%
\pgfpathlineto{\pgfqpoint{0.944355in}{1.815005in}}%
\pgfpathlineto{\pgfqpoint{0.927598in}{1.816715in}}%
\pgfpathlineto{\pgfqpoint{0.910842in}{1.818425in}}%
\pgfpathlineto{\pgfqpoint{0.894086in}{1.820134in}}%
\pgfpathlineto{\pgfqpoint{0.877330in}{1.821772in}}%
\pgfpathlineto{\pgfqpoint{0.860573in}{1.823375in}}%
\pgfpathlineto{\pgfqpoint{0.843817in}{1.824898in}}%
\pgfpathlineto{\pgfqpoint{0.827061in}{1.826417in}}%
\pgfpathlineto{\pgfqpoint{0.810304in}{1.828112in}}%
\pgfpathlineto{\pgfqpoint{0.793548in}{1.829689in}}%
\pgfpathlineto{\pgfqpoint{0.776792in}{1.831334in}}%
\pgfpathlineto{\pgfqpoint{0.760036in}{1.833045in}}%
\pgfpathlineto{\pgfqpoint{0.743279in}{1.834796in}}%
\pgfpathlineto{\pgfqpoint{0.726523in}{1.836501in}}%
\pgfpathlineto{\pgfqpoint{0.709767in}{1.838197in}}%
\pgfpathlineto{\pgfqpoint{0.693011in}{1.839830in}}%
\pgfpathlineto{\pgfqpoint{0.676254in}{1.841533in}}%
\pgfpathlineto{\pgfqpoint{0.659498in}{1.843176in}}%
\pgfpathlineto{\pgfqpoint{0.642742in}{1.844820in}}%
\pgfpathlineto{\pgfqpoint{0.625985in}{1.846566in}}%
\pgfpathlineto{\pgfqpoint{0.609229in}{1.848362in}}%
\pgfpathlineto{\pgfqpoint{0.592473in}{1.850134in}}%
\pgfpathlineto{\pgfqpoint{0.575717in}{1.851928in}}%
\pgfpathlineto{\pgfqpoint{0.558960in}{1.853601in}}%
\pgfpathlineto{\pgfqpoint{0.542204in}{1.855254in}}%
\pgfpathlineto{\pgfqpoint{0.525448in}{1.856961in}}%
\pgfpathlineto{\pgfqpoint{0.508691in}{1.858668in}}%
\pgfpathlineto{\pgfqpoint{0.491935in}{1.860374in}}%
\pgfpathlineto{\pgfqpoint{0.475179in}{1.862081in}}%
\pgfpathlineto{\pgfqpoint{0.458423in}{1.863788in}}%
\pgfpathlineto{\pgfqpoint{0.441666in}{1.865583in}}%
\pgfpathlineto{\pgfqpoint{0.424910in}{1.867462in}}%
\pgfpathlineto{\pgfqpoint{0.408154in}{1.869341in}}%
\pgfpathlineto{\pgfqpoint{0.391398in}{1.871220in}}%
\pgfpathlineto{\pgfqpoint{0.374641in}{1.872994in}}%
\pgfpathlineto{\pgfqpoint{0.357885in}{1.874659in}}%
\pgfpathlineto{\pgfqpoint{0.341129in}{1.876323in}}%
\pgfpathclose%
\pgfusepath{stroke,fill}%
}%
\begin{pgfscope}%
\pgfsys@transformshift{0.000000in}{0.000000in}%
\pgfsys@useobject{currentmarker}{}%
\end{pgfscope}%
\end{pgfscope}%
\begin{pgfscope}%
\pgfpathrectangle{\pgfqpoint{0.341129in}{0.466613in}}{\pgfqpoint{1.658871in}{1.711598in}}%
\pgfusepath{clip}%
\pgfsetbuttcap%
\pgfsetroundjoin%
\definecolor{currentfill}{rgb}{0.333333,0.658824,0.407843}%
\pgfsetfillcolor{currentfill}%
\pgfsetfillopacity{0.250000}%
\pgfsetlinewidth{1.003750pt}%
\definecolor{currentstroke}{rgb}{0.333333,0.658824,0.407843}%
\pgfsetstrokecolor{currentstroke}%
\pgfsetstrokeopacity{0.250000}%
\pgfsetdash{}{0pt}%
\pgfsys@defobject{currentmarker}{\pgfqpoint{-0.017010in}{-0.017010in}}{\pgfqpoint{0.017010in}{0.017010in}}{%
\pgfpathmoveto{\pgfqpoint{0.000000in}{-0.017010in}}%
\pgfpathcurveto{\pgfqpoint{0.004511in}{-0.017010in}}{\pgfqpoint{0.008838in}{-0.015218in}}{\pgfqpoint{0.012028in}{-0.012028in}}%
\pgfpathcurveto{\pgfqpoint{0.015218in}{-0.008838in}}{\pgfqpoint{0.017010in}{-0.004511in}}{\pgfqpoint{0.017010in}{0.000000in}}%
\pgfpathcurveto{\pgfqpoint{0.017010in}{0.004511in}}{\pgfqpoint{0.015218in}{0.008838in}}{\pgfqpoint{0.012028in}{0.012028in}}%
\pgfpathcurveto{\pgfqpoint{0.008838in}{0.015218in}}{\pgfqpoint{0.004511in}{0.017010in}}{\pgfqpoint{0.000000in}{0.017010in}}%
\pgfpathcurveto{\pgfqpoint{-0.004511in}{0.017010in}}{\pgfqpoint{-0.008838in}{0.015218in}}{\pgfqpoint{-0.012028in}{0.012028in}}%
\pgfpathcurveto{\pgfqpoint{-0.015218in}{0.008838in}}{\pgfqpoint{-0.017010in}{0.004511in}}{\pgfqpoint{-0.017010in}{0.000000in}}%
\pgfpathcurveto{\pgfqpoint{-0.017010in}{-0.004511in}}{\pgfqpoint{-0.015218in}{-0.008838in}}{\pgfqpoint{-0.012028in}{-0.012028in}}%
\pgfpathcurveto{\pgfqpoint{-0.008838in}{-0.015218in}}{\pgfqpoint{-0.004511in}{-0.017010in}}{\pgfqpoint{0.000000in}{-0.017010in}}%
\pgfpathclose%
\pgfusepath{stroke,fill}%
}%
\begin{pgfscope}%
\pgfsys@transformshift{0.667456in}{1.604386in}%
\pgfsys@useobject{currentmarker}{}%
\end{pgfscope}%
\begin{pgfscope}%
\pgfsys@transformshift{0.872203in}{1.615570in}%
\pgfsys@useobject{currentmarker}{}%
\end{pgfscope}%
\begin{pgfscope}%
\pgfsys@transformshift{1.174456in}{1.513186in}%
\pgfsys@useobject{currentmarker}{}%
\end{pgfscope}%
\begin{pgfscope}%
\pgfsys@transformshift{1.594005in}{1.416879in}%
\pgfsys@useobject{currentmarker}{}%
\end{pgfscope}%
\begin{pgfscope}%
\pgfsys@transformshift{1.217413in}{1.348529in}%
\pgfsys@useobject{currentmarker}{}%
\end{pgfscope}%
\begin{pgfscope}%
\pgfsys@transformshift{0.553973in}{1.611492in}%
\pgfsys@useobject{currentmarker}{}%
\end{pgfscope}%
\begin{pgfscope}%
\pgfsys@transformshift{1.639332in}{1.287294in}%
\pgfsys@useobject{currentmarker}{}%
\end{pgfscope}%
\begin{pgfscope}%
\pgfsys@transformshift{0.709729in}{1.671881in}%
\pgfsys@useobject{currentmarker}{}%
\end{pgfscope}%
\begin{pgfscope}%
\pgfsys@transformshift{1.202352in}{1.603966in}%
\pgfsys@useobject{currentmarker}{}%
\end{pgfscope}%
\begin{pgfscope}%
\pgfsys@transformshift{0.671018in}{1.518927in}%
\pgfsys@useobject{currentmarker}{}%
\end{pgfscope}%
\begin{pgfscope}%
\pgfsys@transformshift{1.442003in}{1.454959in}%
\pgfsys@useobject{currentmarker}{}%
\end{pgfscope}%
\begin{pgfscope}%
\pgfsys@transformshift{1.398849in}{1.483030in}%
\pgfsys@useobject{currentmarker}{}%
\end{pgfscope}%
\begin{pgfscope}%
\pgfsys@transformshift{1.552752in}{1.315171in}%
\pgfsys@useobject{currentmarker}{}%
\end{pgfscope}%
\begin{pgfscope}%
\pgfsys@transformshift{1.177147in}{1.518919in}%
\pgfsys@useobject{currentmarker}{}%
\end{pgfscope}%
\begin{pgfscope}%
\pgfsys@transformshift{1.887294in}{1.437366in}%
\pgfsys@useobject{currentmarker}{}%
\end{pgfscope}%
\begin{pgfscope}%
\pgfsys@transformshift{1.188970in}{1.447620in}%
\pgfsys@useobject{currentmarker}{}%
\end{pgfscope}%
\begin{pgfscope}%
\pgfsys@transformshift{1.534509in}{1.525909in}%
\pgfsys@useobject{currentmarker}{}%
\end{pgfscope}%
\begin{pgfscope}%
\pgfsys@transformshift{0.588572in}{1.618968in}%
\pgfsys@useobject{currentmarker}{}%
\end{pgfscope}%
\begin{pgfscope}%
\pgfsys@transformshift{0.952296in}{1.539905in}%
\pgfsys@useobject{currentmarker}{}%
\end{pgfscope}%
\begin{pgfscope}%
\pgfsys@transformshift{0.705522in}{1.548171in}%
\pgfsys@useobject{currentmarker}{}%
\end{pgfscope}%
\begin{pgfscope}%
\pgfsys@transformshift{1.121384in}{1.440595in}%
\pgfsys@useobject{currentmarker}{}%
\end{pgfscope}%
\begin{pgfscope}%
\pgfsys@transformshift{1.103574in}{1.440666in}%
\pgfsys@useobject{currentmarker}{}%
\end{pgfscope}%
\begin{pgfscope}%
\pgfsys@transformshift{1.817005in}{1.454745in}%
\pgfsys@useobject{currentmarker}{}%
\end{pgfscope}%
\begin{pgfscope}%
\pgfsys@transformshift{0.496492in}{1.680987in}%
\pgfsys@useobject{currentmarker}{}%
\end{pgfscope}%
\begin{pgfscope}%
\pgfsys@transformshift{1.711311in}{1.432636in}%
\pgfsys@useobject{currentmarker}{}%
\end{pgfscope}%
\begin{pgfscope}%
\pgfsys@transformshift{1.750408in}{1.163092in}%
\pgfsys@useobject{currentmarker}{}%
\end{pgfscope}%
\begin{pgfscope}%
\pgfsys@transformshift{1.333406in}{1.391788in}%
\pgfsys@useobject{currentmarker}{}%
\end{pgfscope}%
\begin{pgfscope}%
\pgfsys@transformshift{1.724223in}{1.349455in}%
\pgfsys@useobject{currentmarker}{}%
\end{pgfscope}%
\begin{pgfscope}%
\pgfsys@transformshift{1.256999in}{1.493328in}%
\pgfsys@useobject{currentmarker}{}%
\end{pgfscope}%
\begin{pgfscope}%
\pgfsys@transformshift{1.047649in}{1.501668in}%
\pgfsys@useobject{currentmarker}{}%
\end{pgfscope}%
\begin{pgfscope}%
\pgfsys@transformshift{0.878614in}{1.601000in}%
\pgfsys@useobject{currentmarker}{}%
\end{pgfscope}%
\begin{pgfscope}%
\pgfsys@transformshift{0.761954in}{1.612705in}%
\pgfsys@useobject{currentmarker}{}%
\end{pgfscope}%
\begin{pgfscope}%
\pgfsys@transformshift{1.796268in}{1.276415in}%
\pgfsys@useobject{currentmarker}{}%
\end{pgfscope}%
\begin{pgfscope}%
\pgfsys@transformshift{0.726378in}{1.593350in}%
\pgfsys@useobject{currentmarker}{}%
\end{pgfscope}%
\begin{pgfscope}%
\pgfsys@transformshift{1.408196in}{1.471706in}%
\pgfsys@useobject{currentmarker}{}%
\end{pgfscope}%
\begin{pgfscope}%
\pgfsys@transformshift{0.612217in}{1.323516in}%
\pgfsys@useobject{currentmarker}{}%
\end{pgfscope}%
\begin{pgfscope}%
\pgfsys@transformshift{1.180902in}{1.624478in}%
\pgfsys@useobject{currentmarker}{}%
\end{pgfscope}%
\begin{pgfscope}%
\pgfsys@transformshift{0.627344in}{1.649096in}%
\pgfsys@useobject{currentmarker}{}%
\end{pgfscope}%
\begin{pgfscope}%
\pgfsys@transformshift{0.628282in}{1.585633in}%
\pgfsys@useobject{currentmarker}{}%
\end{pgfscope}%
\begin{pgfscope}%
\pgfsys@transformshift{0.437992in}{1.317357in}%
\pgfsys@useobject{currentmarker}{}%
\end{pgfscope}%
\begin{pgfscope}%
\pgfsys@transformshift{0.758691in}{1.562821in}%
\pgfsys@useobject{currentmarker}{}%
\end{pgfscope}%
\begin{pgfscope}%
\pgfsys@transformshift{1.614220in}{1.532482in}%
\pgfsys@useobject{currentmarker}{}%
\end{pgfscope}%
\begin{pgfscope}%
\pgfsys@transformshift{1.846563in}{1.341515in}%
\pgfsys@useobject{currentmarker}{}%
\end{pgfscope}%
\begin{pgfscope}%
\pgfsys@transformshift{1.460173in}{1.505933in}%
\pgfsys@useobject{currentmarker}{}%
\end{pgfscope}%
\begin{pgfscope}%
\pgfsys@transformshift{1.637112in}{1.349124in}%
\pgfsys@useobject{currentmarker}{}%
\end{pgfscope}%
\begin{pgfscope}%
\pgfsys@transformshift{1.615512in}{1.276969in}%
\pgfsys@useobject{currentmarker}{}%
\end{pgfscope}%
\begin{pgfscope}%
\pgfsys@transformshift{1.242112in}{1.664582in}%
\pgfsys@useobject{currentmarker}{}%
\end{pgfscope}%
\begin{pgfscope}%
\pgfsys@transformshift{0.688214in}{1.554814in}%
\pgfsys@useobject{currentmarker}{}%
\end{pgfscope}%
\begin{pgfscope}%
\pgfsys@transformshift{0.911315in}{1.306707in}%
\pgfsys@useobject{currentmarker}{}%
\end{pgfscope}%
\begin{pgfscope}%
\pgfsys@transformshift{0.551462in}{1.712968in}%
\pgfsys@useobject{currentmarker}{}%
\end{pgfscope}%
\begin{pgfscope}%
\pgfsys@transformshift{1.069973in}{1.514444in}%
\pgfsys@useobject{currentmarker}{}%
\end{pgfscope}%
\begin{pgfscope}%
\pgfsys@transformshift{1.211180in}{1.631055in}%
\pgfsys@useobject{currentmarker}{}%
\end{pgfscope}%
\begin{pgfscope}%
\pgfsys@transformshift{1.485757in}{1.613793in}%
\pgfsys@useobject{currentmarker}{}%
\end{pgfscope}%
\begin{pgfscope}%
\pgfsys@transformshift{0.743119in}{1.489676in}%
\pgfsys@useobject{currentmarker}{}%
\end{pgfscope}%
\begin{pgfscope}%
\pgfsys@transformshift{1.407385in}{1.368569in}%
\pgfsys@useobject{currentmarker}{}%
\end{pgfscope}%
\begin{pgfscope}%
\pgfsys@transformshift{1.618799in}{1.634128in}%
\pgfsys@useobject{currentmarker}{}%
\end{pgfscope}%
\begin{pgfscope}%
\pgfsys@transformshift{1.477965in}{1.359375in}%
\pgfsys@useobject{currentmarker}{}%
\end{pgfscope}%
\begin{pgfscope}%
\pgfsys@transformshift{1.156480in}{1.623104in}%
\pgfsys@useobject{currentmarker}{}%
\end{pgfscope}%
\begin{pgfscope}%
\pgfsys@transformshift{1.583657in}{1.383877in}%
\pgfsys@useobject{currentmarker}{}%
\end{pgfscope}%
\begin{pgfscope}%
\pgfsys@transformshift{1.440624in}{1.400285in}%
\pgfsys@useobject{currentmarker}{}%
\end{pgfscope}%
\begin{pgfscope}%
\pgfsys@transformshift{1.318935in}{1.420215in}%
\pgfsys@useobject{currentmarker}{}%
\end{pgfscope}%
\begin{pgfscope}%
\pgfsys@transformshift{0.547443in}{1.721011in}%
\pgfsys@useobject{currentmarker}{}%
\end{pgfscope}%
\begin{pgfscope}%
\pgfsys@transformshift{1.436824in}{1.507777in}%
\pgfsys@useobject{currentmarker}{}%
\end{pgfscope}%
\begin{pgfscope}%
\pgfsys@transformshift{1.658179in}{1.310661in}%
\pgfsys@useobject{currentmarker}{}%
\end{pgfscope}%
\begin{pgfscope}%
\pgfsys@transformshift{1.709942in}{1.374891in}%
\pgfsys@useobject{currentmarker}{}%
\end{pgfscope}%
\begin{pgfscope}%
\pgfsys@transformshift{1.638310in}{1.328522in}%
\pgfsys@useobject{currentmarker}{}%
\end{pgfscope}%
\begin{pgfscope}%
\pgfsys@transformshift{1.723779in}{1.376423in}%
\pgfsys@useobject{currentmarker}{}%
\end{pgfscope}%
\begin{pgfscope}%
\pgfsys@transformshift{1.124544in}{1.505426in}%
\pgfsys@useobject{currentmarker}{}%
\end{pgfscope}%
\begin{pgfscope}%
\pgfsys@transformshift{1.267442in}{1.535808in}%
\pgfsys@useobject{currentmarker}{}%
\end{pgfscope}%
\begin{pgfscope}%
\pgfsys@transformshift{0.562064in}{1.228280in}%
\pgfsys@useobject{currentmarker}{}%
\end{pgfscope}%
\begin{pgfscope}%
\pgfsys@transformshift{0.642722in}{1.564469in}%
\pgfsys@useobject{currentmarker}{}%
\end{pgfscope}%
\begin{pgfscope}%
\pgfsys@transformshift{1.150269in}{1.685641in}%
\pgfsys@useobject{currentmarker}{}%
\end{pgfscope}%
\begin{pgfscope}%
\pgfsys@transformshift{1.173281in}{1.486255in}%
\pgfsys@useobject{currentmarker}{}%
\end{pgfscope}%
\begin{pgfscope}%
\pgfsys@transformshift{1.891719in}{1.472874in}%
\pgfsys@useobject{currentmarker}{}%
\end{pgfscope}%
\begin{pgfscope}%
\pgfsys@transformshift{0.694241in}{1.577801in}%
\pgfsys@useobject{currentmarker}{}%
\end{pgfscope}%
\begin{pgfscope}%
\pgfsys@transformshift{0.790322in}{1.571977in}%
\pgfsys@useobject{currentmarker}{}%
\end{pgfscope}%
\begin{pgfscope}%
\pgfsys@transformshift{1.424550in}{1.336925in}%
\pgfsys@useobject{currentmarker}{}%
\end{pgfscope}%
\begin{pgfscope}%
\pgfsys@transformshift{0.384790in}{1.293073in}%
\pgfsys@useobject{currentmarker}{}%
\end{pgfscope}%
\begin{pgfscope}%
\pgfsys@transformshift{0.729116in}{1.581876in}%
\pgfsys@useobject{currentmarker}{}%
\end{pgfscope}%
\begin{pgfscope}%
\pgfsys@transformshift{1.217637in}{1.546742in}%
\pgfsys@useobject{currentmarker}{}%
\end{pgfscope}%
\begin{pgfscope}%
\pgfsys@transformshift{0.452540in}{1.607863in}%
\pgfsys@useobject{currentmarker}{}%
\end{pgfscope}%
\begin{pgfscope}%
\pgfsys@transformshift{1.493704in}{1.309222in}%
\pgfsys@useobject{currentmarker}{}%
\end{pgfscope}%
\begin{pgfscope}%
\pgfsys@transformshift{1.543302in}{1.471006in}%
\pgfsys@useobject{currentmarker}{}%
\end{pgfscope}%
\begin{pgfscope}%
\pgfsys@transformshift{1.218014in}{1.448924in}%
\pgfsys@useobject{currentmarker}{}%
\end{pgfscope}%
\begin{pgfscope}%
\pgfsys@transformshift{1.521861in}{1.450818in}%
\pgfsys@useobject{currentmarker}{}%
\end{pgfscope}%
\begin{pgfscope}%
\pgfsys@transformshift{1.764286in}{1.427494in}%
\pgfsys@useobject{currentmarker}{}%
\end{pgfscope}%
\begin{pgfscope}%
\pgfsys@transformshift{0.924889in}{1.723651in}%
\pgfsys@useobject{currentmarker}{}%
\end{pgfscope}%
\begin{pgfscope}%
\pgfsys@transformshift{1.222866in}{1.330531in}%
\pgfsys@useobject{currentmarker}{}%
\end{pgfscope}%
\begin{pgfscope}%
\pgfsys@transformshift{1.382222in}{1.465268in}%
\pgfsys@useobject{currentmarker}{}%
\end{pgfscope}%
\begin{pgfscope}%
\pgfsys@transformshift{1.643945in}{1.498042in}%
\pgfsys@useobject{currentmarker}{}%
\end{pgfscope}%
\begin{pgfscope}%
\pgfsys@transformshift{0.801834in}{1.480125in}%
\pgfsys@useobject{currentmarker}{}%
\end{pgfscope}%
\begin{pgfscope}%
\pgfsys@transformshift{1.258384in}{1.505611in}%
\pgfsys@useobject{currentmarker}{}%
\end{pgfscope}%
\begin{pgfscope}%
\pgfsys@transformshift{1.394783in}{1.420347in}%
\pgfsys@useobject{currentmarker}{}%
\end{pgfscope}%
\begin{pgfscope}%
\pgfsys@transformshift{0.341129in}{1.741645in}%
\pgfsys@useobject{currentmarker}{}%
\end{pgfscope}%
\begin{pgfscope}%
\pgfsys@transformshift{1.069581in}{1.589132in}%
\pgfsys@useobject{currentmarker}{}%
\end{pgfscope}%
\begin{pgfscope}%
\pgfsys@transformshift{0.789606in}{1.725873in}%
\pgfsys@useobject{currentmarker}{}%
\end{pgfscope}%
\begin{pgfscope}%
\pgfsys@transformshift{1.273287in}{1.385061in}%
\pgfsys@useobject{currentmarker}{}%
\end{pgfscope}%
\begin{pgfscope}%
\pgfsys@transformshift{0.646998in}{1.640484in}%
\pgfsys@useobject{currentmarker}{}%
\end{pgfscope}%
\begin{pgfscope}%
\pgfsys@transformshift{0.761065in}{1.645310in}%
\pgfsys@useobject{currentmarker}{}%
\end{pgfscope}%
\begin{pgfscope}%
\pgfsys@transformshift{0.628168in}{1.770212in}%
\pgfsys@useobject{currentmarker}{}%
\end{pgfscope}%
\begin{pgfscope}%
\pgfsys@transformshift{1.214800in}{1.529337in}%
\pgfsys@useobject{currentmarker}{}%
\end{pgfscope}%
\begin{pgfscope}%
\pgfsys@transformshift{1.051231in}{1.564008in}%
\pgfsys@useobject{currentmarker}{}%
\end{pgfscope}%
\begin{pgfscope}%
\pgfsys@transformshift{1.010738in}{1.520038in}%
\pgfsys@useobject{currentmarker}{}%
\end{pgfscope}%
\begin{pgfscope}%
\pgfsys@transformshift{1.387136in}{1.396653in}%
\pgfsys@useobject{currentmarker}{}%
\end{pgfscope}%
\begin{pgfscope}%
\pgfsys@transformshift{0.522501in}{1.249974in}%
\pgfsys@useobject{currentmarker}{}%
\end{pgfscope}%
\begin{pgfscope}%
\pgfsys@transformshift{1.737851in}{1.534814in}%
\pgfsys@useobject{currentmarker}{}%
\end{pgfscope}%
\begin{pgfscope}%
\pgfsys@transformshift{1.622475in}{1.404350in}%
\pgfsys@useobject{currentmarker}{}%
\end{pgfscope}%
\begin{pgfscope}%
\pgfsys@transformshift{0.776054in}{1.313771in}%
\pgfsys@useobject{currentmarker}{}%
\end{pgfscope}%
\begin{pgfscope}%
\pgfsys@transformshift{1.260864in}{1.255940in}%
\pgfsys@useobject{currentmarker}{}%
\end{pgfscope}%
\begin{pgfscope}%
\pgfsys@transformshift{1.587232in}{1.379641in}%
\pgfsys@useobject{currentmarker}{}%
\end{pgfscope}%
\begin{pgfscope}%
\pgfsys@transformshift{1.339693in}{1.491067in}%
\pgfsys@useobject{currentmarker}{}%
\end{pgfscope}%
\begin{pgfscope}%
\pgfsys@transformshift{1.448928in}{1.282791in}%
\pgfsys@useobject{currentmarker}{}%
\end{pgfscope}%
\begin{pgfscope}%
\pgfsys@transformshift{1.848941in}{1.184177in}%
\pgfsys@useobject{currentmarker}{}%
\end{pgfscope}%
\begin{pgfscope}%
\pgfsys@transformshift{1.645721in}{1.416584in}%
\pgfsys@useobject{currentmarker}{}%
\end{pgfscope}%
\begin{pgfscope}%
\pgfsys@transformshift{0.980835in}{1.554159in}%
\pgfsys@useobject{currentmarker}{}%
\end{pgfscope}%
\begin{pgfscope}%
\pgfsys@transformshift{0.704938in}{1.683943in}%
\pgfsys@useobject{currentmarker}{}%
\end{pgfscope}%
\begin{pgfscope}%
\pgfsys@transformshift{0.622306in}{1.687763in}%
\pgfsys@useobject{currentmarker}{}%
\end{pgfscope}%
\begin{pgfscope}%
\pgfsys@transformshift{1.213744in}{1.279606in}%
\pgfsys@useobject{currentmarker}{}%
\end{pgfscope}%
\begin{pgfscope}%
\pgfsys@transformshift{0.934075in}{1.610839in}%
\pgfsys@useobject{currentmarker}{}%
\end{pgfscope}%
\begin{pgfscope}%
\pgfsys@transformshift{1.307417in}{1.346548in}%
\pgfsys@useobject{currentmarker}{}%
\end{pgfscope}%
\begin{pgfscope}%
\pgfsys@transformshift{1.605349in}{1.586684in}%
\pgfsys@useobject{currentmarker}{}%
\end{pgfscope}%
\begin{pgfscope}%
\pgfsys@transformshift{0.750394in}{1.607925in}%
\pgfsys@useobject{currentmarker}{}%
\end{pgfscope}%
\begin{pgfscope}%
\pgfsys@transformshift{1.771258in}{1.469420in}%
\pgfsys@useobject{currentmarker}{}%
\end{pgfscope}%
\begin{pgfscope}%
\pgfsys@transformshift{1.939172in}{1.448781in}%
\pgfsys@useobject{currentmarker}{}%
\end{pgfscope}%
\begin{pgfscope}%
\pgfsys@transformshift{1.599583in}{1.550546in}%
\pgfsys@useobject{currentmarker}{}%
\end{pgfscope}%
\begin{pgfscope}%
\pgfsys@transformshift{0.881257in}{1.560353in}%
\pgfsys@useobject{currentmarker}{}%
\end{pgfscope}%
\begin{pgfscope}%
\pgfsys@transformshift{1.201768in}{1.606930in}%
\pgfsys@useobject{currentmarker}{}%
\end{pgfscope}%
\begin{pgfscope}%
\pgfsys@transformshift{1.244134in}{1.541934in}%
\pgfsys@useobject{currentmarker}{}%
\end{pgfscope}%
\begin{pgfscope}%
\pgfsys@transformshift{1.750150in}{1.463959in}%
\pgfsys@useobject{currentmarker}{}%
\end{pgfscope}%
\begin{pgfscope}%
\pgfsys@transformshift{1.541153in}{1.443522in}%
\pgfsys@useobject{currentmarker}{}%
\end{pgfscope}%
\begin{pgfscope}%
\pgfsys@transformshift{0.964663in}{1.338376in}%
\pgfsys@useobject{currentmarker}{}%
\end{pgfscope}%
\begin{pgfscope}%
\pgfsys@transformshift{0.818109in}{1.522293in}%
\pgfsys@useobject{currentmarker}{}%
\end{pgfscope}%
\begin{pgfscope}%
\pgfsys@transformshift{1.755264in}{1.322526in}%
\pgfsys@useobject{currentmarker}{}%
\end{pgfscope}%
\begin{pgfscope}%
\pgfsys@transformshift{1.450480in}{1.524912in}%
\pgfsys@useobject{currentmarker}{}%
\end{pgfscope}%
\begin{pgfscope}%
\pgfsys@transformshift{1.246595in}{1.599574in}%
\pgfsys@useobject{currentmarker}{}%
\end{pgfscope}%
\begin{pgfscope}%
\pgfsys@transformshift{1.110348in}{1.632937in}%
\pgfsys@useobject{currentmarker}{}%
\end{pgfscope}%
\begin{pgfscope}%
\pgfsys@transformshift{0.853154in}{1.465758in}%
\pgfsys@useobject{currentmarker}{}%
\end{pgfscope}%
\begin{pgfscope}%
\pgfsys@transformshift{0.803206in}{1.601650in}%
\pgfsys@useobject{currentmarker}{}%
\end{pgfscope}%
\begin{pgfscope}%
\pgfsys@transformshift{0.800584in}{1.589769in}%
\pgfsys@useobject{currentmarker}{}%
\end{pgfscope}%
\begin{pgfscope}%
\pgfsys@transformshift{0.828786in}{1.575128in}%
\pgfsys@useobject{currentmarker}{}%
\end{pgfscope}%
\begin{pgfscope}%
\pgfsys@transformshift{0.802472in}{1.673947in}%
\pgfsys@useobject{currentmarker}{}%
\end{pgfscope}%
\begin{pgfscope}%
\pgfsys@transformshift{0.778610in}{1.674974in}%
\pgfsys@useobject{currentmarker}{}%
\end{pgfscope}%
\begin{pgfscope}%
\pgfsys@transformshift{1.678430in}{1.473371in}%
\pgfsys@useobject{currentmarker}{}%
\end{pgfscope}%
\begin{pgfscope}%
\pgfsys@transformshift{1.678534in}{1.508917in}%
\pgfsys@useobject{currentmarker}{}%
\end{pgfscope}%
\begin{pgfscope}%
\pgfsys@transformshift{1.164510in}{1.463634in}%
\pgfsys@useobject{currentmarker}{}%
\end{pgfscope}%
\begin{pgfscope}%
\pgfsys@transformshift{1.533885in}{1.418117in}%
\pgfsys@useobject{currentmarker}{}%
\end{pgfscope}%
\begin{pgfscope}%
\pgfsys@transformshift{0.532239in}{1.586026in}%
\pgfsys@useobject{currentmarker}{}%
\end{pgfscope}%
\begin{pgfscope}%
\pgfsys@transformshift{1.059630in}{1.435146in}%
\pgfsys@useobject{currentmarker}{}%
\end{pgfscope}%
\begin{pgfscope}%
\pgfsys@transformshift{1.471016in}{1.462355in}%
\pgfsys@useobject{currentmarker}{}%
\end{pgfscope}%
\begin{pgfscope}%
\pgfsys@transformshift{0.764422in}{1.553737in}%
\pgfsys@useobject{currentmarker}{}%
\end{pgfscope}%
\begin{pgfscope}%
\pgfsys@transformshift{1.724626in}{1.327380in}%
\pgfsys@useobject{currentmarker}{}%
\end{pgfscope}%
\begin{pgfscope}%
\pgfsys@transformshift{1.557866in}{1.345664in}%
\pgfsys@useobject{currentmarker}{}%
\end{pgfscope}%
\begin{pgfscope}%
\pgfsys@transformshift{1.322665in}{1.407131in}%
\pgfsys@useobject{currentmarker}{}%
\end{pgfscope}%
\begin{pgfscope}%
\pgfsys@transformshift{1.043939in}{1.533900in}%
\pgfsys@useobject{currentmarker}{}%
\end{pgfscope}%
\begin{pgfscope}%
\pgfsys@transformshift{1.615078in}{1.466294in}%
\pgfsys@useobject{currentmarker}{}%
\end{pgfscope}%
\begin{pgfscope}%
\pgfsys@transformshift{1.467591in}{1.571247in}%
\pgfsys@useobject{currentmarker}{}%
\end{pgfscope}%
\begin{pgfscope}%
\pgfsys@transformshift{0.991846in}{1.470424in}%
\pgfsys@useobject{currentmarker}{}%
\end{pgfscope}%
\begin{pgfscope}%
\pgfsys@transformshift{0.697891in}{1.516796in}%
\pgfsys@useobject{currentmarker}{}%
\end{pgfscope}%
\begin{pgfscope}%
\pgfsys@transformshift{1.294350in}{1.645582in}%
\pgfsys@useobject{currentmarker}{}%
\end{pgfscope}%
\begin{pgfscope}%
\pgfsys@transformshift{1.409354in}{1.307189in}%
\pgfsys@useobject{currentmarker}{}%
\end{pgfscope}%
\begin{pgfscope}%
\pgfsys@transformshift{0.733011in}{1.756530in}%
\pgfsys@useobject{currentmarker}{}%
\end{pgfscope}%
\begin{pgfscope}%
\pgfsys@transformshift{0.634257in}{1.599423in}%
\pgfsys@useobject{currentmarker}{}%
\end{pgfscope}%
\begin{pgfscope}%
\pgfsys@transformshift{0.526826in}{1.703566in}%
\pgfsys@useobject{currentmarker}{}%
\end{pgfscope}%
\begin{pgfscope}%
\pgfsys@transformshift{1.225515in}{1.567509in}%
\pgfsys@useobject{currentmarker}{}%
\end{pgfscope}%
\begin{pgfscope}%
\pgfsys@transformshift{0.578368in}{1.372758in}%
\pgfsys@useobject{currentmarker}{}%
\end{pgfscope}%
\begin{pgfscope}%
\pgfsys@transformshift{1.515054in}{1.503835in}%
\pgfsys@useobject{currentmarker}{}%
\end{pgfscope}%
\begin{pgfscope}%
\pgfsys@transformshift{1.989373in}{1.417246in}%
\pgfsys@useobject{currentmarker}{}%
\end{pgfscope}%
\begin{pgfscope}%
\pgfsys@transformshift{1.360084in}{1.626446in}%
\pgfsys@useobject{currentmarker}{}%
\end{pgfscope}%
\begin{pgfscope}%
\pgfsys@transformshift{0.851541in}{1.556922in}%
\pgfsys@useobject{currentmarker}{}%
\end{pgfscope}%
\begin{pgfscope}%
\pgfsys@transformshift{1.688966in}{1.416271in}%
\pgfsys@useobject{currentmarker}{}%
\end{pgfscope}%
\begin{pgfscope}%
\pgfsys@transformshift{1.144955in}{1.495837in}%
\pgfsys@useobject{currentmarker}{}%
\end{pgfscope}%
\begin{pgfscope}%
\pgfsys@transformshift{1.681418in}{1.291307in}%
\pgfsys@useobject{currentmarker}{}%
\end{pgfscope}%
\begin{pgfscope}%
\pgfsys@transformshift{1.133992in}{1.667493in}%
\pgfsys@useobject{currentmarker}{}%
\end{pgfscope}%
\begin{pgfscope}%
\pgfsys@transformshift{1.480556in}{1.487928in}%
\pgfsys@useobject{currentmarker}{}%
\end{pgfscope}%
\begin{pgfscope}%
\pgfsys@transformshift{1.339321in}{1.519389in}%
\pgfsys@useobject{currentmarker}{}%
\end{pgfscope}%
\begin{pgfscope}%
\pgfsys@transformshift{0.874156in}{1.466421in}%
\pgfsys@useobject{currentmarker}{}%
\end{pgfscope}%
\begin{pgfscope}%
\pgfsys@transformshift{1.592077in}{1.303139in}%
\pgfsys@useobject{currentmarker}{}%
\end{pgfscope}%
\begin{pgfscope}%
\pgfsys@transformshift{1.288980in}{1.537801in}%
\pgfsys@useobject{currentmarker}{}%
\end{pgfscope}%
\begin{pgfscope}%
\pgfsys@transformshift{0.987975in}{1.398170in}%
\pgfsys@useobject{currentmarker}{}%
\end{pgfscope}%
\begin{pgfscope}%
\pgfsys@transformshift{1.714750in}{1.358357in}%
\pgfsys@useobject{currentmarker}{}%
\end{pgfscope}%
\begin{pgfscope}%
\pgfsys@transformshift{1.365090in}{1.439494in}%
\pgfsys@useobject{currentmarker}{}%
\end{pgfscope}%
\begin{pgfscope}%
\pgfsys@transformshift{1.930143in}{1.364247in}%
\pgfsys@useobject{currentmarker}{}%
\end{pgfscope}%
\begin{pgfscope}%
\pgfsys@transformshift{1.850728in}{1.214445in}%
\pgfsys@useobject{currentmarker}{}%
\end{pgfscope}%
\begin{pgfscope}%
\pgfsys@transformshift{0.856167in}{1.636122in}%
\pgfsys@useobject{currentmarker}{}%
\end{pgfscope}%
\begin{pgfscope}%
\pgfsys@transformshift{1.724113in}{1.448644in}%
\pgfsys@useobject{currentmarker}{}%
\end{pgfscope}%
\begin{pgfscope}%
\pgfsys@transformshift{0.746656in}{1.654995in}%
\pgfsys@useobject{currentmarker}{}%
\end{pgfscope}%
\begin{pgfscope}%
\pgfsys@transformshift{0.981125in}{1.569414in}%
\pgfsys@useobject{currentmarker}{}%
\end{pgfscope}%
\begin{pgfscope}%
\pgfsys@transformshift{0.840360in}{1.481223in}%
\pgfsys@useobject{currentmarker}{}%
\end{pgfscope}%
\begin{pgfscope}%
\pgfsys@transformshift{0.780491in}{1.521121in}%
\pgfsys@useobject{currentmarker}{}%
\end{pgfscope}%
\begin{pgfscope}%
\pgfsys@transformshift{1.873816in}{1.373536in}%
\pgfsys@useobject{currentmarker}{}%
\end{pgfscope}%
\begin{pgfscope}%
\pgfsys@transformshift{1.594601in}{1.481824in}%
\pgfsys@useobject{currentmarker}{}%
\end{pgfscope}%
\begin{pgfscope}%
\pgfsys@transformshift{0.815736in}{1.577967in}%
\pgfsys@useobject{currentmarker}{}%
\end{pgfscope}%
\begin{pgfscope}%
\pgfsys@transformshift{1.864796in}{1.123165in}%
\pgfsys@useobject{currentmarker}{}%
\end{pgfscope}%
\begin{pgfscope}%
\pgfsys@transformshift{1.367631in}{1.625000in}%
\pgfsys@useobject{currentmarker}{}%
\end{pgfscope}%
\begin{pgfscope}%
\pgfsys@transformshift{1.653345in}{1.489220in}%
\pgfsys@useobject{currentmarker}{}%
\end{pgfscope}%
\begin{pgfscope}%
\pgfsys@transformshift{1.768545in}{1.327780in}%
\pgfsys@useobject{currentmarker}{}%
\end{pgfscope}%
\begin{pgfscope}%
\pgfsys@transformshift{1.244730in}{1.469697in}%
\pgfsys@useobject{currentmarker}{}%
\end{pgfscope}%
\begin{pgfscope}%
\pgfsys@transformshift{0.782047in}{1.667936in}%
\pgfsys@useobject{currentmarker}{}%
\end{pgfscope}%
\begin{pgfscope}%
\pgfsys@transformshift{1.846677in}{1.401679in}%
\pgfsys@useobject{currentmarker}{}%
\end{pgfscope}%
\begin{pgfscope}%
\pgfsys@transformshift{1.775739in}{1.348547in}%
\pgfsys@useobject{currentmarker}{}%
\end{pgfscope}%
\begin{pgfscope}%
\pgfsys@transformshift{1.703721in}{1.384711in}%
\pgfsys@useobject{currentmarker}{}%
\end{pgfscope}%
\begin{pgfscope}%
\pgfsys@transformshift{0.834121in}{1.558527in}%
\pgfsys@useobject{currentmarker}{}%
\end{pgfscope}%
\begin{pgfscope}%
\pgfsys@transformshift{1.119982in}{1.617817in}%
\pgfsys@useobject{currentmarker}{}%
\end{pgfscope}%
\begin{pgfscope}%
\pgfsys@transformshift{1.891923in}{1.326912in}%
\pgfsys@useobject{currentmarker}{}%
\end{pgfscope}%
\begin{pgfscope}%
\pgfsys@transformshift{1.690465in}{1.229530in}%
\pgfsys@useobject{currentmarker}{}%
\end{pgfscope}%
\begin{pgfscope}%
\pgfsys@transformshift{0.559984in}{1.624140in}%
\pgfsys@useobject{currentmarker}{}%
\end{pgfscope}%
\begin{pgfscope}%
\pgfsys@transformshift{1.447613in}{1.460160in}%
\pgfsys@useobject{currentmarker}{}%
\end{pgfscope}%
\begin{pgfscope}%
\pgfsys@transformshift{0.897928in}{1.687986in}%
\pgfsys@useobject{currentmarker}{}%
\end{pgfscope}%
\begin{pgfscope}%
\pgfsys@transformshift{1.417775in}{1.514487in}%
\pgfsys@useobject{currentmarker}{}%
\end{pgfscope}%
\begin{pgfscope}%
\pgfsys@transformshift{1.673252in}{1.310991in}%
\pgfsys@useobject{currentmarker}{}%
\end{pgfscope}%
\begin{pgfscope}%
\pgfsys@transformshift{1.412143in}{1.312782in}%
\pgfsys@useobject{currentmarker}{}%
\end{pgfscope}%
\begin{pgfscope}%
\pgfsys@transformshift{1.641912in}{1.460367in}%
\pgfsys@useobject{currentmarker}{}%
\end{pgfscope}%
\begin{pgfscope}%
\pgfsys@transformshift{1.388318in}{1.598844in}%
\pgfsys@useobject{currentmarker}{}%
\end{pgfscope}%
\begin{pgfscope}%
\pgfsys@transformshift{1.495487in}{1.420211in}%
\pgfsys@useobject{currentmarker}{}%
\end{pgfscope}%
\begin{pgfscope}%
\pgfsys@transformshift{1.918935in}{1.164978in}%
\pgfsys@useobject{currentmarker}{}%
\end{pgfscope}%
\begin{pgfscope}%
\pgfsys@transformshift{0.550025in}{1.566392in}%
\pgfsys@useobject{currentmarker}{}%
\end{pgfscope}%
\begin{pgfscope}%
\pgfsys@transformshift{1.821318in}{1.268702in}%
\pgfsys@useobject{currentmarker}{}%
\end{pgfscope}%
\begin{pgfscope}%
\pgfsys@transformshift{0.956955in}{1.515535in}%
\pgfsys@useobject{currentmarker}{}%
\end{pgfscope}%
\begin{pgfscope}%
\pgfsys@transformshift{0.895984in}{1.563220in}%
\pgfsys@useobject{currentmarker}{}%
\end{pgfscope}%
\begin{pgfscope}%
\pgfsys@transformshift{0.467116in}{1.345581in}%
\pgfsys@useobject{currentmarker}{}%
\end{pgfscope}%
\begin{pgfscope}%
\pgfsys@transformshift{1.606478in}{1.483342in}%
\pgfsys@useobject{currentmarker}{}%
\end{pgfscope}%
\begin{pgfscope}%
\pgfsys@transformshift{1.423682in}{1.506685in}%
\pgfsys@useobject{currentmarker}{}%
\end{pgfscope}%
\begin{pgfscope}%
\pgfsys@transformshift{1.436312in}{1.595645in}%
\pgfsys@useobject{currentmarker}{}%
\end{pgfscope}%
\begin{pgfscope}%
\pgfsys@transformshift{0.970257in}{1.708676in}%
\pgfsys@useobject{currentmarker}{}%
\end{pgfscope}%
\begin{pgfscope}%
\pgfsys@transformshift{1.482495in}{1.490892in}%
\pgfsys@useobject{currentmarker}{}%
\end{pgfscope}%
\begin{pgfscope}%
\pgfsys@transformshift{1.363658in}{1.571452in}%
\pgfsys@useobject{currentmarker}{}%
\end{pgfscope}%
\begin{pgfscope}%
\pgfsys@transformshift{1.566350in}{1.544093in}%
\pgfsys@useobject{currentmarker}{}%
\end{pgfscope}%
\begin{pgfscope}%
\pgfsys@transformshift{0.737168in}{1.544650in}%
\pgfsys@useobject{currentmarker}{}%
\end{pgfscope}%
\begin{pgfscope}%
\pgfsys@transformshift{0.416424in}{1.318866in}%
\pgfsys@useobject{currentmarker}{}%
\end{pgfscope}%
\begin{pgfscope}%
\pgfsys@transformshift{0.524334in}{1.576678in}%
\pgfsys@useobject{currentmarker}{}%
\end{pgfscope}%
\begin{pgfscope}%
\pgfsys@transformshift{0.739043in}{1.328970in}%
\pgfsys@useobject{currentmarker}{}%
\end{pgfscope}%
\begin{pgfscope}%
\pgfsys@transformshift{1.840100in}{1.350651in}%
\pgfsys@useobject{currentmarker}{}%
\end{pgfscope}%
\begin{pgfscope}%
\pgfsys@transformshift{1.824568in}{1.393679in}%
\pgfsys@useobject{currentmarker}{}%
\end{pgfscope}%
\begin{pgfscope}%
\pgfsys@transformshift{1.470009in}{1.546355in}%
\pgfsys@useobject{currentmarker}{}%
\end{pgfscope}%
\begin{pgfscope}%
\pgfsys@transformshift{0.471982in}{1.583247in}%
\pgfsys@useobject{currentmarker}{}%
\end{pgfscope}%
\begin{pgfscope}%
\pgfsys@transformshift{1.573451in}{1.733607in}%
\pgfsys@useobject{currentmarker}{}%
\end{pgfscope}%
\begin{pgfscope}%
\pgfsys@transformshift{1.738247in}{1.347984in}%
\pgfsys@useobject{currentmarker}{}%
\end{pgfscope}%
\begin{pgfscope}%
\pgfsys@transformshift{1.607428in}{1.325642in}%
\pgfsys@useobject{currentmarker}{}%
\end{pgfscope}%
\begin{pgfscope}%
\pgfsys@transformshift{1.759927in}{1.173501in}%
\pgfsys@useobject{currentmarker}{}%
\end{pgfscope}%
\begin{pgfscope}%
\pgfsys@transformshift{1.551624in}{1.379730in}%
\pgfsys@useobject{currentmarker}{}%
\end{pgfscope}%
\begin{pgfscope}%
\pgfsys@transformshift{1.356607in}{1.540110in}%
\pgfsys@useobject{currentmarker}{}%
\end{pgfscope}%
\begin{pgfscope}%
\pgfsys@transformshift{1.649686in}{1.123584in}%
\pgfsys@useobject{currentmarker}{}%
\end{pgfscope}%
\begin{pgfscope}%
\pgfsys@transformshift{1.305691in}{1.428819in}%
\pgfsys@useobject{currentmarker}{}%
\end{pgfscope}%
\begin{pgfscope}%
\pgfsys@transformshift{1.393587in}{1.332452in}%
\pgfsys@useobject{currentmarker}{}%
\end{pgfscope}%
\begin{pgfscope}%
\pgfsys@transformshift{1.153074in}{1.382847in}%
\pgfsys@useobject{currentmarker}{}%
\end{pgfscope}%
\begin{pgfscope}%
\pgfsys@transformshift{1.150217in}{1.575952in}%
\pgfsys@useobject{currentmarker}{}%
\end{pgfscope}%
\begin{pgfscope}%
\pgfsys@transformshift{0.541563in}{1.182647in}%
\pgfsys@useobject{currentmarker}{}%
\end{pgfscope}%
\begin{pgfscope}%
\pgfsys@transformshift{0.907947in}{1.578240in}%
\pgfsys@useobject{currentmarker}{}%
\end{pgfscope}%
\begin{pgfscope}%
\pgfsys@transformshift{1.404883in}{1.487207in}%
\pgfsys@useobject{currentmarker}{}%
\end{pgfscope}%
\begin{pgfscope}%
\pgfsys@transformshift{1.766444in}{1.321540in}%
\pgfsys@useobject{currentmarker}{}%
\end{pgfscope}%
\begin{pgfscope}%
\pgfsys@transformshift{1.727091in}{1.473987in}%
\pgfsys@useobject{currentmarker}{}%
\end{pgfscope}%
\begin{pgfscope}%
\pgfsys@transformshift{1.794191in}{1.479016in}%
\pgfsys@useobject{currentmarker}{}%
\end{pgfscope}%
\begin{pgfscope}%
\pgfsys@transformshift{1.006287in}{1.576334in}%
\pgfsys@useobject{currentmarker}{}%
\end{pgfscope}%
\begin{pgfscope}%
\pgfsys@transformshift{1.581569in}{1.430552in}%
\pgfsys@useobject{currentmarker}{}%
\end{pgfscope}%
\begin{pgfscope}%
\pgfsys@transformshift{0.941329in}{1.557151in}%
\pgfsys@useobject{currentmarker}{}%
\end{pgfscope}%
\begin{pgfscope}%
\pgfsys@transformshift{0.691952in}{1.670783in}%
\pgfsys@useobject{currentmarker}{}%
\end{pgfscope}%
\begin{pgfscope}%
\pgfsys@transformshift{1.057616in}{1.667000in}%
\pgfsys@useobject{currentmarker}{}%
\end{pgfscope}%
\begin{pgfscope}%
\pgfsys@transformshift{1.673517in}{1.492277in}%
\pgfsys@useobject{currentmarker}{}%
\end{pgfscope}%
\begin{pgfscope}%
\pgfsys@transformshift{1.477276in}{1.571833in}%
\pgfsys@useobject{currentmarker}{}%
\end{pgfscope}%
\begin{pgfscope}%
\pgfsys@transformshift{1.418685in}{1.505915in}%
\pgfsys@useobject{currentmarker}{}%
\end{pgfscope}%
\begin{pgfscope}%
\pgfsys@transformshift{1.590325in}{1.317212in}%
\pgfsys@useobject{currentmarker}{}%
\end{pgfscope}%
\begin{pgfscope}%
\pgfsys@transformshift{0.779671in}{1.626132in}%
\pgfsys@useobject{currentmarker}{}%
\end{pgfscope}%
\begin{pgfscope}%
\pgfsys@transformshift{1.674297in}{1.346926in}%
\pgfsys@useobject{currentmarker}{}%
\end{pgfscope}%
\begin{pgfscope}%
\pgfsys@transformshift{0.676244in}{1.592839in}%
\pgfsys@useobject{currentmarker}{}%
\end{pgfscope}%
\begin{pgfscope}%
\pgfsys@transformshift{1.715152in}{1.276970in}%
\pgfsys@useobject{currentmarker}{}%
\end{pgfscope}%
\begin{pgfscope}%
\pgfsys@transformshift{0.491079in}{1.307690in}%
\pgfsys@useobject{currentmarker}{}%
\end{pgfscope}%
\begin{pgfscope}%
\pgfsys@transformshift{1.084356in}{1.667983in}%
\pgfsys@useobject{currentmarker}{}%
\end{pgfscope}%
\begin{pgfscope}%
\pgfsys@transformshift{0.525316in}{1.309273in}%
\pgfsys@useobject{currentmarker}{}%
\end{pgfscope}%
\begin{pgfscope}%
\pgfsys@transformshift{0.530939in}{1.324880in}%
\pgfsys@useobject{currentmarker}{}%
\end{pgfscope}%
\begin{pgfscope}%
\pgfsys@transformshift{0.632029in}{1.751908in}%
\pgfsys@useobject{currentmarker}{}%
\end{pgfscope}%
\begin{pgfscope}%
\pgfsys@transformshift{0.952115in}{1.693863in}%
\pgfsys@useobject{currentmarker}{}%
\end{pgfscope}%
\begin{pgfscope}%
\pgfsys@transformshift{1.781361in}{1.173113in}%
\pgfsys@useobject{currentmarker}{}%
\end{pgfscope}%
\begin{pgfscope}%
\pgfsys@transformshift{0.893435in}{1.483645in}%
\pgfsys@useobject{currentmarker}{}%
\end{pgfscope}%
\begin{pgfscope}%
\pgfsys@transformshift{1.630080in}{1.358351in}%
\pgfsys@useobject{currentmarker}{}%
\end{pgfscope}%
\begin{pgfscope}%
\pgfsys@transformshift{1.495001in}{1.483327in}%
\pgfsys@useobject{currentmarker}{}%
\end{pgfscope}%
\begin{pgfscope}%
\pgfsys@transformshift{1.807197in}{1.450815in}%
\pgfsys@useobject{currentmarker}{}%
\end{pgfscope}%
\begin{pgfscope}%
\pgfsys@transformshift{0.872166in}{1.537107in}%
\pgfsys@useobject{currentmarker}{}%
\end{pgfscope}%
\begin{pgfscope}%
\pgfsys@transformshift{1.447090in}{1.616219in}%
\pgfsys@useobject{currentmarker}{}%
\end{pgfscope}%
\begin{pgfscope}%
\pgfsys@transformshift{1.534740in}{1.476670in}%
\pgfsys@useobject{currentmarker}{}%
\end{pgfscope}%
\begin{pgfscope}%
\pgfsys@transformshift{0.648228in}{1.832216in}%
\pgfsys@useobject{currentmarker}{}%
\end{pgfscope}%
\begin{pgfscope}%
\pgfsys@transformshift{0.939750in}{1.317544in}%
\pgfsys@useobject{currentmarker}{}%
\end{pgfscope}%
\begin{pgfscope}%
\pgfsys@transformshift{0.437103in}{1.351928in}%
\pgfsys@useobject{currentmarker}{}%
\end{pgfscope}%
\begin{pgfscope}%
\pgfsys@transformshift{1.534965in}{1.494684in}%
\pgfsys@useobject{currentmarker}{}%
\end{pgfscope}%
\begin{pgfscope}%
\pgfsys@transformshift{1.227246in}{1.619967in}%
\pgfsys@useobject{currentmarker}{}%
\end{pgfscope}%
\begin{pgfscope}%
\pgfsys@transformshift{1.816498in}{1.277633in}%
\pgfsys@useobject{currentmarker}{}%
\end{pgfscope}%
\begin{pgfscope}%
\pgfsys@transformshift{0.733769in}{1.649407in}%
\pgfsys@useobject{currentmarker}{}%
\end{pgfscope}%
\begin{pgfscope}%
\pgfsys@transformshift{0.664484in}{1.511367in}%
\pgfsys@useobject{currentmarker}{}%
\end{pgfscope}%
\begin{pgfscope}%
\pgfsys@transformshift{1.076644in}{1.635868in}%
\pgfsys@useobject{currentmarker}{}%
\end{pgfscope}%
\begin{pgfscope}%
\pgfsys@transformshift{1.548521in}{1.555511in}%
\pgfsys@useobject{currentmarker}{}%
\end{pgfscope}%
\begin{pgfscope}%
\pgfsys@transformshift{0.536243in}{1.679306in}%
\pgfsys@useobject{currentmarker}{}%
\end{pgfscope}%
\begin{pgfscope}%
\pgfsys@transformshift{0.889736in}{1.612599in}%
\pgfsys@useobject{currentmarker}{}%
\end{pgfscope}%
\begin{pgfscope}%
\pgfsys@transformshift{1.818644in}{1.311585in}%
\pgfsys@useobject{currentmarker}{}%
\end{pgfscope}%
\begin{pgfscope}%
\pgfsys@transformshift{1.073205in}{1.657939in}%
\pgfsys@useobject{currentmarker}{}%
\end{pgfscope}%
\begin{pgfscope}%
\pgfsys@transformshift{0.882269in}{1.677255in}%
\pgfsys@useobject{currentmarker}{}%
\end{pgfscope}%
\begin{pgfscope}%
\pgfsys@transformshift{0.755753in}{1.645083in}%
\pgfsys@useobject{currentmarker}{}%
\end{pgfscope}%
\begin{pgfscope}%
\pgfsys@transformshift{1.318495in}{1.537911in}%
\pgfsys@useobject{currentmarker}{}%
\end{pgfscope}%
\begin{pgfscope}%
\pgfsys@transformshift{1.102293in}{1.488693in}%
\pgfsys@useobject{currentmarker}{}%
\end{pgfscope}%
\begin{pgfscope}%
\pgfsys@transformshift{1.814698in}{1.537800in}%
\pgfsys@useobject{currentmarker}{}%
\end{pgfscope}%
\begin{pgfscope}%
\pgfsys@transformshift{1.595111in}{1.348078in}%
\pgfsys@useobject{currentmarker}{}%
\end{pgfscope}%
\begin{pgfscope}%
\pgfsys@transformshift{0.726840in}{1.520989in}%
\pgfsys@useobject{currentmarker}{}%
\end{pgfscope}%
\begin{pgfscope}%
\pgfsys@transformshift{1.492602in}{1.425892in}%
\pgfsys@useobject{currentmarker}{}%
\end{pgfscope}%
\begin{pgfscope}%
\pgfsys@transformshift{1.393179in}{1.322157in}%
\pgfsys@useobject{currentmarker}{}%
\end{pgfscope}%
\begin{pgfscope}%
\pgfsys@transformshift{1.166943in}{1.297711in}%
\pgfsys@useobject{currentmarker}{}%
\end{pgfscope}%
\begin{pgfscope}%
\pgfsys@transformshift{1.444452in}{1.320235in}%
\pgfsys@useobject{currentmarker}{}%
\end{pgfscope}%
\begin{pgfscope}%
\pgfsys@transformshift{1.693005in}{1.488964in}%
\pgfsys@useobject{currentmarker}{}%
\end{pgfscope}%
\begin{pgfscope}%
\pgfsys@transformshift{0.817674in}{1.395519in}%
\pgfsys@useobject{currentmarker}{}%
\end{pgfscope}%
\begin{pgfscope}%
\pgfsys@transformshift{0.521183in}{1.728531in}%
\pgfsys@useobject{currentmarker}{}%
\end{pgfscope}%
\begin{pgfscope}%
\pgfsys@transformshift{1.298494in}{1.608031in}%
\pgfsys@useobject{currentmarker}{}%
\end{pgfscope}%
\begin{pgfscope}%
\pgfsys@transformshift{0.537740in}{1.626422in}%
\pgfsys@useobject{currentmarker}{}%
\end{pgfscope}%
\begin{pgfscope}%
\pgfsys@transformshift{1.711514in}{1.592620in}%
\pgfsys@useobject{currentmarker}{}%
\end{pgfscope}%
\begin{pgfscope}%
\pgfsys@transformshift{1.607575in}{1.535536in}%
\pgfsys@useobject{currentmarker}{}%
\end{pgfscope}%
\begin{pgfscope}%
\pgfsys@transformshift{1.569340in}{1.498031in}%
\pgfsys@useobject{currentmarker}{}%
\end{pgfscope}%
\begin{pgfscope}%
\pgfsys@transformshift{0.423151in}{1.648938in}%
\pgfsys@useobject{currentmarker}{}%
\end{pgfscope}%
\begin{pgfscope}%
\pgfsys@transformshift{0.636198in}{1.195227in}%
\pgfsys@useobject{currentmarker}{}%
\end{pgfscope}%
\begin{pgfscope}%
\pgfsys@transformshift{1.742823in}{1.264753in}%
\pgfsys@useobject{currentmarker}{}%
\end{pgfscope}%
\begin{pgfscope}%
\pgfsys@transformshift{0.849878in}{1.515806in}%
\pgfsys@useobject{currentmarker}{}%
\end{pgfscope}%
\begin{pgfscope}%
\pgfsys@transformshift{1.353468in}{1.559808in}%
\pgfsys@useobject{currentmarker}{}%
\end{pgfscope}%
\begin{pgfscope}%
\pgfsys@transformshift{1.622982in}{1.071482in}%
\pgfsys@useobject{currentmarker}{}%
\end{pgfscope}%
\begin{pgfscope}%
\pgfsys@transformshift{1.599783in}{1.168543in}%
\pgfsys@useobject{currentmarker}{}%
\end{pgfscope}%
\begin{pgfscope}%
\pgfsys@transformshift{1.895187in}{1.235088in}%
\pgfsys@useobject{currentmarker}{}%
\end{pgfscope}%
\begin{pgfscope}%
\pgfsys@transformshift{1.129530in}{1.591859in}%
\pgfsys@useobject{currentmarker}{}%
\end{pgfscope}%
\begin{pgfscope}%
\pgfsys@transformshift{0.991382in}{1.497394in}%
\pgfsys@useobject{currentmarker}{}%
\end{pgfscope}%
\begin{pgfscope}%
\pgfsys@transformshift{0.795664in}{1.820008in}%
\pgfsys@useobject{currentmarker}{}%
\end{pgfscope}%
\begin{pgfscope}%
\pgfsys@transformshift{1.780025in}{1.438491in}%
\pgfsys@useobject{currentmarker}{}%
\end{pgfscope}%
\begin{pgfscope}%
\pgfsys@transformshift{1.712749in}{1.428494in}%
\pgfsys@useobject{currentmarker}{}%
\end{pgfscope}%
\begin{pgfscope}%
\pgfsys@transformshift{1.145211in}{1.620266in}%
\pgfsys@useobject{currentmarker}{}%
\end{pgfscope}%
\begin{pgfscope}%
\pgfsys@transformshift{1.328349in}{1.414532in}%
\pgfsys@useobject{currentmarker}{}%
\end{pgfscope}%
\begin{pgfscope}%
\pgfsys@transformshift{1.405073in}{1.478224in}%
\pgfsys@useobject{currentmarker}{}%
\end{pgfscope}%
\begin{pgfscope}%
\pgfsys@transformshift{0.669867in}{1.661528in}%
\pgfsys@useobject{currentmarker}{}%
\end{pgfscope}%
\begin{pgfscope}%
\pgfsys@transformshift{1.898744in}{1.202589in}%
\pgfsys@useobject{currentmarker}{}%
\end{pgfscope}%
\begin{pgfscope}%
\pgfsys@transformshift{0.874940in}{1.552929in}%
\pgfsys@useobject{currentmarker}{}%
\end{pgfscope}%
\begin{pgfscope}%
\pgfsys@transformshift{1.459937in}{1.757535in}%
\pgfsys@useobject{currentmarker}{}%
\end{pgfscope}%
\begin{pgfscope}%
\pgfsys@transformshift{1.895928in}{1.214669in}%
\pgfsys@useobject{currentmarker}{}%
\end{pgfscope}%
\begin{pgfscope}%
\pgfsys@transformshift{0.853418in}{1.519203in}%
\pgfsys@useobject{currentmarker}{}%
\end{pgfscope}%
\begin{pgfscope}%
\pgfsys@transformshift{0.717208in}{1.597269in}%
\pgfsys@useobject{currentmarker}{}%
\end{pgfscope}%
\begin{pgfscope}%
\pgfsys@transformshift{1.075298in}{1.533234in}%
\pgfsys@useobject{currentmarker}{}%
\end{pgfscope}%
\begin{pgfscope}%
\pgfsys@transformshift{1.920218in}{1.452531in}%
\pgfsys@useobject{currentmarker}{}%
\end{pgfscope}%
\begin{pgfscope}%
\pgfsys@transformshift{0.377829in}{1.620768in}%
\pgfsys@useobject{currentmarker}{}%
\end{pgfscope}%
\begin{pgfscope}%
\pgfsys@transformshift{1.155745in}{1.670298in}%
\pgfsys@useobject{currentmarker}{}%
\end{pgfscope}%
\begin{pgfscope}%
\pgfsys@transformshift{1.712524in}{1.263974in}%
\pgfsys@useobject{currentmarker}{}%
\end{pgfscope}%
\begin{pgfscope}%
\pgfsys@transformshift{0.534596in}{1.652967in}%
\pgfsys@useobject{currentmarker}{}%
\end{pgfscope}%
\begin{pgfscope}%
\pgfsys@transformshift{0.622741in}{1.729369in}%
\pgfsys@useobject{currentmarker}{}%
\end{pgfscope}%
\begin{pgfscope}%
\pgfsys@transformshift{0.575527in}{1.777388in}%
\pgfsys@useobject{currentmarker}{}%
\end{pgfscope}%
\begin{pgfscope}%
\pgfsys@transformshift{1.804863in}{1.253584in}%
\pgfsys@useobject{currentmarker}{}%
\end{pgfscope}%
\begin{pgfscope}%
\pgfsys@transformshift{1.749745in}{1.155424in}%
\pgfsys@useobject{currentmarker}{}%
\end{pgfscope}%
\begin{pgfscope}%
\pgfsys@transformshift{1.674396in}{1.426241in}%
\pgfsys@useobject{currentmarker}{}%
\end{pgfscope}%
\begin{pgfscope}%
\pgfsys@transformshift{1.859027in}{1.171602in}%
\pgfsys@useobject{currentmarker}{}%
\end{pgfscope}%
\begin{pgfscope}%
\pgfsys@transformshift{1.603827in}{1.273680in}%
\pgfsys@useobject{currentmarker}{}%
\end{pgfscope}%
\begin{pgfscope}%
\pgfsys@transformshift{1.255917in}{1.590281in}%
\pgfsys@useobject{currentmarker}{}%
\end{pgfscope}%
\begin{pgfscope}%
\pgfsys@transformshift{0.419746in}{1.325141in}%
\pgfsys@useobject{currentmarker}{}%
\end{pgfscope}%
\begin{pgfscope}%
\pgfsys@transformshift{0.903589in}{1.297537in}%
\pgfsys@useobject{currentmarker}{}%
\end{pgfscope}%
\begin{pgfscope}%
\pgfsys@transformshift{1.411837in}{1.549918in}%
\pgfsys@useobject{currentmarker}{}%
\end{pgfscope}%
\begin{pgfscope}%
\pgfsys@transformshift{1.888179in}{1.209072in}%
\pgfsys@useobject{currentmarker}{}%
\end{pgfscope}%
\begin{pgfscope}%
\pgfsys@transformshift{1.463948in}{1.458464in}%
\pgfsys@useobject{currentmarker}{}%
\end{pgfscope}%
\begin{pgfscope}%
\pgfsys@transformshift{1.037147in}{1.479446in}%
\pgfsys@useobject{currentmarker}{}%
\end{pgfscope}%
\begin{pgfscope}%
\pgfsys@transformshift{1.744144in}{1.262712in}%
\pgfsys@useobject{currentmarker}{}%
\end{pgfscope}%
\begin{pgfscope}%
\pgfsys@transformshift{0.867355in}{1.464033in}%
\pgfsys@useobject{currentmarker}{}%
\end{pgfscope}%
\begin{pgfscope}%
\pgfsys@transformshift{1.331798in}{1.598254in}%
\pgfsys@useobject{currentmarker}{}%
\end{pgfscope}%
\begin{pgfscope}%
\pgfsys@transformshift{0.968745in}{1.592420in}%
\pgfsys@useobject{currentmarker}{}%
\end{pgfscope}%
\begin{pgfscope}%
\pgfsys@transformshift{1.193049in}{1.626323in}%
\pgfsys@useobject{currentmarker}{}%
\end{pgfscope}%
\begin{pgfscope}%
\pgfsys@transformshift{1.664246in}{1.297183in}%
\pgfsys@useobject{currentmarker}{}%
\end{pgfscope}%
\begin{pgfscope}%
\pgfsys@transformshift{0.835587in}{1.595983in}%
\pgfsys@useobject{currentmarker}{}%
\end{pgfscope}%
\begin{pgfscope}%
\pgfsys@transformshift{1.701974in}{1.214719in}%
\pgfsys@useobject{currentmarker}{}%
\end{pgfscope}%
\begin{pgfscope}%
\pgfsys@transformshift{1.269033in}{1.535186in}%
\pgfsys@useobject{currentmarker}{}%
\end{pgfscope}%
\begin{pgfscope}%
\pgfsys@transformshift{1.329352in}{1.444552in}%
\pgfsys@useobject{currentmarker}{}%
\end{pgfscope}%
\begin{pgfscope}%
\pgfsys@transformshift{1.471661in}{1.578355in}%
\pgfsys@useobject{currentmarker}{}%
\end{pgfscope}%
\begin{pgfscope}%
\pgfsys@transformshift{1.916157in}{1.457010in}%
\pgfsys@useobject{currentmarker}{}%
\end{pgfscope}%
\begin{pgfscope}%
\pgfsys@transformshift{1.533379in}{1.286299in}%
\pgfsys@useobject{currentmarker}{}%
\end{pgfscope}%
\begin{pgfscope}%
\pgfsys@transformshift{1.358583in}{1.434037in}%
\pgfsys@useobject{currentmarker}{}%
\end{pgfscope}%
\begin{pgfscope}%
\pgfsys@transformshift{1.599245in}{1.362133in}%
\pgfsys@useobject{currentmarker}{}%
\end{pgfscope}%
\begin{pgfscope}%
\pgfsys@transformshift{0.703172in}{1.600380in}%
\pgfsys@useobject{currentmarker}{}%
\end{pgfscope}%
\begin{pgfscope}%
\pgfsys@transformshift{1.320412in}{1.533543in}%
\pgfsys@useobject{currentmarker}{}%
\end{pgfscope}%
\begin{pgfscope}%
\pgfsys@transformshift{1.136406in}{1.517529in}%
\pgfsys@useobject{currentmarker}{}%
\end{pgfscope}%
\begin{pgfscope}%
\pgfsys@transformshift{0.721527in}{1.667600in}%
\pgfsys@useobject{currentmarker}{}%
\end{pgfscope}%
\begin{pgfscope}%
\pgfsys@transformshift{1.722543in}{1.161262in}%
\pgfsys@useobject{currentmarker}{}%
\end{pgfscope}%
\begin{pgfscope}%
\pgfsys@transformshift{1.835288in}{1.395915in}%
\pgfsys@useobject{currentmarker}{}%
\end{pgfscope}%
\begin{pgfscope}%
\pgfsys@transformshift{1.984248in}{1.414540in}%
\pgfsys@useobject{currentmarker}{}%
\end{pgfscope}%
\begin{pgfscope}%
\pgfsys@transformshift{1.658536in}{1.471135in}%
\pgfsys@useobject{currentmarker}{}%
\end{pgfscope}%
\begin{pgfscope}%
\pgfsys@transformshift{0.561607in}{1.656013in}%
\pgfsys@useobject{currentmarker}{}%
\end{pgfscope}%
\begin{pgfscope}%
\pgfsys@transformshift{1.906539in}{1.375862in}%
\pgfsys@useobject{currentmarker}{}%
\end{pgfscope}%
\begin{pgfscope}%
\pgfsys@transformshift{1.521333in}{1.482607in}%
\pgfsys@useobject{currentmarker}{}%
\end{pgfscope}%
\begin{pgfscope}%
\pgfsys@transformshift{1.136396in}{1.549229in}%
\pgfsys@useobject{currentmarker}{}%
\end{pgfscope}%
\begin{pgfscope}%
\pgfsys@transformshift{1.435818in}{1.305933in}%
\pgfsys@useobject{currentmarker}{}%
\end{pgfscope}%
\begin{pgfscope}%
\pgfsys@transformshift{1.643271in}{1.254269in}%
\pgfsys@useobject{currentmarker}{}%
\end{pgfscope}%
\begin{pgfscope}%
\pgfsys@transformshift{1.445064in}{1.412236in}%
\pgfsys@useobject{currentmarker}{}%
\end{pgfscope}%
\begin{pgfscope}%
\pgfsys@transformshift{0.658007in}{1.484491in}%
\pgfsys@useobject{currentmarker}{}%
\end{pgfscope}%
\begin{pgfscope}%
\pgfsys@transformshift{0.786972in}{1.543972in}%
\pgfsys@useobject{currentmarker}{}%
\end{pgfscope}%
\begin{pgfscope}%
\pgfsys@transformshift{1.102575in}{1.555751in}%
\pgfsys@useobject{currentmarker}{}%
\end{pgfscope}%
\begin{pgfscope}%
\pgfsys@transformshift{0.532598in}{1.670164in}%
\pgfsys@useobject{currentmarker}{}%
\end{pgfscope}%
\begin{pgfscope}%
\pgfsys@transformshift{0.357574in}{1.705316in}%
\pgfsys@useobject{currentmarker}{}%
\end{pgfscope}%
\begin{pgfscope}%
\pgfsys@transformshift{1.683669in}{1.339211in}%
\pgfsys@useobject{currentmarker}{}%
\end{pgfscope}%
\begin{pgfscope}%
\pgfsys@transformshift{1.612445in}{1.460811in}%
\pgfsys@useobject{currentmarker}{}%
\end{pgfscope}%
\begin{pgfscope}%
\pgfsys@transformshift{1.620066in}{1.308582in}%
\pgfsys@useobject{currentmarker}{}%
\end{pgfscope}%
\begin{pgfscope}%
\pgfsys@transformshift{0.680343in}{1.346365in}%
\pgfsys@useobject{currentmarker}{}%
\end{pgfscope}%
\begin{pgfscope}%
\pgfsys@transformshift{0.675258in}{1.632518in}%
\pgfsys@useobject{currentmarker}{}%
\end{pgfscope}%
\begin{pgfscope}%
\pgfsys@transformshift{0.963927in}{1.503364in}%
\pgfsys@useobject{currentmarker}{}%
\end{pgfscope}%
\begin{pgfscope}%
\pgfsys@transformshift{1.684562in}{1.282793in}%
\pgfsys@useobject{currentmarker}{}%
\end{pgfscope}%
\begin{pgfscope}%
\pgfsys@transformshift{1.473245in}{1.565901in}%
\pgfsys@useobject{currentmarker}{}%
\end{pgfscope}%
\begin{pgfscope}%
\pgfsys@transformshift{1.020383in}{1.671507in}%
\pgfsys@useobject{currentmarker}{}%
\end{pgfscope}%
\begin{pgfscope}%
\pgfsys@transformshift{0.681597in}{1.750266in}%
\pgfsys@useobject{currentmarker}{}%
\end{pgfscope}%
\begin{pgfscope}%
\pgfsys@transformshift{1.730115in}{1.305743in}%
\pgfsys@useobject{currentmarker}{}%
\end{pgfscope}%
\begin{pgfscope}%
\pgfsys@transformshift{1.808933in}{1.295675in}%
\pgfsys@useobject{currentmarker}{}%
\end{pgfscope}%
\begin{pgfscope}%
\pgfsys@transformshift{1.251675in}{1.575224in}%
\pgfsys@useobject{currentmarker}{}%
\end{pgfscope}%
\begin{pgfscope}%
\pgfsys@transformshift{1.717946in}{1.128849in}%
\pgfsys@useobject{currentmarker}{}%
\end{pgfscope}%
\begin{pgfscope}%
\pgfsys@transformshift{1.597312in}{1.284162in}%
\pgfsys@useobject{currentmarker}{}%
\end{pgfscope}%
\begin{pgfscope}%
\pgfsys@transformshift{1.148136in}{1.629186in}%
\pgfsys@useobject{currentmarker}{}%
\end{pgfscope}%
\begin{pgfscope}%
\pgfsys@transformshift{0.720143in}{1.648337in}%
\pgfsys@useobject{currentmarker}{}%
\end{pgfscope}%
\begin{pgfscope}%
\pgfsys@transformshift{1.574970in}{1.526770in}%
\pgfsys@useobject{currentmarker}{}%
\end{pgfscope}%
\begin{pgfscope}%
\pgfsys@transformshift{0.869972in}{1.448404in}%
\pgfsys@useobject{currentmarker}{}%
\end{pgfscope}%
\begin{pgfscope}%
\pgfsys@transformshift{1.713193in}{1.314515in}%
\pgfsys@useobject{currentmarker}{}%
\end{pgfscope}%
\begin{pgfscope}%
\pgfsys@transformshift{1.673692in}{1.244374in}%
\pgfsys@useobject{currentmarker}{}%
\end{pgfscope}%
\begin{pgfscope}%
\pgfsys@transformshift{1.344861in}{1.416219in}%
\pgfsys@useobject{currentmarker}{}%
\end{pgfscope}%
\begin{pgfscope}%
\pgfsys@transformshift{1.182101in}{1.545297in}%
\pgfsys@useobject{currentmarker}{}%
\end{pgfscope}%
\begin{pgfscope}%
\pgfsys@transformshift{1.449668in}{1.483830in}%
\pgfsys@useobject{currentmarker}{}%
\end{pgfscope}%
\begin{pgfscope}%
\pgfsys@transformshift{1.561680in}{1.443812in}%
\pgfsys@useobject{currentmarker}{}%
\end{pgfscope}%
\begin{pgfscope}%
\pgfsys@transformshift{1.192063in}{1.465298in}%
\pgfsys@useobject{currentmarker}{}%
\end{pgfscope}%
\begin{pgfscope}%
\pgfsys@transformshift{1.264374in}{1.588618in}%
\pgfsys@useobject{currentmarker}{}%
\end{pgfscope}%
\begin{pgfscope}%
\pgfsys@transformshift{0.727375in}{1.712847in}%
\pgfsys@useobject{currentmarker}{}%
\end{pgfscope}%
\begin{pgfscope}%
\pgfsys@transformshift{0.687912in}{1.227498in}%
\pgfsys@useobject{currentmarker}{}%
\end{pgfscope}%
\begin{pgfscope}%
\pgfsys@transformshift{1.823596in}{1.434464in}%
\pgfsys@useobject{currentmarker}{}%
\end{pgfscope}%
\begin{pgfscope}%
\pgfsys@transformshift{1.420153in}{1.422629in}%
\pgfsys@useobject{currentmarker}{}%
\end{pgfscope}%
\begin{pgfscope}%
\pgfsys@transformshift{0.672265in}{1.746074in}%
\pgfsys@useobject{currentmarker}{}%
\end{pgfscope}%
\begin{pgfscope}%
\pgfsys@transformshift{0.574267in}{1.668087in}%
\pgfsys@useobject{currentmarker}{}%
\end{pgfscope}%
\begin{pgfscope}%
\pgfsys@transformshift{1.570297in}{1.385945in}%
\pgfsys@useobject{currentmarker}{}%
\end{pgfscope}%
\begin{pgfscope}%
\pgfsys@transformshift{1.650981in}{1.244300in}%
\pgfsys@useobject{currentmarker}{}%
\end{pgfscope}%
\begin{pgfscope}%
\pgfsys@transformshift{0.894282in}{1.511529in}%
\pgfsys@useobject{currentmarker}{}%
\end{pgfscope}%
\begin{pgfscope}%
\pgfsys@transformshift{1.157746in}{1.569106in}%
\pgfsys@useobject{currentmarker}{}%
\end{pgfscope}%
\begin{pgfscope}%
\pgfsys@transformshift{0.608353in}{1.718167in}%
\pgfsys@useobject{currentmarker}{}%
\end{pgfscope}%
\begin{pgfscope}%
\pgfsys@transformshift{1.631665in}{1.272291in}%
\pgfsys@useobject{currentmarker}{}%
\end{pgfscope}%
\begin{pgfscope}%
\pgfsys@transformshift{1.419963in}{1.399698in}%
\pgfsys@useobject{currentmarker}{}%
\end{pgfscope}%
\begin{pgfscope}%
\pgfsys@transformshift{1.180003in}{1.593969in}%
\pgfsys@useobject{currentmarker}{}%
\end{pgfscope}%
\begin{pgfscope}%
\pgfsys@transformshift{1.469789in}{1.576147in}%
\pgfsys@useobject{currentmarker}{}%
\end{pgfscope}%
\begin{pgfscope}%
\pgfsys@transformshift{1.493240in}{1.622142in}%
\pgfsys@useobject{currentmarker}{}%
\end{pgfscope}%
\begin{pgfscope}%
\pgfsys@transformshift{1.725815in}{1.303890in}%
\pgfsys@useobject{currentmarker}{}%
\end{pgfscope}%
\begin{pgfscope}%
\pgfsys@transformshift{1.002068in}{1.508121in}%
\pgfsys@useobject{currentmarker}{}%
\end{pgfscope}%
\begin{pgfscope}%
\pgfsys@transformshift{1.111774in}{1.612832in}%
\pgfsys@useobject{currentmarker}{}%
\end{pgfscope}%
\begin{pgfscope}%
\pgfsys@transformshift{0.361781in}{1.774126in}%
\pgfsys@useobject{currentmarker}{}%
\end{pgfscope}%
\begin{pgfscope}%
\pgfsys@transformshift{1.608152in}{1.424221in}%
\pgfsys@useobject{currentmarker}{}%
\end{pgfscope}%
\begin{pgfscope}%
\pgfsys@transformshift{1.468569in}{1.632723in}%
\pgfsys@useobject{currentmarker}{}%
\end{pgfscope}%
\begin{pgfscope}%
\pgfsys@transformshift{1.646778in}{1.258959in}%
\pgfsys@useobject{currentmarker}{}%
\end{pgfscope}%
\begin{pgfscope}%
\pgfsys@transformshift{1.356664in}{1.557455in}%
\pgfsys@useobject{currentmarker}{}%
\end{pgfscope}%
\begin{pgfscope}%
\pgfsys@transformshift{0.657110in}{1.285053in}%
\pgfsys@useobject{currentmarker}{}%
\end{pgfscope}%
\begin{pgfscope}%
\pgfsys@transformshift{1.700456in}{1.216999in}%
\pgfsys@useobject{currentmarker}{}%
\end{pgfscope}%
\begin{pgfscope}%
\pgfsys@transformshift{0.741052in}{1.591259in}%
\pgfsys@useobject{currentmarker}{}%
\end{pgfscope}%
\begin{pgfscope}%
\pgfsys@transformshift{0.576055in}{1.933366in}%
\pgfsys@useobject{currentmarker}{}%
\end{pgfscope}%
\begin{pgfscope}%
\pgfsys@transformshift{1.509849in}{1.493758in}%
\pgfsys@useobject{currentmarker}{}%
\end{pgfscope}%
\begin{pgfscope}%
\pgfsys@transformshift{1.838065in}{1.326054in}%
\pgfsys@useobject{currentmarker}{}%
\end{pgfscope}%
\begin{pgfscope}%
\pgfsys@transformshift{1.699823in}{1.521510in}%
\pgfsys@useobject{currentmarker}{}%
\end{pgfscope}%
\begin{pgfscope}%
\pgfsys@transformshift{1.575775in}{1.423280in}%
\pgfsys@useobject{currentmarker}{}%
\end{pgfscope}%
\begin{pgfscope}%
\pgfsys@transformshift{1.000652in}{1.590882in}%
\pgfsys@useobject{currentmarker}{}%
\end{pgfscope}%
\begin{pgfscope}%
\pgfsys@transformshift{1.275329in}{1.583317in}%
\pgfsys@useobject{currentmarker}{}%
\end{pgfscope}%
\begin{pgfscope}%
\pgfsys@transformshift{0.787302in}{1.577204in}%
\pgfsys@useobject{currentmarker}{}%
\end{pgfscope}%
\begin{pgfscope}%
\pgfsys@transformshift{0.941308in}{1.375284in}%
\pgfsys@useobject{currentmarker}{}%
\end{pgfscope}%
\begin{pgfscope}%
\pgfsys@transformshift{1.041866in}{1.393058in}%
\pgfsys@useobject{currentmarker}{}%
\end{pgfscope}%
\begin{pgfscope}%
\pgfsys@transformshift{1.471835in}{1.484323in}%
\pgfsys@useobject{currentmarker}{}%
\end{pgfscope}%
\begin{pgfscope}%
\pgfsys@transformshift{1.225077in}{1.523493in}%
\pgfsys@useobject{currentmarker}{}%
\end{pgfscope}%
\begin{pgfscope}%
\pgfsys@transformshift{1.294631in}{1.344824in}%
\pgfsys@useobject{currentmarker}{}%
\end{pgfscope}%
\begin{pgfscope}%
\pgfsys@transformshift{1.597060in}{1.449726in}%
\pgfsys@useobject{currentmarker}{}%
\end{pgfscope}%
\begin{pgfscope}%
\pgfsys@transformshift{1.136873in}{1.396056in}%
\pgfsys@useobject{currentmarker}{}%
\end{pgfscope}%
\begin{pgfscope}%
\pgfsys@transformshift{1.080421in}{1.496439in}%
\pgfsys@useobject{currentmarker}{}%
\end{pgfscope}%
\begin{pgfscope}%
\pgfsys@transformshift{1.185638in}{1.555825in}%
\pgfsys@useobject{currentmarker}{}%
\end{pgfscope}%
\begin{pgfscope}%
\pgfsys@transformshift{1.407062in}{1.530091in}%
\pgfsys@useobject{currentmarker}{}%
\end{pgfscope}%
\begin{pgfscope}%
\pgfsys@transformshift{1.504034in}{1.586854in}%
\pgfsys@useobject{currentmarker}{}%
\end{pgfscope}%
\begin{pgfscope}%
\pgfsys@transformshift{1.396930in}{1.538350in}%
\pgfsys@useobject{currentmarker}{}%
\end{pgfscope}%
\begin{pgfscope}%
\pgfsys@transformshift{2.000000in}{1.332895in}%
\pgfsys@useobject{currentmarker}{}%
\end{pgfscope}%
\begin{pgfscope}%
\pgfsys@transformshift{1.200619in}{1.421709in}%
\pgfsys@useobject{currentmarker}{}%
\end{pgfscope}%
\begin{pgfscope}%
\pgfsys@transformshift{0.950093in}{1.212466in}%
\pgfsys@useobject{currentmarker}{}%
\end{pgfscope}%
\begin{pgfscope}%
\pgfsys@transformshift{0.632457in}{1.646747in}%
\pgfsys@useobject{currentmarker}{}%
\end{pgfscope}%
\begin{pgfscope}%
\pgfsys@transformshift{0.673561in}{1.514944in}%
\pgfsys@useobject{currentmarker}{}%
\end{pgfscope}%
\begin{pgfscope}%
\pgfsys@transformshift{0.694799in}{1.703062in}%
\pgfsys@useobject{currentmarker}{}%
\end{pgfscope}%
\begin{pgfscope}%
\pgfsys@transformshift{0.518198in}{1.694434in}%
\pgfsys@useobject{currentmarker}{}%
\end{pgfscope}%
\begin{pgfscope}%
\pgfsys@transformshift{1.324442in}{1.478561in}%
\pgfsys@useobject{currentmarker}{}%
\end{pgfscope}%
\begin{pgfscope}%
\pgfsys@transformshift{0.827155in}{1.691080in}%
\pgfsys@useobject{currentmarker}{}%
\end{pgfscope}%
\begin{pgfscope}%
\pgfsys@transformshift{1.894588in}{1.309811in}%
\pgfsys@useobject{currentmarker}{}%
\end{pgfscope}%
\begin{pgfscope}%
\pgfsys@transformshift{0.446321in}{1.587737in}%
\pgfsys@useobject{currentmarker}{}%
\end{pgfscope}%
\begin{pgfscope}%
\pgfsys@transformshift{1.741761in}{1.483870in}%
\pgfsys@useobject{currentmarker}{}%
\end{pgfscope}%
\begin{pgfscope}%
\pgfsys@transformshift{1.307351in}{1.468399in}%
\pgfsys@useobject{currentmarker}{}%
\end{pgfscope}%
\begin{pgfscope}%
\pgfsys@transformshift{1.835875in}{1.355065in}%
\pgfsys@useobject{currentmarker}{}%
\end{pgfscope}%
\begin{pgfscope}%
\pgfsys@transformshift{1.237368in}{1.548716in}%
\pgfsys@useobject{currentmarker}{}%
\end{pgfscope}%
\begin{pgfscope}%
\pgfsys@transformshift{1.875271in}{1.061104in}%
\pgfsys@useobject{currentmarker}{}%
\end{pgfscope}%
\begin{pgfscope}%
\pgfsys@transformshift{1.669995in}{1.338566in}%
\pgfsys@useobject{currentmarker}{}%
\end{pgfscope}%
\begin{pgfscope}%
\pgfsys@transformshift{0.628877in}{1.744669in}%
\pgfsys@useobject{currentmarker}{}%
\end{pgfscope}%
\begin{pgfscope}%
\pgfsys@transformshift{1.644915in}{1.382908in}%
\pgfsys@useobject{currentmarker}{}%
\end{pgfscope}%
\begin{pgfscope}%
\pgfsys@transformshift{1.179959in}{1.465961in}%
\pgfsys@useobject{currentmarker}{}%
\end{pgfscope}%
\begin{pgfscope}%
\pgfsys@transformshift{1.753339in}{1.402159in}%
\pgfsys@useobject{currentmarker}{}%
\end{pgfscope}%
\begin{pgfscope}%
\pgfsys@transformshift{1.477915in}{1.476883in}%
\pgfsys@useobject{currentmarker}{}%
\end{pgfscope}%
\begin{pgfscope}%
\pgfsys@transformshift{1.742609in}{1.452592in}%
\pgfsys@useobject{currentmarker}{}%
\end{pgfscope}%
\begin{pgfscope}%
\pgfsys@transformshift{1.094044in}{1.633551in}%
\pgfsys@useobject{currentmarker}{}%
\end{pgfscope}%
\begin{pgfscope}%
\pgfsys@transformshift{1.645687in}{1.293352in}%
\pgfsys@useobject{currentmarker}{}%
\end{pgfscope}%
\begin{pgfscope}%
\pgfsys@transformshift{0.717249in}{1.521022in}%
\pgfsys@useobject{currentmarker}{}%
\end{pgfscope}%
\begin{pgfscope}%
\pgfsys@transformshift{0.963925in}{1.406064in}%
\pgfsys@useobject{currentmarker}{}%
\end{pgfscope}%
\begin{pgfscope}%
\pgfsys@transformshift{1.342874in}{1.537000in}%
\pgfsys@useobject{currentmarker}{}%
\end{pgfscope}%
\begin{pgfscope}%
\pgfsys@transformshift{1.522619in}{1.429553in}%
\pgfsys@useobject{currentmarker}{}%
\end{pgfscope}%
\begin{pgfscope}%
\pgfsys@transformshift{0.649847in}{1.740663in}%
\pgfsys@useobject{currentmarker}{}%
\end{pgfscope}%
\begin{pgfscope}%
\pgfsys@transformshift{1.826185in}{1.352320in}%
\pgfsys@useobject{currentmarker}{}%
\end{pgfscope}%
\begin{pgfscope}%
\pgfsys@transformshift{1.619806in}{1.399407in}%
\pgfsys@useobject{currentmarker}{}%
\end{pgfscope}%
\begin{pgfscope}%
\pgfsys@transformshift{1.136702in}{1.596659in}%
\pgfsys@useobject{currentmarker}{}%
\end{pgfscope}%
\begin{pgfscope}%
\pgfsys@transformshift{1.894425in}{1.262657in}%
\pgfsys@useobject{currentmarker}{}%
\end{pgfscope}%
\begin{pgfscope}%
\pgfsys@transformshift{0.668971in}{1.739252in}%
\pgfsys@useobject{currentmarker}{}%
\end{pgfscope}%
\end{pgfscope}%
\begin{pgfscope}%
\pgfpathrectangle{\pgfqpoint{0.341129in}{0.466613in}}{\pgfqpoint{1.658871in}{1.711598in}}%
\pgfusepath{clip}%
\pgfsetbuttcap%
\pgfsetroundjoin%
\definecolor{currentfill}{rgb}{0.333333,0.658824,0.407843}%
\pgfsetfillcolor{currentfill}%
\pgfsetfillopacity{0.150000}%
\pgfsetlinewidth{1.003750pt}%
\definecolor{currentstroke}{rgb}{1.000000,1.000000,1.000000}%
\pgfsetstrokecolor{currentstroke}%
\pgfsetstrokeopacity{0.150000}%
\pgfsetdash{}{0pt}%
\pgfsys@defobject{currentmarker}{\pgfqpoint{0.341129in}{1.311214in}}{\pgfqpoint{2.000000in}{1.682003in}}{%
\pgfpathmoveto{\pgfqpoint{0.341129in}{1.682003in}}%
\pgfpathlineto{\pgfqpoint{0.341129in}{1.619110in}}%
\pgfpathlineto{\pgfqpoint{0.357885in}{1.616404in}}%
\pgfpathlineto{\pgfqpoint{0.374641in}{1.613693in}}%
\pgfpathlineto{\pgfqpoint{0.391398in}{1.610989in}}%
\pgfpathlineto{\pgfqpoint{0.408154in}{1.608280in}}%
\pgfpathlineto{\pgfqpoint{0.424910in}{1.605571in}}%
\pgfpathlineto{\pgfqpoint{0.441666in}{1.602864in}}%
\pgfpathlineto{\pgfqpoint{0.458423in}{1.600177in}}%
\pgfpathlineto{\pgfqpoint{0.475179in}{1.597449in}}%
\pgfpathlineto{\pgfqpoint{0.491935in}{1.594738in}}%
\pgfpathlineto{\pgfqpoint{0.508691in}{1.592027in}}%
\pgfpathlineto{\pgfqpoint{0.525448in}{1.589316in}}%
\pgfpathlineto{\pgfqpoint{0.542204in}{1.586563in}}%
\pgfpathlineto{\pgfqpoint{0.558960in}{1.583751in}}%
\pgfpathlineto{\pgfqpoint{0.575717in}{1.581143in}}%
\pgfpathlineto{\pgfqpoint{0.592473in}{1.578495in}}%
\pgfpathlineto{\pgfqpoint{0.609229in}{1.575787in}}%
\pgfpathlineto{\pgfqpoint{0.625985in}{1.573077in}}%
\pgfpathlineto{\pgfqpoint{0.642742in}{1.570366in}}%
\pgfpathlineto{\pgfqpoint{0.659498in}{1.567656in}}%
\pgfpathlineto{\pgfqpoint{0.676254in}{1.564946in}}%
\pgfpathlineto{\pgfqpoint{0.693011in}{1.562236in}}%
\pgfpathlineto{\pgfqpoint{0.709767in}{1.559506in}}%
\pgfpathlineto{\pgfqpoint{0.726523in}{1.556424in}}%
\pgfpathlineto{\pgfqpoint{0.743279in}{1.553754in}}%
\pgfpathlineto{\pgfqpoint{0.760036in}{1.551101in}}%
\pgfpathlineto{\pgfqpoint{0.776792in}{1.548134in}}%
\pgfpathlineto{\pgfqpoint{0.793548in}{1.545406in}}%
\pgfpathlineto{\pgfqpoint{0.810304in}{1.542666in}}%
\pgfpathlineto{\pgfqpoint{0.827061in}{1.539875in}}%
\pgfpathlineto{\pgfqpoint{0.843817in}{1.537091in}}%
\pgfpathlineto{\pgfqpoint{0.860573in}{1.534232in}}%
\pgfpathlineto{\pgfqpoint{0.877330in}{1.531361in}}%
\pgfpathlineto{\pgfqpoint{0.894086in}{1.528361in}}%
\pgfpathlineto{\pgfqpoint{0.910842in}{1.525478in}}%
\pgfpathlineto{\pgfqpoint{0.927598in}{1.522756in}}%
\pgfpathlineto{\pgfqpoint{0.944355in}{1.519893in}}%
\pgfpathlineto{\pgfqpoint{0.961111in}{1.517016in}}%
\pgfpathlineto{\pgfqpoint{0.977867in}{1.514180in}}%
\pgfpathlineto{\pgfqpoint{0.994623in}{1.511319in}}%
\pgfpathlineto{\pgfqpoint{1.011380in}{1.508499in}}%
\pgfpathlineto{\pgfqpoint{1.028136in}{1.505767in}}%
\pgfpathlineto{\pgfqpoint{1.044892in}{1.502966in}}%
\pgfpathlineto{\pgfqpoint{1.061649in}{1.500009in}}%
\pgfpathlineto{\pgfqpoint{1.078405in}{1.497052in}}%
\pgfpathlineto{\pgfqpoint{1.095161in}{1.494134in}}%
\pgfpathlineto{\pgfqpoint{1.111917in}{1.491200in}}%
\pgfpathlineto{\pgfqpoint{1.128674in}{1.488318in}}%
\pgfpathlineto{\pgfqpoint{1.145430in}{1.485432in}}%
\pgfpathlineto{\pgfqpoint{1.162186in}{1.482419in}}%
\pgfpathlineto{\pgfqpoint{1.178942in}{1.479311in}}%
\pgfpathlineto{\pgfqpoint{1.195699in}{1.476054in}}%
\pgfpathlineto{\pgfqpoint{1.212455in}{1.472998in}}%
\pgfpathlineto{\pgfqpoint{1.229211in}{1.469989in}}%
\pgfpathlineto{\pgfqpoint{1.245968in}{1.466980in}}%
\pgfpathlineto{\pgfqpoint{1.262724in}{1.463975in}}%
\pgfpathlineto{\pgfqpoint{1.279480in}{1.461092in}}%
\pgfpathlineto{\pgfqpoint{1.296236in}{1.458009in}}%
\pgfpathlineto{\pgfqpoint{1.312993in}{1.454875in}}%
\pgfpathlineto{\pgfqpoint{1.329749in}{1.451741in}}%
\pgfpathlineto{\pgfqpoint{1.346505in}{1.448613in}}%
\pgfpathlineto{\pgfqpoint{1.363262in}{1.445472in}}%
\pgfpathlineto{\pgfqpoint{1.380018in}{1.442107in}}%
\pgfpathlineto{\pgfqpoint{1.396774in}{1.439038in}}%
\pgfpathlineto{\pgfqpoint{1.413530in}{1.435986in}}%
\pgfpathlineto{\pgfqpoint{1.430287in}{1.432692in}}%
\pgfpathlineto{\pgfqpoint{1.447043in}{1.429276in}}%
\pgfpathlineto{\pgfqpoint{1.463799in}{1.426069in}}%
\pgfpathlineto{\pgfqpoint{1.480555in}{1.422626in}}%
\pgfpathlineto{\pgfqpoint{1.497312in}{1.419074in}}%
\pgfpathlineto{\pgfqpoint{1.514068in}{1.415681in}}%
\pgfpathlineto{\pgfqpoint{1.530824in}{1.411828in}}%
\pgfpathlineto{\pgfqpoint{1.547581in}{1.408228in}}%
\pgfpathlineto{\pgfqpoint{1.564337in}{1.404745in}}%
\pgfpathlineto{\pgfqpoint{1.581093in}{1.401143in}}%
\pgfpathlineto{\pgfqpoint{1.597849in}{1.397629in}}%
\pgfpathlineto{\pgfqpoint{1.614606in}{1.394117in}}%
\pgfpathlineto{\pgfqpoint{1.631362in}{1.390492in}}%
\pgfpathlineto{\pgfqpoint{1.648118in}{1.387087in}}%
\pgfpathlineto{\pgfqpoint{1.664874in}{1.383574in}}%
\pgfpathlineto{\pgfqpoint{1.681631in}{1.379954in}}%
\pgfpathlineto{\pgfqpoint{1.698387in}{1.376215in}}%
\pgfpathlineto{\pgfqpoint{1.715143in}{1.372689in}}%
\pgfpathlineto{\pgfqpoint{1.731900in}{1.369033in}}%
\pgfpathlineto{\pgfqpoint{1.748656in}{1.365558in}}%
\pgfpathlineto{\pgfqpoint{1.765412in}{1.362004in}}%
\pgfpathlineto{\pgfqpoint{1.782168in}{1.358188in}}%
\pgfpathlineto{\pgfqpoint{1.798925in}{1.354351in}}%
\pgfpathlineto{\pgfqpoint{1.815681in}{1.350826in}}%
\pgfpathlineto{\pgfqpoint{1.832437in}{1.347336in}}%
\pgfpathlineto{\pgfqpoint{1.849193in}{1.343819in}}%
\pgfpathlineto{\pgfqpoint{1.865950in}{1.340113in}}%
\pgfpathlineto{\pgfqpoint{1.882706in}{1.336407in}}%
\pgfpathlineto{\pgfqpoint{1.899462in}{1.332722in}}%
\pgfpathlineto{\pgfqpoint{1.916219in}{1.329086in}}%
\pgfpathlineto{\pgfqpoint{1.932975in}{1.325553in}}%
\pgfpathlineto{\pgfqpoint{1.949731in}{1.322021in}}%
\pgfpathlineto{\pgfqpoint{1.966487in}{1.318410in}}%
\pgfpathlineto{\pgfqpoint{1.983244in}{1.314728in}}%
\pgfpathlineto{\pgfqpoint{2.000000in}{1.311214in}}%
\pgfpathlineto{\pgfqpoint{2.000000in}{1.357832in}}%
\pgfpathlineto{\pgfqpoint{2.000000in}{1.357832in}}%
\pgfpathlineto{\pgfqpoint{1.983244in}{1.360434in}}%
\pgfpathlineto{\pgfqpoint{1.966487in}{1.363069in}}%
\pgfpathlineto{\pgfqpoint{1.949731in}{1.365851in}}%
\pgfpathlineto{\pgfqpoint{1.932975in}{1.368795in}}%
\pgfpathlineto{\pgfqpoint{1.916219in}{1.371698in}}%
\pgfpathlineto{\pgfqpoint{1.899462in}{1.374465in}}%
\pgfpathlineto{\pgfqpoint{1.882706in}{1.377228in}}%
\pgfpathlineto{\pgfqpoint{1.865950in}{1.379987in}}%
\pgfpathlineto{\pgfqpoint{1.849193in}{1.382700in}}%
\pgfpathlineto{\pgfqpoint{1.832437in}{1.385328in}}%
\pgfpathlineto{\pgfqpoint{1.815681in}{1.388116in}}%
\pgfpathlineto{\pgfqpoint{1.798925in}{1.390922in}}%
\pgfpathlineto{\pgfqpoint{1.782168in}{1.393727in}}%
\pgfpathlineto{\pgfqpoint{1.765412in}{1.396564in}}%
\pgfpathlineto{\pgfqpoint{1.748656in}{1.399347in}}%
\pgfpathlineto{\pgfqpoint{1.731900in}{1.402150in}}%
\pgfpathlineto{\pgfqpoint{1.715143in}{1.404955in}}%
\pgfpathlineto{\pgfqpoint{1.698387in}{1.407765in}}%
\pgfpathlineto{\pgfqpoint{1.681631in}{1.410574in}}%
\pgfpathlineto{\pgfqpoint{1.664874in}{1.413379in}}%
\pgfpathlineto{\pgfqpoint{1.648118in}{1.416187in}}%
\pgfpathlineto{\pgfqpoint{1.631362in}{1.419057in}}%
\pgfpathlineto{\pgfqpoint{1.614606in}{1.421902in}}%
\pgfpathlineto{\pgfqpoint{1.597849in}{1.424779in}}%
\pgfpathlineto{\pgfqpoint{1.581093in}{1.427724in}}%
\pgfpathlineto{\pgfqpoint{1.564337in}{1.430633in}}%
\pgfpathlineto{\pgfqpoint{1.547581in}{1.433484in}}%
\pgfpathlineto{\pgfqpoint{1.530824in}{1.436480in}}%
\pgfpathlineto{\pgfqpoint{1.514068in}{1.439385in}}%
\pgfpathlineto{\pgfqpoint{1.497312in}{1.442141in}}%
\pgfpathlineto{\pgfqpoint{1.480555in}{1.445208in}}%
\pgfpathlineto{\pgfqpoint{1.463799in}{1.448393in}}%
\pgfpathlineto{\pgfqpoint{1.447043in}{1.451436in}}%
\pgfpathlineto{\pgfqpoint{1.430287in}{1.454661in}}%
\pgfpathlineto{\pgfqpoint{1.413530in}{1.457807in}}%
\pgfpathlineto{\pgfqpoint{1.396774in}{1.460859in}}%
\pgfpathlineto{\pgfqpoint{1.380018in}{1.463901in}}%
\pgfpathlineto{\pgfqpoint{1.363262in}{1.466946in}}%
\pgfpathlineto{\pgfqpoint{1.346505in}{1.470001in}}%
\pgfpathlineto{\pgfqpoint{1.329749in}{1.473206in}}%
\pgfpathlineto{\pgfqpoint{1.312993in}{1.476600in}}%
\pgfpathlineto{\pgfqpoint{1.296236in}{1.479945in}}%
\pgfpathlineto{\pgfqpoint{1.279480in}{1.483046in}}%
\pgfpathlineto{\pgfqpoint{1.262724in}{1.486369in}}%
\pgfpathlineto{\pgfqpoint{1.245968in}{1.489388in}}%
\pgfpathlineto{\pgfqpoint{1.229211in}{1.492657in}}%
\pgfpathlineto{\pgfqpoint{1.212455in}{1.495906in}}%
\pgfpathlineto{\pgfqpoint{1.195699in}{1.499274in}}%
\pgfpathlineto{\pgfqpoint{1.178942in}{1.502628in}}%
\pgfpathlineto{\pgfqpoint{1.162186in}{1.506133in}}%
\pgfpathlineto{\pgfqpoint{1.145430in}{1.509395in}}%
\pgfpathlineto{\pgfqpoint{1.128674in}{1.512757in}}%
\pgfpathlineto{\pgfqpoint{1.111917in}{1.516223in}}%
\pgfpathlineto{\pgfqpoint{1.095161in}{1.519690in}}%
\pgfpathlineto{\pgfqpoint{1.078405in}{1.523159in}}%
\pgfpathlineto{\pgfqpoint{1.061649in}{1.526748in}}%
\pgfpathlineto{\pgfqpoint{1.044892in}{1.530364in}}%
\pgfpathlineto{\pgfqpoint{1.028136in}{1.533889in}}%
\pgfpathlineto{\pgfqpoint{1.011380in}{1.537445in}}%
\pgfpathlineto{\pgfqpoint{0.994623in}{1.540971in}}%
\pgfpathlineto{\pgfqpoint{0.977867in}{1.544513in}}%
\pgfpathlineto{\pgfqpoint{0.961111in}{1.548055in}}%
\pgfpathlineto{\pgfqpoint{0.944355in}{1.551913in}}%
\pgfpathlineto{\pgfqpoint{0.927598in}{1.555433in}}%
\pgfpathlineto{\pgfqpoint{0.910842in}{1.558896in}}%
\pgfpathlineto{\pgfqpoint{0.894086in}{1.562362in}}%
\pgfpathlineto{\pgfqpoint{0.877330in}{1.565933in}}%
\pgfpathlineto{\pgfqpoint{0.860573in}{1.569317in}}%
\pgfpathlineto{\pgfqpoint{0.843817in}{1.572865in}}%
\pgfpathlineto{\pgfqpoint{0.827061in}{1.576591in}}%
\pgfpathlineto{\pgfqpoint{0.810304in}{1.580505in}}%
\pgfpathlineto{\pgfqpoint{0.793548in}{1.584418in}}%
\pgfpathlineto{\pgfqpoint{0.776792in}{1.587997in}}%
\pgfpathlineto{\pgfqpoint{0.760036in}{1.591551in}}%
\pgfpathlineto{\pgfqpoint{0.743279in}{1.595105in}}%
\pgfpathlineto{\pgfqpoint{0.726523in}{1.598659in}}%
\pgfpathlineto{\pgfqpoint{0.709767in}{1.602214in}}%
\pgfpathlineto{\pgfqpoint{0.693011in}{1.605768in}}%
\pgfpathlineto{\pgfqpoint{0.676254in}{1.609322in}}%
\pgfpathlineto{\pgfqpoint{0.659498in}{1.612922in}}%
\pgfpathlineto{\pgfqpoint{0.642742in}{1.616460in}}%
\pgfpathlineto{\pgfqpoint{0.625985in}{1.620009in}}%
\pgfpathlineto{\pgfqpoint{0.609229in}{1.623660in}}%
\pgfpathlineto{\pgfqpoint{0.592473in}{1.627311in}}%
\pgfpathlineto{\pgfqpoint{0.575717in}{1.630955in}}%
\pgfpathlineto{\pgfqpoint{0.558960in}{1.634599in}}%
\pgfpathlineto{\pgfqpoint{0.542204in}{1.638243in}}%
\pgfpathlineto{\pgfqpoint{0.525448in}{1.641887in}}%
\pgfpathlineto{\pgfqpoint{0.508691in}{1.645531in}}%
\pgfpathlineto{\pgfqpoint{0.491935in}{1.649174in}}%
\pgfpathlineto{\pgfqpoint{0.475179in}{1.652817in}}%
\pgfpathlineto{\pgfqpoint{0.458423in}{1.656461in}}%
\pgfpathlineto{\pgfqpoint{0.441666in}{1.660126in}}%
\pgfpathlineto{\pgfqpoint{0.424910in}{1.663773in}}%
\pgfpathlineto{\pgfqpoint{0.408154in}{1.667402in}}%
\pgfpathlineto{\pgfqpoint{0.391398in}{1.671052in}}%
\pgfpathlineto{\pgfqpoint{0.374641in}{1.674703in}}%
\pgfpathlineto{\pgfqpoint{0.357885in}{1.678353in}}%
\pgfpathlineto{\pgfqpoint{0.341129in}{1.682003in}}%
\pgfpathclose%
\pgfusepath{stroke,fill}%
}%
\begin{pgfscope}%
\pgfsys@transformshift{0.000000in}{0.000000in}%
\pgfsys@useobject{currentmarker}{}%
\end{pgfscope}%
\end{pgfscope}%
\begin{pgfscope}%
\pgfpathrectangle{\pgfqpoint{0.341129in}{0.466613in}}{\pgfqpoint{1.658871in}{1.711598in}}%
\pgfusepath{clip}%
\pgfsetbuttcap%
\pgfsetroundjoin%
\definecolor{currentfill}{rgb}{0.768627,0.305882,0.321569}%
\pgfsetfillcolor{currentfill}%
\pgfsetfillopacity{0.250000}%
\pgfsetlinewidth{1.003750pt}%
\definecolor{currentstroke}{rgb}{0.768627,0.305882,0.321569}%
\pgfsetstrokecolor{currentstroke}%
\pgfsetstrokeopacity{0.250000}%
\pgfsetdash{}{0pt}%
\pgfsys@defobject{currentmarker}{\pgfqpoint{-0.017010in}{-0.017010in}}{\pgfqpoint{0.017010in}{0.017010in}}{%
\pgfpathmoveto{\pgfqpoint{0.000000in}{-0.017010in}}%
\pgfpathcurveto{\pgfqpoint{0.004511in}{-0.017010in}}{\pgfqpoint{0.008838in}{-0.015218in}}{\pgfqpoint{0.012028in}{-0.012028in}}%
\pgfpathcurveto{\pgfqpoint{0.015218in}{-0.008838in}}{\pgfqpoint{0.017010in}{-0.004511in}}{\pgfqpoint{0.017010in}{0.000000in}}%
\pgfpathcurveto{\pgfqpoint{0.017010in}{0.004511in}}{\pgfqpoint{0.015218in}{0.008838in}}{\pgfqpoint{0.012028in}{0.012028in}}%
\pgfpathcurveto{\pgfqpoint{0.008838in}{0.015218in}}{\pgfqpoint{0.004511in}{0.017010in}}{\pgfqpoint{0.000000in}{0.017010in}}%
\pgfpathcurveto{\pgfqpoint{-0.004511in}{0.017010in}}{\pgfqpoint{-0.008838in}{0.015218in}}{\pgfqpoint{-0.012028in}{0.012028in}}%
\pgfpathcurveto{\pgfqpoint{-0.015218in}{0.008838in}}{\pgfqpoint{-0.017010in}{0.004511in}}{\pgfqpoint{-0.017010in}{0.000000in}}%
\pgfpathcurveto{\pgfqpoint{-0.017010in}{-0.004511in}}{\pgfqpoint{-0.015218in}{-0.008838in}}{\pgfqpoint{-0.012028in}{-0.012028in}}%
\pgfpathcurveto{\pgfqpoint{-0.008838in}{-0.015218in}}{\pgfqpoint{-0.004511in}{-0.017010in}}{\pgfqpoint{0.000000in}{-0.017010in}}%
\pgfpathclose%
\pgfusepath{stroke,fill}%
}%
\begin{pgfscope}%
\pgfsys@transformshift{0.667456in}{1.487621in}%
\pgfsys@useobject{currentmarker}{}%
\end{pgfscope}%
\begin{pgfscope}%
\pgfsys@transformshift{0.872203in}{1.427337in}%
\pgfsys@useobject{currentmarker}{}%
\end{pgfscope}%
\begin{pgfscope}%
\pgfsys@transformshift{1.174456in}{1.265604in}%
\pgfsys@useobject{currentmarker}{}%
\end{pgfscope}%
\begin{pgfscope}%
\pgfsys@transformshift{1.594005in}{1.152049in}%
\pgfsys@useobject{currentmarker}{}%
\end{pgfscope}%
\begin{pgfscope}%
\pgfsys@transformshift{1.217413in}{1.118678in}%
\pgfsys@useobject{currentmarker}{}%
\end{pgfscope}%
\begin{pgfscope}%
\pgfsys@transformshift{0.553973in}{1.458827in}%
\pgfsys@useobject{currentmarker}{}%
\end{pgfscope}%
\begin{pgfscope}%
\pgfsys@transformshift{1.639332in}{1.090915in}%
\pgfsys@useobject{currentmarker}{}%
\end{pgfscope}%
\begin{pgfscope}%
\pgfsys@transformshift{0.709729in}{1.521982in}%
\pgfsys@useobject{currentmarker}{}%
\end{pgfscope}%
\begin{pgfscope}%
\pgfsys@transformshift{1.202352in}{1.476897in}%
\pgfsys@useobject{currentmarker}{}%
\end{pgfscope}%
\begin{pgfscope}%
\pgfsys@transformshift{0.671018in}{1.326899in}%
\pgfsys@useobject{currentmarker}{}%
\end{pgfscope}%
\begin{pgfscope}%
\pgfsys@transformshift{1.442003in}{1.215456in}%
\pgfsys@useobject{currentmarker}{}%
\end{pgfscope}%
\begin{pgfscope}%
\pgfsys@transformshift{1.398849in}{1.251349in}%
\pgfsys@useobject{currentmarker}{}%
\end{pgfscope}%
\begin{pgfscope}%
\pgfsys@transformshift{1.552752in}{1.072476in}%
\pgfsys@useobject{currentmarker}{}%
\end{pgfscope}%
\begin{pgfscope}%
\pgfsys@transformshift{1.177147in}{1.332324in}%
\pgfsys@useobject{currentmarker}{}%
\end{pgfscope}%
\begin{pgfscope}%
\pgfsys@transformshift{1.887294in}{1.284761in}%
\pgfsys@useobject{currentmarker}{}%
\end{pgfscope}%
\begin{pgfscope}%
\pgfsys@transformshift{1.188970in}{1.228126in}%
\pgfsys@useobject{currentmarker}{}%
\end{pgfscope}%
\begin{pgfscope}%
\pgfsys@transformshift{1.534509in}{1.308942in}%
\pgfsys@useobject{currentmarker}{}%
\end{pgfscope}%
\begin{pgfscope}%
\pgfsys@transformshift{0.588572in}{1.456402in}%
\pgfsys@useobject{currentmarker}{}%
\end{pgfscope}%
\begin{pgfscope}%
\pgfsys@transformshift{0.952296in}{1.346079in}%
\pgfsys@useobject{currentmarker}{}%
\end{pgfscope}%
\begin{pgfscope}%
\pgfsys@transformshift{0.705522in}{1.433052in}%
\pgfsys@useobject{currentmarker}{}%
\end{pgfscope}%
\begin{pgfscope}%
\pgfsys@transformshift{1.121384in}{1.214484in}%
\pgfsys@useobject{currentmarker}{}%
\end{pgfscope}%
\begin{pgfscope}%
\pgfsys@transformshift{1.103574in}{1.206455in}%
\pgfsys@useobject{currentmarker}{}%
\end{pgfscope}%
\begin{pgfscope}%
\pgfsys@transformshift{1.817005in}{1.200958in}%
\pgfsys@useobject{currentmarker}{}%
\end{pgfscope}%
\begin{pgfscope}%
\pgfsys@transformshift{0.496492in}{1.524847in}%
\pgfsys@useobject{currentmarker}{}%
\end{pgfscope}%
\begin{pgfscope}%
\pgfsys@transformshift{1.711311in}{1.234917in}%
\pgfsys@useobject{currentmarker}{}%
\end{pgfscope}%
\begin{pgfscope}%
\pgfsys@transformshift{1.750408in}{0.924368in}%
\pgfsys@useobject{currentmarker}{}%
\end{pgfscope}%
\begin{pgfscope}%
\pgfsys@transformshift{1.333406in}{1.148236in}%
\pgfsys@useobject{currentmarker}{}%
\end{pgfscope}%
\begin{pgfscope}%
\pgfsys@transformshift{1.724223in}{1.083610in}%
\pgfsys@useobject{currentmarker}{}%
\end{pgfscope}%
\begin{pgfscope}%
\pgfsys@transformshift{1.256999in}{1.263815in}%
\pgfsys@useobject{currentmarker}{}%
\end{pgfscope}%
\begin{pgfscope}%
\pgfsys@transformshift{1.047649in}{1.342683in}%
\pgfsys@useobject{currentmarker}{}%
\end{pgfscope}%
\begin{pgfscope}%
\pgfsys@transformshift{0.878614in}{1.422431in}%
\pgfsys@useobject{currentmarker}{}%
\end{pgfscope}%
\begin{pgfscope}%
\pgfsys@transformshift{0.761954in}{1.468650in}%
\pgfsys@useobject{currentmarker}{}%
\end{pgfscope}%
\begin{pgfscope}%
\pgfsys@transformshift{1.796268in}{1.069346in}%
\pgfsys@useobject{currentmarker}{}%
\end{pgfscope}%
\begin{pgfscope}%
\pgfsys@transformshift{0.726378in}{1.481342in}%
\pgfsys@useobject{currentmarker}{}%
\end{pgfscope}%
\begin{pgfscope}%
\pgfsys@transformshift{1.408196in}{1.212673in}%
\pgfsys@useobject{currentmarker}{}%
\end{pgfscope}%
\begin{pgfscope}%
\pgfsys@transformshift{0.612217in}{1.127193in}%
\pgfsys@useobject{currentmarker}{}%
\end{pgfscope}%
\begin{pgfscope}%
\pgfsys@transformshift{1.180902in}{1.405953in}%
\pgfsys@useobject{currentmarker}{}%
\end{pgfscope}%
\begin{pgfscope}%
\pgfsys@transformshift{0.627344in}{1.481645in}%
\pgfsys@useobject{currentmarker}{}%
\end{pgfscope}%
\begin{pgfscope}%
\pgfsys@transformshift{0.628282in}{1.421390in}%
\pgfsys@useobject{currentmarker}{}%
\end{pgfscope}%
\begin{pgfscope}%
\pgfsys@transformshift{0.437992in}{1.126044in}%
\pgfsys@useobject{currentmarker}{}%
\end{pgfscope}%
\begin{pgfscope}%
\pgfsys@transformshift{0.758691in}{1.421956in}%
\pgfsys@useobject{currentmarker}{}%
\end{pgfscope}%
\begin{pgfscope}%
\pgfsys@transformshift{1.614220in}{1.323720in}%
\pgfsys@useobject{currentmarker}{}%
\end{pgfscope}%
\begin{pgfscope}%
\pgfsys@transformshift{1.846563in}{1.109957in}%
\pgfsys@useobject{currentmarker}{}%
\end{pgfscope}%
\begin{pgfscope}%
\pgfsys@transformshift{1.460173in}{1.292715in}%
\pgfsys@useobject{currentmarker}{}%
\end{pgfscope}%
\begin{pgfscope}%
\pgfsys@transformshift{1.637112in}{1.081361in}%
\pgfsys@useobject{currentmarker}{}%
\end{pgfscope}%
\begin{pgfscope}%
\pgfsys@transformshift{1.615512in}{1.060600in}%
\pgfsys@useobject{currentmarker}{}%
\end{pgfscope}%
\begin{pgfscope}%
\pgfsys@transformshift{1.242112in}{1.488579in}%
\pgfsys@useobject{currentmarker}{}%
\end{pgfscope}%
\begin{pgfscope}%
\pgfsys@transformshift{0.688214in}{1.378134in}%
\pgfsys@useobject{currentmarker}{}%
\end{pgfscope}%
\begin{pgfscope}%
\pgfsys@transformshift{0.911315in}{1.104103in}%
\pgfsys@useobject{currentmarker}{}%
\end{pgfscope}%
\begin{pgfscope}%
\pgfsys@transformshift{0.551462in}{1.576300in}%
\pgfsys@useobject{currentmarker}{}%
\end{pgfscope}%
\begin{pgfscope}%
\pgfsys@transformshift{1.069973in}{1.329957in}%
\pgfsys@useobject{currentmarker}{}%
\end{pgfscope}%
\begin{pgfscope}%
\pgfsys@transformshift{1.211180in}{1.441060in}%
\pgfsys@useobject{currentmarker}{}%
\end{pgfscope}%
\begin{pgfscope}%
\pgfsys@transformshift{1.485757in}{1.389649in}%
\pgfsys@useobject{currentmarker}{}%
\end{pgfscope}%
\begin{pgfscope}%
\pgfsys@transformshift{0.743119in}{1.340930in}%
\pgfsys@useobject{currentmarker}{}%
\end{pgfscope}%
\begin{pgfscope}%
\pgfsys@transformshift{1.407385in}{1.135908in}%
\pgfsys@useobject{currentmarker}{}%
\end{pgfscope}%
\begin{pgfscope}%
\pgfsys@transformshift{1.618799in}{1.425750in}%
\pgfsys@useobject{currentmarker}{}%
\end{pgfscope}%
\begin{pgfscope}%
\pgfsys@transformshift{1.477965in}{1.094038in}%
\pgfsys@useobject{currentmarker}{}%
\end{pgfscope}%
\begin{pgfscope}%
\pgfsys@transformshift{1.156480in}{1.405909in}%
\pgfsys@useobject{currentmarker}{}%
\end{pgfscope}%
\begin{pgfscope}%
\pgfsys@transformshift{1.583657in}{1.155991in}%
\pgfsys@useobject{currentmarker}{}%
\end{pgfscope}%
\begin{pgfscope}%
\pgfsys@transformshift{1.440624in}{1.167599in}%
\pgfsys@useobject{currentmarker}{}%
\end{pgfscope}%
\begin{pgfscope}%
\pgfsys@transformshift{1.318935in}{1.188448in}%
\pgfsys@useobject{currentmarker}{}%
\end{pgfscope}%
\begin{pgfscope}%
\pgfsys@transformshift{0.547443in}{1.661860in}%
\pgfsys@useobject{currentmarker}{}%
\end{pgfscope}%
\begin{pgfscope}%
\pgfsys@transformshift{1.436824in}{1.300280in}%
\pgfsys@useobject{currentmarker}{}%
\end{pgfscope}%
\begin{pgfscope}%
\pgfsys@transformshift{1.658179in}{1.045589in}%
\pgfsys@useobject{currentmarker}{}%
\end{pgfscope}%
\begin{pgfscope}%
\pgfsys@transformshift{1.709942in}{1.103323in}%
\pgfsys@useobject{currentmarker}{}%
\end{pgfscope}%
\begin{pgfscope}%
\pgfsys@transformshift{1.638310in}{1.054395in}%
\pgfsys@useobject{currentmarker}{}%
\end{pgfscope}%
\begin{pgfscope}%
\pgfsys@transformshift{1.723779in}{1.118134in}%
\pgfsys@useobject{currentmarker}{}%
\end{pgfscope}%
\begin{pgfscope}%
\pgfsys@transformshift{1.124544in}{1.324805in}%
\pgfsys@useobject{currentmarker}{}%
\end{pgfscope}%
\begin{pgfscope}%
\pgfsys@transformshift{1.267442in}{1.377459in}%
\pgfsys@useobject{currentmarker}{}%
\end{pgfscope}%
\begin{pgfscope}%
\pgfsys@transformshift{0.562064in}{1.046386in}%
\pgfsys@useobject{currentmarker}{}%
\end{pgfscope}%
\begin{pgfscope}%
\pgfsys@transformshift{0.642722in}{1.427247in}%
\pgfsys@useobject{currentmarker}{}%
\end{pgfscope}%
\begin{pgfscope}%
\pgfsys@transformshift{1.150269in}{1.534900in}%
\pgfsys@useobject{currentmarker}{}%
\end{pgfscope}%
\begin{pgfscope}%
\pgfsys@transformshift{1.173281in}{1.291442in}%
\pgfsys@useobject{currentmarker}{}%
\end{pgfscope}%
\begin{pgfscope}%
\pgfsys@transformshift{1.891719in}{1.315794in}%
\pgfsys@useobject{currentmarker}{}%
\end{pgfscope}%
\begin{pgfscope}%
\pgfsys@transformshift{0.694241in}{1.455330in}%
\pgfsys@useobject{currentmarker}{}%
\end{pgfscope}%
\begin{pgfscope}%
\pgfsys@transformshift{0.790322in}{1.427226in}%
\pgfsys@useobject{currentmarker}{}%
\end{pgfscope}%
\begin{pgfscope}%
\pgfsys@transformshift{1.424550in}{1.083965in}%
\pgfsys@useobject{currentmarker}{}%
\end{pgfscope}%
\begin{pgfscope}%
\pgfsys@transformshift{0.384790in}{1.115208in}%
\pgfsys@useobject{currentmarker}{}%
\end{pgfscope}%
\begin{pgfscope}%
\pgfsys@transformshift{0.729116in}{1.418632in}%
\pgfsys@useobject{currentmarker}{}%
\end{pgfscope}%
\begin{pgfscope}%
\pgfsys@transformshift{1.217637in}{1.333195in}%
\pgfsys@useobject{currentmarker}{}%
\end{pgfscope}%
\begin{pgfscope}%
\pgfsys@transformshift{0.452540in}{1.512835in}%
\pgfsys@useobject{currentmarker}{}%
\end{pgfscope}%
\begin{pgfscope}%
\pgfsys@transformshift{1.493704in}{1.048900in}%
\pgfsys@useobject{currentmarker}{}%
\end{pgfscope}%
\begin{pgfscope}%
\pgfsys@transformshift{1.543302in}{1.234563in}%
\pgfsys@useobject{currentmarker}{}%
\end{pgfscope}%
\begin{pgfscope}%
\pgfsys@transformshift{1.218014in}{1.236837in}%
\pgfsys@useobject{currentmarker}{}%
\end{pgfscope}%
\begin{pgfscope}%
\pgfsys@transformshift{1.521861in}{1.202368in}%
\pgfsys@useobject{currentmarker}{}%
\end{pgfscope}%
\begin{pgfscope}%
\pgfsys@transformshift{1.764286in}{1.194625in}%
\pgfsys@useobject{currentmarker}{}%
\end{pgfscope}%
\begin{pgfscope}%
\pgfsys@transformshift{0.924889in}{1.635571in}%
\pgfsys@useobject{currentmarker}{}%
\end{pgfscope}%
\begin{pgfscope}%
\pgfsys@transformshift{1.222866in}{1.136287in}%
\pgfsys@useobject{currentmarker}{}%
\end{pgfscope}%
\begin{pgfscope}%
\pgfsys@transformshift{1.382222in}{1.249559in}%
\pgfsys@useobject{currentmarker}{}%
\end{pgfscope}%
\begin{pgfscope}%
\pgfsys@transformshift{1.643945in}{1.264014in}%
\pgfsys@useobject{currentmarker}{}%
\end{pgfscope}%
\begin{pgfscope}%
\pgfsys@transformshift{0.801834in}{1.318721in}%
\pgfsys@useobject{currentmarker}{}%
\end{pgfscope}%
\begin{pgfscope}%
\pgfsys@transformshift{1.258384in}{1.365843in}%
\pgfsys@useobject{currentmarker}{}%
\end{pgfscope}%
\begin{pgfscope}%
\pgfsys@transformshift{1.394783in}{1.200560in}%
\pgfsys@useobject{currentmarker}{}%
\end{pgfscope}%
\begin{pgfscope}%
\pgfsys@transformshift{0.341129in}{1.587387in}%
\pgfsys@useobject{currentmarker}{}%
\end{pgfscope}%
\begin{pgfscope}%
\pgfsys@transformshift{1.069581in}{1.418323in}%
\pgfsys@useobject{currentmarker}{}%
\end{pgfscope}%
\begin{pgfscope}%
\pgfsys@transformshift{0.789606in}{1.635358in}%
\pgfsys@useobject{currentmarker}{}%
\end{pgfscope}%
\begin{pgfscope}%
\pgfsys@transformshift{1.273287in}{1.131720in}%
\pgfsys@useobject{currentmarker}{}%
\end{pgfscope}%
\begin{pgfscope}%
\pgfsys@transformshift{0.646998in}{1.470948in}%
\pgfsys@useobject{currentmarker}{}%
\end{pgfscope}%
\begin{pgfscope}%
\pgfsys@transformshift{0.761065in}{1.556885in}%
\pgfsys@useobject{currentmarker}{}%
\end{pgfscope}%
\begin{pgfscope}%
\pgfsys@transformshift{0.628168in}{1.754135in}%
\pgfsys@useobject{currentmarker}{}%
\end{pgfscope}%
\begin{pgfscope}%
\pgfsys@transformshift{1.214800in}{1.316163in}%
\pgfsys@useobject{currentmarker}{}%
\end{pgfscope}%
\begin{pgfscope}%
\pgfsys@transformshift{1.051231in}{1.333520in}%
\pgfsys@useobject{currentmarker}{}%
\end{pgfscope}%
\begin{pgfscope}%
\pgfsys@transformshift{1.010738in}{1.357168in}%
\pgfsys@useobject{currentmarker}{}%
\end{pgfscope}%
\begin{pgfscope}%
\pgfsys@transformshift{1.387136in}{1.167057in}%
\pgfsys@useobject{currentmarker}{}%
\end{pgfscope}%
\begin{pgfscope}%
\pgfsys@transformshift{0.522501in}{1.061252in}%
\pgfsys@useobject{currentmarker}{}%
\end{pgfscope}%
\begin{pgfscope}%
\pgfsys@transformshift{1.737851in}{1.295987in}%
\pgfsys@useobject{currentmarker}{}%
\end{pgfscope}%
\begin{pgfscope}%
\pgfsys@transformshift{1.622475in}{1.140820in}%
\pgfsys@useobject{currentmarker}{}%
\end{pgfscope}%
\begin{pgfscope}%
\pgfsys@transformshift{0.776054in}{1.113773in}%
\pgfsys@useobject{currentmarker}{}%
\end{pgfscope}%
\begin{pgfscope}%
\pgfsys@transformshift{1.260864in}{1.064531in}%
\pgfsys@useobject{currentmarker}{}%
\end{pgfscope}%
\begin{pgfscope}%
\pgfsys@transformshift{1.587232in}{1.140520in}%
\pgfsys@useobject{currentmarker}{}%
\end{pgfscope}%
\begin{pgfscope}%
\pgfsys@transformshift{1.339693in}{1.268904in}%
\pgfsys@useobject{currentmarker}{}%
\end{pgfscope}%
\begin{pgfscope}%
\pgfsys@transformshift{1.448928in}{1.041973in}%
\pgfsys@useobject{currentmarker}{}%
\end{pgfscope}%
\begin{pgfscope}%
\pgfsys@transformshift{1.848941in}{0.961047in}%
\pgfsys@useobject{currentmarker}{}%
\end{pgfscope}%
\begin{pgfscope}%
\pgfsys@transformshift{1.645721in}{1.174268in}%
\pgfsys@useobject{currentmarker}{}%
\end{pgfscope}%
\begin{pgfscope}%
\pgfsys@transformshift{0.980835in}{1.388876in}%
\pgfsys@useobject{currentmarker}{}%
\end{pgfscope}%
\begin{pgfscope}%
\pgfsys@transformshift{0.704938in}{1.581507in}%
\pgfsys@useobject{currentmarker}{}%
\end{pgfscope}%
\begin{pgfscope}%
\pgfsys@transformshift{0.622306in}{1.530408in}%
\pgfsys@useobject{currentmarker}{}%
\end{pgfscope}%
\begin{pgfscope}%
\pgfsys@transformshift{1.213744in}{1.076592in}%
\pgfsys@useobject{currentmarker}{}%
\end{pgfscope}%
\begin{pgfscope}%
\pgfsys@transformshift{0.934075in}{1.461289in}%
\pgfsys@useobject{currentmarker}{}%
\end{pgfscope}%
\begin{pgfscope}%
\pgfsys@transformshift{1.307417in}{1.138443in}%
\pgfsys@useobject{currentmarker}{}%
\end{pgfscope}%
\begin{pgfscope}%
\pgfsys@transformshift{1.605349in}{1.378099in}%
\pgfsys@useobject{currentmarker}{}%
\end{pgfscope}%
\begin{pgfscope}%
\pgfsys@transformshift{0.750394in}{1.442857in}%
\pgfsys@useobject{currentmarker}{}%
\end{pgfscope}%
\begin{pgfscope}%
\pgfsys@transformshift{1.771258in}{1.209871in}%
\pgfsys@useobject{currentmarker}{}%
\end{pgfscope}%
\begin{pgfscope}%
\pgfsys@transformshift{1.939172in}{1.337862in}%
\pgfsys@useobject{currentmarker}{}%
\end{pgfscope}%
\begin{pgfscope}%
\pgfsys@transformshift{1.599583in}{1.357555in}%
\pgfsys@useobject{currentmarker}{}%
\end{pgfscope}%
\begin{pgfscope}%
\pgfsys@transformshift{0.881257in}{1.348575in}%
\pgfsys@useobject{currentmarker}{}%
\end{pgfscope}%
\begin{pgfscope}%
\pgfsys@transformshift{1.201768in}{1.395887in}%
\pgfsys@useobject{currentmarker}{}%
\end{pgfscope}%
\begin{pgfscope}%
\pgfsys@transformshift{1.244134in}{1.353053in}%
\pgfsys@useobject{currentmarker}{}%
\end{pgfscope}%
\begin{pgfscope}%
\pgfsys@transformshift{1.750150in}{1.177885in}%
\pgfsys@useobject{currentmarker}{}%
\end{pgfscope}%
\begin{pgfscope}%
\pgfsys@transformshift{1.541153in}{1.222493in}%
\pgfsys@useobject{currentmarker}{}%
\end{pgfscope}%
\begin{pgfscope}%
\pgfsys@transformshift{0.964663in}{1.107836in}%
\pgfsys@useobject{currentmarker}{}%
\end{pgfscope}%
\begin{pgfscope}%
\pgfsys@transformshift{0.818109in}{1.353320in}%
\pgfsys@useobject{currentmarker}{}%
\end{pgfscope}%
\begin{pgfscope}%
\pgfsys@transformshift{1.755264in}{1.065316in}%
\pgfsys@useobject{currentmarker}{}%
\end{pgfscope}%
\begin{pgfscope}%
\pgfsys@transformshift{1.450480in}{1.340079in}%
\pgfsys@useobject{currentmarker}{}%
\end{pgfscope}%
\begin{pgfscope}%
\pgfsys@transformshift{1.246595in}{1.392964in}%
\pgfsys@useobject{currentmarker}{}%
\end{pgfscope}%
\begin{pgfscope}%
\pgfsys@transformshift{1.110348in}{1.495608in}%
\pgfsys@useobject{currentmarker}{}%
\end{pgfscope}%
\begin{pgfscope}%
\pgfsys@transformshift{0.853154in}{1.271264in}%
\pgfsys@useobject{currentmarker}{}%
\end{pgfscope}%
\begin{pgfscope}%
\pgfsys@transformshift{0.803206in}{1.466251in}%
\pgfsys@useobject{currentmarker}{}%
\end{pgfscope}%
\begin{pgfscope}%
\pgfsys@transformshift{0.800584in}{1.476564in}%
\pgfsys@useobject{currentmarker}{}%
\end{pgfscope}%
\begin{pgfscope}%
\pgfsys@transformshift{0.828786in}{1.453264in}%
\pgfsys@useobject{currentmarker}{}%
\end{pgfscope}%
\begin{pgfscope}%
\pgfsys@transformshift{0.802472in}{1.499793in}%
\pgfsys@useobject{currentmarker}{}%
\end{pgfscope}%
\begin{pgfscope}%
\pgfsys@transformshift{0.778610in}{1.601066in}%
\pgfsys@useobject{currentmarker}{}%
\end{pgfscope}%
\begin{pgfscope}%
\pgfsys@transformshift{1.678430in}{1.231471in}%
\pgfsys@useobject{currentmarker}{}%
\end{pgfscope}%
\begin{pgfscope}%
\pgfsys@transformshift{1.678534in}{1.308634in}%
\pgfsys@useobject{currentmarker}{}%
\end{pgfscope}%
\begin{pgfscope}%
\pgfsys@transformshift{1.164510in}{1.211948in}%
\pgfsys@useobject{currentmarker}{}%
\end{pgfscope}%
\begin{pgfscope}%
\pgfsys@transformshift{1.533885in}{1.158270in}%
\pgfsys@useobject{currentmarker}{}%
\end{pgfscope}%
\begin{pgfscope}%
\pgfsys@transformshift{0.532239in}{1.418390in}%
\pgfsys@useobject{currentmarker}{}%
\end{pgfscope}%
\begin{pgfscope}%
\pgfsys@transformshift{1.059630in}{1.224029in}%
\pgfsys@useobject{currentmarker}{}%
\end{pgfscope}%
\begin{pgfscope}%
\pgfsys@transformshift{1.471016in}{1.247864in}%
\pgfsys@useobject{currentmarker}{}%
\end{pgfscope}%
\begin{pgfscope}%
\pgfsys@transformshift{0.764422in}{1.425059in}%
\pgfsys@useobject{currentmarker}{}%
\end{pgfscope}%
\begin{pgfscope}%
\pgfsys@transformshift{1.724626in}{1.070873in}%
\pgfsys@useobject{currentmarker}{}%
\end{pgfscope}%
\begin{pgfscope}%
\pgfsys@transformshift{1.557866in}{1.103186in}%
\pgfsys@useobject{currentmarker}{}%
\end{pgfscope}%
\begin{pgfscope}%
\pgfsys@transformshift{1.322665in}{1.222093in}%
\pgfsys@useobject{currentmarker}{}%
\end{pgfscope}%
\begin{pgfscope}%
\pgfsys@transformshift{1.043939in}{1.331682in}%
\pgfsys@useobject{currentmarker}{}%
\end{pgfscope}%
\begin{pgfscope}%
\pgfsys@transformshift{1.615078in}{1.199121in}%
\pgfsys@useobject{currentmarker}{}%
\end{pgfscope}%
\begin{pgfscope}%
\pgfsys@transformshift{1.467591in}{1.350069in}%
\pgfsys@useobject{currentmarker}{}%
\end{pgfscope}%
\begin{pgfscope}%
\pgfsys@transformshift{0.991846in}{1.254044in}%
\pgfsys@useobject{currentmarker}{}%
\end{pgfscope}%
\begin{pgfscope}%
\pgfsys@transformshift{0.697891in}{1.356252in}%
\pgfsys@useobject{currentmarker}{}%
\end{pgfscope}%
\begin{pgfscope}%
\pgfsys@transformshift{1.294350in}{1.450035in}%
\pgfsys@useobject{currentmarker}{}%
\end{pgfscope}%
\begin{pgfscope}%
\pgfsys@transformshift{1.409354in}{1.046236in}%
\pgfsys@useobject{currentmarker}{}%
\end{pgfscope}%
\begin{pgfscope}%
\pgfsys@transformshift{0.733011in}{1.727609in}%
\pgfsys@useobject{currentmarker}{}%
\end{pgfscope}%
\begin{pgfscope}%
\pgfsys@transformshift{0.634257in}{1.441302in}%
\pgfsys@useobject{currentmarker}{}%
\end{pgfscope}%
\begin{pgfscope}%
\pgfsys@transformshift{0.526826in}{1.662876in}%
\pgfsys@useobject{currentmarker}{}%
\end{pgfscope}%
\begin{pgfscope}%
\pgfsys@transformshift{1.225515in}{1.416535in}%
\pgfsys@useobject{currentmarker}{}%
\end{pgfscope}%
\begin{pgfscope}%
\pgfsys@transformshift{0.578368in}{1.171075in}%
\pgfsys@useobject{currentmarker}{}%
\end{pgfscope}%
\begin{pgfscope}%
\pgfsys@transformshift{1.515054in}{1.277981in}%
\pgfsys@useobject{currentmarker}{}%
\end{pgfscope}%
\begin{pgfscope}%
\pgfsys@transformshift{1.989373in}{1.167835in}%
\pgfsys@useobject{currentmarker}{}%
\end{pgfscope}%
\begin{pgfscope}%
\pgfsys@transformshift{1.360084in}{1.420045in}%
\pgfsys@useobject{currentmarker}{}%
\end{pgfscope}%
\begin{pgfscope}%
\pgfsys@transformshift{0.851541in}{1.434573in}%
\pgfsys@useobject{currentmarker}{}%
\end{pgfscope}%
\begin{pgfscope}%
\pgfsys@transformshift{1.688966in}{1.165732in}%
\pgfsys@useobject{currentmarker}{}%
\end{pgfscope}%
\begin{pgfscope}%
\pgfsys@transformshift{1.144955in}{1.288450in}%
\pgfsys@useobject{currentmarker}{}%
\end{pgfscope}%
\begin{pgfscope}%
\pgfsys@transformshift{1.681418in}{1.041719in}%
\pgfsys@useobject{currentmarker}{}%
\end{pgfscope}%
\begin{pgfscope}%
\pgfsys@transformshift{1.133992in}{1.479087in}%
\pgfsys@useobject{currentmarker}{}%
\end{pgfscope}%
\begin{pgfscope}%
\pgfsys@transformshift{1.480556in}{1.250971in}%
\pgfsys@useobject{currentmarker}{}%
\end{pgfscope}%
\begin{pgfscope}%
\pgfsys@transformshift{1.339321in}{1.303805in}%
\pgfsys@useobject{currentmarker}{}%
\end{pgfscope}%
\begin{pgfscope}%
\pgfsys@transformshift{0.874156in}{1.305615in}%
\pgfsys@useobject{currentmarker}{}%
\end{pgfscope}%
\begin{pgfscope}%
\pgfsys@transformshift{1.592077in}{1.031449in}%
\pgfsys@useobject{currentmarker}{}%
\end{pgfscope}%
\begin{pgfscope}%
\pgfsys@transformshift{1.288980in}{1.339750in}%
\pgfsys@useobject{currentmarker}{}%
\end{pgfscope}%
\begin{pgfscope}%
\pgfsys@transformshift{0.987975in}{1.169121in}%
\pgfsys@useobject{currentmarker}{}%
\end{pgfscope}%
\begin{pgfscope}%
\pgfsys@transformshift{1.714750in}{1.135600in}%
\pgfsys@useobject{currentmarker}{}%
\end{pgfscope}%
\begin{pgfscope}%
\pgfsys@transformshift{1.365090in}{1.187150in}%
\pgfsys@useobject{currentmarker}{}%
\end{pgfscope}%
\begin{pgfscope}%
\pgfsys@transformshift{1.930143in}{1.169545in}%
\pgfsys@useobject{currentmarker}{}%
\end{pgfscope}%
\begin{pgfscope}%
\pgfsys@transformshift{1.850728in}{1.027789in}%
\pgfsys@useobject{currentmarker}{}%
\end{pgfscope}%
\begin{pgfscope}%
\pgfsys@transformshift{0.856167in}{1.440806in}%
\pgfsys@useobject{currentmarker}{}%
\end{pgfscope}%
\begin{pgfscope}%
\pgfsys@transformshift{1.724113in}{1.212076in}%
\pgfsys@useobject{currentmarker}{}%
\end{pgfscope}%
\begin{pgfscope}%
\pgfsys@transformshift{0.746656in}{1.488022in}%
\pgfsys@useobject{currentmarker}{}%
\end{pgfscope}%
\begin{pgfscope}%
\pgfsys@transformshift{0.981125in}{1.350204in}%
\pgfsys@useobject{currentmarker}{}%
\end{pgfscope}%
\begin{pgfscope}%
\pgfsys@transformshift{0.840360in}{1.333470in}%
\pgfsys@useobject{currentmarker}{}%
\end{pgfscope}%
\begin{pgfscope}%
\pgfsys@transformshift{0.780491in}{1.378040in}%
\pgfsys@useobject{currentmarker}{}%
\end{pgfscope}%
\begin{pgfscope}%
\pgfsys@transformshift{1.873816in}{1.170039in}%
\pgfsys@useobject{currentmarker}{}%
\end{pgfscope}%
\begin{pgfscope}%
\pgfsys@transformshift{1.594601in}{1.247120in}%
\pgfsys@useobject{currentmarker}{}%
\end{pgfscope}%
\begin{pgfscope}%
\pgfsys@transformshift{0.815736in}{1.416962in}%
\pgfsys@useobject{currentmarker}{}%
\end{pgfscope}%
\begin{pgfscope}%
\pgfsys@transformshift{1.864796in}{0.897817in}%
\pgfsys@useobject{currentmarker}{}%
\end{pgfscope}%
\begin{pgfscope}%
\pgfsys@transformshift{1.367631in}{1.436080in}%
\pgfsys@useobject{currentmarker}{}%
\end{pgfscope}%
\begin{pgfscope}%
\pgfsys@transformshift{1.653345in}{1.268414in}%
\pgfsys@useobject{currentmarker}{}%
\end{pgfscope}%
\begin{pgfscope}%
\pgfsys@transformshift{1.768545in}{1.085050in}%
\pgfsys@useobject{currentmarker}{}%
\end{pgfscope}%
\begin{pgfscope}%
\pgfsys@transformshift{1.244730in}{1.250945in}%
\pgfsys@useobject{currentmarker}{}%
\end{pgfscope}%
\begin{pgfscope}%
\pgfsys@transformshift{0.782047in}{1.472184in}%
\pgfsys@useobject{currentmarker}{}%
\end{pgfscope}%
\begin{pgfscope}%
\pgfsys@transformshift{1.846677in}{1.211480in}%
\pgfsys@useobject{currentmarker}{}%
\end{pgfscope}%
\begin{pgfscope}%
\pgfsys@transformshift{1.775739in}{1.081422in}%
\pgfsys@useobject{currentmarker}{}%
\end{pgfscope}%
\begin{pgfscope}%
\pgfsys@transformshift{1.703721in}{1.126552in}%
\pgfsys@useobject{currentmarker}{}%
\end{pgfscope}%
\begin{pgfscope}%
\pgfsys@transformshift{0.834121in}{1.403531in}%
\pgfsys@useobject{currentmarker}{}%
\end{pgfscope}%
\begin{pgfscope}%
\pgfsys@transformshift{1.119982in}{1.550746in}%
\pgfsys@useobject{currentmarker}{}%
\end{pgfscope}%
\begin{pgfscope}%
\pgfsys@transformshift{1.891923in}{1.116839in}%
\pgfsys@useobject{currentmarker}{}%
\end{pgfscope}%
\begin{pgfscope}%
\pgfsys@transformshift{1.690465in}{0.963250in}%
\pgfsys@useobject{currentmarker}{}%
\end{pgfscope}%
\begin{pgfscope}%
\pgfsys@transformshift{0.559984in}{1.458725in}%
\pgfsys@useobject{currentmarker}{}%
\end{pgfscope}%
\begin{pgfscope}%
\pgfsys@transformshift{1.447613in}{1.229769in}%
\pgfsys@useobject{currentmarker}{}%
\end{pgfscope}%
\begin{pgfscope}%
\pgfsys@transformshift{0.897928in}{1.574831in}%
\pgfsys@useobject{currentmarker}{}%
\end{pgfscope}%
\begin{pgfscope}%
\pgfsys@transformshift{1.417775in}{1.320600in}%
\pgfsys@useobject{currentmarker}{}%
\end{pgfscope}%
\begin{pgfscope}%
\pgfsys@transformshift{1.673252in}{1.042247in}%
\pgfsys@useobject{currentmarker}{}%
\end{pgfscope}%
\begin{pgfscope}%
\pgfsys@transformshift{1.412143in}{1.064058in}%
\pgfsys@useobject{currentmarker}{}%
\end{pgfscope}%
\begin{pgfscope}%
\pgfsys@transformshift{1.641912in}{1.184051in}%
\pgfsys@useobject{currentmarker}{}%
\end{pgfscope}%
\begin{pgfscope}%
\pgfsys@transformshift{1.388318in}{1.430971in}%
\pgfsys@useobject{currentmarker}{}%
\end{pgfscope}%
\begin{pgfscope}%
\pgfsys@transformshift{1.495487in}{1.165754in}%
\pgfsys@useobject{currentmarker}{}%
\end{pgfscope}%
\begin{pgfscope}%
\pgfsys@transformshift{1.918935in}{0.953298in}%
\pgfsys@useobject{currentmarker}{}%
\end{pgfscope}%
\begin{pgfscope}%
\pgfsys@transformshift{0.550025in}{1.402013in}%
\pgfsys@useobject{currentmarker}{}%
\end{pgfscope}%
\begin{pgfscope}%
\pgfsys@transformshift{1.821318in}{1.045586in}%
\pgfsys@useobject{currentmarker}{}%
\end{pgfscope}%
\begin{pgfscope}%
\pgfsys@transformshift{0.956955in}{1.331980in}%
\pgfsys@useobject{currentmarker}{}%
\end{pgfscope}%
\begin{pgfscope}%
\pgfsys@transformshift{0.895984in}{1.441183in}%
\pgfsys@useobject{currentmarker}{}%
\end{pgfscope}%
\begin{pgfscope}%
\pgfsys@transformshift{0.467116in}{1.154061in}%
\pgfsys@useobject{currentmarker}{}%
\end{pgfscope}%
\begin{pgfscope}%
\pgfsys@transformshift{1.606478in}{1.273955in}%
\pgfsys@useobject{currentmarker}{}%
\end{pgfscope}%
\begin{pgfscope}%
\pgfsys@transformshift{1.423682in}{1.300638in}%
\pgfsys@useobject{currentmarker}{}%
\end{pgfscope}%
\begin{pgfscope}%
\pgfsys@transformshift{1.436312in}{1.496310in}%
\pgfsys@useobject{currentmarker}{}%
\end{pgfscope}%
\begin{pgfscope}%
\pgfsys@transformshift{0.970257in}{1.581264in}%
\pgfsys@useobject{currentmarker}{}%
\end{pgfscope}%
\begin{pgfscope}%
\pgfsys@transformshift{1.482495in}{1.270482in}%
\pgfsys@useobject{currentmarker}{}%
\end{pgfscope}%
\begin{pgfscope}%
\pgfsys@transformshift{1.363658in}{1.344054in}%
\pgfsys@useobject{currentmarker}{}%
\end{pgfscope}%
\begin{pgfscope}%
\pgfsys@transformshift{1.566350in}{1.323415in}%
\pgfsys@useobject{currentmarker}{}%
\end{pgfscope}%
\begin{pgfscope}%
\pgfsys@transformshift{0.737168in}{1.390621in}%
\pgfsys@useobject{currentmarker}{}%
\end{pgfscope}%
\begin{pgfscope}%
\pgfsys@transformshift{0.416424in}{1.141139in}%
\pgfsys@useobject{currentmarker}{}%
\end{pgfscope}%
\begin{pgfscope}%
\pgfsys@transformshift{0.524334in}{1.410391in}%
\pgfsys@useobject{currentmarker}{}%
\end{pgfscope}%
\begin{pgfscope}%
\pgfsys@transformshift{0.739043in}{1.129262in}%
\pgfsys@useobject{currentmarker}{}%
\end{pgfscope}%
\begin{pgfscope}%
\pgfsys@transformshift{1.840100in}{1.135351in}%
\pgfsys@useobject{currentmarker}{}%
\end{pgfscope}%
\begin{pgfscope}%
\pgfsys@transformshift{1.824568in}{1.214543in}%
\pgfsys@useobject{currentmarker}{}%
\end{pgfscope}%
\begin{pgfscope}%
\pgfsys@transformshift{1.470009in}{1.352075in}%
\pgfsys@useobject{currentmarker}{}%
\end{pgfscope}%
\begin{pgfscope}%
\pgfsys@transformshift{0.471982in}{1.391172in}%
\pgfsys@useobject{currentmarker}{}%
\end{pgfscope}%
\begin{pgfscope}%
\pgfsys@transformshift{1.573451in}{1.576385in}%
\pgfsys@useobject{currentmarker}{}%
\end{pgfscope}%
\begin{pgfscope}%
\pgfsys@transformshift{1.738247in}{1.071753in}%
\pgfsys@useobject{currentmarker}{}%
\end{pgfscope}%
\begin{pgfscope}%
\pgfsys@transformshift{1.607428in}{1.077201in}%
\pgfsys@useobject{currentmarker}{}%
\end{pgfscope}%
\begin{pgfscope}%
\pgfsys@transformshift{1.759927in}{0.919237in}%
\pgfsys@useobject{currentmarker}{}%
\end{pgfscope}%
\begin{pgfscope}%
\pgfsys@transformshift{1.551624in}{1.155289in}%
\pgfsys@useobject{currentmarker}{}%
\end{pgfscope}%
\begin{pgfscope}%
\pgfsys@transformshift{1.356607in}{1.302890in}%
\pgfsys@useobject{currentmarker}{}%
\end{pgfscope}%
\begin{pgfscope}%
\pgfsys@transformshift{1.649686in}{0.892849in}%
\pgfsys@useobject{currentmarker}{}%
\end{pgfscope}%
\begin{pgfscope}%
\pgfsys@transformshift{1.305691in}{1.200747in}%
\pgfsys@useobject{currentmarker}{}%
\end{pgfscope}%
\begin{pgfscope}%
\pgfsys@transformshift{1.393587in}{1.082301in}%
\pgfsys@useobject{currentmarker}{}%
\end{pgfscope}%
\begin{pgfscope}%
\pgfsys@transformshift{1.153074in}{1.134772in}%
\pgfsys@useobject{currentmarker}{}%
\end{pgfscope}%
\begin{pgfscope}%
\pgfsys@transformshift{1.150217in}{1.457841in}%
\pgfsys@useobject{currentmarker}{}%
\end{pgfscope}%
\begin{pgfscope}%
\pgfsys@transformshift{0.541563in}{1.009026in}%
\pgfsys@useobject{currentmarker}{}%
\end{pgfscope}%
\begin{pgfscope}%
\pgfsys@transformshift{0.907947in}{1.386013in}%
\pgfsys@useobject{currentmarker}{}%
\end{pgfscope}%
\begin{pgfscope}%
\pgfsys@transformshift{1.404883in}{1.247239in}%
\pgfsys@useobject{currentmarker}{}%
\end{pgfscope}%
\begin{pgfscope}%
\pgfsys@transformshift{1.766444in}{1.083764in}%
\pgfsys@useobject{currentmarker}{}%
\end{pgfscope}%
\begin{pgfscope}%
\pgfsys@transformshift{1.727091in}{1.220013in}%
\pgfsys@useobject{currentmarker}{}%
\end{pgfscope}%
\begin{pgfscope}%
\pgfsys@transformshift{1.794191in}{1.233616in}%
\pgfsys@useobject{currentmarker}{}%
\end{pgfscope}%
\begin{pgfscope}%
\pgfsys@transformshift{1.006287in}{1.356762in}%
\pgfsys@useobject{currentmarker}{}%
\end{pgfscope}%
\begin{pgfscope}%
\pgfsys@transformshift{1.581569in}{1.193311in}%
\pgfsys@useobject{currentmarker}{}%
\end{pgfscope}%
\begin{pgfscope}%
\pgfsys@transformshift{0.941329in}{1.369494in}%
\pgfsys@useobject{currentmarker}{}%
\end{pgfscope}%
\begin{pgfscope}%
\pgfsys@transformshift{0.691952in}{1.529486in}%
\pgfsys@useobject{currentmarker}{}%
\end{pgfscope}%
\begin{pgfscope}%
\pgfsys@transformshift{1.057616in}{1.484361in}%
\pgfsys@useobject{currentmarker}{}%
\end{pgfscope}%
\begin{pgfscope}%
\pgfsys@transformshift{1.673517in}{1.295847in}%
\pgfsys@useobject{currentmarker}{}%
\end{pgfscope}%
\begin{pgfscope}%
\pgfsys@transformshift{1.477276in}{1.379425in}%
\pgfsys@useobject{currentmarker}{}%
\end{pgfscope}%
\begin{pgfscope}%
\pgfsys@transformshift{1.418685in}{1.261890in}%
\pgfsys@useobject{currentmarker}{}%
\end{pgfscope}%
\begin{pgfscope}%
\pgfsys@transformshift{1.590325in}{1.060657in}%
\pgfsys@useobject{currentmarker}{}%
\end{pgfscope}%
\begin{pgfscope}%
\pgfsys@transformshift{0.779671in}{1.437509in}%
\pgfsys@useobject{currentmarker}{}%
\end{pgfscope}%
\begin{pgfscope}%
\pgfsys@transformshift{1.674297in}{1.127368in}%
\pgfsys@useobject{currentmarker}{}%
\end{pgfscope}%
\begin{pgfscope}%
\pgfsys@transformshift{0.676244in}{1.430452in}%
\pgfsys@useobject{currentmarker}{}%
\end{pgfscope}%
\begin{pgfscope}%
\pgfsys@transformshift{1.715152in}{1.036652in}%
\pgfsys@useobject{currentmarker}{}%
\end{pgfscope}%
\begin{pgfscope}%
\pgfsys@transformshift{0.491079in}{1.112657in}%
\pgfsys@useobject{currentmarker}{}%
\end{pgfscope}%
\begin{pgfscope}%
\pgfsys@transformshift{1.084356in}{1.484373in}%
\pgfsys@useobject{currentmarker}{}%
\end{pgfscope}%
\begin{pgfscope}%
\pgfsys@transformshift{0.525316in}{1.110510in}%
\pgfsys@useobject{currentmarker}{}%
\end{pgfscope}%
\begin{pgfscope}%
\pgfsys@transformshift{0.530939in}{1.130040in}%
\pgfsys@useobject{currentmarker}{}%
\end{pgfscope}%
\begin{pgfscope}%
\pgfsys@transformshift{0.632029in}{1.705700in}%
\pgfsys@useobject{currentmarker}{}%
\end{pgfscope}%
\begin{pgfscope}%
\pgfsys@transformshift{0.952115in}{1.542357in}%
\pgfsys@useobject{currentmarker}{}%
\end{pgfscope}%
\begin{pgfscope}%
\pgfsys@transformshift{1.781361in}{0.930793in}%
\pgfsys@useobject{currentmarker}{}%
\end{pgfscope}%
\begin{pgfscope}%
\pgfsys@transformshift{0.893435in}{1.351484in}%
\pgfsys@useobject{currentmarker}{}%
\end{pgfscope}%
\begin{pgfscope}%
\pgfsys@transformshift{1.630080in}{1.162555in}%
\pgfsys@useobject{currentmarker}{}%
\end{pgfscope}%
\begin{pgfscope}%
\pgfsys@transformshift{1.495001in}{1.255506in}%
\pgfsys@useobject{currentmarker}{}%
\end{pgfscope}%
\begin{pgfscope}%
\pgfsys@transformshift{1.807197in}{1.225830in}%
\pgfsys@useobject{currentmarker}{}%
\end{pgfscope}%
\begin{pgfscope}%
\pgfsys@transformshift{0.872166in}{1.393384in}%
\pgfsys@useobject{currentmarker}{}%
\end{pgfscope}%
\begin{pgfscope}%
\pgfsys@transformshift{1.447090in}{1.395996in}%
\pgfsys@useobject{currentmarker}{}%
\end{pgfscope}%
\begin{pgfscope}%
\pgfsys@transformshift{1.534740in}{1.288212in}%
\pgfsys@useobject{currentmarker}{}%
\end{pgfscope}%
\begin{pgfscope}%
\pgfsys@transformshift{0.648228in}{1.800615in}%
\pgfsys@useobject{currentmarker}{}%
\end{pgfscope}%
\begin{pgfscope}%
\pgfsys@transformshift{0.939750in}{1.131333in}%
\pgfsys@useobject{currentmarker}{}%
\end{pgfscope}%
\begin{pgfscope}%
\pgfsys@transformshift{0.437103in}{1.156900in}%
\pgfsys@useobject{currentmarker}{}%
\end{pgfscope}%
\begin{pgfscope}%
\pgfsys@transformshift{1.534965in}{1.240450in}%
\pgfsys@useobject{currentmarker}{}%
\end{pgfscope}%
\begin{pgfscope}%
\pgfsys@transformshift{1.227246in}{1.412479in}%
\pgfsys@useobject{currentmarker}{}%
\end{pgfscope}%
\begin{pgfscope}%
\pgfsys@transformshift{1.816498in}{1.048332in}%
\pgfsys@useobject{currentmarker}{}%
\end{pgfscope}%
\begin{pgfscope}%
\pgfsys@transformshift{0.733769in}{1.549828in}%
\pgfsys@useobject{currentmarker}{}%
\end{pgfscope}%
\begin{pgfscope}%
\pgfsys@transformshift{0.664484in}{1.359412in}%
\pgfsys@useobject{currentmarker}{}%
\end{pgfscope}%
\begin{pgfscope}%
\pgfsys@transformshift{1.076644in}{1.441379in}%
\pgfsys@useobject{currentmarker}{}%
\end{pgfscope}%
\begin{pgfscope}%
\pgfsys@transformshift{1.548521in}{1.328383in}%
\pgfsys@useobject{currentmarker}{}%
\end{pgfscope}%
\begin{pgfscope}%
\pgfsys@transformshift{0.536243in}{1.604214in}%
\pgfsys@useobject{currentmarker}{}%
\end{pgfscope}%
\begin{pgfscope}%
\pgfsys@transformshift{0.889736in}{1.425142in}%
\pgfsys@useobject{currentmarker}{}%
\end{pgfscope}%
\begin{pgfscope}%
\pgfsys@transformshift{1.818644in}{1.119905in}%
\pgfsys@useobject{currentmarker}{}%
\end{pgfscope}%
\begin{pgfscope}%
\pgfsys@transformshift{1.073205in}{1.497259in}%
\pgfsys@useobject{currentmarker}{}%
\end{pgfscope}%
\begin{pgfscope}%
\pgfsys@transformshift{0.882269in}{1.578339in}%
\pgfsys@useobject{currentmarker}{}%
\end{pgfscope}%
\begin{pgfscope}%
\pgfsys@transformshift{0.755753in}{1.476984in}%
\pgfsys@useobject{currentmarker}{}%
\end{pgfscope}%
\begin{pgfscope}%
\pgfsys@transformshift{1.318495in}{1.325724in}%
\pgfsys@useobject{currentmarker}{}%
\end{pgfscope}%
\begin{pgfscope}%
\pgfsys@transformshift{1.102293in}{1.264045in}%
\pgfsys@useobject{currentmarker}{}%
\end{pgfscope}%
\begin{pgfscope}%
\pgfsys@transformshift{1.814698in}{1.303825in}%
\pgfsys@useobject{currentmarker}{}%
\end{pgfscope}%
\begin{pgfscope}%
\pgfsys@transformshift{1.595111in}{1.095996in}%
\pgfsys@useobject{currentmarker}{}%
\end{pgfscope}%
\begin{pgfscope}%
\pgfsys@transformshift{0.726840in}{1.323961in}%
\pgfsys@useobject{currentmarker}{}%
\end{pgfscope}%
\begin{pgfscope}%
\pgfsys@transformshift{1.492602in}{1.239647in}%
\pgfsys@useobject{currentmarker}{}%
\end{pgfscope}%
\begin{pgfscope}%
\pgfsys@transformshift{1.393179in}{1.107225in}%
\pgfsys@useobject{currentmarker}{}%
\end{pgfscope}%
\begin{pgfscope}%
\pgfsys@transformshift{1.166943in}{1.100291in}%
\pgfsys@useobject{currentmarker}{}%
\end{pgfscope}%
\begin{pgfscope}%
\pgfsys@transformshift{1.444452in}{1.102691in}%
\pgfsys@useobject{currentmarker}{}%
\end{pgfscope}%
\begin{pgfscope}%
\pgfsys@transformshift{1.693005in}{1.251831in}%
\pgfsys@useobject{currentmarker}{}%
\end{pgfscope}%
\begin{pgfscope}%
\pgfsys@transformshift{0.817674in}{1.207898in}%
\pgfsys@useobject{currentmarker}{}%
\end{pgfscope}%
\begin{pgfscope}%
\pgfsys@transformshift{0.521183in}{1.688427in}%
\pgfsys@useobject{currentmarker}{}%
\end{pgfscope}%
\begin{pgfscope}%
\pgfsys@transformshift{1.298494in}{1.404075in}%
\pgfsys@useobject{currentmarker}{}%
\end{pgfscope}%
\begin{pgfscope}%
\pgfsys@transformshift{0.537740in}{1.484413in}%
\pgfsys@useobject{currentmarker}{}%
\end{pgfscope}%
\begin{pgfscope}%
\pgfsys@transformshift{1.711514in}{1.333377in}%
\pgfsys@useobject{currentmarker}{}%
\end{pgfscope}%
\begin{pgfscope}%
\pgfsys@transformshift{1.607575in}{1.316032in}%
\pgfsys@useobject{currentmarker}{}%
\end{pgfscope}%
\begin{pgfscope}%
\pgfsys@transformshift{1.569340in}{1.262109in}%
\pgfsys@useobject{currentmarker}{}%
\end{pgfscope}%
\begin{pgfscope}%
\pgfsys@transformshift{0.423151in}{1.500296in}%
\pgfsys@useobject{currentmarker}{}%
\end{pgfscope}%
\begin{pgfscope}%
\pgfsys@transformshift{0.636198in}{1.013319in}%
\pgfsys@useobject{currentmarker}{}%
\end{pgfscope}%
\begin{pgfscope}%
\pgfsys@transformshift{1.742823in}{0.976090in}%
\pgfsys@useobject{currentmarker}{}%
\end{pgfscope}%
\begin{pgfscope}%
\pgfsys@transformshift{0.849878in}{1.369284in}%
\pgfsys@useobject{currentmarker}{}%
\end{pgfscope}%
\begin{pgfscope}%
\pgfsys@transformshift{1.353468in}{1.338992in}%
\pgfsys@useobject{currentmarker}{}%
\end{pgfscope}%
\begin{pgfscope}%
\pgfsys@transformshift{1.622982in}{0.848914in}%
\pgfsys@useobject{currentmarker}{}%
\end{pgfscope}%
\begin{pgfscope}%
\pgfsys@transformshift{1.599783in}{0.931032in}%
\pgfsys@useobject{currentmarker}{}%
\end{pgfscope}%
\begin{pgfscope}%
\pgfsys@transformshift{1.895187in}{1.010774in}%
\pgfsys@useobject{currentmarker}{}%
\end{pgfscope}%
\begin{pgfscope}%
\pgfsys@transformshift{1.129530in}{1.456998in}%
\pgfsys@useobject{currentmarker}{}%
\end{pgfscope}%
\begin{pgfscope}%
\pgfsys@transformshift{0.991382in}{1.283228in}%
\pgfsys@useobject{currentmarker}{}%
\end{pgfscope}%
\begin{pgfscope}%
\pgfsys@transformshift{0.795664in}{1.797052in}%
\pgfsys@useobject{currentmarker}{}%
\end{pgfscope}%
\begin{pgfscope}%
\pgfsys@transformshift{1.780025in}{1.192874in}%
\pgfsys@useobject{currentmarker}{}%
\end{pgfscope}%
\begin{pgfscope}%
\pgfsys@transformshift{1.712749in}{1.171973in}%
\pgfsys@useobject{currentmarker}{}%
\end{pgfscope}%
\begin{pgfscope}%
\pgfsys@transformshift{1.145211in}{1.407525in}%
\pgfsys@useobject{currentmarker}{}%
\end{pgfscope}%
\begin{pgfscope}%
\pgfsys@transformshift{1.328349in}{1.189702in}%
\pgfsys@useobject{currentmarker}{}%
\end{pgfscope}%
\begin{pgfscope}%
\pgfsys@transformshift{1.405073in}{1.236292in}%
\pgfsys@useobject{currentmarker}{}%
\end{pgfscope}%
\begin{pgfscope}%
\pgfsys@transformshift{0.669867in}{1.507604in}%
\pgfsys@useobject{currentmarker}{}%
\end{pgfscope}%
\begin{pgfscope}%
\pgfsys@transformshift{1.898744in}{0.993421in}%
\pgfsys@useobject{currentmarker}{}%
\end{pgfscope}%
\begin{pgfscope}%
\pgfsys@transformshift{0.874940in}{1.352737in}%
\pgfsys@useobject{currentmarker}{}%
\end{pgfscope}%
\begin{pgfscope}%
\pgfsys@transformshift{1.459937in}{1.641709in}%
\pgfsys@useobject{currentmarker}{}%
\end{pgfscope}%
\begin{pgfscope}%
\pgfsys@transformshift{1.895928in}{0.976271in}%
\pgfsys@useobject{currentmarker}{}%
\end{pgfscope}%
\begin{pgfscope}%
\pgfsys@transformshift{0.853418in}{1.368911in}%
\pgfsys@useobject{currentmarker}{}%
\end{pgfscope}%
\begin{pgfscope}%
\pgfsys@transformshift{0.717208in}{1.431722in}%
\pgfsys@useobject{currentmarker}{}%
\end{pgfscope}%
\begin{pgfscope}%
\pgfsys@transformshift{1.075298in}{1.321096in}%
\pgfsys@useobject{currentmarker}{}%
\end{pgfscope}%
\begin{pgfscope}%
\pgfsys@transformshift{1.920218in}{1.337550in}%
\pgfsys@useobject{currentmarker}{}%
\end{pgfscope}%
\begin{pgfscope}%
\pgfsys@transformshift{0.377829in}{1.434797in}%
\pgfsys@useobject{currentmarker}{}%
\end{pgfscope}%
\begin{pgfscope}%
\pgfsys@transformshift{1.155745in}{1.520920in}%
\pgfsys@useobject{currentmarker}{}%
\end{pgfscope}%
\begin{pgfscope}%
\pgfsys@transformshift{1.712524in}{1.016039in}%
\pgfsys@useobject{currentmarker}{}%
\end{pgfscope}%
\begin{pgfscope}%
\pgfsys@transformshift{0.534596in}{1.500797in}%
\pgfsys@useobject{currentmarker}{}%
\end{pgfscope}%
\begin{pgfscope}%
\pgfsys@transformshift{0.622741in}{1.582416in}%
\pgfsys@useobject{currentmarker}{}%
\end{pgfscope}%
\begin{pgfscope}%
\pgfsys@transformshift{0.575527in}{1.694059in}%
\pgfsys@useobject{currentmarker}{}%
\end{pgfscope}%
\begin{pgfscope}%
\pgfsys@transformshift{1.804863in}{1.005039in}%
\pgfsys@useobject{currentmarker}{}%
\end{pgfscope}%
\begin{pgfscope}%
\pgfsys@transformshift{1.749745in}{0.946409in}%
\pgfsys@useobject{currentmarker}{}%
\end{pgfscope}%
\begin{pgfscope}%
\pgfsys@transformshift{1.674396in}{1.188555in}%
\pgfsys@useobject{currentmarker}{}%
\end{pgfscope}%
\begin{pgfscope}%
\pgfsys@transformshift{1.859027in}{0.954110in}%
\pgfsys@useobject{currentmarker}{}%
\end{pgfscope}%
\begin{pgfscope}%
\pgfsys@transformshift{1.603827in}{1.014159in}%
\pgfsys@useobject{currentmarker}{}%
\end{pgfscope}%
\begin{pgfscope}%
\pgfsys@transformshift{1.255917in}{1.362886in}%
\pgfsys@useobject{currentmarker}{}%
\end{pgfscope}%
\begin{pgfscope}%
\pgfsys@transformshift{0.419746in}{1.141644in}%
\pgfsys@useobject{currentmarker}{}%
\end{pgfscope}%
\begin{pgfscope}%
\pgfsys@transformshift{0.903589in}{1.109522in}%
\pgfsys@useobject{currentmarker}{}%
\end{pgfscope}%
\begin{pgfscope}%
\pgfsys@transformshift{1.411837in}{1.322066in}%
\pgfsys@useobject{currentmarker}{}%
\end{pgfscope}%
\begin{pgfscope}%
\pgfsys@transformshift{1.888179in}{0.981671in}%
\pgfsys@useobject{currentmarker}{}%
\end{pgfscope}%
\begin{pgfscope}%
\pgfsys@transformshift{1.463948in}{1.207746in}%
\pgfsys@useobject{currentmarker}{}%
\end{pgfscope}%
\begin{pgfscope}%
\pgfsys@transformshift{1.037147in}{1.261818in}%
\pgfsys@useobject{currentmarker}{}%
\end{pgfscope}%
\begin{pgfscope}%
\pgfsys@transformshift{1.744144in}{0.966337in}%
\pgfsys@useobject{currentmarker}{}%
\end{pgfscope}%
\begin{pgfscope}%
\pgfsys@transformshift{0.867355in}{1.271159in}%
\pgfsys@useobject{currentmarker}{}%
\end{pgfscope}%
\begin{pgfscope}%
\pgfsys@transformshift{1.331798in}{1.361312in}%
\pgfsys@useobject{currentmarker}{}%
\end{pgfscope}%
\begin{pgfscope}%
\pgfsys@transformshift{0.968745in}{1.410445in}%
\pgfsys@useobject{currentmarker}{}%
\end{pgfscope}%
\begin{pgfscope}%
\pgfsys@transformshift{1.193049in}{1.535659in}%
\pgfsys@useobject{currentmarker}{}%
\end{pgfscope}%
\begin{pgfscope}%
\pgfsys@transformshift{1.664246in}{1.067367in}%
\pgfsys@useobject{currentmarker}{}%
\end{pgfscope}%
\begin{pgfscope}%
\pgfsys@transformshift{0.835587in}{1.425613in}%
\pgfsys@useobject{currentmarker}{}%
\end{pgfscope}%
\begin{pgfscope}%
\pgfsys@transformshift{1.701974in}{0.961714in}%
\pgfsys@useobject{currentmarker}{}%
\end{pgfscope}%
\begin{pgfscope}%
\pgfsys@transformshift{1.269033in}{1.323590in}%
\pgfsys@useobject{currentmarker}{}%
\end{pgfscope}%
\begin{pgfscope}%
\pgfsys@transformshift{1.329352in}{1.203550in}%
\pgfsys@useobject{currentmarker}{}%
\end{pgfscope}%
\begin{pgfscope}%
\pgfsys@transformshift{1.471661in}{1.414248in}%
\pgfsys@useobject{currentmarker}{}%
\end{pgfscope}%
\begin{pgfscope}%
\pgfsys@transformshift{1.916157in}{1.291414in}%
\pgfsys@useobject{currentmarker}{}%
\end{pgfscope}%
\begin{pgfscope}%
\pgfsys@transformshift{1.533379in}{1.049759in}%
\pgfsys@useobject{currentmarker}{}%
\end{pgfscope}%
\begin{pgfscope}%
\pgfsys@transformshift{1.358583in}{1.222970in}%
\pgfsys@useobject{currentmarker}{}%
\end{pgfscope}%
\begin{pgfscope}%
\pgfsys@transformshift{1.599245in}{1.105818in}%
\pgfsys@useobject{currentmarker}{}%
\end{pgfscope}%
\begin{pgfscope}%
\pgfsys@transformshift{0.703172in}{1.466013in}%
\pgfsys@useobject{currentmarker}{}%
\end{pgfscope}%
\begin{pgfscope}%
\pgfsys@transformshift{1.320412in}{1.307948in}%
\pgfsys@useobject{currentmarker}{}%
\end{pgfscope}%
\begin{pgfscope}%
\pgfsys@transformshift{1.136406in}{1.340190in}%
\pgfsys@useobject{currentmarker}{}%
\end{pgfscope}%
\begin{pgfscope}%
\pgfsys@transformshift{0.721527in}{1.588117in}%
\pgfsys@useobject{currentmarker}{}%
\end{pgfscope}%
\begin{pgfscope}%
\pgfsys@transformshift{1.722543in}{0.903142in}%
\pgfsys@useobject{currentmarker}{}%
\end{pgfscope}%
\begin{pgfscope}%
\pgfsys@transformshift{1.835288in}{1.168785in}%
\pgfsys@useobject{currentmarker}{}%
\end{pgfscope}%
\begin{pgfscope}%
\pgfsys@transformshift{1.984248in}{1.263776in}%
\pgfsys@useobject{currentmarker}{}%
\end{pgfscope}%
\begin{pgfscope}%
\pgfsys@transformshift{1.658536in}{1.252565in}%
\pgfsys@useobject{currentmarker}{}%
\end{pgfscope}%
\begin{pgfscope}%
\pgfsys@transformshift{0.561607in}{1.490866in}%
\pgfsys@useobject{currentmarker}{}%
\end{pgfscope}%
\begin{pgfscope}%
\pgfsys@transformshift{1.906539in}{1.171810in}%
\pgfsys@useobject{currentmarker}{}%
\end{pgfscope}%
\begin{pgfscope}%
\pgfsys@transformshift{1.521333in}{1.273560in}%
\pgfsys@useobject{currentmarker}{}%
\end{pgfscope}%
\begin{pgfscope}%
\pgfsys@transformshift{1.136396in}{1.385189in}%
\pgfsys@useobject{currentmarker}{}%
\end{pgfscope}%
\begin{pgfscope}%
\pgfsys@transformshift{1.435818in}{1.059888in}%
\pgfsys@useobject{currentmarker}{}%
\end{pgfscope}%
\begin{pgfscope}%
\pgfsys@transformshift{1.643271in}{1.031859in}%
\pgfsys@useobject{currentmarker}{}%
\end{pgfscope}%
\begin{pgfscope}%
\pgfsys@transformshift{1.445064in}{1.180613in}%
\pgfsys@useobject{currentmarker}{}%
\end{pgfscope}%
\begin{pgfscope}%
\pgfsys@transformshift{0.658007in}{1.294459in}%
\pgfsys@useobject{currentmarker}{}%
\end{pgfscope}%
\begin{pgfscope}%
\pgfsys@transformshift{0.786972in}{1.376634in}%
\pgfsys@useobject{currentmarker}{}%
\end{pgfscope}%
\begin{pgfscope}%
\pgfsys@transformshift{1.102575in}{1.361664in}%
\pgfsys@useobject{currentmarker}{}%
\end{pgfscope}%
\begin{pgfscope}%
\pgfsys@transformshift{0.532598in}{1.531928in}%
\pgfsys@useobject{currentmarker}{}%
\end{pgfscope}%
\begin{pgfscope}%
\pgfsys@transformshift{0.357574in}{1.586640in}%
\pgfsys@useobject{currentmarker}{}%
\end{pgfscope}%
\begin{pgfscope}%
\pgfsys@transformshift{1.683669in}{1.130497in}%
\pgfsys@useobject{currentmarker}{}%
\end{pgfscope}%
\begin{pgfscope}%
\pgfsys@transformshift{1.612445in}{1.216600in}%
\pgfsys@useobject{currentmarker}{}%
\end{pgfscope}%
\begin{pgfscope}%
\pgfsys@transformshift{1.620066in}{1.056461in}%
\pgfsys@useobject{currentmarker}{}%
\end{pgfscope}%
\begin{pgfscope}%
\pgfsys@transformshift{0.680343in}{1.137978in}%
\pgfsys@useobject{currentmarker}{}%
\end{pgfscope}%
\begin{pgfscope}%
\pgfsys@transformshift{0.675258in}{1.470723in}%
\pgfsys@useobject{currentmarker}{}%
\end{pgfscope}%
\begin{pgfscope}%
\pgfsys@transformshift{0.963927in}{1.318761in}%
\pgfsys@useobject{currentmarker}{}%
\end{pgfscope}%
\begin{pgfscope}%
\pgfsys@transformshift{1.684562in}{1.046531in}%
\pgfsys@useobject{currentmarker}{}%
\end{pgfscope}%
\begin{pgfscope}%
\pgfsys@transformshift{1.473245in}{1.391786in}%
\pgfsys@useobject{currentmarker}{}%
\end{pgfscope}%
\begin{pgfscope}%
\pgfsys@transformshift{1.020383in}{1.497243in}%
\pgfsys@useobject{currentmarker}{}%
\end{pgfscope}%
\begin{pgfscope}%
\pgfsys@transformshift{0.681597in}{1.618225in}%
\pgfsys@useobject{currentmarker}{}%
\end{pgfscope}%
\begin{pgfscope}%
\pgfsys@transformshift{1.730115in}{1.072802in}%
\pgfsys@useobject{currentmarker}{}%
\end{pgfscope}%
\begin{pgfscope}%
\pgfsys@transformshift{1.808933in}{1.039272in}%
\pgfsys@useobject{currentmarker}{}%
\end{pgfscope}%
\begin{pgfscope}%
\pgfsys@transformshift{1.251675in}{1.359445in}%
\pgfsys@useobject{currentmarker}{}%
\end{pgfscope}%
\begin{pgfscope}%
\pgfsys@transformshift{1.717946in}{0.886518in}%
\pgfsys@useobject{currentmarker}{}%
\end{pgfscope}%
\begin{pgfscope}%
\pgfsys@transformshift{1.597312in}{1.025738in}%
\pgfsys@useobject{currentmarker}{}%
\end{pgfscope}%
\begin{pgfscope}%
\pgfsys@transformshift{1.148136in}{1.513075in}%
\pgfsys@useobject{currentmarker}{}%
\end{pgfscope}%
\begin{pgfscope}%
\pgfsys@transformshift{0.720143in}{1.485507in}%
\pgfsys@useobject{currentmarker}{}%
\end{pgfscope}%
\begin{pgfscope}%
\pgfsys@transformshift{1.574970in}{1.316713in}%
\pgfsys@useobject{currentmarker}{}%
\end{pgfscope}%
\begin{pgfscope}%
\pgfsys@transformshift{0.869972in}{1.238657in}%
\pgfsys@useobject{currentmarker}{}%
\end{pgfscope}%
\begin{pgfscope}%
\pgfsys@transformshift{1.713193in}{1.032321in}%
\pgfsys@useobject{currentmarker}{}%
\end{pgfscope}%
\begin{pgfscope}%
\pgfsys@transformshift{1.673692in}{1.017183in}%
\pgfsys@useobject{currentmarker}{}%
\end{pgfscope}%
\begin{pgfscope}%
\pgfsys@transformshift{1.344861in}{1.186349in}%
\pgfsys@useobject{currentmarker}{}%
\end{pgfscope}%
\begin{pgfscope}%
\pgfsys@transformshift{1.182101in}{1.352872in}%
\pgfsys@useobject{currentmarker}{}%
\end{pgfscope}%
\begin{pgfscope}%
\pgfsys@transformshift{1.449668in}{1.290964in}%
\pgfsys@useobject{currentmarker}{}%
\end{pgfscope}%
\begin{pgfscope}%
\pgfsys@transformshift{1.561680in}{1.204087in}%
\pgfsys@useobject{currentmarker}{}%
\end{pgfscope}%
\begin{pgfscope}%
\pgfsys@transformshift{1.192063in}{1.255055in}%
\pgfsys@useobject{currentmarker}{}%
\end{pgfscope}%
\begin{pgfscope}%
\pgfsys@transformshift{1.264374in}{1.384474in}%
\pgfsys@useobject{currentmarker}{}%
\end{pgfscope}%
\begin{pgfscope}%
\pgfsys@transformshift{0.727375in}{1.618892in}%
\pgfsys@useobject{currentmarker}{}%
\end{pgfscope}%
\begin{pgfscope}%
\pgfsys@transformshift{0.687912in}{1.034231in}%
\pgfsys@useobject{currentmarker}{}%
\end{pgfscope}%
\begin{pgfscope}%
\pgfsys@transformshift{1.823596in}{1.169364in}%
\pgfsys@useobject{currentmarker}{}%
\end{pgfscope}%
\begin{pgfscope}%
\pgfsys@transformshift{1.420153in}{1.188755in}%
\pgfsys@useobject{currentmarker}{}%
\end{pgfscope}%
\begin{pgfscope}%
\pgfsys@transformshift{0.672265in}{1.678275in}%
\pgfsys@useobject{currentmarker}{}%
\end{pgfscope}%
\begin{pgfscope}%
\pgfsys@transformshift{0.574267in}{1.507720in}%
\pgfsys@useobject{currentmarker}{}%
\end{pgfscope}%
\begin{pgfscope}%
\pgfsys@transformshift{1.570297in}{1.136188in}%
\pgfsys@useobject{currentmarker}{}%
\end{pgfscope}%
\begin{pgfscope}%
\pgfsys@transformshift{1.650981in}{1.030561in}%
\pgfsys@useobject{currentmarker}{}%
\end{pgfscope}%
\begin{pgfscope}%
\pgfsys@transformshift{0.894282in}{1.362772in}%
\pgfsys@useobject{currentmarker}{}%
\end{pgfscope}%
\begin{pgfscope}%
\pgfsys@transformshift{1.157746in}{1.380789in}%
\pgfsys@useobject{currentmarker}{}%
\end{pgfscope}%
\begin{pgfscope}%
\pgfsys@transformshift{0.608353in}{1.591399in}%
\pgfsys@useobject{currentmarker}{}%
\end{pgfscope}%
\begin{pgfscope}%
\pgfsys@transformshift{1.631665in}{1.041077in}%
\pgfsys@useobject{currentmarker}{}%
\end{pgfscope}%
\begin{pgfscope}%
\pgfsys@transformshift{1.419963in}{1.175597in}%
\pgfsys@useobject{currentmarker}{}%
\end{pgfscope}%
\begin{pgfscope}%
\pgfsys@transformshift{1.180003in}{1.437014in}%
\pgfsys@useobject{currentmarker}{}%
\end{pgfscope}%
\begin{pgfscope}%
\pgfsys@transformshift{1.469789in}{1.406096in}%
\pgfsys@useobject{currentmarker}{}%
\end{pgfscope}%
\begin{pgfscope}%
\pgfsys@transformshift{1.493240in}{1.449350in}%
\pgfsys@useobject{currentmarker}{}%
\end{pgfscope}%
\begin{pgfscope}%
\pgfsys@transformshift{1.725815in}{1.085971in}%
\pgfsys@useobject{currentmarker}{}%
\end{pgfscope}%
\begin{pgfscope}%
\pgfsys@transformshift{1.002068in}{1.339966in}%
\pgfsys@useobject{currentmarker}{}%
\end{pgfscope}%
\begin{pgfscope}%
\pgfsys@transformshift{1.111774in}{1.448212in}%
\pgfsys@useobject{currentmarker}{}%
\end{pgfscope}%
\begin{pgfscope}%
\pgfsys@transformshift{0.361781in}{1.673053in}%
\pgfsys@useobject{currentmarker}{}%
\end{pgfscope}%
\begin{pgfscope}%
\pgfsys@transformshift{1.608152in}{1.221009in}%
\pgfsys@useobject{currentmarker}{}%
\end{pgfscope}%
\begin{pgfscope}%
\pgfsys@transformshift{1.468569in}{1.481908in}%
\pgfsys@useobject{currentmarker}{}%
\end{pgfscope}%
\begin{pgfscope}%
\pgfsys@transformshift{1.646778in}{1.010543in}%
\pgfsys@useobject{currentmarker}{}%
\end{pgfscope}%
\begin{pgfscope}%
\pgfsys@transformshift{1.356664in}{1.328855in}%
\pgfsys@useobject{currentmarker}{}%
\end{pgfscope}%
\begin{pgfscope}%
\pgfsys@transformshift{0.657110in}{1.079428in}%
\pgfsys@useobject{currentmarker}{}%
\end{pgfscope}%
\begin{pgfscope}%
\pgfsys@transformshift{1.700456in}{0.960872in}%
\pgfsys@useobject{currentmarker}{}%
\end{pgfscope}%
\begin{pgfscope}%
\pgfsys@transformshift{0.741052in}{1.421787in}%
\pgfsys@useobject{currentmarker}{}%
\end{pgfscope}%
\begin{pgfscope}%
\pgfsys@transformshift{0.576055in}{1.918446in}%
\pgfsys@useobject{currentmarker}{}%
\end{pgfscope}%
\begin{pgfscope}%
\pgfsys@transformshift{1.509849in}{1.274430in}%
\pgfsys@useobject{currentmarker}{}%
\end{pgfscope}%
\begin{pgfscope}%
\pgfsys@transformshift{1.838065in}{1.100580in}%
\pgfsys@useobject{currentmarker}{}%
\end{pgfscope}%
\begin{pgfscope}%
\pgfsys@transformshift{1.699823in}{1.297981in}%
\pgfsys@useobject{currentmarker}{}%
\end{pgfscope}%
\begin{pgfscope}%
\pgfsys@transformshift{1.575775in}{1.185245in}%
\pgfsys@useobject{currentmarker}{}%
\end{pgfscope}%
\begin{pgfscope}%
\pgfsys@transformshift{1.000652in}{1.471195in}%
\pgfsys@useobject{currentmarker}{}%
\end{pgfscope}%
\begin{pgfscope}%
\pgfsys@transformshift{1.275329in}{1.353199in}%
\pgfsys@useobject{currentmarker}{}%
\end{pgfscope}%
\begin{pgfscope}%
\pgfsys@transformshift{0.787302in}{1.380567in}%
\pgfsys@useobject{currentmarker}{}%
\end{pgfscope}%
\begin{pgfscope}%
\pgfsys@transformshift{0.941308in}{1.147894in}%
\pgfsys@useobject{currentmarker}{}%
\end{pgfscope}%
\begin{pgfscope}%
\pgfsys@transformshift{1.041866in}{1.132618in}%
\pgfsys@useobject{currentmarker}{}%
\end{pgfscope}%
\begin{pgfscope}%
\pgfsys@transformshift{1.471835in}{1.284117in}%
\pgfsys@useobject{currentmarker}{}%
\end{pgfscope}%
\begin{pgfscope}%
\pgfsys@transformshift{1.225077in}{1.314088in}%
\pgfsys@useobject{currentmarker}{}%
\end{pgfscope}%
\begin{pgfscope}%
\pgfsys@transformshift{1.294631in}{1.106811in}%
\pgfsys@useobject{currentmarker}{}%
\end{pgfscope}%
\begin{pgfscope}%
\pgfsys@transformshift{1.597060in}{1.207995in}%
\pgfsys@useobject{currentmarker}{}%
\end{pgfscope}%
\begin{pgfscope}%
\pgfsys@transformshift{1.136873in}{1.146536in}%
\pgfsys@useobject{currentmarker}{}%
\end{pgfscope}%
\begin{pgfscope}%
\pgfsys@transformshift{1.080421in}{1.303906in}%
\pgfsys@useobject{currentmarker}{}%
\end{pgfscope}%
\begin{pgfscope}%
\pgfsys@transformshift{1.185638in}{1.340841in}%
\pgfsys@useobject{currentmarker}{}%
\end{pgfscope}%
\begin{pgfscope}%
\pgfsys@transformshift{1.407062in}{1.311422in}%
\pgfsys@useobject{currentmarker}{}%
\end{pgfscope}%
\begin{pgfscope}%
\pgfsys@transformshift{1.504034in}{1.356760in}%
\pgfsys@useobject{currentmarker}{}%
\end{pgfscope}%
\begin{pgfscope}%
\pgfsys@transformshift{1.396930in}{1.315680in}%
\pgfsys@useobject{currentmarker}{}%
\end{pgfscope}%
\begin{pgfscope}%
\pgfsys@transformshift{2.000000in}{1.103847in}%
\pgfsys@useobject{currentmarker}{}%
\end{pgfscope}%
\begin{pgfscope}%
\pgfsys@transformshift{1.200619in}{1.221958in}%
\pgfsys@useobject{currentmarker}{}%
\end{pgfscope}%
\begin{pgfscope}%
\pgfsys@transformshift{0.950093in}{1.038964in}%
\pgfsys@useobject{currentmarker}{}%
\end{pgfscope}%
\begin{pgfscope}%
\pgfsys@transformshift{0.632457in}{1.506327in}%
\pgfsys@useobject{currentmarker}{}%
\end{pgfscope}%
\begin{pgfscope}%
\pgfsys@transformshift{0.673561in}{1.354639in}%
\pgfsys@useobject{currentmarker}{}%
\end{pgfscope}%
\begin{pgfscope}%
\pgfsys@transformshift{0.694799in}{1.541297in}%
\pgfsys@useobject{currentmarker}{}%
\end{pgfscope}%
\begin{pgfscope}%
\pgfsys@transformshift{0.518198in}{1.548783in}%
\pgfsys@useobject{currentmarker}{}%
\end{pgfscope}%
\begin{pgfscope}%
\pgfsys@transformshift{1.324442in}{1.228927in}%
\pgfsys@useobject{currentmarker}{}%
\end{pgfscope}%
\begin{pgfscope}%
\pgfsys@transformshift{0.827155in}{1.580904in}%
\pgfsys@useobject{currentmarker}{}%
\end{pgfscope}%
\begin{pgfscope}%
\pgfsys@transformshift{1.894588in}{1.078349in}%
\pgfsys@useobject{currentmarker}{}%
\end{pgfscope}%
\begin{pgfscope}%
\pgfsys@transformshift{0.446321in}{1.396466in}%
\pgfsys@useobject{currentmarker}{}%
\end{pgfscope}%
\begin{pgfscope}%
\pgfsys@transformshift{1.741761in}{1.262607in}%
\pgfsys@useobject{currentmarker}{}%
\end{pgfscope}%
\begin{pgfscope}%
\pgfsys@transformshift{1.307351in}{1.245040in}%
\pgfsys@useobject{currentmarker}{}%
\end{pgfscope}%
\begin{pgfscope}%
\pgfsys@transformshift{1.835875in}{1.095760in}%
\pgfsys@useobject{currentmarker}{}%
\end{pgfscope}%
\begin{pgfscope}%
\pgfsys@transformshift{1.237368in}{1.338604in}%
\pgfsys@useobject{currentmarker}{}%
\end{pgfscope}%
\begin{pgfscope}%
\pgfsys@transformshift{1.875271in}{0.842494in}%
\pgfsys@useobject{currentmarker}{}%
\end{pgfscope}%
\begin{pgfscope}%
\pgfsys@transformshift{1.669995in}{1.118722in}%
\pgfsys@useobject{currentmarker}{}%
\end{pgfscope}%
\begin{pgfscope}%
\pgfsys@transformshift{0.628877in}{1.628281in}%
\pgfsys@useobject{currentmarker}{}%
\end{pgfscope}%
\begin{pgfscope}%
\pgfsys@transformshift{1.644915in}{1.149965in}%
\pgfsys@useobject{currentmarker}{}%
\end{pgfscope}%
\begin{pgfscope}%
\pgfsys@transformshift{1.179959in}{1.282084in}%
\pgfsys@useobject{currentmarker}{}%
\end{pgfscope}%
\begin{pgfscope}%
\pgfsys@transformshift{1.753339in}{1.180007in}%
\pgfsys@useobject{currentmarker}{}%
\end{pgfscope}%
\begin{pgfscope}%
\pgfsys@transformshift{1.477915in}{1.301271in}%
\pgfsys@useobject{currentmarker}{}%
\end{pgfscope}%
\begin{pgfscope}%
\pgfsys@transformshift{1.742609in}{1.192028in}%
\pgfsys@useobject{currentmarker}{}%
\end{pgfscope}%
\begin{pgfscope}%
\pgfsys@transformshift{1.094044in}{1.442385in}%
\pgfsys@useobject{currentmarker}{}%
\end{pgfscope}%
\begin{pgfscope}%
\pgfsys@transformshift{1.645687in}{1.051128in}%
\pgfsys@useobject{currentmarker}{}%
\end{pgfscope}%
\begin{pgfscope}%
\pgfsys@transformshift{0.717249in}{1.317906in}%
\pgfsys@useobject{currentmarker}{}%
\end{pgfscope}%
\begin{pgfscope}%
\pgfsys@transformshift{0.963925in}{1.180427in}%
\pgfsys@useobject{currentmarker}{}%
\end{pgfscope}%
\begin{pgfscope}%
\pgfsys@transformshift{1.342874in}{1.351604in}%
\pgfsys@useobject{currentmarker}{}%
\end{pgfscope}%
\begin{pgfscope}%
\pgfsys@transformshift{1.522619in}{1.213087in}%
\pgfsys@useobject{currentmarker}{}%
\end{pgfscope}%
\begin{pgfscope}%
\pgfsys@transformshift{0.649847in}{1.614736in}%
\pgfsys@useobject{currentmarker}{}%
\end{pgfscope}%
\begin{pgfscope}%
\pgfsys@transformshift{1.826185in}{1.136241in}%
\pgfsys@useobject{currentmarker}{}%
\end{pgfscope}%
\begin{pgfscope}%
\pgfsys@transformshift{1.619806in}{1.178807in}%
\pgfsys@useobject{currentmarker}{}%
\end{pgfscope}%
\begin{pgfscope}%
\pgfsys@transformshift{1.136702in}{1.461762in}%
\pgfsys@useobject{currentmarker}{}%
\end{pgfscope}%
\begin{pgfscope}%
\pgfsys@transformshift{1.894425in}{1.021112in}%
\pgfsys@useobject{currentmarker}{}%
\end{pgfscope}%
\begin{pgfscope}%
\pgfsys@transformshift{0.668971in}{1.616146in}%
\pgfsys@useobject{currentmarker}{}%
\end{pgfscope}%
\end{pgfscope}%
\begin{pgfscope}%
\pgfpathrectangle{\pgfqpoint{0.341129in}{0.466613in}}{\pgfqpoint{1.658871in}{1.711598in}}%
\pgfusepath{clip}%
\pgfsetbuttcap%
\pgfsetroundjoin%
\definecolor{currentfill}{rgb}{0.768627,0.305882,0.321569}%
\pgfsetfillcolor{currentfill}%
\pgfsetfillopacity{0.150000}%
\pgfsetlinewidth{1.003750pt}%
\definecolor{currentstroke}{rgb}{1.000000,1.000000,1.000000}%
\pgfsetstrokecolor{currentstroke}%
\pgfsetstrokeopacity{0.150000}%
\pgfsetdash{}{0pt}%
\pgfsys@defobject{currentmarker}{\pgfqpoint{0.341129in}{1.057681in}}{\pgfqpoint{2.000000in}{1.554597in}}{%
\pgfpathmoveto{\pgfqpoint{0.341129in}{1.554597in}}%
\pgfpathlineto{\pgfqpoint{0.341129in}{1.482894in}}%
\pgfpathlineto{\pgfqpoint{0.357885in}{1.478932in}}%
\pgfpathlineto{\pgfqpoint{0.374641in}{1.474974in}}%
\pgfpathlineto{\pgfqpoint{0.391398in}{1.471018in}}%
\pgfpathlineto{\pgfqpoint{0.408154in}{1.467062in}}%
\pgfpathlineto{\pgfqpoint{0.424910in}{1.463106in}}%
\pgfpathlineto{\pgfqpoint{0.441666in}{1.459150in}}%
\pgfpathlineto{\pgfqpoint{0.458423in}{1.455194in}}%
\pgfpathlineto{\pgfqpoint{0.475179in}{1.451289in}}%
\pgfpathlineto{\pgfqpoint{0.491935in}{1.447532in}}%
\pgfpathlineto{\pgfqpoint{0.508691in}{1.443775in}}%
\pgfpathlineto{\pgfqpoint{0.525448in}{1.440018in}}%
\pgfpathlineto{\pgfqpoint{0.542204in}{1.436261in}}%
\pgfpathlineto{\pgfqpoint{0.558960in}{1.432501in}}%
\pgfpathlineto{\pgfqpoint{0.575717in}{1.428361in}}%
\pgfpathlineto{\pgfqpoint{0.592473in}{1.424221in}}%
\pgfpathlineto{\pgfqpoint{0.609229in}{1.420085in}}%
\pgfpathlineto{\pgfqpoint{0.625985in}{1.415951in}}%
\pgfpathlineto{\pgfqpoint{0.642742in}{1.412183in}}%
\pgfpathlineto{\pgfqpoint{0.659498in}{1.408422in}}%
\pgfpathlineto{\pgfqpoint{0.676254in}{1.404556in}}%
\pgfpathlineto{\pgfqpoint{0.693011in}{1.400504in}}%
\pgfpathlineto{\pgfqpoint{0.709767in}{1.396456in}}%
\pgfpathlineto{\pgfqpoint{0.726523in}{1.392409in}}%
\pgfpathlineto{\pgfqpoint{0.743279in}{1.388361in}}%
\pgfpathlineto{\pgfqpoint{0.760036in}{1.384468in}}%
\pgfpathlineto{\pgfqpoint{0.776792in}{1.380659in}}%
\pgfpathlineto{\pgfqpoint{0.793548in}{1.376925in}}%
\pgfpathlineto{\pgfqpoint{0.810304in}{1.373230in}}%
\pgfpathlineto{\pgfqpoint{0.827061in}{1.369556in}}%
\pgfpathlineto{\pgfqpoint{0.843817in}{1.365855in}}%
\pgfpathlineto{\pgfqpoint{0.860573in}{1.361945in}}%
\pgfpathlineto{\pgfqpoint{0.877330in}{1.358238in}}%
\pgfpathlineto{\pgfqpoint{0.894086in}{1.354259in}}%
\pgfpathlineto{\pgfqpoint{0.910842in}{1.350159in}}%
\pgfpathlineto{\pgfqpoint{0.927598in}{1.346297in}}%
\pgfpathlineto{\pgfqpoint{0.944355in}{1.342315in}}%
\pgfpathlineto{\pgfqpoint{0.961111in}{1.338335in}}%
\pgfpathlineto{\pgfqpoint{0.977867in}{1.334188in}}%
\pgfpathlineto{\pgfqpoint{0.994623in}{1.329908in}}%
\pgfpathlineto{\pgfqpoint{1.011380in}{1.325848in}}%
\pgfpathlineto{\pgfqpoint{1.028136in}{1.321788in}}%
\pgfpathlineto{\pgfqpoint{1.044892in}{1.317660in}}%
\pgfpathlineto{\pgfqpoint{1.061649in}{1.313670in}}%
\pgfpathlineto{\pgfqpoint{1.078405in}{1.309760in}}%
\pgfpathlineto{\pgfqpoint{1.095161in}{1.305734in}}%
\pgfpathlineto{\pgfqpoint{1.111917in}{1.301267in}}%
\pgfpathlineto{\pgfqpoint{1.128674in}{1.296998in}}%
\pgfpathlineto{\pgfqpoint{1.145430in}{1.293064in}}%
\pgfpathlineto{\pgfqpoint{1.162186in}{1.288968in}}%
\pgfpathlineto{\pgfqpoint{1.178942in}{1.284641in}}%
\pgfpathlineto{\pgfqpoint{1.195699in}{1.280562in}}%
\pgfpathlineto{\pgfqpoint{1.212455in}{1.276453in}}%
\pgfpathlineto{\pgfqpoint{1.229211in}{1.272554in}}%
\pgfpathlineto{\pgfqpoint{1.245968in}{1.268426in}}%
\pgfpathlineto{\pgfqpoint{1.262724in}{1.264110in}}%
\pgfpathlineto{\pgfqpoint{1.279480in}{1.259958in}}%
\pgfpathlineto{\pgfqpoint{1.296236in}{1.255660in}}%
\pgfpathlineto{\pgfqpoint{1.312993in}{1.251130in}}%
\pgfpathlineto{\pgfqpoint{1.329749in}{1.246714in}}%
\pgfpathlineto{\pgfqpoint{1.346505in}{1.242350in}}%
\pgfpathlineto{\pgfqpoint{1.363262in}{1.238179in}}%
\pgfpathlineto{\pgfqpoint{1.380018in}{1.233734in}}%
\pgfpathlineto{\pgfqpoint{1.396774in}{1.229326in}}%
\pgfpathlineto{\pgfqpoint{1.413530in}{1.224513in}}%
\pgfpathlineto{\pgfqpoint{1.430287in}{1.220079in}}%
\pgfpathlineto{\pgfqpoint{1.447043in}{1.215358in}}%
\pgfpathlineto{\pgfqpoint{1.463799in}{1.210616in}}%
\pgfpathlineto{\pgfqpoint{1.480555in}{1.206198in}}%
\pgfpathlineto{\pgfqpoint{1.497312in}{1.201615in}}%
\pgfpathlineto{\pgfqpoint{1.514068in}{1.196966in}}%
\pgfpathlineto{\pgfqpoint{1.530824in}{1.192432in}}%
\pgfpathlineto{\pgfqpoint{1.547581in}{1.187720in}}%
\pgfpathlineto{\pgfqpoint{1.564337in}{1.183052in}}%
\pgfpathlineto{\pgfqpoint{1.581093in}{1.178474in}}%
\pgfpathlineto{\pgfqpoint{1.597849in}{1.173834in}}%
\pgfpathlineto{\pgfqpoint{1.614606in}{1.169085in}}%
\pgfpathlineto{\pgfqpoint{1.631362in}{1.164184in}}%
\pgfpathlineto{\pgfqpoint{1.648118in}{1.159428in}}%
\pgfpathlineto{\pgfqpoint{1.664874in}{1.154673in}}%
\pgfpathlineto{\pgfqpoint{1.681631in}{1.149765in}}%
\pgfpathlineto{\pgfqpoint{1.698387in}{1.144856in}}%
\pgfpathlineto{\pgfqpoint{1.715143in}{1.140088in}}%
\pgfpathlineto{\pgfqpoint{1.731900in}{1.135251in}}%
\pgfpathlineto{\pgfqpoint{1.748656in}{1.130249in}}%
\pgfpathlineto{\pgfqpoint{1.765412in}{1.125371in}}%
\pgfpathlineto{\pgfqpoint{1.782168in}{1.120630in}}%
\pgfpathlineto{\pgfqpoint{1.798925in}{1.115892in}}%
\pgfpathlineto{\pgfqpoint{1.815681in}{1.111070in}}%
\pgfpathlineto{\pgfqpoint{1.832437in}{1.106087in}}%
\pgfpathlineto{\pgfqpoint{1.849193in}{1.101292in}}%
\pgfpathlineto{\pgfqpoint{1.865950in}{1.096350in}}%
\pgfpathlineto{\pgfqpoint{1.882706in}{1.091317in}}%
\pgfpathlineto{\pgfqpoint{1.899462in}{1.086615in}}%
\pgfpathlineto{\pgfqpoint{1.916219in}{1.082014in}}%
\pgfpathlineto{\pgfqpoint{1.932975in}{1.077024in}}%
\pgfpathlineto{\pgfqpoint{1.949731in}{1.072036in}}%
\pgfpathlineto{\pgfqpoint{1.966487in}{1.067300in}}%
\pgfpathlineto{\pgfqpoint{1.983244in}{1.062526in}}%
\pgfpathlineto{\pgfqpoint{2.000000in}{1.057681in}}%
\pgfpathlineto{\pgfqpoint{2.000000in}{1.106202in}}%
\pgfpathlineto{\pgfqpoint{2.000000in}{1.106202in}}%
\pgfpathlineto{\pgfqpoint{1.983244in}{1.110359in}}%
\pgfpathlineto{\pgfqpoint{1.966487in}{1.114552in}}%
\pgfpathlineto{\pgfqpoint{1.949731in}{1.118447in}}%
\pgfpathlineto{\pgfqpoint{1.932975in}{1.122324in}}%
\pgfpathlineto{\pgfqpoint{1.916219in}{1.126373in}}%
\pgfpathlineto{\pgfqpoint{1.899462in}{1.130401in}}%
\pgfpathlineto{\pgfqpoint{1.882706in}{1.134363in}}%
\pgfpathlineto{\pgfqpoint{1.865950in}{1.138270in}}%
\pgfpathlineto{\pgfqpoint{1.849193in}{1.142132in}}%
\pgfpathlineto{\pgfqpoint{1.832437in}{1.146225in}}%
\pgfpathlineto{\pgfqpoint{1.815681in}{1.150169in}}%
\pgfpathlineto{\pgfqpoint{1.798925in}{1.154087in}}%
\pgfpathlineto{\pgfqpoint{1.782168in}{1.158138in}}%
\pgfpathlineto{\pgfqpoint{1.765412in}{1.162102in}}%
\pgfpathlineto{\pgfqpoint{1.748656in}{1.166089in}}%
\pgfpathlineto{\pgfqpoint{1.731900in}{1.170022in}}%
\pgfpathlineto{\pgfqpoint{1.715143in}{1.173986in}}%
\pgfpathlineto{\pgfqpoint{1.698387in}{1.178026in}}%
\pgfpathlineto{\pgfqpoint{1.681631in}{1.182132in}}%
\pgfpathlineto{\pgfqpoint{1.664874in}{1.186046in}}%
\pgfpathlineto{\pgfqpoint{1.648118in}{1.190115in}}%
\pgfpathlineto{\pgfqpoint{1.631362in}{1.194286in}}%
\pgfpathlineto{\pgfqpoint{1.614606in}{1.198292in}}%
\pgfpathlineto{\pgfqpoint{1.597849in}{1.202314in}}%
\pgfpathlineto{\pgfqpoint{1.581093in}{1.206427in}}%
\pgfpathlineto{\pgfqpoint{1.564337in}{1.210572in}}%
\pgfpathlineto{\pgfqpoint{1.547581in}{1.214611in}}%
\pgfpathlineto{\pgfqpoint{1.530824in}{1.218635in}}%
\pgfpathlineto{\pgfqpoint{1.514068in}{1.222717in}}%
\pgfpathlineto{\pgfqpoint{1.497312in}{1.226734in}}%
\pgfpathlineto{\pgfqpoint{1.480555in}{1.231084in}}%
\pgfpathlineto{\pgfqpoint{1.463799in}{1.235315in}}%
\pgfpathlineto{\pgfqpoint{1.447043in}{1.239554in}}%
\pgfpathlineto{\pgfqpoint{1.430287in}{1.243736in}}%
\pgfpathlineto{\pgfqpoint{1.413530in}{1.247914in}}%
\pgfpathlineto{\pgfqpoint{1.396774in}{1.252209in}}%
\pgfpathlineto{\pgfqpoint{1.380018in}{1.256836in}}%
\pgfpathlineto{\pgfqpoint{1.363262in}{1.261102in}}%
\pgfpathlineto{\pgfqpoint{1.346505in}{1.265646in}}%
\pgfpathlineto{\pgfqpoint{1.329749in}{1.270111in}}%
\pgfpathlineto{\pgfqpoint{1.312993in}{1.274592in}}%
\pgfpathlineto{\pgfqpoint{1.296236in}{1.279208in}}%
\pgfpathlineto{\pgfqpoint{1.279480in}{1.283758in}}%
\pgfpathlineto{\pgfqpoint{1.262724in}{1.288302in}}%
\pgfpathlineto{\pgfqpoint{1.245968in}{1.292802in}}%
\pgfpathlineto{\pgfqpoint{1.229211in}{1.297527in}}%
\pgfpathlineto{\pgfqpoint{1.212455in}{1.302240in}}%
\pgfpathlineto{\pgfqpoint{1.195699in}{1.307014in}}%
\pgfpathlineto{\pgfqpoint{1.178942in}{1.311668in}}%
\pgfpathlineto{\pgfqpoint{1.162186in}{1.316469in}}%
\pgfpathlineto{\pgfqpoint{1.145430in}{1.321086in}}%
\pgfpathlineto{\pgfqpoint{1.128674in}{1.325424in}}%
\pgfpathlineto{\pgfqpoint{1.111917in}{1.330030in}}%
\pgfpathlineto{\pgfqpoint{1.095161in}{1.334843in}}%
\pgfpathlineto{\pgfqpoint{1.078405in}{1.339655in}}%
\pgfpathlineto{\pgfqpoint{1.061649in}{1.344467in}}%
\pgfpathlineto{\pgfqpoint{1.044892in}{1.349280in}}%
\pgfpathlineto{\pgfqpoint{1.028136in}{1.354092in}}%
\pgfpathlineto{\pgfqpoint{1.011380in}{1.358691in}}%
\pgfpathlineto{\pgfqpoint{0.994623in}{1.363544in}}%
\pgfpathlineto{\pgfqpoint{0.977867in}{1.368440in}}%
\pgfpathlineto{\pgfqpoint{0.961111in}{1.373339in}}%
\pgfpathlineto{\pgfqpoint{0.944355in}{1.378181in}}%
\pgfpathlineto{\pgfqpoint{0.927598in}{1.383053in}}%
\pgfpathlineto{\pgfqpoint{0.910842in}{1.387840in}}%
\pgfpathlineto{\pgfqpoint{0.894086in}{1.392620in}}%
\pgfpathlineto{\pgfqpoint{0.877330in}{1.397470in}}%
\pgfpathlineto{\pgfqpoint{0.860573in}{1.402271in}}%
\pgfpathlineto{\pgfqpoint{0.843817in}{1.407086in}}%
\pgfpathlineto{\pgfqpoint{0.827061in}{1.411894in}}%
\pgfpathlineto{\pgfqpoint{0.810304in}{1.416787in}}%
\pgfpathlineto{\pgfqpoint{0.793548in}{1.421685in}}%
\pgfpathlineto{\pgfqpoint{0.776792in}{1.426538in}}%
\pgfpathlineto{\pgfqpoint{0.760036in}{1.431393in}}%
\pgfpathlineto{\pgfqpoint{0.743279in}{1.436279in}}%
\pgfpathlineto{\pgfqpoint{0.726523in}{1.441207in}}%
\pgfpathlineto{\pgfqpoint{0.709767in}{1.446064in}}%
\pgfpathlineto{\pgfqpoint{0.693011in}{1.451003in}}%
\pgfpathlineto{\pgfqpoint{0.676254in}{1.455978in}}%
\pgfpathlineto{\pgfqpoint{0.659498in}{1.460932in}}%
\pgfpathlineto{\pgfqpoint{0.642742in}{1.465745in}}%
\pgfpathlineto{\pgfqpoint{0.625985in}{1.470659in}}%
\pgfpathlineto{\pgfqpoint{0.609229in}{1.475573in}}%
\pgfpathlineto{\pgfqpoint{0.592473in}{1.480487in}}%
\pgfpathlineto{\pgfqpoint{0.575717in}{1.485399in}}%
\pgfpathlineto{\pgfqpoint{0.558960in}{1.490308in}}%
\pgfpathlineto{\pgfqpoint{0.542204in}{1.495161in}}%
\pgfpathlineto{\pgfqpoint{0.525448in}{1.499927in}}%
\pgfpathlineto{\pgfqpoint{0.508691in}{1.504762in}}%
\pgfpathlineto{\pgfqpoint{0.491935in}{1.509671in}}%
\pgfpathlineto{\pgfqpoint{0.475179in}{1.514579in}}%
\pgfpathlineto{\pgfqpoint{0.458423in}{1.519488in}}%
\pgfpathlineto{\pgfqpoint{0.441666in}{1.524440in}}%
\pgfpathlineto{\pgfqpoint{0.424910in}{1.529473in}}%
\pgfpathlineto{\pgfqpoint{0.408154in}{1.534506in}}%
\pgfpathlineto{\pgfqpoint{0.391398in}{1.539544in}}%
\pgfpathlineto{\pgfqpoint{0.374641in}{1.544583in}}%
\pgfpathlineto{\pgfqpoint{0.357885in}{1.549616in}}%
\pgfpathlineto{\pgfqpoint{0.341129in}{1.554597in}}%
\pgfpathclose%
\pgfusepath{stroke,fill}%
}%
\begin{pgfscope}%
\pgfsys@transformshift{0.000000in}{0.000000in}%
\pgfsys@useobject{currentmarker}{}%
\end{pgfscope}%
\end{pgfscope}%
\begin{pgfscope}%
\pgfpathrectangle{\pgfqpoint{0.341129in}{0.466613in}}{\pgfqpoint{1.658871in}{1.711598in}}%
\pgfusepath{clip}%
\pgfsetbuttcap%
\pgfsetroundjoin%
\definecolor{currentfill}{rgb}{0.505882,0.447059,0.701961}%
\pgfsetfillcolor{currentfill}%
\pgfsetfillopacity{0.250000}%
\pgfsetlinewidth{1.003750pt}%
\definecolor{currentstroke}{rgb}{0.505882,0.447059,0.701961}%
\pgfsetstrokecolor{currentstroke}%
\pgfsetstrokeopacity{0.250000}%
\pgfsetdash{}{0pt}%
\pgfsys@defobject{currentmarker}{\pgfqpoint{-0.017010in}{-0.017010in}}{\pgfqpoint{0.017010in}{0.017010in}}{%
\pgfpathmoveto{\pgfqpoint{0.000000in}{-0.017010in}}%
\pgfpathcurveto{\pgfqpoint{0.004511in}{-0.017010in}}{\pgfqpoint{0.008838in}{-0.015218in}}{\pgfqpoint{0.012028in}{-0.012028in}}%
\pgfpathcurveto{\pgfqpoint{0.015218in}{-0.008838in}}{\pgfqpoint{0.017010in}{-0.004511in}}{\pgfqpoint{0.017010in}{0.000000in}}%
\pgfpathcurveto{\pgfqpoint{0.017010in}{0.004511in}}{\pgfqpoint{0.015218in}{0.008838in}}{\pgfqpoint{0.012028in}{0.012028in}}%
\pgfpathcurveto{\pgfqpoint{0.008838in}{0.015218in}}{\pgfqpoint{0.004511in}{0.017010in}}{\pgfqpoint{0.000000in}{0.017010in}}%
\pgfpathcurveto{\pgfqpoint{-0.004511in}{0.017010in}}{\pgfqpoint{-0.008838in}{0.015218in}}{\pgfqpoint{-0.012028in}{0.012028in}}%
\pgfpathcurveto{\pgfqpoint{-0.015218in}{0.008838in}}{\pgfqpoint{-0.017010in}{0.004511in}}{\pgfqpoint{-0.017010in}{0.000000in}}%
\pgfpathcurveto{\pgfqpoint{-0.017010in}{-0.004511in}}{\pgfqpoint{-0.015218in}{-0.008838in}}{\pgfqpoint{-0.012028in}{-0.012028in}}%
\pgfpathcurveto{\pgfqpoint{-0.008838in}{-0.015218in}}{\pgfqpoint{-0.004511in}{-0.017010in}}{\pgfqpoint{0.000000in}{-0.017010in}}%
\pgfpathclose%
\pgfusepath{stroke,fill}%
}%
\begin{pgfscope}%
\pgfsys@transformshift{0.667456in}{1.400393in}%
\pgfsys@useobject{currentmarker}{}%
\end{pgfscope}%
\begin{pgfscope}%
\pgfsys@transformshift{0.872203in}{1.183090in}%
\pgfsys@useobject{currentmarker}{}%
\end{pgfscope}%
\begin{pgfscope}%
\pgfsys@transformshift{1.174456in}{0.849586in}%
\pgfsys@useobject{currentmarker}{}%
\end{pgfscope}%
\begin{pgfscope}%
\pgfsys@transformshift{1.594005in}{0.764066in}%
\pgfsys@useobject{currentmarker}{}%
\end{pgfscope}%
\begin{pgfscope}%
\pgfsys@transformshift{1.217413in}{0.734372in}%
\pgfsys@useobject{currentmarker}{}%
\end{pgfscope}%
\begin{pgfscope}%
\pgfsys@transformshift{0.553973in}{1.348406in}%
\pgfsys@useobject{currentmarker}{}%
\end{pgfscope}%
\begin{pgfscope}%
\pgfsys@transformshift{1.639332in}{0.804347in}%
\pgfsys@useobject{currentmarker}{}%
\end{pgfscope}%
\begin{pgfscope}%
\pgfsys@transformshift{0.709729in}{1.300277in}%
\pgfsys@useobject{currentmarker}{}%
\end{pgfscope}%
\begin{pgfscope}%
\pgfsys@transformshift{1.202352in}{1.379272in}%
\pgfsys@useobject{currentmarker}{}%
\end{pgfscope}%
\begin{pgfscope}%
\pgfsys@transformshift{0.671018in}{0.983944in}%
\pgfsys@useobject{currentmarker}{}%
\end{pgfscope}%
\begin{pgfscope}%
\pgfsys@transformshift{1.442003in}{0.848880in}%
\pgfsys@useobject{currentmarker}{}%
\end{pgfscope}%
\begin{pgfscope}%
\pgfsys@transformshift{1.398849in}{0.805848in}%
\pgfsys@useobject{currentmarker}{}%
\end{pgfscope}%
\begin{pgfscope}%
\pgfsys@transformshift{1.552752in}{0.692826in}%
\pgfsys@useobject{currentmarker}{}%
\end{pgfscope}%
\begin{pgfscope}%
\pgfsys@transformshift{1.177147in}{1.053761in}%
\pgfsys@useobject{currentmarker}{}%
\end{pgfscope}%
\begin{pgfscope}%
\pgfsys@transformshift{1.887294in}{1.160445in}%
\pgfsys@useobject{currentmarker}{}%
\end{pgfscope}%
\begin{pgfscope}%
\pgfsys@transformshift{1.188970in}{0.864508in}%
\pgfsys@useobject{currentmarker}{}%
\end{pgfscope}%
\begin{pgfscope}%
\pgfsys@transformshift{1.534509in}{0.973333in}%
\pgfsys@useobject{currentmarker}{}%
\end{pgfscope}%
\begin{pgfscope}%
\pgfsys@transformshift{0.588572in}{1.313571in}%
\pgfsys@useobject{currentmarker}{}%
\end{pgfscope}%
\begin{pgfscope}%
\pgfsys@transformshift{0.952296in}{1.084010in}%
\pgfsys@useobject{currentmarker}{}%
\end{pgfscope}%
\begin{pgfscope}%
\pgfsys@transformshift{0.705522in}{1.356227in}%
\pgfsys@useobject{currentmarker}{}%
\end{pgfscope}%
\begin{pgfscope}%
\pgfsys@transformshift{1.121384in}{0.775731in}%
\pgfsys@useobject{currentmarker}{}%
\end{pgfscope}%
\begin{pgfscope}%
\pgfsys@transformshift{1.103574in}{0.832111in}%
\pgfsys@useobject{currentmarker}{}%
\end{pgfscope}%
\begin{pgfscope}%
\pgfsys@transformshift{1.817005in}{0.820277in}%
\pgfsys@useobject{currentmarker}{}%
\end{pgfscope}%
\begin{pgfscope}%
\pgfsys@transformshift{0.496492in}{1.395740in}%
\pgfsys@useobject{currentmarker}{}%
\end{pgfscope}%
\begin{pgfscope}%
\pgfsys@transformshift{1.711311in}{0.974168in}%
\pgfsys@useobject{currentmarker}{}%
\end{pgfscope}%
\begin{pgfscope}%
\pgfsys@transformshift{1.750408in}{0.616944in}%
\pgfsys@useobject{currentmarker}{}%
\end{pgfscope}%
\begin{pgfscope}%
\pgfsys@transformshift{1.333406in}{0.767265in}%
\pgfsys@useobject{currentmarker}{}%
\end{pgfscope}%
\begin{pgfscope}%
\pgfsys@transformshift{1.724223in}{0.804976in}%
\pgfsys@useobject{currentmarker}{}%
\end{pgfscope}%
\begin{pgfscope}%
\pgfsys@transformshift{1.256999in}{0.842651in}%
\pgfsys@useobject{currentmarker}{}%
\end{pgfscope}%
\begin{pgfscope}%
\pgfsys@transformshift{1.047649in}{1.130423in}%
\pgfsys@useobject{currentmarker}{}%
\end{pgfscope}%
\begin{pgfscope}%
\pgfsys@transformshift{0.878614in}{1.161884in}%
\pgfsys@useobject{currentmarker}{}%
\end{pgfscope}%
\begin{pgfscope}%
\pgfsys@transformshift{0.761954in}{1.229862in}%
\pgfsys@useobject{currentmarker}{}%
\end{pgfscope}%
\begin{pgfscope}%
\pgfsys@transformshift{1.796268in}{0.815551in}%
\pgfsys@useobject{currentmarker}{}%
\end{pgfscope}%
\begin{pgfscope}%
\pgfsys@transformshift{0.726378in}{1.394211in}%
\pgfsys@useobject{currentmarker}{}%
\end{pgfscope}%
\begin{pgfscope}%
\pgfsys@transformshift{1.408196in}{0.789889in}%
\pgfsys@useobject{currentmarker}{}%
\end{pgfscope}%
\begin{pgfscope}%
\pgfsys@transformshift{0.612217in}{0.877609in}%
\pgfsys@useobject{currentmarker}{}%
\end{pgfscope}%
\begin{pgfscope}%
\pgfsys@transformshift{1.180902in}{1.018124in}%
\pgfsys@useobject{currentmarker}{}%
\end{pgfscope}%
\begin{pgfscope}%
\pgfsys@transformshift{0.627344in}{1.347705in}%
\pgfsys@useobject{currentmarker}{}%
\end{pgfscope}%
\begin{pgfscope}%
\pgfsys@transformshift{0.628282in}{1.167178in}%
\pgfsys@useobject{currentmarker}{}%
\end{pgfscope}%
\begin{pgfscope}%
\pgfsys@transformshift{0.437992in}{0.880890in}%
\pgfsys@useobject{currentmarker}{}%
\end{pgfscope}%
\begin{pgfscope}%
\pgfsys@transformshift{0.758691in}{1.292137in}%
\pgfsys@useobject{currentmarker}{}%
\end{pgfscope}%
\begin{pgfscope}%
\pgfsys@transformshift{1.614220in}{0.981721in}%
\pgfsys@useobject{currentmarker}{}%
\end{pgfscope}%
\begin{pgfscope}%
\pgfsys@transformshift{1.846563in}{0.879350in}%
\pgfsys@useobject{currentmarker}{}%
\end{pgfscope}%
\begin{pgfscope}%
\pgfsys@transformshift{1.460173in}{0.940329in}%
\pgfsys@useobject{currentmarker}{}%
\end{pgfscope}%
\begin{pgfscope}%
\pgfsys@transformshift{1.637112in}{0.805618in}%
\pgfsys@useobject{currentmarker}{}%
\end{pgfscope}%
\begin{pgfscope}%
\pgfsys@transformshift{1.615512in}{0.729130in}%
\pgfsys@useobject{currentmarker}{}%
\end{pgfscope}%
\begin{pgfscope}%
\pgfsys@transformshift{1.242112in}{1.150423in}%
\pgfsys@useobject{currentmarker}{}%
\end{pgfscope}%
\begin{pgfscope}%
\pgfsys@transformshift{0.688214in}{1.117976in}%
\pgfsys@useobject{currentmarker}{}%
\end{pgfscope}%
\begin{pgfscope}%
\pgfsys@transformshift{0.911315in}{0.778003in}%
\pgfsys@useobject{currentmarker}{}%
\end{pgfscope}%
\begin{pgfscope}%
\pgfsys@transformshift{0.551462in}{1.378383in}%
\pgfsys@useobject{currentmarker}{}%
\end{pgfscope}%
\begin{pgfscope}%
\pgfsys@transformshift{1.069973in}{1.068958in}%
\pgfsys@useobject{currentmarker}{}%
\end{pgfscope}%
\begin{pgfscope}%
\pgfsys@transformshift{1.211180in}{1.076857in}%
\pgfsys@useobject{currentmarker}{}%
\end{pgfscope}%
\begin{pgfscope}%
\pgfsys@transformshift{1.485757in}{0.988621in}%
\pgfsys@useobject{currentmarker}{}%
\end{pgfscope}%
\begin{pgfscope}%
\pgfsys@transformshift{0.743119in}{1.142170in}%
\pgfsys@useobject{currentmarker}{}%
\end{pgfscope}%
\begin{pgfscope}%
\pgfsys@transformshift{1.407385in}{0.719340in}%
\pgfsys@useobject{currentmarker}{}%
\end{pgfscope}%
\begin{pgfscope}%
\pgfsys@transformshift{1.618799in}{1.208350in}%
\pgfsys@useobject{currentmarker}{}%
\end{pgfscope}%
\begin{pgfscope}%
\pgfsys@transformshift{1.477965in}{0.682895in}%
\pgfsys@useobject{currentmarker}{}%
\end{pgfscope}%
\begin{pgfscope}%
\pgfsys@transformshift{1.156480in}{1.040034in}%
\pgfsys@useobject{currentmarker}{}%
\end{pgfscope}%
\begin{pgfscope}%
\pgfsys@transformshift{1.583657in}{0.716436in}%
\pgfsys@useobject{currentmarker}{}%
\end{pgfscope}%
\begin{pgfscope}%
\pgfsys@transformshift{1.440624in}{0.766082in}%
\pgfsys@useobject{currentmarker}{}%
\end{pgfscope}%
\begin{pgfscope}%
\pgfsys@transformshift{1.318935in}{0.776744in}%
\pgfsys@useobject{currentmarker}{}%
\end{pgfscope}%
\begin{pgfscope}%
\pgfsys@transformshift{0.547443in}{1.635460in}%
\pgfsys@useobject{currentmarker}{}%
\end{pgfscope}%
\begin{pgfscope}%
\pgfsys@transformshift{1.436824in}{1.034957in}%
\pgfsys@useobject{currentmarker}{}%
\end{pgfscope}%
\begin{pgfscope}%
\pgfsys@transformshift{1.658179in}{0.793819in}%
\pgfsys@useobject{currentmarker}{}%
\end{pgfscope}%
\begin{pgfscope}%
\pgfsys@transformshift{1.709942in}{0.762400in}%
\pgfsys@useobject{currentmarker}{}%
\end{pgfscope}%
\begin{pgfscope}%
\pgfsys@transformshift{1.638310in}{0.662365in}%
\pgfsys@useobject{currentmarker}{}%
\end{pgfscope}%
\begin{pgfscope}%
\pgfsys@transformshift{1.723779in}{0.866469in}%
\pgfsys@useobject{currentmarker}{}%
\end{pgfscope}%
\begin{pgfscope}%
\pgfsys@transformshift{1.124544in}{1.109819in}%
\pgfsys@useobject{currentmarker}{}%
\end{pgfscope}%
\begin{pgfscope}%
\pgfsys@transformshift{1.267442in}{1.150272in}%
\pgfsys@useobject{currentmarker}{}%
\end{pgfscope}%
\begin{pgfscope}%
\pgfsys@transformshift{0.562064in}{0.869055in}%
\pgfsys@useobject{currentmarker}{}%
\end{pgfscope}%
\begin{pgfscope}%
\pgfsys@transformshift{0.642722in}{1.296186in}%
\pgfsys@useobject{currentmarker}{}%
\end{pgfscope}%
\begin{pgfscope}%
\pgfsys@transformshift{1.150269in}{1.221422in}%
\pgfsys@useobject{currentmarker}{}%
\end{pgfscope}%
\begin{pgfscope}%
\pgfsys@transformshift{1.173281in}{1.030925in}%
\pgfsys@useobject{currentmarker}{}%
\end{pgfscope}%
\begin{pgfscope}%
\pgfsys@transformshift{1.891719in}{1.179308in}%
\pgfsys@useobject{currentmarker}{}%
\end{pgfscope}%
\begin{pgfscope}%
\pgfsys@transformshift{0.694241in}{1.376178in}%
\pgfsys@useobject{currentmarker}{}%
\end{pgfscope}%
\begin{pgfscope}%
\pgfsys@transformshift{0.790322in}{1.203174in}%
\pgfsys@useobject{currentmarker}{}%
\end{pgfscope}%
\begin{pgfscope}%
\pgfsys@transformshift{1.424550in}{0.694186in}%
\pgfsys@useobject{currentmarker}{}%
\end{pgfscope}%
\begin{pgfscope}%
\pgfsys@transformshift{0.384790in}{0.910674in}%
\pgfsys@useobject{currentmarker}{}%
\end{pgfscope}%
\begin{pgfscope}%
\pgfsys@transformshift{0.729116in}{1.264912in}%
\pgfsys@useobject{currentmarker}{}%
\end{pgfscope}%
\begin{pgfscope}%
\pgfsys@transformshift{1.217637in}{0.980450in}%
\pgfsys@useobject{currentmarker}{}%
\end{pgfscope}%
\begin{pgfscope}%
\pgfsys@transformshift{0.452540in}{1.446065in}%
\pgfsys@useobject{currentmarker}{}%
\end{pgfscope}%
\begin{pgfscope}%
\pgfsys@transformshift{1.493704in}{0.665502in}%
\pgfsys@useobject{currentmarker}{}%
\end{pgfscope}%
\begin{pgfscope}%
\pgfsys@transformshift{1.543302in}{0.876420in}%
\pgfsys@useobject{currentmarker}{}%
\end{pgfscope}%
\begin{pgfscope}%
\pgfsys@transformshift{1.218014in}{0.884265in}%
\pgfsys@useobject{currentmarker}{}%
\end{pgfscope}%
\begin{pgfscope}%
\pgfsys@transformshift{1.521861in}{0.771777in}%
\pgfsys@useobject{currentmarker}{}%
\end{pgfscope}%
\begin{pgfscope}%
\pgfsys@transformshift{1.764286in}{0.914610in}%
\pgfsys@useobject{currentmarker}{}%
\end{pgfscope}%
\begin{pgfscope}%
\pgfsys@transformshift{0.924889in}{1.555032in}%
\pgfsys@useobject{currentmarker}{}%
\end{pgfscope}%
\begin{pgfscope}%
\pgfsys@transformshift{1.222866in}{0.850378in}%
\pgfsys@useobject{currentmarker}{}%
\end{pgfscope}%
\begin{pgfscope}%
\pgfsys@transformshift{1.382222in}{0.916365in}%
\pgfsys@useobject{currentmarker}{}%
\end{pgfscope}%
\begin{pgfscope}%
\pgfsys@transformshift{1.643945in}{0.894342in}%
\pgfsys@useobject{currentmarker}{}%
\end{pgfscope}%
\begin{pgfscope}%
\pgfsys@transformshift{0.801834in}{1.165553in}%
\pgfsys@useobject{currentmarker}{}%
\end{pgfscope}%
\begin{pgfscope}%
\pgfsys@transformshift{1.258384in}{1.261924in}%
\pgfsys@useobject{currentmarker}{}%
\end{pgfscope}%
\begin{pgfscope}%
\pgfsys@transformshift{1.394783in}{0.849420in}%
\pgfsys@useobject{currentmarker}{}%
\end{pgfscope}%
\begin{pgfscope}%
\pgfsys@transformshift{0.341129in}{1.357758in}%
\pgfsys@useobject{currentmarker}{}%
\end{pgfscope}%
\begin{pgfscope}%
\pgfsys@transformshift{1.069581in}{1.235414in}%
\pgfsys@useobject{currentmarker}{}%
\end{pgfscope}%
\begin{pgfscope}%
\pgfsys@transformshift{0.789606in}{1.566566in}%
\pgfsys@useobject{currentmarker}{}%
\end{pgfscope}%
\begin{pgfscope}%
\pgfsys@transformshift{1.273287in}{0.724422in}%
\pgfsys@useobject{currentmarker}{}%
\end{pgfscope}%
\begin{pgfscope}%
\pgfsys@transformshift{0.646998in}{1.171236in}%
\pgfsys@useobject{currentmarker}{}%
\end{pgfscope}%
\begin{pgfscope}%
\pgfsys@transformshift{0.761065in}{1.514052in}%
\pgfsys@useobject{currentmarker}{}%
\end{pgfscope}%
\begin{pgfscope}%
\pgfsys@transformshift{0.628168in}{1.734108in}%
\pgfsys@useobject{currentmarker}{}%
\end{pgfscope}%
\begin{pgfscope}%
\pgfsys@transformshift{1.214800in}{0.894715in}%
\pgfsys@useobject{currentmarker}{}%
\end{pgfscope}%
\begin{pgfscope}%
\pgfsys@transformshift{1.051231in}{0.986803in}%
\pgfsys@useobject{currentmarker}{}%
\end{pgfscope}%
\begin{pgfscope}%
\pgfsys@transformshift{1.010738in}{1.128774in}%
\pgfsys@useobject{currentmarker}{}%
\end{pgfscope}%
\begin{pgfscope}%
\pgfsys@transformshift{1.387136in}{0.758703in}%
\pgfsys@useobject{currentmarker}{}%
\end{pgfscope}%
\begin{pgfscope}%
\pgfsys@transformshift{0.522501in}{0.830339in}%
\pgfsys@useobject{currentmarker}{}%
\end{pgfscope}%
\begin{pgfscope}%
\pgfsys@transformshift{1.737851in}{1.026499in}%
\pgfsys@useobject{currentmarker}{}%
\end{pgfscope}%
\begin{pgfscope}%
\pgfsys@transformshift{1.622475in}{0.785136in}%
\pgfsys@useobject{currentmarker}{}%
\end{pgfscope}%
\begin{pgfscope}%
\pgfsys@transformshift{0.776054in}{0.811674in}%
\pgfsys@useobject{currentmarker}{}%
\end{pgfscope}%
\begin{pgfscope}%
\pgfsys@transformshift{1.260864in}{0.778883in}%
\pgfsys@useobject{currentmarker}{}%
\end{pgfscope}%
\begin{pgfscope}%
\pgfsys@transformshift{1.587232in}{0.750144in}%
\pgfsys@useobject{currentmarker}{}%
\end{pgfscope}%
\begin{pgfscope}%
\pgfsys@transformshift{1.339693in}{0.853793in}%
\pgfsys@useobject{currentmarker}{}%
\end{pgfscope}%
\begin{pgfscope}%
\pgfsys@transformshift{1.448928in}{0.682177in}%
\pgfsys@useobject{currentmarker}{}%
\end{pgfscope}%
\begin{pgfscope}%
\pgfsys@transformshift{1.848941in}{0.716102in}%
\pgfsys@useobject{currentmarker}{}%
\end{pgfscope}%
\begin{pgfscope}%
\pgfsys@transformshift{1.645721in}{0.815216in}%
\pgfsys@useobject{currentmarker}{}%
\end{pgfscope}%
\begin{pgfscope}%
\pgfsys@transformshift{0.980835in}{1.181676in}%
\pgfsys@useobject{currentmarker}{}%
\end{pgfscope}%
\begin{pgfscope}%
\pgfsys@transformshift{0.704938in}{1.485008in}%
\pgfsys@useobject{currentmarker}{}%
\end{pgfscope}%
\begin{pgfscope}%
\pgfsys@transformshift{0.622306in}{1.296155in}%
\pgfsys@useobject{currentmarker}{}%
\end{pgfscope}%
\begin{pgfscope}%
\pgfsys@transformshift{1.213744in}{0.775778in}%
\pgfsys@useobject{currentmarker}{}%
\end{pgfscope}%
\begin{pgfscope}%
\pgfsys@transformshift{0.934075in}{1.320093in}%
\pgfsys@useobject{currentmarker}{}%
\end{pgfscope}%
\begin{pgfscope}%
\pgfsys@transformshift{1.307417in}{0.787749in}%
\pgfsys@useobject{currentmarker}{}%
\end{pgfscope}%
\begin{pgfscope}%
\pgfsys@transformshift{1.605349in}{0.934857in}%
\pgfsys@useobject{currentmarker}{}%
\end{pgfscope}%
\begin{pgfscope}%
\pgfsys@transformshift{0.750394in}{1.202368in}%
\pgfsys@useobject{currentmarker}{}%
\end{pgfscope}%
\begin{pgfscope}%
\pgfsys@transformshift{1.771258in}{0.917183in}%
\pgfsys@useobject{currentmarker}{}%
\end{pgfscope}%
\begin{pgfscope}%
\pgfsys@transformshift{1.939172in}{1.241185in}%
\pgfsys@useobject{currentmarker}{}%
\end{pgfscope}%
\begin{pgfscope}%
\pgfsys@transformshift{1.599583in}{0.979835in}%
\pgfsys@useobject{currentmarker}{}%
\end{pgfscope}%
\begin{pgfscope}%
\pgfsys@transformshift{0.881257in}{1.012165in}%
\pgfsys@useobject{currentmarker}{}%
\end{pgfscope}%
\begin{pgfscope}%
\pgfsys@transformshift{1.201768in}{1.003688in}%
\pgfsys@useobject{currentmarker}{}%
\end{pgfscope}%
\begin{pgfscope}%
\pgfsys@transformshift{1.244134in}{1.131411in}%
\pgfsys@useobject{currentmarker}{}%
\end{pgfscope}%
\begin{pgfscope}%
\pgfsys@transformshift{1.750150in}{0.755243in}%
\pgfsys@useobject{currentmarker}{}%
\end{pgfscope}%
\begin{pgfscope}%
\pgfsys@transformshift{1.541153in}{0.876416in}%
\pgfsys@useobject{currentmarker}{}%
\end{pgfscope}%
\begin{pgfscope}%
\pgfsys@transformshift{0.964663in}{0.752436in}%
\pgfsys@useobject{currentmarker}{}%
\end{pgfscope}%
\begin{pgfscope}%
\pgfsys@transformshift{0.818109in}{1.198701in}%
\pgfsys@useobject{currentmarker}{}%
\end{pgfscope}%
\begin{pgfscope}%
\pgfsys@transformshift{1.755264in}{0.799533in}%
\pgfsys@useobject{currentmarker}{}%
\end{pgfscope}%
\begin{pgfscope}%
\pgfsys@transformshift{1.450480in}{1.105924in}%
\pgfsys@useobject{currentmarker}{}%
\end{pgfscope}%
\begin{pgfscope}%
\pgfsys@transformshift{1.246595in}{1.032298in}%
\pgfsys@useobject{currentmarker}{}%
\end{pgfscope}%
\begin{pgfscope}%
\pgfsys@transformshift{1.110348in}{1.274777in}%
\pgfsys@useobject{currentmarker}{}%
\end{pgfscope}%
\begin{pgfscope}%
\pgfsys@transformshift{0.853154in}{1.050533in}%
\pgfsys@useobject{currentmarker}{}%
\end{pgfscope}%
\begin{pgfscope}%
\pgfsys@transformshift{0.803206in}{1.368268in}%
\pgfsys@useobject{currentmarker}{}%
\end{pgfscope}%
\begin{pgfscope}%
\pgfsys@transformshift{0.800584in}{1.380848in}%
\pgfsys@useobject{currentmarker}{}%
\end{pgfscope}%
\begin{pgfscope}%
\pgfsys@transformshift{0.828786in}{1.368848in}%
\pgfsys@useobject{currentmarker}{}%
\end{pgfscope}%
\begin{pgfscope}%
\pgfsys@transformshift{0.802472in}{1.225333in}%
\pgfsys@useobject{currentmarker}{}%
\end{pgfscope}%
\begin{pgfscope}%
\pgfsys@transformshift{0.778610in}{1.568891in}%
\pgfsys@useobject{currentmarker}{}%
\end{pgfscope}%
\begin{pgfscope}%
\pgfsys@transformshift{1.678430in}{0.856308in}%
\pgfsys@useobject{currentmarker}{}%
\end{pgfscope}%
\begin{pgfscope}%
\pgfsys@transformshift{1.678534in}{0.939082in}%
\pgfsys@useobject{currentmarker}{}%
\end{pgfscope}%
\begin{pgfscope}%
\pgfsys@transformshift{1.164510in}{0.816378in}%
\pgfsys@useobject{currentmarker}{}%
\end{pgfscope}%
\begin{pgfscope}%
\pgfsys@transformshift{1.533885in}{0.772783in}%
\pgfsys@useobject{currentmarker}{}%
\end{pgfscope}%
\begin{pgfscope}%
\pgfsys@transformshift{0.532239in}{1.272907in}%
\pgfsys@useobject{currentmarker}{}%
\end{pgfscope}%
\begin{pgfscope}%
\pgfsys@transformshift{1.059630in}{0.825513in}%
\pgfsys@useobject{currentmarker}{}%
\end{pgfscope}%
\begin{pgfscope}%
\pgfsys@transformshift{1.471016in}{0.845010in}%
\pgfsys@useobject{currentmarker}{}%
\end{pgfscope}%
\begin{pgfscope}%
\pgfsys@transformshift{0.764422in}{1.315133in}%
\pgfsys@useobject{currentmarker}{}%
\end{pgfscope}%
\begin{pgfscope}%
\pgfsys@transformshift{1.724626in}{0.753193in}%
\pgfsys@useobject{currentmarker}{}%
\end{pgfscope}%
\begin{pgfscope}%
\pgfsys@transformshift{1.557866in}{0.668006in}%
\pgfsys@useobject{currentmarker}{}%
\end{pgfscope}%
\begin{pgfscope}%
\pgfsys@transformshift{1.322665in}{0.889238in}%
\pgfsys@useobject{currentmarker}{}%
\end{pgfscope}%
\begin{pgfscope}%
\pgfsys@transformshift{1.043939in}{1.047546in}%
\pgfsys@useobject{currentmarker}{}%
\end{pgfscope}%
\begin{pgfscope}%
\pgfsys@transformshift{1.615078in}{0.852044in}%
\pgfsys@useobject{currentmarker}{}%
\end{pgfscope}%
\begin{pgfscope}%
\pgfsys@transformshift{1.467591in}{0.942435in}%
\pgfsys@useobject{currentmarker}{}%
\end{pgfscope}%
\begin{pgfscope}%
\pgfsys@transformshift{0.991846in}{0.869448in}%
\pgfsys@useobject{currentmarker}{}%
\end{pgfscope}%
\begin{pgfscope}%
\pgfsys@transformshift{0.697891in}{1.106037in}%
\pgfsys@useobject{currentmarker}{}%
\end{pgfscope}%
\begin{pgfscope}%
\pgfsys@transformshift{1.294350in}{1.118680in}%
\pgfsys@useobject{currentmarker}{}%
\end{pgfscope}%
\begin{pgfscope}%
\pgfsys@transformshift{1.409354in}{0.650148in}%
\pgfsys@useobject{currentmarker}{}%
\end{pgfscope}%
\begin{pgfscope}%
\pgfsys@transformshift{0.733011in}{1.712593in}%
\pgfsys@useobject{currentmarker}{}%
\end{pgfscope}%
\begin{pgfscope}%
\pgfsys@transformshift{0.634257in}{1.199631in}%
\pgfsys@useobject{currentmarker}{}%
\end{pgfscope}%
\begin{pgfscope}%
\pgfsys@transformshift{0.526826in}{1.648129in}%
\pgfsys@useobject{currentmarker}{}%
\end{pgfscope}%
\begin{pgfscope}%
\pgfsys@transformshift{1.225515in}{1.299700in}%
\pgfsys@useobject{currentmarker}{}%
\end{pgfscope}%
\begin{pgfscope}%
\pgfsys@transformshift{0.578368in}{0.887206in}%
\pgfsys@useobject{currentmarker}{}%
\end{pgfscope}%
\begin{pgfscope}%
\pgfsys@transformshift{1.515054in}{0.872511in}%
\pgfsys@useobject{currentmarker}{}%
\end{pgfscope}%
\begin{pgfscope}%
\pgfsys@transformshift{1.989373in}{0.872944in}%
\pgfsys@useobject{currentmarker}{}%
\end{pgfscope}%
\begin{pgfscope}%
\pgfsys@transformshift{1.360084in}{1.081775in}%
\pgfsys@useobject{currentmarker}{}%
\end{pgfscope}%
\begin{pgfscope}%
\pgfsys@transformshift{0.851541in}{1.331925in}%
\pgfsys@useobject{currentmarker}{}%
\end{pgfscope}%
\begin{pgfscope}%
\pgfsys@transformshift{1.688966in}{0.779334in}%
\pgfsys@useobject{currentmarker}{}%
\end{pgfscope}%
\begin{pgfscope}%
\pgfsys@transformshift{1.144955in}{0.988055in}%
\pgfsys@useobject{currentmarker}{}%
\end{pgfscope}%
\begin{pgfscope}%
\pgfsys@transformshift{1.681418in}{0.693383in}%
\pgfsys@useobject{currentmarker}{}%
\end{pgfscope}%
\begin{pgfscope}%
\pgfsys@transformshift{1.133992in}{1.098713in}%
\pgfsys@useobject{currentmarker}{}%
\end{pgfscope}%
\begin{pgfscope}%
\pgfsys@transformshift{1.480556in}{0.918766in}%
\pgfsys@useobject{currentmarker}{}%
\end{pgfscope}%
\begin{pgfscope}%
\pgfsys@transformshift{1.339321in}{0.887962in}%
\pgfsys@useobject{currentmarker}{}%
\end{pgfscope}%
\begin{pgfscope}%
\pgfsys@transformshift{0.874156in}{1.146494in}%
\pgfsys@useobject{currentmarker}{}%
\end{pgfscope}%
\begin{pgfscope}%
\pgfsys@transformshift{1.592077in}{0.601406in}%
\pgfsys@useobject{currentmarker}{}%
\end{pgfscope}%
\begin{pgfscope}%
\pgfsys@transformshift{1.288980in}{1.024668in}%
\pgfsys@useobject{currentmarker}{}%
\end{pgfscope}%
\begin{pgfscope}%
\pgfsys@transformshift{0.987975in}{0.796139in}%
\pgfsys@useobject{currentmarker}{}%
\end{pgfscope}%
\begin{pgfscope}%
\pgfsys@transformshift{1.714750in}{0.849831in}%
\pgfsys@useobject{currentmarker}{}%
\end{pgfscope}%
\begin{pgfscope}%
\pgfsys@transformshift{1.365090in}{0.736032in}%
\pgfsys@useobject{currentmarker}{}%
\end{pgfscope}%
\begin{pgfscope}%
\pgfsys@transformshift{1.930143in}{0.995665in}%
\pgfsys@useobject{currentmarker}{}%
\end{pgfscope}%
\begin{pgfscope}%
\pgfsys@transformshift{1.850728in}{0.829047in}%
\pgfsys@useobject{currentmarker}{}%
\end{pgfscope}%
\begin{pgfscope}%
\pgfsys@transformshift{0.856167in}{1.125165in}%
\pgfsys@useobject{currentmarker}{}%
\end{pgfscope}%
\begin{pgfscope}%
\pgfsys@transformshift{1.724113in}{0.888463in}%
\pgfsys@useobject{currentmarker}{}%
\end{pgfscope}%
\begin{pgfscope}%
\pgfsys@transformshift{0.746656in}{1.252349in}%
\pgfsys@useobject{currentmarker}{}%
\end{pgfscope}%
\begin{pgfscope}%
\pgfsys@transformshift{0.981125in}{1.067944in}%
\pgfsys@useobject{currentmarker}{}%
\end{pgfscope}%
\begin{pgfscope}%
\pgfsys@transformshift{0.840360in}{1.193219in}%
\pgfsys@useobject{currentmarker}{}%
\end{pgfscope}%
\begin{pgfscope}%
\pgfsys@transformshift{0.780491in}{1.263093in}%
\pgfsys@useobject{currentmarker}{}%
\end{pgfscope}%
\begin{pgfscope}%
\pgfsys@transformshift{1.873816in}{0.977234in}%
\pgfsys@useobject{currentmarker}{}%
\end{pgfscope}%
\begin{pgfscope}%
\pgfsys@transformshift{1.594601in}{0.840903in}%
\pgfsys@useobject{currentmarker}{}%
\end{pgfscope}%
\begin{pgfscope}%
\pgfsys@transformshift{0.815736in}{1.177549in}%
\pgfsys@useobject{currentmarker}{}%
\end{pgfscope}%
\begin{pgfscope}%
\pgfsys@transformshift{1.864796in}{0.670195in}%
\pgfsys@useobject{currentmarker}{}%
\end{pgfscope}%
\begin{pgfscope}%
\pgfsys@transformshift{1.367631in}{1.029347in}%
\pgfsys@useobject{currentmarker}{}%
\end{pgfscope}%
\begin{pgfscope}%
\pgfsys@transformshift{1.653345in}{0.949671in}%
\pgfsys@useobject{currentmarker}{}%
\end{pgfscope}%
\begin{pgfscope}%
\pgfsys@transformshift{1.768545in}{0.853569in}%
\pgfsys@useobject{currentmarker}{}%
\end{pgfscope}%
\begin{pgfscope}%
\pgfsys@transformshift{1.244730in}{0.882186in}%
\pgfsys@useobject{currentmarker}{}%
\end{pgfscope}%
\begin{pgfscope}%
\pgfsys@transformshift{0.782047in}{1.211739in}%
\pgfsys@useobject{currentmarker}{}%
\end{pgfscope}%
\begin{pgfscope}%
\pgfsys@transformshift{1.846677in}{1.032130in}%
\pgfsys@useobject{currentmarker}{}%
\end{pgfscope}%
\begin{pgfscope}%
\pgfsys@transformshift{1.775739in}{0.819469in}%
\pgfsys@useobject{currentmarker}{}%
\end{pgfscope}%
\begin{pgfscope}%
\pgfsys@transformshift{1.703721in}{0.899749in}%
\pgfsys@useobject{currentmarker}{}%
\end{pgfscope}%
\begin{pgfscope}%
\pgfsys@transformshift{0.834121in}{1.274367in}%
\pgfsys@useobject{currentmarker}{}%
\end{pgfscope}%
\begin{pgfscope}%
\pgfsys@transformshift{1.119982in}{1.453499in}%
\pgfsys@useobject{currentmarker}{}%
\end{pgfscope}%
\begin{pgfscope}%
\pgfsys@transformshift{1.891923in}{0.885221in}%
\pgfsys@useobject{currentmarker}{}%
\end{pgfscope}%
\begin{pgfscope}%
\pgfsys@transformshift{1.690465in}{0.582225in}%
\pgfsys@useobject{currentmarker}{}%
\end{pgfscope}%
\begin{pgfscope}%
\pgfsys@transformshift{0.559984in}{1.323887in}%
\pgfsys@useobject{currentmarker}{}%
\end{pgfscope}%
\begin{pgfscope}%
\pgfsys@transformshift{1.447613in}{0.797293in}%
\pgfsys@useobject{currentmarker}{}%
\end{pgfscope}%
\begin{pgfscope}%
\pgfsys@transformshift{0.897928in}{1.405797in}%
\pgfsys@useobject{currentmarker}{}%
\end{pgfscope}%
\begin{pgfscope}%
\pgfsys@transformshift{1.417775in}{0.957063in}%
\pgfsys@useobject{currentmarker}{}%
\end{pgfscope}%
\begin{pgfscope}%
\pgfsys@transformshift{1.673252in}{0.669659in}%
\pgfsys@useobject{currentmarker}{}%
\end{pgfscope}%
\begin{pgfscope}%
\pgfsys@transformshift{1.412143in}{0.691703in}%
\pgfsys@useobject{currentmarker}{}%
\end{pgfscope}%
\begin{pgfscope}%
\pgfsys@transformshift{1.641912in}{0.803658in}%
\pgfsys@useobject{currentmarker}{}%
\end{pgfscope}%
\begin{pgfscope}%
\pgfsys@transformshift{1.388318in}{1.205041in}%
\pgfsys@useobject{currentmarker}{}%
\end{pgfscope}%
\begin{pgfscope}%
\pgfsys@transformshift{1.495487in}{0.727725in}%
\pgfsys@useobject{currentmarker}{}%
\end{pgfscope}%
\begin{pgfscope}%
\pgfsys@transformshift{1.918935in}{0.727118in}%
\pgfsys@useobject{currentmarker}{}%
\end{pgfscope}%
\begin{pgfscope}%
\pgfsys@transformshift{0.550025in}{1.118595in}%
\pgfsys@useobject{currentmarker}{}%
\end{pgfscope}%
\begin{pgfscope}%
\pgfsys@transformshift{1.821318in}{0.862274in}%
\pgfsys@useobject{currentmarker}{}%
\end{pgfscope}%
\begin{pgfscope}%
\pgfsys@transformshift{0.956955in}{0.969980in}%
\pgfsys@useobject{currentmarker}{}%
\end{pgfscope}%
\begin{pgfscope}%
\pgfsys@transformshift{0.895984in}{1.364034in}%
\pgfsys@useobject{currentmarker}{}%
\end{pgfscope}%
\begin{pgfscope}%
\pgfsys@transformshift{0.467116in}{0.898925in}%
\pgfsys@useobject{currentmarker}{}%
\end{pgfscope}%
\begin{pgfscope}%
\pgfsys@transformshift{1.606478in}{0.846292in}%
\pgfsys@useobject{currentmarker}{}%
\end{pgfscope}%
\begin{pgfscope}%
\pgfsys@transformshift{1.423682in}{0.906428in}%
\pgfsys@useobject{currentmarker}{}%
\end{pgfscope}%
\begin{pgfscope}%
\pgfsys@transformshift{1.436312in}{1.251879in}%
\pgfsys@useobject{currentmarker}{}%
\end{pgfscope}%
\begin{pgfscope}%
\pgfsys@transformshift{0.970257in}{1.432334in}%
\pgfsys@useobject{currentmarker}{}%
\end{pgfscope}%
\begin{pgfscope}%
\pgfsys@transformshift{1.482495in}{0.914756in}%
\pgfsys@useobject{currentmarker}{}%
\end{pgfscope}%
\begin{pgfscope}%
\pgfsys@transformshift{1.363658in}{0.956451in}%
\pgfsys@useobject{currentmarker}{}%
\end{pgfscope}%
\begin{pgfscope}%
\pgfsys@transformshift{1.566350in}{0.947260in}%
\pgfsys@useobject{currentmarker}{}%
\end{pgfscope}%
\begin{pgfscope}%
\pgfsys@transformshift{0.737168in}{1.150991in}%
\pgfsys@useobject{currentmarker}{}%
\end{pgfscope}%
\begin{pgfscope}%
\pgfsys@transformshift{0.416424in}{0.932008in}%
\pgfsys@useobject{currentmarker}{}%
\end{pgfscope}%
\begin{pgfscope}%
\pgfsys@transformshift{0.524334in}{1.102919in}%
\pgfsys@useobject{currentmarker}{}%
\end{pgfscope}%
\begin{pgfscope}%
\pgfsys@transformshift{0.739043in}{0.820881in}%
\pgfsys@useobject{currentmarker}{}%
\end{pgfscope}%
\begin{pgfscope}%
\pgfsys@transformshift{1.840100in}{0.859270in}%
\pgfsys@useobject{currentmarker}{}%
\end{pgfscope}%
\begin{pgfscope}%
\pgfsys@transformshift{1.824568in}{1.045270in}%
\pgfsys@useobject{currentmarker}{}%
\end{pgfscope}%
\begin{pgfscope}%
\pgfsys@transformshift{1.470009in}{1.038977in}%
\pgfsys@useobject{currentmarker}{}%
\end{pgfscope}%
\begin{pgfscope}%
\pgfsys@transformshift{0.471982in}{1.013864in}%
\pgfsys@useobject{currentmarker}{}%
\end{pgfscope}%
\begin{pgfscope}%
\pgfsys@transformshift{1.573451in}{1.009521in}%
\pgfsys@useobject{currentmarker}{}%
\end{pgfscope}%
\begin{pgfscope}%
\pgfsys@transformshift{1.738247in}{0.787463in}%
\pgfsys@useobject{currentmarker}{}%
\end{pgfscope}%
\begin{pgfscope}%
\pgfsys@transformshift{1.607428in}{0.681402in}%
\pgfsys@useobject{currentmarker}{}%
\end{pgfscope}%
\begin{pgfscope}%
\pgfsys@transformshift{1.759927in}{0.635702in}%
\pgfsys@useobject{currentmarker}{}%
\end{pgfscope}%
\begin{pgfscope}%
\pgfsys@transformshift{1.551624in}{0.819658in}%
\pgfsys@useobject{currentmarker}{}%
\end{pgfscope}%
\begin{pgfscope}%
\pgfsys@transformshift{1.356607in}{0.885826in}%
\pgfsys@useobject{currentmarker}{}%
\end{pgfscope}%
\begin{pgfscope}%
\pgfsys@transformshift{1.649686in}{0.586059in}%
\pgfsys@useobject{currentmarker}{}%
\end{pgfscope}%
\begin{pgfscope}%
\pgfsys@transformshift{1.305691in}{0.787677in}%
\pgfsys@useobject{currentmarker}{}%
\end{pgfscope}%
\begin{pgfscope}%
\pgfsys@transformshift{1.393587in}{0.674819in}%
\pgfsys@useobject{currentmarker}{}%
\end{pgfscope}%
\begin{pgfscope}%
\pgfsys@transformshift{1.153074in}{0.766359in}%
\pgfsys@useobject{currentmarker}{}%
\end{pgfscope}%
\begin{pgfscope}%
\pgfsys@transformshift{1.150217in}{1.365401in}%
\pgfsys@useobject{currentmarker}{}%
\end{pgfscope}%
\begin{pgfscope}%
\pgfsys@transformshift{0.541563in}{0.818412in}%
\pgfsys@useobject{currentmarker}{}%
\end{pgfscope}%
\begin{pgfscope}%
\pgfsys@transformshift{0.907947in}{1.065313in}%
\pgfsys@useobject{currentmarker}{}%
\end{pgfscope}%
\begin{pgfscope}%
\pgfsys@transformshift{1.404883in}{0.817184in}%
\pgfsys@useobject{currentmarker}{}%
\end{pgfscope}%
\begin{pgfscope}%
\pgfsys@transformshift{1.766444in}{0.783834in}%
\pgfsys@useobject{currentmarker}{}%
\end{pgfscope}%
\begin{pgfscope}%
\pgfsys@transformshift{1.727091in}{0.856021in}%
\pgfsys@useobject{currentmarker}{}%
\end{pgfscope}%
\begin{pgfscope}%
\pgfsys@transformshift{1.794191in}{0.883450in}%
\pgfsys@useobject{currentmarker}{}%
\end{pgfscope}%
\begin{pgfscope}%
\pgfsys@transformshift{1.006287in}{0.971186in}%
\pgfsys@useobject{currentmarker}{}%
\end{pgfscope}%
\begin{pgfscope}%
\pgfsys@transformshift{1.581569in}{0.778855in}%
\pgfsys@useobject{currentmarker}{}%
\end{pgfscope}%
\begin{pgfscope}%
\pgfsys@transformshift{0.941329in}{1.084691in}%
\pgfsys@useobject{currentmarker}{}%
\end{pgfscope}%
\begin{pgfscope}%
\pgfsys@transformshift{0.691952in}{1.280587in}%
\pgfsys@useobject{currentmarker}{}%
\end{pgfscope}%
\begin{pgfscope}%
\pgfsys@transformshift{1.057616in}{1.154528in}%
\pgfsys@useobject{currentmarker}{}%
\end{pgfscope}%
\begin{pgfscope}%
\pgfsys@transformshift{1.673517in}{0.866732in}%
\pgfsys@useobject{currentmarker}{}%
\end{pgfscope}%
\begin{pgfscope}%
\pgfsys@transformshift{1.477276in}{1.064507in}%
\pgfsys@useobject{currentmarker}{}%
\end{pgfscope}%
\begin{pgfscope}%
\pgfsys@transformshift{1.418685in}{0.860019in}%
\pgfsys@useobject{currentmarker}{}%
\end{pgfscope}%
\begin{pgfscope}%
\pgfsys@transformshift{1.590325in}{0.666768in}%
\pgfsys@useobject{currentmarker}{}%
\end{pgfscope}%
\begin{pgfscope}%
\pgfsys@transformshift{0.779671in}{1.148717in}%
\pgfsys@useobject{currentmarker}{}%
\end{pgfscope}%
\begin{pgfscope}%
\pgfsys@transformshift{1.674297in}{0.871795in}%
\pgfsys@useobject{currentmarker}{}%
\end{pgfscope}%
\begin{pgfscope}%
\pgfsys@transformshift{0.676244in}{1.293788in}%
\pgfsys@useobject{currentmarker}{}%
\end{pgfscope}%
\begin{pgfscope}%
\pgfsys@transformshift{1.715152in}{0.817928in}%
\pgfsys@useobject{currentmarker}{}%
\end{pgfscope}%
\begin{pgfscope}%
\pgfsys@transformshift{0.491079in}{0.879999in}%
\pgfsys@useobject{currentmarker}{}%
\end{pgfscope}%
\begin{pgfscope}%
\pgfsys@transformshift{1.084356in}{1.135796in}%
\pgfsys@useobject{currentmarker}{}%
\end{pgfscope}%
\begin{pgfscope}%
\pgfsys@transformshift{0.525316in}{0.869936in}%
\pgfsys@useobject{currentmarker}{}%
\end{pgfscope}%
\begin{pgfscope}%
\pgfsys@transformshift{0.530939in}{0.884771in}%
\pgfsys@useobject{currentmarker}{}%
\end{pgfscope}%
\begin{pgfscope}%
\pgfsys@transformshift{0.632029in}{1.678112in}%
\pgfsys@useobject{currentmarker}{}%
\end{pgfscope}%
\begin{pgfscope}%
\pgfsys@transformshift{0.952115in}{1.247029in}%
\pgfsys@useobject{currentmarker}{}%
\end{pgfscope}%
\begin{pgfscope}%
\pgfsys@transformshift{1.781361in}{0.641639in}%
\pgfsys@useobject{currentmarker}{}%
\end{pgfscope}%
\begin{pgfscope}%
\pgfsys@transformshift{0.893435in}{1.247090in}%
\pgfsys@useobject{currentmarker}{}%
\end{pgfscope}%
\begin{pgfscope}%
\pgfsys@transformshift{1.630080in}{0.795480in}%
\pgfsys@useobject{currentmarker}{}%
\end{pgfscope}%
\begin{pgfscope}%
\pgfsys@transformshift{1.495001in}{0.905080in}%
\pgfsys@useobject{currentmarker}{}%
\end{pgfscope}%
\begin{pgfscope}%
\pgfsys@transformshift{1.807197in}{0.908602in}%
\pgfsys@useobject{currentmarker}{}%
\end{pgfscope}%
\begin{pgfscope}%
\pgfsys@transformshift{0.872166in}{1.267818in}%
\pgfsys@useobject{currentmarker}{}%
\end{pgfscope}%
\begin{pgfscope}%
\pgfsys@transformshift{1.447090in}{1.014549in}%
\pgfsys@useobject{currentmarker}{}%
\end{pgfscope}%
\begin{pgfscope}%
\pgfsys@transformshift{1.534740in}{1.017134in}%
\pgfsys@useobject{currentmarker}{}%
\end{pgfscope}%
\begin{pgfscope}%
\pgfsys@transformshift{0.648228in}{1.775666in}%
\pgfsys@useobject{currentmarker}{}%
\end{pgfscope}%
\begin{pgfscope}%
\pgfsys@transformshift{0.939750in}{0.920289in}%
\pgfsys@useobject{currentmarker}{}%
\end{pgfscope}%
\begin{pgfscope}%
\pgfsys@transformshift{0.437103in}{0.905589in}%
\pgfsys@useobject{currentmarker}{}%
\end{pgfscope}%
\begin{pgfscope}%
\pgfsys@transformshift{1.534965in}{0.807376in}%
\pgfsys@useobject{currentmarker}{}%
\end{pgfscope}%
\begin{pgfscope}%
\pgfsys@transformshift{1.227246in}{1.020336in}%
\pgfsys@useobject{currentmarker}{}%
\end{pgfscope}%
\begin{pgfscope}%
\pgfsys@transformshift{1.816498in}{0.805812in}%
\pgfsys@useobject{currentmarker}{}%
\end{pgfscope}%
\begin{pgfscope}%
\pgfsys@transformshift{0.733769in}{1.485511in}%
\pgfsys@useobject{currentmarker}{}%
\end{pgfscope}%
\begin{pgfscope}%
\pgfsys@transformshift{0.664484in}{1.226139in}%
\pgfsys@useobject{currentmarker}{}%
\end{pgfscope}%
\begin{pgfscope}%
\pgfsys@transformshift{1.076644in}{1.084409in}%
\pgfsys@useobject{currentmarker}{}%
\end{pgfscope}%
\begin{pgfscope}%
\pgfsys@transformshift{1.548521in}{0.975554in}%
\pgfsys@useobject{currentmarker}{}%
\end{pgfscope}%
\begin{pgfscope}%
\pgfsys@transformshift{0.536243in}{1.551429in}%
\pgfsys@useobject{currentmarker}{}%
\end{pgfscope}%
\begin{pgfscope}%
\pgfsys@transformshift{0.889736in}{1.166492in}%
\pgfsys@useobject{currentmarker}{}%
\end{pgfscope}%
\begin{pgfscope}%
\pgfsys@transformshift{1.818644in}{0.922740in}%
\pgfsys@useobject{currentmarker}{}%
\end{pgfscope}%
\begin{pgfscope}%
\pgfsys@transformshift{1.073205in}{1.306969in}%
\pgfsys@useobject{currentmarker}{}%
\end{pgfscope}%
\begin{pgfscope}%
\pgfsys@transformshift{0.882269in}{1.497703in}%
\pgfsys@useobject{currentmarker}{}%
\end{pgfscope}%
\begin{pgfscope}%
\pgfsys@transformshift{0.755753in}{1.236987in}%
\pgfsys@useobject{currentmarker}{}%
\end{pgfscope}%
\begin{pgfscope}%
\pgfsys@transformshift{1.318495in}{0.959274in}%
\pgfsys@useobject{currentmarker}{}%
\end{pgfscope}%
\begin{pgfscope}%
\pgfsys@transformshift{1.102293in}{0.890897in}%
\pgfsys@useobject{currentmarker}{}%
\end{pgfscope}%
\begin{pgfscope}%
\pgfsys@transformshift{1.814698in}{0.930269in}%
\pgfsys@useobject{currentmarker}{}%
\end{pgfscope}%
\begin{pgfscope}%
\pgfsys@transformshift{1.595111in}{0.683753in}%
\pgfsys@useobject{currentmarker}{}%
\end{pgfscope}%
\begin{pgfscope}%
\pgfsys@transformshift{0.726840in}{0.943074in}%
\pgfsys@useobject{currentmarker}{}%
\end{pgfscope}%
\begin{pgfscope}%
\pgfsys@transformshift{1.492602in}{0.936265in}%
\pgfsys@useobject{currentmarker}{}%
\end{pgfscope}%
\begin{pgfscope}%
\pgfsys@transformshift{1.393179in}{0.767110in}%
\pgfsys@useobject{currentmarker}{}%
\end{pgfscope}%
\begin{pgfscope}%
\pgfsys@transformshift{1.166943in}{0.810567in}%
\pgfsys@useobject{currentmarker}{}%
\end{pgfscope}%
\begin{pgfscope}%
\pgfsys@transformshift{1.444452in}{0.747792in}%
\pgfsys@useobject{currentmarker}{}%
\end{pgfscope}%
\begin{pgfscope}%
\pgfsys@transformshift{1.693005in}{0.831255in}%
\pgfsys@useobject{currentmarker}{}%
\end{pgfscope}%
\begin{pgfscope}%
\pgfsys@transformshift{0.817674in}{0.901217in}%
\pgfsys@useobject{currentmarker}{}%
\end{pgfscope}%
\begin{pgfscope}%
\pgfsys@transformshift{0.521183in}{1.668583in}%
\pgfsys@useobject{currentmarker}{}%
\end{pgfscope}%
\begin{pgfscope}%
\pgfsys@transformshift{1.298494in}{1.032290in}%
\pgfsys@useobject{currentmarker}{}%
\end{pgfscope}%
\begin{pgfscope}%
\pgfsys@transformshift{0.537740in}{1.266114in}%
\pgfsys@useobject{currentmarker}{}%
\end{pgfscope}%
\begin{pgfscope}%
\pgfsys@transformshift{1.711514in}{1.022285in}%
\pgfsys@useobject{currentmarker}{}%
\end{pgfscope}%
\begin{pgfscope}%
\pgfsys@transformshift{1.607575in}{0.833123in}%
\pgfsys@useobject{currentmarker}{}%
\end{pgfscope}%
\begin{pgfscope}%
\pgfsys@transformshift{1.569340in}{0.899315in}%
\pgfsys@useobject{currentmarker}{}%
\end{pgfscope}%
\begin{pgfscope}%
\pgfsys@transformshift{0.423151in}{1.270112in}%
\pgfsys@useobject{currentmarker}{}%
\end{pgfscope}%
\begin{pgfscope}%
\pgfsys@transformshift{0.636198in}{0.812136in}%
\pgfsys@useobject{currentmarker}{}%
\end{pgfscope}%
\begin{pgfscope}%
\pgfsys@transformshift{1.742823in}{0.604278in}%
\pgfsys@useobject{currentmarker}{}%
\end{pgfscope}%
\begin{pgfscope}%
\pgfsys@transformshift{0.849878in}{1.265321in}%
\pgfsys@useobject{currentmarker}{}%
\end{pgfscope}%
\begin{pgfscope}%
\pgfsys@transformshift{1.353468in}{0.962529in}%
\pgfsys@useobject{currentmarker}{}%
\end{pgfscope}%
\begin{pgfscope}%
\pgfsys@transformshift{1.622982in}{0.544413in}%
\pgfsys@useobject{currentmarker}{}%
\end{pgfscope}%
\begin{pgfscope}%
\pgfsys@transformshift{1.599783in}{0.596888in}%
\pgfsys@useobject{currentmarker}{}%
\end{pgfscope}%
\begin{pgfscope}%
\pgfsys@transformshift{1.895187in}{0.755347in}%
\pgfsys@useobject{currentmarker}{}%
\end{pgfscope}%
\begin{pgfscope}%
\pgfsys@transformshift{1.129530in}{1.352853in}%
\pgfsys@useobject{currentmarker}{}%
\end{pgfscope}%
\begin{pgfscope}%
\pgfsys@transformshift{0.991382in}{0.898977in}%
\pgfsys@useobject{currentmarker}{}%
\end{pgfscope}%
\begin{pgfscope}%
\pgfsys@transformshift{0.795664in}{1.759871in}%
\pgfsys@useobject{currentmarker}{}%
\end{pgfscope}%
\begin{pgfscope}%
\pgfsys@transformshift{1.780025in}{0.778722in}%
\pgfsys@useobject{currentmarker}{}%
\end{pgfscope}%
\begin{pgfscope}%
\pgfsys@transformshift{1.712749in}{0.906774in}%
\pgfsys@useobject{currentmarker}{}%
\end{pgfscope}%
\begin{pgfscope}%
\pgfsys@transformshift{1.145211in}{1.109821in}%
\pgfsys@useobject{currentmarker}{}%
\end{pgfscope}%
\begin{pgfscope}%
\pgfsys@transformshift{1.328349in}{0.799643in}%
\pgfsys@useobject{currentmarker}{}%
\end{pgfscope}%
\begin{pgfscope}%
\pgfsys@transformshift{1.405073in}{0.793095in}%
\pgfsys@useobject{currentmarker}{}%
\end{pgfscope}%
\begin{pgfscope}%
\pgfsys@transformshift{0.669867in}{1.251236in}%
\pgfsys@useobject{currentmarker}{}%
\end{pgfscope}%
\begin{pgfscope}%
\pgfsys@transformshift{1.898744in}{0.765754in}%
\pgfsys@useobject{currentmarker}{}%
\end{pgfscope}%
\begin{pgfscope}%
\pgfsys@transformshift{0.874940in}{1.065855in}%
\pgfsys@useobject{currentmarker}{}%
\end{pgfscope}%
\begin{pgfscope}%
\pgfsys@transformshift{1.459937in}{1.340388in}%
\pgfsys@useobject{currentmarker}{}%
\end{pgfscope}%
\begin{pgfscope}%
\pgfsys@transformshift{1.895928in}{0.754729in}%
\pgfsys@useobject{currentmarker}{}%
\end{pgfscope}%
\begin{pgfscope}%
\pgfsys@transformshift{0.853418in}{1.228861in}%
\pgfsys@useobject{currentmarker}{}%
\end{pgfscope}%
\begin{pgfscope}%
\pgfsys@transformshift{0.717208in}{1.295959in}%
\pgfsys@useobject{currentmarker}{}%
\end{pgfscope}%
\begin{pgfscope}%
\pgfsys@transformshift{1.075298in}{0.938317in}%
\pgfsys@useobject{currentmarker}{}%
\end{pgfscope}%
\begin{pgfscope}%
\pgfsys@transformshift{1.920218in}{1.239057in}%
\pgfsys@useobject{currentmarker}{}%
\end{pgfscope}%
\begin{pgfscope}%
\pgfsys@transformshift{0.377829in}{1.049835in}%
\pgfsys@useobject{currentmarker}{}%
\end{pgfscope}%
\begin{pgfscope}%
\pgfsys@transformshift{1.155745in}{1.214818in}%
\pgfsys@useobject{currentmarker}{}%
\end{pgfscope}%
\begin{pgfscope}%
\pgfsys@transformshift{1.712524in}{0.759478in}%
\pgfsys@useobject{currentmarker}{}%
\end{pgfscope}%
\begin{pgfscope}%
\pgfsys@transformshift{0.534596in}{1.262146in}%
\pgfsys@useobject{currentmarker}{}%
\end{pgfscope}%
\begin{pgfscope}%
\pgfsys@transformshift{0.622741in}{1.390862in}%
\pgfsys@useobject{currentmarker}{}%
\end{pgfscope}%
\begin{pgfscope}%
\pgfsys@transformshift{0.575527in}{1.567422in}%
\pgfsys@useobject{currentmarker}{}%
\end{pgfscope}%
\begin{pgfscope}%
\pgfsys@transformshift{1.804863in}{0.649169in}%
\pgfsys@useobject{currentmarker}{}%
\end{pgfscope}%
\begin{pgfscope}%
\pgfsys@transformshift{1.749745in}{0.727041in}%
\pgfsys@useobject{currentmarker}{}%
\end{pgfscope}%
\begin{pgfscope}%
\pgfsys@transformshift{1.674396in}{0.921657in}%
\pgfsys@useobject{currentmarker}{}%
\end{pgfscope}%
\begin{pgfscope}%
\pgfsys@transformshift{1.859027in}{0.731934in}%
\pgfsys@useobject{currentmarker}{}%
\end{pgfscope}%
\begin{pgfscope}%
\pgfsys@transformshift{1.603827in}{0.665348in}%
\pgfsys@useobject{currentmarker}{}%
\end{pgfscope}%
\begin{pgfscope}%
\pgfsys@transformshift{1.255917in}{0.979346in}%
\pgfsys@useobject{currentmarker}{}%
\end{pgfscope}%
\begin{pgfscope}%
\pgfsys@transformshift{0.419746in}{0.919529in}%
\pgfsys@useobject{currentmarker}{}%
\end{pgfscope}%
\begin{pgfscope}%
\pgfsys@transformshift{0.903589in}{0.876338in}%
\pgfsys@useobject{currentmarker}{}%
\end{pgfscope}%
\begin{pgfscope}%
\pgfsys@transformshift{1.411837in}{0.923698in}%
\pgfsys@useobject{currentmarker}{}%
\end{pgfscope}%
\begin{pgfscope}%
\pgfsys@transformshift{1.888179in}{0.786283in}%
\pgfsys@useobject{currentmarker}{}%
\end{pgfscope}%
\begin{pgfscope}%
\pgfsys@transformshift{1.463948in}{0.826534in}%
\pgfsys@useobject{currentmarker}{}%
\end{pgfscope}%
\begin{pgfscope}%
\pgfsys@transformshift{1.037147in}{0.917603in}%
\pgfsys@useobject{currentmarker}{}%
\end{pgfscope}%
\begin{pgfscope}%
\pgfsys@transformshift{1.744144in}{0.584143in}%
\pgfsys@useobject{currentmarker}{}%
\end{pgfscope}%
\begin{pgfscope}%
\pgfsys@transformshift{0.867355in}{0.982965in}%
\pgfsys@useobject{currentmarker}{}%
\end{pgfscope}%
\begin{pgfscope}%
\pgfsys@transformshift{1.331798in}{1.025957in}%
\pgfsys@useobject{currentmarker}{}%
\end{pgfscope}%
\begin{pgfscope}%
\pgfsys@transformshift{0.968745in}{1.157914in}%
\pgfsys@useobject{currentmarker}{}%
\end{pgfscope}%
\begin{pgfscope}%
\pgfsys@transformshift{1.193049in}{1.445327in}%
\pgfsys@useobject{currentmarker}{}%
\end{pgfscope}%
\begin{pgfscope}%
\pgfsys@transformshift{1.664246in}{0.754127in}%
\pgfsys@useobject{currentmarker}{}%
\end{pgfscope}%
\begin{pgfscope}%
\pgfsys@transformshift{0.835587in}{1.237600in}%
\pgfsys@useobject{currentmarker}{}%
\end{pgfscope}%
\begin{pgfscope}%
\pgfsys@transformshift{1.701974in}{0.715666in}%
\pgfsys@useobject{currentmarker}{}%
\end{pgfscope}%
\begin{pgfscope}%
\pgfsys@transformshift{1.269033in}{0.940103in}%
\pgfsys@useobject{currentmarker}{}%
\end{pgfscope}%
\begin{pgfscope}%
\pgfsys@transformshift{1.329352in}{0.747421in}%
\pgfsys@useobject{currentmarker}{}%
\end{pgfscope}%
\begin{pgfscope}%
\pgfsys@transformshift{1.471661in}{1.201453in}%
\pgfsys@useobject{currentmarker}{}%
\end{pgfscope}%
\begin{pgfscope}%
\pgfsys@transformshift{1.916157in}{1.119294in}%
\pgfsys@useobject{currentmarker}{}%
\end{pgfscope}%
\begin{pgfscope}%
\pgfsys@transformshift{1.533379in}{0.710070in}%
\pgfsys@useobject{currentmarker}{}%
\end{pgfscope}%
\begin{pgfscope}%
\pgfsys@transformshift{1.358583in}{0.841886in}%
\pgfsys@useobject{currentmarker}{}%
\end{pgfscope}%
\begin{pgfscope}%
\pgfsys@transformshift{1.599245in}{0.719730in}%
\pgfsys@useobject{currentmarker}{}%
\end{pgfscope}%
\begin{pgfscope}%
\pgfsys@transformshift{0.703172in}{1.367920in}%
\pgfsys@useobject{currentmarker}{}%
\end{pgfscope}%
\begin{pgfscope}%
\pgfsys@transformshift{1.320412in}{0.994846in}%
\pgfsys@useobject{currentmarker}{}%
\end{pgfscope}%
\begin{pgfscope}%
\pgfsys@transformshift{1.136406in}{1.084987in}%
\pgfsys@useobject{currentmarker}{}%
\end{pgfscope}%
\begin{pgfscope}%
\pgfsys@transformshift{0.721527in}{1.546991in}%
\pgfsys@useobject{currentmarker}{}%
\end{pgfscope}%
\begin{pgfscope}%
\pgfsys@transformshift{1.722543in}{0.603716in}%
\pgfsys@useobject{currentmarker}{}%
\end{pgfscope}%
\begin{pgfscope}%
\pgfsys@transformshift{1.835288in}{0.712488in}%
\pgfsys@useobject{currentmarker}{}%
\end{pgfscope}%
\begin{pgfscope}%
\pgfsys@transformshift{1.984248in}{1.145491in}%
\pgfsys@useobject{currentmarker}{}%
\end{pgfscope}%
\begin{pgfscope}%
\pgfsys@transformshift{1.658536in}{0.924221in}%
\pgfsys@useobject{currentmarker}{}%
\end{pgfscope}%
\begin{pgfscope}%
\pgfsys@transformshift{0.561607in}{1.260450in}%
\pgfsys@useobject{currentmarker}{}%
\end{pgfscope}%
\begin{pgfscope}%
\pgfsys@transformshift{1.906539in}{0.969953in}%
\pgfsys@useobject{currentmarker}{}%
\end{pgfscope}%
\begin{pgfscope}%
\pgfsys@transformshift{1.521333in}{0.951654in}%
\pgfsys@useobject{currentmarker}{}%
\end{pgfscope}%
\begin{pgfscope}%
\pgfsys@transformshift{1.136396in}{1.155363in}%
\pgfsys@useobject{currentmarker}{}%
\end{pgfscope}%
\begin{pgfscope}%
\pgfsys@transformshift{1.435818in}{0.696081in}%
\pgfsys@useobject{currentmarker}{}%
\end{pgfscope}%
\begin{pgfscope}%
\pgfsys@transformshift{1.643271in}{0.706638in}%
\pgfsys@useobject{currentmarker}{}%
\end{pgfscope}%
\begin{pgfscope}%
\pgfsys@transformshift{1.445064in}{0.769952in}%
\pgfsys@useobject{currentmarker}{}%
\end{pgfscope}%
\begin{pgfscope}%
\pgfsys@transformshift{0.658007in}{1.046020in}%
\pgfsys@useobject{currentmarker}{}%
\end{pgfscope}%
\begin{pgfscope}%
\pgfsys@transformshift{0.786972in}{1.182827in}%
\pgfsys@useobject{currentmarker}{}%
\end{pgfscope}%
\begin{pgfscope}%
\pgfsys@transformshift{1.102575in}{1.044640in}%
\pgfsys@useobject{currentmarker}{}%
\end{pgfscope}%
\begin{pgfscope}%
\pgfsys@transformshift{0.532598in}{1.324438in}%
\pgfsys@useobject{currentmarker}{}%
\end{pgfscope}%
\begin{pgfscope}%
\pgfsys@transformshift{0.357574in}{1.408929in}%
\pgfsys@useobject{currentmarker}{}%
\end{pgfscope}%
\begin{pgfscope}%
\pgfsys@transformshift{1.683669in}{0.743181in}%
\pgfsys@useobject{currentmarker}{}%
\end{pgfscope}%
\begin{pgfscope}%
\pgfsys@transformshift{1.612445in}{0.786576in}%
\pgfsys@useobject{currentmarker}{}%
\end{pgfscope}%
\begin{pgfscope}%
\pgfsys@transformshift{1.620066in}{0.677706in}%
\pgfsys@useobject{currentmarker}{}%
\end{pgfscope}%
\begin{pgfscope}%
\pgfsys@transformshift{0.680343in}{0.782450in}%
\pgfsys@useobject{currentmarker}{}%
\end{pgfscope}%
\begin{pgfscope}%
\pgfsys@transformshift{0.675258in}{1.223032in}%
\pgfsys@useobject{currentmarker}{}%
\end{pgfscope}%
\begin{pgfscope}%
\pgfsys@transformshift{0.963927in}{1.041966in}%
\pgfsys@useobject{currentmarker}{}%
\end{pgfscope}%
\begin{pgfscope}%
\pgfsys@transformshift{1.684562in}{0.737447in}%
\pgfsys@useobject{currentmarker}{}%
\end{pgfscope}%
\begin{pgfscope}%
\pgfsys@transformshift{1.473245in}{1.167924in}%
\pgfsys@useobject{currentmarker}{}%
\end{pgfscope}%
\begin{pgfscope}%
\pgfsys@transformshift{1.020383in}{1.233160in}%
\pgfsys@useobject{currentmarker}{}%
\end{pgfscope}%
\begin{pgfscope}%
\pgfsys@transformshift{0.681597in}{1.436798in}%
\pgfsys@useobject{currentmarker}{}%
\end{pgfscope}%
\begin{pgfscope}%
\pgfsys@transformshift{1.730115in}{0.768599in}%
\pgfsys@useobject{currentmarker}{}%
\end{pgfscope}%
\begin{pgfscope}%
\pgfsys@transformshift{1.808933in}{0.711076in}%
\pgfsys@useobject{currentmarker}{}%
\end{pgfscope}%
\begin{pgfscope}%
\pgfsys@transformshift{1.251675in}{0.944124in}%
\pgfsys@useobject{currentmarker}{}%
\end{pgfscope}%
\begin{pgfscope}%
\pgfsys@transformshift{1.717946in}{0.622559in}%
\pgfsys@useobject{currentmarker}{}%
\end{pgfscope}%
\begin{pgfscope}%
\pgfsys@transformshift{1.597312in}{0.676007in}%
\pgfsys@useobject{currentmarker}{}%
\end{pgfscope}%
\begin{pgfscope}%
\pgfsys@transformshift{1.148136in}{1.412295in}%
\pgfsys@useobject{currentmarker}{}%
\end{pgfscope}%
\begin{pgfscope}%
\pgfsys@transformshift{0.720143in}{1.254116in}%
\pgfsys@useobject{currentmarker}{}%
\end{pgfscope}%
\begin{pgfscope}%
\pgfsys@transformshift{1.574970in}{0.942445in}%
\pgfsys@useobject{currentmarker}{}%
\end{pgfscope}%
\begin{pgfscope}%
\pgfsys@transformshift{0.869972in}{0.923663in}%
\pgfsys@useobject{currentmarker}{}%
\end{pgfscope}%
\begin{pgfscope}%
\pgfsys@transformshift{1.713193in}{0.644602in}%
\pgfsys@useobject{currentmarker}{}%
\end{pgfscope}%
\begin{pgfscope}%
\pgfsys@transformshift{1.673692in}{0.695537in}%
\pgfsys@useobject{currentmarker}{}%
\end{pgfscope}%
\begin{pgfscope}%
\pgfsys@transformshift{1.344861in}{0.808518in}%
\pgfsys@useobject{currentmarker}{}%
\end{pgfscope}%
\begin{pgfscope}%
\pgfsys@transformshift{1.182101in}{1.051436in}%
\pgfsys@useobject{currentmarker}{}%
\end{pgfscope}%
\begin{pgfscope}%
\pgfsys@transformshift{1.449668in}{0.932446in}%
\pgfsys@useobject{currentmarker}{}%
\end{pgfscope}%
\begin{pgfscope}%
\pgfsys@transformshift{1.561680in}{0.775386in}%
\pgfsys@useobject{currentmarker}{}%
\end{pgfscope}%
\begin{pgfscope}%
\pgfsys@transformshift{1.192063in}{0.865131in}%
\pgfsys@useobject{currentmarker}{}%
\end{pgfscope}%
\begin{pgfscope}%
\pgfsys@transformshift{1.264374in}{0.964533in}%
\pgfsys@useobject{currentmarker}{}%
\end{pgfscope}%
\begin{pgfscope}%
\pgfsys@transformshift{0.727375in}{1.535408in}%
\pgfsys@useobject{currentmarker}{}%
\end{pgfscope}%
\begin{pgfscope}%
\pgfsys@transformshift{0.687912in}{0.821256in}%
\pgfsys@useobject{currentmarker}{}%
\end{pgfscope}%
\begin{pgfscope}%
\pgfsys@transformshift{1.823596in}{0.759360in}%
\pgfsys@useobject{currentmarker}{}%
\end{pgfscope}%
\begin{pgfscope}%
\pgfsys@transformshift{1.420153in}{0.779176in}%
\pgfsys@useobject{currentmarker}{}%
\end{pgfscope}%
\begin{pgfscope}%
\pgfsys@transformshift{0.672265in}{1.648282in}%
\pgfsys@useobject{currentmarker}{}%
\end{pgfscope}%
\begin{pgfscope}%
\pgfsys@transformshift{0.574267in}{1.354942in}%
\pgfsys@useobject{currentmarker}{}%
\end{pgfscope}%
\begin{pgfscope}%
\pgfsys@transformshift{1.570297in}{0.763075in}%
\pgfsys@useobject{currentmarker}{}%
\end{pgfscope}%
\begin{pgfscope}%
\pgfsys@transformshift{1.650981in}{0.728778in}%
\pgfsys@useobject{currentmarker}{}%
\end{pgfscope}%
\begin{pgfscope}%
\pgfsys@transformshift{0.894282in}{1.228264in}%
\pgfsys@useobject{currentmarker}{}%
\end{pgfscope}%
\begin{pgfscope}%
\pgfsys@transformshift{1.157746in}{1.103999in}%
\pgfsys@useobject{currentmarker}{}%
\end{pgfscope}%
\begin{pgfscope}%
\pgfsys@transformshift{0.608353in}{1.426865in}%
\pgfsys@useobject{currentmarker}{}%
\end{pgfscope}%
\begin{pgfscope}%
\pgfsys@transformshift{1.631665in}{0.685656in}%
\pgfsys@useobject{currentmarker}{}%
\end{pgfscope}%
\begin{pgfscope}%
\pgfsys@transformshift{1.419963in}{0.793810in}%
\pgfsys@useobject{currentmarker}{}%
\end{pgfscope}%
\begin{pgfscope}%
\pgfsys@transformshift{1.180003in}{1.193845in}%
\pgfsys@useobject{currentmarker}{}%
\end{pgfscope}%
\begin{pgfscope}%
\pgfsys@transformshift{1.469789in}{1.138990in}%
\pgfsys@useobject{currentmarker}{}%
\end{pgfscope}%
\begin{pgfscope}%
\pgfsys@transformshift{1.493240in}{1.218757in}%
\pgfsys@useobject{currentmarker}{}%
\end{pgfscope}%
\begin{pgfscope}%
\pgfsys@transformshift{1.725815in}{0.729711in}%
\pgfsys@useobject{currentmarker}{}%
\end{pgfscope}%
\begin{pgfscope}%
\pgfsys@transformshift{1.002068in}{1.207471in}%
\pgfsys@useobject{currentmarker}{}%
\end{pgfscope}%
\begin{pgfscope}%
\pgfsys@transformshift{1.111774in}{1.107384in}%
\pgfsys@useobject{currentmarker}{}%
\end{pgfscope}%
\begin{pgfscope}%
\pgfsys@transformshift{0.361781in}{1.548277in}%
\pgfsys@useobject{currentmarker}{}%
\end{pgfscope}%
\begin{pgfscope}%
\pgfsys@transformshift{1.608152in}{0.881076in}%
\pgfsys@useobject{currentmarker}{}%
\end{pgfscope}%
\begin{pgfscope}%
\pgfsys@transformshift{1.468569in}{1.247721in}%
\pgfsys@useobject{currentmarker}{}%
\end{pgfscope}%
\begin{pgfscope}%
\pgfsys@transformshift{1.646778in}{0.679322in}%
\pgfsys@useobject{currentmarker}{}%
\end{pgfscope}%
\begin{pgfscope}%
\pgfsys@transformshift{1.356664in}{0.915799in}%
\pgfsys@useobject{currentmarker}{}%
\end{pgfscope}%
\begin{pgfscope}%
\pgfsys@transformshift{0.657110in}{0.736781in}%
\pgfsys@useobject{currentmarker}{}%
\end{pgfscope}%
\begin{pgfscope}%
\pgfsys@transformshift{1.700456in}{0.636259in}%
\pgfsys@useobject{currentmarker}{}%
\end{pgfscope}%
\begin{pgfscope}%
\pgfsys@transformshift{0.741052in}{1.110689in}%
\pgfsys@useobject{currentmarker}{}%
\end{pgfscope}%
\begin{pgfscope}%
\pgfsys@transformshift{0.576055in}{1.901818in}%
\pgfsys@useobject{currentmarker}{}%
\end{pgfscope}%
\begin{pgfscope}%
\pgfsys@transformshift{1.509849in}{0.889235in}%
\pgfsys@useobject{currentmarker}{}%
\end{pgfscope}%
\begin{pgfscope}%
\pgfsys@transformshift{1.838065in}{0.688437in}%
\pgfsys@useobject{currentmarker}{}%
\end{pgfscope}%
\begin{pgfscope}%
\pgfsys@transformshift{1.699823in}{0.957457in}%
\pgfsys@useobject{currentmarker}{}%
\end{pgfscope}%
\begin{pgfscope}%
\pgfsys@transformshift{1.575775in}{0.923162in}%
\pgfsys@useobject{currentmarker}{}%
\end{pgfscope}%
\begin{pgfscope}%
\pgfsys@transformshift{1.000652in}{1.399520in}%
\pgfsys@useobject{currentmarker}{}%
\end{pgfscope}%
\begin{pgfscope}%
\pgfsys@transformshift{1.275329in}{0.943478in}%
\pgfsys@useobject{currentmarker}{}%
\end{pgfscope}%
\begin{pgfscope}%
\pgfsys@transformshift{0.787302in}{1.072814in}%
\pgfsys@useobject{currentmarker}{}%
\end{pgfscope}%
\begin{pgfscope}%
\pgfsys@transformshift{0.941308in}{0.784178in}%
\pgfsys@useobject{currentmarker}{}%
\end{pgfscope}%
\begin{pgfscope}%
\pgfsys@transformshift{1.041866in}{0.728239in}%
\pgfsys@useobject{currentmarker}{}%
\end{pgfscope}%
\begin{pgfscope}%
\pgfsys@transformshift{1.471835in}{0.917704in}%
\pgfsys@useobject{currentmarker}{}%
\end{pgfscope}%
\begin{pgfscope}%
\pgfsys@transformshift{1.225077in}{1.005964in}%
\pgfsys@useobject{currentmarker}{}%
\end{pgfscope}%
\begin{pgfscope}%
\pgfsys@transformshift{1.294631in}{0.735371in}%
\pgfsys@useobject{currentmarker}{}%
\end{pgfscope}%
\begin{pgfscope}%
\pgfsys@transformshift{1.597060in}{0.873113in}%
\pgfsys@useobject{currentmarker}{}%
\end{pgfscope}%
\begin{pgfscope}%
\pgfsys@transformshift{1.136873in}{0.760635in}%
\pgfsys@useobject{currentmarker}{}%
\end{pgfscope}%
\begin{pgfscope}%
\pgfsys@transformshift{1.080421in}{1.013617in}%
\pgfsys@useobject{currentmarker}{}%
\end{pgfscope}%
\begin{pgfscope}%
\pgfsys@transformshift{1.185638in}{0.959586in}%
\pgfsys@useobject{currentmarker}{}%
\end{pgfscope}%
\begin{pgfscope}%
\pgfsys@transformshift{1.407062in}{0.941080in}%
\pgfsys@useobject{currentmarker}{}%
\end{pgfscope}%
\begin{pgfscope}%
\pgfsys@transformshift{1.504034in}{0.928170in}%
\pgfsys@useobject{currentmarker}{}%
\end{pgfscope}%
\begin{pgfscope}%
\pgfsys@transformshift{1.396930in}{0.862853in}%
\pgfsys@useobject{currentmarker}{}%
\end{pgfscope}%
\begin{pgfscope}%
\pgfsys@transformshift{2.000000in}{0.714466in}%
\pgfsys@useobject{currentmarker}{}%
\end{pgfscope}%
\begin{pgfscope}%
\pgfsys@transformshift{1.200619in}{0.864692in}%
\pgfsys@useobject{currentmarker}{}%
\end{pgfscope}%
\begin{pgfscope}%
\pgfsys@transformshift{0.950093in}{0.851249in}%
\pgfsys@useobject{currentmarker}{}%
\end{pgfscope}%
\begin{pgfscope}%
\pgfsys@transformshift{0.632457in}{1.314158in}%
\pgfsys@useobject{currentmarker}{}%
\end{pgfscope}%
\begin{pgfscope}%
\pgfsys@transformshift{0.673561in}{1.156581in}%
\pgfsys@useobject{currentmarker}{}%
\end{pgfscope}%
\begin{pgfscope}%
\pgfsys@transformshift{0.694799in}{1.219238in}%
\pgfsys@useobject{currentmarker}{}%
\end{pgfscope}%
\begin{pgfscope}%
\pgfsys@transformshift{0.518198in}{1.328838in}%
\pgfsys@useobject{currentmarker}{}%
\end{pgfscope}%
\begin{pgfscope}%
\pgfsys@transformshift{1.324442in}{0.857176in}%
\pgfsys@useobject{currentmarker}{}%
\end{pgfscope}%
\begin{pgfscope}%
\pgfsys@transformshift{0.827155in}{1.474977in}%
\pgfsys@useobject{currentmarker}{}%
\end{pgfscope}%
\begin{pgfscope}%
\pgfsys@transformshift{1.894588in}{0.803523in}%
\pgfsys@useobject{currentmarker}{}%
\end{pgfscope}%
\begin{pgfscope}%
\pgfsys@transformshift{0.446321in}{1.007537in}%
\pgfsys@useobject{currentmarker}{}%
\end{pgfscope}%
\begin{pgfscope}%
\pgfsys@transformshift{1.741761in}{0.885783in}%
\pgfsys@useobject{currentmarker}{}%
\end{pgfscope}%
\begin{pgfscope}%
\pgfsys@transformshift{1.307351in}{0.837204in}%
\pgfsys@useobject{currentmarker}{}%
\end{pgfscope}%
\begin{pgfscope}%
\pgfsys@transformshift{1.835875in}{0.684668in}%
\pgfsys@useobject{currentmarker}{}%
\end{pgfscope}%
\begin{pgfscope}%
\pgfsys@transformshift{1.237368in}{0.984282in}%
\pgfsys@useobject{currentmarker}{}%
\end{pgfscope}%
\begin{pgfscope}%
\pgfsys@transformshift{1.875271in}{0.562344in}%
\pgfsys@useobject{currentmarker}{}%
\end{pgfscope}%
\begin{pgfscope}%
\pgfsys@transformshift{1.669995in}{0.843785in}%
\pgfsys@useobject{currentmarker}{}%
\end{pgfscope}%
\begin{pgfscope}%
\pgfsys@transformshift{0.628877in}{1.454928in}%
\pgfsys@useobject{currentmarker}{}%
\end{pgfscope}%
\begin{pgfscope}%
\pgfsys@transformshift{1.644915in}{0.863963in}%
\pgfsys@useobject{currentmarker}{}%
\end{pgfscope}%
\begin{pgfscope}%
\pgfsys@transformshift{1.179959in}{0.982357in}%
\pgfsys@useobject{currentmarker}{}%
\end{pgfscope}%
\begin{pgfscope}%
\pgfsys@transformshift{1.753339in}{0.841398in}%
\pgfsys@useobject{currentmarker}{}%
\end{pgfscope}%
\begin{pgfscope}%
\pgfsys@transformshift{1.477915in}{1.047666in}%
\pgfsys@useobject{currentmarker}{}%
\end{pgfscope}%
\begin{pgfscope}%
\pgfsys@transformshift{1.742609in}{0.823285in}%
\pgfsys@useobject{currentmarker}{}%
\end{pgfscope}%
\begin{pgfscope}%
\pgfsys@transformshift{1.094044in}{1.074157in}%
\pgfsys@useobject{currentmarker}{}%
\end{pgfscope}%
\begin{pgfscope}%
\pgfsys@transformshift{1.645687in}{0.674852in}%
\pgfsys@useobject{currentmarker}{}%
\end{pgfscope}%
\begin{pgfscope}%
\pgfsys@transformshift{0.717249in}{1.000856in}%
\pgfsys@useobject{currentmarker}{}%
\end{pgfscope}%
\begin{pgfscope}%
\pgfsys@transformshift{0.963925in}{0.813812in}%
\pgfsys@useobject{currentmarker}{}%
\end{pgfscope}%
\begin{pgfscope}%
\pgfsys@transformshift{1.342874in}{1.025307in}%
\pgfsys@useobject{currentmarker}{}%
\end{pgfscope}%
\begin{pgfscope}%
\pgfsys@transformshift{1.522619in}{0.834739in}%
\pgfsys@useobject{currentmarker}{}%
\end{pgfscope}%
\begin{pgfscope}%
\pgfsys@transformshift{0.649847in}{1.381801in}%
\pgfsys@useobject{currentmarker}{}%
\end{pgfscope}%
\begin{pgfscope}%
\pgfsys@transformshift{1.826185in}{0.949598in}%
\pgfsys@useobject{currentmarker}{}%
\end{pgfscope}%
\begin{pgfscope}%
\pgfsys@transformshift{1.619806in}{0.754500in}%
\pgfsys@useobject{currentmarker}{}%
\end{pgfscope}%
\begin{pgfscope}%
\pgfsys@transformshift{1.136702in}{1.356879in}%
\pgfsys@useobject{currentmarker}{}%
\end{pgfscope}%
\begin{pgfscope}%
\pgfsys@transformshift{1.894425in}{0.717403in}%
\pgfsys@useobject{currentmarker}{}%
\end{pgfscope}%
\begin{pgfscope}%
\pgfsys@transformshift{0.668971in}{1.452903in}%
\pgfsys@useobject{currentmarker}{}%
\end{pgfscope}%
\end{pgfscope}%
\begin{pgfscope}%
\pgfpathrectangle{\pgfqpoint{0.341129in}{0.466613in}}{\pgfqpoint{1.658871in}{1.711598in}}%
\pgfusepath{clip}%
\pgfsetbuttcap%
\pgfsetroundjoin%
\definecolor{currentfill}{rgb}{0.505882,0.447059,0.701961}%
\pgfsetfillcolor{currentfill}%
\pgfsetfillopacity{0.150000}%
\pgfsetlinewidth{1.003750pt}%
\definecolor{currentstroke}{rgb}{1.000000,1.000000,1.000000}%
\pgfsetstrokecolor{currentstroke}%
\pgfsetstrokeopacity{0.150000}%
\pgfsetdash{}{0pt}%
\pgfsys@defobject{currentmarker}{\pgfqpoint{0.341129in}{0.673751in}}{\pgfqpoint{2.000000in}{1.390634in}}{%
\pgfpathmoveto{\pgfqpoint{0.341129in}{1.390634in}}%
\pgfpathlineto{\pgfqpoint{0.341129in}{1.290978in}}%
\pgfpathlineto{\pgfqpoint{0.357885in}{1.285332in}}%
\pgfpathlineto{\pgfqpoint{0.374641in}{1.279604in}}%
\pgfpathlineto{\pgfqpoint{0.391398in}{1.273972in}}%
\pgfpathlineto{\pgfqpoint{0.408154in}{1.268341in}}%
\pgfpathlineto{\pgfqpoint{0.424910in}{1.262709in}}%
\pgfpathlineto{\pgfqpoint{0.441666in}{1.257041in}}%
\pgfpathlineto{\pgfqpoint{0.458423in}{1.251363in}}%
\pgfpathlineto{\pgfqpoint{0.475179in}{1.245690in}}%
\pgfpathlineto{\pgfqpoint{0.491935in}{1.239852in}}%
\pgfpathlineto{\pgfqpoint{0.508691in}{1.233896in}}%
\pgfpathlineto{\pgfqpoint{0.525448in}{1.227939in}}%
\pgfpathlineto{\pgfqpoint{0.542204in}{1.221983in}}%
\pgfpathlineto{\pgfqpoint{0.558960in}{1.216027in}}%
\pgfpathlineto{\pgfqpoint{0.575717in}{1.210260in}}%
\pgfpathlineto{\pgfqpoint{0.592473in}{1.204921in}}%
\pgfpathlineto{\pgfqpoint{0.609229in}{1.199595in}}%
\pgfpathlineto{\pgfqpoint{0.625985in}{1.193906in}}%
\pgfpathlineto{\pgfqpoint{0.642742in}{1.188213in}}%
\pgfpathlineto{\pgfqpoint{0.659498in}{1.182658in}}%
\pgfpathlineto{\pgfqpoint{0.676254in}{1.176809in}}%
\pgfpathlineto{\pgfqpoint{0.693011in}{1.171320in}}%
\pgfpathlineto{\pgfqpoint{0.709767in}{1.165833in}}%
\pgfpathlineto{\pgfqpoint{0.726523in}{1.160347in}}%
\pgfpathlineto{\pgfqpoint{0.743279in}{1.154761in}}%
\pgfpathlineto{\pgfqpoint{0.760036in}{1.149006in}}%
\pgfpathlineto{\pgfqpoint{0.776792in}{1.143251in}}%
\pgfpathlineto{\pgfqpoint{0.793548in}{1.137497in}}%
\pgfpathlineto{\pgfqpoint{0.810304in}{1.131742in}}%
\pgfpathlineto{\pgfqpoint{0.827061in}{1.125988in}}%
\pgfpathlineto{\pgfqpoint{0.843817in}{1.120233in}}%
\pgfpathlineto{\pgfqpoint{0.860573in}{1.114469in}}%
\pgfpathlineto{\pgfqpoint{0.877330in}{1.108690in}}%
\pgfpathlineto{\pgfqpoint{0.894086in}{1.102860in}}%
\pgfpathlineto{\pgfqpoint{0.910842in}{1.097190in}}%
\pgfpathlineto{\pgfqpoint{0.927598in}{1.091468in}}%
\pgfpathlineto{\pgfqpoint{0.944355in}{1.085670in}}%
\pgfpathlineto{\pgfqpoint{0.961111in}{1.079730in}}%
\pgfpathlineto{\pgfqpoint{0.977867in}{1.073549in}}%
\pgfpathlineto{\pgfqpoint{0.994623in}{1.067658in}}%
\pgfpathlineto{\pgfqpoint{1.011380in}{1.061703in}}%
\pgfpathlineto{\pgfqpoint{1.028136in}{1.055910in}}%
\pgfpathlineto{\pgfqpoint{1.044892in}{1.049965in}}%
\pgfpathlineto{\pgfqpoint{1.061649in}{1.044186in}}%
\pgfpathlineto{\pgfqpoint{1.078405in}{1.038325in}}%
\pgfpathlineto{\pgfqpoint{1.095161in}{1.032522in}}%
\pgfpathlineto{\pgfqpoint{1.111917in}{1.026597in}}%
\pgfpathlineto{\pgfqpoint{1.128674in}{1.020723in}}%
\pgfpathlineto{\pgfqpoint{1.145430in}{1.014700in}}%
\pgfpathlineto{\pgfqpoint{1.162186in}{1.008316in}}%
\pgfpathlineto{\pgfqpoint{1.178942in}{1.002345in}}%
\pgfpathlineto{\pgfqpoint{1.195699in}{0.996298in}}%
\pgfpathlineto{\pgfqpoint{1.212455in}{0.990253in}}%
\pgfpathlineto{\pgfqpoint{1.229211in}{0.984087in}}%
\pgfpathlineto{\pgfqpoint{1.245968in}{0.977874in}}%
\pgfpathlineto{\pgfqpoint{1.262724in}{0.971853in}}%
\pgfpathlineto{\pgfqpoint{1.279480in}{0.965347in}}%
\pgfpathlineto{\pgfqpoint{1.296236in}{0.959132in}}%
\pgfpathlineto{\pgfqpoint{1.312993in}{0.952894in}}%
\pgfpathlineto{\pgfqpoint{1.329749in}{0.946788in}}%
\pgfpathlineto{\pgfqpoint{1.346505in}{0.940438in}}%
\pgfpathlineto{\pgfqpoint{1.363262in}{0.934213in}}%
\pgfpathlineto{\pgfqpoint{1.380018in}{0.927979in}}%
\pgfpathlineto{\pgfqpoint{1.396774in}{0.921397in}}%
\pgfpathlineto{\pgfqpoint{1.413530in}{0.914819in}}%
\pgfpathlineto{\pgfqpoint{1.430287in}{0.908247in}}%
\pgfpathlineto{\pgfqpoint{1.447043in}{0.901832in}}%
\pgfpathlineto{\pgfqpoint{1.463799in}{0.895381in}}%
\pgfpathlineto{\pgfqpoint{1.480555in}{0.888999in}}%
\pgfpathlineto{\pgfqpoint{1.497312in}{0.882615in}}%
\pgfpathlineto{\pgfqpoint{1.514068in}{0.876104in}}%
\pgfpathlineto{\pgfqpoint{1.530824in}{0.869448in}}%
\pgfpathlineto{\pgfqpoint{1.547581in}{0.862736in}}%
\pgfpathlineto{\pgfqpoint{1.564337in}{0.855811in}}%
\pgfpathlineto{\pgfqpoint{1.581093in}{0.848964in}}%
\pgfpathlineto{\pgfqpoint{1.597849in}{0.842104in}}%
\pgfpathlineto{\pgfqpoint{1.614606in}{0.835111in}}%
\pgfpathlineto{\pgfqpoint{1.631362in}{0.828118in}}%
\pgfpathlineto{\pgfqpoint{1.648118in}{0.821127in}}%
\pgfpathlineto{\pgfqpoint{1.664874in}{0.814158in}}%
\pgfpathlineto{\pgfqpoint{1.681631in}{0.807479in}}%
\pgfpathlineto{\pgfqpoint{1.698387in}{0.800460in}}%
\pgfpathlineto{\pgfqpoint{1.715143in}{0.793499in}}%
\pgfpathlineto{\pgfqpoint{1.731900in}{0.786779in}}%
\pgfpathlineto{\pgfqpoint{1.748656in}{0.779883in}}%
\pgfpathlineto{\pgfqpoint{1.765412in}{0.772871in}}%
\pgfpathlineto{\pgfqpoint{1.782168in}{0.765629in}}%
\pgfpathlineto{\pgfqpoint{1.798925in}{0.758846in}}%
\pgfpathlineto{\pgfqpoint{1.815681in}{0.751655in}}%
\pgfpathlineto{\pgfqpoint{1.832437in}{0.744472in}}%
\pgfpathlineto{\pgfqpoint{1.849193in}{0.737535in}}%
\pgfpathlineto{\pgfqpoint{1.865950in}{0.730638in}}%
\pgfpathlineto{\pgfqpoint{1.882706in}{0.723755in}}%
\pgfpathlineto{\pgfqpoint{1.899462in}{0.716294in}}%
\pgfpathlineto{\pgfqpoint{1.916219in}{0.709234in}}%
\pgfpathlineto{\pgfqpoint{1.932975in}{0.702241in}}%
\pgfpathlineto{\pgfqpoint{1.949731in}{0.695018in}}%
\pgfpathlineto{\pgfqpoint{1.966487in}{0.687761in}}%
\pgfpathlineto{\pgfqpoint{1.983244in}{0.680754in}}%
\pgfpathlineto{\pgfqpoint{2.000000in}{0.673751in}}%
\pgfpathlineto{\pgfqpoint{2.000000in}{0.742328in}}%
\pgfpathlineto{\pgfqpoint{2.000000in}{0.742328in}}%
\pgfpathlineto{\pgfqpoint{1.983244in}{0.748189in}}%
\pgfpathlineto{\pgfqpoint{1.966487in}{0.754041in}}%
\pgfpathlineto{\pgfqpoint{1.949731in}{0.759686in}}%
\pgfpathlineto{\pgfqpoint{1.932975in}{0.765425in}}%
\pgfpathlineto{\pgfqpoint{1.916219in}{0.771064in}}%
\pgfpathlineto{\pgfqpoint{1.899462in}{0.776698in}}%
\pgfpathlineto{\pgfqpoint{1.882706in}{0.782324in}}%
\pgfpathlineto{\pgfqpoint{1.865950in}{0.787736in}}%
\pgfpathlineto{\pgfqpoint{1.849193in}{0.793289in}}%
\pgfpathlineto{\pgfqpoint{1.832437in}{0.798933in}}%
\pgfpathlineto{\pgfqpoint{1.815681in}{0.804883in}}%
\pgfpathlineto{\pgfqpoint{1.798925in}{0.810530in}}%
\pgfpathlineto{\pgfqpoint{1.782168in}{0.816177in}}%
\pgfpathlineto{\pgfqpoint{1.765412in}{0.821820in}}%
\pgfpathlineto{\pgfqpoint{1.748656in}{0.827544in}}%
\pgfpathlineto{\pgfqpoint{1.731900in}{0.833316in}}%
\pgfpathlineto{\pgfqpoint{1.715143in}{0.839046in}}%
\pgfpathlineto{\pgfqpoint{1.698387in}{0.844732in}}%
\pgfpathlineto{\pgfqpoint{1.681631in}{0.850570in}}%
\pgfpathlineto{\pgfqpoint{1.664874in}{0.856196in}}%
\pgfpathlineto{\pgfqpoint{1.648118in}{0.862123in}}%
\pgfpathlineto{\pgfqpoint{1.631362in}{0.868071in}}%
\pgfpathlineto{\pgfqpoint{1.614606in}{0.874010in}}%
\pgfpathlineto{\pgfqpoint{1.597849in}{0.879672in}}%
\pgfpathlineto{\pgfqpoint{1.581093in}{0.885828in}}%
\pgfpathlineto{\pgfqpoint{1.564337in}{0.891476in}}%
\pgfpathlineto{\pgfqpoint{1.547581in}{0.897586in}}%
\pgfpathlineto{\pgfqpoint{1.530824in}{0.903639in}}%
\pgfpathlineto{\pgfqpoint{1.514068in}{0.909694in}}%
\pgfpathlineto{\pgfqpoint{1.497312in}{0.915508in}}%
\pgfpathlineto{\pgfqpoint{1.480555in}{0.921745in}}%
\pgfpathlineto{\pgfqpoint{1.463799in}{0.927985in}}%
\pgfpathlineto{\pgfqpoint{1.447043in}{0.934231in}}%
\pgfpathlineto{\pgfqpoint{1.430287in}{0.940492in}}%
\pgfpathlineto{\pgfqpoint{1.413530in}{0.946855in}}%
\pgfpathlineto{\pgfqpoint{1.396774in}{0.953137in}}%
\pgfpathlineto{\pgfqpoint{1.380018in}{0.959269in}}%
\pgfpathlineto{\pgfqpoint{1.363262in}{0.965614in}}%
\pgfpathlineto{\pgfqpoint{1.346505in}{0.971966in}}%
\pgfpathlineto{\pgfqpoint{1.329749in}{0.978674in}}%
\pgfpathlineto{\pgfqpoint{1.312993in}{0.985198in}}%
\pgfpathlineto{\pgfqpoint{1.296236in}{0.991778in}}%
\pgfpathlineto{\pgfqpoint{1.279480in}{0.998568in}}%
\pgfpathlineto{\pgfqpoint{1.262724in}{1.005114in}}%
\pgfpathlineto{\pgfqpoint{1.245968in}{1.011452in}}%
\pgfpathlineto{\pgfqpoint{1.229211in}{1.017948in}}%
\pgfpathlineto{\pgfqpoint{1.212455in}{1.024687in}}%
\pgfpathlineto{\pgfqpoint{1.195699in}{1.031470in}}%
\pgfpathlineto{\pgfqpoint{1.178942in}{1.038220in}}%
\pgfpathlineto{\pgfqpoint{1.162186in}{1.044943in}}%
\pgfpathlineto{\pgfqpoint{1.145430in}{1.051754in}}%
\pgfpathlineto{\pgfqpoint{1.128674in}{1.058319in}}%
\pgfpathlineto{\pgfqpoint{1.111917in}{1.065143in}}%
\pgfpathlineto{\pgfqpoint{1.095161in}{1.072053in}}%
\pgfpathlineto{\pgfqpoint{1.078405in}{1.079064in}}%
\pgfpathlineto{\pgfqpoint{1.061649in}{1.085765in}}%
\pgfpathlineto{\pgfqpoint{1.044892in}{1.092387in}}%
\pgfpathlineto{\pgfqpoint{1.028136in}{1.099260in}}%
\pgfpathlineto{\pgfqpoint{1.011380in}{1.106171in}}%
\pgfpathlineto{\pgfqpoint{0.994623in}{1.113088in}}%
\pgfpathlineto{\pgfqpoint{0.977867in}{1.120004in}}%
\pgfpathlineto{\pgfqpoint{0.961111in}{1.126715in}}%
\pgfpathlineto{\pgfqpoint{0.944355in}{1.133841in}}%
\pgfpathlineto{\pgfqpoint{0.927598in}{1.140759in}}%
\pgfpathlineto{\pgfqpoint{0.910842in}{1.147774in}}%
\pgfpathlineto{\pgfqpoint{0.894086in}{1.154925in}}%
\pgfpathlineto{\pgfqpoint{0.877330in}{1.162026in}}%
\pgfpathlineto{\pgfqpoint{0.860573in}{1.168939in}}%
\pgfpathlineto{\pgfqpoint{0.843817in}{1.176077in}}%
\pgfpathlineto{\pgfqpoint{0.827061in}{1.183257in}}%
\pgfpathlineto{\pgfqpoint{0.810304in}{1.190288in}}%
\pgfpathlineto{\pgfqpoint{0.793548in}{1.197500in}}%
\pgfpathlineto{\pgfqpoint{0.776792in}{1.204711in}}%
\pgfpathlineto{\pgfqpoint{0.760036in}{1.211841in}}%
\pgfpathlineto{\pgfqpoint{0.743279in}{1.218695in}}%
\pgfpathlineto{\pgfqpoint{0.726523in}{1.225730in}}%
\pgfpathlineto{\pgfqpoint{0.709767in}{1.232898in}}%
\pgfpathlineto{\pgfqpoint{0.693011in}{1.240079in}}%
\pgfpathlineto{\pgfqpoint{0.676254in}{1.247282in}}%
\pgfpathlineto{\pgfqpoint{0.659498in}{1.254478in}}%
\pgfpathlineto{\pgfqpoint{0.642742in}{1.261585in}}%
\pgfpathlineto{\pgfqpoint{0.625985in}{1.268758in}}%
\pgfpathlineto{\pgfqpoint{0.609229in}{1.275859in}}%
\pgfpathlineto{\pgfqpoint{0.592473in}{1.282798in}}%
\pgfpathlineto{\pgfqpoint{0.575717in}{1.289746in}}%
\pgfpathlineto{\pgfqpoint{0.558960in}{1.296951in}}%
\pgfpathlineto{\pgfqpoint{0.542204in}{1.304236in}}%
\pgfpathlineto{\pgfqpoint{0.525448in}{1.311534in}}%
\pgfpathlineto{\pgfqpoint{0.508691in}{1.318832in}}%
\pgfpathlineto{\pgfqpoint{0.491935in}{1.326130in}}%
\pgfpathlineto{\pgfqpoint{0.475179in}{1.333309in}}%
\pgfpathlineto{\pgfqpoint{0.458423in}{1.340473in}}%
\pgfpathlineto{\pgfqpoint{0.441666in}{1.347596in}}%
\pgfpathlineto{\pgfqpoint{0.424910in}{1.354719in}}%
\pgfpathlineto{\pgfqpoint{0.408154in}{1.361841in}}%
\pgfpathlineto{\pgfqpoint{0.391398in}{1.368964in}}%
\pgfpathlineto{\pgfqpoint{0.374641in}{1.376133in}}%
\pgfpathlineto{\pgfqpoint{0.357885in}{1.383383in}}%
\pgfpathlineto{\pgfqpoint{0.341129in}{1.390634in}}%
\pgfpathclose%
\pgfusepath{stroke,fill}%
}%
\begin{pgfscope}%
\pgfsys@transformshift{0.000000in}{0.000000in}%
\pgfsys@useobject{currentmarker}{}%
\end{pgfscope}%
\end{pgfscope}%
\begin{pgfscope}%
\pgfpathrectangle{\pgfqpoint{0.341129in}{0.466613in}}{\pgfqpoint{1.658871in}{1.711598in}}%
\pgfusepath{clip}%
\pgfsetroundcap%
\pgfsetroundjoin%
\pgfsetlinewidth{1.505625pt}%
\definecolor{currentstroke}{rgb}{0.298039,0.447059,0.690196}%
\pgfsetstrokecolor{currentstroke}%
\pgfsetdash{}{0pt}%
\pgfpathmoveto{\pgfqpoint{0.341129in}{1.999952in}}%
\pgfpathlineto{\pgfqpoint{0.357885in}{1.999334in}}%
\pgfpathlineto{\pgfqpoint{0.374641in}{1.998717in}}%
\pgfpathlineto{\pgfqpoint{0.391398in}{1.998099in}}%
\pgfpathlineto{\pgfqpoint{0.408154in}{1.997482in}}%
\pgfpathlineto{\pgfqpoint{0.424910in}{1.996864in}}%
\pgfpathlineto{\pgfqpoint{0.441666in}{1.996247in}}%
\pgfpathlineto{\pgfqpoint{0.458423in}{1.995629in}}%
\pgfpathlineto{\pgfqpoint{0.475179in}{1.995012in}}%
\pgfpathlineto{\pgfqpoint{0.491935in}{1.994394in}}%
\pgfpathlineto{\pgfqpoint{0.508691in}{1.993777in}}%
\pgfpathlineto{\pgfqpoint{0.525448in}{1.993159in}}%
\pgfpathlineto{\pgfqpoint{0.542204in}{1.992542in}}%
\pgfpathlineto{\pgfqpoint{0.558960in}{1.991925in}}%
\pgfpathlineto{\pgfqpoint{0.575717in}{1.991307in}}%
\pgfpathlineto{\pgfqpoint{0.592473in}{1.990690in}}%
\pgfpathlineto{\pgfqpoint{0.609229in}{1.990072in}}%
\pgfpathlineto{\pgfqpoint{0.625985in}{1.989455in}}%
\pgfpathlineto{\pgfqpoint{0.642742in}{1.988837in}}%
\pgfpathlineto{\pgfqpoint{0.659498in}{1.988220in}}%
\pgfpathlineto{\pgfqpoint{0.676254in}{1.987602in}}%
\pgfpathlineto{\pgfqpoint{0.693011in}{1.986985in}}%
\pgfpathlineto{\pgfqpoint{0.709767in}{1.986367in}}%
\pgfpathlineto{\pgfqpoint{0.726523in}{1.985750in}}%
\pgfpathlineto{\pgfqpoint{0.743279in}{1.985132in}}%
\pgfpathlineto{\pgfqpoint{0.760036in}{1.984515in}}%
\pgfpathlineto{\pgfqpoint{0.776792in}{1.983897in}}%
\pgfpathlineto{\pgfqpoint{0.793548in}{1.983280in}}%
\pgfpathlineto{\pgfqpoint{0.810304in}{1.982663in}}%
\pgfpathlineto{\pgfqpoint{0.827061in}{1.982045in}}%
\pgfpathlineto{\pgfqpoint{0.843817in}{1.981428in}}%
\pgfpathlineto{\pgfqpoint{0.860573in}{1.980810in}}%
\pgfpathlineto{\pgfqpoint{0.877330in}{1.980193in}}%
\pgfpathlineto{\pgfqpoint{0.894086in}{1.979575in}}%
\pgfpathlineto{\pgfqpoint{0.910842in}{1.978958in}}%
\pgfpathlineto{\pgfqpoint{0.927598in}{1.978340in}}%
\pgfpathlineto{\pgfqpoint{0.944355in}{1.977723in}}%
\pgfpathlineto{\pgfqpoint{0.961111in}{1.977105in}}%
\pgfpathlineto{\pgfqpoint{0.977867in}{1.976488in}}%
\pgfpathlineto{\pgfqpoint{0.994623in}{1.975870in}}%
\pgfpathlineto{\pgfqpoint{1.011380in}{1.975253in}}%
\pgfpathlineto{\pgfqpoint{1.028136in}{1.974635in}}%
\pgfpathlineto{\pgfqpoint{1.044892in}{1.974018in}}%
\pgfpathlineto{\pgfqpoint{1.061649in}{1.973401in}}%
\pgfpathlineto{\pgfqpoint{1.078405in}{1.972783in}}%
\pgfpathlineto{\pgfqpoint{1.095161in}{1.972166in}}%
\pgfpathlineto{\pgfqpoint{1.111917in}{1.971548in}}%
\pgfpathlineto{\pgfqpoint{1.128674in}{1.970931in}}%
\pgfpathlineto{\pgfqpoint{1.145430in}{1.970313in}}%
\pgfpathlineto{\pgfqpoint{1.162186in}{1.969696in}}%
\pgfpathlineto{\pgfqpoint{1.178942in}{1.969078in}}%
\pgfpathlineto{\pgfqpoint{1.195699in}{1.968461in}}%
\pgfpathlineto{\pgfqpoint{1.212455in}{1.967843in}}%
\pgfpathlineto{\pgfqpoint{1.229211in}{1.967226in}}%
\pgfpathlineto{\pgfqpoint{1.245968in}{1.966608in}}%
\pgfpathlineto{\pgfqpoint{1.262724in}{1.965991in}}%
\pgfpathlineto{\pgfqpoint{1.279480in}{1.965373in}}%
\pgfpathlineto{\pgfqpoint{1.296236in}{1.964756in}}%
\pgfpathlineto{\pgfqpoint{1.312993in}{1.964138in}}%
\pgfpathlineto{\pgfqpoint{1.329749in}{1.963521in}}%
\pgfpathlineto{\pgfqpoint{1.346505in}{1.962904in}}%
\pgfpathlineto{\pgfqpoint{1.363262in}{1.962286in}}%
\pgfpathlineto{\pgfqpoint{1.380018in}{1.961669in}}%
\pgfpathlineto{\pgfqpoint{1.396774in}{1.961051in}}%
\pgfpathlineto{\pgfqpoint{1.413530in}{1.960434in}}%
\pgfpathlineto{\pgfqpoint{1.430287in}{1.959816in}}%
\pgfpathlineto{\pgfqpoint{1.447043in}{1.959199in}}%
\pgfpathlineto{\pgfqpoint{1.463799in}{1.958581in}}%
\pgfpathlineto{\pgfqpoint{1.480555in}{1.957964in}}%
\pgfpathlineto{\pgfqpoint{1.497312in}{1.957346in}}%
\pgfpathlineto{\pgfqpoint{1.514068in}{1.956729in}}%
\pgfpathlineto{\pgfqpoint{1.530824in}{1.956111in}}%
\pgfpathlineto{\pgfqpoint{1.547581in}{1.955494in}}%
\pgfpathlineto{\pgfqpoint{1.564337in}{1.954876in}}%
\pgfpathlineto{\pgfqpoint{1.581093in}{1.954259in}}%
\pgfpathlineto{\pgfqpoint{1.597849in}{1.953642in}}%
\pgfpathlineto{\pgfqpoint{1.614606in}{1.953024in}}%
\pgfpathlineto{\pgfqpoint{1.631362in}{1.952407in}}%
\pgfpathlineto{\pgfqpoint{1.648118in}{1.951789in}}%
\pgfpathlineto{\pgfqpoint{1.664874in}{1.951172in}}%
\pgfpathlineto{\pgfqpoint{1.681631in}{1.950554in}}%
\pgfpathlineto{\pgfqpoint{1.698387in}{1.949937in}}%
\pgfpathlineto{\pgfqpoint{1.715143in}{1.949319in}}%
\pgfpathlineto{\pgfqpoint{1.731900in}{1.948702in}}%
\pgfpathlineto{\pgfqpoint{1.748656in}{1.948084in}}%
\pgfpathlineto{\pgfqpoint{1.765412in}{1.947467in}}%
\pgfpathlineto{\pgfqpoint{1.782168in}{1.946849in}}%
\pgfpathlineto{\pgfqpoint{1.798925in}{1.946232in}}%
\pgfpathlineto{\pgfqpoint{1.815681in}{1.945614in}}%
\pgfpathlineto{\pgfqpoint{1.832437in}{1.944997in}}%
\pgfpathlineto{\pgfqpoint{1.849193in}{1.944379in}}%
\pgfpathlineto{\pgfqpoint{1.865950in}{1.943762in}}%
\pgfpathlineto{\pgfqpoint{1.882706in}{1.943145in}}%
\pgfpathlineto{\pgfqpoint{1.899462in}{1.942527in}}%
\pgfpathlineto{\pgfqpoint{1.916219in}{1.941910in}}%
\pgfpathlineto{\pgfqpoint{1.932975in}{1.941292in}}%
\pgfpathlineto{\pgfqpoint{1.949731in}{1.940675in}}%
\pgfpathlineto{\pgfqpoint{1.966487in}{1.940057in}}%
\pgfpathlineto{\pgfqpoint{1.983244in}{1.939440in}}%
\pgfpathlineto{\pgfqpoint{2.000000in}{1.938822in}}%
\pgfusepath{stroke}%
\end{pgfscope}%
\begin{pgfscope}%
\pgfpathrectangle{\pgfqpoint{0.341129in}{0.466613in}}{\pgfqpoint{1.658871in}{1.711598in}}%
\pgfusepath{clip}%
\pgfsetroundcap%
\pgfsetroundjoin%
\pgfsetlinewidth{1.505625pt}%
\definecolor{currentstroke}{rgb}{0.866667,0.517647,0.321569}%
\pgfsetstrokecolor{currentstroke}%
\pgfsetdash{}{0pt}%
\pgfpathmoveto{\pgfqpoint{0.341129in}{1.855071in}}%
\pgfpathlineto{\pgfqpoint{0.357885in}{1.853639in}}%
\pgfpathlineto{\pgfqpoint{0.374641in}{1.852206in}}%
\pgfpathlineto{\pgfqpoint{0.391398in}{1.850774in}}%
\pgfpathlineto{\pgfqpoint{0.408154in}{1.849342in}}%
\pgfpathlineto{\pgfqpoint{0.424910in}{1.847909in}}%
\pgfpathlineto{\pgfqpoint{0.441666in}{1.846477in}}%
\pgfpathlineto{\pgfqpoint{0.458423in}{1.845045in}}%
\pgfpathlineto{\pgfqpoint{0.475179in}{1.843612in}}%
\pgfpathlineto{\pgfqpoint{0.491935in}{1.842180in}}%
\pgfpathlineto{\pgfqpoint{0.508691in}{1.840748in}}%
\pgfpathlineto{\pgfqpoint{0.525448in}{1.839315in}}%
\pgfpathlineto{\pgfqpoint{0.542204in}{1.837883in}}%
\pgfpathlineto{\pgfqpoint{0.558960in}{1.836451in}}%
\pgfpathlineto{\pgfqpoint{0.575717in}{1.835018in}}%
\pgfpathlineto{\pgfqpoint{0.592473in}{1.833586in}}%
\pgfpathlineto{\pgfqpoint{0.609229in}{1.832154in}}%
\pgfpathlineto{\pgfqpoint{0.625985in}{1.830721in}}%
\pgfpathlineto{\pgfqpoint{0.642742in}{1.829289in}}%
\pgfpathlineto{\pgfqpoint{0.659498in}{1.827856in}}%
\pgfpathlineto{\pgfqpoint{0.676254in}{1.826424in}}%
\pgfpathlineto{\pgfqpoint{0.693011in}{1.824992in}}%
\pgfpathlineto{\pgfqpoint{0.709767in}{1.823559in}}%
\pgfpathlineto{\pgfqpoint{0.726523in}{1.822127in}}%
\pgfpathlineto{\pgfqpoint{0.743279in}{1.820695in}}%
\pgfpathlineto{\pgfqpoint{0.760036in}{1.819262in}}%
\pgfpathlineto{\pgfqpoint{0.776792in}{1.817830in}}%
\pgfpathlineto{\pgfqpoint{0.793548in}{1.816398in}}%
\pgfpathlineto{\pgfqpoint{0.810304in}{1.814965in}}%
\pgfpathlineto{\pgfqpoint{0.827061in}{1.813533in}}%
\pgfpathlineto{\pgfqpoint{0.843817in}{1.812101in}}%
\pgfpathlineto{\pgfqpoint{0.860573in}{1.810668in}}%
\pgfpathlineto{\pgfqpoint{0.877330in}{1.809236in}}%
\pgfpathlineto{\pgfqpoint{0.894086in}{1.807804in}}%
\pgfpathlineto{\pgfqpoint{0.910842in}{1.806371in}}%
\pgfpathlineto{\pgfqpoint{0.927598in}{1.804939in}}%
\pgfpathlineto{\pgfqpoint{0.944355in}{1.803506in}}%
\pgfpathlineto{\pgfqpoint{0.961111in}{1.802074in}}%
\pgfpathlineto{\pgfqpoint{0.977867in}{1.800642in}}%
\pgfpathlineto{\pgfqpoint{0.994623in}{1.799209in}}%
\pgfpathlineto{\pgfqpoint{1.011380in}{1.797777in}}%
\pgfpathlineto{\pgfqpoint{1.028136in}{1.796345in}}%
\pgfpathlineto{\pgfqpoint{1.044892in}{1.794912in}}%
\pgfpathlineto{\pgfqpoint{1.061649in}{1.793480in}}%
\pgfpathlineto{\pgfqpoint{1.078405in}{1.792048in}}%
\pgfpathlineto{\pgfqpoint{1.095161in}{1.790615in}}%
\pgfpathlineto{\pgfqpoint{1.111917in}{1.789183in}}%
\pgfpathlineto{\pgfqpoint{1.128674in}{1.787751in}}%
\pgfpathlineto{\pgfqpoint{1.145430in}{1.786318in}}%
\pgfpathlineto{\pgfqpoint{1.162186in}{1.784886in}}%
\pgfpathlineto{\pgfqpoint{1.178942in}{1.783454in}}%
\pgfpathlineto{\pgfqpoint{1.195699in}{1.782021in}}%
\pgfpathlineto{\pgfqpoint{1.212455in}{1.780589in}}%
\pgfpathlineto{\pgfqpoint{1.229211in}{1.779156in}}%
\pgfpathlineto{\pgfqpoint{1.245968in}{1.777724in}}%
\pgfpathlineto{\pgfqpoint{1.262724in}{1.776292in}}%
\pgfpathlineto{\pgfqpoint{1.279480in}{1.774859in}}%
\pgfpathlineto{\pgfqpoint{1.296236in}{1.773427in}}%
\pgfpathlineto{\pgfqpoint{1.312993in}{1.771995in}}%
\pgfpathlineto{\pgfqpoint{1.329749in}{1.770562in}}%
\pgfpathlineto{\pgfqpoint{1.346505in}{1.769130in}}%
\pgfpathlineto{\pgfqpoint{1.363262in}{1.767698in}}%
\pgfpathlineto{\pgfqpoint{1.380018in}{1.766265in}}%
\pgfpathlineto{\pgfqpoint{1.396774in}{1.764833in}}%
\pgfpathlineto{\pgfqpoint{1.413530in}{1.763401in}}%
\pgfpathlineto{\pgfqpoint{1.430287in}{1.761968in}}%
\pgfpathlineto{\pgfqpoint{1.447043in}{1.760536in}}%
\pgfpathlineto{\pgfqpoint{1.463799in}{1.759104in}}%
\pgfpathlineto{\pgfqpoint{1.480555in}{1.757671in}}%
\pgfpathlineto{\pgfqpoint{1.497312in}{1.756239in}}%
\pgfpathlineto{\pgfqpoint{1.514068in}{1.754806in}}%
\pgfpathlineto{\pgfqpoint{1.530824in}{1.753374in}}%
\pgfpathlineto{\pgfqpoint{1.547581in}{1.751942in}}%
\pgfpathlineto{\pgfqpoint{1.564337in}{1.750509in}}%
\pgfpathlineto{\pgfqpoint{1.581093in}{1.749077in}}%
\pgfpathlineto{\pgfqpoint{1.597849in}{1.747645in}}%
\pgfpathlineto{\pgfqpoint{1.614606in}{1.746212in}}%
\pgfpathlineto{\pgfqpoint{1.631362in}{1.744780in}}%
\pgfpathlineto{\pgfqpoint{1.648118in}{1.743348in}}%
\pgfpathlineto{\pgfqpoint{1.664874in}{1.741915in}}%
\pgfpathlineto{\pgfqpoint{1.681631in}{1.740483in}}%
\pgfpathlineto{\pgfqpoint{1.698387in}{1.739051in}}%
\pgfpathlineto{\pgfqpoint{1.715143in}{1.737618in}}%
\pgfpathlineto{\pgfqpoint{1.731900in}{1.736186in}}%
\pgfpathlineto{\pgfqpoint{1.748656in}{1.734754in}}%
\pgfpathlineto{\pgfqpoint{1.765412in}{1.733321in}}%
\pgfpathlineto{\pgfqpoint{1.782168in}{1.731889in}}%
\pgfpathlineto{\pgfqpoint{1.798925in}{1.730456in}}%
\pgfpathlineto{\pgfqpoint{1.815681in}{1.729024in}}%
\pgfpathlineto{\pgfqpoint{1.832437in}{1.727592in}}%
\pgfpathlineto{\pgfqpoint{1.849193in}{1.726159in}}%
\pgfpathlineto{\pgfqpoint{1.865950in}{1.724727in}}%
\pgfpathlineto{\pgfqpoint{1.882706in}{1.723295in}}%
\pgfpathlineto{\pgfqpoint{1.899462in}{1.721862in}}%
\pgfpathlineto{\pgfqpoint{1.916219in}{1.720430in}}%
\pgfpathlineto{\pgfqpoint{1.932975in}{1.718998in}}%
\pgfpathlineto{\pgfqpoint{1.949731in}{1.717565in}}%
\pgfpathlineto{\pgfqpoint{1.966487in}{1.716133in}}%
\pgfpathlineto{\pgfqpoint{1.983244in}{1.714701in}}%
\pgfpathlineto{\pgfqpoint{2.000000in}{1.713268in}}%
\pgfusepath{stroke}%
\end{pgfscope}%
\begin{pgfscope}%
\pgfpathrectangle{\pgfqpoint{0.341129in}{0.466613in}}{\pgfqpoint{1.658871in}{1.711598in}}%
\pgfusepath{clip}%
\pgfsetroundcap%
\pgfsetroundjoin%
\pgfsetlinewidth{1.505625pt}%
\definecolor{currentstroke}{rgb}{0.333333,0.658824,0.407843}%
\pgfsetstrokecolor{currentstroke}%
\pgfsetdash{}{0pt}%
\pgfpathmoveto{\pgfqpoint{0.341129in}{1.649951in}}%
\pgfpathlineto{\pgfqpoint{0.357885in}{1.646768in}}%
\pgfpathlineto{\pgfqpoint{0.374641in}{1.643585in}}%
\pgfpathlineto{\pgfqpoint{0.391398in}{1.640402in}}%
\pgfpathlineto{\pgfqpoint{0.408154in}{1.637219in}}%
\pgfpathlineto{\pgfqpoint{0.424910in}{1.634036in}}%
\pgfpathlineto{\pgfqpoint{0.441666in}{1.630853in}}%
\pgfpathlineto{\pgfqpoint{0.458423in}{1.627670in}}%
\pgfpathlineto{\pgfqpoint{0.475179in}{1.624487in}}%
\pgfpathlineto{\pgfqpoint{0.491935in}{1.621304in}}%
\pgfpathlineto{\pgfqpoint{0.508691in}{1.618121in}}%
\pgfpathlineto{\pgfqpoint{0.525448in}{1.614938in}}%
\pgfpathlineto{\pgfqpoint{0.542204in}{1.611755in}}%
\pgfpathlineto{\pgfqpoint{0.558960in}{1.608572in}}%
\pgfpathlineto{\pgfqpoint{0.575717in}{1.605389in}}%
\pgfpathlineto{\pgfqpoint{0.592473in}{1.602206in}}%
\pgfpathlineto{\pgfqpoint{0.609229in}{1.599023in}}%
\pgfpathlineto{\pgfqpoint{0.625985in}{1.595840in}}%
\pgfpathlineto{\pgfqpoint{0.642742in}{1.592657in}}%
\pgfpathlineto{\pgfqpoint{0.659498in}{1.589475in}}%
\pgfpathlineto{\pgfqpoint{0.676254in}{1.586292in}}%
\pgfpathlineto{\pgfqpoint{0.693011in}{1.583109in}}%
\pgfpathlineto{\pgfqpoint{0.709767in}{1.579926in}}%
\pgfpathlineto{\pgfqpoint{0.726523in}{1.576743in}}%
\pgfpathlineto{\pgfqpoint{0.743279in}{1.573560in}}%
\pgfpathlineto{\pgfqpoint{0.760036in}{1.570377in}}%
\pgfpathlineto{\pgfqpoint{0.776792in}{1.567194in}}%
\pgfpathlineto{\pgfqpoint{0.793548in}{1.564011in}}%
\pgfpathlineto{\pgfqpoint{0.810304in}{1.560828in}}%
\pgfpathlineto{\pgfqpoint{0.827061in}{1.557645in}}%
\pgfpathlineto{\pgfqpoint{0.843817in}{1.554462in}}%
\pgfpathlineto{\pgfqpoint{0.860573in}{1.551279in}}%
\pgfpathlineto{\pgfqpoint{0.877330in}{1.548096in}}%
\pgfpathlineto{\pgfqpoint{0.894086in}{1.544913in}}%
\pgfpathlineto{\pgfqpoint{0.910842in}{1.541730in}}%
\pgfpathlineto{\pgfqpoint{0.927598in}{1.538547in}}%
\pgfpathlineto{\pgfqpoint{0.944355in}{1.535364in}}%
\pgfpathlineto{\pgfqpoint{0.961111in}{1.532181in}}%
\pgfpathlineto{\pgfqpoint{0.977867in}{1.528998in}}%
\pgfpathlineto{\pgfqpoint{0.994623in}{1.525815in}}%
\pgfpathlineto{\pgfqpoint{1.011380in}{1.522632in}}%
\pgfpathlineto{\pgfqpoint{1.028136in}{1.519449in}}%
\pgfpathlineto{\pgfqpoint{1.044892in}{1.516266in}}%
\pgfpathlineto{\pgfqpoint{1.061649in}{1.513084in}}%
\pgfpathlineto{\pgfqpoint{1.078405in}{1.509901in}}%
\pgfpathlineto{\pgfqpoint{1.095161in}{1.506718in}}%
\pgfpathlineto{\pgfqpoint{1.111917in}{1.503535in}}%
\pgfpathlineto{\pgfqpoint{1.128674in}{1.500352in}}%
\pgfpathlineto{\pgfqpoint{1.145430in}{1.497169in}}%
\pgfpathlineto{\pgfqpoint{1.162186in}{1.493986in}}%
\pgfpathlineto{\pgfqpoint{1.178942in}{1.490803in}}%
\pgfpathlineto{\pgfqpoint{1.195699in}{1.487620in}}%
\pgfpathlineto{\pgfqpoint{1.212455in}{1.484437in}}%
\pgfpathlineto{\pgfqpoint{1.229211in}{1.481254in}}%
\pgfpathlineto{\pgfqpoint{1.245968in}{1.478071in}}%
\pgfpathlineto{\pgfqpoint{1.262724in}{1.474888in}}%
\pgfpathlineto{\pgfqpoint{1.279480in}{1.471705in}}%
\pgfpathlineto{\pgfqpoint{1.296236in}{1.468522in}}%
\pgfpathlineto{\pgfqpoint{1.312993in}{1.465339in}}%
\pgfpathlineto{\pgfqpoint{1.329749in}{1.462156in}}%
\pgfpathlineto{\pgfqpoint{1.346505in}{1.458973in}}%
\pgfpathlineto{\pgfqpoint{1.363262in}{1.455790in}}%
\pgfpathlineto{\pgfqpoint{1.380018in}{1.452607in}}%
\pgfpathlineto{\pgfqpoint{1.396774in}{1.449424in}}%
\pgfpathlineto{\pgfqpoint{1.413530in}{1.446241in}}%
\pgfpathlineto{\pgfqpoint{1.430287in}{1.443058in}}%
\pgfpathlineto{\pgfqpoint{1.447043in}{1.439875in}}%
\pgfpathlineto{\pgfqpoint{1.463799in}{1.436693in}}%
\pgfpathlineto{\pgfqpoint{1.480555in}{1.433510in}}%
\pgfpathlineto{\pgfqpoint{1.497312in}{1.430327in}}%
\pgfpathlineto{\pgfqpoint{1.514068in}{1.427144in}}%
\pgfpathlineto{\pgfqpoint{1.530824in}{1.423961in}}%
\pgfpathlineto{\pgfqpoint{1.547581in}{1.420778in}}%
\pgfpathlineto{\pgfqpoint{1.564337in}{1.417595in}}%
\pgfpathlineto{\pgfqpoint{1.581093in}{1.414412in}}%
\pgfpathlineto{\pgfqpoint{1.597849in}{1.411229in}}%
\pgfpathlineto{\pgfqpoint{1.614606in}{1.408046in}}%
\pgfpathlineto{\pgfqpoint{1.631362in}{1.404863in}}%
\pgfpathlineto{\pgfqpoint{1.648118in}{1.401680in}}%
\pgfpathlineto{\pgfqpoint{1.664874in}{1.398497in}}%
\pgfpathlineto{\pgfqpoint{1.681631in}{1.395314in}}%
\pgfpathlineto{\pgfqpoint{1.698387in}{1.392131in}}%
\pgfpathlineto{\pgfqpoint{1.715143in}{1.388948in}}%
\pgfpathlineto{\pgfqpoint{1.731900in}{1.385765in}}%
\pgfpathlineto{\pgfqpoint{1.748656in}{1.382582in}}%
\pgfpathlineto{\pgfqpoint{1.765412in}{1.379399in}}%
\pgfpathlineto{\pgfqpoint{1.782168in}{1.376216in}}%
\pgfpathlineto{\pgfqpoint{1.798925in}{1.373033in}}%
\pgfpathlineto{\pgfqpoint{1.815681in}{1.369850in}}%
\pgfpathlineto{\pgfqpoint{1.832437in}{1.366667in}}%
\pgfpathlineto{\pgfqpoint{1.849193in}{1.363484in}}%
\pgfpathlineto{\pgfqpoint{1.865950in}{1.360301in}}%
\pgfpathlineto{\pgfqpoint{1.882706in}{1.357119in}}%
\pgfpathlineto{\pgfqpoint{1.899462in}{1.353936in}}%
\pgfpathlineto{\pgfqpoint{1.916219in}{1.350753in}}%
\pgfpathlineto{\pgfqpoint{1.932975in}{1.347570in}}%
\pgfpathlineto{\pgfqpoint{1.949731in}{1.344387in}}%
\pgfpathlineto{\pgfqpoint{1.966487in}{1.341204in}}%
\pgfpathlineto{\pgfqpoint{1.983244in}{1.338021in}}%
\pgfpathlineto{\pgfqpoint{2.000000in}{1.334838in}}%
\pgfusepath{stroke}%
\end{pgfscope}%
\begin{pgfscope}%
\pgfpathrectangle{\pgfqpoint{0.341129in}{0.466613in}}{\pgfqpoint{1.658871in}{1.711598in}}%
\pgfusepath{clip}%
\pgfsetroundcap%
\pgfsetroundjoin%
\pgfsetlinewidth{1.505625pt}%
\definecolor{currentstroke}{rgb}{0.768627,0.305882,0.321569}%
\pgfsetstrokecolor{currentstroke}%
\pgfsetdash{}{0pt}%
\pgfpathmoveto{\pgfqpoint{0.341129in}{1.520460in}}%
\pgfpathlineto{\pgfqpoint{0.357885in}{1.516021in}}%
\pgfpathlineto{\pgfqpoint{0.374641in}{1.511583in}}%
\pgfpathlineto{\pgfqpoint{0.391398in}{1.507144in}}%
\pgfpathlineto{\pgfqpoint{0.408154in}{1.502705in}}%
\pgfpathlineto{\pgfqpoint{0.424910in}{1.498266in}}%
\pgfpathlineto{\pgfqpoint{0.441666in}{1.493827in}}%
\pgfpathlineto{\pgfqpoint{0.458423in}{1.489388in}}%
\pgfpathlineto{\pgfqpoint{0.475179in}{1.484949in}}%
\pgfpathlineto{\pgfqpoint{0.491935in}{1.480511in}}%
\pgfpathlineto{\pgfqpoint{0.508691in}{1.476072in}}%
\pgfpathlineto{\pgfqpoint{0.525448in}{1.471633in}}%
\pgfpathlineto{\pgfqpoint{0.542204in}{1.467194in}}%
\pgfpathlineto{\pgfqpoint{0.558960in}{1.462755in}}%
\pgfpathlineto{\pgfqpoint{0.575717in}{1.458316in}}%
\pgfpathlineto{\pgfqpoint{0.592473in}{1.453877in}}%
\pgfpathlineto{\pgfqpoint{0.609229in}{1.449439in}}%
\pgfpathlineto{\pgfqpoint{0.625985in}{1.445000in}}%
\pgfpathlineto{\pgfqpoint{0.642742in}{1.440561in}}%
\pgfpathlineto{\pgfqpoint{0.659498in}{1.436122in}}%
\pgfpathlineto{\pgfqpoint{0.676254in}{1.431683in}}%
\pgfpathlineto{\pgfqpoint{0.693011in}{1.427244in}}%
\pgfpathlineto{\pgfqpoint{0.709767in}{1.422806in}}%
\pgfpathlineto{\pgfqpoint{0.726523in}{1.418367in}}%
\pgfpathlineto{\pgfqpoint{0.743279in}{1.413928in}}%
\pgfpathlineto{\pgfqpoint{0.760036in}{1.409489in}}%
\pgfpathlineto{\pgfqpoint{0.776792in}{1.405050in}}%
\pgfpathlineto{\pgfqpoint{0.793548in}{1.400611in}}%
\pgfpathlineto{\pgfqpoint{0.810304in}{1.396172in}}%
\pgfpathlineto{\pgfqpoint{0.827061in}{1.391734in}}%
\pgfpathlineto{\pgfqpoint{0.843817in}{1.387295in}}%
\pgfpathlineto{\pgfqpoint{0.860573in}{1.382856in}}%
\pgfpathlineto{\pgfqpoint{0.877330in}{1.378417in}}%
\pgfpathlineto{\pgfqpoint{0.894086in}{1.373978in}}%
\pgfpathlineto{\pgfqpoint{0.910842in}{1.369539in}}%
\pgfpathlineto{\pgfqpoint{0.927598in}{1.365100in}}%
\pgfpathlineto{\pgfqpoint{0.944355in}{1.360662in}}%
\pgfpathlineto{\pgfqpoint{0.961111in}{1.356223in}}%
\pgfpathlineto{\pgfqpoint{0.977867in}{1.351784in}}%
\pgfpathlineto{\pgfqpoint{0.994623in}{1.347345in}}%
\pgfpathlineto{\pgfqpoint{1.011380in}{1.342906in}}%
\pgfpathlineto{\pgfqpoint{1.028136in}{1.338467in}}%
\pgfpathlineto{\pgfqpoint{1.044892in}{1.334029in}}%
\pgfpathlineto{\pgfqpoint{1.061649in}{1.329590in}}%
\pgfpathlineto{\pgfqpoint{1.078405in}{1.325151in}}%
\pgfpathlineto{\pgfqpoint{1.095161in}{1.320712in}}%
\pgfpathlineto{\pgfqpoint{1.111917in}{1.316273in}}%
\pgfpathlineto{\pgfqpoint{1.128674in}{1.311834in}}%
\pgfpathlineto{\pgfqpoint{1.145430in}{1.307395in}}%
\pgfpathlineto{\pgfqpoint{1.162186in}{1.302957in}}%
\pgfpathlineto{\pgfqpoint{1.178942in}{1.298518in}}%
\pgfpathlineto{\pgfqpoint{1.195699in}{1.294079in}}%
\pgfpathlineto{\pgfqpoint{1.212455in}{1.289640in}}%
\pgfpathlineto{\pgfqpoint{1.229211in}{1.285201in}}%
\pgfpathlineto{\pgfqpoint{1.245968in}{1.280762in}}%
\pgfpathlineto{\pgfqpoint{1.262724in}{1.276324in}}%
\pgfpathlineto{\pgfqpoint{1.279480in}{1.271885in}}%
\pgfpathlineto{\pgfqpoint{1.296236in}{1.267446in}}%
\pgfpathlineto{\pgfqpoint{1.312993in}{1.263007in}}%
\pgfpathlineto{\pgfqpoint{1.329749in}{1.258568in}}%
\pgfpathlineto{\pgfqpoint{1.346505in}{1.254129in}}%
\pgfpathlineto{\pgfqpoint{1.363262in}{1.249690in}}%
\pgfpathlineto{\pgfqpoint{1.380018in}{1.245252in}}%
\pgfpathlineto{\pgfqpoint{1.396774in}{1.240813in}}%
\pgfpathlineto{\pgfqpoint{1.413530in}{1.236374in}}%
\pgfpathlineto{\pgfqpoint{1.430287in}{1.231935in}}%
\pgfpathlineto{\pgfqpoint{1.447043in}{1.227496in}}%
\pgfpathlineto{\pgfqpoint{1.463799in}{1.223057in}}%
\pgfpathlineto{\pgfqpoint{1.480555in}{1.218618in}}%
\pgfpathlineto{\pgfqpoint{1.497312in}{1.214180in}}%
\pgfpathlineto{\pgfqpoint{1.514068in}{1.209741in}}%
\pgfpathlineto{\pgfqpoint{1.530824in}{1.205302in}}%
\pgfpathlineto{\pgfqpoint{1.547581in}{1.200863in}}%
\pgfpathlineto{\pgfqpoint{1.564337in}{1.196424in}}%
\pgfpathlineto{\pgfqpoint{1.581093in}{1.191985in}}%
\pgfpathlineto{\pgfqpoint{1.597849in}{1.187547in}}%
\pgfpathlineto{\pgfqpoint{1.614606in}{1.183108in}}%
\pgfpathlineto{\pgfqpoint{1.631362in}{1.178669in}}%
\pgfpathlineto{\pgfqpoint{1.648118in}{1.174230in}}%
\pgfpathlineto{\pgfqpoint{1.664874in}{1.169791in}}%
\pgfpathlineto{\pgfqpoint{1.681631in}{1.165352in}}%
\pgfpathlineto{\pgfqpoint{1.698387in}{1.160913in}}%
\pgfpathlineto{\pgfqpoint{1.715143in}{1.156475in}}%
\pgfpathlineto{\pgfqpoint{1.731900in}{1.152036in}}%
\pgfpathlineto{\pgfqpoint{1.748656in}{1.147597in}}%
\pgfpathlineto{\pgfqpoint{1.765412in}{1.143158in}}%
\pgfpathlineto{\pgfqpoint{1.782168in}{1.138719in}}%
\pgfpathlineto{\pgfqpoint{1.798925in}{1.134280in}}%
\pgfpathlineto{\pgfqpoint{1.815681in}{1.129841in}}%
\pgfpathlineto{\pgfqpoint{1.832437in}{1.125403in}}%
\pgfpathlineto{\pgfqpoint{1.849193in}{1.120964in}}%
\pgfpathlineto{\pgfqpoint{1.865950in}{1.116525in}}%
\pgfpathlineto{\pgfqpoint{1.882706in}{1.112086in}}%
\pgfpathlineto{\pgfqpoint{1.899462in}{1.107647in}}%
\pgfpathlineto{\pgfqpoint{1.916219in}{1.103208in}}%
\pgfpathlineto{\pgfqpoint{1.932975in}{1.098770in}}%
\pgfpathlineto{\pgfqpoint{1.949731in}{1.094331in}}%
\pgfpathlineto{\pgfqpoint{1.966487in}{1.089892in}}%
\pgfpathlineto{\pgfqpoint{1.983244in}{1.085453in}}%
\pgfpathlineto{\pgfqpoint{2.000000in}{1.081014in}}%
\pgfusepath{stroke}%
\end{pgfscope}%
\begin{pgfscope}%
\pgfpathrectangle{\pgfqpoint{0.341129in}{0.466613in}}{\pgfqpoint{1.658871in}{1.711598in}}%
\pgfusepath{clip}%
\pgfsetroundcap%
\pgfsetroundjoin%
\pgfsetlinewidth{1.505625pt}%
\definecolor{currentstroke}{rgb}{0.505882,0.447059,0.701961}%
\pgfsetstrokecolor{currentstroke}%
\pgfsetdash{}{0pt}%
\pgfpathmoveto{\pgfqpoint{0.341129in}{1.338002in}}%
\pgfpathlineto{\pgfqpoint{0.357885in}{1.331643in}}%
\pgfpathlineto{\pgfqpoint{0.374641in}{1.325285in}}%
\pgfpathlineto{\pgfqpoint{0.391398in}{1.318927in}}%
\pgfpathlineto{\pgfqpoint{0.408154in}{1.312568in}}%
\pgfpathlineto{\pgfqpoint{0.424910in}{1.306210in}}%
\pgfpathlineto{\pgfqpoint{0.441666in}{1.299852in}}%
\pgfpathlineto{\pgfqpoint{0.458423in}{1.293493in}}%
\pgfpathlineto{\pgfqpoint{0.475179in}{1.287135in}}%
\pgfpathlineto{\pgfqpoint{0.491935in}{1.280776in}}%
\pgfpathlineto{\pgfqpoint{0.508691in}{1.274418in}}%
\pgfpathlineto{\pgfqpoint{0.525448in}{1.268060in}}%
\pgfpathlineto{\pgfqpoint{0.542204in}{1.261701in}}%
\pgfpathlineto{\pgfqpoint{0.558960in}{1.255343in}}%
\pgfpathlineto{\pgfqpoint{0.575717in}{1.248985in}}%
\pgfpathlineto{\pgfqpoint{0.592473in}{1.242626in}}%
\pgfpathlineto{\pgfqpoint{0.609229in}{1.236268in}}%
\pgfpathlineto{\pgfqpoint{0.625985in}{1.229909in}}%
\pgfpathlineto{\pgfqpoint{0.642742in}{1.223551in}}%
\pgfpathlineto{\pgfqpoint{0.659498in}{1.217193in}}%
\pgfpathlineto{\pgfqpoint{0.676254in}{1.210834in}}%
\pgfpathlineto{\pgfqpoint{0.693011in}{1.204476in}}%
\pgfpathlineto{\pgfqpoint{0.709767in}{1.198118in}}%
\pgfpathlineto{\pgfqpoint{0.726523in}{1.191759in}}%
\pgfpathlineto{\pgfqpoint{0.743279in}{1.185401in}}%
\pgfpathlineto{\pgfqpoint{0.760036in}{1.179042in}}%
\pgfpathlineto{\pgfqpoint{0.776792in}{1.172684in}}%
\pgfpathlineto{\pgfqpoint{0.793548in}{1.166326in}}%
\pgfpathlineto{\pgfqpoint{0.810304in}{1.159967in}}%
\pgfpathlineto{\pgfqpoint{0.827061in}{1.153609in}}%
\pgfpathlineto{\pgfqpoint{0.843817in}{1.147251in}}%
\pgfpathlineto{\pgfqpoint{0.860573in}{1.140892in}}%
\pgfpathlineto{\pgfqpoint{0.877330in}{1.134534in}}%
\pgfpathlineto{\pgfqpoint{0.894086in}{1.128175in}}%
\pgfpathlineto{\pgfqpoint{0.910842in}{1.121817in}}%
\pgfpathlineto{\pgfqpoint{0.927598in}{1.115459in}}%
\pgfpathlineto{\pgfqpoint{0.944355in}{1.109100in}}%
\pgfpathlineto{\pgfqpoint{0.961111in}{1.102742in}}%
\pgfpathlineto{\pgfqpoint{0.977867in}{1.096384in}}%
\pgfpathlineto{\pgfqpoint{0.994623in}{1.090025in}}%
\pgfpathlineto{\pgfqpoint{1.011380in}{1.083667in}}%
\pgfpathlineto{\pgfqpoint{1.028136in}{1.077308in}}%
\pgfpathlineto{\pgfqpoint{1.044892in}{1.070950in}}%
\pgfpathlineto{\pgfqpoint{1.061649in}{1.064592in}}%
\pgfpathlineto{\pgfqpoint{1.078405in}{1.058233in}}%
\pgfpathlineto{\pgfqpoint{1.095161in}{1.051875in}}%
\pgfpathlineto{\pgfqpoint{1.111917in}{1.045517in}}%
\pgfpathlineto{\pgfqpoint{1.128674in}{1.039158in}}%
\pgfpathlineto{\pgfqpoint{1.145430in}{1.032800in}}%
\pgfpathlineto{\pgfqpoint{1.162186in}{1.026441in}}%
\pgfpathlineto{\pgfqpoint{1.178942in}{1.020083in}}%
\pgfpathlineto{\pgfqpoint{1.195699in}{1.013725in}}%
\pgfpathlineto{\pgfqpoint{1.212455in}{1.007366in}}%
\pgfpathlineto{\pgfqpoint{1.229211in}{1.001008in}}%
\pgfpathlineto{\pgfqpoint{1.245968in}{0.994650in}}%
\pgfpathlineto{\pgfqpoint{1.262724in}{0.988291in}}%
\pgfpathlineto{\pgfqpoint{1.279480in}{0.981933in}}%
\pgfpathlineto{\pgfqpoint{1.296236in}{0.975574in}}%
\pgfpathlineto{\pgfqpoint{1.312993in}{0.969216in}}%
\pgfpathlineto{\pgfqpoint{1.329749in}{0.962858in}}%
\pgfpathlineto{\pgfqpoint{1.346505in}{0.956499in}}%
\pgfpathlineto{\pgfqpoint{1.363262in}{0.950141in}}%
\pgfpathlineto{\pgfqpoint{1.380018in}{0.943783in}}%
\pgfpathlineto{\pgfqpoint{1.396774in}{0.937424in}}%
\pgfpathlineto{\pgfqpoint{1.413530in}{0.931066in}}%
\pgfpathlineto{\pgfqpoint{1.430287in}{0.924707in}}%
\pgfpathlineto{\pgfqpoint{1.447043in}{0.918349in}}%
\pgfpathlineto{\pgfqpoint{1.463799in}{0.911991in}}%
\pgfpathlineto{\pgfqpoint{1.480555in}{0.905632in}}%
\pgfpathlineto{\pgfqpoint{1.497312in}{0.899274in}}%
\pgfpathlineto{\pgfqpoint{1.514068in}{0.892915in}}%
\pgfpathlineto{\pgfqpoint{1.530824in}{0.886557in}}%
\pgfpathlineto{\pgfqpoint{1.547581in}{0.880199in}}%
\pgfpathlineto{\pgfqpoint{1.564337in}{0.873840in}}%
\pgfpathlineto{\pgfqpoint{1.581093in}{0.867482in}}%
\pgfpathlineto{\pgfqpoint{1.597849in}{0.861124in}}%
\pgfpathlineto{\pgfqpoint{1.614606in}{0.854765in}}%
\pgfpathlineto{\pgfqpoint{1.631362in}{0.848407in}}%
\pgfpathlineto{\pgfqpoint{1.648118in}{0.842048in}}%
\pgfpathlineto{\pgfqpoint{1.664874in}{0.835690in}}%
\pgfpathlineto{\pgfqpoint{1.681631in}{0.829332in}}%
\pgfpathlineto{\pgfqpoint{1.698387in}{0.822973in}}%
\pgfpathlineto{\pgfqpoint{1.715143in}{0.816615in}}%
\pgfpathlineto{\pgfqpoint{1.731900in}{0.810257in}}%
\pgfpathlineto{\pgfqpoint{1.748656in}{0.803898in}}%
\pgfpathlineto{\pgfqpoint{1.765412in}{0.797540in}}%
\pgfpathlineto{\pgfqpoint{1.782168in}{0.791181in}}%
\pgfpathlineto{\pgfqpoint{1.798925in}{0.784823in}}%
\pgfpathlineto{\pgfqpoint{1.815681in}{0.778465in}}%
\pgfpathlineto{\pgfqpoint{1.832437in}{0.772106in}}%
\pgfpathlineto{\pgfqpoint{1.849193in}{0.765748in}}%
\pgfpathlineto{\pgfqpoint{1.865950in}{0.759390in}}%
\pgfpathlineto{\pgfqpoint{1.882706in}{0.753031in}}%
\pgfpathlineto{\pgfqpoint{1.899462in}{0.746673in}}%
\pgfpathlineto{\pgfqpoint{1.916219in}{0.740314in}}%
\pgfpathlineto{\pgfqpoint{1.932975in}{0.733956in}}%
\pgfpathlineto{\pgfqpoint{1.949731in}{0.727598in}}%
\pgfpathlineto{\pgfqpoint{1.966487in}{0.721239in}}%
\pgfpathlineto{\pgfqpoint{1.983244in}{0.714881in}}%
\pgfpathlineto{\pgfqpoint{2.000000in}{0.708523in}}%
\pgfusepath{stroke}%
\end{pgfscope}%
\begin{pgfscope}%
\pgfsetrectcap%
\pgfsetmiterjoin%
\pgfsetlinewidth{0.752812pt}%
\definecolor{currentstroke}{rgb}{0.700000,0.700000,0.700000}%
\pgfsetstrokecolor{currentstroke}%
\pgfsetdash{}{0pt}%
\pgfpathmoveto{\pgfqpoint{0.341129in}{0.466613in}}%
\pgfpathlineto{\pgfqpoint{0.341129in}{2.178211in}}%
\pgfusepath{stroke}%
\end{pgfscope}%
\begin{pgfscope}%
\pgfsetrectcap%
\pgfsetmiterjoin%
\pgfsetlinewidth{0.752812pt}%
\definecolor{currentstroke}{rgb}{0.700000,0.700000,0.700000}%
\pgfsetstrokecolor{currentstroke}%
\pgfsetdash{}{0pt}%
\pgfpathmoveto{\pgfqpoint{0.341129in}{0.466613in}}%
\pgfpathlineto{\pgfqpoint{2.000000in}{0.466613in}}%
\pgfusepath{stroke}%
\end{pgfscope}%
\end{pgfpicture}%
\makeatother%
\endgroup%
}} 
      \resizebox{0.125\linewidth}{!}{\input{assets/global_prbcd_arxiv_joint_n_nodes_largest_component_vs_robustness_legend.pgf}} \\
    \end{array}\)
  }
  \caption{For each of the 500 experiments, we sample 8 classes out of the 40 classes of the arXiv dataset and then take the largest connected component. We argue that the difficulty of a classification task is strongly influenced by the number of classes which is apparent by the lower random chance for more classes. In (a) we see, that there is some variance in the results, but the slope is not significant. For low perturbation budgets (b), we see a clear trend that the large graph is less robust (i.e.\ the  drop in accuracy is stronger for large graphs). Around a budget of \(\epsilon\) the slope becomes insignificant and (d) with a large budget the smaller graphs are less robust. In (b) we observe a correlation of \(\rho_b=0.36\) with a significance of \(\alpha_b=2e-16\) and in (d) \(\rho_b=-0.28\) with a significance of \(\alpha_b=2e-10\) \label{fig:graphsizevsrobustness}}
\end{figure*}

\textbf{Time and memory cost.} On arXiv, we train for 500 epochs and run the PR-BCD attack for 500 epochs. The whole training and attacking procedure requires less than 2 minutes and the peak usage of GPU memory is below 2.5 GB.
%If we use checkpointing and chunk matrix multiplications into 16 parts, the same procedure takes 6 minutes but only requires 1.9 GB. 
Note that only loading the adjacency matrix for traditional attacks (no training etc.) would require around 115 GB (see Table~\ref{tab:datasets}). Naively attacking the dense adjacency matrix would certainly require more than 1 TB.

\todo{PPRGo / PRBCD vs. NETTACK}

\section{Conclusion}\label{sec:conclusion} % Open

\todo{We propose three new attacks that all have the potential to scale to much larger graphs. We are this first to study adversarial attacks on graphs of practical size and, hence, set the cornerstone for the important evaluation of adversarial robustness at scale. We give some intriguing insights. For example, our experiments suggest that adversarial robustness seems to decrease with the size of the graph.Moreover, it seems to be very different to defend against adversarially added nodes than edge additions or deletions within the existing graph structure. For most applications, we believe that adding new nodes is more realistic than adding edges between existing nodes and, hence, we argue that this setting should be studied more in future work.}

%%
%% The acknowledgments section is defined using the "acks" environment
%% (and NOT an unnumbered section). This ensures the proper
%% identification of the section in the article metadata, and the
%% consistent spelling of the heading.
\begin{acks}
  To Robert, for the bagels and explaining CMYK and color spaces.
\end{acks}

%%
%% The next two lines define the bibliography style to be used, and
%% the bibliography file.
\bibliographystyle{ACM-Reference-Format}
\bibliography{references}

%%
%% If your work has an appendix, this is the place to put it.
\newpage
\appendix

\todo{table of hyperparameters}

\section{Alternative Motivation: Cross Entropy is a Bad Surrogate}\label{sec:related} % Simon

In the context of images, typically a single sample is attacked. In the context of graph neural networks this corresponds to a local attack. For such a scenario an untargeted attack it is often sufficient to maximize the cross entropy 
\begin{equation}\label{oldeq:crossentropy}
\text{CE}^{(n)}(y, \vp) = \sum_{c \in \sC} \mathbb{I}[y^{(n)} = c] \log(\evp_{c})^{(n)}\,.
\end{equation}
Many \emph{global} attacks~\citet{Chen2018, Wu2019, Xu2018, Zugner2019a} also attack via maximizing the cross entropy \(\max_{\adj} \text{CE}(f_{\theta}(\adj, \features))\). However, in our experiments, while attacking GNNs on large graphs, we have often observed that the CE loss increased while the accuracy did not decline. This can be explained by a bias of CE towards nodes which had a low confidence score (misclassified). This is apparent in Figure~\ref{oldfig:negceprob}. Intuitively, in contrast of attacking a single image/node, a global structure attack has to 1) keep house with the budget \(\Delta\) and 2) find edges that degrade the accuracy maximally.

\begin{figure}[t]
  \centering
  \makebox[\linewidth][c]{
    \(\begin{array}{cc}
      \subfloat[Clean graph]{\resizebox{0.5\linewidth}{!}{%% Creator: Matplotlib, PGF backend
%%
%% To include the figure in your LaTeX document, write
%%   \input{<filename>.pgf}
%%
%% Make sure the required packages are loaded in your preamble
%%   \usepackage{pgf}
%%
%% and, on pdftex
%%   \usepackage[utf8]{inputenc}\DeclareUnicodeCharacter{2212}{-}
%%
%% or, on luatex and xetex
%%   \usepackage{unicode-math}
%%
%% Figures using additional raster images can only be included by \input if
%% they are in the same directory as the main LaTeX file. For loading figures
%% from other directories you can use the `import` package
%%   \usepackage{import}
%%
%% and then include the figures with
%%   \import{<path to file>}{<filename>.pgf}
%%
%% Matplotlib used the following preamble
%%   \usepackage[utf8]{inputenc}
%%   \usepackage[T1]{fontenc}
%%   \usepackage{amsmath}
%%   \newcommand*{\mat}[1]{\boldsymbol{#1}}
%%
\begingroup%
\makeatletter%
\begin{pgfpicture}%
\pgfpathrectangle{\pgfpointorigin}{\pgfqpoint{10.063632in}{9.626613in}}%
\pgfusepath{use as bounding box, clip}%
\begin{pgfscope}%
\pgfsetbuttcap%
\pgfsetmiterjoin%
\definecolor{currentfill}{rgb}{1.000000,1.000000,1.000000}%
\pgfsetfillcolor{currentfill}%
\pgfsetlinewidth{0.000000pt}%
\definecolor{currentstroke}{rgb}{1.000000,1.000000,1.000000}%
\pgfsetstrokecolor{currentstroke}%
\pgfsetstrokeopacity{0.000000}%
\pgfsetdash{}{0pt}%
\pgfpathmoveto{\pgfqpoint{0.000000in}{0.000000in}}%
\pgfpathlineto{\pgfqpoint{10.063632in}{0.000000in}}%
\pgfpathlineto{\pgfqpoint{10.063632in}{9.626613in}}%
\pgfpathlineto{\pgfqpoint{0.000000in}{9.626613in}}%
\pgfpathclose%
\pgfusepath{fill}%
\end{pgfscope}%
\begin{pgfscope}%
\pgfsetbuttcap%
\pgfsetmiterjoin%
\definecolor{currentfill}{rgb}{1.000000,1.000000,1.000000}%
\pgfsetfillcolor{currentfill}%
\pgfsetlinewidth{0.000000pt}%
\definecolor{currentstroke}{rgb}{0.000000,0.000000,0.000000}%
\pgfsetstrokecolor{currentstroke}%
\pgfsetstrokeopacity{0.000000}%
\pgfsetdash{}{0pt}%
\pgfpathmoveto{\pgfqpoint{0.663632in}{0.466613in}}%
\pgfpathlineto{\pgfqpoint{9.963632in}{0.466613in}}%
\pgfpathlineto{\pgfqpoint{9.963632in}{9.526613in}}%
\pgfpathlineto{\pgfqpoint{0.663632in}{9.526613in}}%
\pgfpathclose%
\pgfusepath{fill}%
\end{pgfscope}%
\begin{pgfscope}%
\pgfpathrectangle{\pgfqpoint{0.663632in}{0.466613in}}{\pgfqpoint{9.300000in}{9.060000in}}%
\pgfusepath{clip}%
\pgfsetroundcap%
\pgfsetroundjoin%
\pgfsetlinewidth{0.501875pt}%
\definecolor{currentstroke}{rgb}{0.800000,0.800000,0.800000}%
\pgfsetstrokecolor{currentstroke}%
\pgfsetdash{}{0pt}%
\pgfpathmoveto{\pgfqpoint{1.086359in}{0.466613in}}%
\pgfpathlineto{\pgfqpoint{1.086359in}{9.526613in}}%
\pgfusepath{stroke}%
\end{pgfscope}%
\begin{pgfscope}%
\definecolor{textcolor}{rgb}{0.150000,0.150000,0.150000}%
\pgfsetstrokecolor{textcolor}%
\pgfsetfillcolor{textcolor}%
\pgftext[x=1.086359in,y=0.376335in,,top]{\color{textcolor}\rmfamily\fontsize{8.000000}{9.600000}\selectfont \(\displaystyle {0.0}\)}%
\end{pgfscope}%
\begin{pgfscope}%
\pgfpathrectangle{\pgfqpoint{0.663632in}{0.466613in}}{\pgfqpoint{9.300000in}{9.060000in}}%
\pgfusepath{clip}%
\pgfsetroundcap%
\pgfsetroundjoin%
\pgfsetlinewidth{0.501875pt}%
\definecolor{currentstroke}{rgb}{0.800000,0.800000,0.800000}%
\pgfsetstrokecolor{currentstroke}%
\pgfsetdash{}{0pt}%
\pgfpathmoveto{\pgfqpoint{2.777269in}{0.466613in}}%
\pgfpathlineto{\pgfqpoint{2.777269in}{9.526613in}}%
\pgfusepath{stroke}%
\end{pgfscope}%
\begin{pgfscope}%
\definecolor{textcolor}{rgb}{0.150000,0.150000,0.150000}%
\pgfsetstrokecolor{textcolor}%
\pgfsetfillcolor{textcolor}%
\pgftext[x=2.777269in,y=0.376335in,,top]{\color{textcolor}\rmfamily\fontsize{8.000000}{9.600000}\selectfont \(\displaystyle {0.2}\)}%
\end{pgfscope}%
\begin{pgfscope}%
\pgfpathrectangle{\pgfqpoint{0.663632in}{0.466613in}}{\pgfqpoint{9.300000in}{9.060000in}}%
\pgfusepath{clip}%
\pgfsetroundcap%
\pgfsetroundjoin%
\pgfsetlinewidth{0.501875pt}%
\definecolor{currentstroke}{rgb}{0.800000,0.800000,0.800000}%
\pgfsetstrokecolor{currentstroke}%
\pgfsetdash{}{0pt}%
\pgfpathmoveto{\pgfqpoint{4.468180in}{0.466613in}}%
\pgfpathlineto{\pgfqpoint{4.468180in}{9.526613in}}%
\pgfusepath{stroke}%
\end{pgfscope}%
\begin{pgfscope}%
\definecolor{textcolor}{rgb}{0.150000,0.150000,0.150000}%
\pgfsetstrokecolor{textcolor}%
\pgfsetfillcolor{textcolor}%
\pgftext[x=4.468180in,y=0.376335in,,top]{\color{textcolor}\rmfamily\fontsize{8.000000}{9.600000}\selectfont \(\displaystyle {0.4}\)}%
\end{pgfscope}%
\begin{pgfscope}%
\pgfpathrectangle{\pgfqpoint{0.663632in}{0.466613in}}{\pgfqpoint{9.300000in}{9.060000in}}%
\pgfusepath{clip}%
\pgfsetroundcap%
\pgfsetroundjoin%
\pgfsetlinewidth{0.501875pt}%
\definecolor{currentstroke}{rgb}{0.800000,0.800000,0.800000}%
\pgfsetstrokecolor{currentstroke}%
\pgfsetdash{}{0pt}%
\pgfpathmoveto{\pgfqpoint{6.159091in}{0.466613in}}%
\pgfpathlineto{\pgfqpoint{6.159091in}{9.526613in}}%
\pgfusepath{stroke}%
\end{pgfscope}%
\begin{pgfscope}%
\definecolor{textcolor}{rgb}{0.150000,0.150000,0.150000}%
\pgfsetstrokecolor{textcolor}%
\pgfsetfillcolor{textcolor}%
\pgftext[x=6.159091in,y=0.376335in,,top]{\color{textcolor}\rmfamily\fontsize{8.000000}{9.600000}\selectfont \(\displaystyle {0.6}\)}%
\end{pgfscope}%
\begin{pgfscope}%
\pgfpathrectangle{\pgfqpoint{0.663632in}{0.466613in}}{\pgfqpoint{9.300000in}{9.060000in}}%
\pgfusepath{clip}%
\pgfsetroundcap%
\pgfsetroundjoin%
\pgfsetlinewidth{0.501875pt}%
\definecolor{currentstroke}{rgb}{0.800000,0.800000,0.800000}%
\pgfsetstrokecolor{currentstroke}%
\pgfsetdash{}{0pt}%
\pgfpathmoveto{\pgfqpoint{7.850001in}{0.466613in}}%
\pgfpathlineto{\pgfqpoint{7.850001in}{9.526613in}}%
\pgfusepath{stroke}%
\end{pgfscope}%
\begin{pgfscope}%
\definecolor{textcolor}{rgb}{0.150000,0.150000,0.150000}%
\pgfsetstrokecolor{textcolor}%
\pgfsetfillcolor{textcolor}%
\pgftext[x=7.850001in,y=0.376335in,,top]{\color{textcolor}\rmfamily\fontsize{8.000000}{9.600000}\selectfont \(\displaystyle {0.8}\)}%
\end{pgfscope}%
\begin{pgfscope}%
\pgfpathrectangle{\pgfqpoint{0.663632in}{0.466613in}}{\pgfqpoint{9.300000in}{9.060000in}}%
\pgfusepath{clip}%
\pgfsetroundcap%
\pgfsetroundjoin%
\pgfsetlinewidth{0.501875pt}%
\definecolor{currentstroke}{rgb}{0.800000,0.800000,0.800000}%
\pgfsetstrokecolor{currentstroke}%
\pgfsetdash{}{0pt}%
\pgfpathmoveto{\pgfqpoint{9.540912in}{0.466613in}}%
\pgfpathlineto{\pgfqpoint{9.540912in}{9.526613in}}%
\pgfusepath{stroke}%
\end{pgfscope}%
\begin{pgfscope}%
\definecolor{textcolor}{rgb}{0.150000,0.150000,0.150000}%
\pgfsetstrokecolor{textcolor}%
\pgfsetfillcolor{textcolor}%
\pgftext[x=9.540912in,y=0.376335in,,top]{\color{textcolor}\rmfamily\fontsize{8.000000}{9.600000}\selectfont \(\displaystyle {1.0}\)}%
\end{pgfscope}%
\begin{pgfscope}%
\definecolor{textcolor}{rgb}{0.150000,0.150000,0.150000}%
\pgfsetstrokecolor{textcolor}%
\pgfsetfillcolor{textcolor}%
\pgftext[x=5.313632in,y=0.222655in,,top]{\color{textcolor}\rmfamily\fontsize{10.000000}{12.000000}\selectfont Probability of attacked node}%
\end{pgfscope}%
\begin{pgfscope}%
\pgfpathrectangle{\pgfqpoint{0.663632in}{0.466613in}}{\pgfqpoint{9.300000in}{9.060000in}}%
\pgfusepath{clip}%
\pgfsetroundcap%
\pgfsetroundjoin%
\pgfsetlinewidth{0.501875pt}%
\definecolor{currentstroke}{rgb}{0.800000,0.800000,0.800000}%
\pgfsetstrokecolor{currentstroke}%
\pgfsetdash{}{0pt}%
\pgfpathmoveto{\pgfqpoint{0.663632in}{0.466613in}}%
\pgfpathlineto{\pgfqpoint{9.963632in}{0.466613in}}%
\pgfusepath{stroke}%
\end{pgfscope}%
\begin{pgfscope}%
\definecolor{textcolor}{rgb}{0.150000,0.150000,0.150000}%
\pgfsetstrokecolor{textcolor}%
\pgfsetfillcolor{textcolor}%
\pgftext[x=0.514325in, y=0.428350in, left, base]{\color{textcolor}\rmfamily\fontsize{8.000000}{9.600000}\selectfont \(\displaystyle {0}\)}%
\end{pgfscope}%
\begin{pgfscope}%
\pgfpathrectangle{\pgfqpoint{0.663632in}{0.466613in}}{\pgfqpoint{9.300000in}{9.060000in}}%
\pgfusepath{clip}%
\pgfsetroundcap%
\pgfsetroundjoin%
\pgfsetlinewidth{0.501875pt}%
\definecolor{currentstroke}{rgb}{0.800000,0.800000,0.800000}%
\pgfsetstrokecolor{currentstroke}%
\pgfsetdash{}{0pt}%
\pgfpathmoveto{\pgfqpoint{0.663632in}{1.886837in}}%
\pgfpathlineto{\pgfqpoint{9.963632in}{1.886837in}}%
\pgfusepath{stroke}%
\end{pgfscope}%
\begin{pgfscope}%
\definecolor{textcolor}{rgb}{0.150000,0.150000,0.150000}%
\pgfsetstrokecolor{textcolor}%
\pgfsetfillcolor{textcolor}%
\pgftext[x=0.337239in, y=1.848575in, left, base]{\color{textcolor}\rmfamily\fontsize{8.000000}{9.600000}\selectfont \(\displaystyle {2000}\)}%
\end{pgfscope}%
\begin{pgfscope}%
\pgfpathrectangle{\pgfqpoint{0.663632in}{0.466613in}}{\pgfqpoint{9.300000in}{9.060000in}}%
\pgfusepath{clip}%
\pgfsetroundcap%
\pgfsetroundjoin%
\pgfsetlinewidth{0.501875pt}%
\definecolor{currentstroke}{rgb}{0.800000,0.800000,0.800000}%
\pgfsetstrokecolor{currentstroke}%
\pgfsetdash{}{0pt}%
\pgfpathmoveto{\pgfqpoint{0.663632in}{3.307061in}}%
\pgfpathlineto{\pgfqpoint{9.963632in}{3.307061in}}%
\pgfusepath{stroke}%
\end{pgfscope}%
\begin{pgfscope}%
\definecolor{textcolor}{rgb}{0.150000,0.150000,0.150000}%
\pgfsetstrokecolor{textcolor}%
\pgfsetfillcolor{textcolor}%
\pgftext[x=0.337239in, y=3.268799in, left, base]{\color{textcolor}\rmfamily\fontsize{8.000000}{9.600000}\selectfont \(\displaystyle {4000}\)}%
\end{pgfscope}%
\begin{pgfscope}%
\pgfpathrectangle{\pgfqpoint{0.663632in}{0.466613in}}{\pgfqpoint{9.300000in}{9.060000in}}%
\pgfusepath{clip}%
\pgfsetroundcap%
\pgfsetroundjoin%
\pgfsetlinewidth{0.501875pt}%
\definecolor{currentstroke}{rgb}{0.800000,0.800000,0.800000}%
\pgfsetstrokecolor{currentstroke}%
\pgfsetdash{}{0pt}%
\pgfpathmoveto{\pgfqpoint{0.663632in}{4.727285in}}%
\pgfpathlineto{\pgfqpoint{9.963632in}{4.727285in}}%
\pgfusepath{stroke}%
\end{pgfscope}%
\begin{pgfscope}%
\definecolor{textcolor}{rgb}{0.150000,0.150000,0.150000}%
\pgfsetstrokecolor{textcolor}%
\pgfsetfillcolor{textcolor}%
\pgftext[x=0.337239in, y=4.689023in, left, base]{\color{textcolor}\rmfamily\fontsize{8.000000}{9.600000}\selectfont \(\displaystyle {6000}\)}%
\end{pgfscope}%
\begin{pgfscope}%
\pgfpathrectangle{\pgfqpoint{0.663632in}{0.466613in}}{\pgfqpoint{9.300000in}{9.060000in}}%
\pgfusepath{clip}%
\pgfsetroundcap%
\pgfsetroundjoin%
\pgfsetlinewidth{0.501875pt}%
\definecolor{currentstroke}{rgb}{0.800000,0.800000,0.800000}%
\pgfsetstrokecolor{currentstroke}%
\pgfsetdash{}{0pt}%
\pgfpathmoveto{\pgfqpoint{0.663632in}{6.147509in}}%
\pgfpathlineto{\pgfqpoint{9.963632in}{6.147509in}}%
\pgfusepath{stroke}%
\end{pgfscope}%
\begin{pgfscope}%
\definecolor{textcolor}{rgb}{0.150000,0.150000,0.150000}%
\pgfsetstrokecolor{textcolor}%
\pgfsetfillcolor{textcolor}%
\pgftext[x=0.337239in, y=6.109247in, left, base]{\color{textcolor}\rmfamily\fontsize{8.000000}{9.600000}\selectfont \(\displaystyle {8000}\)}%
\end{pgfscope}%
\begin{pgfscope}%
\pgfpathrectangle{\pgfqpoint{0.663632in}{0.466613in}}{\pgfqpoint{9.300000in}{9.060000in}}%
\pgfusepath{clip}%
\pgfsetroundcap%
\pgfsetroundjoin%
\pgfsetlinewidth{0.501875pt}%
\definecolor{currentstroke}{rgb}{0.800000,0.800000,0.800000}%
\pgfsetstrokecolor{currentstroke}%
\pgfsetdash{}{0pt}%
\pgfpathmoveto{\pgfqpoint{0.663632in}{7.567733in}}%
\pgfpathlineto{\pgfqpoint{9.963632in}{7.567733in}}%
\pgfusepath{stroke}%
\end{pgfscope}%
\begin{pgfscope}%
\definecolor{textcolor}{rgb}{0.150000,0.150000,0.150000}%
\pgfsetstrokecolor{textcolor}%
\pgfsetfillcolor{textcolor}%
\pgftext[x=0.278211in, y=7.529471in, left, base]{\color{textcolor}\rmfamily\fontsize{8.000000}{9.600000}\selectfont \(\displaystyle {10000}\)}%
\end{pgfscope}%
\begin{pgfscope}%
\pgfpathrectangle{\pgfqpoint{0.663632in}{0.466613in}}{\pgfqpoint{9.300000in}{9.060000in}}%
\pgfusepath{clip}%
\pgfsetroundcap%
\pgfsetroundjoin%
\pgfsetlinewidth{0.501875pt}%
\definecolor{currentstroke}{rgb}{0.800000,0.800000,0.800000}%
\pgfsetstrokecolor{currentstroke}%
\pgfsetdash{}{0pt}%
\pgfpathmoveto{\pgfqpoint{0.663632in}{8.987957in}}%
\pgfpathlineto{\pgfqpoint{9.963632in}{8.987957in}}%
\pgfusepath{stroke}%
\end{pgfscope}%
\begin{pgfscope}%
\definecolor{textcolor}{rgb}{0.150000,0.150000,0.150000}%
\pgfsetstrokecolor{textcolor}%
\pgfsetfillcolor{textcolor}%
\pgftext[x=0.278211in, y=8.949695in, left, base]{\color{textcolor}\rmfamily\fontsize{8.000000}{9.600000}\selectfont \(\displaystyle {12000}\)}%
\end{pgfscope}%
\begin{pgfscope}%
\definecolor{textcolor}{rgb}{0.150000,0.150000,0.150000}%
\pgfsetstrokecolor{textcolor}%
\pgfsetfillcolor{textcolor}%
\pgftext[x=0.222655in,y=4.996613in,,bottom,rotate=90.000000]{\color{textcolor}\rmfamily\fontsize{10.000000}{12.000000}\selectfont Probability density}%
\end{pgfscope}%
\begin{pgfscope}%
\pgfpathrectangle{\pgfqpoint{0.663632in}{0.466613in}}{\pgfqpoint{9.300000in}{9.060000in}}%
\pgfusepath{clip}%
\pgfsetbuttcap%
\pgfsetmiterjoin%
\definecolor{currentfill}{rgb}{0.298039,0.447059,0.690196}%
\pgfsetfillcolor{currentfill}%
\pgfsetlinewidth{1.003750pt}%
\definecolor{currentstroke}{rgb}{1.000000,1.000000,1.000000}%
\pgfsetstrokecolor{currentstroke}%
\pgfsetdash{}{0pt}%
\pgfpathmoveto{\pgfqpoint{1.086359in}{0.466613in}}%
\pgfpathlineto{\pgfqpoint{1.931813in}{0.466613in}}%
\pgfpathlineto{\pgfqpoint{1.931813in}{4.958071in}}%
\pgfpathlineto{\pgfqpoint{1.086359in}{4.958071in}}%
\pgfpathclose%
\pgfusepath{stroke,fill}%
\end{pgfscope}%
\begin{pgfscope}%
\pgfpathrectangle{\pgfqpoint{0.663632in}{0.466613in}}{\pgfqpoint{9.300000in}{9.060000in}}%
\pgfusepath{clip}%
\pgfsetbuttcap%
\pgfsetmiterjoin%
\definecolor{currentfill}{rgb}{0.298039,0.447059,0.690196}%
\pgfsetfillcolor{currentfill}%
\pgfsetlinewidth{1.003750pt}%
\definecolor{currentstroke}{rgb}{1.000000,1.000000,1.000000}%
\pgfsetstrokecolor{currentstroke}%
\pgfsetdash{}{0pt}%
\pgfpathmoveto{\pgfqpoint{1.931813in}{0.466613in}}%
\pgfpathlineto{\pgfqpoint{2.777268in}{0.466613in}}%
\pgfpathlineto{\pgfqpoint{2.777268in}{2.986800in}}%
\pgfpathlineto{\pgfqpoint{1.931813in}{2.986800in}}%
\pgfpathclose%
\pgfusepath{stroke,fill}%
\end{pgfscope}%
\begin{pgfscope}%
\pgfpathrectangle{\pgfqpoint{0.663632in}{0.466613in}}{\pgfqpoint{9.300000in}{9.060000in}}%
\pgfusepath{clip}%
\pgfsetbuttcap%
\pgfsetmiterjoin%
\definecolor{currentfill}{rgb}{0.298039,0.447059,0.690196}%
\pgfsetfillcolor{currentfill}%
\pgfsetlinewidth{1.003750pt}%
\definecolor{currentstroke}{rgb}{1.000000,1.000000,1.000000}%
\pgfsetstrokecolor{currentstroke}%
\pgfsetdash{}{0pt}%
\pgfpathmoveto{\pgfqpoint{2.777268in}{0.466613in}}%
\pgfpathlineto{\pgfqpoint{3.622723in}{0.466613in}}%
\pgfpathlineto{\pgfqpoint{3.622723in}{2.903717in}}%
\pgfpathlineto{\pgfqpoint{2.777268in}{2.903717in}}%
\pgfpathclose%
\pgfusepath{stroke,fill}%
\end{pgfscope}%
\begin{pgfscope}%
\pgfpathrectangle{\pgfqpoint{0.663632in}{0.466613in}}{\pgfqpoint{9.300000in}{9.060000in}}%
\pgfusepath{clip}%
\pgfsetbuttcap%
\pgfsetmiterjoin%
\definecolor{currentfill}{rgb}{0.298039,0.447059,0.690196}%
\pgfsetfillcolor{currentfill}%
\pgfsetlinewidth{1.003750pt}%
\definecolor{currentstroke}{rgb}{1.000000,1.000000,1.000000}%
\pgfsetstrokecolor{currentstroke}%
\pgfsetdash{}{0pt}%
\pgfpathmoveto{\pgfqpoint{3.622722in}{0.466613in}}%
\pgfpathlineto{\pgfqpoint{4.468177in}{0.466613in}}%
\pgfpathlineto{\pgfqpoint{4.468177in}{2.853299in}}%
\pgfpathlineto{\pgfqpoint{3.622722in}{2.853299in}}%
\pgfpathclose%
\pgfusepath{stroke,fill}%
\end{pgfscope}%
\begin{pgfscope}%
\pgfpathrectangle{\pgfqpoint{0.663632in}{0.466613in}}{\pgfqpoint{9.300000in}{9.060000in}}%
\pgfusepath{clip}%
\pgfsetbuttcap%
\pgfsetmiterjoin%
\definecolor{currentfill}{rgb}{0.298039,0.447059,0.690196}%
\pgfsetfillcolor{currentfill}%
\pgfsetlinewidth{1.003750pt}%
\definecolor{currentstroke}{rgb}{1.000000,1.000000,1.000000}%
\pgfsetstrokecolor{currentstroke}%
\pgfsetdash{}{0pt}%
\pgfpathmoveto{\pgfqpoint{4.468177in}{0.466613in}}%
\pgfpathlineto{\pgfqpoint{5.313631in}{0.466613in}}%
\pgfpathlineto{\pgfqpoint{5.313631in}{2.920050in}}%
\pgfpathlineto{\pgfqpoint{4.468177in}{2.920050in}}%
\pgfpathclose%
\pgfusepath{stroke,fill}%
\end{pgfscope}%
\begin{pgfscope}%
\pgfpathrectangle{\pgfqpoint{0.663632in}{0.466613in}}{\pgfqpoint{9.300000in}{9.060000in}}%
\pgfusepath{clip}%
\pgfsetbuttcap%
\pgfsetmiterjoin%
\definecolor{currentfill}{rgb}{0.298039,0.447059,0.690196}%
\pgfsetfillcolor{currentfill}%
\pgfsetlinewidth{1.003750pt}%
\definecolor{currentstroke}{rgb}{1.000000,1.000000,1.000000}%
\pgfsetstrokecolor{currentstroke}%
\pgfsetdash{}{0pt}%
\pgfpathmoveto{\pgfqpoint{5.313632in}{0.466613in}}%
\pgfpathlineto{\pgfqpoint{6.159087in}{0.466613in}}%
\pgfpathlineto{\pgfqpoint{6.159087in}{2.888095in}}%
\pgfpathlineto{\pgfqpoint{5.313632in}{2.888095in}}%
\pgfpathclose%
\pgfusepath{stroke,fill}%
\end{pgfscope}%
\begin{pgfscope}%
\pgfpathrectangle{\pgfqpoint{0.663632in}{0.466613in}}{\pgfqpoint{9.300000in}{9.060000in}}%
\pgfusepath{clip}%
\pgfsetbuttcap%
\pgfsetmiterjoin%
\definecolor{currentfill}{rgb}{0.298039,0.447059,0.690196}%
\pgfsetfillcolor{currentfill}%
\pgfsetlinewidth{1.003750pt}%
\definecolor{currentstroke}{rgb}{1.000000,1.000000,1.000000}%
\pgfsetstrokecolor{currentstroke}%
\pgfsetdash{}{0pt}%
\pgfpathmoveto{\pgfqpoint{6.159086in}{0.466613in}}%
\pgfpathlineto{\pgfqpoint{7.004541in}{0.466613in}}%
\pgfpathlineto{\pgfqpoint{7.004541in}{2.919340in}}%
\pgfpathlineto{\pgfqpoint{6.159086in}{2.919340in}}%
\pgfpathclose%
\pgfusepath{stroke,fill}%
\end{pgfscope}%
\begin{pgfscope}%
\pgfpathrectangle{\pgfqpoint{0.663632in}{0.466613in}}{\pgfqpoint{9.300000in}{9.060000in}}%
\pgfusepath{clip}%
\pgfsetbuttcap%
\pgfsetmiterjoin%
\definecolor{currentfill}{rgb}{0.298039,0.447059,0.690196}%
\pgfsetfillcolor{currentfill}%
\pgfsetlinewidth{1.003750pt}%
\definecolor{currentstroke}{rgb}{1.000000,1.000000,1.000000}%
\pgfsetstrokecolor{currentstroke}%
\pgfsetdash{}{0pt}%
\pgfpathmoveto{\pgfqpoint{7.004541in}{0.466613in}}%
\pgfpathlineto{\pgfqpoint{7.849995in}{0.466613in}}%
\pgfpathlineto{\pgfqpoint{7.849995in}{3.255933in}}%
\pgfpathlineto{\pgfqpoint{7.004541in}{3.255933in}}%
\pgfpathclose%
\pgfusepath{stroke,fill}%
\end{pgfscope}%
\begin{pgfscope}%
\pgfpathrectangle{\pgfqpoint{0.663632in}{0.466613in}}{\pgfqpoint{9.300000in}{9.060000in}}%
\pgfusepath{clip}%
\pgfsetbuttcap%
\pgfsetmiterjoin%
\definecolor{currentfill}{rgb}{0.298039,0.447059,0.690196}%
\pgfsetfillcolor{currentfill}%
\pgfsetlinewidth{1.003750pt}%
\definecolor{currentstroke}{rgb}{1.000000,1.000000,1.000000}%
\pgfsetstrokecolor{currentstroke}%
\pgfsetdash{}{0pt}%
\pgfpathmoveto{\pgfqpoint{7.849995in}{0.466613in}}%
\pgfpathlineto{\pgfqpoint{8.695450in}{0.466613in}}%
\pgfpathlineto{\pgfqpoint{8.695450in}{4.399213in}}%
\pgfpathlineto{\pgfqpoint{7.849995in}{4.399213in}}%
\pgfpathclose%
\pgfusepath{stroke,fill}%
\end{pgfscope}%
\begin{pgfscope}%
\pgfpathrectangle{\pgfqpoint{0.663632in}{0.466613in}}{\pgfqpoint{9.300000in}{9.060000in}}%
\pgfusepath{clip}%
\pgfsetbuttcap%
\pgfsetmiterjoin%
\definecolor{currentfill}{rgb}{0.298039,0.447059,0.690196}%
\pgfsetfillcolor{currentfill}%
\pgfsetlinewidth{1.003750pt}%
\definecolor{currentstroke}{rgb}{1.000000,1.000000,1.000000}%
\pgfsetstrokecolor{currentstroke}%
\pgfsetdash{}{0pt}%
\pgfpathmoveto{\pgfqpoint{8.695450in}{0.466613in}}%
\pgfpathlineto{\pgfqpoint{9.540904in}{0.466613in}}%
\pgfpathlineto{\pgfqpoint{9.540904in}{9.095184in}}%
\pgfpathlineto{\pgfqpoint{8.695450in}{9.095184in}}%
\pgfpathclose%
\pgfusepath{stroke,fill}%
\end{pgfscope}%
\begin{pgfscope}%
\pgfsetrectcap%
\pgfsetmiterjoin%
\pgfsetlinewidth{0.752812pt}%
\definecolor{currentstroke}{rgb}{0.700000,0.700000,0.700000}%
\pgfsetstrokecolor{currentstroke}%
\pgfsetdash{}{0pt}%
\pgfpathmoveto{\pgfqpoint{0.663632in}{0.466613in}}%
\pgfpathlineto{\pgfqpoint{0.663632in}{9.526613in}}%
\pgfusepath{stroke}%
\end{pgfscope}%
\begin{pgfscope}%
\pgfsetrectcap%
\pgfsetmiterjoin%
\pgfsetlinewidth{0.752812pt}%
\definecolor{currentstroke}{rgb}{0.700000,0.700000,0.700000}%
\pgfsetstrokecolor{currentstroke}%
\pgfsetdash{}{0pt}%
\pgfpathmoveto{\pgfqpoint{9.963632in}{0.466613in}}%
\pgfpathlineto{\pgfqpoint{9.963632in}{9.526613in}}%
\pgfusepath{stroke}%
\end{pgfscope}%
\begin{pgfscope}%
\pgfsetrectcap%
\pgfsetmiterjoin%
\pgfsetlinewidth{0.752812pt}%
\definecolor{currentstroke}{rgb}{0.700000,0.700000,0.700000}%
\pgfsetstrokecolor{currentstroke}%
\pgfsetdash{}{0pt}%
\pgfpathmoveto{\pgfqpoint{0.663632in}{0.466613in}}%
\pgfpathlineto{\pgfqpoint{9.963632in}{0.466613in}}%
\pgfusepath{stroke}%
\end{pgfscope}%
\begin{pgfscope}%
\pgfsetrectcap%
\pgfsetmiterjoin%
\pgfsetlinewidth{0.752812pt}%
\definecolor{currentstroke}{rgb}{0.700000,0.700000,0.700000}%
\pgfsetstrokecolor{currentstroke}%
\pgfsetdash{}{0pt}%
\pgfpathmoveto{\pgfqpoint{0.663632in}{9.526613in}}%
\pgfpathlineto{\pgfqpoint{9.963632in}{9.526613in}}%
\pgfusepath{stroke}%
\end{pgfscope}%
\end{pgfpicture}%
\makeatother%
\endgroup%
}} &
      \subfloat[Attacked nodes]{\resizebox{0.5\linewidth}{!}{%% Creator: Matplotlib, PGF backend
%%
%% To include the figure in your LaTeX document, write
%%   \input{<filename>.pgf}
%%
%% Make sure the required packages are loaded in your preamble
%%   \usepackage{pgf}
%%
%% and, on pdftex
%%   \usepackage[utf8]{inputenc}\DeclareUnicodeCharacter{2212}{-}
%%
%% or, on luatex and xetex
%%   \usepackage{unicode-math}
%%
%% Figures using additional raster images can only be included by \input if
%% they are in the same directory as the main LaTeX file. For loading figures
%% from other directories you can use the `import` package
%%   \usepackage{import}
%%
%% and then include the figures with
%%   \import{<path to file>}{<filename>.pgf}
%%
%% Matplotlib used the following preamble
%%   \usepackage[utf8]{inputenc}
%%   \usepackage[T1]{fontenc}
%%   \usepackage{amsmath}
%%   \newcommand*{\mat}[1]{\boldsymbol{#1}}
%%
\begingroup%
\makeatletter%
\begin{pgfpicture}%
\pgfpathrectangle{\pgfpointorigin}{\pgfqpoint{2.000721in}{1.819295in}}%
\pgfusepath{use as bounding box, clip}%
\begin{pgfscope}%
\pgfsetbuttcap%
\pgfsetmiterjoin%
\definecolor{currentfill}{rgb}{1.000000,1.000000,1.000000}%
\pgfsetfillcolor{currentfill}%
\pgfsetlinewidth{0.000000pt}%
\definecolor{currentstroke}{rgb}{1.000000,1.000000,1.000000}%
\pgfsetstrokecolor{currentstroke}%
\pgfsetstrokeopacity{0.000000}%
\pgfsetdash{}{0pt}%
\pgfpathmoveto{\pgfqpoint{-0.000000in}{0.000000in}}%
\pgfpathlineto{\pgfqpoint{2.000721in}{0.000000in}}%
\pgfpathlineto{\pgfqpoint{2.000721in}{1.819295in}}%
\pgfpathlineto{\pgfqpoint{-0.000000in}{1.819295in}}%
\pgfpathclose%
\pgfusepath{fill}%
\end{pgfscope}%
\begin{pgfscope}%
\pgfsetbuttcap%
\pgfsetmiterjoin%
\definecolor{currentfill}{rgb}{1.000000,1.000000,1.000000}%
\pgfsetfillcolor{currentfill}%
\pgfsetlinewidth{0.000000pt}%
\definecolor{currentstroke}{rgb}{0.000000,0.000000,0.000000}%
\pgfsetstrokecolor{currentstroke}%
\pgfsetstrokeopacity{0.000000}%
\pgfsetdash{}{0pt}%
\pgfpathmoveto{\pgfqpoint{0.604603in}{0.473545in}}%
\pgfpathlineto{\pgfqpoint{1.883353in}{0.473545in}}%
\pgfpathlineto{\pgfqpoint{1.883353in}{1.719295in}}%
\pgfpathlineto{\pgfqpoint{0.604603in}{1.719295in}}%
\pgfpathclose%
\pgfusepath{fill}%
\end{pgfscope}%
\begin{pgfscope}%
\pgfpathrectangle{\pgfqpoint{0.604603in}{0.473545in}}{\pgfqpoint{1.278750in}{1.245750in}}%
\pgfusepath{clip}%
\pgfsetroundcap%
\pgfsetroundjoin%
\pgfsetlinewidth{0.501875pt}%
\definecolor{currentstroke}{rgb}{0.800000,0.800000,0.800000}%
\pgfsetstrokecolor{currentstroke}%
\pgfsetdash{}{0pt}%
\pgfpathmoveto{\pgfqpoint{0.662728in}{0.473545in}}%
\pgfpathlineto{\pgfqpoint{0.662728in}{1.719295in}}%
\pgfusepath{stroke}%
\end{pgfscope}%
\begin{pgfscope}%
\definecolor{textcolor}{rgb}{0.150000,0.150000,0.150000}%
\pgfsetstrokecolor{textcolor}%
\pgfsetfillcolor{textcolor}%
\pgftext[x=0.662728in,y=0.383267in,,top]{\color{textcolor}\rmfamily\fontsize{8.000000}{9.600000}\selectfont \(\displaystyle {0.0}\)}%
\end{pgfscope}%
\begin{pgfscope}%
\pgfpathrectangle{\pgfqpoint{0.604603in}{0.473545in}}{\pgfqpoint{1.278750in}{1.245750in}}%
\pgfusepath{clip}%
\pgfsetroundcap%
\pgfsetroundjoin%
\pgfsetlinewidth{0.501875pt}%
\definecolor{currentstroke}{rgb}{0.800000,0.800000,0.800000}%
\pgfsetstrokecolor{currentstroke}%
\pgfsetdash{}{0pt}%
\pgfpathmoveto{\pgfqpoint{1.244012in}{0.473545in}}%
\pgfpathlineto{\pgfqpoint{1.244012in}{1.719295in}}%
\pgfusepath{stroke}%
\end{pgfscope}%
\begin{pgfscope}%
\definecolor{textcolor}{rgb}{0.150000,0.150000,0.150000}%
\pgfsetstrokecolor{textcolor}%
\pgfsetfillcolor{textcolor}%
\pgftext[x=1.244012in,y=0.383267in,,top]{\color{textcolor}\rmfamily\fontsize{8.000000}{9.600000}\selectfont \(\displaystyle {0.5}\)}%
\end{pgfscope}%
\begin{pgfscope}%
\pgfpathrectangle{\pgfqpoint{0.604603in}{0.473545in}}{\pgfqpoint{1.278750in}{1.245750in}}%
\pgfusepath{clip}%
\pgfsetroundcap%
\pgfsetroundjoin%
\pgfsetlinewidth{0.501875pt}%
\definecolor{currentstroke}{rgb}{0.800000,0.800000,0.800000}%
\pgfsetstrokecolor{currentstroke}%
\pgfsetdash{}{0pt}%
\pgfpathmoveto{\pgfqpoint{1.825296in}{0.473545in}}%
\pgfpathlineto{\pgfqpoint{1.825296in}{1.719295in}}%
\pgfusepath{stroke}%
\end{pgfscope}%
\begin{pgfscope}%
\definecolor{textcolor}{rgb}{0.150000,0.150000,0.150000}%
\pgfsetstrokecolor{textcolor}%
\pgfsetfillcolor{textcolor}%
\pgftext[x=1.825296in,y=0.383267in,,top]{\color{textcolor}\rmfamily\fontsize{8.000000}{9.600000}\selectfont \(\displaystyle {1.0}\)}%
\end{pgfscope}%
\begin{pgfscope}%
\definecolor{textcolor}{rgb}{0.150000,0.150000,0.150000}%
\pgfsetstrokecolor{textcolor}%
\pgfsetfillcolor{textcolor}%
\pgftext[x=1.243978in,y=0.229588in,,top]{\color{textcolor}\rmfamily\fontsize{10.000000}{12.000000}\selectfont Prob. score \(\displaystyle p^*\)}%
\end{pgfscope}%
\begin{pgfscope}%
\pgfpathrectangle{\pgfqpoint{0.604603in}{0.473545in}}{\pgfqpoint{1.278750in}{1.245750in}}%
\pgfusepath{clip}%
\pgfsetroundcap%
\pgfsetroundjoin%
\pgfsetlinewidth{0.501875pt}%
\definecolor{currentstroke}{rgb}{0.800000,0.800000,0.800000}%
\pgfsetstrokecolor{currentstroke}%
\pgfsetdash{}{0pt}%
\pgfpathmoveto{\pgfqpoint{0.604603in}{0.473545in}}%
\pgfpathlineto{\pgfqpoint{1.883353in}{0.473545in}}%
\pgfusepath{stroke}%
\end{pgfscope}%
\begin{pgfscope}%
\definecolor{textcolor}{rgb}{0.150000,0.150000,0.150000}%
\pgfsetstrokecolor{textcolor}%
\pgfsetfillcolor{textcolor}%
\pgftext[x=0.455297in, y=0.435283in, left, base]{\color{textcolor}\rmfamily\fontsize{8.000000}{9.600000}\selectfont \(\displaystyle {0}\)}%
\end{pgfscope}%
\begin{pgfscope}%
\pgfpathrectangle{\pgfqpoint{0.604603in}{0.473545in}}{\pgfqpoint{1.278750in}{1.245750in}}%
\pgfusepath{clip}%
\pgfsetroundcap%
\pgfsetroundjoin%
\pgfsetlinewidth{0.501875pt}%
\definecolor{currentstroke}{rgb}{0.800000,0.800000,0.800000}%
\pgfsetstrokecolor{currentstroke}%
\pgfsetdash{}{0pt}%
\pgfpathmoveto{\pgfqpoint{0.604603in}{0.810790in}}%
\pgfpathlineto{\pgfqpoint{1.883353in}{0.810790in}}%
\pgfusepath{stroke}%
\end{pgfscope}%
\begin{pgfscope}%
\definecolor{textcolor}{rgb}{0.150000,0.150000,0.150000}%
\pgfsetstrokecolor{textcolor}%
\pgfsetfillcolor{textcolor}%
\pgftext[x=0.278211in, y=0.772528in, left, base]{\color{textcolor}\rmfamily\fontsize{8.000000}{9.600000}\selectfont \(\displaystyle {1000}\)}%
\end{pgfscope}%
\begin{pgfscope}%
\pgfpathrectangle{\pgfqpoint{0.604603in}{0.473545in}}{\pgfqpoint{1.278750in}{1.245750in}}%
\pgfusepath{clip}%
\pgfsetroundcap%
\pgfsetroundjoin%
\pgfsetlinewidth{0.501875pt}%
\definecolor{currentstroke}{rgb}{0.800000,0.800000,0.800000}%
\pgfsetstrokecolor{currentstroke}%
\pgfsetdash{}{0pt}%
\pgfpathmoveto{\pgfqpoint{0.604603in}{1.148035in}}%
\pgfpathlineto{\pgfqpoint{1.883353in}{1.148035in}}%
\pgfusepath{stroke}%
\end{pgfscope}%
\begin{pgfscope}%
\definecolor{textcolor}{rgb}{0.150000,0.150000,0.150000}%
\pgfsetstrokecolor{textcolor}%
\pgfsetfillcolor{textcolor}%
\pgftext[x=0.278211in, y=1.109773in, left, base]{\color{textcolor}\rmfamily\fontsize{8.000000}{9.600000}\selectfont \(\displaystyle {2000}\)}%
\end{pgfscope}%
\begin{pgfscope}%
\pgfpathrectangle{\pgfqpoint{0.604603in}{0.473545in}}{\pgfqpoint{1.278750in}{1.245750in}}%
\pgfusepath{clip}%
\pgfsetroundcap%
\pgfsetroundjoin%
\pgfsetlinewidth{0.501875pt}%
\definecolor{currentstroke}{rgb}{0.800000,0.800000,0.800000}%
\pgfsetstrokecolor{currentstroke}%
\pgfsetdash{}{0pt}%
\pgfpathmoveto{\pgfqpoint{0.604603in}{1.485281in}}%
\pgfpathlineto{\pgfqpoint{1.883353in}{1.485281in}}%
\pgfusepath{stroke}%
\end{pgfscope}%
\begin{pgfscope}%
\definecolor{textcolor}{rgb}{0.150000,0.150000,0.150000}%
\pgfsetstrokecolor{textcolor}%
\pgfsetfillcolor{textcolor}%
\pgftext[x=0.278211in, y=1.447018in, left, base]{\color{textcolor}\rmfamily\fontsize{8.000000}{9.600000}\selectfont \(\displaystyle {3000}\)}%
\end{pgfscope}%
\begin{pgfscope}%
\definecolor{textcolor}{rgb}{0.150000,0.150000,0.150000}%
\pgfsetstrokecolor{textcolor}%
\pgfsetfillcolor{textcolor}%
\pgftext[x=0.222655in,y=1.096420in,,bottom,rotate=90.000000]{\color{textcolor}\rmfamily\fontsize{10.000000}{12.000000}\selectfont Frequency}%
\end{pgfscope}%
\begin{pgfscope}%
\pgfpathrectangle{\pgfqpoint{0.604603in}{0.473545in}}{\pgfqpoint{1.278750in}{1.245750in}}%
\pgfusepath{clip}%
\pgfsetbuttcap%
\pgfsetmiterjoin%
\definecolor{currentfill}{rgb}{0.298039,0.447059,0.690196}%
\pgfsetfillcolor{currentfill}%
\pgfsetlinewidth{1.003750pt}%
\definecolor{currentstroke}{rgb}{1.000000,1.000000,1.000000}%
\pgfsetstrokecolor{currentstroke}%
\pgfsetdash{}{0pt}%
\pgfpathmoveto{\pgfqpoint{0.662728in}{0.473545in}}%
\pgfpathlineto{\pgfqpoint{0.778978in}{0.473545in}}%
\pgfpathlineto{\pgfqpoint{0.778978in}{1.659974in}}%
\pgfpathlineto{\pgfqpoint{0.662728in}{1.659974in}}%
\pgfpathclose%
\pgfusepath{stroke,fill}%
\end{pgfscope}%
\begin{pgfscope}%
\pgfpathrectangle{\pgfqpoint{0.604603in}{0.473545in}}{\pgfqpoint{1.278750in}{1.245750in}}%
\pgfusepath{clip}%
\pgfsetbuttcap%
\pgfsetmiterjoin%
\definecolor{currentfill}{rgb}{0.298039,0.447059,0.690196}%
\pgfsetfillcolor{currentfill}%
\pgfsetlinewidth{1.003750pt}%
\definecolor{currentstroke}{rgb}{1.000000,1.000000,1.000000}%
\pgfsetstrokecolor{currentstroke}%
\pgfsetdash{}{0pt}%
\pgfpathmoveto{\pgfqpoint{0.778978in}{0.473545in}}%
\pgfpathlineto{\pgfqpoint{0.895228in}{0.473545in}}%
\pgfpathlineto{\pgfqpoint{0.895228in}{1.064399in}}%
\pgfpathlineto{\pgfqpoint{0.778978in}{1.064399in}}%
\pgfpathclose%
\pgfusepath{stroke,fill}%
\end{pgfscope}%
\begin{pgfscope}%
\pgfpathrectangle{\pgfqpoint{0.604603in}{0.473545in}}{\pgfqpoint{1.278750in}{1.245750in}}%
\pgfusepath{clip}%
\pgfsetbuttcap%
\pgfsetmiterjoin%
\definecolor{currentfill}{rgb}{0.298039,0.447059,0.690196}%
\pgfsetfillcolor{currentfill}%
\pgfsetlinewidth{1.003750pt}%
\definecolor{currentstroke}{rgb}{1.000000,1.000000,1.000000}%
\pgfsetstrokecolor{currentstroke}%
\pgfsetdash{}{0pt}%
\pgfpathmoveto{\pgfqpoint{0.895228in}{0.473545in}}%
\pgfpathlineto{\pgfqpoint{1.011478in}{0.473545in}}%
\pgfpathlineto{\pgfqpoint{1.011478in}{0.896788in}}%
\pgfpathlineto{\pgfqpoint{0.895228in}{0.896788in}}%
\pgfpathclose%
\pgfusepath{stroke,fill}%
\end{pgfscope}%
\begin{pgfscope}%
\pgfpathrectangle{\pgfqpoint{0.604603in}{0.473545in}}{\pgfqpoint{1.278750in}{1.245750in}}%
\pgfusepath{clip}%
\pgfsetbuttcap%
\pgfsetmiterjoin%
\definecolor{currentfill}{rgb}{0.298039,0.447059,0.690196}%
\pgfsetfillcolor{currentfill}%
\pgfsetlinewidth{1.003750pt}%
\definecolor{currentstroke}{rgb}{1.000000,1.000000,1.000000}%
\pgfsetstrokecolor{currentstroke}%
\pgfsetdash{}{0pt}%
\pgfpathmoveto{\pgfqpoint{1.011478in}{0.473545in}}%
\pgfpathlineto{\pgfqpoint{1.127728in}{0.473545in}}%
\pgfpathlineto{\pgfqpoint{1.127728in}{0.868122in}}%
\pgfpathlineto{\pgfqpoint{1.011478in}{0.868122in}}%
\pgfpathclose%
\pgfusepath{stroke,fill}%
\end{pgfscope}%
\begin{pgfscope}%
\pgfpathrectangle{\pgfqpoint{0.604603in}{0.473545in}}{\pgfqpoint{1.278750in}{1.245750in}}%
\pgfusepath{clip}%
\pgfsetbuttcap%
\pgfsetmiterjoin%
\definecolor{currentfill}{rgb}{0.298039,0.447059,0.690196}%
\pgfsetfillcolor{currentfill}%
\pgfsetlinewidth{1.003750pt}%
\definecolor{currentstroke}{rgb}{1.000000,1.000000,1.000000}%
\pgfsetstrokecolor{currentstroke}%
\pgfsetdash{}{0pt}%
\pgfpathmoveto{\pgfqpoint{1.127728in}{0.473545in}}%
\pgfpathlineto{\pgfqpoint{1.243978in}{0.473545in}}%
\pgfpathlineto{\pgfqpoint{1.243978in}{0.857667in}}%
\pgfpathlineto{\pgfqpoint{1.127728in}{0.857667in}}%
\pgfpathclose%
\pgfusepath{stroke,fill}%
\end{pgfscope}%
\begin{pgfscope}%
\pgfpathrectangle{\pgfqpoint{0.604603in}{0.473545in}}{\pgfqpoint{1.278750in}{1.245750in}}%
\pgfusepath{clip}%
\pgfsetbuttcap%
\pgfsetmiterjoin%
\definecolor{currentfill}{rgb}{0.298039,0.447059,0.690196}%
\pgfsetfillcolor{currentfill}%
\pgfsetlinewidth{1.003750pt}%
\definecolor{currentstroke}{rgb}{1.000000,1.000000,1.000000}%
\pgfsetstrokecolor{currentstroke}%
\pgfsetdash{}{0pt}%
\pgfpathmoveto{\pgfqpoint{1.243978in}{0.473545in}}%
\pgfpathlineto{\pgfqpoint{1.360228in}{0.473545in}}%
\pgfpathlineto{\pgfqpoint{1.360228in}{0.829339in}}%
\pgfpathlineto{\pgfqpoint{1.243978in}{0.829339in}}%
\pgfpathclose%
\pgfusepath{stroke,fill}%
\end{pgfscope}%
\begin{pgfscope}%
\pgfpathrectangle{\pgfqpoint{0.604603in}{0.473545in}}{\pgfqpoint{1.278750in}{1.245750in}}%
\pgfusepath{clip}%
\pgfsetbuttcap%
\pgfsetmiterjoin%
\definecolor{currentfill}{rgb}{0.298039,0.447059,0.690196}%
\pgfsetfillcolor{currentfill}%
\pgfsetlinewidth{1.003750pt}%
\definecolor{currentstroke}{rgb}{1.000000,1.000000,1.000000}%
\pgfsetstrokecolor{currentstroke}%
\pgfsetdash{}{0pt}%
\pgfpathmoveto{\pgfqpoint{1.360228in}{0.473545in}}%
\pgfpathlineto{\pgfqpoint{1.476478in}{0.473545in}}%
\pgfpathlineto{\pgfqpoint{1.476478in}{0.809779in}}%
\pgfpathlineto{\pgfqpoint{1.360228in}{0.809779in}}%
\pgfpathclose%
\pgfusepath{stroke,fill}%
\end{pgfscope}%
\begin{pgfscope}%
\pgfpathrectangle{\pgfqpoint{0.604603in}{0.473545in}}{\pgfqpoint{1.278750in}{1.245750in}}%
\pgfusepath{clip}%
\pgfsetbuttcap%
\pgfsetmiterjoin%
\definecolor{currentfill}{rgb}{0.298039,0.447059,0.690196}%
\pgfsetfillcolor{currentfill}%
\pgfsetlinewidth{1.003750pt}%
\definecolor{currentstroke}{rgb}{1.000000,1.000000,1.000000}%
\pgfsetstrokecolor{currentstroke}%
\pgfsetdash{}{0pt}%
\pgfpathmoveto{\pgfqpoint{1.476478in}{0.473545in}}%
\pgfpathlineto{\pgfqpoint{1.592728in}{0.473545in}}%
\pgfpathlineto{\pgfqpoint{1.592728in}{0.800336in}}%
\pgfpathlineto{\pgfqpoint{1.476478in}{0.800336in}}%
\pgfpathclose%
\pgfusepath{stroke,fill}%
\end{pgfscope}%
\begin{pgfscope}%
\pgfpathrectangle{\pgfqpoint{0.604603in}{0.473545in}}{\pgfqpoint{1.278750in}{1.245750in}}%
\pgfusepath{clip}%
\pgfsetbuttcap%
\pgfsetmiterjoin%
\definecolor{currentfill}{rgb}{0.298039,0.447059,0.690196}%
\pgfsetfillcolor{currentfill}%
\pgfsetlinewidth{1.003750pt}%
\definecolor{currentstroke}{rgb}{1.000000,1.000000,1.000000}%
\pgfsetstrokecolor{currentstroke}%
\pgfsetdash{}{0pt}%
\pgfpathmoveto{\pgfqpoint{1.592728in}{0.473545in}}%
\pgfpathlineto{\pgfqpoint{1.708978in}{0.473545in}}%
\pgfpathlineto{\pgfqpoint{1.708978in}{0.826641in}}%
\pgfpathlineto{\pgfqpoint{1.592728in}{0.826641in}}%
\pgfpathclose%
\pgfusepath{stroke,fill}%
\end{pgfscope}%
\begin{pgfscope}%
\pgfpathrectangle{\pgfqpoint{0.604603in}{0.473545in}}{\pgfqpoint{1.278750in}{1.245750in}}%
\pgfusepath{clip}%
\pgfsetbuttcap%
\pgfsetmiterjoin%
\definecolor{currentfill}{rgb}{0.298039,0.447059,0.690196}%
\pgfsetfillcolor{currentfill}%
\pgfsetlinewidth{1.003750pt}%
\definecolor{currentstroke}{rgb}{1.000000,1.000000,1.000000}%
\pgfsetstrokecolor{currentstroke}%
\pgfsetdash{}{0pt}%
\pgfpathmoveto{\pgfqpoint{1.708978in}{0.473545in}}%
\pgfpathlineto{\pgfqpoint{1.825228in}{0.473545in}}%
\pgfpathlineto{\pgfqpoint{1.825228in}{0.899823in}}%
\pgfpathlineto{\pgfqpoint{1.708978in}{0.899823in}}%
\pgfpathclose%
\pgfusepath{stroke,fill}%
\end{pgfscope}%
\begin{pgfscope}%
\pgfsetrectcap%
\pgfsetmiterjoin%
\pgfsetlinewidth{0.752812pt}%
\definecolor{currentstroke}{rgb}{0.700000,0.700000,0.700000}%
\pgfsetstrokecolor{currentstroke}%
\pgfsetdash{}{0pt}%
\pgfpathmoveto{\pgfqpoint{0.604603in}{0.473545in}}%
\pgfpathlineto{\pgfqpoint{0.604603in}{1.719295in}}%
\pgfusepath{stroke}%
\end{pgfscope}%
\begin{pgfscope}%
\pgfsetrectcap%
\pgfsetmiterjoin%
\pgfsetlinewidth{0.752812pt}%
\definecolor{currentstroke}{rgb}{0.700000,0.700000,0.700000}%
\pgfsetstrokecolor{currentstroke}%
\pgfsetdash{}{0pt}%
\pgfpathmoveto{\pgfqpoint{1.883353in}{0.473545in}}%
\pgfpathlineto{\pgfqpoint{1.883353in}{1.719295in}}%
\pgfusepath{stroke}%
\end{pgfscope}%
\begin{pgfscope}%
\pgfsetrectcap%
\pgfsetmiterjoin%
\pgfsetlinewidth{0.752812pt}%
\definecolor{currentstroke}{rgb}{0.700000,0.700000,0.700000}%
\pgfsetstrokecolor{currentstroke}%
\pgfsetdash{}{0pt}%
\pgfpathmoveto{\pgfqpoint{0.604603in}{0.473545in}}%
\pgfpathlineto{\pgfqpoint{1.883353in}{0.473545in}}%
\pgfusepath{stroke}%
\end{pgfscope}%
\begin{pgfscope}%
\pgfsetrectcap%
\pgfsetmiterjoin%
\pgfsetlinewidth{0.752812pt}%
\definecolor{currentstroke}{rgb}{0.700000,0.700000,0.700000}%
\pgfsetstrokecolor{currentstroke}%
\pgfsetdash{}{0pt}%
\pgfpathmoveto{\pgfqpoint{0.604603in}{1.719295in}}%
\pgfpathlineto{\pgfqpoint{1.883353in}{1.719295in}}%
\pgfusepath{stroke}%
\end{pgfscope}%
\end{pgfpicture}%
\makeatother%
\endgroup%
}}  \\
    \end{array}\)
  }
  \caption{In (a) we show the distribution of confidence scores for the correct class \(p^*\) over all test nodes on the clean graph. We observe a large fraction of very confident nodes. In stark contrast, in (b) we analyze the distribution of the directly attacked test nodes before the evasion attack started (i.e.\ if the attack would randomly attack nodes this distribution should match (a)). Here we small budget of one percent of edges (\(\Delta=\epsilon=0.01\))\label{oldfig:negceprob}}
\end{figure}

In Theorem~\ref{oldtheorem:goodsurrogate}, we propose. According to Corollary~\ref{oldcorollary:ce} and Corollary~\ref{oldcorollary:margin} it follows that maximizing the CE or the margin loss do not work well for the constructed scenario. For the following discussion, we define the classification margin as \(\psi = \min_{c \,\text{s.t.}\, c \ne c^*} \evp_{c^*} - \evp{c}\).

%Despite its wide use, the surrogate loss cross entropy can be completely uncorrelated with the accuracy if the budget \(\Delta\) of flipping edges is limited. Using the negative CE leads to a strong focus on the nodes which which had a low confidence score (misclassified) before the attack even started (see Figure~\ref{oldfig:negceprob}). %In contrast, in the traditional (semi-) supervised learning task the budget is only limited via sharing parameters over all samples and regularization.
%The CE loss is unbounded and if the probability of the correct class \(\evp_{c^*}\) approaches 0 it diverges (i.e.\ \(\lim_{p_{c^*} \to 0} \text{CE} = \infty\)). Hence, for small enough budgets \(\Delta\) on a large graph it is possible that the attack only focuses on nodes that are \emph{already wrongly classified} before the attack has even started (for those nodes the gradient is the largest).

\begin{theorem}\label{oldtheorem:goodsurrogate}
  Let \(f_{\theta}(\adj, \features)\) be a GNN applied to a large (possibly infinite), diverse graph with the adjacency matrix \(\adj\). Due to the size and diversity of the graph, the attack can move any node's predicted probability it chooses by exactly \(d\) and there exists a node for every possible confidence score \(p^*\) for the correct class on the clean graph. The budget \(\Delta\) is chosen such that the global optimum of
  \begin{equation}\label{oldeq:goodsurrogate}
    \max_{\tilde{\adj}\text{ s.t.\ }\|\tilde{\adj} - \adj\|_0 < \Delta} \mathcal{L}(f_{\theta}(\tilde{\adj}, \features))
  \end{equation}
  can be obtained by moving a single node's prediction such that the decision boundary is just crossed. That is, after attacking \(d \le \tilde{\psi} < 0\) with an arbitrary small constant \(d\). The derivative of the surrogate loss \(\nicefrac{\partial \mathcal{L}'}{\partial p^*}\) that maximizes Eq.~\ref{oldeq:goodsurrogate}, must have its unique global minimum s.t. for \(\psi \to 0^+\).
\end{theorem}

\begin{proof}
  We know that \(\mathcal{L}'(\psi - d) > \mathcal{L}'(\psi)\) or equivalently we can analyze the normalized gain \(g(\psi)\) and let \(d \to 0\): \[
    g(\psi) = \lim_{d \to 0^+} 
    %\frac{\mathcal{L}'(\psi - d) - \mathcal{L}'(\psi)}{\psi - d - \psi} = 
    -\frac{\mathcal{L}'(\psi - d) - \mathcal{L}'(\psi)}{d} = -\frac{\partial \mathcal{L}'(\psi)}{\partial \psi}
  \]
  Further, to make sure that \(\arg\max_\psi \mathcal{L}' = \arg\max_\psi \mathcal{L}\) we must perturb a node that is within \(0 \le \psi < d\) of the decision boundary. Hence,  \[
    \frac{\partial \mathcal{L}'(\psi)}{\partial \psi} \Big|_{\psi \to 0^+} < \frac{\partial \mathcal{L}'(\psi)}{\partial \psi} \Big|_{r}\,\,\,\forall r \in \mathbb{R}_*^-
  \] and \[
    \frac{\partial \mathcal{L}'(\psi)}{\partial \psi} \Big|_{\psi \to 0^+} < \lim_{d \to 0^+} \frac{\partial \mathcal{L}'(\psi)}{\partial \psi} \Big|_{d + r}\,\,\,\forall r \in \mathbb{R}_*^+
  \] must hold. Or equivalently, \[
    \arg\max_\psi \mathcal{L}' = \arg\min_\psi \frac{\partial \mathcal{L}'(\psi)}{\partial \psi} = \frac{\partial \mathcal{L}'(\psi)}{\partial \psi} \Big|_{\psi \to 0^+}
  \] with the unique global minima for \(\psi \to 0^+\). 
\end{proof}

\begin{corollary}\label{oldcorollary:ce}
  The cross entropy surrogate loss \(\mathcal{L} = \text{Accuracy} \approx CE\) (Eq.~\ref{oldeq:crossentropy}) does not obtain the global optimum since the loss is maximal for nodes with \(p^* \to 0^+\).
\end{corollary}

\begin{corollary}\label{oldcorollary:margin}
  The margin loss \(\mathcal{L} = \text{Accuracy} \approx \text{Margin Loss} = min(0, \psi)\) does not obtain the global optimum, since its gradient is constant for \(\psi > 0\).
\end{corollary}

Of course, the formal statements up to now just covered a very basic scenario where we do not worry about effects such as dependencies between the nodes. However, we argue that the surrogate is certainly not well suited if it even does not work in such a basic scenario. More generally, we state the subsequent conjectures a well-suited, monotonically decreasing surrogate loss \(\mathcal{L^*}(y, \vp)\) should obey for globally attacking a node-classification algorithm.
\begin{conjecture}\label{oldconjecture:conjecture1}
  The loss \(\mathcal{L^*}(y, \vp)\) should saturate for low confidence values of the correct class: \(\lim_{\psi \to -1^+} \mathcal{L}(y, \vp) = k < \infty\).
\end{conjecture}
\begin{conjecture}\label{oldconjecture:conjecture2}
  The loss should favour points close to the decision boundary: \(\nicefrac{\partial \mathcal{L}(y, \vp)}{\partial \evp_{c^*}} |_{\psi = 1}  > \nicefrac{\partial \mathcal{L}(y, \vp)}{\partial \evp_{c^*}} |_{\psi \to 0^+}\).
\end{conjecture}

One natural choice that obeys the restrictions of the conjectures and the theorem is the masked cross entropy
\begin{equation}\label{oldeq:mce}
  \text{MCE} = \frac{1}{|\sV^+|} \sum_{n \in \sV^+} \sum_{c \in \sC} \mathbb{I}[y^{(n)} = c] \log(\evp_{c}^{(n)})
\end{equation}
where \(\sV^+\) is the set of correctly classified nodes. %Conjecture 1 holds since for a negative margin \(\psi < 0\) the node will be simply omitted. Conjecture 2 holds since the gradient \(\nicefrac{\partial \text{MCE}}{\partial \evp_{c^*}}\) is strictly decreasing w.r.t.\ \(\psi\). This loss is well suited for greedy attacks.

However, \(\text{MCE}\) is not a good choice for a projected gradient descent method like the one we are going to propose in Section~\ref{sec:prbcd}. For such an optimization, we typically tune the learning rate such that the budget \(\Delta\) is exceeded after each gradient update and then the project operation maps the parameters back into the feasible region. If we now set the loss to zero / mask it out if wrongly classified, the contributing edges will not gain anything in the gradient update and likely loose strength in the upcoming project step. Hence, those nodes will oscillate at the decision boundary. Therefore, we use tanh of the classification margin
\begin{equation}\label{oldeq:mce}
  \tanh\text{Margin} = \frac{1}{|\sV|} \sum_{n \in \sV} \tanh(\psi^{(n)})
\end{equation}
where \(\psi\) denotes the classification margin. This loss obviously fulfills both conjectures.

In Table~\ref{oldtab:losscompare}, we compare the convectional CE loss with our newly proposed losses. For the admittedly large budget of \(\epsilon=0.25\), we see gains of up to 40\% on the perturbed accuracy. Moreover, if we compare the accuracy drop (i.e.\ clean minus perturbed accuracy) we also achieve for low budgets like \(\epsilon=0.01\) an improvements of more then 100\%. And those are only the numbers for the small datasets. On larger graphs such as arXiv we even observe gains of more than 100\% directly on the perturbed accuracy.

\begin{table*}
  \centering
  \caption{Perturbed accuracy comparing the conventional losses with our loss. We report the mean over three different seeds. \(\epsilon\) denotes the fraction of edges perturbed (relative to the clean graph). We use random split with 20 nodes per class. For each architecture and budget we embolden the better loss. For details about the set up we refer to Section~\ref{sec:empirical}.}
  \label{oldtab:losscompare}
  \resizebox{\linewidth}{!}{
    \begin{tabular}{lcl|ccccccc|ccccccc}
    \toprule
                               &      &     & \multicolumn{7}{c|}{\textbf{Cora ML~\citep{Bojchevski2018}}} & \multicolumn{7}{c}{\textbf{Citeseer~\citep{McCallum2000}}} \\
                               \rotatebox{90}{\textbf{Attack}} & \makecell{\textbf{Frac.}\\\textbf{edges}\\\(\boldsymbol{\epsilon}\)} & \textbf{Loss} &                 \makecell{Vanilla\\GCN} & \makecell{Vanilla\\GDC} & \makecell{SVD\\GCN} & \makecell{Jaccard\\GCN} &  \makecell{RGCN} & \makecell{Soft\\Medoid\\GDC} & \makecell{Soft\\Median\\GDC} &                \makecell{Vanilla\\GCN} & \makecell{Vanilla\\GDC} & \makecell{SVD\\GCN} & \makecell{Jaccard\\GCN} &  \makecell{RGCN} & \makecell{Soft\\Medoid\\GDC} & \makecell{Soft\\Median\\GDC} \\
    \midrule
    \multirow{8}{*}{\rotatebox{90}{\textbf{greedy FGSM}}} & \multirow{2}{*}{0.01} & CE &                                  0.8087 &                  0.8144 &              0.7576 &                  0.8066 &           0.7864 &                       0.8061 &                       0.8092 &                                 0.7052 &                  0.7000 &              0.6401 &                  0.7091 &           0.6389 &              \textbf{0.7045} &                       0.7061 \\
                                &      & MCE &                         \textbf{0.7859} &         \textbf{0.7953} &     \textbf{0.7573} &         \textbf{0.7871} &  \textbf{0.7730} &              \textbf{0.8070} &              \textbf{0.8078} &                        \textbf{0.6850} &         \textbf{0.6832} &     \textbf{0.6376} &         \textbf{0.6959} &  \textbf{0.6305} &                       0.7037 &              \textbf{0.7046} \\
    \cline{2-17}
                                & \multirow{2}{*}{0.05} & CE &                                  0.7577 &                  0.7586 &              0.7414 &                  0.7605 &           0.7428 &              \textbf{0.7722} &              \textbf{0.7722} &                                 0.6693 &                  0.6594 &              0.6244 &                  0.6799 &           0.6077 &              \textbf{0.6852} &              \textbf{0.6831} \\
                                &      & MCE &                         \textbf{0.6908} &         \textbf{0.7045} &     \textbf{0.7426} &         \textbf{0.7116} &  \textbf{0.7004} &                       0.7885 &                       0.7850 &                        \textbf{0.6064} &         \textbf{0.6036} &     \textbf{0.6212} &         \textbf{0.6410} &  \textbf{0.5815} &                       0.6952 &                       0.6911 \\
    \cline{2-17}
                                & \multirow{2}{*}{0.10} & CE &                                  0.7188 &                  0.7179 &              0.7153 &                  0.7221 &           0.7080 &              \textbf{0.7426} &              \textbf{0.7410} &                                 0.6316 &                  0.6223 &              0.6025 &                  0.6476 &           0.5766 &              \textbf{0.6665} &              \textbf{0.6590} \\
                                &      & MCE &                         \textbf{0.6086} &         \textbf{0.6385} &     \textbf{0.7215} &         \textbf{0.6441} &  \textbf{0.6387} &                       0.7773 &                       0.7693 &                        \textbf{0.5335} &         \textbf{0.5346} &     \textbf{0.5991} &         \textbf{0.5922} &  \textbf{0.5323} &                       0.6893 &                       0.6791 \\
    \cline{2-17}
                                & \multirow{2}{*}{0.25} & CE &                                  0.6353 &                  0.6391 &              0.6399 &                  0.6427 &           0.6312 &              \textbf{0.6785} &              \textbf{0.6746} &                                 0.5401 &                  0.5330 &     \textbf{0.3626} &                  0.5729 &           0.5130 &              \textbf{0.6201} &              \textbf{0.6073} \\
                                &      & MCE &                         \textbf{0.4599} &         \textbf{0.5275} &     \textbf{0.6374} &         \textbf{0.5245} &  \textbf{0.5101} &                       0.7632 &                       0.7524 &                        \textbf{0.3898} &         \textbf{0.4128} &              0.5036 &         \textbf{0.4973} &  \textbf{0.4244} &                       0.6820 &                       0.6674 \\
    \cline{1-17}
    \cline{2-17}
    \multirow{8}{*}{\rotatebox{90}{\textbf{PGD}}} & \multirow{2}{*}{0.01} & CE &                                  0.8047 &                  0.8107 &     \textbf{0.7568} &                  0.8020 &           0.7848 &              \textbf{0.8053} &              \textbf{0.8069} &                                 0.7032 &                  0.6980 &              0.6378 &                  0.7057 &           0.6373 &              \textbf{0.7030} &              \textbf{0.7041} \\
                                &      & tanh Margin &                         \textbf{0.7892} &         \textbf{0.7960} &              0.7567 &         \textbf{0.7903} &  \textbf{0.7758} &                       0.8050 &                       0.8053 &                        \textbf{0.6873} &         \textbf{0.6840} &     \textbf{0.6383} &         \textbf{0.6957} &  \textbf{0.6355} &                       0.7045 &                       0.7045 \\
    \cline{2-17}
                                & \multirow{2}{*}{0.05} & CE &                                  0.7547 &                  0.7555 &     \textbf{0.7348} &                  0.7564 &           0.7390 &              \textbf{0.7735} &              \textbf{0.7742} &                                 0.6595 &                  0.6551 &     \textbf{0.6144} &                  0.6718 &           0.6068 &              \textbf{0.6852} &              \textbf{0.6829} \\
                                &      & tanh Margin &                         \textbf{0.7065} &         \textbf{0.7165} &              0.7253 &         \textbf{0.7153} &  \textbf{0.7155} &                       0.7816 &                       0.7762 &                        \textbf{0.6264} &         \textbf{0.6205} &              0.4137 &         \textbf{0.6419} &  \textbf{0.6020} &                       0.6884 &                       0.6868 \\
    \cline{2-17}
                                & \multirow{2}{*}{0.10} & CE &                                  0.7053 &                  0.7045 &     \textbf{0.6957} &                  0.7080 &           0.6991 &              \textbf{0.7432} &              \textbf{0.7402} &                                 0.6184 &                  0.6102 &     \textbf{0.5925} &                  0.6392 &           0.5836 &              \textbf{0.6708} &              \textbf{0.6617} \\
                                &      & tanh Margin &                         \textbf{0.6379} &         \textbf{0.6481} &              0.6905 &         \textbf{0.6577} &  \textbf{0.6560} &                       0.7603 &                       0.7532 &                        \textbf{0.5592} &         \textbf{0.5620} &              0.3947 &         \textbf{0.5932} &  \textbf{0.5554} &                       0.6838 &                       0.6713 \\
    \cline{2-17}
                                & \multirow{2}{*}{0.25} & CE &                                  0.5901 &                  0.5953 &     \textbf{0.6028} &                  0.6011 &           0.5942 &              \textbf{0.6830} &              \textbf{0.6775} &                                 0.5260 &                  0.5175 &     \textbf{0.4904} &                  0.5542 &           0.5039 &              \textbf{0.6283} &              \textbf{0.6086} \\
                                &      & tanh Margin &                         \textbf{0.4993} &         \textbf{0.5254} &              0.5791 &         \textbf{0.5332} &  \textbf{0.5311} &                       0.7246 &                       0.7175 &                        \textbf{0.4283} &         \textbf{0.4342} &              0.4713 &         \textbf{0.5043} &  \textbf{0.4599} &                       0.6674 &                       0.6487 \\
    \bottomrule
  \end{tabular}
  }
\end{table*}

\end{document}
\endinput
%%
%% End of file `sample-authordraft.tex'.
